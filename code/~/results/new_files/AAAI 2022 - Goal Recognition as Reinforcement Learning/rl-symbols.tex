%% Generic mathematical stuff
%% Requires \usepackage{algorithm,bbm,amsmath,amssymb}
% TODO - Add \usepackage{bm}  % For bold math to all previous slides
\newcommand\mapsfrom{\mathrel{\reflectbox{\ensuremath{\mapsto}}}}

\DeclareMathOperator{\agent}{G}
\DeclareMathOperator{\agentspace}{\mathcal{G}}
\DeclareMathOperator{\kldiv}{D_{\textsc{KL}}}

\providecommand{\textred}[1]{{\color{red}#1}}
\providecommand{\textblue}[1]{{\color{blue}#1}}

\providecommand{\e}{\ensuremath{\mathrm{e}}}
\providecommand{\prob}[1]{\ensuremath{\mathbb{P}\left[ #1 \right]}}
\providecommand{\expectation}[2][]{\ensuremath{\mathbb{E}_{#1}\left[#2\right]}}
\DeclareMathOperator{\mdp}{\mathcal{M}}
\DeclareMathOperator{\model}{\mathcal{M}}
% \providecommand{\transprob}[1][]{\ensuremath{\mathcal{P}_{ {#1} }}}
\DeclareMathOperator{\transprob}{\mathcal{P}}
% \providecommand{\transprobf}[4][]{\ensuremath{\mathcal{P}^{{#4},{#1}}_{{#2}{#3}}}}
%% Requires \usepackage{xifthen}
\providecommand{\transprobf}[4][]{\ensuremath{\mathcal{P}({#4}{\ifthenelse{\equal{#1}{}}{}{, #1}}\mid{#2},{#3})}}
%% Does not require that
% \providecommand{\transprobf}[4][]{\ensuremath{\mathcal{P}({#4}{, #1}\mid{#2},{#3})}}
\DeclareMathOperator{\approxtransprob}{\hat{\mathcal{P}}}
\providecommand{\prob}[1]{\ensuremath{p({#1})}}
\providecommand{\normaldist}[1]{\ensuremath{\mathcal{N}({#1})}}
\providecommand{\tderror}[1][]{\ensuremath{\delta_{#1}}}

% \providecommand{\statespace}[1][]{\ensuremath{\mathcal{S}_{ {#1} }}}
\DeclareMathOperator{\statespace}{\mathcal{S}}
\DeclareMathOperator{\statespacefull}{\mathcal{S}^{+}}
\DeclareMathOperator{\samplestatespace}{\mathcal{\tilde{S}}}
\DeclareMathOperator{\actionspace}{\mathcal{A}}
\DeclareMathOperator{\observationspace}{\mathcal{O}}
\DeclareMathOperator{\observationfunction}{\mathcal{Z}}
\DeclareMathOperator{\vectorspace}{\mathcal{V}}
% \providecommand{\reward}[1][]{\ensuremath{\mathcal{R}_{#1}}}
\DeclareMathOperator{\reward}{\mathcal{R}}
\DeclareMathOperator{\approxreward}{\hat{\mathcal{R}}}
\DeclareMathOperator{\singlereward}{R} % A single possible reward (decide)
\DeclareMathOperator{\averagereward}{\bar{R}} % Average rewards up to this point
\DeclareMathOperator{\areward}{r} % An actual reward
\DeclareBoldMathCommand{\observation}{O}
\DeclareMathOperator{\singleobservation}{O} % A single possible observation (decide)
\DeclareMathOperator{\anobservation}{o}
\DeclareMathOperator{\discount}{\gamma}
\DeclareMathOperator{\return}{G}
\DeclareMathOperator{\state}{S}
\DeclareMathOperator{\astate}{s} % An actual state
\DeclareMathOperator{\singleaction}{A} % An action
\DeclareMathOperator{\anaction}{a}
\DeclareMathOperator{\baction}{b}
\DeclareMathOperator{\identity}{I}
\DeclareMathOperator{\history}{H}
\providecommand{\optimalaction}{\ensuremath{\singleaction_{*}}}

\DeclareMathOperator{\eligibilityvector}{\ensuremath{Z}}
\providecommand{\eligibilityfunction}[2][]{\eligibilityvector_{#1}({#2})}
\DeclareMathOperator{\experience}{\mathcal{D}}

\providecommand{\policy}{\ensuremath{\pi}}
% \providecommand{\behaviorpolicy}{\ensuremath{\mu}}% This is what silver uses
\providecommand{\behaviorpolicy}{\ensuremath{\pi_{b}}}% This is what silver uses
\providecommand{\optimalpolicy}{\ensuremath{\pi_{*}}}
\providecommand{\policyfunction}[2][]{\ensuremath{\pi_{#1}({#2})}}
\providecommand{\featurefunction}[2][]{\ensuremath{\phi_{#1}({#2})}}
\providecommand{\policyreward}[2][]{\ensuremath{\areward_{#1}({#2})}}
\providecommand{\baselinefunction}[2][]{\ensuremath{b_{#1}({#2})}}


\providecommand{\norm}[1]{\ensuremath{\left\lVert#1\right\rVert}}
\providecommand{\inftynorm}[1]{\ensuremath{\left\lVert#1\right\rVert_{\infty}}}

%% Various Value functions
% \DeclareMathOperator{\valuevector}{v}
\providecommand{\valuevector}{\ensuremath{v}}
\providecommand{\qvaluevector}{\ensuremath{q}}
\providecommand{\valuefunction}[2][]{\ensuremath{\valuevector_{#1}( #2 )}}
\providecommand{\estimatedvaluefunction}[2][]{\ensuremath{\tilde{\valuevector}_{#1}( #2 )}}
\providecommand{\optimalvaluefunction}[1]{\ensuremath{v_{*}( #1 )}}
\providecommand{\qvaluefunction}[2][]{\ensuremath{q_{#1}( #2 )}}
\providecommand{\optimalqvaluefunction}[1]{\ensuremath{q_{*}( #1 )}}
\providecommand{\optimalvaluevector}{\valuevector_{*}}
\providecommand{\optimalqvaluevector}{\qvaluevector_{*}}
\providecommand{\avgreward}[2][]{\ensuremath{r_{#1}({#2})}}



\providecommand{\arrayvalue}{\ensuremath{V}}
\providecommand{\arrayvaluefunction}[2][]{\ensuremath{\arrayvalue_{#1}({#2})}}

\providecommand{\arrayqvalue}[1][]{\ensuremath{Q_{{#1}}}}
\providecommand{\arrayqvaluefunction}[2][]{\ensuremath{\estimatedqvalue[{#1}]({#2})}}

%% TODO triple check that this is a good name for the below (check the array thing above)
\providecommand{\estimatedqvalue}[1][]{\ensuremath{Q_{{#1}}}}
\providecommand{\estimatedqvaluefunction}[2][]{\ensuremath{\estimatedqvalue[{#1}]({#2})}}

\providecommand{\preferencevalue}[1][]{\ensuremath{H_{{#1}}}}
\providecommand{\preferencevaluefunction}[2][]{\ensuremath{\preferencevalue[{#1}]({#2})}}


\providecommand{\numberofvisits}[2][]{\ensuremath{N_{#1}({#2})}}


\providecommand{\approxvaluefunction}[2][]{\ensuremath{\hat{\valuevector}_{#1}({#2})}}
\providecommand{\approxqvaluefunction}[2][]{\ensuremath{\hat{\qvaluevector}_{#1}({#2})}}
\providecommand{\weightsvector}{\ensuremath{\bm{w}}} 
\providecommand{\modelweights}[1][]{\ensuremath{\weightsvector_{#1}}}
\providecommand{\policyweights}[1][]{\ensuremath{\bm{\theta}_{#1}}}
\providecommand{\weight}[1][]{\ensuremath{w_{#1}}} 
\providecommand{\statedistributionfunction}[2][]{\ensuremath{\mu_{#1}({#2})}}
\providecommand{\predictionobjfunction}[1]{\ensuremath{\bar{VE}({#1})}}
\providecommand{\featurevector}[2][]{\ensuremath{\bm{x}_{#1}({#2})}}
\providecommand{\leastsquares}[1]{\ensuremath{LS({#1})}}
\providecommand{\mean}[1]{\ensuremath{\mu({#1})}}

\providecommand{\objectiveSymbol}{\ensuremath{J}}
\providecommand{\derfunction}[2][]{\ensuremath{{\objectiveSymbol}_{#1}({#2})}}
\providecommand{\objectivefunction}[2][]{\ensuremath{{\objectiveSymbol}_{#1}({#2})}}
\providecommand{\gradient}[1][]{\ensuremath{\nabla_{#1}}}
\providecommand{\lossfunction}[2][]{\ensuremath{\mathcal{L}_{#1}({#2})}}
\providecommand{\preferencefunction}[2][]{\ensuremath{h_{#1}({#2})}}


\providecommand{\tuple}[1]{\ensuremath{\langle{#1}\rangle}}
\providecommand{\set}[1]{\ensuremath{\left\{{#1}\right\}}}

\providecommand{\highlight}[1]{{\color{red}#1}}

\DeclareMathOperator*{\argmax}{arg\,max}
\DeclareMathOperator*{\argmin}{arg\,min}

%% MDP diagrams (http://www.actual.world/resources/tex/doc/TikZ.pdf)
\usepackage{tikz}
   \usetikzlibrary{positioning,arrows,calc,trees}
   \tikzset{
   mdp/.style={>=stealth,shorten >=1pt,shorten <=1pt,auto,node distance=1.5cm,
   semithick,font=\tiny},
   state/.style={circle, draw, minimum size=0.8cm,fill=gray!15},
   transition/.style={circle, draw, minimum size=0.05cm,fill=black},
   tstate/.style={rectangle,draw,minimum size=0.8cm,fill=gray!15},
   mstate/.style={circle, draw, minimum size=0.2cm,fill=white},
   reflexive above/.style={->,loop,looseness=7,in=120,out=60},
   reflexive below/.style={->,loop,looseness=7,in=240,out=300},
   reflexive left/.style={->,loop,looseness=7,in=150,out=210},
   reflexive right/.style={->,loop,looseness=7,in=30,out=330}
   }

%% Matrices with brackets
\makeatletter
\def\bbordermatrix#1{\begingroup \m@th
  \@tempdima 4.75\p@
  \setbox\z@\vbox{%
    \def\cr{\crcr\noalign{\kern2\p@\global\let\cr\endline}}%
    \ialign{$##$\hfil\kern2\p@\kern\@tempdima&\thinspace\hfil$##$\hfil
      &&\quad\hfil$##$\hfil\crcr
      \omit\strut\hfil\crcr\noalign{\kern-\baselineskip}%
      #1\crcr\omit\strut\cr}}%
  \setbox\tw@\vbox{\unvcopy\z@\global\setbox\@ne\lastbox}%
  \setbox\tw@\hbox{\unhbox\@ne\unskip\global\setbox\@ne\lastbox}%
  \setbox\tw@\hbox{$\kern\wd\@ne\kern-\@tempdima\left[\kern-\wd\@ne
    \global\setbox\@ne\vbox{\box\@ne\kern2\p@}%
    \vcenter{\kern-\ht\@ne\unvbox\z@\kern-\baselineskip}\,\right]$}%
  \null\;\vbox{\kern\ht\@ne\box\tw@}\endgroup}
\makeatother


%% Planning stuff I might want

\DeclareMathOperator{\cost}{cost}
\DeclareMathOperator{\pre}{pre}
\DeclareMathOperator{\post}{post}
\DeclareMathOperator{\vars}{vars}
% \newtheorem{definition}{Definition}
% \newtheorem{proposition}{Proposition}
% \newtheorem{corollary}{Corollary}
% \newtheorem{theorem}{Theorem}
\providecommand\tuple[1]{\ensuremath\langle#1\rangle}
\providecommand\Z{\ensuremath{\mathbb{Z}}}
\providecommand\R{\ensuremath{\mathbb{R}}}
\providecommand\N{\ensuremath{\mathbb{N}}}
\providecommand\Y[1]{\ensuremath{\mathsf{Y}_{#1}}}
\providecommand\C{\ensuremath{\mathsf{C}}}

%% Definitions for the paper
\DeclareMathOperator{\variables}{\mathcal{V}}
\DeclareMathOperator{\operators}{\mathcal{O}}

\DeclareMathOperator{\planningtask}{\Pi}
\DeclareMathOperator{\goalconditions}{\Gamma}
\DeclareMathOperator{\plan}{\pi}
\DeclareMathOperator{\optimalplan}{\pi^{*}}
\DeclareMathOperator{\plancost}{\mathit{cost}}
\DeclareMathOperator{\occur}{\mathit{occur}}
\providecommand\initialstate{\ensuremath{s_{0}}}
\providecommand\goalstate{\ensuremath{s^{*}}}

\providecommand\h{\ensuremath{h}}
\providecommand\hoptimal{\ensuremath{h^{*}}}
\providecommand\hip{\ensuremath{h}^{\textup{IP}}}

\DeclareMathOperator{\grtask}{\Pi_{\goalconditions}^{\observations}}
\DeclareMathOperator{\grsolution}{\Gamma^{*}}
\DeclareMathOperator{\observations}{\Omega}
\DeclareMathOperator{\noisyobservations}{\Omega^{N}}
% \DeclareMathOperator{\obsevation}{o}
\providecommand\varvalue[2]{\ensuremath{#1[#2]}}
\providecommand\obs[1]{\ensuremath{\vec{#1}}}

\providecommand\variableoccurrences{\ensuremath{\mathcal{K}}}
\providecommand\lpvariables{\ensuremath{\mathcal{Y}}}
\providecommand\constraints{\ensuremath{C}}
\providecommand\uncertainty{\ensuremath{\mu}}
% \providecommand\unreliability{\ensuremath{\mathcal{R}}}
\providecommand\unreliability{\ensuremath{\epsilon}}

\providecommand\theory{\mathbb{T}}
\providecommand\goals{\mathcal{G}}
\providecommand\goal{g}
%% End of Definitions
