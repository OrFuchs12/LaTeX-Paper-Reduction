\section{Conclusion}

In this paper, we studied the automation of product placement in retail settings. The problem is motivated by the fact that well placed products can maximize impulse buys and minimize search costs for consumers. Solving this allocation problem is difficult because location-based, historical data is limited in most retail settings. Consequently, the number of possible allocation strategies is massive compared to the number of strategies typically explored in historical data. Additionally, it is generally costly to experiment and explore new policies because of the economic costs of sub optimal strategies, and operational cost of deploying a new allocation strategy. Therefore, we propose a probabilistic environment model called that is designed to mirror the real world, and allow for automated search, simulation and exploration of new product allocation strategies. We train the proposed model on real data collected from two large retail environments. We show that the proposed model can make accurate predictions on test data. Additionally, we do a preliminary study into various optimization methods using the proposed model as a simulator. We discover that Deep $Q$-learning techniques can learn a more effective policy than baselines. On average, DQN offers an improvement of 24.5\% over Tabu search in terms of cumulative test reward.