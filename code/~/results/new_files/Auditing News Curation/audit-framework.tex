\section{Audit Framework}\label{sec-framework}

In this section we develop and motivate a conceptual framework to guide our audit of Apple News, which we intend to capture important aspects of news curation systems. The framework is especially motivated by the way these systems can express editorial values and potentially influence personal, political, and/or economic dimensions of society through phenomena such as information overload, agenda-setting, and/or impacting  revenue for publishers.

It has become increasingly clear that algorithmic tools are ``heterogeneous and diffuse sociotechnical systems, rather than rigidly constrained and procedural formulas'' \citep{Seaver2017}. Therefore, our framework accounts for the role humans may play in such tools, and not merely their technical components. Contrary to the notion of ``neutral'' or ``purely technical,'' systems inevitably embed, embody, and propagate human values and biases \citep{Weizenbaum1976,Friedman1996,Introna2000,ONeil}, such as, in this case, editorial values about what and when to publish. As gatekeepers, news curation systems operate levers that  influence news distribution and consumption. These levers include the algorithms, content policies, and presentation design (i.e. layout) of the system.

Based on our review of prior studies and audits of platforms, we observe at least three important aspects of an intermediary such as Apple News that an audit should examine: (1) mechanism, (2) content, and (3) consumption. These three dimensions loosely align with the goals of better understanding (1) editorial bias arising as a result of a technological apparatus (i.e. mechanism), (2) exposure to information provided through a sociotechnical editorial process (i.e. content availability), and (3) attention patterns which encompass behavior of end users (i.e. consumption). Below is the framework along with some motivating questions:

\begin{itemize}
\item \textbf{Mechanism}
\begin{itemize}
\item How often does the curator update the content displayed? When do updates typically occur?
\item To what extent does the curator adapt content across users: is content personalized, localized, or uniform?
\end{itemize}

\item \textbf{Content}
\begin{itemize}
\item What sources does the curator display most often?
\item What topics does the curator display most often?
\end{itemize}

\item \textbf{Consumption}
\begin{itemize}
\item How does an appearance in the curator affect the attention given to a story?

\end{itemize}
\end{itemize}

\subsubsection{Mechanism}
While particularities of the mechanism may seem inconsequential, aspects such as update frequency and degree of adaptation play a crucial role in determining the type and quality of content a person sees in the curator. For example, the curator's update frequency can determine whether it emphasizes \textit{recent} content or \textit{relevant} content \citep{Chakraborty2015}, that is, information of present importance or longer-term importance. The churn rate for popularity-ranked content lists (such as Trending Stories) also effects the quality of that content, because high-quality content may not ``bubble up'' if the list is updated too frequently \citep{Salganik,Chaney2019,Ciampaglia2018}. Finally, more frequent updates increase the overall throughput of information, contributing to a phenomenon commonly known as ``information overload,'' whereby excessive amounts of content harms a person's ability to make sense of information \citep{Himma,Bawden2009}.

The political implications of news personalization mechanisms have also been pointed out in a wave of concern about the various threats to democratic efficacy posed by the ``filter bubble'' \citep{Pariser,Bozdag2015,Helberger2018,Flaxman}. These concerns range from diminished individual autonomy and undermined civic discourse to a lack of diverse exposure to information which may limit awareness and ability to contest ideas. While these concerns have recently been met with evidence that the filter bubble phenomenon is relatively minimal in practice \citep{Haim2018,Trilling2018}, evolving algorithmic curators need ongoing investigation to characterize exposure to differences of ideas or sources \citep{Diakopoulos2019}.

\subsubsection{Content}
In addition to mechanistic qualities such as update frequency and adaptation to different users, a curator's inclusion and exclusion of content is also of interest. The selection of news headlines represents ``a hierarchy of moral salience'' \citep{Schudson1995}. This is because many editorial decisions play a role in agenda-setting, sending topics from the mass media's attention into the general public's attention \citep{McCombs2005}. This relationship to the public discourse means that a curation system's content can have substantive political implications.

The content in a given curation system can be evaluated for source diversity, content diversity, and exposure diversity: variation in a system's information providers, topics/ideas/perspectives represented, and audience consumption behavior, respectively \citep{Napoli2011}. Investigating the diversity of news curators helps in understanding how they are impacting public discourse and democratic efficacy \citep{Nechushtai2019}. For instance, exposure diversity can help sustain democracy by supporting individual autonomy, creating a more inclusive public debate, and introducing new ways of contesting ideas \citep{Helberger2018}.

In addition to political implications, the sources of content in a curation system such as Apple News also has economic implications. For instance, publishing businesses have come to recognize the economic benefits of ranking well in Google search results, which drives more clicks, advertisement views, and therefore more revenue. Similarly, a story that ``ranks well'' in Apple News converts to revenue, at least to some extent, for the story's publisher. Since all 85 million users see the same content in several prominent sections of the app (e.g., Top Stories, Top Videos), an appearance in one of these sections can lead to a boost in traffic which can in turn increase revenue \citep{Nicas2018,Oremus2018,Dotan2018}.

\subsubsection{Consumption}
The potential for impact on user exposure to diverse content as well as on revenue for publishers makes it helpful to know about actual news \textit{consumption} resulting from a news curator, namely, how an appearance in the curator affects the attention given to a story. While monetizing in-app traffic has proven difficult thus far, publishers report substantial website traffic from Apple News referrals, which is already influencing revenue streams \citep{Nicas2018,Oremus2018,Tran,Dotan2018,Weiss}.  The current study, however, does not investigate such effects because Apple heavily restricts access to the relevant data, especially data about in-app behavior. Future work may consider alternative methodological approaches, such as panel studies, surveys, or data donations, to better understand the attentional effects of Apple News compared to other systems. For example, in an audit study of the Google Top Stories box, \cite{Trielli} were able to obtain timestamped referral data for 2,639 news articles and quantify the increase in referral traffic associated with appearances in different positions of the Top Stories box.

Having clarified how the mechanism, content, and consumption effects of news curation systems can have personal, political, and economic impact, we now apply this conceptual audit framework to Apple News. Again, because Apple closely guards the data needed to directly investigate specific effects on news consumption, our audit of Apple News focuses particularly on mechanism and content. 