\section{Apple News \label{sec:apple-news}}
In this section, we contextualize the audit with an overview of Apple News' origin, evolution, and functionality.

A massive potential readership, low entry costs, an editorial staff, and declining traffic from other platforms are a few factors that have made Apple News a prominent figure in the news ecosystem within the last four years. Apple released the app in September 2015, making it just a few taps away from more than 1 billion iOS devices. Apple reported more than 85 million monthly active users in early 2019 \citep{Feiner2019}. As the publishing environment shifts, referral traffic from Apple News to publishers has surged \citep{Oremus2018,Tran}. Facebook's modified News Feed algorithms, which promote friends over news outlets, left a void that media executives hope Apple News can fill \citep{Weiss}. Besides the large iOS user base and potential for traffic, Apple provides an accessible publishing workflow. Publishers can create their own channel(s) and release articles using either the Apple News Format, RSS, or their own content management system. However, while these are flexible options for basic integration with Apple News, the editors only features content if it is in Apple News Format \citep{AppleInc.}.

%This blend of human curation and algorithmic curation in Apple News raises several research questions about the news intermediary which we investigate in this paper.

%Publishers have noted that although the staff's decisions can be similarly inscrutable at times, human curation allows them to better control traffic to their content compared to the effort required by Facebook's algorithms \citep{Oremus2018}.



%In response, Cook has explicitly positioned Apple’s approach to news as a countermeasure to this craziness. While algorithms determine trending stories and user-specific suggested topics, human editors curate the list of top stories in the app \citep{Nicas}.   As Apple CEO Tim Cook said in June of 2018, the news ecosystem was ``kind of going a little crazy.''



\begin{figure}[!t] 
    \begin{center}
        
        \includegraphics[width=0.22\textwidth]{top.png}    		\includegraphics[width=0.22\textwidth]{trending.png}
    \end{center}
    \caption{Two screenshots from the Apple News app taken on an iPhone X simulator. The \textit{Top} Stories section is shown on the left, and the \textit{Trending} Stories section on the right.}
    \label{fig-screenshots}
\end{figure}


The Apple News app features a variety of content: top stories, popular stories, and stories relevant to a user's personal interests. Even though Apple has released several redesigns of the app, including one in March 2019, the various designs consistently make some sections more important and impactful. The primary tab (titled ``Today'') shows headlines and topical news from the current day. The second tab generally featured longer-form ``digest'' material before being renamed to ``News+'' in March 2019, when it was also redesigned to feature magazine content. A third tab (renamed from ``Channels'' to ``Following'' in March 2019) displays a list of publishers and news topics to view, like, or block. %Beginning in June 2018 and lasting through the United States midterm elections in November 2018, the app featured a ``Midterms' tab with stories curated by the editorial staff.

Notably, the primary tab highlights five aggregated ``Top Stories'' from Apple's editorial staff, which selects from about 100-200 pitches each day \citep{Nicas2018}. Thus, rather than a search engine or a news feed algorithm determining the prominence of a given story, the most prominent stories in Apple News are chosen by a staff of human editors. This editorial approach is a distinctive feature, which Apple CEO Tim Cook positioned as a way to combat the ``craziness'' of digital news, adding that human curation is not intended to be political, but instead to ensure the platform avoids ``content that strictly has the goal of enraging people'' \citep{Burch2018}. Still, while human editors curate the Top Stories section of the app, algorithms are used to determine what appears in the Trending Stories section \citep{Nicas2018}.

The Trending Stories section appears in the ``Today'' tab, below the Top Stories section (see Figure \ref{fig-screenshots}), and prominently displays a list of algorithmically-selected content. The first two Top Stories and the first two Trending Stories also appear in the Apple News widget, which is displayed by default when swiping left on the first iOS home screen.

Since at least 85 million users regularly see content in Top Stories and Trending Stories, we aim to better understand the mechanisms behind them and the content they display. By performing a focused audit on these two sections, we hope to characterize some of the ways Apple News is mediating human attention in today's news ecosystem.

