\def\year{2021}\relax
\documentclass[letterpaper]{article}
\usepackage{adjustbox}
 % DO NOT CHANGE THIS
\usepackage{aaai21}  % DO NOT CHANGE THIS
\usepackage{times}  % DO NOT CHANGE THIS
\usepackage{helvet} % DO NOT CHANGE THIS
\usepackage{courier}  % DO NOT CHANGE THIS
\usepackage[hyphens]{url}  % DO NOT CHANGE THIS
\usepackage{graphicx} % DO NOT CHANGE THIS
\urlstyle{rm} % DO NOT CHANGE THIS
\def\UrlFont{\rm}  % DO NOT CHANGE THIS
\usepackage{natbib}  % DO NOT CHANGE THIS AND DO NOT ADD ANY OPTIONS TO IT
\usepackage{caption} % DO NOT CHANGE THIS AND DO NOT ADD ANY OPTIONS TO IT
\frenchspacing  % DO NOT CHANGE THIS
\setlength{\pdfpagewidth}{8.5in}  % DO NOT CHANGE THIS
\setlength{\pdfpageheight}{11in}  % DO NOT CHANGE THIS


\usepackage{cite}
\usepackage{amssymb}
\usepackage{amstext}
\usepackage{subfigure}
\usepackage{booktabs}
\usepackage{enumitem}
\usepackage{multirow}
\usepackage[switch]{lineno}





\setcounter{secnumdepth}{2} %May be changed to 1 or 2 if section numbers are desired.




\title{A Graph-based Relevance Matching Model for Ad-hoc Retrieval}
\author{\textbf{Yufeng Zhang\textsuperscript{\rm 1}\thanks{Equal contribution}, Jinghao Zhang\textsuperscript{\rm 1,\rm 2}\footnotemark[1], Zeyu Cui\textsuperscript{\rm 1,\rm 2}, Shu Wu\textsuperscript{\rm 1,\rm2,\rm 3}\thanks{Corresponding author} and Liang Wang\textsuperscript{\rm 1,\rm 2}} \\
}
\affiliations{\textsuperscript{\rm 1}Institute of Automation, Chinese Academy of Sciences \\
  \textsuperscript{\rm 2}University of Chinese Academy of Sciences \\
  \textsuperscript{\rm 3}Artificial Intelligence Research, Chinese Academy of Sciences \\
  \texttt{\{yufeng.zhang,jinghao.zhang\}@cripac.ia.ac.cn} \\ \texttt{\{zeyu.cui,shu.wu,wangliang\}@nlpr.ia.ac.cn}\\}

\begin{document}

\maketitle

\begin{abstract}
To retrieve more relevant, appropriate and useful documents given a query, finding clues about that query through the text is crucial. Recent deep learning models regard the task as a term-level matching problem, which seeks exact or similar query patterns in the document. However, we argue that they are inherently based on local interactions and do not generalise to ubiquitous, non-consecutive contextual relationships. In this work, we propose a novel relevance matching model based on graph neural networks to leverage the document-level word relationships for ad-hoc retrieval. In addition to the local interactions, we explicitly incorporate all contexts of a term through the graph-of-word text format. Matching patterns can be revealed accordingly to provide a more accurate relevance score. Our approach significantly outperforms strong baselines on two ad-hoc benchmarks. We also experimentally compare our model with BERT and show our advantages on long documents.



\end{abstract}

\section{Introduction}
%%%%%%%%%%%%%%%%%%%%%%%%%%%%%%
% 1.定义image captioning任务 
%%%%%%%%%%%%%%%%%%%%%%%%%%%%%%
Image captioning is a fundamental task in vision-language understanding that involves generating natural language descriptions for a given image. It plays a critical role in facilitating more complex vision-language tasks, such as visual question answering \cite{Agrawal2015VQAVQ,gqa,okvqa} and visual dialog \cite{Das2016VisualD,Niu2018RecursiveVA,llava}.
%%%%%%%%%%%%%%%%%%%%%%%%%%%%%%
% text-only training 的介绍
%%%%%%%%%%%%%%%%%%%%%%%%%%%%%%
The mainstream image captioning methods \cite{conimgcap4,conimgcap1,conimgcap3,conimgcap2} require expensive human annotation of image-text pairs for training neural network models in an end-to-end manner. Recent developments in Contrastive Image Language Pre-training (CLIP) \cite{clip} have enabled researchers to explore a new paradigm, zero-shot image captioning, through text-only training. In particular, CLIP learns a multi-modal embedding space where semantically related images and text are encoded into features with close proximity. As such, if a model learns to map the CLIP text features to their corresponding texts, it is feasible to generate image captions from the CLIP image features without needing supervision from caption annotations.

%%%%%%%%%%%%%%%%%%%%%%%%%%%%%%
% text-only training 的优势
%%%%%%%%%%%%%%%%%%%%%%%%%%%%%%

One main advantage of this zero-shot captioning paradigm is that it enables a Large Language Model (LLM) \cite{gpt3, Zhang2022OPTOP} with image captioning capabilities using only text data and affordable computational resources. Despite the impressive performance achieved by recent powerful multimodal models \cite{miniGPT4,llava}, they typically require large-scale, high-quality human-annotated data and expensive computational resources for fine-tuning an LLM. Zero-shot captioning methods can significantly reduce such costs, which is particularly important in situations of data scarcity and limited resources. Moreover, recent work \cite{Guo2022FromIT, Changpinyo2022AllYM,Tiong2022PlugandPlayVZ} demonstrates that other vision-language tasks, such as VQA, can be addressed by LLMs and image captions. Consequently, the paradigm of zero-shot captioning has the potential to pave the way to solving complex vision-language tasks with LLMs through efficient text-only training. 


%%%%%%%%%%%%%%%%%%%%%%%%%%%%%%
% zero-shot image captioning via text-only training 的challenge
%%%%%%%%%%%%%%%%%%%%%%%%%%%%%%
A critical challenge in zero-shot image captioning through text-only training is to mitigate a widely observed phenomenon known as the \textit{modality gap}. While the features of paired texts and images are close in the CLIP embedding space, there remains a gap between them \cite{MindGap}. This gap often results in inaccurate mappings from the image embeddings to the text ones. Consequently, without fine-tuning with paired data, it significantly impairs the performance of zero-shot image captioning.
%%%%%%%%%%%%%%%%%%%%%%%%%%%%%%
% current works intro
%%%%%%%%%%%%%%%%%%%%%%%%%%%%%%
Several works have attempted to address the modality gap in zero-shot image captioning, relying mainly on two strategies: (1) The first strategy leverages a memory bank from training text data to project visual embeddings into the text embedding space \cite{DeCap}. However, this projection prevents it from representing any semantic content outside the distribution of the memory bank features and introduces extra inference costs; (2) The second approach injects noise during training to encourage the visual embeddings to be included inside the semantic neighborhood of the corresponding text embeddings \cite{CapDec}. Nonetheless, the noise injection tends to diffuse the distribution of visual inputs at the cost of weakening the semantic correlation between paired images and text embeddings. 

%However, in the first strategy, the projection of visual embeddings prevents them from  For the second strategy, noise injection during training diffuses the input distribution at the cost of degrading the semantic correlation between paired images and text embeddings.

%Previous attempts \cite{CapDec,DeCap} to reduce the modality gap in zero-shot image captioning can be summarized into two aspects: (1) Decap\cite{DeCap} leverages a memory bank from training text data to project visual embeddings into text embedding space. However, the projection of visual embeddings prevents it from representing any semantic content outside the distribution of the memory bank and introduce extra inference cost. (2) CapDec\cite{CapDec}proposes to inject noise during training to encourage the visual embedding to be included inside the text embedding space. 
% current work weakness
%Nevertheless, noise injection during training diffuses the input distribution at the cost of degrading the semantic correlation between paired images and text embeddings.


%%%%%%%%%%%%%%%%%%%%%%%%%%%%%%
% 我们工作的流程
% 分析得到两个结论:1.subregion带来更好的匹配2.image text gap符合高斯分布
%%%%%%%%%%%%%%%%%%%%%%%%%%%%%%
To tackle these challenges, we first conduct a thorough analysis of the CLIP feature space, leading to two key observations. First, most text descriptions are unable to fully capture the content of their paired images. However, we empirically find that the visual embedding of certain local regions of an image, named image subregions, have closer proximity to the text embedding of the paired caption. Integrating such image subregions with the global image representation generates a tighter alignment between image and text. Additionally, we analyze the distribution of the gap between the CLIP features of image or subregion-text pairs and find that it closely resembles a zero-mean Gaussian distribution.
%initiate our investigation by conducting a thorough analysis of the CLIP latent space. Building upon the insights from the work \cite{MindGap}, we identify a key factor contributing to the existence of a modality gap. Due to the inherent disparities between textual and visual modalities, text is incapable of comprehensively describing the information within an image. However, we empirically demonstrate that the CLIP embedding of some part of image, named image subregions, exhibit closer proximity to the CLIP embedding of the paired caption. The integration between image subregion information and global image feature leads to more compact image text alignment. Besides, we collect the statistics of the gap between CLIP image and text feature. The results demonstrate the gap is close to gaussian distribution. 

%%%%%%%%%%%%%%%%%%%%%%%%%%%%%%
% 我们的方法简略介绍
%%%%%%%%%%%%%%%%%%%%%%%%%%%%%%

Based on our findings, we propose a novel zero-shot image captioning framework, named \textit{\textbf{M}ining Fine-Grained Image-Text \textbf{A}lignment in \textbf{C}LIP for \textbf{Cap}tioning} (MacCap), to address the aforementioned challenges. In this framework, we introduce a region-aware cross-modal representation based on CLIP and an effective unimodal training strategy for an LLM-based caption generator. Our cross-modal representation maps an input image into the language space of LLMs and consists of two main components. First, we design a \textit{sub-region feature aggregation} module to fuse both global and subregion-level CLIP image features, resulting in a smaller gap between the corresponding CLIP text embedding. Next, we introduce a learnable adaptor-decoder to transform the CLIP representation into the LLM's language space.
To train our model with text-only data, we develop a robust procedure to learn a projection from the CLIP embedding space to a language representation, enabling the LLM to reconstruct captions. Specifically, our learning procedure first injects noise into our region-aware CLIP-text representation, mimicking the modality gap between image and text features. This is followed by a multiple sampling and filtering step that leverages the CLIP knowledge to improve the quality of the captioning.
%tackles the problem from three key perspectives. Firstly, we focus on learning a robust projection from CLIP embedding space to language model space by text reconstruction training, which enable model to generate text based on both CLIP image and text feature. The region noise injection in training alleviate the \textit{modality gap} between image and text feature, which makes the projection works for both image and text features. Secondly, we design \textit{sub-region feature aggregation} to obtain a more compact CLIP image feature, which is based on the observation that CLIP subregion feature exhibit closer disntance with corresponding text feature. Third, we propose multiple sampling and filtering to mitigate the drawbacks of noise injection, which leverage CLIP knowledge to further boost caption performance. Finally, we design a pipeline for zero-shot VQA to demonstrate the extensibility of ouir methods to more intricate vision-language tasks.
In addition to the image captioning task, we further extend our framework to build a zero-shot VQA pipeline, demonstrating the generality of our cross-modal representation for more complex vision-language tasks.

%%%%%%%%%%%%%%%%%%%%%%%%%%%%%%
% 我们的方法简略介绍
%%%%%%%%%%%%%%%%%%%%%%%%%%%%%%

We evaluate our framework on several widely-adopted image captioning benchmarks, such as MSCOCO \cite{mscoco} and Flickr30k \cite{Flickr30k}, as well as a standard VQA benchmark, VQAV2 \cite{vqav2}. Our extensive experiments cover multiple vision-language tasks, including zero-shot in-domain image captioning, zero-shot cross-domain image, and zero-shot VQA. The results not only demonstrate the superiority of our methods but also validate our findings on the CLIP embedding space.

% demonstrate through experiments that our proposed methods outperform previous approaches on popular captioning benchmarks, such as MSCOCO, Flickr30k, which further verify our understanding of \textit{concept region}



% Specifically, we evaluate the distribution of the image and text embedding space under hyperspherical coordinates and observe a geometric phenomenon \textit{concept region} 
% where semantically correlated image and text embedding tend to clustering despite the \textit{modality gap}.
% 我们基于concept region的观察提出的方法:concept region和modality gap的cause里面有mismatch pair data导致的semantic ambiguity,总体思路是在train的时候模拟在concept region。在training的时候,我们给text embedding加上region noise,具体而言就是以原本text embedding为中心,一定范围内的多个随机sample的related text embedding,这样的获得的text embedding全都是在输入text对应的concept region内部。在zs captioning的inference时,部分image sub-region inforamtion 会比global image 对text匹配度更高,因此我们基于部分image sub-region inforamtion
% Motivated by the semantic ambiguity of mismatched data observed in \textit{concept region}, we propose two 
% an image sub-region information aggregation strategy for .In detail

% result summary


\section{Related Work}
In this section, we briefly review some existing neural matching models and graph neural networks.

\subsection{Neural Matching Models}
Most neural matching models fall within two categories: representation-focused models, e.g. DSSM \cite{huang2013learning}, ARC-I \cite{hu2014convolutional}, CDSSM \cite{shen2014latent}, and interaction-focused models, e.g. MatchPyramid \cite{pang2016text}, DRMM \cite{guo2016deep}, PACRR \cite{hui2017pacrr}, KNRM \cite{xiong2017end}.

The representation-focused models follow the representation learning approach adopted in many natural language processing tasks. Queries and documents are projected into the same semantic space individually. The cosine similarity is then used between their high-level text representations to produce the final relevance score. For example, DSSM \cite{huang2013learning}, one of the earliest neural relevance matching models, employs simple dense neural layers to learn high-level representations for queries and documents. To enhance the projecting function, ARC-I \cite{hu2014convolutional} and CDSSM \cite{shen2014latent} devoted much effort into convolutional layers later on. 

In comparison, interaction-focused methods model the two text sequences jointly, by directly exploiting detailed query-document interaction signals rather than high-level representations of individual texts. For example, DRMM \cite{guo2016deep} maps the local query-document interaction signals into a fixed-length histogram, and dense neural layers are followed to produce final ranking scores. \citet{xiong2017end} and \citet{dai2018convolutional} both use kernel pooling to extract multi-level soft match features. Many other works rely on convolutional layers or spatial GRU over interaction signals to extract ranking features  
such as \cite{pang2016text,pang2017deeprank,hui2017pacrr,hui2018co,fan2018modeling}, which considers just local word connections. 

There are also several studies investigating how to apply BERT in ranking, e.g.  \citet{dai2019deeper} and \citet{macavaney2019cedr}. A common approach is to concatenate the document and query text together and feed them into the next sentence prediction task, where the `[CLS]' token embeds the representation of the query-document pair. 
\begin{figure*}[h]
	\centering
	\includegraphics[width=\textwidth]{./pics/grmm.pdf}
	\caption{The workflow of the GRMM model. The document is first transformed into the graph-of-word form, where the node feature is the similarity between the word and each query term. Then, graph neural networks are applied to propagate these matching signals on the document graph. Finally, to estimate a relevance score, top-$k$ signals of each query term are chosen to filter out irrelevant noisy information, and their features are fed into a dense neural layer. }
	\label{fig:2} 
\end{figure*}

Nevertheless, the majority of existing neural matching models only take the linear text sequence, inevitably limiting the model capability. To this end, we propose to break the linear text format and represent the document in a flexible graph structure, where comprehensive interactions can be explicitly modeled. 



\subsection{Graph Neural Networks}
Graph is a kind of data structure which cooperates with a set of objects (nodes) and their relationships (edges). Recently, researches of analysing graphs with machine learning have attracted much attention because of its great representative power in many fields. 

Graph neural networks (GNNs) are deep learning based methods that operate in the graph domain. The concept of GNNs is previously proposed by  \cite{scarselli2008graph}. Generally, nodes in GNNs update own hidden states by aggregating neighbourhood information and mixing things up into a new context-aware state. There are also many variants of GNNs with various kinds of aggregators and updaters, such as \cite{li2016gated,kipf2017semi,hamilton2017inductive,velivckovic2018graph}. 

Due to the convincing performance and high interpretability, GNNs have become a widely applied structural analysis tool. Recently, there are many applications covering from recommendation \cite{wu2019session,li2019fi} to NLP area, including text classification \cite{yao2019graph,zhang2020every}, question answering \cite{de2019question}, and spam review detection \cite{li2019spam}.

In this work, we employ GNNs in the relevance matching task to extract implicit matching patterns from the query-document interaction signals, which is intrinsically difficult to be revealed by existing methods. 



\section{Proposed Method}

In this section, we introduce thoroughly our proposed Graph-based Relevance Matching Model (GRMM). We first formulate the problem and demonstrate how to construct the graph-of-word formation from the query and document, and then describe the graph-based matching method in details. Figure \ref{fig:2} illustrates the overall process of our proposed architecture.

\subsection{Problem Statement}
Given a query $q$ and a document $d$, they are represented as a sequence of  words 
$q=\left[w_{1}^{(q)}, \ldots, w_{M}^{(q)}\right]$  and $d=\left[w_{1}^{(d)}, \ldots, w_{N}^{(d)}\right]$, where $w_{i}^{(q)}$ denotes the $i$-th word in the query, $w_{i}^{(d)}$ denotes the $i$-th word in the document, $M$ and $N$ denote the length of the query and the document respectively.
The aim is to compute a relevance score $rel(q,d)$ regarding the query words and the document words.


\subsection{Graph Construction}
\label{sec:graphconstruct}
To leverage the long-distance term dependency information, the first step is to construct a graph $\mathcal{G}$ for the document. It typically consists of two components denoted as $\mathcal{G}=(\mathcal{V}, \mathcal{E})$, 
where $\mathcal{V}$ is the set of vertexes with \emph{node features}, and $\mathcal{E}$ is the set of edges as the \emph{topological structure}.

\subsubsection{Node features.}
We represent each unique word instead of sentence or paragraph in the document as a node. Thus the word sequence is squeezed to a node set $\left\{w_{1}^{(d)}, \ldots, w_{n}^{(d)}\right\}$, where $n$ is the number of unique words in the document ($|\mathcal{V}| = n  \leq N$). Each node feature is set the interaction signal between its word embedding and query term embeddings. We simply employ the cosine similarity matrix as the interaction matrix, denoted as $\mathbf{S} \in \mathbb{R}^{n \times M}$, where each element $\mathbf{S}_{ij}$ between document node $w^{(d)}_i$ and query term $w^{(q)}_j$ is defined as:
\begin{equation}\mathbf{S}_{i j}=cosine\left(\mathbf{e}_i^{(d)}, \mathbf{e}_j^{(q)}\right)
\end{equation}
where $\mathbf{e}_{i}^{(d)}$ and $\mathbf{e}_{j}^{(q)} $ are embedding vectors for $w_{i}^{(d)}$ and $w_{j}^{(q)}$ respectively. In this work, we use word2vec \cite{mikolov2013distributed} technique to convert words into dense and semantic embedding vectors.

\subsubsection{Topological structure.}
In addition to the node feature matrix, the adjacency matrix representing the topological structure constitutes for the graph as well. The structure generally describes the connection between the nodes and reveals their relationships. We build bi-directional connections for each pair of word nodes that co-occur within a sliding window, along with the original document word sequence $d$. By restricting the size of the window, every word can connect with their neighbourhood words which may share related contextual meanings. However, GRMM differs from those local relevance matching methods in that the combined word node can bridge all neighbourhoods together and therefore possess a document-level receptive field. In other words, it breaks the constraints of local context and can model the long-distance word dependencies that we concern. Note that in the worst case where there are no duplicate words, the graph would still perform as a sequential and local scheme. 

Formally, the adjacency matrix $\mathbf{A} \in \mathbb{R}^{n \times n}$ is defined as:
\begin{equation}
\mathbf{A}_{i j}=\left\{\begin{array}{ll}
count(i, j) & \text{if } i \not= j \\
0 & \text{otherwise}
\end{array}\right.
\end{equation}
where $count(i, j)$ is the number of times that the words $w_{i}^{(d)}$ and $w_{j}^{(d)}$ appear in the same sliding window. To alleviate the exploding/vanishing gradient problem \cite{kipf2017semi}, we normalise the adjacency matrix as $\tilde{\mathbf{A}} = \mathbf{D}^{-\frac{1}{2}} \mathbf{A} \mathbf{D}^{-\frac{1}{2}}$, where $\mathbf{D} \in \mathbb{R}^{n \times n}$ is the diagonal degree matrix and $\mathbf{D}_{ii} = \sum_j \mathbf{A}_{ij}$.



\subsection{Graph-based Matching}
Once we obtain the graph $\mathcal{G}$, we focus on making use of its node features and structure information with graph neural networks. In particular, the query-document interaction and the intra-document word interaction are learned mutually following the procedures - \emph{neighbourhood aggregation}, \emph{state update} and \emph{feature election}. 

\subsubsection{Neighbourhood Aggregation.}
As discussed in Section \ref{sec:graphconstruct}, we initialise the node state $\mathbf{h}^0_i$ with the query-document interaction matrix:
\begin{equation}\mathbf{h}^0_i =  \mathbf{S}_{i,:}
\end{equation}
where $\forall i\in [1, n]$ denotes the $i$-th node in the graph, and $\mathbf{S}_{i,:}$ is the $i$-th row of the interaction matrix $\mathbf{S}$.

Assume each word node either holds the core information or serves as a bridge connecting others, it is necessary to make the information flow and enrich the related fractions on the graph.
Through propagating the state representations to a node from its neighbours, it can receive the contextual information within the first-order connectivity as:
\begin{equation}\mathbf{a}_{i}^{t}=\sum_{(w_{i}, w_{j}) \in \mathcal{E}} \mathbf{\tilde{A}}_{ij} \mathbf{W}_{a} \mathbf{h}_{j}^{t}\end{equation}
where $\mathbf{a}_i^t \in \mathbb{R}^{M}$ denotes the summed message from neighbours, $t$ denotes the current timestamp, and $\mathbf{W}_a$ is a trainable transformation matrix to project features into a new relation space. When aggregate $t$ times recursively, a node can receive the information propagated from its $t$-hop neighbours. In this way, the model can achieve \emph{high-order aggregation} of the query-document interaction as well as the intra-document interaction.

\subsubsection{State Update.}
To incorporate the contextual information into the word nodes, we engage a GRU-like function \cite{li2016gated} to automatically adjust the merge proportion of its current representation $\mathbf{h}^{t}_i$ and the received representation $\mathbf{a}^{t}_i$, which is formulated as:
\begin{equation}\begin{array}{l}
\mathbf{z}_{i}^{t}=\sigma\left(\mathbf{W}_{z} \mathbf{a}_{i}^{t}+\mathbf{U}_{z} \mathbf{h}_{i}^{t}+\mathbf{b}_{z}\right)
\end{array}\end{equation}
\begin{equation}
\mathbf{r}_{i}^{t}=\sigma\left(\mathbf{W}_{r} \mathbf{a}_{i}^{t}+\mathbf{U}_{r} \mathbf{h}_{i}^{t}+\mathbf{b}_{r}\right)
\end{equation}
\begin{equation}\tilde{\mathbf{h}}_{i}^{t}=\tanh \left(\mathbf{W}_{h} \mathbf{a}_{i}^{t}+\mathbf{U}_{h}\left(\mathbf{r}_{i}^{t} \odot \mathbf{h}_{i}^{t}\right)+\mathbf{b}_{h}\right)\end{equation}
\begin{equation}\mathbf{h}_{i}^{t+1}=\tilde{\mathbf{h}}_{i}^{t} \odot \mathbf{z}_{i}^{t}+\mathbf{h}_{i}^{t} \odot\left(1-\mathbf{z}_{i}^{t}\right)\end{equation}
where $\sigma(\cdot)$ is the sigmoid function, $\odot$ is the Hardamard product operation, tanh$(\cdot)$ is the non-linear tangent hyperbolic activation function, and all $\mathbf{W_*, U_*}$ and $\mathbf{b_*}$ are trainable weights and biases. 

Specifically, $\mathbf{r}^{t}_i$ determines irrelevant information for hidden state $\tilde{\mathbf{h}}^{t}_i$ to forget (reset gate), while $\mathbf{z}^{t}_i$ determines which part of past information to discard and which to push forward (update gate). With the layer $t$ going deep, high-order information becomes complicated, and it is necessary to identify useful dependencies with the two gates. We have also tried plain updater such as GCN \cite{kipf2017semi} in our experiments but did not observe satisfying performance due to its simplicity.


\subsubsection{Graph Readout.}
The last phase involves locating the position where relevance matching happens as a delegate for the entire graph. Since it is suggested that not all words make contributions, and some may cause adverse influences \cite{guo2016deep}, here we only select the most informative features to represent the query-document matching signals. Intuitively, higher similarity means higher relevance possibility. Hence we perform a $k$-max-pooling strategy over the query dimension and select the top $k$ signals for each query term, which also prevents the model from being biased by the document length. The formulas are expressed as:
\begin{equation}\mathbf{H}=\mathbf{h}_{1}^{t} \parallel \mathbf{h}_{2}^{t} \parallel \ldots \parallel \mathbf{h}_{n}^{t}\end{equation}
\begin{equation}
\mathbf{x}_{j} = {topk}(\mathbf{H}_{:,j})
\end{equation}
where $\forall j\in [1, M]$ denotes the $j$-th query term, and $\mathbf{H}_{:,j}$ is the $j$-th column of the feature matrix $\mathbf{H}$.

\subsection{Matching Score and Training}
After obtaining low-dimensional and informative matching features $\mathbf{x}_j$, we move towards converting them into actual relevance scores for training and inference. Considering different terms may have different importances \cite{guo2016deep}, we assign each with a soft gating network as:
\begin{equation}g_{j}=\frac{\exp \left({c} \cdot idf_j \right)}{\sum_{j=1}^{M} \exp \left({c} \cdot idf_j \right)}\end{equation}
where $g_j$ denotes the term weight, $idf_j$ is the inverse document frequency of the $j$-th query term, and $c$ is a trainable parameter. To reduce the amount of parameters and avoid over-fitting, we score each query term with a weight-shared multi-layer perceptron (MLP) and sum them up as the final result:
\begin{equation}{rel}(q, d)=\sum_{j=1}^{M} g_j \cdot \tanh \left(\mathbf{W}_x \mathbf{x}_{j}+{b}_x \right)\end{equation}
where $\mathbf{W}_x, b_x$ are trainable parameters for MLP.

Finally, we adopt the pairwise hinge loss which is commonly used in information retrieval to optimise the model parameters:
\begin{small}
	\begin{equation}\mathcal{L}\left(q, d^{+}, d^{-}\right)=\max \left(0, 1-rel\left(q, d^{+}\right)+rel\left(q, d^{-}\right)\right)\end{equation} 
\end{small}
where $\mathcal{L}\left(q, d^{+}, d^{-}\right)$ denotes the pairwise loss based on a triplet of the query $q$, a relevant (positive) document sample $d^+$, and an irrelevant (negative) document sample $d^-$.



\section{EXPERIMENTS}
In this section, we conduct experiments on two widely used datasets to answer the following research questions:
\begin{itemize}
	\item RQ1: How does GRMM perform compared with different retrieval methods (typically traditional, local interaction-based, and BERT-based matching methods)?
	\item RQ2: How effective is the graph structure as well as the long-dependency in ad-hoc retrieval?
	\item RQ3: How sensitive (or robust) is GRMM with different hyper-parameter settings?
\end{itemize}

\subsection{Experiment Setup}
\subsubsection{Datasets.}
We evaluate our proposed model on two datasets: Robust04 and ClueWeb09-B.
\begin{itemize}
    \item Robust04\footnote{https://trec.nist.gov/data/cd45/index.html} is a standard ad-hoc retrieval dataset with 0.47M documents and 250 queries, using TREC disks 4 and 5 as document collections.
    \item ClueWeb09-B\footnote{https://lemurproject.org/clueweb09/} is the "Category B" subset of the full web collection ClueWeb09. It has 50M web pages and 200 queries, whose topics are accumulated from TREC Web Tracks 2009-2012.
\end{itemize}
Table \ref{tab:1} summarises the statistic of the two collections. For both datasets, there are two available versions of the query: a keyword title and a natural language description. In our experiments, we only use the title for each query.


\subsubsection{Baselines.}
To examine the performance of GRMM, we take three categories of retrieval models as baselines, including traditional (QL and BM25), local interaction-based (MP, DRMM, KNRM, and PACRR), and BERT-based (BERT-MaxP) matching methods, as follows: 

\begin{itemize}
    \item \textbf{QL} (Query likelihood model) \cite{zhai2004study} is one of the best performing language models that based on Dirichlet smoothing.
    \item \textbf{BM25} \cite{robertson1994some} is another effective and commonly used classical probabilistic retrieval model.
    \item \textbf{MP} (MatchPyramid) \cite{pang2016text} employs CNN to extract the matching features from interaction matrix, and dense neural layers are followed to produce final ranking scores.
    \item \textbf{DRMM} \cite{guo2016deep} performs a histogram pooling over the local query-document interaction signals. 
    \item \textbf{KNRM} \cite{xiong2017end} introduces a new kernel-pooling technique that extracts multi-level soft matching features.
    \item \textbf{PACRR} \cite{hui2017pacrr} uses well-designed convolutional layers and $k$-max-pooling layers over the interaction signals to model sequential word relations in the document.
    \item \textbf{Co-PACRR} \cite{hui2018co} is a context-aware variant of PACRR that takes the local and global context of matching signals into account.
    \item \textbf{BERT-MaxP} \cite{dai2019deeper} applies BERT to provide deeper text understanding for retrieval. The neural ranker predicts the relevance for each passage independently, and the document score is set as the best score among all passages.
\end{itemize}


\begin{table}[]
	\footnotesize
	\begin{tabular}{@{}ccccc@{}}
		\toprule
		\textbf{Dataset}     & \textbf{Genre} & \textbf{\# of qrys} & \textbf{\# of docs} & \textbf{avg.length} \\ \midrule
		\textbf{Robust04}    & news           & 250                 & 0.47M                & 460                         \\
		\textbf{ClueWeb09-B} & webpages       & 200                 & 50M                 & 1506                        \\ \bottomrule
	\end{tabular}
	\caption{Statistics of datasets.}
	\label{tab:1}
\end{table}


\subsubsection{Implementation Details.}
All document and query words were white-space tokenised, lowercased, and lemmatised using the WordNet\footnote{https://www.nltk.org/howto/wordnet.html}. We discarded stopwords as well as low-frequency words with less than ten occurrences in the corpus. Regarding the word embeddings, we trained 300-dimensional vectors with the Continuous Bag-of-Words (CBOW) model \cite{mikolov2013distributed} on Robust04 and ClueWeb-09-B collections. For a fair comparison, the other baseline models shared the same embeddings, except those who do not need. Implementation of baselines followed their original paper.

Both datasets were divided into five folds. We used them to conduct 5-fold cross-validation, where four of them are for tuning parameters, and one for testing \cite{macavaney2019cedr}. The process repeated five times with different random seeds each turn, and we took an average as the performance.

We implemented our method in PyTorch\footnote{Our code is at https://github.com/CRIPAC-DIG/GRMM}. The optimal hyper-parameters were determined via grid search on the validation set: the number of graph layers $t$ was searched in \{1, 2, 3, 4\}, the $k$ value of $k$-max-pooling was tuned in \{10, 20, 30, 40, 50, 60, 70\}, the sliding window size in \{3,5,7,9\}, the learning rate in \{0.0001, 0.0005, 0.001, 0.005, 0.01\}, and the batch size in \{8, 16, 32, 48, 64\}.
Unless otherwise specified, we set $t$ = 2 and $k$ = 40 to report the performance (see Section \ref{sec:neighbouraggre} and \ref{sec:featureelect} for different settings), and the model was trained with a window size of 5, a learning rate of 0.001 by Adam optimiser for 300 epochs, each with 32 batches times 16 triplets. All experiments were conducted on a Linux server equipped with 8 NVIDIA Titan X GPUs.

\subsubsection{Evaluation Methodology.}
Like many ad-hoc retrieval works, we adopted a re-ranking strategy that is more efficient and practical than ranking all query-document pairs. In particular, we re-ranked top 100 candidate documents for each query that were initially ranked by BM25. To evaluate the re-ranking result, we used the normalised discounted cumulative gain at rank 20 (nDCG@20) and the precision at rank 20 (P@20) as evaluation matrices. 


\subsection{Model Comparison (RQ1)}
Table \ref{tab:2} lists the overall performance of different models, from which we have the following observations:
\begin{itemize}
	\item GRMM significantly outperforms traditional and local interaction-based models, and it is comparable to BERT-MaxP, though without massive external pre-training. To be specific, GRMM advances the performance of nDCG@20 by 14.4\% on ClueWeb09-B much more than by 5.4\% on Robust04, compared to the best-performed baselines excluding BERT-MaxP. It is reasonably due to the diversity between the two datasets. ClueWeb09-B contains webpages that are usually long and casual, whereas Robust04 contains news that is correspondingly shorter and formal. It suggests that useful information may have distributed non-consecutively, and it is beneficial to capture them together, especially for long documents. GRMM can achieve long-distance relevance matching through the graph structure regardless of the document length. 
	
	\item On the contrary, BERT-MaxP performs relatively better on Robust04 than on ClueWeb09-B. We explain the observation with the following two points. First, since the input sequence length is restricted by a maximum of 512 tokens, BERT has to truncate those long documents from ClueWeb09-B into several passages. It, therefore, loses relations among different passages, i.e. the long-distance dependency. Second, documents from Robust04 are generally written in formal languages. BERT primarily depends on the pre-trained semantics, which could naturally gain benefit from that. 
	
	\item Regarding the local interaction-based models, their performances slightly fluctuate around the initial ranking result by BM25. However, exceptions are DRMM and KNRM on ClueWeb09-B, where the global histogram and kernel pooling strategy may cause the difference. It implies that the local interaction is insufficient in ad-hoc retrieval task. Document-level information also needs to be considered. 
	
	\item Traditional approaches like QL and BM25 remain a strong baseline though quite straightforward, which means the exact matching of terms is still of necessity as \citet{guo2016deep} proposed. These models also avoid the problem of over-fitting, since they do not require parameter optimisation. 
\end{itemize}                       

\label{sec:modelcompare}
\begin{table}[]
	\fontsize{9.3pt}{11pt}\selectfont
    \begin{tabular}{@{}cllll@{}}
    \toprule
    \multirow{2}{*}{Model} & \multicolumn{2}{c}{Robust04}                           & \multicolumn{2}{c}{ClueWeb09-B}                        \\ \cmidrule(l){2-5} 
                           & \multicolumn{1}{c}{nDCG@20} & \multicolumn{1}{c}{P@20} & \multicolumn{1}{c}{nDCG@20} & \multicolumn{1}{c}{P@20} \\ \midrule
    QL                     & 0.415$^-$                   & 0.369$^-$                & 0.224$^-$                   & 0.328$^-$                \\
    BM25                   & 0.418$^-$                   & 0.370$^-$                & 0.225$^-$                   & 0.326$^-$                \\ \midrule
    MP                     & 0.318$^-$                   & 0.278$^-$                & 0.227$^-$                   & 0.262$^-$                \\
    DRMM                   & 0.406$^-$                   & 0.350$^-$                & 0.271$^-$                   & 0.324$^-$                \\
    KNRM                   & 0.415$^-$                   & 0.359$^-$                & 0.270$^-$                   & 0.330$^-$                \\
    PACRR                  & 0.415$^-$                   & 0.371$^-$                & 0.245$^-$                   & 0.278$^-$                \\
    Co-PACRR               & 0.426$^-$                   & 0.378$^-$                & 0.252$^-$                   & 0.289$^-$                \\ \midrule
    BERT-MaxP              & \textbf{0.469}                       & -                        & 0.293                       & -                        \\ \midrule
    GRMM                   & 0.449                        & \textbf{0.387}                    & \textbf{0.310}                       & \textbf{0.354}                    \\ \bottomrule
    \end{tabular}
	\caption{Performance comparison of different methods. The best performances on each dataset and metric are highlighted. Significant performance degradation with respect to GRMM is indicated (-) with p-value $\leq$ 0.05.}
	\label{tab:2}
\end{table}

\subsection{Study of Graph Structure (RQ2)}
\label{sec:graphstructure}
To dig in the effectiveness of the document-level word relationships of GRMM, we conduct further ablation experiments to study their impact. Specifically, we keep all settings fixed except substituting the adjacency matrix with: 
\begin{itemize}
	\item \textbf{Zero matrix}: Word nodes can only see themselves, and no neighbourhood information is aggregated. This alternative can be viewed as not using any contextual information. The model learns directly from the query-document term similarity.
	\item \textbf{Word sequence}, the original document format: No words are bound together, and they can see themselves as well as their previous and next ones. This alternative can be viewed as only using local contextual information. It does not consider long-distance dependencies. 
\end{itemize}


\begin{figure}[h]
	\centering
	\includegraphics[width=.47\textwidth]{./pics/graph_ablation.png}
	\caption{Ablation study on graph structure of GRMM.}
	\label{fig:3} 
\end{figure}

Figure \ref{fig:3} illustrates the comparison between the original GRMM and the alternatives. We can see that:
\begin{itemize}
    \item GRMM (zero matrix) performs inferior to others in all cases. Since it merely depends on the junior term similarities, the model becomes approximate to term-based matching. Without contextualised refinement, some words and their synonyms can be misleading, which makes it even hard to discriminate the actual matching signals. 
    \item GRMM (word sequence) promotes GRMM (zero matrix) by fusing local neighbourhood information but still underperforms the original GRMM by a margin of 2-3 points. This observation resembles some results in Table \ref{tab:2}. It shows that such text format could advantage local context understanding but is insufficient in more comprehensive relationships. 
    \item  From an overall view of the comparison, the document-level word relationships along the graph structure is proved effective for ad-hoc retrieval. Moreover, a relatively greater gain on ClueWeb09-B indicates that longer texts can benefit more from the document-level respective field.
\end{itemize}

\subsection{Study of Neighbourhood Aggregation (RQ2 \& RQ3)}
\label{sec:neighbouraggre}
Figure \ref{fig:4} summarises the experimental performance w.r.t a different number of graph layers. The idea is to investigate the effect of high-order neighbourhood aggregations. For convenience, we notate GRMM-0 for the model with no graph layer, GRMM-1 for the model with a single graph layer, and so forth for the others. From the figure, we find that:

\begin{figure}[h]
	\centering
	\includegraphics[width=.47\textwidth]{./pics/num_of_layers.png}
	\caption{Influence of different graph layer numbers.}
	\label{fig:4} 
\end{figure}

\begin{itemize}
	\item GRMM-1 dramatically boosts the performance against GRMM-0. This observation is consistent with Section \ref{sec:graphstructure} that propagating the information within the graph helps to understand both query-term interaction and document-level word relationships. The exact/similar query-document matching signals are likely to be strengthened or weakened according to intra-document word relationships. 
	\item GRMM-2 improves, not as much though, GRMM-1 by incorporating second-order neighbours. It suggests that the information from 2-hops away also contributes to the term relations. The nodes serving as a bridge can exchange the message from two ends in this way.
	\item However, when further stacking more layers, GRMM-3 and GRMM-4 suffer from slight performance degradation. The reason could be nodes receive more noises from high-order neighbours which burdens the training of parameters. Too much propagation may also lead to the issue of over-smooth \cite{kipf2017semi}. A two-layer propagation seems to be sufficient for capturing useful word relationships.
	\item Overall, there is a tremendous gap between using and not using the contextual information, and the model peaks at layer $t$ = 2 on both datasets. The tendency supports our hypothesis that it is essential to consider term-level interaction and document-level word relationships jointly for ad-hoc retrieval. 
\end{itemize}

\subsection{Study of Graph Readout (RQ3)}
\label{sec:featureelect}
\begin{figure}[h]
	\centering
	\includegraphics[width=.47\textwidth]{./pics/k.png}
	\caption{Influence of different $k$ values of $k$-max pooling.}
	\label{fig:5} 
\end{figure}

We also explored the effect of graph readout for each query term. Figure \ref{fig:5} summarises the experimental performance w.r.t different $k$ values of $k$-max-pooling. From the figure, we find that: 
\begin{itemize}
	\item The performance steadily grows from $k$ = 10 to $k$ = 40, which implies that a small feature dimension may limit the representation of terms. By enlarging the $k$ value, the relevant term with more matching signals can distinguish from the irrelevant one with less. 
	\item The trend, however, declines until $k$ = 70, which implies that a large feature dimension may bring negative influence. It can be explained that a large $k$ value may have a bias to the document length, where longer documents tend to have more matching signals. 
	\item Overall, there are no apparent sharp rises and falls in the figure, which tells that GRMM is not that sensitive to the selection of $k$ value. Notably, almost all performances (except $k$ = 70) exceed the baselines in Table \ref{tab:2}, suggesting that determinative matching signals are acquired during the graph-based interactions before feeding into the readout layer. 
\end{itemize}

% \vspace{-1em}
\section{Conclusions}
% \vspace{-1em}
In this paper, we introduced a benchmark task for commonsense reasoning that aims at uncovering unspoken intents that humans can easily uncover in a given statement by making presumptions supported by their common sense. In order to solve this task, we propose
CORGI (COmmon-sense ReasoninG by Instruction),  a neuro-symbolic theorem prover that performs commonsense reasoning by initiating a conversation with a user. CORGI has access to a small knowledge base of commonsense facts and completes it as she interacts with the user. We further conduct a user study that indicates the possibility of using conversational interactions with humans for evoking commonsense knowledge and verifies the effectiveness of our proposed theorem prover.
% We defined common-sense reasoning as the process of finding a chain of reasoning in a logic program given an if/then/because statement. We showed that obtaining the because statement is crucial in extracting a relevant chain of reasoning given an if/then statement. Moreover, we introduced a soft backward chaining algorithm that allows us to combat variations in natural language by learning embeddings for the facts and rules in the knowledge base. This algorithm combines symbolic AI with neural approaches allowing us to bridge a gap between symbolic AI and the recent advances in deep learning.

\section*{Acknowledgements}
This work is supported by National Key Research and Development Program (2018YFB1402605, 2018YFB1402600), National Natural Science Foundation of China (U19B2038, 61772528), and Beijing National Natural Science Foundation (4182066).

Repudiandae provident neque, eum consequatur fugiat aliquam earum, quibusdam molestiae nesciunt aperiam non sapiente mollitia iste, quaerat delectus in harum iusto voluptate error eligendi dolor, fugit aperiam est voluptatem?Quos iusto nulla ut, aliquid incidunt ad reprehenderit rem voluptate exercitationem eum vel molestiae ratione, expedita sit pariatur autem placeat quae tempore, eaque laborum quia libero nulla culpa accusamus?Ut quae nam illum, iste inventore saepe, obcaecati maiores blanditiis libero tenetur similique itaque, hic est fugit repudiandae in perferendis quasi?Sit repudiandae deleniti corporis neque velit quaerat maiores est ut repellat enim, voluptate amet
\bibliography{ref.bib}
\end{document}