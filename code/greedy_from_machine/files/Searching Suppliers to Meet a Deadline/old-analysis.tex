

is 
We conjecture that the \samd problem is NP-hard. The reason for this is due to the following observations:

\begin{enumerate}
\item Computing the probability that with a given supplier assignment we will meet the project deadline, i.e., computing $M(\Pi, \varphi, d)$, is NP-hard.
\item The solution space of all possible supplier assignments is exponential, and a dynamic programming approach does not work because \samd does not have the \emph{optimal substructure} property. %partial assignments are not optimal substructures, which prevents the use of dynamic programming.
\end{enumerate}
Next, we explain these observations in more details. 

\paragraph{Hardness of the $M$ function.}
Computing the probability that a \emph{task network} satisfies a deadline is known to be NP-hard \cite[Theorem 5]{cohen2015estimating}. 
A task network consists of a set of tasks that must be performed, where some of these tasks should be performed in sequence and some in parallel. In fact, even if the task network consists only of a sequence of tasks, computing the probability that it satisfies a deadline is still NP-hard \cite[Lemma 5]{cohen2015estimating}. 

Note that a given supplier assignment $\varphi$ in an \samd problem is equivalent to a task network that consists solely of a sequence of tasks. %Therefore, even the computation of $M(\Pi, \varphi, d)$ for a single \samd supplier assignment $\varphi$ is already NP-hard.
\begin{corollary}
The computation of $M(\Pi, \varphi, d)$ is NP-hard.
\label{cor:m}
\end{corollary}
The above corollary refers to the complexity of assessing a single supplier assignment. We believe that the \samd problem is at least as hard, since we expect that one needs to assess a \emph{set} of possible supplier assignments in order to find the one that optimizes the probability to meet the deadline. %achieve a deasolves the optimization problem at hand.





%\section{Problem Complexity}
\section{Theoretical Analysis}
[[Roni: I think we should move this section later, near the end of the paper, to a ``Discussion'' section]]

\paragraph{The \samd search space.}
We conjecture that \samd is still NP-hard, even in the presence of some \emph{oracle} that can compute $M(\Pi, \varphi, d)$ instantly. This conjecture is based on the fact that 
\emph{partial} supplier assignments are not necessarily  optimal substructures. 


A brute-force solution to an \samd problem $\langle 
\Pi, S, X, d \rangle$ is to enumerate all possible supplier assignments, 
compute the value of $M(\Pi,\varphi, d)$ for each assignment $\varphi$ and return the assignment that yields the maximal value. Observe that the number of possible supplier assignments is exponential in the number of tasks $(|T|)$, and for each assignment we need to compute $M(\Pi, \varphi, d)$, which, according to Corollary~\ref{cor:m}, is a computationally hard problem on its own. 


Consider the example from Section~\ref{sec:def} with the deadline 2, i.e., the \samd problem $\langle \Pi, S, X, 2\rangle$. The optimal supplier assignment for this problem is $\varphi_{2,4}$. Now, consider a reduced problem that consists of only the first task with $\Pi'=\{T_1\}$, $S'=\{s_1,s_2\}$, and the task completion distributions $X'$, which are equal to those of $X$ for the suppliers of the reduced problem. The optimal supplier assignment for $\langle \Pi', S', X', 2\rangle$ is $\varphi_{1}$ for which $M(\Pi',\varphi_{1}, 2)=1$. Obviously, supplier $s_1$ that constitutes the optimal assignment for the first task $T_1$ is not part of the optimal assignment of the extended two-task problem, hence partial supplier assignments in \samd are not necessarily optimal substructures. 

The absence of optimal substructures prevents the use of efficient dynamic programming approaches (cf. the \emph{relax} operation when solving the shortest path problem~\citep{bellman1958routing,dijkstra1959note}). This observation, together with Corollary~\ref{cor:m}, leads to our main conjecture.
\begin{conjecture}
The \samd problem is NP-hard.
\end{conjecture}
We leave the proof for future research.

%There exist, of course, many examples of problems with an exponential solution space that can be efficiently traversed. For instance, the number of possible solutions to the shortest path problem is exponential, but there exist polynomial-time algorithms that solve the problem~\citep{dijkstra1959note,bellmanford}. These algorithms apply the \emph{relax} operation that enables discarding many partial paths, which in turn leads to efficient traversal of the search space. 
%This is similar to discarding some \emph{partial} supplier assignments beforehand by identifying partial assignments that are undoubtedly inferior to others. In order to compare between partial supplier assignments we turn to the following definition.

%\begin{definition}[Pareto domination]
%A partial supplier assignment $\varphi'$ Pareto dominates a partial supplier assignment $\varphi''$ if for each ...

%\end{definition}


%Optional: discuss the DP solution that works in pseudo-polynomial time when the deadline and discretization are small. 
















\section{D}



Following our conjecture, we propose several heuristic approaches for finding task assignments that are not provably optimal, 
and propose two complete and optimal \samd 

two approaches:
\begin{itemize}
    \item Efficient suboptimal al
\end{itemize}heuristic approaches 
for finding supplier assignments that
for finding 


Parting from the above described brute-force solution, we next introduce an alternative, more efficient, way to solve \samd, based on defining \samd as a graph search problem and using the well-known \astar~\citep{hart1968formal} algorithm along with a domain-specific admissible heuristic. 
