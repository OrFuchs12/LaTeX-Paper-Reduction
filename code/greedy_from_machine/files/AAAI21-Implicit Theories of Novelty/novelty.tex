\def\year{2021}\relax
%File: formatting-instructions-latex-2021.tex
%release 2021.1
\documentclass[letterpaper]{article} % DO NOT CHANGE THIS
\usepackage{aaai21}  % DO NOT CHANGE THIS
\usepackage{times}  % DO NOT CHANGE THIS
\usepackage{helvet} % DO NOT CHANGE THIS
\usepackage{courier}  % DO NOT CHANGE THIS
\usepackage[hyphens]{url}  % DO NOT CHANGE THIS
\usepackage{graphicx} % DO NOT CHANGE THIS



\urlstyle{rm} % DO NOT CHANGE THIS
\def\UrlFont{\rm}  % DO NOT CHANGE THIS
\usepackage{natbib}  % DO NOT CHANGE THIS AND DO NOT ADD ANY OPTIONS TO IT
\usepackage{caption} % DO NOT CHANGE THIS AND DO NOT ADD ANY OPTIONS TO IT
\frenchspacing  % DO NOT CHANGE THIS
\setlength{\pdfpagewidth}{8.5in}  % DO NOT CHANGE THIS
\setlength{\pdfpageheight}{11in}  % DO NOT CHANGE THIS
 \usepackage{multirow}
%\nocopyright
%PDF Info Is REQUIRED.
% For /Author, add all authors within the parentheses, separated by commas. No accents or commands.
% For /Title, add Title in Mixed Case. No accents or commands. Retain the parentheses.
\pdfinfo{
/Title (Formalizing Novelty)
/Author (
    T. E. Boult,
    P. A. Grabowicz,
    D. S. Prijatelj,
    R. Stern,
    L. Holder,
    J. Alspector,
    M. Jafarzadeh,
    T. Ahmad,
    A. R. Dhamija,
    C. Li,
    S. Cruz, 
    A. Shrivastava,
    C. Vondrick,
    W. J. Scheirer
) 
/TemplateVersion (2021.1)
} %Leave this
% /Title ()
% Put your actual complete title (no codes, scripts, shortcuts, or LaTeX commands) within the parentheses in mixed case
% Leave the space between \Title and the beginning parenthesis alone
% /Author ()
% Put your actual complete list of authors (no codes, scripts, shortcuts, or LaTeX commands) within the parentheses in mixed case.
% Each author should be only by a comma. If the name contains accents, remove them. If there are any LaTeX commands,
% remove them.


\usepackage{times}
\usepackage{epsfig}
\usepackage{graphicx}
\usepackage{algpseudocode}
\usepackage{algorithm}
\usepackage{amsmath}
\usepackage{enumitem}
\usepackage{amssymb}
\usepackage{amsthm}
\usepackage{microtype}
\frenchspacing
\newtheoremstyle{mydefstyle}{}{}{\itshape}{}{\bfseries}{:}{.5em}{#1 #2 (\thmnote{#3})}
\theoremstyle{mydefstyle}
\newtheorem{mydef}{Definition}
\newtheorem{mytheory}{Theorem}
\newtheorem{mycorr}{Corollary}
\def\real{\mathbb{R}}
\def\posnat{\mathbb{N}^+}
\def\labs{\mathbb{N}}
\def\ints{\mathbb{Z}}
\newcommand{\argmin}{\operatornamewithlimits{argmin}}
\newcommand{\argmax}{\operatornamewithlimits{argmax}}
\newcommand{\argsort}{\operatornamewithlimits{argsort}}
\long\def\advercomment#1{}
\long\def\comment#1{}
\newcommand{\E}{\mathbb{E}}

\def\httilde{\mbox{\tt\raisebox{-.5ex}{\symbol{126}}}}

% Pages are numbered in submission mode, and unnumbered in camera-ready

\begin{document}

%%%%%%%%% TITLE
\title{Towards a Unifying Framework for Formal Theories of Novelty}
\author {
    % Authors
    T. E. Boult\textsuperscript{\rm 1},
    P. A. Grabowicz\textsuperscript{\rm 5},
    D. S. Prijatelj\textsuperscript{\rm 2},
    R. Stern\textsuperscript{\rm 6},
    L. Holder\textsuperscript{\rm 4},
    J. Alspector\textsuperscript{\rm 3},
    M. Jafarzadeh\textsuperscript{\rm 1},
    T. Ahmad\textsuperscript{\rm 1},
    A. R. Dhamija\textsuperscript{\rm 1},
    C. Li\textsuperscript{\rm 1},
    S. Cruz\textsuperscript{\rm 1},
    A. Shrivastava\textsuperscript{\rm 7},
    C. Vondrick\textsuperscript{\rm 8},
    W. J. Scheirer\textsuperscript{\rm 2} \\
}
\affiliations {
    % Affiliations
    \textsuperscript{\rm 1} U. Col. Col. Springs,
    \textsuperscript{\rm 2} U. Notre Dame,
    \textsuperscript{\rm 3} IDA/ITSD,
    \textsuperscript{\rm 4} Wash. State U.,
    \textsuperscript{\rm 5} U. Mass.,
    \textsuperscript{\rm 6} PARC, BGU, 
    \textsuperscript{\rm 7} U. Maryland,
    \textsuperscript{\rm 8} Columbia U.\\
    \{tboult $\vert$ mjafarzadeh $\vert$ tahmad $\vert$ adhamija\}@vast.uccs.edu, grabowicz@cs.umass.edu,  derek.prijatelj@nd.edu,  rstern@parc.com,  jalspect@ida.org, holder@wsu.edu,  abhinav@cs.umd.edu, vondrick@cs.columbia.edu,  walter.scheirer@nd.edu
}



\maketitle
%\thispagestyle{empty}

% %%%%%%%%% ABSTRACT
% \begin{abstract}

% Initial formulation of a theory of Perceptual Novelty
  
% \end{abstract}

% %\vspace*{-3mm}
%%%%%%%%% BODY TEXT

\begin{abstract}
Managing inputs that are novel, unknown, or out-of-distribution is critical as an agent moves from the lab to the open world. Novelty-related problems include being tolerant to novel perturbations of the normal input, detecting when the input includes novel items, and adapting to novel inputs.  While significant research has been undertaken in these areas, a noticeable gap exists in the lack of a formalized definition of novelty that transcends problem domains. As a team of researchers spanning multiple research groups and different domains,  we have seen, first hand, the difficulties that arise from ill-specified novelty problems, as well as inconsistent definitions and terminology. Therefore, we present the first unified framework for formal theories of novelty and use the framework to formally define a family of novelty types. Our framework can be applied across a wide range of domains, from symbolic AI to reinforcement learning, and beyond to open world image recognition. Thus, it can be used to help kick-start new research efforts and accelerate ongoing work on these important novelty-related problems.  
\end{abstract}



\section{Introduction} 

``What is novel?" is an important AI research question that informs the design of agents tolerant to novel inputs. 
Is a noticeable change in the world that does not impact an agent's task performance a novelty? 
How about a change that impacts performance but is not directly perceptible?  
If the world has not changed but the agent senses a random error that produces an input that leads to an unexpected state, is that novel?  

With decades of work and thousands of papers covering novelty detection and related research in anomaly detection, out-of-distribution detection, open set recognition, and open world recognition, one would think that a consistent unified definition of novelty would have been developed.  Unfortunately, that is not the case. Instead, we find a plethora of variations on this theme, as well as \textit{ad hoc} use and inconsistent reuse of terminology, all of which injects confusion as researchers discuss these topics.   

This paper introduces a unifying formal framework of novelty. The framework seeks to formalize what it means for an input to be a novelty in the context of agents in artificial intelligence or in other learning-based systems.
%This paper seeks to formalize the definition of a novel sample in the context of agents in artificial intelligence or in other learning-based systems.
Using the proposed framework, we formally define multiple types of novelty an agent can encounter.
The goal of these definitions is to be broad enough to encompass and unify the full range of novelty models that have been proposed in the literature~\cite{pimentel2014review,markou2003novelty,markou2003novelty2,openset-pami13, openworld_2015,langley2020open}. An important generalization beyond prior work is that we consider novelty in the world, observed space, and agent space (see Fig.~\ref{fig:elements}), with dissimilarity and regret operators critical to our definitions.
The overarching goal is a framework such that researchers have clear definitions for the development of agents that must handle novelty, including support for agents / algorithms that incrementally learn from novel inputs.   A longer version of this theory with example applications to three different domains can be found at~\cite{Boult-eta-al-novelty20}.



Our framework supports \textit{implicit theories of novelty}, meaning the definitions use functions to implicitly specify if something is novel. The framework does not require a way to generate novelties, but rather it provides functions that can be used to evaluate if a given input is novel. This is similar to how any 2D shape can be implicitly defined by a function $f(x,y)=0$, whether or not there is a procedure for generating the shape. 
We contend any constructive or generative theory of novelty~\cite{langley2020open} must be incomplete because the construction or generation of defined worlds, states, and any enumerable set of transformations between them form, by definition, a closed world. We note, however, that a constructive model can be consistent with our definition, but we do not require a constructive model. 




\input noveltytheory.tex


\section{Conclusion }


We see three primary contributions of this formalization of novelty that will spur further research.
First, formalization forces one to specify (or intentionally disregard) the required items in the theory.
This can lead to insights about the problem and fill in knowledge gaps.
For example, when applying the theory to the CartPole problem, numerous unanticipated issues were highlighted, new predictions made, and new experiments validated the new insights.


Second, formalization provides a common language to define and compare models of novelty across problems.
The precision of terms reduces confusion, while the flexibility allows it to be applied to a wide range of problems.


Third, the formalization allows one to make predictions about where or why experiments incorporating some form of novelty might run into difficulties.
For example, when the world-level and perceptual-level dissimilarity assessments disagree, we predict novelty problems will be more difficult.
One example of difficulty is world-disparity using variables not represented in perceptual space.
Another is when there are many possible world labels, but the input is only assigned one label that is used for assessing world-level dissimilarity.
In this case, the theory predicts a greater difficulty with such novelty, especially if the assigned label is associated with a physically smaller aspect of the observation.


Biological intelligence has a remarkable capacity to generalize novel inputs with ease, yet artificial agents continue to struggle with this behavior.
It is our hope that the adoption and use of the framework proposed here leads to the development of more effective solutions for novelty management and to make agents more robust to novel changes in their world.



By formalizing CartPole using our novelty framework, we gained insights into what are meaningful ``novelty'' problems for this task.
We showed how to develop better measures to predict when novelty would be easy or hard to manage or to detect.
In line with this, our team of researchers has been refining this theory and applying it to multiple problem domains.
More details can be found in the longer arXiv version~\cite{Boult-eta-al-novelty20}.

{\footnotesize
\section*{Acknowledgments} 
This research was  sponsored  by the Defense Advanced Research Projects Agency (DARPA) and the Army Research Office (ARO) under multiple contracts/agreements including  HR001120C0055, W911NF-20-2-0005,W911NF-20-2-0004,HQ0034-19-D-0001, W911NF2020009. The views contained in this document are those of the authors and should not be interpreted as representing the official policies, either expressed or implied, of the DARPA or ARO, or the U.S. Government.
}



{
\bibliography{novelty}
%\nocite{*}
}


\end{document}














