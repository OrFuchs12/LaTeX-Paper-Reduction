\documentclass[letterpaper]{article} %DO NOT CHANGE THIS
\usepackage{aaai23}  %Required
\usepackage[ruled,vlined,linesnumbered]{algorithm2e}
\usepackage{times}
\usepackage{helvet}
\usepackage{color}
\usepackage{graphicx}
\usepackage{courier}
\usepackage{amsthm}
\usepackage{amsmath}
\usepackage{url}
\usepackage{algorithmic}
\usepackage{subcaption}
\usepackage{todonotes}
\usepackage{multirow}
\usepackage{xcolor}

\usepackage{anyfontsize}
\usepackage{dcolumn,caption,booktabs}
\usepackage{lipsum}

\usepackage{tabu}
\usepackage{array}
\usepackage{natbib}  % DO NOT CHANGE THIS AND DO NOT ADD ANY OPTIONS TO IT
\usepackage{caption} % DO NOT CHANGE THIS AND DO NOT ADD ANY OPTIONS TO IT
\urlstyle{rm} % DO NOT CHANGE THIS
\def\UrlFont{\rm}  % DO NOT CHANGE THIS
\usepackage{arydshln}

\newcommand{\var}[1]{{\operatorname{\mathit{#1}}}}

\newtheorem{example}{Example}
\newtheorem{lemma}{Lemma}
\newtheorem{theorem}{Theorem}
%\newtheorem{definition}{Definition}


\newcommand\note[1]{\textcolor{red}{#1}}
\newcommand\commentout[1]{}

\theoremstyle{definition}
\newtheorem{definition}{Definition}[section]

\newcommand{\ronen}[1]{\textbf{\color{blue}[RONEN:#1]}}
%\newcommand{\guy}[1]{\textbf{\color{red}[GUY:#1]}}
%\newcommand{\orw}[1]{\textbf{\color{cyan}[OrWert:#1]}}

\frenchspacing
\setlength{\pdfpagewidth}{8.5in}  % DO NOT CHANGE THIS
\setlength{\pdfpageheight}{11in}  % DO NOT CHANGE THIS

\urlstyle{rm} % DO NOT CHANGE THIS
\def\UrlFont{\rm}  % DO NOT CHANGE THIS
\usepackage{natbib}  % DO NOT CHANGE THIS AND DO NOT ADD ANY OPTIONS TO IT
\usepackage{caption} % DO NOT CHANGE THIS AND DO NOT ADD ANY OPTIONS TO IT
\frenchspacing  % DO NOT CHANGE THIS
\setlength{\pdfpagewidth}{8.5in} % DO NOT CHANGE THIS
\setlength{\pdfpageheight}{11in} % DO NOT CHANGE THIS
%
% These are recommended to typeset algorithms but not required. See the subsubsection on algorithms. Remove them if you don't have algorithms in your paper.
\usepackage{algorithm}
\usepackage{algorithmic}

%
% These are are recommended to typeset listings but not required. See the subsubsection on listing. Remove this block if you don't have listings in your paper.
\usepackage{newfloat}
\usepackage{listings}
\DeclareCaptionStyle{ruled}{labelfont=normalfont,labelsep=colon,strut=off} % DO NOT CHANGE THIS
\lstset{%
	basicstyle={\footnotesize\ttfamily},% footnotesize acceptable for monospace
	numbers=left,numberstyle=\footnotesize,xleftmargin=2em,% show line numbers, remove this entire line if you don't want the numbers.
	aboveskip=0pt,belowskip=0pt,%
	showstringspaces=false,tabsize=2,breaklines=true}
\floatstyle{ruled}
\newfloat{listing}{tb}{lst}{}
\floatname{listing}{Listing}
%
% Keep the \pdfinfo as shown here. There's no need
% for you to add the /Title and /Author tags.


\newcommand\gNote[1]{\todo[inline, author=Guy, color=pink]{#1}}
%%\ccsPaper{9999} % TODO: replace with your paper number once obtained



 %for nice json code
 \usepackage{listings}
\usepackage{xcolor}
\colorlet{punct}{red!60!black}
\definecolor{background}{HTML}{EEEEEE}
\definecolor{delim}{RGB}{20,105,176}
\colorlet{numb}{magenta!60!black}
\lstdefinelanguage{json}{
    basicstyle=\normalfont\ttfamily,
    numbers=left,
    numberstyle=\scriptsize,
    stepnumber=1,
    numbersep=8pt,
    showstringspaces=false,
    breaklines=true,
    frame=lines,
    backgroundcolor=\color{background},
    literate=
     %*{0}{{{\color{numb}0}}}{1}
      %{1}{{{\color{numb}1}}}{1}
      %{2}{{{\color{numb}2}}}{1}
      %{3}{{{\color{numb}3}}}{1}
      %{4}{{{\color{numb}4}}}{1}
      %{5}{{{\color{numb}5}}}{1}
      %{6}{{{\color{numb}6}}}{1}
      %{7}{{{\color{numb}7}}}{1}
      %{8}{{{\color{numb}8}}}{1}
      %{9}{{{\color{numb}9}}}{1}
      %{:}{{{\color{punct}{:}}}}{1}
      %{<}{{{\color{punct}{<}}}}{1}
      %{>}{{{\color{punct}{>}}}}{1}
      %{=}{{{\color{punct}{=}}}}{1}
      {,}{{{\color{punct}{,}}}}{1}
      {\{}{{{\color{delim}{\{}}}}{1}
      {\}}{{{\color{delim}{\}}}}}{1}
      {[}{{{\color{delim}{[}}}}{1}
      {]}{{{\color{delim}{]}}}}{1},
} 
\lstset{
language=json,
  basicstyle=\fontsize{2}{2}\selectfont\ttfamily,
    numbers=left,
    stepnumber=1,
    showstringspaces=false,
    tabsize=1,
    breaklines=true,
    breakatwhitespace=false,
}
 % end for nice json code

%\setlength{\belowcaptionskip}{-10pt}


\pdfinfo{
    /Title (Probabilistic Programs as an Action Description Language) % change this
    /Author (Ronen I Brafman and David Tolpin and Or Wertheim) % change this
    /TemplateVersion (2021.2) % Leave this
}
%%%%%%

\setcounter{secnumdepth}{0}  

\begin{document}

\title{Demo Proposal:\\
The AOS System: Plug'n Play Task-Level Autonomy for Robotics\\ Using POMDPs and Probabilistic Programs}

\author{Or Wertheim, Dan R.~Suissa, Orel Hamamy, and Ronen I. Brafman}
%~\IEEEmembership{Staff,~IEEE,}
        % <-this % stops a space
%\thanks{This paper was produced by the IEEE Publication Technology Group. They are in Piscataway, NJ.}% <-this % stops a space
%\thanks{Manuscript received April 19, 2021; revised August 16, 2021.}}

% The paper headers
%\markboth{Journal of IEEE ROBOTICS & AUTOMATION LETTERS}%
%{Shell \MakeLowercase{\textit{et al.}}: A Sample Article Using IEEEtran.cls for IEEE Journals}
 

\maketitle

\section{The System}
A growing body of code for diverse robotic skills is available to roboticists. The AOS system describe a principled approach based on POMDPs, generative models and probabilistic programming for integrating such code into a working autonomous robot controller that can perform diverse tasks by appropriately scheduling skill-code execution and responding to new observations.
The AOS leverages skill models and POMDP solvers to make near-optimal choices
for scheduling robotic skills to achieve diverse user-specified tasks. The AOS
makes a number of conceptual contributions in the form of an explicit mapping between the more abstract decision model and the actual code and the use of  probabilistic programs for model specification. But ultimately, the AOS system aims to offer a practical contribution in the form of a true Plug'n Play experience: any skill that the robot can execute, properly documented, can be used by the controller without any additional programming effort, greatly reducing the integration effort required.

\section{Relevance to the Bridge Program}
Our system is explicitly focused on integrating AI planning techniques with robotics. Moreover, it is also interesting because of its use of probabilistic programs for modeling, providing another interesting connection between AI research and robotics.
Thus, AOS is in itself a bridge between the areas.

\section{Suggested Demo Format}
Rather than try to demo the actual system, we would like to present a pre-prepared video that will have the following parts
1. Brief overview of the system and its main ideas. 2. Videos that show a use-case on a Panda robotic arms which is used to play various variants of tic-tac-toe played by the arm against a human player. 


%\bibliographystyle{aaai23}
%\bibliography{all}

%%%
\end{document}
