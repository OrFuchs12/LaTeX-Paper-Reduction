%
\def\year{2018}\relax
\documentclass[letterpaper]{article} %DO NOT CHANGE THIS
\newcounter{bab}
\renewcommand{\thebab}{The Bounded Alpha-Beta Algorithm}

\usepackage{aaai18}  %Required
\usepackage{times}  %Required
\usepackage{helvet}  %Required
\usepackage{courier}  %Required
\usepackage{url}  %Required
\usepackage{graphicx}  %Required
\usepackage{subfig}
\usepackage{soul}


\usepackage{tikz}
\usetikzlibrary{arrows,shapes,positioning,mindmap,calc}
\tikzset{style={>=stealth'}}
\tikzset{minnode/.style={align=left,font=\footnotesize,circle,    fill=white,draw=black, minimum size=7mm,inner sep=1mm}}
\tikzset{maxnode/.style={align=left,font=\footnotesize,rectangle,fill=white,draw=black,  minimum size=6mm,inner sep=1mm}}
%\tikzset{targetk/.style={star,     fill=white,draw=red,  text=red,  minimum size=8mm,inner sep=0mm}}
\tikzset{infobox/.style={align=left,font=\footnotesize,minimum size=6mm}}

\frenchspacing  %Required
\setlength{\pdfpagewidth}{8.5in}  %Required
\setlength{\pdfpageheight}{11in}  %Required

\usepackage[ruled,vlined,linesnumbered]{algorithm2e}
\usepackage{amsmath}
\usepackage{amsthm}
\usepackage{url}
\usepackage{multirow}

%PDF Info Is Required:
\setcounter{secnumdepth}{0}


\newtheorem{definition}{Definition}
\newtheorem{lemma}{Lemma}
\newtheorem{theorem}{Theorem}
\newcommand{\ignore}[1]{}
%\newcommand{\pess}{\mathit{pess}}
%\newcommand{\opti}{\mathit{opti}}
\newcommand{\MM}{\mathit{V}}
\newcommand{\pess}{\mathit{L}}
\newcommand{\opti}{\mathit{U}}
\newcommand{\besto}{\mathit{best}_{\opti}}
\newcommand{\bestp}{\mathit{best}_{\pess}}
\newcommand{\eval}{\mathit{eval}}
\newcommand{\bab}{\mathit{BAB}}
\newcommand{\vmax}{v_{\text{max}}}
\newcommand{\vmin}{v_{\text{min}}}
\newcommand{\rootnode}{\mathit{n_1}}
\newcommand{\amb}{\textit{AMB}}
\newcommand{\er}{\textit{ER}}
\newcommand{\tuple}[1]{\langle #1 \rangle}
% correct bad hyphenation here


\begin{document}
\title{Bounded Suboptimal Game Tree Search}
\author{Dor Atzmon \and Roni Stern\\
Ben Gurion University of the Negev\\
Be'er Sheva, Israel
\And
Abdallah Saffidine \\
Australian National University\\
Canberra, Australia}

\maketitle

% As a general rule, do not put math, special symbols or citations
% in the abstract
\begin{abstract}
.......
\end{abstract}

%-----------------------------------------------------------------------------------------------------------


\section{Bounded Alpha-Beta with multiplicative bound (BAB/m)}

A bounded-suboptimal game tree search algorithm with multiplicative bound is defined as a game tree search algorithm that accepts $\gamma \geq 0$ as input and outputs a solution with suboptimality at most $\gamma \cdot \MM$ from the true minimax value $\MM$, where $\gamma$ is a user-provided parameter. In our case, to adjust BAB to this type of bound, both $\pess$ and $\opti$ must be inside these bounds as they are part of the solution, thus: $\MM - \gamma \cdot \MM \leq \pess \leq \opti \leq \MM + \gamma \cdot \MM$, or $1 - \gamma \leq \frac{\pess}{\MM} \leq \frac{\opti}{\MM} \leq 1 + \gamma$.
Algorithm \ref{alg:weightedAlphaBeta} presents the changes needed to be done in BAB in order to return a bounded solution.


%Describing the pseudo code + improvement
 \refstepcounter{wab}
 \label{sec:bab}

\begin{algorithm}[t]
\DontPrintSemicolon
%\KwIn{$(n, p)$ - a game tree node and its parent}
\KwIn{$n$ - a game tree node}
%\KwIn{$(\alpha_p, \beta_p)$ - the $\alpha$ and $\beta$ values of the parent node}
\KwIn{$\epsilon$ - an error margin}

\If{$n$ is a terminal node}{
  \nllabel{line:term-start}
  %$\alpha = \max(\alpha_p, \eval(n))$; $\beta = \min(\beta_p, \eval(n))$\;
  \Return $( \MM(n), \MM(n) )$\;
  \nllabel{line:term-end}
}

$\pess(n)\gets \vmin$;
$\opti(n)\gets \vmax$\; \nllabel{line:init-start}
  \lIf{$n$ is a MAX node}{
  	$\besto \gets \vmin$
  }\lElseIf{$n$ is a MIN node}{
    $\bestp\gets \vmax$
  }

  \lIf{$n$ is $\rootnode$ \nllabel{line:alpha-update-start}}{
     $\alpha(n)=\vmin$; $\beta(n)=\vmax$
  }
  \lElseIf{$p(n)$ is a MAX node or a MIN node \nllabel{line:alpha-update-start}}{
     $\alpha(n)=\alpha(p(n))$; $\beta(n)=\beta(p(n))$
  }
  \Else{
    $\alpha(n)\gets \max(\vmin, \frac{\alpha(p(n))-\opti(p(n))}{\pi(n)} + \opti(n))$\;
    $\beta(n)\gets \min(\vmax, \frac{\beta(p(n))-\pess(p(n))}{\pi(n)} + \pess(n))$\;
    \nllabel{line:init-end}
  }

\ForEach{child $c$ of $n$}{
  \nllabel{line:loop-start}
  %$\zeta\gets$ ComputeChildBound($\epsilon$, $\opti(n)$,$\pess(n)$,$\pi(c)$)\;
  $\big(\pess(c),\opti(c)\big)\gets$ WeightedAB($c$, $\epsilon$) \nllabel{line:recursive}\;

  \If{$n$ is a MAX node}{
    \nllabel{line:lu-update-start}
    $\besto\gets \max(\besto,\opti(c))$\;
    $\pess(n)\gets \max(\pess(n),\pess(c))$\;
%    $\opti(n)\gets \max(\opti(n),\opti(c))$\; Always vmax
  }
  \ElseIf{$n$ is a MIN node}{
    $\bestp\gets \min(\bestp,\pess(c))$\;
%    $\pess(n)\gets \min(\pess(n),\pess(c))$\;  Always vmin
	$\opti(n)\gets \min(\opti(n),\opti(c))$\;
    \nllabel{line:lu-update-end-det}
    }\Else{
    $\pess(n)\gets \pess(n)+\pi(c)(\pess(c)-\vmin)$\; \nllabel{line:lu-update-chance-start}
%    $\opti(n)\gets \opti(n)+\pi(\opti(c)-\vmax)$\;
	$\opti(n)\gets \opti(n)-\pi(c)(\vmax-\opti(c))$
    \nllabel{line:lu-update-end}
    \;
  }

  $\alpha(n)\gets \max(\pess(n), \alpha(n))$\;
  $\beta(n)\gets \min(\opti(n), \beta(n))$\;

    \textcolor{red}{
    $\epsilon\gets w\cdot \min(|\alpha(n)|,|\beta(n)|)$\;
}

  \lIf{$\beta(n)\leq \alpha(n)+\epsilon$ and $c$ is not the last child of $n$}{
    \Return $(\pess(n),\opti(n))$
    \nllabel{line:break-loop}
  }
}
\lIf{$n$ is a MAX node}{
  \nllabel{line:update-best-start}
  $\opti(n)\gets \besto$
}\lElseIf{$n$ is a MIN node}{
  $\pess(n)\gets \bestp$
  \nllabel{line:update-best-end}
}

\Return $(\pess(n),\opti(n))$\;

\caption{WeightedAB}
\label{alg:weightedAlphaBeta}
\end{algorithm}....


%-----------------------------------------------------------------------------------------------------------

\section{Worst / best case of BAB}



%-----------------------------------------------------------------------------------------------------------

\section{Game plying with BAB}



%-----------------------------------------------------------------------------------------------------------

\section{Improvements and comparison (null window search, history heuristic, and more)}



%-----------------------------------------------------------------------------------------------------------

% old text ------------------------
\ssection{Choosing a Bounded-Suboptimal Action}
\label{sec:action}
BAB (Algorithm~\ref{alg:weightedAlphaBeta}) returns a range $[\pess(\rootnode), \opti(\rootnode)]$ for a given game tree and $\epsilon$. As proven in Theorem~\ref{the:basic}, the minimax value is in that range, and the size of the range is at most $\epsilon$. Thus, any value chosen in that range can serve as a bounded-suboptimal solution for the given $\epsilon$.

To obtain a concrete policy from the output of BAB the MAX player needs to choose the action that leads to the highest $\pess$ values. This policy is guaranteed (on expectation) to provide at least a bounded-suboptimal minimax value, for any action the MIN player may choose. By contrast, choosing the action that leads to the highest $\opti$ values may end up with a strategy whose expected value is smaller than the minimax value by more than $\epsilon$. %Similarly, the MIN player can choose the action that leads to the lowest $\opti$ values in order to guarantee a bounded-suboptimal minimax value. [[Roni: I don't see the value in this, and it is confusing - the reader may be tempetd to thing we assume that the MIN player MUST do this. ]]

\section{$\epsilon$-equilibrium}
The optimal policy for MAX and MIN players that use classical Alpha-Beta pruning is known to be a Nash equilibruim, that is, no player can gain by unilaterally deviating from its action. Obviously, a the policy proposed in~"\ref{sec:bab}" section obtained by BAB in not necessarily a Nash equilibruim. An  $\epsilon$-equilibrium~\cite{nisan2007algorithmic} is a joint policy in which no player can gain more than $\epsilon$ by unilaterally deviating from his strategy. Proving this conjucture is a topic for future research.

\begin{theorem}
For a game tree $G$ and an $\epsilon$ bounded solution, playing the policy proposed in~"\ref{sec:bab}" results in $\epsilon$-equilibrium
\end{theorem}
\begin{proof}
%Assume there is a node $n$ that by the policy in~"\ref{sec:bab}" child $c^*$ is selected and that exist other child $c'$ such that the player can gain more than $\epsilon$ by deviating from choosing $c^*$.

Assume by the policy in~"\ref{sec:bab}" child $c'$ is selected.

%$\MM(c')-\MM(c) \geq \epsilon$.
\paragraph{Case \#1: $n$ is a MAX node.}
By theorem \ref{theorem_error_bound}, for each child of $n$
\begin{align}
    \opti(c) - \pess(c) \leq \epsilon
\end{align}
And
\begin{align}
   \MM(c) \leq \opti(c)
\end{align}
By the policy in~"\ref{sec:bab}"
\begin{align}
    \max\limits_{c\in C(n)} \pess(c) =  L(c') \leq \MM(c')
\end{align}
Using (2) and (3) in (1) we get that for each child $c$ of $n$
\begin{align}
    \MM(c) - \MM(c') \leq \epsilon
\end{align}

And thus playing the policy proposed in~"\ref{sec:bab}" results in $\epsilon$-equilibrium
\end{proof}


%-----------------------------------------------------------------------------------------------------------


\section{Related Work}
....




\section{Acknowledgements}
This research was supported by the Israel Ministry of Science, the Czech Ministry of Education, and by ISF grant \#210/17 to Roni Stern.

\clearpage
\bibliography{games}
\bibliographystyle{aaai}
\end{document}

