
% It is immediate that satisfying assignments to $\cnf_{\mathcal{T}}$ correspond to encodings of action models and the complete trajectories those action models would yield, given the values observed in the trajectories of $\mathcal{T}$. We note that the clauses of this formula each only contain literals for a single fluent, so the formula is satisfiable iff the formulas $\cnf_{\mathcal{T}}(l)$ for each literal $l$ are satisfiable.



% \begin{enumerate}
% \item $\ispre(l,a_i)\rightarrow (l\in s_{i-1})$
% \item $\iseff(l,a_i)\rightarrow (l\in s_i)$
% \item $\neg\iseff(l,a_i)\rightarrow (s_{i-1}[l]=s_i[l])$

% % \item $\neg \ispre(l,a_i)\vee \state(l,i-1,T)$
% % \item $\neg \iseff(l,a_i)\vee \state(l,i,T)$
% % \item $\neg \state(l,i-1,T)\vee \iseff(\neg l,a_i)\vee \state(l,i,T)$
% \item $\neg \ispre(l,a_i)\vee (l\in s_{i-1})$
% \item $\neg \iseff(l,a_i)\vee (l\in s_i)$
% \item $(\neg l\in s_{i-1})\vee \iseff(\neg l,a_i)\vee (l\in s_i)$
% \end{enumerate}


% a CNF, $\cnf_{\mathcal{T}}$ defined over variables of the form $\iseff(l,a)$, $\ispre(l,a)$, and $\state(l,i,T)$, representing that 
% $l$ is a precondition of $a$, 
% $l$ is an effect of $a$, 
% and $l=\true$ in the $i^{th}$ state of trajectory $T$, respectively.  
% The clauses for $\cnf_\mathcal{T}$ are obtained from clausal encodings of the STRIPS axioms, instantiated at each step of each trajectory in $\mathcal{T}$:
% \begin{enumerate}
% \item $\neg \ispre(l,a_i)\vee State(l,i-1,T)$
% \item $\neg \iseff(l,a_i)\vee State(l,i,T)$
% \item $\neg State(l,i-1,T)\vee \iseff(\neg l,a_i)\vee State(l,i,T)$
% \end{enumerate}
% with $State(l,i,T)$ replaced by true or false when $l$ is observed true or false, respectively, at step $i$ in $T$. 
% It is immediate that satisfying assignments to $\cnf_{\mathcal{T}}$ correspond to encodings of action models and the complete trajectories those action models would yield, given the values observed in the trajectories of $\mathcal{T}$. We note that the clauses of this formula each only contain literals for a single fluent, so the formula is satisfiable iff the formulas $\cnf_{\mathcal{T}}(l)$ for each literal $l$ are satisfiable.

We then use the refutation-completeness of resolution to reduce the problem to identifying the clauses that may appear in resolution refutations. The $\ispre(a,l)$ literals, appearing only negatively, cannot appear in a refutation, and the literals $State(l,i,T)$ must be eliminated, but these only appear in consecutive instances of the second and third type of clauses where the literal $l$ is unobserved; and only the third clause has the negative literal. Reordering the applications of the resolution rule on these literals to the beginning of the proof, we see that we must create clauses that correspond to consecutive runs of unobserved literals using the resolution rule on the third type of clause for each step, beginning with either an observed literal or with using the second type of clause to eliminate the first $State(l,i,T)$ literal. These are, respectively, the clauses of $CNF_{\eff}(l)$ created on lines~\ref{epi:refute1} and~\ref{epi:refute2} in Algorithm~\ref{alg:episam-effects}. %4--5 and 6--7. See the supplemental material for details.
% \roni{I am not super happy with this proof. Don't we need a claim about the preconditions?} \brendan{OK, properly the statement should only concern the effects part of the model. But remember that we can always find a consistent set of preconditions by just saying there are no preconditions, so the question about preconditions is not important for the existence of a consistent action model.}
\iffalse{
%% Full proof below punted to supplemental
\begin{proof}
We first consider the following CNF encoding of the possible trajectories that could yield the observations appearing in $\mathcal{T}$: let $CNF_{\mathcal{T}}$ be a formula with variables $\iseff(l,a)$ for each literal $l$ and action $a$, $\ispre(l,a)$ for each literal $l$ and action $a$, and $State(l,t,h)$ for each literal $l$, trajectory $h\in\mathcal{T}$, and $t=1,\ldots,k$ where $k$ is the length of $h$. The clauses of $CNF_{\mathcal{T}}$ are obtained from clausal encodings of the STRIPS axioms as follows: For each $l$ and $a_t$ at step $t$ of some trajectory $h\in\mathcal{T}$, include
\begin{compactenum}
\item $\neg \ispre(l,a_t)\vee State(l,t-1,h)$ if $l$ is not observed at step $t-1$ or $\neg \ispre(l,a_t)$ if $l$ is observed false.
\item $\neg \iseff(l,a_t)\vee State(l,t,h)$, if $l$ is not observed at step $t$, or $\neg \iseff(l,a_t)$ if $l$ is observed false at step $t$, and
\item $\neg State(l,t-1,h)\vee \iseff(\neg l,a_t)\vee State(l,t,h)$ if $l$ is not observed at either step $t$ or $t-1$, $\neg State(l,t-1,h)\vee \iseff(\neg l,a_t)$ if $l$ is not observed at $t-1$ and observed false at $t$, and $\iseff(\neg l,a_t)\vee State(l,t,h)$ if $l$ is observed true at $t-1$ and not observed at $t$.
\end{compactenum}
We also include the mutual exclusion axioms $\neg \iseff(l,a)\vee \neg \iseff(\neg l,a)$ and $\neg State(l,0,h)\vee \neg State(\neg l,0,h)$ for each unobserved literal $l$.
Observe that the claim holds for $CNF_{\mathcal{T}}$ if we additionally set the variables $State(l,t,h)$ to true if $l$ would be true in the corresponding fully-observed trajectory at step $t$ (and conversely): indeed, these are encodings of precisely the action models that obey the STRIPS axioms and corresponding trajectories. We also note that the clauses of $CNF_{\mathcal{T}}$ each only use variables corresponding to a single fluent, so the set of satisfying assignments correspond to products of assignments to the formulas $CNF_{\mathcal{T}}(l)$ that only include the clauses using the same fluent as $l$. (Note that the STRIPS axioms ensure that the post-state is determined uniquely given an assignment to the pre-state.)

We now show that the satisfying assignments to each $CNF_{\eff}(l)$ have the corresponding property for the literal $l$. Recall that resolution is refutation complete, so it suffices to show that for any resolution refutation of a CNF formula $\varphi\wedge CNF_{\mathcal{T}}(l)$, where $\varphi$ (like $CNF_{\eff}(l)$) only contains variables $\iseff(l,a)$, a corresponding refutation of $\varphi\wedge CNF_{\eff}(l)$ exists. 

We first observe that $\ispre(l,a)$ only appears negatively in all clauses, and hence clauses containing these variables cannot be used in any refutation. Similarly, the initial state variables $State(l,0,h)$ only appear negatively and hence we cannot use the mutual exclusion clauses for the state variables either.

Second, observe that to obtain a refutation using the clauses of $CNF_{\mathcal{T}}(l)$, the variables $State(l,t,h)$ must be eliminated, but this variable only appears in a negative literal in the third type of clause (these do not appear in $\varphi$ by assumption), created for step $t+1$ of trajectory $h\in\mathcal{T}$. (And in the mutual exclusion axiom for unobserved variables in the initial state.) The variable only appears positively in the clauses created for step $t$ of the second and third type. Observe that we can rewrite the proof (possibly increasing its size) so that these applications of the resolution rule occur first.

We now claim that the clauses resulting from the final application of the resolution rule on a variable $State(l,t,h)$ appear in $CNF_{\eff}(l)$, which will prove the lemma. Indeed, we can only apply the resolution rule to some run of consecutive clauses of the third type in which the literal $l$ was not observed, ending with a step $t+\Delta$ where $l$ was observed, and beginning with either eliminating $State(l,t-1,h)$ using the clause $\neg \iseff(l,a_{t-1})\vee State(l,t-1,h)$ (created on lines 6--7), or with a clause corresponding to a step $t$ where $l$ was observed in step $t-1$ (created on lines 4--5). 
\end{proof}
}\fi





% Let $A_\sam$ be the action model of the Conformant Planning problem returned by EPI-SAM. 
% First, we prove the following key property about $A_\sam$. of the EPI-SAM action model 
% The key property is that the output of EPI-SAM, $\left( \{\cnf_{\eff}(l)\}_l, \{\pre_a\}_a \right)$ characterize the possible action models that are consistent with our trajectories. 
% We say that the EPI-SAM output $\left( \{\cnf_{\eff}(l)\}_l, \{\pre_a\}_a \right)$ is consistent with an action model if the 

% Throughout this analysis, we will denote by $\pre_A(a)$ and $\eff_A(a)$ the set of preconditions and effects, respectively, of an action $a$ according to an action model $A$. Let $A_\sam$ be the action models of $\Pi_\sam$. 
% First, we prove that for every action we observed, we learned the smallest set of preconditions that ensures safety. 
% \begin{lemma}
% For every action $a\in A_\sam$, 
% a literal $\ell\notin\pre(a)$ if and only if 
% $\ell\notin A$ for every action model $A$ consistent with $\mathcal{T}$.
% \label{lem:preconditions-strong}
% \end{lemma}

% \begin{lemma}
% If an action model $A$ is consistent with $\mathcal{T}$ then it satisfies the CNF, 
% and every satisfying assignment of the CNF describe the effects of at least one action model that is consistent with $\mathcal{T}$. 

% For every action model $A$, 
% $a\in A_\sam$, 
% a literal $\ell\notin\pre(a)$ if and only if 
% $\ell\notin A$ for every action model $A$ consistent with $\mathcal{T}$.
% \label{lem:preconditions-strong}
% \end{lemma}







% First, we prove the following relation between the effect clauses $\{\cnf_\ell\}_\ell$
% returned by the effect learning part of EPI-SAM and the set of action models consistent with the given trajectories $\mathcal{T}$. 
% % Let $R(T)$ be the set of all realizations of $T$. 


% \begin{lemma}
% An action model $A$ is consistent with $\mathcal{T}$
% if and only if the assignment obtained by setting $\iseff(l,a)$ to true if $l$ is an effect of $a$ according to $A$ for each literal $l$ and action $a$ is a satisfying assignment to $\cnf_{\eff}(l)$. 
% % the assignment obtained by setting $\iseff(l,a)$ to true if $l\in\eff_A(a)$ for each literal $l$ and action $a$ is a satisfying assignment to $\cnf_{\eff}(l)$. 
% % the assignment obtained by setting $\iseff(l,a)$ to true if $l$ is an effect of $a$ according to $A$ for each literal $l$ and action $a$ is a satisfying assignment to $\cnf_{\eff}(l)$. 
% % , 
% % the assignment obtained by setting $\iseff(l,a)$ to true if $l$ is an effect of $a$ for each literal $l$ and action $a$ is a satisfying assignment to $\cnf_{\eff}(l)$. 
% % Conversely, for any satisfying assignment to $\cnf_{\eff}(l)$, the corresponding action model is an action model that is consistent with the trajectories $\mathcal{T}$.
% \label{lem:cnf-char}
% \end{lemma}



\begin{lemma}
If an action model $A$ is consistent with $\mathcal{T}$
then the assignment obtained by setting $\iseff(l,a)$ to true if $l$ is an effect of $a$ for each literal $l$ and action $a$ is a satisfying assignment to $\cnf_{\eff}(l)$. 
Conversely, for any satisfying assignment to $\cnf_{\eff}(l)$, the exists an action model consistent with $\mathcal{T}$ whose effects comprise 

corresponding action model is an action model that is consistent with the trajectories $\mathcal{T}$.
\label{lem:cnf-char}
\end{lemma}
\noindent
{\em Sketch of proof.}
Consider the CNF created by the clausal encodings of the STRIPS axioms, instantiated at each step of each trajectory in $\mathcal{T}$. 
This CNF is defined over variables of the form $\iseff(l,a)$, $\ispre(l,a)$, and $\state(l,i,T)$, representing that 
$l$ is a precondition of $a$, 
$l$ is an effect of $a$, 
and $l=\true$ in the $i^{th}$ state of trajectory $T$, respectively.  
This CNF, denoted $\cnf_{\mathcal{T}}$, includes the following clauses for every transition $\tuple{s_{i-1},a_i,s_i}$ in every trajectory $T\in\mathcal{T}$:
\begin{enumerate}
% \item $\neg \ispre(l,a_i)\vee \state(l,i-1,T)$
% \item $\neg \iseff(l,a_i)\vee \state(l,i,T)$
% \item $\neg \state(l,i-1,T)\vee \iseff(\neg l,a_i)\vee \state(l,i,T)$
\item $\neg \ispre(l,a_i)\vee (l\in s_{i-1})$
\item $\neg \iseff(l,a_i)\vee (l\in s_i)$
\item $(\neg l\in s_{i-1})\vee \iseff(\neg l,a_i)\vee (l\in s_i)$
\end{enumerate}
\begin{enumerate}
% \item $\neg \ispre(l,a_i)\vee \state(l,i-1,T)$
% \item $\neg \iseff(l,a_i)\vee \state(l,i,T)$
% \item $\neg \state(l,i-1,T)\vee \iseff(\neg l,a_i)\vee \state(l,i,T)$
\item $\ispre(l,a_i)\rightarrow (l\in s_{i-1})$
\item $\iseff(l,a_i)\rightarrow (l\in s_i)$
\item $\neg\iseff(l,a_i)\rightarrow (s_{i-1}[l]=s_i[l])$
\end{enumerate}
with $State(l,i,T)$ replaced by true or false when $l$ is observed true or false, respectively, at step $i$ in $T$. 
It is immediate that satisfying assignments to $\cnf_{\mathcal{T}}$ correspond to encodings of action models and the complete trajectories those action models would yield, given the values observed in the trajectories of $\mathcal{T}$. We note that the clauses of this formula each only contain literals for a single fluent, so the formula is satisfiable iff the formulas $\cnf_{\mathcal{T}}(l)$ for each literal $l$ are satisfiable.

% a CNF, $\cnf_{\mathcal{T}}$ defined over variables of the form $\iseff(l,a)$, $\ispre(l,a)$, and $\state(l,i,T)$, representing that 
% $l$ is a precondition of $a$, 
% $l$ is an effect of $a$, 
% and $l=\true$ in the $i^{th}$ state of trajectory $T$, respectively.  
% The clauses for $\cnf_\mathcal{T}$ are obtained from clausal encodings of the STRIPS axioms, instantiated at each step of each trajectory in $\mathcal{T}$:
% \begin{enumerate}
% \item $\neg \ispre(l,a_i)\vee State(l,i-1,T)$
% \item $\neg \iseff(l,a_i)\vee State(l,i,T)$
% \item $\neg State(l,i-1,T)\vee \iseff(\neg l,a_i)\vee State(l,i,T)$
% \end{enumerate}
% with $State(l,i,T)$ replaced by true or false when $l$ is observed true or false, respectively, at step $i$ in $T$. 
% It is immediate that satisfying assignments to $\cnf_{\mathcal{T}}$ correspond to encodings of action models and the complete trajectories those action models would yield, given the values observed in the trajectories of $\mathcal{T}$. We note that the clauses of this formula each only contain literals for a single fluent, so the formula is satisfiable iff the formulas $\cnf_{\mathcal{T}}(l)$ for each literal $l$ are satisfiable.

We then use the refutation-completeness of resolution to reduce the problem to identifying the clauses that may appear in resolution refutations. The $\ispre(a,l)$ literals, appearing only negatively, cannot appear in a refutation, and the literals $State(l,i,T)$ must be eliminated, but these only appear in consecutive instances of the second and third type of clauses where the literal $l$ is unobserved; and only the third clause has the negative literal. Reordering the applications of the resolution rule on these literals to the beginning of the proof, we see that we must create clauses that correspond to consecutive runs of unobserved literals using the resolution rule on the third type of clause for each step, beginning with either an observed literal or with using the second type of clause to eliminate the first $State(l,i,T)$ literal. These are, respectively, the clauses of $CNF_{\eff}(l)$ created on lines 4--5 and 6--7. See the supplemental material for details.\roni{I am not super happy with this proof. Don't we need a claim about the preconditions?} \brendan{OK, properly the statement should only concern the effects part of the model. But remember that we can always find a consistent set of preconditions by just saying there are no preconditions, so the question about preconditions is not important for the existence of a consistent action model.}
\iffalse{
%% Full proof below punted to supplemental
\begin{proof}
We first consider the following CNF encoding of the possible trajectories that could yield the observations appearing in $\mathcal{T}$: let $CNF_{\mathcal{T}}$ be a formula with variables $\iseff(l,a)$ for each literal $l$ and action $a$, $\ispre(l,a)$ for each literal $l$ and action $a$, and $State(l,t,h)$ for each literal $l$, trajectory $h\in\mathcal{T}$, and $t=1,\ldots,k$ where $k$ is the length of $h$. The clauses of $CNF_{\mathcal{T}}$ are obtained from clausal encodings of the STRIPS axioms as follows: For each $l$ and $a_t$ at step $t$ of some trajectory $h\in\mathcal{T}$, include
\begin{compactenum}
\item $\neg \ispre(l,a_t)\vee State(l,t-1,h)$ if $l$ is not observed at step $t-1$ or $\neg \ispre(l,a_t)$ if $l$ is observed false.
\item $\neg \iseff(l,a_t)\vee State(l,t,h)$, if $l$ is not observed at step $t$, or $\neg \iseff(l,a_t)$ if $l$ is observed false at step $t$, and
\item $\neg State(l,t-1,h)\vee \iseff(\neg l,a_t)\vee State(l,t,h)$ if $l$ is not observed at either step $t$ or $t-1$, $\neg State(l,t-1,h)\vee \iseff(\neg l,a_t)$ if $l$ is not observed at $t-1$ and observed false at $t$, and $\iseff(\neg l,a_t)\vee State(l,t,h)$ if $l$ is observed true at $t-1$ and not observed at $t$.
\end{compactenum}
We also include the mutual exclusion axioms $\neg \iseff(l,a)\vee \neg \iseff(\neg l,a)$ and $\neg State(l,0,h)\vee \neg State(\neg l,0,h)$ for each unobserved literal $l$.
Observe that the claim holds for $CNF_{\mathcal{T}}$ if we additionally set the variables $State(l,t,h)$ to true if $l$ would be true in the corresponding fully-observed trajectory at step $t$ (and conversely): indeed, these are encodings of precisely the action models that obey the STRIPS axioms and corresponding trajectories. We also note that the clauses of $CNF_{\mathcal{T}}$ each only use variables corresponding to a single fluent, so the set of satisfying assignments correspond to products of assignments to the formulas $CNF_{\mathcal{T}}(l)$ that only include the clauses using the same fluent as $l$. (Note that the STRIPS axioms ensure that the post-state is determined uniquely given an assignment to the pre-state.)

We now show that the satisfying assignments to each $CNF_{\eff}(l)$ have the corresponding property for the literal $l$. Recall that resolution is refutation complete, so it suffices to show that for any resolution refutation of a CNF formula $\varphi\wedge CNF_{\mathcal{T}}(l)$, where $\varphi$ (like $CNF_{\eff}(l)$) only contains variables $\iseff(l,a)$, a corresponding refutation of $\varphi\wedge CNF_{\eff}(l)$ exists. 

We first observe that $\ispre(l,a)$ only appears negatively in all clauses, and hence clauses containing these variables cannot be used in any refutation. Similarly, the initial state variables $State(l,0,h)$ only appear negatively and hence we cannot use the mutual exclusion clauses for the state variables either.

Second, observe that to obtain a refutation using the clauses of $CNF_{\mathcal{T}}(l)$, the variables $State(l,t,h)$ must be eliminated, but this variable only appears in a negative literal in the third type of clause (these do not appear in $\varphi$ by assumption), created for step $t+1$ of trajectory $h\in\mathcal{T}$. (And in the mutual exclusion axiom for unobserved variables in the initial state.) The variable only appears positively in the clauses created for step $t$ of the second and third type. Observe that we can rewrite the proof (possibly increasing its size) so that these applications of the resolution rule occur first.

We now claim that the clauses resulting from the final application of the resolution rule on a variable $State(l,t,h)$ appear in $CNF_{\eff}(l)$, which will prove the lemma. Indeed, we can only apply the resolution rule to some run of consecutive clauses of the third type in which the literal $l$ was not observed, ending with a step $t+\Delta$ where $l$ was observed, and beginning with either eliminating $State(l,t-1,h)$ using the clause $\neg \iseff(l,a_{t-1})\vee State(l,t-1,h)$ (created on lines 6--7), or with a clause corresponding to a step $t$ where $l$ was observed in step $t-1$ (created on lines 4--5). 
\end{proof}
}\fi







% The definition of a safe action model  given earlier (Definition~\ref{def:safe-action-model}) is limited to classical action models. 
% To prove that EPI-SAM is safe, we extend the definition of 
% We formally define the notions of safety and algorithmic strength




% prove that any plan generated 

% Here we prove that the action model returned by EPI-SAM is safe, and the runtime of EPI-SAM is tractable. 
% To prove that EPI-SAM is safe
% The definition of a safe action model given in Definition~\ref{def:safe-action-model}


% Let $A_\sam(\mathcal{T})=\left( \{\cnf_{\eff}(l)\}_l, \{\pre_a\}_a \right)$ denote the output of EPI-SAM given $\mathcal{T}$. 
% We refer to $A_\sam$ as an action model, representing the action model of the corresponding conformant planning problem. 
% We say that $A_\sam$ is safe wrt an action model $A$ 
% if for every action $a\in A_\sam$ and state $s$ it holds that if $a$ is applicable in $s$ according to $A_\sam$ then it is also applicable according to $A$, 
% and for any goal $g$, if a plan achieves $g$ according to $A_\sam$ then it also achieves it according to $A$. 


% To connect to the proofs below: 
% (1) every action model consistent with $\mathcal{T}$ is consistent with $A_\sam$ and vice-versa. 
% (2) $A_\sam$ is safe


% The key property is that the formulas $CNF_{\eff}(l)$ created by EPI-SAM characterize the possible action models that are consistent with our trajectories. 


% \begin{lemma}\label{lem:cnf-char}
% At line 10 in Algorithm \ref{alg:episam}, for every action model consistent with the set of partially observed trajectories $\mathcal{T}$, the assignment obtained by setting $\iseff(l,a)$ to true if $l$ is an effect of $a$ for each literal $l$ and action $a$ is a satisfying assignment to $CNF_{\eff}(l)$, \roni{shouldn't it be as follows: any action model consistent with the trajectories satisfies these CNFs?} and conversely, for any satisfying assignment to $CNF_{\eff}(l)$, the corresponding action model is an action model consistent with the trajectories $\mathcal{T}$.
% \end{lemma}
% % \begin{lemma}\label{lem:cnf-char}
% % Let $\mathcal{T}$ be a set of partially observable trajectories, 
% % and let $A_\sam=\left(\{\cnf_\ell\}_\ell, \{\pre_a\}_a \right)$ be the output of EPI-SAM when given $\mathcal{T}$. 
% % Every action model consistent with $\mathcal{T}$ 
% % must satisfy $\{\cnf_\ell\}_\ell$ and for every action $a$
% % and every action model that satisfies $\{\cnf_\ell\}_\ell$ is consistent with $\mathcal{T}$. 
% % \end{lemma}
% \noindent
% {\em Sketch of proof.}
% We consider a CNF, $\cnf_{\mathcal{T}}$, that has variables of the form $\iseff(l,a)$, $\ispre(l,a)$, and $State(l,i,T)$, representing that 
% $l$ is a precondition of $a$, 
% $l$ is an effect of $a$, 
% and $l=\true$ in the $i^{th}$ state of trajectory $T$, respectively.  
% The clauses for $\cnf_\mathcal{T}$ are obtained from clausal encodings of the STRIPS axioms, instantiated at each step of each trajectory in $\mathcal{T}$:
% \begin{enumerate}
% \item $\neg \ispre(l,a_i)\vee State(l,i-1,T)$
% \item $\neg \iseff(l,a_i)\vee State(l,i,T)$
% \item $\neg State(l,i-1,T)\vee \iseff(\neg l,a_i)\vee State(l,i,T)$
% \end{enumerate}
% with $State(l,i,T)$ replaced by true or false when $l$ is observed true or false, respectively, at step $i$ in $T$. 
% It is immediate that satisfying assignments to $\cnf_{\mathcal{T}}$ correspond to encodings of action models and the complete trajectories those action models would yield, given the values observed in the trajectories of $\mathcal{T}$. We note that the clauses of this formula each only contain literals for a single fluent, so the formula is satisfiable iff the formulas $\cnf_{\mathcal{T}}(l)$ for each literal $l$ are satisfiable.

% We then use the refutation-completeness of resolution to reduce the problem to identifying the clauses that may appear in resolution refutations. The $\ispre(a,l)$ literals, appearing only negatively, cannot appear in a refutation, and the literals $State(l,i,T)$ must be eliminated, but these only appear in consecutive instances of the second and third type of clauses where the literal $l$ is unobserved; and only the third clause has the negative literal. Reordering the applications of the resolution rule on these literals to the beginning of the proof, we see that we must create clauses that correspond to consecutive runs of unobserved literals using the resolution rule on the third type of clause for each step, beginning with either an observed literal or with using the second type of clause to eliminate the first $State(l,i,T)$ literal. These are, respectively, the clauses of $CNF_{\eff}(l)$ created on lines 4--5 and 6--7. See the supplemental material for details.\roni{I am not super happy with this proof. Don't we need a claim about the preconditions?} 
% \iffalse{
% %% Full proof below punted to supplemental
% \begin{proof}
% We first consider the following CNF encoding of the possible trajectories that could yield the observations appearing in $\mathcal{T}$: let $CNF_{\mathcal{T}}$ be a formula with variables $\iseff(l,a)$ for each literal $l$ and action $a$, $\ispre(l,a)$ for each literal $l$ and action $a$, and $State(l,t,h)$ for each literal $l$, trajectory $h\in\mathcal{T}$, and $t=1,\ldots,k$ where $k$ is the length of $h$. The clauses of $CNF_{\mathcal{T}}$ are obtained from clausal encodings of the STRIPS axioms as follows: For each $l$ and $a_t$ at step $t$ of some trajectory $h\in\mathcal{T}$, include
% \begin{compactenum}
% \item $\neg \ispre(l,a_t)\vee State(l,t-1,h)$ if $l$ is not observed at step $t-1$ or $\neg \ispre(l,a_t)$ if $l$ is observed false.
% \item $\neg \iseff(l,a_t)\vee State(l,t,h)$, if $l$ is not observed at step $t$, or $\neg \iseff(l,a_t)$ if $l$ is observed false at step $t$, and
% \item $\neg State(l,t-1,h)\vee \iseff(\neg l,a_t)\vee State(l,t,h)$ if $l$ is not observed at either step $t$ or $t-1$, $\neg State(l,t-1,h)\vee \iseff(\neg l,a_t)$ if $l$ is not observed at $t-1$ and observed false at $t$, and $\iseff(\neg l,a_t)\vee State(l,t,h)$ if $l$ is observed true at $t-1$ and not observed at $t$.
% \end{compactenum}
% We also include the mutual exclusion axioms $\neg \iseff(l,a)\vee \neg \iseff(\neg l,a)$ and $\neg State(l,0,h)\vee \neg State(\neg l,0,h)$ for each unobserved literal $l$.
% Observe that the claim holds for $CNF_{\mathcal{T}}$ if we additionally set the variables $State(l,t,h)$ to true if $l$ would be true in the corresponding fully-observed trajectory at step $t$ (and conversely): indeed, these are encodings of precisely the action models that obey the STRIPS axioms and corresponding trajectories. We also note that the clauses of $CNF_{\mathcal{T}}$ each only use variables corresponding to a single fluent, so the set of satisfying assignments correspond to products of assignments to the formulas $CNF_{\mathcal{T}}(l)$ that only include the clauses using the same fluent as $l$. (Note that the STRIPS axioms ensure that the post-state is determined uniquely given an assignment to the pre-state.)

% We now show that the satisfying assignments to each $CNF_{\eff}(l)$ have the corresponding property for the literal $l$. Recall that resolution is refutation complete, so it suffices to show that for any resolution refutation of a CNF formula $\varphi\wedge CNF_{\mathcal{T}}(l)$, where $\varphi$ (like $CNF_{\eff}(l)$) only contains variables $\iseff(l,a)$, a corresponding refutation of $\varphi\wedge CNF_{\eff}(l)$ exists. 

% We first observe that $\ispre(l,a)$ only appears negatively in all clauses, and hence clauses containing these variables cannot be used in any refutation. Similarly, the initial state variables $State(l,0,h)$ only appear negatively and hence we cannot use the mutual exclusion clauses for the state variables either.

% Second, observe that to obtain a refutation using the clauses of $CNF_{\mathcal{T}}(l)$, the variables $State(l,t,h)$ must be eliminated, but this variable only appears in a negative literal in the third type of clause (these do not appear in $\varphi$ by assumption), created for step $t+1$ of trajectory $h\in\mathcal{T}$. (And in the mutual exclusion axiom for unobserved variables in the initial state.) The variable only appears positively in the clauses created for step $t$ of the second and third type. Observe that we can rewrite the proof (possibly increasing its size) so that these applications of the resolution rule occur first.

% We now claim that the clauses resulting from the final application of the resolution rule on a variable $State(l,t,h)$ appear in $CNF_{\eff}(l)$, which will prove the lemma. Indeed, we can only apply the resolution rule to some run of consecutive clauses of the third type in which the literal $l$ was not observed, ending with a step $t+\Delta$ where $l$ was observed, and beginning with either eliminating $State(l,t-1,h)$ using the clause $\neg \iseff(l,a_{t-1})\vee State(l,t-1,h)$ (created on lines 6--7), or with a clause corresponding to a step $t$ where $l$ was observed in step $t-1$ (created on lines 4--5). 
% \end{proof}
% }\fi
