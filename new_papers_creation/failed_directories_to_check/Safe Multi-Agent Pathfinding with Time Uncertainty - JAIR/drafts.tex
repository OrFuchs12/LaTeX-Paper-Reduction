

% However, in contrast to~\citeauthor{CimattiDMRS18}'s framework, in this paper we endeavor to look for \emph{optimal} and \emph{strong} solutions for the MAPF-TU problem, where we do not model explicit actions with information like their \emph{durations}, \emph{preconditions} and \emph{effects}, and \emph{exogenous events}. 
% Rather we work with explicit graphs. We place uncertainty over edges ($e \in \edges$) -- indicating their traversal time ranges in a given graph $G=(\vertices,\edges)$.
% However, the notion of a graph is implicit in the area of automated planning, but one could generate the whole search space using the actions modeled by a modeler. Also, with some simple tweaks and tricks, we can use their exact solver to solve the MAPF-TU problems. 
% Perhaps, if we model each \emph{move} action as $move^{e}_{a_i}$, one for each edge $e \in \edges$, it is possible for an agent $a_i$ to traverse the edge $e$. 
% The $move^{e}_{a_i}$ action keeps the \emph{duration} on the basis of the traversal time of $e$. 
% We also need to model the initial and goal states accordingly along with other minor details, e.g.,  a vertex cannot be occupied by more than one agent. 
% We need to do that for all the agents and consider a meta-agent that has access to all the actions of all the agents. 
% This turns the MA problem to a single-agent STPUD problem~\cite{CimattiDMRS18}. 
% Therefore, we can employ their temporal planner to obtain a satisficing %{satisficing is correct}
% plan for the original MAPF-TU problem and which can be transformed back. 




% , directly
%  MAPF problem, the CBS~\cite{SharonSFS12} approach also exploits this fact and being optimistic one could realize that our proposed approaches to solve the MAPF-TU problem would be much more efficient than using \citeauthor{CimattiDMRS18}'s temporal planner. 
% Since the transformation we discussed in the last paragraph would always take a single-agent view to the multi-agent problem, it could never be able to exploit the degree of decoupledness of the problem. 
% Using their temporal planner might also generate a non-optimal plan unlike our framework. Although we also suspect that such modeling is not feasible for the explicit graph problems like ours. 
% Any further study in this direction is left for future work. 






% Such approaches exploit the fact that loosely coupled multi-agent systems are easier to plan for because they require less coordination between agent sub-plans. 
% These sub-plans can even be obtained in a distributed manner by solving the independent sub-problems on a distributed system. Later combine the obtained solutions to form a complete joint-solution.
% To solve the MAPF problem, the CBS~\cite{SharonSFS12} approach also exploits this fact and being optimistic one could realize that our proposed approaches to solve the MAPF-TU problem would be much more efficient than using \citeauthor{CimattiDMRS18}'s temporal planner. 
% Since the transformation we discussed in the last paragraph would always take a single-agent view to the multi-agent problem, it could never be able to exploit the degree of decoupledness of the problem. 
% Using their temporal planner might also generate a non-optimal plan unlike our framework. Although we also suspect that such modeling is not feasible for the explicit graph problems like ours. 
% Any further study in this direction is left for future work. 







% However, in contrast to~\citeauthor{CimattiDMRS18}'s framework, in this paper we endeavor to look for \emph{optimal} and \emph{strong} solutions for the MAPF-TU problem, where we do not model explicit actions with information like their \emph{durations}, \emph{preconditions} and \emph{effects}, and \emph{exogenous events}. 
% Rather we work with explicit graphs. We place uncertainty over edges ($e \in \edges$) -- indicating their traversal time ranges in a given graph $G=(\vertices,\edges)$.
% However, the notion of a graph is implicit in the area of automated planning, but one could generate the whole search space using the actions modeled by a modeler. Also, with some simple tweaks and tricks, we can use their exact solver to solve the MAPF-TU problems. 
% Perhaps, if we model each \emph{move} action as $move^{e}_{a_i}$, one for each edge $e \in \edges$, it is possible for an agent $a_i$ to traverse the edge $e$. 
% The $move^{e}_{a_i}$ action keeps the \emph{duration} on the basis of the traversal time of $e$. 
% We also need to model the initial and goal states accordingly along with other minor details, e.g.,  a vertex cannot be occupied by more than one agent. 
% We need to do that for all the agents and consider a meta-agent that has access to all the actions of all the agents. 
% This turns the MA problem to a single-agent STPUD problem~\cite{CimattiDMRS18}. 
% Therefore, we can employ their temporal planner to obtain a satisficing %{satisficing is correct}
% plan for the original MAPF-TU problem and which can be transformed back. 


% It is evident from the planning literature that, for the multi-agent planning problems, factored-approaches can be very efficient~\cite{BrafmanD08,PajarinenP11,ShekharBS19} sometimes.  
% Such approaches exploit the fact that loosely coupled multi-agent systems are easier to plan for because they require less coordination between agent sub-plans. 
% These sub-plans can even be obtained in a distributed manner by solving the independent sub-problems on a distributed system. Later combine the obtained solutions to form a complete joint-solution.
% To solve the MAPF problem, the CBS~\cite{SharonSFS12} approach also exploits this fact and being optimistic one could realize that our proposed approaches to solve the MAPF-TU problem would be much more efficient than using \citeauthor{CimattiDMRS18}'s temporal planner. 
% Since the transformation we discussed in the last paragraph would always take a single-agent view to the multi-agent problem, it could never be able to exploit the degree of decoupledness of the problem. 
% Using their temporal planner might also generate a non-optimal plan unlike our framework. Although we also suspect that such modeling is not feasible for the explicit graph problems like ours. 
% Any further study in this direction is left for future work. 









% : the planner can to decide when the action starts while the environment chooses when the action ends \emph{end} is chosen (within a \emph{known bound}) by the environment. 


% , such problems are abbreviated as STPUD.
% Temporal planning~\cite{ColesFLS08,ColesC14,CashmoreCCKMR19} is a popular and well researched sub-area of AI planning in which we are allowed to look for temporal plans that specify \emph{start} and \emph{end} time points for each action in the plan, and to model and reason about them. 
% \citeauthor{CimattiDMRS18} (2018) address that, in many practical scenarios, it is hard to have a control over the execution time of an action and the same depends on exogenous events. 
% Their planning approach handles cases where actions have uncertain \emph{durations}, where the planner is just allowed to decide the \emph{start} of an action while the \emph{end} is chosen (within a \emph{known bound}) by the environment. 
% Like ours, their approach too looks for a \emph{strong} plan such that its execution is always safe, while the plan must be consistent with the imposed temporal constraints, given whatever choices the environment makes. 
 


% Their framework tackles the problem of strong \emph{temporal planning} in which the action durations are uncontrollable, such problems are abbreviated as STPUD.
% Temporal planning~\cite{ColesFLS08,ColesC14,CashmoreCCKMR19} is a popular and well researched sub-area of AI planning in which we are allowed to look for temporal plans that specify \emph{start} and \emph{end} time points for each action in the plan, and to model and reason about them. 
% \citeauthor{CimattiDMRS18} (2018) address that, in many practical scenarios, it is hard to have a control over the execution time of an action and the same depends on exogenous events. 
% Their planning approach handles cases where actions have uncertain \emph{durations}, where the planner is just allowed to decide the \emph{start} of an action while the \emph{end} is chosen (within a \emph{known bound}) by the environment. 
% Like ours, their approach too looks for a \emph{strong} plan such that its execution is always safe, while the plan must be consistent with the imposed temporal constraints, given whatever choices the environment makes. 
 


% we are allowed to look for temporal plans that specify \emph{start} and \emph{end} time points for each action in the plan, and to model and reason about them. 
% \citeauthor{CimattiDMRS18} (2018) address that, in many practical scenarios, it is hard to have a control over the execution time of an action and the same depends on exogenous events. 
% Their planning approach handles cases where actions have uncertain \emph{durations}, where the planner is just allowed to decide the \emph{start} of an action while the \emph{end} is chosen (within a \emph{known bound}) by the environment. 
% Like ours, their approach too looks for a \emph{strong} plan such that its execution is always safe, while the plan must be consistent with the imposed temporal constraints, given whatever choices the environment makes. 













% While 
% The solutions sought for STPU problems are either \emph{strong}, \emph{weak}, or \emph{dynamic}. 



% Thus, the output 

% STPU is fundamentally different from MAPF-TU in that it 
%  Perhaps it is trivial to verify whether the solution for the MAPF-TU problem is strongly controllable by employing their approach. 
% We note that such verification is not required in our case as our planning framework always guarantees that all the temporal constraints are respected, and causal links and precedence relations between two actions/activities are maintained and consistent. 





% The solutions sought for STPU problems are either \emph{strong}, \emph{weak}, or \emph{dynamic}. 

% An STPU is strongly controllable if there exists an assignment (called a strong schedule) of real values to each controllable time point, such that all free constraints are satisfied for every possible assignment of the uncontrollable time points satisfying the contingent constraints. 

% \citeauthor{VidalF99} (1999) show that the strong controllability problem for an STPU can be solved in polynomial-time. Perhaps it is trivial to verify whether the solution for the MAPF-TU problem is strongly controllable by employing their approach. 
% We note that such verification is not required in our case as our planning framework always guarantees that all the temporal constraints are respected, and causal links and precedence relations between two actions/activities are maintained and consistent. 

% solutions~\cite{VidalF99,PeintnerVY07} or solutions that are \emph{dynamically controllable}~\cite{MorrisMV01}. 


% is time uncertainty means that, exactly like in our case, the 
% , and hence observed online during execution. 



% In general, a consistent solution for an STPU problem is a \emph{schedule}  schedules a set of activities to be performed such that all the temporal constraints among them are met. 
% For STPU problems, the time uncertainty means that, exactly like in our case, the duration of an activity may lie within some bounds but cannot be decided upfront, and hence observed online during execution. 

%  the time uncertainty means that, exactly like in our case, the duration of an activity may lie within some bounds but cannot be decided upfront, and hence observed online during execution. 


% Planning under temporal uncertainty is a well-studied topic in Artificial Intelligence. 
% Foundational work on Simple Temporal Problem (STP~\cite{DechterMP91}) addressed the problem of \emph{scheduling} a set of activities with temporal constraints. 
% STP under Uncertainty (STPU~\cite{VidalF99}) 

% , which is a scheduling problem. 

% The \emph{solution} for an MAPF-TU problem can be seen as a solution of a Simple Temporal Problem (STP~\cite{DechterMP91}) under Uncertainty (STPU~\cite{VidalF99}), which is a scheduling problem. 
% However, viewing the MAPF-TU solution this way is true when all the actions (activities) in this solution are the only activities that need to be scheduled in that corresponding STPU problem. 
% In general, a consistent solution for an STPU problem is obtained via an activity scheduler that schedules these activities to be performed such that all the temporal constraints among them are met. 
% For STPU problems, the time uncertainty means that, exactly like in our case, the duration of an activity may lie within some bounds but cannot be decided upfront, and hence observed online during execution. \roni{This ``hence'' is not clear}
% %
% %A consistent schedule for the STPU problem is obtained via a scheduler that respects all the temporal constraints and relationships among the activities given in the activity space. 
% %

% For the STPU problems the solutions sought are either strong solutions~\cite{VidalF99,PeintnerVY07} or  solutions that are Dynamically Controllable~\cite{MorrisMV01}. 
% Since in this paper our main focus is on exact techniques for \emph{strong planning} with uncontrollable interval durations (i.e., traversal times) over the edges of an explicit graph, we keep this discussion only up to strong controllability. 
% %%
% An STPU is strongly controllable if there exists an assignment (called a strong schedule) of real values to each controllable time point, such that all free constraints are satisfied for every possible assignment of the uncontrollable time points satisfying the contingent constraints. 

% \citeauthor{VidalF99} (1999) show that the strong controllability problem for an STPU can be solved in polynomial-time. Perhaps it is trivial to verify whether the solution for the MAPF-TU problem is strongly controllable by employing their approach. 
% We note that such verification is not required in our case as our planning framework always guarantees that all the temporal constraints are respected, and causal links and precedence relations between two actions/activities are maintained and consistent. 

% However, a more general framework should plan with activities having uncontrollable durations and schedule these activities later in a consistent way, or intertwine these two processes. 
% %such that the activities are not already known. Which means that  it should plan the activities first and schedule them later. 
% In this work we propose the MAPF-TU framework that plans for and schedules agents' movements such that all the temporal constraints appear due to the nature of the problem graphs, are respected, and all the causal links and ordering constraints are consistent. 
% Perhaps following their movement schedule the agents reach their goal locations without collision.   



%
%
%
%Orthogonal to the above line of work done by the MAPF community, the temporal planning and scheduling community deals with action time intervals (\emph{time} an action could possibly take), duration uncertainty (deals in time ranges), and uncontrollability (the execution time of an action cannot be predicted upfront and can only be decided by the environment but decision of starting the execution of an action is controlled by the agent), and so on. [[Roni: too-long sentence]
%Simple Temporal Problems (STPs) represent constraints over the timing of activities, as arising in many robotic applications, such as scheduling and temporal planning~\cite{DechterMP91}. 
%STP with \emph{uncertainty} (STPU) is characterized by activities with \emph{uncontrollable} durations~\cite{CuiH19,cimatti2003weak,CimattiDMRS18}.
%

% A strong temporal planning framework ---  more general and close to our proposed framework, is proposed and studied in~\cite{CimattiDMRS18}. 
% Their framework tackles the problem of strong \emph{temporal planning} in which the action durations are uncontrollable, such problems are abbreviated as STPUD.
% Temporal planning~\cite{ColesFLS08,ColesC14,CashmoreCCKMR19} is a popular and well researched sub-area of AI planning in which we are allowed to look for temporal plans that specify \emph{start} and \emph{end} time points for each action in the plan, and to model and reason about them. 
% \citeauthor{CimattiDMRS18} (2018) address that, in many practical scenarios, it is hard to have a control over the execution time of an action and the same depends on exogenous events. 
% Their planning approach handles cases where actions have uncertain \emph{durations}, where the planner is just allowed to decide the \emph{start} of an action while the \emph{end} is chosen (within a \emph{known bound}) by the environment. 
% Like ours, their approach too looks for a \emph{strong} plan such that its execution is always safe, while the plan must be consistent with the imposed temporal constraints, given whatever choices the environment makes. 
 

% In contrast to~\citeauthor{CimattiDMRS18}'s framework, in this paper we endeavor to look for \emph{optimal} and \emph{strong} solutions for the MAPF-TU problem, where we do not model explicit actions with information like their \emph{durations}, \emph{preconditions} and \emph{effects}, and \emph{exogenous events}. 
% Rather we work with explicit graphs. We place uncertainty over edges ($e \in \edges$) -- indicating their traversal time ranges in a given graph $G=(\vertices,\edges)$.
% However, the notion of a graph is implicit in the area of automated planning, but one could generate the whole search space using the actions modeled by a modeler. Also, with some simple tweaks and tricks, we can use their exact solver to solve the MAPF-TU problems. 
% Perhaps, if we model each \emph{move} action as $move^{e}_{a_i}$, one for each edge $e \in \edges$, it is possible for an agent $a_i$ to traverse the edge $e$. 
% The $move^{e}_{a_i}$ action keeps the \emph{duration} on the basis of the traversal time of $e$. 
% We also need to model the initial and goal states accordingly along with other minor details, e.g.,  a vertex cannot be occupied by more than one agent. 
% We need to do that for all the agents and consider a meta-agent that has access to all the actions of all the agents. 
% This turns the MA problem to a single-agent STPUD problem~\cite{CimattiDMRS18}. 
% Therefore, we can employ their temporal planner to obtain a satisficing %{satisficing is correct}
% plan for the original MAPF-TU problem and which can be transformed back. 


% It is evident from the planning literature that, for the multi-agent planning problems, factored-approaches can be very efficient~\cite{BrafmanD08,PajarinenP11,ShekharBS19} sometimes.  
% Such approaches exploit the fact that loosely coupled multi-agent systems are easier to plan for because they require less coordination between agent sub-plans. 
% These sub-plans can even be obtained in a distributed manner by solving the independent sub-problems on a distributed system. Later combine the obtained solutions to form a complete joint-solution.
% To solve the MAPF problem, the CBS~\cite{SharonSFS12} approach also exploits this fact and being optimistic one could realize that our proposed approaches to solve the MAPF-TU problem would be much more efficient than using \citeauthor{CimattiDMRS18}'s temporal planner. 
% Since the transformation we discussed in the last paragraph would always take a single-agent view to the multi-agent problem, it could never be able to exploit the degree of decoupledness of the problem. 
% Using their temporal planner might also generate a non-optimal plan unlike our framework. Although we also suspect that such modeling is not feasible for the explicit graph problems like ours. 
% Any further study in this direction is left for future work. 

