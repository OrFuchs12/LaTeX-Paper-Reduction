\def\year{2021}\relax
%File: fmatting-instructions-latex-2021.tex
%release 2021.1
\documentclass[letterpaper]{article} % DO NOT CHANGE THIS
\usepackage{aaai21}  % DO NOT CHANGE THIS
\usepackage{times}  % DO NOT CHANGE THIS
\usepackage{helvet} % DO NOT CHANGE THIS
\usepackage{courier}  % DO NOT CHANGE THIS
\usepackage[hyphens]{url}  % DO NOT CHANGE THIS
\usepackage{graphicx} % DO NOT CHANGE THIS



\urlstyle{rm} % DO NOT CHANGE THIS
\def\UrlFont{\rm}  % DO NOT CHANGE THIS
\usepackage{natbib}  % DO NOT CHANGE THIS AND DO NOT ADD ANY OPTIONS TO IT
\usepackage{caption} % DO NOT CHANGE THIS AND DO NOT ADD ANY OPTIONS TO IT
\frenchspacing  % DO NOT CHANGE THIS
\setlength{\pdfpagewidth}{8.5in}  % DO NOT CHANGE THIS
\setlength{\pdfpageheight}{11in}  % DO NOT CHANGE THIS
 \usepackage{multirow}
%\nocopyright
%PDF Info IREQUIRED.
% F /Auth, add all authwithin the parentheses, separated by commas. No accent commands.
% F /Title, add Title in Mixed Case. No accent commands. Retain the parentheses.
% /Title ()
% Put your actual complete title (no codes, scripts, shtcuts,  LaTeX commands) within the parenthesein mixed case
% Leave the space between \Title and the beginning parenthesialone
% /Auth ()
% Put your actual complete list of auth(no codes, scripts, shtcuts,  LaTeX commands) within the parenthesein mixed case.
% Each auth should be only by a comma. If the name containaccents, remove them. If there are any LaTeX commands,
% remove them.


\usepackage{times}
\usepackage{epsfig}
\usepackage{graphicx}
\usepackage{algpseudocode}
\usepackage{algithm}
\usepackage{amsmath}
\usepackage{enumitem}
\usepackage{amssymb}
\usepackage{amsthm}
\usepackage{microtype}
\frenchspacing
\newtheemstyle{mydefstyle}{}{}{\itshape}{}{\bfseries}{:}{.5em}{#1 #2 (\thmnote{#3})}
\theemstyle{mydefstyle}
\newtheem{mydef}{Definition}
\newtheem{mythey}{Theem}
\newtheem{mycr}{Collary}
\def\real{\mathbb{R}}
\def\posnat{\mathbb{N}^+}
\def\labs{\mathbb{N}}
\def\ints{\mathbb{Z}}
\newcommand{\argmin}{\operatnamewithlimits{argmin}}
\newcommand{\argmax}{\operatnamewithlimits{argmax}}
\newcommand{\argst}{\operatnamewithlimits{argst}}
\long\def\advercomment#1{}
\long\def\comment#1{}
\newcommand{\E}{\mathbb{E}}

\def\httilde{\mbox{\tt\raisebox{-.5ex}{\symbol{126}}}}

% Pageare numbered in submission mode, and unnumbered in camera-ready

\begin{document}

%%%%%%%%% TITLE
\title{Towarda Unifying Framewk f Fmal Theieof Novelty}
\auth {
    % Auths
    T. E. Boult\textsuperscript{\rm 1},
    P. A. Grabowicz\textsuperscript{\rm 5},
    D. S. Prijatelj\textsuperscript{\rm 2},
    R. Stern\textsuperscript{\rm 6},
    L. Holder\textsuperscript{\rm 4},
    J. Alspect\textsuperscript{\rm 3},
    M. Jafarzadeh\textsuperscript{\rm 1},
    T. Ahmad\textsuperscript{\rm 1},
    A. R. Dhamija\textsuperscript{\rm 1},
    C. Li\textsuperscript{\rm 1},
    S. Cruz\textsuperscript{\rm 1},
    A. Shrivastava\textsuperscript{\rm 7},
    C. Vondrick\textsuperscript{\rm 8},
    W. J. Scheirer\textsuperscript{\rm 2} \\
}
\affiliation{
    % Affiliations
    \textsuperscript{\rm 1} U. Col. Col. Springs,
    \textsuperscript{\rm 2} U. Notre Dame,
    \textsuperscript{\rm 3} IDA/ITSD,
    \textsuperscript{\rm 4} Wash. State U.,
    \textsuperscript{\rm 5} U. Mass.,
    \textsuperscript{\rm 6} PARC, BGU,
    \textsuperscript{\rm 7} U. Maryland,
    \textsuperscript{\rm 8} Columbia U.\\
    \{tboult $\vert$ mjafarzadeh $\vert$ tahmad $\vert$ adhamija\}@vast.uccs.edu, grabowicz@cs.umass.edu,  derek.prijatelj@nd.edu,  rstern@parc.com,  jalspect@ida.g, holder@wsu.edu,  abhinav@cs.umd.edu, vondrick@cs.columbia.edu,  walter.scheirer@nd.edu
}



\maketitle
%\thispagestyle{empty}

% %%%%%%%%% ABSTRACT
% \begin{abstract}

% Initial fmulation of a they of Perceptual Novelty

% \end{abstract}

% %\vspace*{-3mm}
%%%%%%%%% BODY TEXT

\begin{abstract}
Managing inputthat are novel, unknown,  out-of-distribution icritical aan agent movefrom the lab to the open wld. Novelty-related include being tolerant to novel perturbationof the nmal input, detecting when the input includenovel items, and adapting to novel inputs.  While significant research habeen undertaken in these areas, a noticeable gap existin the lack of a fmalized definition of novelty that transcendproblem domains. Aa team of researcherspanning multiple research groupand different domains,  we have seen, first hand, the difficultiethat arise from ill-specified novelty  awell ainconsistent definitionand terminology. Therefe, we present the first unified framewk f fmal theieof novelty and use the framewk to fmally define a family of novelty types. Our framewk can be applied acrosa wide range of domains, from symbolic AI to reinfcement learning, and beyond to open wld image recognition. Thus, it can be used to help kick-start new research efftand accelerate ongoing wk on these imptant novelty-related 
\end{abstract}



\section{Introduction}

``What inovel?" ian imptant AI research question that infmthe design of agenttolerant to novel inputs.
Ia noticeable change in the wld that doenot impact an agent'task perfmance a novelty?
How about a change that impactperfmance but inot directly perceptible?
If the wld hanot changed but the agent sensea random err that producean input that leadto an unexpected state, ithat novel?

With decadeof wk and thousandof papercovering novelty detection and related research in anomaly detection, out-of-distribution detection, open set recognition, and open wld recognition, one would think that a consistent unified definition of novelty would have been developed.  Unftunately, that inot the case. Instead, we find a pletha of variationon thitheme, as well as \textit{ad hoc} use and inconsistent reuse of terminology, all of which injects confusion as researchers discuss these topics.

This paper introduces a unifying fmal framewk of novelty. The framewk seeks to fmalize what it means f an input to be a novelty in the context of agents in artificial intelligence  in other learning-based systems.
%This paper seeks to fmalize the definition of a novel sample in the context of agents in artificial intelligence  in other learning-based systems.
Using the proposed framewk, we fmally define multiple types of novelty an agent can encounter.
The goal of these definitions is to be broad enough to encompass and unify the full range of novelty models that have been proposed in the literature~\cite{pimentel2014review,markou2003novelty,markou2003novelty2,openset-pami13, openwld_2015,langley2020open}. An imptant generalization beyond pri wk is that we consider novelty in the wld, observed space, and agent space (see Fig.~\ref{fig:elements}), with dissimilarity and regret operats critical to our definitions.
The overarching goal is a framewk such that researchers have clear definitions f the development of agents that must handle novelty, including suppt f agents / algithms that incrementally learn from novel inputs.   A longer version of this they with example applications to three different domains can be found at~\cite{Boult-eta-al-novelty20}.



Our framewk suppts \textit{implicit theies of novelty}, meaning the definitions use functions to implicitly specify if something is novel. The framewk does not require a way to generate novelties, but rather it provides functions that can be used to evaluate if a given input is novel. This is similar to how any 2D shape can be implicitly defined by a function $f(x,y)=0$, whether  not there is a procedure f generating the shape.
We contend any constructive  generative they of novelty~\cite{langley2020open} must be incomplete because the construction  generation of defined wlds, states, and any enumerable set of transfmations between them fm, by definition, a closed wld. We note, however, that a constructive model can be consistent with our definition, but we do not require a constructive model.




\input noveltythey.tex


\section{ }


We see three primary contributions of this fmalization of novelty that will spur further research.
First, fmalization fces one to specify ( intentionally disregard) the required items in the they.
This can lead to in
F example, when applying the they to the CartPole problem, numerous unanticipated issues were high 


Second, fmalization provides a common language to define and compare models of novelty across 
The precision of terms reduces confusion, while  


Third, the fmalization allows one to make predictions about where  why experiments incpating some fm of novelty might run into difficulties.
F example, when the wld-level and perceptual-level dissimilarity assessments disagree, we predict novelty will be me difficult.
One example of difficulty is wld-disparity using variables not represented in perceptual space.
Another is when there are many possible wld labels, but the input is only assigned one label that is used f assessing wld-level dissimilarity.
In this case, the they predicts a greater difficulty with ated with a physically smaller aspect of the observation.


  continue to struggle with this behavi.
It is our hope that the adoption and use of the framewk proposed here leads to the development of me effective solutions f novelty management and to make agents me robust to novel changes in their wld.



By fmalizing CartPole using our novelty framewk, we gained insights into what are meaningful ``novelty'' f this task.
We showed how to develop better measures to predict when novelty would be easy  hard to manage  to detect.
In line with this, our team of researchers has been refining this they and applying it to multiple problem domains.
Me details can be found in the longer arXiv version~\cite{Boult-eta-al-novelty20}.

{\footnotesize
\section*{}
This research was  sponsed  by the Defense Advanced Research Projects Agency (DARPA) and the Army Research Office (ARO) under multiple contracts/agreements including  HR001120C0055, W911NF-20-2-0005,W911NF- contained in this document are those of the auths and should not be interpreted as representing the official policies, either expressed  implied, of the DARPA  ARO,  
}



{\small
\clearpage
\bibliography{novelty}
%\nocite{*}
}


\end{document}













