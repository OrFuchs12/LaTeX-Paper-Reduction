\section{Introduction}
Consider a scenario in the world of e-commerce, where a single
consumer is seeking to buy a set of products through an online website with
multiple vendors, such as Amazon or eBay. Given the items available
for sale and their prices, the buyer will purchase some subset
of them, according to his valuation of the items and their prices.

Naturally, the vendors (our agents) strive to
maximize their profits.\footnote{For ease of exposition we assume
production costs are zero, hence profits can be equated with
revenues, or the sum of the prices of their
sold items.}
% \footnote{We implicitly assume that the items have zero
%  production costs, which would otherwise need to be discounted from
%  the revenues.} 
A vendor can both competitively tailor the set
of items it offers and adjust
the prices of these items to react to their competitors (pricing an item sufficiently high can be
regarded as not offering it). 
Indeed, automatic mechanisms for rapid online price optimization 
exist in many markets and
% beyond the ``natural'' ones (e.g., the airline 
industries~\cite{angwin2012coming}. This practice,
sometimes called \emph{competitive price intelligence}~\cite{skorupa14},
is a growing phenomenon within online retail. The specific
question it addresses is how companies should price products
in this competitive environment. %As P. K. Kannan states\cite{moutinho2014routledge}: \paragraph{}\emph{In this framework, value is defined on the basis of the next best competitive alternative... If the firm's offering totally undifferentiated from this competitive offering, then this is the maximum level at which the firm can price their offering.}

Furthermore, as argued by Babaioff et al.~\citeyear{BabaioffNL14}, such a
setting introduces subtle algorithmic questions, since changing the
prices of the products may affect the resulting revenues in a
complex fashion, which may induce responses by one's
competitors. Therefore, studying the convergence properties of such
pricing dynamics is of interest.

In this paper, we take a game-theoretic approach to price competition
among multiple sellers, each with a set of items to sell. As in
Babaioff et al.~\citeyear{BabaioffNL14}, we study a setting with a single
buyer with a (combinatorial) valuation function, taken to be a
monotone and submodular set function over the set of items, which is fully known to the vendors. However, unlike that earlier
work, we examine the more general case where each of the $k$ vendors
controls a disjoint set of items $A_i$, rather than a single item. Given the prices of all of the
items, the buyer will buy the set with the highest net-payoff
(valuation minus the total price). Our model induces a game
in which the vendors' strategy is a pricing of their
items.%This framework extends recent work done by Babioff, who studied the case where each of the vendors owns a single item.

\paragraph{Contributions}
We begin by discussing a related two-phase game, that serves as a
way-station in our study of the main game. In this intermediate
game, vendors can only modify the sets of items being offered,
whereas their prices for these items are subsequently set by a specific pricing
mechanism. We show that this game, which results only from this
modification of game dynamics (without changing its
parameters), has key properties that its equilibria share with those of the
original game. This allows us to reduce the pricing decision to a decision over what items to
sell, thereby significantly simplifying the problem.


\commentout{
two critical properties. First, any
strategy profile in the original game (a price vector) has a
corresponding strategy profile that results in at least as much
revenue for each of the vendors (Proposition~\ref{prop:discrete}).

More importantly, we show that for any pure Nash equilibrium in the
original game, there is a corresponding pure Nash equilibrium in the
intermediate game, 
in which each vendor sells the same items at the same prices as
in the original equilibrium
% that not only admits identical revenues for the vendors,
% by having them sell the same sets of items, but also that the prices
% of these items are identical to their respective prices in the discrete game
(Theorem~\ref{thm:NE}). %These results extend and build on a recent study by Babioff et al.~\cite{BabaioffNL14}, who studied the case where each vendor own a single item.
Hence, we reduce the pricing decision to a decision over what items to
sell, allowing a significant simplification of the problem.
}

We next study basic game theoretic properties of our game. We first show
that there are games which admit no pure Nash equilibrium. To do
so, we show that our two-phase game admits no pure Nash
equilibria, which then implies the nonexistence of a pure Nash
equilibrium in the original game, using Theorem~\ref{thm:NE}
(see Proposition~\ref{prop:no_PNE}).  This result suggests the
following question: suppose we restrict attention to instances of the game
that have some pure Nash equilibria---can we then say anything about
their value?  To accomplish this, we analyze the price of anarchy
(PoA) of this subset of games, where the objective function in
question is the social welfare value, taken to be simply the buyer's
valuation of the set of items that he purchased. We provide a tight
bound of $\Theta(\log m)$, where $m$ is the maximal number of items
controlled by any of the vendors.

Finally, as an additional way of dealing with the consequences of
Proposition~\ref {prop:no_PNE}, we investigate a special class of
valuation functions, which we call \emph{category-substitutable},
that, informally, partition products into ``equivalence
classes'' or categories, such that only a single item
will be chosen within a specific category, while 
different categories do not influence one
another. %\jo{I believe that the term ``Semi-Equivalent'' is misleading here: the products aren't semi-equivalent at all, not even among items in their own category (they have different valuations).}\ol{Replaced}%There is a partition of the total set of items, and a non-negative value assignment $\mathbf{x}$ to these items. The value of a set $S$ is taken to be a linear combination of the highest $x$ values of the items in $S$, per category.
We show not only that \emph{efficient} pure Nash equilibria always
exist given such buyer valuations, but provide a
precise characterization of this equilibria.
%%% Local Variables: 
%%% mode: latex
%%% TeX-master: "vendor_competition"
%%% End: 
