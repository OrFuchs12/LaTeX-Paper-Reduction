\section{Previous Work}

Multi-item pricing has been a significant topic  of research for many
years~\cite{HN12,HR12}, including analyses of the price of anarchy
(see Christodoulou et al. \citeyear{CKS08} and follow-up papers).
The work of Babaioff et
al.~\citeyear{BabaioffNL14} is the most directly related to the model
developed here; the game that they study is a special case of
our game, in which each vendor sells only a single item. They show that for a
buyer with a general valuation function, pure Nash equilibria may not
exist, and they prove several properties of the equilibria for games where 
one does. Furthermore, for \emph{submodular} valuation functions (the focus of
our paper), they show not only that pure Nash equilibria always exist,
but also that they are unique (they give a closed-form characterization of
the prices of the items that are sold) and efficient---i.e.,
have a \emph{price of anarchy (PoA)} of one. In contrast
to their setting, we show in
our more general case there exist games with no pure Nash equilibria.
In cases where they exist, we provide a characterization similar to theirs,
though in our more general case, PoA is significantly higher.
 
Non-competitive (i.e., single-vendor) optimal-pricing problems
% , given the valuations of potential buyers, 
have been studied in the theoretical computer
science community. \cite{Guruswami:2005:PEP:1070432.1070598} study
a number of settings with \emph{multiple} buyers possessing various
valuation functions. They show that even with unit-demand buyers and an
unlimited supply of each item, selecting the optimal price vector is APX-hard;
they then provide a logarithmic approximation algorithm for the same case. It
should be noted that Babaioff et al.~\citeyear{BabaioffNL14} also provide a $\log n$ approximation
algorithm for the case of a single vendor, and for that of a single buyer with
a submodular valuation function.


%\jo{write a short blurb about the main results in this area.}\ol{Should we just throw in many of the papers Babaioff mentions as well?}\jo{I guess we should do something like this --- but tread lightly, I don't want to raise any suspicions.}

In a recent paper, Oren et al.~\citeyear{oren2014game} analyze a model in which
fixed prices are given exogenously, and there are multiple
unit-demand buyers. As above, their model assumes an unlimited supply
of each item.
The strategies of the vendors are which
\textbf{sets} of items to sell. Having the vendor make
decisions only about the set of items to sell has traditionally
been studied in the field of operation research. In particular, 
\emph{assortment optimization}~\cite{schon} deals with optimizing a seller's
``assortment'' (e.g., his catalog, or shelf), under various circumstances.
Although our game does not fall directly into this category, we do define a
discrete game in which the vendors' decisions are similar to
those in assortment
optimization (although the pricing procedure differs).

%%% Local Variables: 
%%% mode: latex
%%% TeX-master: "vendor_competition"
%%% End: 
