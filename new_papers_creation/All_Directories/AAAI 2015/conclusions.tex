\section{Conclusions and Future Work}

The multi-item, multi-vendor problem is a practical instance of rival agents' problems, with their actions directly affecting actions to be taken by the others. This relationship among the agents is, in many cases, intractable to handle. However, in the simplified model, which is robust enough to incorporate realistic limitations, we were able to analyze the effects of each agent's moves. 

We defined a discrete game that allowed us to consider a related game that was instrumental in analyzing our original game. The main property of the discrete game was to transform player strategies from pricing, to selecting what items to sell. To paraphrase Clausewitz's famous dictum, displaying (what to sell) became pricing by other means. Utilizing this discrete game, we were able to prove that a multi-item, multi-vendor game with submodular buyers valuations does not necessarily have a Nash equilibrium (unlike the ``single item per vendor'' model). Furthermore, even when equilibria exist, it may provide only a logarithmic price of anarchy% (again, unlike the single item per vendor model). 
Building on these results, we showed that in a particular% (yet, in our view, common) 
category-substitute model, there will always be an efficient pure Nash equilibrium.

Many open problems remain, even before the ``holy grail'' of pricing multi-item multi-buyer scenarios. We believe that there is a need to establish the characteristics of valuation functions that guarantee the existence of  Nash equilibria.% (both in our model and in simpler ones, such as the one item per vendor model).

Adding more buyers changes the model significantly, as vendors do not simply construct some ``buyer in expectation'' and act according to it, but rather have a wider range of options to pursue (primarily bundling). Perhaps using a metric to define a set of similar, yet not identical, buyers, it might be possible to build on our results, and construct extensions to the current model incorporating multiple buyers.
{\footnotesize
\paragraph{Acknowledgements} This research has been partly funded by Microsoft Research through its PhD Scholarship Programme, Israel Science Foundation grant \#1227/12, and the Israel Ministry of Science and Technology --- Knowledge Center in Machine Learning and Artificial Intelligence grant \#3-9243. Oren and Boutilier acknowledge the support of NSERC.}
%%% Local Variables: 
%%% mode: latex
%%% TeX-master: "vendor_competition"
%%% End: 
