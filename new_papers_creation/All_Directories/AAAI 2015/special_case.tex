\section{Special Case: Product Categories}
\label{semi-equivalent}

A particular VC game of interest is one in which we have classes of items that are
roughly equivalent; as such the buyer is interested
in at most one item from each class (e.g., TV sets of a
certain size, with different manufacturers and sets of feature). 
Items in different classes however are ``unrelated'' so
the buyer's valuation for any set of items is additive \emph{across} 
these classes.
This scenario reflects the case of shops selling very similar products, of
which the buyer only needs one, and we seek to try to understand the model's pricing behavior.

\begin{definition}
A Category-Divided Substitutable-Product Vendor Competition game (CDSP-VC) is a VC game with a buyer with a category-product-substitutable valuation: 
\begin{itemize}
\item $A^*$ is partitioned into $r$ pairwise-disjoint sets, $T^{(1)},\ldots, T^{(r)}$. That is, $T^{(i)}\cap T^{(j)}=\emptyset$ for $i\neq j$, and $\bigcup_{i=1}^{r}T^{(i)}=A^*$. We refer to each set $T^{(j)}$ for $j=1,\ldots,r$, as a category.
\item For $S\subseteq A^*$, $v(S)=\sum_{i=1}^{r}v(S\cap T^{(i)})$.
\item For $S\subseteq A^*$ and $1\leq i\leq r$, $v(S\cap T^{(i)})=max_{a\in S\cap T^{(i)}}v(a)$.
\end{itemize}
\end{definition}
The additivity allows us to focus on the pricing dynamic within a specific category and easily generalize the results.
\begin{observation}
\label{obs:single_item_sell}
For a category $T^{(j)}$, regardless of the other vendors' strategies, no vendor can profit by selling any items other than his most valuable one in category $T^{(j)}$.
\end{observation}
(Proof omitted due to space constraints.)\\
\commentout{
\begin{proof}4
As the buyer only buys a single item from each category, vendors can make positive revenue from at most one item.
Consider a vendor $i$ and category $T^{(j)}$ such that $A_i \cap T^{(j)} \neq \emptyset$. 
%cannot gain from offering several items, as the valuation depends solely on the most valuable product of the item-class bought. 
Suppose that $a\in \arg\max_{b\in T^{(j)}\cap A_{i}}v(b)$, and consider a strategy profile $\mathbf{p}$. For any item $b \in (A_{i}\setminus a) \cap T^{(j)}$, if $b \in X(v;p_i,\mathbf{p_{-i}})$, then if vendor $i$ switches to price vector $p'_i$ which prices $a$ same as $b$ and prices $b$ at $v(A^*)+1$, then $a \in X(v;p'_i,\mathbf{p_{-i}})$, and player's profit does not decrease.
% then for any strategy of player $i$ $S^{j}_{i}$ and that other players $S^{j}_{-i}$, for player $i$:

% $$
% v(S'^{j}_{i},S^{j}_{-i})=max_{a\in S'^{j}_{i}\cup S^{j}_{-i}}\leq max(a,max_{a\in S^{j}_{i}})=v(a,S^{j}_{-i})
% $$


% Moreover note that due to the submodularity, as the number of sold products in the item-class grows, the maximal price buyers will be willing to pay in an equilibrium is reduced (as the marginal prices are reduced).
\end{proof}
}
Observation~\ref{obs:single_item_sell} implies that within every category, each of the vendors is better off effectively trying to sell his highest valued item. In other words, we can assume w.l.o.g. that for every vendor $i$ and category $T^{(j)}$, the vendor can pick an item $a^{(j)}_{i} \in \arg\max_{a \in T^{(j)} \cap A_i}v(a)$ (if such item exists) and set $p(b)=v(A^*)+1$, for all $b \neq a^{(j)}_i$, without incurring a loss as a result. Therefore, this reduces our game to $r$ independent special cases of the VC game, in which each vendor owns a \emph{single} item.

We turn to the result given by Babaioff et al.~\citeyear{BabaioffNL14} (Theorem~\ref{thm:babaioff14} in the preliminaries). Their result implies
the following characterization of the prices in a pure Nash equilibrium. 
% In a sense, the CDSP has been reduced, for each item-class, to the Babaioff et al. model~\cite{BabaioffNL14}, as each vendor sells exactly one item. In this case, we know the prices at equilibrium are the marginal values. This means every vendor will sell at price $0$, except for the sole vendor with the most valuable item in the class ($a*^{j}$), which would sell it for $v(a*^{j})-max_{a\in A^{j}\setminus A^{j}_{i}}v(a)$.

\begin{corollary}
Every CDSP-VC game has a pure Nash equilibrium of the following form. For every category $T^{(j)}$, let $c_{i}^{(j)} = \arg\max_{a \in (T^{(j)}\cap A_{i})}v(a)$, and $w=\arg\max_{i} c_{i}^{(j)}$. Let $b^{(j)}=\arg\max_{a \in (T^{(j)}\setminus A_{w})}v(a)$ or $b^{(j)}=0$ if $|T^{(j)}|=1$. Then $p(c_{w}^{(j)})=v(c_{w}^{(j)}) - b^{(j)}$, and for every player $i\neq w$, $p(c_{i}^{(j)})=0$. For all other items $a\in T^{(j)}$, $p(a)=v(A^*)+1$.
\end{corollary}
\begin{proof}
Once we know that each player sells only a single product, unsold products need to be priced high, and the rest of the result stems from Theorem~\ref{thm:babaioff14}, as the above difference constitutes the marginal contribution of item $c^{(j)}$.
\end{proof}

%%% Local Variables: 
%%% mode: latex
%%% TeX-master: "vendor_competition"
%%% End: 
