\section*{Broader Impact}
Performing successful commonsense reasoning is dependent on having a large knowledge base of commonsense facts.
% Commonsense reasoning relies on a knowledge base (KB) of commonsense facts, but 
However, there is currently no technology that has the capacity to store the sum total of all human knowledge.
% one has access to a KB that is large enough to do commonsense with, 
Therefore, we propose to conversationally complete knowledge bases by interacting with humans. Learning through conversational interactions is a research direction that can impact the machine learning community and can be combined with many learning algorithms to address issues with data sparsity. 
Moreover, it is expected for a future version of Siri or Google home or any other smart phone/home assistant to be equipped with this type of interaction strategy, so that they can build more personalized beliefs about their users. Therefore, leveraging conversational interactions for learning is a plausible opportunity. However, as is the case with any model that collects user information, this collected data could be susceptible to attacks or privacy breach.

The importance of commonsense reasoning and its lack of it in current AI systems is a known impediment to having truly intelligent systems \cite{davis2015commonsense}. Therefore, this research is an attempt towards taking the literature on commonsense reasoning a small step forward 
by proposing a new unstudied aspect of commonsense reasoning which is to attempt to explicitly uncover unspoken commonsense presumptions from a given natural language utterance.
% commonsense reasoning benchmark and a novel commonsense reasoning engine. %and can benefit many applications in natural language understanding, machine translation, computer vision and robotics.
% Commonsense reasoning remains an unsolved problem at the heart of AI. This paper proposes an open direction for research in commonsense reasoning and
% proposes . 
This can be further extended to benchmarks that uncover presumptions given an image of a scene, to uncover objects that are presumably in the photo but are not shown. For example, to presume that in an image of a room there exists a wall hook that supports a frame on the wall.

% For example, consider showing a human a photo of a dining room. They would immediately presume that there must be a table underneath the table-cloth or that the objects hanging on the wall are presumably supported by hooks and many more [example taken from \cite{davis2015commonsense}]. This can also be extended to explicitly uncover presumptions while watching a movie. %\facomment{consider replacing the example with an example in your own words}
One of the immediate impacts of this work is to make AI assistants that can engage in natural sounding conversations with humans. In order to be good at conversation, computers should be able to make presumptions driven by common sense about an input statement. If solved, we would have conversational agents that sound smarter to humans. We propose a neuro-symbolic reasoning engine that aims to address this problem.
% Another impact of this work is in proposing a neuro-symbolic solution for the proposed commonsense reasoning problem.