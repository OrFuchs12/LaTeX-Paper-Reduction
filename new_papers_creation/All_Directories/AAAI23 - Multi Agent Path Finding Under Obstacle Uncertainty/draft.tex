We implemented a more efficient method to compute conflicts, based on the following definition of a MAPFOU conflict. 
% This method is based on the observation that every node in the plan trees is reachable under a specific partial obstacle configuration. Let $K(n^i_j)$ denote this partial obstacle configuration for node $n^i_j$. 
\begin{definition}[MAPFOU Conflict]
A MAPFOU conflict is a tuple $\tuple{i,j,v,t,c_i,c_j}$ 
where $i$ and $j$ are different agents, 
$v$ is a vertex,
$t$ is a time step, 
and $c_i$ and $c_j$ are consistent partial obstacle configurations.
\label{def:mapfou-conflict}
\end{definition}

The following observation connects conflicts between plan trees (Definition~\ref{def:conflicting-plan-trees} and the above definition of a MAPFOU conflict (Definition~\ref{def:mapfou-conflict}).  
\begin{observation}
A conflict exists between two plan trees $\tau^i$ and $\tau^j$ 
iff there exists two nodes $n^i\in\tau^i$ and $n^j\in\tau^j$ such that 
(1) $n^i.v = n^j.v$, 
(2) $n^i.t = n^j.t$, 
and (3) $\tuple{i,j,n^i.v,n^i.t,n^i.c, n^j.c}$ is a MAPFOU conflict.
\label{obs:conditionsForConflicts}
\end{observation} 
% \begin{observation}
% There is a conflict between a pair of plan trees $\tau^i$ and $\tau^j$ 
% if and only if there exists a pair of nodes $n^i\in\tau^i$ and $n^j\in\tau^j$ 
% such that (1) both nodes are in the same depth in the tree (i.e., represent the lcoation of the respective agent at the same time step), 
% (2) $n^i.v = n^j.v$, 
% and (3) $K(n^i)$ is consistent with $K(n^j)$. 
% \label{obs:conditionsForConflicts}
% \end{observation} 
Checking for a conflict between a pair of plan trees by searching for a pair of nodes that form a MAPFOU conflict is, in practice, a more efficient method to detect conflicts than the brute force method described above. 
