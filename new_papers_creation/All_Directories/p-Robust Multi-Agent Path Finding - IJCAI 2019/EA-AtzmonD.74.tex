

\newcommand{\tuple}[1]{\ensuremath{\left \langle #1 \right \rangle }}


\newcommand{\qed}{\hfill\ensuremath{\blacksquare}}
\newcommand{\astar}{A$^*$}
\newcommand{\SAS}{SAS$^+$}
\newcommand{\wastar}{WA$^*$}
\newcommand{\arastar}{ARA$^*$}
\newcommand{\open}{\textsc{Open}}
\newcommand{\closed}{\textsc{Closed}}
\newcommand{\cmfp}{conformant model-free planning}
\newcommand{\eff}{\textit{eff}}
\newcommand{\pre}{\textit{pre}}
\newcommand{\solvable}{\textit{S}}
\newcommand{\plannable}{\textit{P}}
\newcommand{\true}{\textit{T}}
\newcommand{\false}{\textit{$\bot$}}



\def\year{2019}\relax
%File: formatting-instruction.tex
\documentclass[letterpaper]{article} % DO NOT CHANGE THIS
\usepackage{aaai19}  % DO NOT CHANGE THIS
\usepackage{times}  % DO NOT CHANGE THIS
\usepackage{helvet} % DO NOT CHANGE THIS
\usepackage{courier}  % DO NOT CHANGE THIS
\usepackage[hyphens]{url}  % DO NOT CHANGE THIS
\usepackage{graphicx} % DO NOT CHANGE THIS
\urlstyle{rm} % DO NOT CHANGE THIS
\def\UrlFont{\rm}  % DO NOT CHANGE THIS
\usepackage{graphicx}  % DO NOT CHANGE THIS
\frenchspacing  % DO NOT CHANGE THIS
\setlength{\pdfpagewidth}{8.5in}  % DO NOT CHANGE THIS
\setlength{\pdfpageheight}{11in}  % DO NOT CHANGE THIS

\usepackage{xspace}
\newcommand{\krmapf}{$k$R-MAPF\xspace}
\newcommand{\krcbs}{$k$R-CBS\xspace}
\newcommand{\ikrcbs}{I-$k$R-CBS\xspace}
\newcommand{\prcbs}{$p$R-CBS\xspace}
\newcommand{\iprcbs}{I-$p$R-CBS\xspace}
\newcommand{\prmapf}{$p$R-MAPF\xspace}





%PDF Info Is REQUIRED.
% For /Author, add all authors within the parentheses, separated by commas. No accents or commands.
% For /Title, add Title in Mixed Case. No accents or commands. Retain the parentheses.
 \pdfinfo{
/Title (Probabilistic Robust Multi-Agent Path Finding)
/Author (Dor Atzmon, Ariel Felner, Roni Stern)
} %Leave this	
% /Title ()
% Put your actual complete title (no codes, scripts, shortcuts, or LaTeX commands) within the parentheses in mixed case
% Leave the space between \Title and the beginning parenthesis alone
% /Author ()
% Put your actual complete list of authors (no codes, scripts, shortcuts, or LaTeX commands) within the parentheses in mixed case. 
% Each author should be only by a comma. If the name contains accents, remove them. If there are any LaTeX commands, 
% remove them. 

% DISALLOWED PACKAGES
% \usepackage{authblk} -- This package is specifically forbidden
% \usepackage{balance} -- This package is specifically forbidden
% \usepackage{caption} -- This package is specifically forbidden
% \usepackage{color (if used in text)
% \usepackage{CJK} -- This package is specifically forbidden
% \usepackage{float} -- This package is specifically forbidden
% \usepackage{flushend} -- This package is specifically forbidden
% \usepackage{fontenc} -- This package is specifically forbidden
% \usepackage{fullpage} -- This package is specifically forbidden
% \usepackage{geometry} -- This package is specifically forbidden
% \usepackage{grffile} -- This package is specifically forbidden
% \usepackage{hyperref} -- This package is specifically forbidden
% \usepackage{navigator} -- This package is specifically forbidden
% (or any other package that embeds links such as navigator or hyperref)
% \indentfirst} -- This package is specifically forbidden
% \layout} -- This package is specifically forbidden
% \multicol} -- This package is specifically forbidden
% \nameref} -- This package is specifically forbidden
% \natbib} -- This package is specifically forbidden -- use the following workaroiqued:
% \usepackage{savetrees} -- This package is specifically forbidden
% \usepackage{setspace} -- This package is specifically forbidden
% \usepackage{stfloats} -- This package is specifically forbidden
% \usepackage{tabu} -- This package is specifically forbidden
% \usepackage{titlesec} -- This package is specifically forbidden
% \usepackage{tocbibind} -- This package is specifically forbidden
% \usepackage{ulem} -- This package is specifically forbidden
% \usepackage{wrapfig} -- This package is specifically forbidden
% DISALLOWED COMMANDS
% \nocopyright -- Your paper will not be published if you use this command
% \addtolength -- This command may not be used
% \balance -- This command may not be used
% \baselinestretch -- Your paper will not be published if you use this command
% \clearpage -- No page breaks of any kind may be used for the final version of your paper
% \columnsep -- This command may not be used
% \newpage -- No page breaks of any kind may be used for the final version of your paper
% \pagebreak -- No page breaks of any kind may be used for the final version of your paperr
% \pagestyle -- This command may not be used
% \tiny -- This is not an acceptable font size.
% \vspace{- -- No negative value may be used in proximity of a caption, figure, table, section, subsection, subsubsection, or reference
% \vskip{- -- No negative value may be used to alter spacing above or below a caption, figure, table, section, subsection, subsubsection, or reference

\setcounter{secnumdepth}{0} %May be changed to 1 or 2 if section numbers are desired.

% The file aaai19.sty is the style file for AAAI Press 
% proceedings, working notes, and technical reports.
%
\setlength\titlebox{2.5in} % If your paper contains an overfull \vbox too high warning at the beginning of the document, use this
% command to correct it. You may not alter the value below 2.5 in
\title{Probabilistic Robust Multi-Agent Path Finding}

\newcommand{\OPEN} {{\textsc{Open}}}

\newtheorem{definition}{Definition}
\newtheorem{lemma}{Lemma}
\newtheorem{theorem}{Theorem}
\newcommand{\ignore}[1]{}

%Your title must be in mixed case, not sentence case. 
% That means all verbs (including short verbs like be, is, using,and go), 
% nouns, adverbs, adjectives should be capitalized, including both words in hyphenated terms, while
% articles, conjunctions, and prepositions are lower case unless they
% directly follow a colon or long dash

%\author{Dor Atzmon\\
%Ben-Gurion University\\
%Israel\\
%dorat@post.bgu.ac.il
%\And
%Ariel Felner\\
%Ben-Gurion University\\
%Israel\\
%felner@bgu.ac.il
%\And
%Roni Stern\\
%Ben-Gurion University\\
%Israel\\
%sternron@post.bgu.ac.il}


\author{\Large \textbf{Dor Atzmon, Ariel Felner, Roni Stern}\\ % All authors must be in the same font size and format. Use \Large and \textbf to achieve this result when breaking a line\
Ben-Gurion University, Israel\\ %If you have multiple authors and multiple affiliations
% use superscripts in text and roman font to identify them. For example, Sunil Issar,\textsuperscript{\rm 2} J. Scott Penberthy\textsuperscript{\rm 3} George Ferguson,\textsuperscript{\rm 4} Hans Guesgen\textsuperscript{\rm 5}. Note that the comma should be placed BEFORE the superscript for optimum readability
dorat@post.bgu.ac.il, felner@bgu.ac.il, sternron@post.bgu.ac.il % email address must be in roman text type, not monospace or sans serif
}
 \begin{document}

\maketitle


% -----------------   Introduction    --------------------------
%\section{Introduction and Definitions}
A {\em Multi-Agent Path Finding} (MAPF) problem is defined by a graph $G=(V,E)$ and a set of agents $\{a_1 \dots a_n\}$. At each time step, an agent can either {\em move} to an adjacent location or {\em wait} in its current location. The task is to find a plan $\pi_i$ for each agent $a_i$ that moves it from its start location $s_i \in V$ to its goal location $g_i \in V$ such that agents do not {\em conflict}, i.e., occupy the same location at the same time.\footnote{This research is supported by ISF grants no. 210/17 to Roni Stern and \#844/17 to Ariel Felner and Eyal Shimony, by BSF grant \#2017692 and by NSF grant \#1815660} 

In practice, unexpected events may delay some of the agents, preventing them from following the plan. Thus, it is desirable to generate a {\em robust plan} that can withstand such delays. Recently, a form of robustness called \emph{$k$-robust MAPF} was introduced~\cite{DBLP:conf/socs/AtzmonSFWBZ18}, in which each agent can be delayed up to $k$ times and no collision will occur. 
In some cases, it is possible to estimate the probability that a delay will occur. 
%By aggregating the probabilities of multiple delays we can estimate the probability that a given conflict will occur. 
In such cases, solving all conflicts with the same fixed value $k$ may be less reasonable, and we might prefer solving conflicts based on their probabilities to occur. To this end, we explore a new form of robustness, $p$-robust, where a $p$-robust plan is a plan that can be executed without any collisions with a probability $\geq p$. 


% ------------------    p -Robust CBS   ------------------------
\section{$p$-Robust CBS}
 \label{sec:probust-cbs}
 
\prcbs is a CBS-based algorithm \cite{CBSJUR} designed to return $p$-robust plans. To present \prcbs, we introduce the notion of \emph{potential conflict} and its relation to finding $p$-robust plans. 

% Potential conflicts and a conflict occurring
\begin{definition}[Potential Conflict]
A plan $\pi$ has a potential conflict $C=\tuple{a_i,a_j,t}$ iff there exists $\Delta(C)\geq 0$ such that agents $a_i$ and $a_j$ are located in the same location in times $t$ and $t+\Delta(C)$, respectively, i.e, when $\pi_i(t) = \pi_j(t+\Delta(C))$.  
\end{definition}

A potential conflict $C=\tuple{a_i,a_j,t}$ 
is said to {\em have occurred} if agent $a_i$ experienced exactly $d_i\geq \Delta(C)$ delays before performing the $t^{th}$ action in $\pi_i$, 
and agent $a_j$ experienced exactly $d_i-\Delta(C)$ delays before performing the $t+\Delta(C)$ action in $\pi_j$. This means the agents will collide  since $\pi_i(t)=\pi_j(t+\Delta(C))$ (they will collide at time $t+d_i$).
 
% Our setting, and the likelihood of a potential conflict to occur
Let $P_0(\pi)$ be the probability that no potential conflict will occur when following plan $\pi$ with a delay probability of $p_d$. 
It is easy to see that a plan $\pi$ is $p$-robust iff the probability that no potential conflicts will occur is $\geq p$, i.e., $P_0(\pi)\geq p$.


\prcbs{} is different than CBS in how it handles CT nodes, in how it chooses and resolves conflicts, and in how it orders nodes in the high-level open list. 

% Handling a CT node
{\bf Handling a CT node.} When a CT node $N$ is chosen for expansion, \prcbs{} scans $N.\pi$ for potential conflicts by checking for locations occupied by more than one agent (even in different time steps). Then, $N.\pi$ is sent to a binary verifier that returns whether the plan is $p$-robust (if $P_0(\pi) \geq p$) or not. If the verifier returns TRUE, then the CT node is declared as goal and $\pi$ is returned. If the verifier returns FALSE, it also returns $P_0(\pi)$ as well as a probability $P_{First}(C)$ for each potential conflict $C$. $P_{First}(C)$ provides the probability that while executing $\pi$, conflict $C$ will occur first in time among all potential conflicts. Note that the sum of $P_0$ and all $P_{First}$ probabilities equals 1, because either the execution succeeded ($P_0$) or one of the conflicts have occurred (one of the $P_{First}$). 


% Choosing a conflict to resolve
{\bf Choosing a Conflict to Resolve.} $p$-robust solution may contain potential conflicts. Thus, we need to resolve a set of conflicts such that the solution will be $p$-robust. \prcbs{} chooses to resolve the conflict with the highest probability of occurring (highest $P_{First}$). This is a greedy approach that has high chances to reach a $p$-robust plan quickly, as it has the highest potential to increase $P_0$ in its children. 

% Resolving a conflict
{\bf Resolving a Conflict.} Let $C=\tuple{a_i,a_j,t}$ be the chosen potential conflict in a non-goal node $N$. To resolve $C$, we add the range constraints
$\tuple{a_i, \pi_i(t), [t, t+\Delta(C)]}$ and $\tuple{a_j, \pi_j(t+\Delta(C)), [t, t+\Delta(C)]}$ to $a_i$ and $a_j$, respectively. This assures that these agents will not both be at the conflicting location in the time frame $[t,t+\Delta(C)]$.

% Choosing CT Nodes
{\bf Choosing CT Nodes.} In this paper we focused on finding a $p$-robust plan as fast as possible. Therefore, we implemented a greedy approach that chooses to expand the node with highest $P_0$ value. Then, when a CT node with $P_0\geq p$ is found by a verifier, that node is returned as a goal. 




% ------------------   Statistical Verifier   ---------------------
\section{Statistical Verifiers}
\label{sec:stat-verifier}
We describe two verifiers that verify statistically whether $P_0$ is {\em greater than or equal to} the desired robustness ($p$).

% Fixed Verifier
{\bf Fixed Verifier.} The fixed verifier is first initialized with the following given parameters: $p$, $p_d$, $s$, and $\alpha$, where $p$ is the desired robustness, $p_d$ is the constant delay probability, $s$ is the number of simulations to be performed, and $1-\alpha$ is the confidence level of the statistical test. Then, a {\em critical value} $c_1$ is calculated by performing a $Z$-test as follows:
\begin{equation}
{   c_1=p+Z_{1-\alpha} \cdot \sqrt{\frac{(1-p) \cdot p}{s}}}
\label{equ:critical}
\end{equation}

$c_1$ is calculated once and used later in every verification to determine whether $P_0 \geq p$ within the confidence level $1-\alpha$.

After \prcbs{} has chosen to expand node $N$, it calls the fixed verifier to verify statistically whether $N$ is a goal node. The verifier executes $s$ simulations of the given plan ($N.\pi$) with delay probability $p_d$. To count collisions during executions we create a table (named $\mathit{occurred}$) that maps a given potential conflict to the number of times it has occurred. We also initialize a parameter: $\mathit{successes}$ that counts the number of executions in which no collision has occurred. During each execution, if a collision has occurred at a potential conflict $C$, the execution halts, and we increment $\mathit{occurred[C]}$ (initialized as $0$). Otherwise, if no collision has occurred, we increment $\mathit{successes}$. When all $s$ simulations ended, it sets $P_0 \gets \mathit{successes}/s$. If $P_0 > c_1$, it returns TRUE. Otherwise, it sets $N.P_0 \gets P_0$. For each conflict $C$ it sets $N.P_{First}(C) \gets \mathit{occurred[C]}/{s}$ and it returns FALSE.

{\bf Dynamic Verifier.} 
%The main drawback of the fixed verifier is that it uses a fixed number of simulations, denoted $s$. However, in some cases $s$ might not be enough and in other cases $s$ is too high. If $P_0 > c_1$ the verify may return true. This may be possible only if $c_1 < 1$. Therefore, we would like to perform enough simulations that will guarantee that $c_1<1$. This is done dynamically with our dynamic verifier, that does not use a fixed number of simulations.  
This verifier chooses dynamically the number of simulations $s$ to be performed in every CT node, as follows.  First, it performs the minimum number of simulation that guarantees that $c_1<1$, which is derived from Equation \ref{equ:critical} to be ${  \left \lceil {Z_{1-\alpha}}^2 \cdot \frac{p}{1-p} \right \rceil}$. 
If $P_0>c_1$ then we return TRUE. Otherwise, we might be able to perform more simulations until $P_0>c_1$. However, the test might always fail and this will lead to an infinite loop.  To overcome this issue, before executing more simulations, we perform another statistical test that checks whether $P_0<c_2$ where $c_2$ is a new critical value which is calculated as follows:
\begin{equation}
{  c_2=p-Z_{1-\alpha} \cdot \sqrt{\frac{(1-p) \cdot p}{s}}}
\label{equ:critical2}
\end{equation}
If the second test passes, return FALSE. Otherwise, perform one more simulation, and check these tests again. The verification phase of the dynamic verifier summarized as follows. (1) Run $s$ simulations and approximate $P_0$. (2) Calculate $c_1$  (Equation~\ref{equ:critical}). (3) If $P_0>c_1$, return TRUE. (4) Calculate $c_2$ (Equation~\ref{equ:critical2}). (5) If $P_0<c_2$, return FALSE. (6) $s \gets s+1$, run one more simulation, and goto step 2.


% ----------------   Experimental Results   ----------------------
\section{Experimental Results}
\label{sec:experiments1}

We compared the performance of \prcbs{} for different values of $p$ with our two verifiers. In all of the following results $\alpha=0.05$, $p_d=0.2$, and $P_0$ was calculated based on 50 executions of the solution. 

% Results - p comparison
\begin{table}[t]
\centering
\resizebox{0.55\columnwidth}{!}{
\begin{tabular}{|c|r|r|r|}
\hline
\multicolumn{1}{|l|}{} & \multicolumn{1}{c|}{Cost} & \multicolumn{1}{c|}{Time(ms)} & \multicolumn{1}{c|}{$P_0$} \\ \hline
CBS                    & 38.5                      & 9                         & 0.41                       \\
%$p=0.6$                  & 41.7                      & 2,553                     & 0.79                       \\
$p=0.7$                  & 43.3                      & 7,620                     & 0.84                       \\
%$p=0.8$                  & 44.9                      & 14,915                    & 0.88                       \\
$p=0.9$                  & 50.1                      & 37,501                    & 0.95                       \\ \hline
\end{tabular}}
\caption{Average planning cost, runtime and for CBS and \prcbs{} with different values of $p$,  over 8x8 open grid.}
\label{tab:p-values}
\end{table}


We compared standard CBS and \prcbs{} with the fixed verifier for different values of $p$ (0.7 and 0.9) and $s=40$, on an 8x8 open grid with 8 randomly allocated agents. Table~\ref{tab:p-values} presents the average cost, planning time (in ms), and $P_0$ for 60 problem instances.
We can see that larger $p$ increases the cost and time but results in less collisions (higher $P_0$). %For example, for $p=0.7$ the cost is 43.3, the time is 7,620ms, and $P_0=0.84$, while for $p=0.9$ the cost is 50.1, the time is 37,501, and $P_0=0.95$. 
The optimal solver achieved the lowest cost (38.5) and the fastest planning time (only 9ms) with a tradeoff that many collisions occurred and only 41\% of the executions were collision-free ($P_0$).


% Dynamic Verifier
\begin{table}[t]
\centering
\resizebox{0.85\columnwidth}{!}{
\begin{tabular}{|c|c|c|c|c|}
\hline
\#Simulations & $p=0.80$ & $p=0.8$5 & $p=0.90$ & $p=0.95$ \\ \hline
20            & \textbf{59}     & 0      & 0      & 0      \\
%30            & \textbf{59}     & \textbf{59}     & 0      & 0      \\
40            & 58     & 57     & \textbf{57}     & 0      \\
%50            & \textbf{59}     & 57     & 55     & 0      \\
60            & 57     & 55     & 55     & 0      \\
%80            & \textbf{59}     & 57     & 54     & 51     \\
160           & 58     & 56     & 52     & 49     \\ 
Dynamic       & \textbf{59}     & \textbf{59}     & \textbf{57}     & \textbf{52}     \\ \hline
\end{tabular}}
\caption{Success rate for \prcbs{} out of 60 instances.}
\label{tab:dynamic}
\end{table}

We also compared the fixed verifier (with a different number of simulations) and the dynamic verifier, with $p=0.8,0.85,0.9,$ and $0.95$. 60 instances were generated and we present the number of instances that could be solved within 5 minutes in Table~\ref{tab:dynamic}. As expected, if the number of simulations was too small, the fixed verifier could not solve any instance as a result of the statistic test ($c_1$ was greater than 1). On the other hand, the dynamic verifier could solve instances for all values of $p$. Moreover, the success rate of the dynamic verifier was at least as the success rate of the fixed verifier that achieved the highest success rate. The quality of solution and running time of the dynamic verifier and the fixed verifier were similar for instances that could be solved by both. The dynamic verifier performs better but it is more complicated. Hence there is a tradeoff.



% ---------------   Conclusions and Future work   -----------------
\section{Conclusions and Future work}

We studied a new form of robustness: $p$-robust, and proposed a greedy CBS-based algorithm for finding a $p$-robust plan with two possible verifiers that have an internal tradeoff. Possible lines of future work, including integrating $p$-robust plans with execution policies, as suggested by Ma et al.~\shortcite{ma2017multiAgent} and better approximating the real $P_0$ probability.

%\section{Acknowledgements}X

%-----------------------------------------------
%-----------------------------------------------


\bibliographystyle{aaai}
\bibliography{sam}


\end{document}
