
In~\cite{sharon2015conflict}, 
the authors suggested that combining multiple agents together (a meta-agent) is sometimes beneficial, which is known as Meta-agent CBS (MA-CBS) that generalizes the CBS approach.
In this case, the main merge policy is to merge agents for which the number of conflicts seen so far exceeds a given  threshold $B$ (a \emph{good} choices of $B$), MA-CBS reduces  the runtime. 
The second strategy was Bypass (BP) improvement to (MA)CBS --- represents CBS \emph{with} or \emph{without} applying the optional agent merging technique, was  suggested in~\cite{BoyraskyFSS15}. 
Note that in (MA)CBS, to resolve a conflict we perform a split and two new constraints are added. However, in this BP improvement, we find an alternative path for one of the conflicting agents, if possible such that it does not not affect the optimality criteria, thus avoiding the need for the split. 
Their work shows that BP improves the overall runtime.

In~\cite{BoyarskiFSSTBS15}, the authors introduced Improved-CBS (ICBS) which further added two new improvements to (MA)CBS+BP.
The first improvement was Merge and Restart (MR). Agents are merged locally at each node in the search tree in MA-CBS approach, however, under MR the authors suggest that merging agents (considering it as a single agent for the entire search tree) and restarting the search from the beginning can be beneficial. 
%
The second improvement they addressed is called Prioritizing Conflicts (PC).
Choosing arbitrarily a conflict, for example, $C=\langle a_i,a_j,v,t\rangle$, to split for a CT node $N$ may lead to poor choices that may increase the size of the high-level search tree. To avoid that the authors suggest a strategy to prioritize conflicts.
Under PC, they classify each conflict while solving a MAPF problem using (MA)CBS as \emph{cardinal}, \emph{semi-cardinal}, and \emph{non-cardinal}.
(a) Constraints derived from the above conflict $C$ for the node $N$ are $\langle a_i,v,t\rangle$ and $\langle a_j,v,t\rangle$.
The conflict $C$ is cardinal when invoking the low-level search for any of the constrained agents, the cost of its path is increased as compared to its cost in $N$. 
(b) The conflict $C$ for the node $N$ is semi-cardinal when adding only one of the above two constraints causes an increase in the cost of $N$, and adding the other constraint does not change the cost. 
(c) Otherwise, the conflict $C$ is non-cardinal. 
The priority of choosing a conflict for a CT node $N$ using the (BP)(MA)CBS approach should be: cardinal $\succ$ semi-cardinal $\succ$ non-cardinal.  

In this work we employ and adapt \emph{bypassing} and \emph{prioritizing conflict} techniques to further enhance the overall runtime and problem \emph{coverage} of the CBSTU approach. 