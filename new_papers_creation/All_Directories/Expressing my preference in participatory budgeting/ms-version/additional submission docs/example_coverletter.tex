%\documentclass[10pt,a4paper]{article}
%\usepackage[utf8]{inputenc}
%\usepackage{amsmath}
%\usepackage{amsfonts}
%\usepackage{amssymb}
%\begin{document}
%\noindent
%To whom this may concern, \bigskip
%
%\noindent
%We are submitting our paper \emph{How make envy vanish over time} for review at the Operations Research special issue honoring Kenneth Arrow.  \smallskip
%
%\noindent
%This paper follows a long line of research about how to divide goods fairly between agents, when the agents have heterogeneous preferences. What makes our setting unique is that goods are indivisible and arrive online. \smallskip
%
%\noindent
%We ask whether there exists online allocation algorithms with \emph{vanishing envy}, that is, for which envy grows sublinearly in the number of goods (even when an adaptive adversary selects item valuations). When items arrive one at a time, we give polynomial time randomized and deterministic allocation algorithms, and prove that the rate at which envy vanishes for these algorithms is asymptotically optimal. When items arrive in batches, we find that, for every batch,  it is possible to round an envy-free fractional allocation in such way that the rate at which envy increases is minimized. Interestingly, we rely on a mixed integer program to find such a such allocation. \smallskip
%
%\noindent
%A shorter version of the manuscript has been accepted to \emph{19th ACM Conference on Economics and Computation} (a copy of this version is attached). We believe there are some valuable contributions in this paper that do not appear in the earlier version:
%\begin{itemize}
%\item When items arrive one by one, our key finding is that an adaptive adversary's optimal strategy against random assignment is in  fact non-adaptive. This allows us to prove an upper bound on the rate at which envy can vanish (and this bound turns out to be optimal).  The proof of this result is developed here in full detail for the first time.
%\item When items arrive in batches, we find a deterministic algorithm for which envy vanishes at a near-optimal rate. However, this algorithm relies on solving an integer program and can, to the best of our knowledge, not be computed in polynomial time. We extend this result here by designing polynomial time allocation algorithms with optimal envy guarantees for several cases special cases including the case where the number of agents is a constant (which is what we'd expect to see in practice). 
%\item In our basic setting, the algorithm has full information about every item's utility before it has to decide how to allocate it. An extension to this, which appears here for the first time, is a model under which item utilities are only revealed after the item has been allocated. This model is studied here for the first time. We find significant differences between the two models, and establish whether or not envy can vanish for randomized and deterministic algorithms under this model, and if so, at which rate. 
%\end{itemize}
%\smallskip
%
%\noindent
%Sincerely, 
%\end{document}


%\documentclass[10pt,a4paper]{article}
\documentclass[11pt,letterpaper,sans]{article}
\usepackage[scale=1,margin=0.8in,top=0.75in]{geometry}

%\usepackage[maxnames=99]{biblatex} % <==================================


%\usepackage{biblatex}
%\addbibresource{ultimate.bib,abb.bib}
% moderncv themes
%\moderncvstyle{classic}                             % style options are 'casual' (default), 'classic', 'oldstyle' and 'banking'
%\moderncvcolor{blue}                               % color options 'blue' (default), 'orange', 'green', 'red', 'purple', 'grey' and 'black'
%\renewcommand{\familydefault}{\sfdefault}         % to set the default font; use '\sfdefault' for the default sans serif font, '\rmdefault' for the default roman one, or any tex font name
%\nopagenumbers{}                                  % uncomment to suppress automatic page numbering for CVs longer than one page

\usepackage[utf8]{inputenc}
\usepackage{amsmath}
\usepackage{amsfonts}
\usepackage{amssymb}
\usepackage{enumitem}

\setlength{\parindent}{0pt}
\setlength{\parskip}{5pt}

\usepackage{natbib}
\renewcommand{\refname}{\normalsize References}


% personal data
%\name{Gerdus}{Benad\`e}
%\title{Resum\'{e}}                               % optional, remove / comment the line if not wanted
%\address{Tepper School of Business, Ph.D.~Program \\ 5000 Forbes Avenue\\ Pittsburgh, PA 15213, USA}{}{}% optional, remove / comment the line if not wanted; the "postcode city" and and "country" arguments can be omitted or provided empty
%\phone[mobile]{+1~(234)~567~890}                   % optional, remove / comment the line if not wanted
%\phone[fixed]{+1~(646)~358~2940}                    % optional, remove / comment the line if not wanted
%\phone[fax]{+3~(456)~789~012}                      % optional, remove / comment the line if not wanted
%\email{akazachk@cmu.edu}                               % optional, remove / comment the line if not wanted
%\homepage{\href{http://www.andrew.cmu.edu/\textasciitilde akazachk/}{{andrew.cmu.edu/\textasciitilde akazachk/}}}                        % optional, remove / comment the line if not wanted
%\homepage{andrew.cmu.edu/\textasciitilde akazachk}

\begin{document}

%-----       letter       ---------------------------------------------------------
%% recipient data
%\recipient{Operations Research}{}
%\date{April 12, 2022}
%\opening{Dear Editors,}
%\closing{Sincerely,}
%%\enclosure[Enclosure]{curriculum vit\ae{}}          % use an optional argument to use a string other than "Enclosure", or redefine \enclname
%\makelettertitle

%\noindent
%To whom this may concern, \bigskip

\begin{flushright}
	\textbf{Gerdus Benad\`e}\\
 \textbf{Operations Research} \hfill	23 June 2022
\end{flushright}

%On behalf of myself and my coauthors (Aleksandr Kazachkov, Ariel D. Procaccia,  Christos-Alexandros Psomas and David Zeng), I am submitting our paper ``{Fair and efficient dynamic allocation}'' for review at  \textit{Operations Research}.

Dear Editors,  \smallskip

We  are submitting our paper ``{Fair and efficient online allocations}'' for review at  \textit{Operations Research}.
 

%The practical setting that motivated this study is that of a food bank which receives donations of perishable goods at arbitrary intervals. 
We consider allocating indivisible goods that arrive online to heterogeneous agents. Our central question is whether there exists simultaneously fair and efficient allocation algorithms. We measure fairness by \emph{envy}, defined as the extent to which any agent prefer another's assignment over their own, and efficiency by  \emph{Pareto optimality}. 

We answer this question fully for two classes of adversary models. When item values are selected by worst-case (adaptive and non-adaptive) adversaries, we find a sharp trade-off: at best you can provide either non-trivial fairness  or non-trivial efficiency guarantees. In contrast, when item values are drawn from (potentially correlated) distributions we provide   algorithms that find    particular  (fractional)  solutions to an offline linear program which, when used to guide the online decisions, provides simultaneously the strongest possible fairness and efficiency guarantees. 

This paper extends results that have been peer-reviewed and accepted at the ACM Conference on Economics and Computation in 2018 and 2020: 
\begin{enumerate}[topsep=0pt, partopsep=0pt, itemsep=0pt]
	\item[\cite{BKPP18}] Published in the conference proceedings. This work asked whether it is possible to find fair online allocation algorithms   against worst-case adversaries. 
	%This work asked whether it is possible to find fair online allocation algorithms with envy sublinear in the number of items against adaptive adversaries. This question was answered for randomized and deterministic  algorithms when items arrive one-by-one or in batches.   
	\item[\cite{PZ20}] Published  as a non-archival one page extended abstract. 
\end{enumerate}

The submitted manuscript contains key results from \cite{BKPP18} around fair allocation algorithms against worst-case adversaries (Theorems 1-2 and Lemmas 1-5). These results are extended in the following ways:
\begin{enumerate}[topsep=2pt, partopsep=0pt, itemsep=0pt,label=(\roman*)]
	\item We study what happens when you desire simultaneously fair \emph{and efficient} algorithms against worst-case adversaries and  find an impossibility: you must choose whether to be fair or efficient but can not be both (Lemma 6, Theorem 3). 
	\item We also study several  Bayesian adversaries. Here  it is possible to   simultaneously be Pareto optimal ex-post and have strong fairness guarantees (Theorems 4-8), and we provide polynomial time algorithms to this end (Algorithms 1-3). 
\end{enumerate}

An earlier manuscript based on \cite{BKPP18} was submitted to  Operations Research    (OPRE-2018-05-290).
 %This submission was desk-rejected for not containing sufficient contributions over the conference version.
  As described above,   the current manuscript extends the original in several substantial and meaningful ways. 

In accordance with the journal's proceedings policy, we include with the submission: (i) the original proceedings paper \cite{BKPP18}; (ii) the one page abstract \cite{PZ20}; (iii) a separate summary of the contributions of this manuscript beyond \cite{BKPP18}; (iv)  a highlighted version of this manuscript showing overlap with \cite{BKPP18} or \cite{PZ20}. % (in blue). 

We recommend the following editors (in rough order of fit): Jay Sethuraman, Itai Ashlagi, Renato Paes Leme, Nicole Immorlica. Finally, we confirm that none of the authors have  financial conflicts to declare.
\smallskip

Sincerely, \\[-3mm]

Gerdus Benad\`e 
%\makeletterclosing

\vspace{-4mm}

%\scriptsize   
\bibliographystyle{plain}
\bibliography{abb,ultimate}
%\printbibliography

\end{document}