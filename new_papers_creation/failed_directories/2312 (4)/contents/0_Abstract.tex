\begin{abstract}
Zero-Shot Object Counting~(ZSOC) aims to count referred instances of arbitrary classes in a query image without human-annotated exemplars.
To deal with ZSOC, preceding studies proposed a \textbf{two-stage} pipeline: discovering exemplars and counting.
However, there remains a challenge of vulnerability to error propagation of the sequentially designed two-stage process.
In this work, we propose an \textbf{one-stage} baseline, Visual-Language Baseline~(VLBase), exploring the implicit association of the semantic-patch embeddings of CLIP.
Subsequently, we extend the VLBase to Visual-language Counter~(VLCounter) by incorporating three modules devised to tailor VLBase for object counting.
First, we introduce Semantic-conditioned Prompt Tuning~(SPT) within the image encoder to acquire target-highlighted representations.
Second, Learnable Affine Transformation~(LAT) is employed to translate the semantic-patch similarity map to be appropriate for the counting task.
Lastly, we transfer the layer-wisely encoded features to the decoder through Segment-aware Skip Connection~(SaSC) to keep the generalization capability for unseen classes.
Through extensive experiments on FSC147, CARPK, and PUCPR+, we demonstrate the benefits of our end-to-end framework, VLCounter.
Code is available at https://github.com/seunggu0305/VLCounter
\end{abstract}
