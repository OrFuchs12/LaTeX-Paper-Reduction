\section{Implementation Details}
We employ pre-trained I3D~\cite{qian_locate_2022} and C3D~\cite{c3d} models to extract video frame features for Charades-STA and ActivityNet-Captions, respectively. We uniformly sample $T\!=
\!128$ features per video to ensure a fixed length. 
During the pseudo-query generation phase, we employ a Faster R-CNN object detector that is trained on objects enumerated in VisualGenome~\cite{krishna_visual_2017}. We employ a top-$k$ strategy to sample the most probable nouns found in the video segment. We choose a \(k\) value of $5$ based on the experimental analysis by \citet{nam_zero-shot_2021}. 
As for the CEM module, we rely on ConceptNet \cite{speer_conceptnet_2017} for commonsense information and extract the English sub-graph for our experiments. Moreover, we prune the commonsense graph \(G_{C}\) by preserving edge types that convey relevant contextual information, as detailed in Table 1 in the main paper.
We randomly initialize the weights for the GCN-based concept encoder.
Experiments for the balancing hyperparameter $\lambda$, spanning a range of $\lambda \in \{0.75, 0.7, 0.3, 0.25\}$ show that performance is consistently high across all metrics for $\lambda=0.7$, indicating the relative importance of temporal attention-guided loss over the overall localization regression loss.

\section{Ablation Studies}
\label{sec:ablations}
We further present ablation studies that focus on the Commonsense Enhancement Module (CEM) design and the overall efficacy of commonsense integration. Unless specified otherwise, we perform ablations on Charades-STA~\cite{Gao_2017_ICCV} and \modelname with 300 seed concepts.

\begin{figure}[t!]
    \centering
    \begin{subfigure}
        \centering
        \begin{tikzpicture}[scale=0.5]
          \begin{axis}[
            ybar,
            bar width=15pt,
            ymin=0,
            enlarge x limits={abs=25pt},
            legend style={draw=none,at={(0.5,-0.15)},
            anchor=north,legend columns=2},
            xlabel={Metric},
            ylabel={~~},
            nodes near coords,
            every node near coord/.append style={font=\normalsize,text width=0.5cm,rotate=90,align=center,
            xshift=-20pt,
            yshift=-6pt
            },
            symbolic x coords={$R@0.3$,$R@0.5$,$R@0.7$,$mIoU$}, 
            point meta=y,  % the displayed number
            xtick=data,
            legend to name={legqcc},
            legend image code/.code={%
                \draw[#1, draw=none] (0cm,-0.1cm) rectangle (0.6cm,0.1cm);
            },  
            legend style={
                draw=none, % ?
                text depth=0pt,
                at={(0.0,-0.15)},
                anchor=north west,
                legend columns=-1,
                % default spacing:
                column sep=5cm,
                % The text "Legend:"
                /tikz/column 2/.style={column sep=15pt},
                %
                % the space between legend image and text:
                /tikz/every odd column/.append style={column sep=0cm},
            },
            cycle list={blueaccent,orangeaccent}
          ]	
            \addplot[fill=blueaccent] coordinates 
            {($mIoU$,32.73) ($R@0.3$,49.21) ($R@0.5$,34.60) ($R@0.7$,17.93)}; %Solucao 1
            \addplot[fill=orangeaccent] coordinates 
            {($mIoU$,30.33) ($R@0.3$,46.72) ($R@0.5$,29.33) ($R@0.7$,13.69)}; %Solucao 1
            \legend{Enhancement, Concatenation}
        \end{axis}
        \end{tikzpicture}
    \end{subfigure}%
    \begin{subfigure}
        \centering
        \begin{tikzpicture}[scale=0.5]
          \begin{axis}[
            ybar,
            bar width=15pt,
            ymin=0,
            enlarge x limits={abs=25pt},
            legend style={draw=none,at={(0.5,-0.15)},
            anchor=north,legend columns=2},
            xlabel={Metric},
            ylabel={~~},
            nodes near coords,
            every node near coord/.append style={
            font=\normalsize,
            text width=0.5cm,
            rotate=90,
            align=center,
            % visualization depends on=y \as \rawy,
            xshift=-20pt,
            yshift=-7pt},
            symbolic x coords={$R@0.3$,$R@0.5$,$R@0.7$,$mIoU$},
            point meta = y,  % the displayed number
            xtick=data,
          ] 			
            \addplot[fill=blueaccent] coordinates 
            {($mIoU$,33.06) ($R@0.3$,50.98) ($R@0.5$,33.18) ($R@0.7$,16.48)}; %Solucao 1
            \addplot[fill=orangeaccent] coordinates 
            {($mIoU$,29.38) ($R@0.3$,42.77) ($R@0.5$,31.40) ($R@0.7$,15.62)}; 
        \end{axis}
        \end{tikzpicture}
    \end{subfigure}
    \ref{legqcc}
    \vspace{-0.2cm}
    \caption{\modelname performance with enhancement \vs query-concepts concatenation for 300 (left) and 250 (right) seed concept sizes.}
\label{fig:ablation_qcc}
\end{figure}

\subsection{Which Modality to Enhance with Commonsense?}
\label{ablation:cem}
To analyze the importance of commonsense enhancement across modalities, we train \modelname with the following configurations: (1) Only query features $Q$ are enhanced (\textbf{Q}), (2) Only video features $V$ are enhanced (\textbf{V}). In addition, we employ two configurations for which both video and query are enhanced. \modelname's CEM makes use of the same concept vectors, but employs separate enhancement steps for video and text query, \ie, we rely on two separate sub-modules $\phi_{C_{\text{vid}}}(V)$ and $\phi_{C_{\text{pq}}}(Q)$. This design choice stems from the hypothesis that the gap between video and query modalities is exacerbated when dealing with less sophisticated queries. Essentially, less sophisticated (or more general) queries may lack specificity or fail to capture the nuances of the desired information accurately. As a result, the gap between the information contained in the video and the intended query widens, making it more challenging to match the two modalities effectively. 
Having separate projection matrices for video and query allows differently enhancing $V$ and $Q$  with the same commonsense information (through the same concept vectors). 
To test this rationale, (3) we train \modelname with shared weights for $V$ and $Q$, \ie, $\phi_{C_{\text{vid}}}(V)$ and $\phi_{C_{\text{pq}}}(Q)$ are identical (\textbf{VQ}). Finally, (4) we represent the original setting of separate enhancement mechanisms for $V$ and $Q$ as \textbf{V+Q}.

\begin{table}[t!]
\centering
\resizebox{\linewidth}{!}{
\begin{tabular}{lcccc}
\toprule
\textbf{Method}            & \textbf{{R@0.3}}       & \textbf{{R@0.5}}       & \textbf{{R@0.7}}       & \textbf{{mIoU}}        \\\midrule
\modelname \textbf{{(Q)}}    & 39.72            & 22.16             & 8.10              & 26.06         \\
\modelname \textbf{{(V)}}    & {\underline{44.11}}  & {31.64}           & 15.20             & {\underline{29.75}}  \\
\modelname \textbf{{(VQ)}}   & 40.21            & \underline{31.78}     & \textbf{18.65}    & 28.45         \\
\modelname \textbf{{(V+Q)}}  & \textbf{49.21}   & \textbf{34.60}    & {\underline{17.93}}   & \textbf{32.73}       \\\bottomrule
\end{tabular}
}
\vspace{-0.2cm}
\caption{\modelname performance with query enhancement only (\textbf{{Q}}), video enhancement only (\textbf{{V}}), shared video and query enhancement (\textbf{{VQ}}) and separate video and query enhancement (\textbf{{V+Q}}). The best and second-best scores are shown in \textbf{bold} and \underline{underline}, respectively.}
\label{tab:weightsCEM}
\end{table}
Table \ref{tab:weightsCEM} presents results for the aforementioned configurations. We observe a significant drop in performance with \textbf{Q} across all metrics, which shows that the localization abilities are negatively impacted by omitting the video feature enhancement. To further support this observation, we see a consistent increase across all metrics for \textbf{V}, where only video features are enhanced and query feature enhancement is omitted. This highlights the positive impact of incorporating important commonsense information in the visual context for boosting model performance. 
Furthermore, we observe a consistent deterioration in model performance across all metrics in \textbf{VQ} except for $R@0.7$ when compared to \textbf{V+Q}. This could be attributed to the fact that a common enhancement flow for $V$ and $Q$ may potentially collapse diverging sources of information into one latent representation. Separating the enhancement for the two modalities allows disentangling the learned latent representations for video and pseudo-query, thereby capturing different relationships, but with the same underlying commonsense knowledge. Finally, \textbf{V+Q} performing the best across all the aforementioned configurations validates our hypothesis of maintaining separate enhancement flows for video and text query features.

\begin{figure}[t!]
    \centering
    \begin{subfigure}
        \centering
        \begin{tikzpicture}[scale=0.5]
          \begin{axis}[
            ybar,
            bar width=15pt,
            ymin=0,
            enlarge x limits={abs=25pt},
            legend style={draw=none,at={(0.5,-0.15)},
            anchor=north,legend columns=2},
            xlabel={Metric},
            ylabel={Value},
            nodes near coords,
            every node near coord/.append style={font=\normalsize,text width=0.5cm,rotate=90,align=center,
            xshift=-20pt,
            yshift=-7pt
            },
            symbolic x coords={$R@0.3$,$R@0.5$,$R@0.7$,$mIoU$},
            point meta=y,  % the displayed number
            xtick=data,
            legend to name={legprepost},
            legend image code/.code={%
                \draw[#1, draw=none] (0cm,-0.1cm) rectangle (0.6cm,0.1cm);
            },  
            legend style={
                draw=none, % ?
                text depth=0pt,
                at={(0.0,-0.15)},
                anchor=north west,
                legend columns=-1,
                % default spacing:
                column sep=5cm,
                % The text "Legend:"
                /tikz/column 2/.style={column sep=15pt},
                %
                % the space between legend image and text:
                /tikz/every odd column/.append style={column sep=0cm},
            },
            cycle list={blueaccent,orangeaccent}
          ]	
            \addplot[fill=blueaccent] coordinates 
            {($mIoU$, 32.73) ($R@0.3$, 49.21) ($R@0.5$,34.60) ($R@0.7$,17.93)}; 
            \addplot[fill=orangeaccent] coordinates 
            {($mIoU$,28.74) ($R@0.3$,44.64) ($R@0.5$,26.50) ($R@0.7$,13.14)}; 
            \legend{Pre-fusion,Post-fusion} 
        \end{axis}
        \end{tikzpicture}
    \end{subfigure}%
    \begin{subfigure}
        \centering
        \begin{tikzpicture}[scale=0.5]
          \begin{axis}[
            ybar,
            bar width=15pt,
            ymin=0,
            enlarge x limits={abs=25pt},
            legend style={draw=none,at={(0.5,-0.15)},
            anchor=north,legend columns=2},
            xlabel={Metric},
            ylabel={Value},
            nodes near coords,
            every node near coord/.append style={
            font=\normalsize,
            text width=0.5cm,
            rotate=90,
            align=center,
            % visualization depends on=y \as \rawy,
            xshift=-20pt,
            yshift=-7pt},
            symbolic x coords={$R@0.3$,$R@0.5$,$R@0.7$,$mIoU$},
            point meta = y,  % the displayed number
            xtick=data,
          ] 			
            \addplot[fill=blueaccent] coordinates 
            {($mIoU$,33.06) ($R@0.3$,50.98) ($R@0.5$,33.18) ($R@0.7$,16.48)}; %Solucao 1
            \addplot[fill=orangeaccent] coordinates 
            {($mIoU$,28.74) ($R@0.3$,43.09) ($R@0.5$,28.63) ($R@0.7$,13.72)}; %Solucao 1
        \end{axis}
        \end{tikzpicture}
    \end{subfigure}
    \ref{legprepost}
    \vspace{-0.2cm}
    \caption{\modelname performance with pre-fusion enhancement \vs post-fusion enhancement for 300 (left) and 250 (right) seed concept sizes.}
    \label{tab:ablation_pre_vs_post}
    \vspace{-0.3cm}
\end{figure}
\subsection{When to Perform Commonsense Enhancement?}
\label{ablation:prevpost}
\modelname separately enhances both video and query features prior to the cross-modal fusion step. However, an  alternative option would be to perform commonsense enhancement on the unified video-query features after cross-modal fusion and interaction. Accordingly, we present results for pre-fusion as well as post-fusion enhancement. 
In Figure \ref{tab:ablation_pre_vs_post}, we observe that our approach of pre-fusion enhancement works significantly better than post-fusion enhancement across both 300 and 250 concept sizes. We believe the underlying reason for this observation is consistent with our previous prior findings, where employing separate enhancement modules for video and query features is best suited to inject necessary information and allowed \modelname to differently approach video and query enhancement.
\subsection{Does Retaining Relational Information Boost Localization?}\label{ablation:relatinal_coronet}
\modelname employs a weighted directed graph as the concept graph $G_C$, where the edge weight between two nodes is the total number of relational edges from source to target nodes. In this ablation study, we analyze the impact of retaining multi-relational information in contrast to collapsing to a single weighted edge between two nodes. We replace the original $G_C$ version with a multi-relational directed graph, where each edge belongs to one of the relation types in Table 1 in the main paper, and two nodes may be connected with multiple different edges. To this end, we employ Relational Graph Convolutional Networks (RGCNs) \cite{schlichtkrull2018modeling}, where each graph convolution step is defined as
\begin{equation}
\label{eq:rgcn}
    C^{\left(l+1\right)}=\sigma\left(\sum_{r \in R} A_{r} C^{(l)} W_{r}^{(l)}\right).
\end{equation}
Here, $W_{r}^{(l)}$ is the trainable weight matrix for layer $l \in \left[0, L\right]$ and $A_r$ is the adjacency matrix for relation $r \in R$, where $R$ denotes our relation set. 
We experiment with both pre- and post-fusion enhancement in this setup and respectively denote them by \modelname-R and \modelname-R$_{post}$. 

\subsection{Does Retaining Relational Information Boost Localization?}\label{ablation:relatinal_coronet}
\modelname employs a weighted directed graph as the concept graph $G_C$, where the edge weight between two nodes is the total number of relational edges from source to target nodes. In this ablation study, we analyze the impact of retaining multi-relational information in contrast to collapsing to a single weighted edge between two nodes. We replace the original $G_C$ version with a multi-relational directed graph, where each edge belongs to one of the relation types in Table 1 in the main paper, and two nodes may be connected with multiple different edges. To this end, we employ Relational Graph Convolutional Networks (RGCNs) \cite{schlichtkrull2018modeling}, where each graph convolution step is defined as
\begin{equation}
\label{eq:rgcn}
    C^{\left(l+1\right)}=\sigma\left(\sum_{r \in R} A_{r} C^{(l)} W_{r}^{(l)}\right).
\end{equation}
Here, $W_{r}^{(l)}$ is the trainable weight matrix for layer $l \in \left[0, L\right]$ and $A_r$ is the adjacency matrix for relation $r \in R$, where $R$ denotes our relation set. 
We experiment with both pre- and post-fusion enhancement in this setup and respectively denote them by \modelname-R and \modelname-R$_{post}$. 

\subsection{Does Retaining Relational Information Boost Localization?}\label{ablation:relatinal_coronet}
\modelname employs a weighted directed graph as the concept graph $G_C$, where the edge weight between two nodes is the total number of relational edges from source to target nodes. In this ablation study, we analyze the impact of retaining multi-relational information in contrast to collapsing to a single weighted edge between two nodes. We replace the original $G_C$ version with a multi-relational directed graph, where each edge belongs to one of the relation types in Table 1 in the main paper, and two nodes may be connected with multiple different edges. To this end, we employ Relational Graph Convolutional Networks (RGCNs) \cite{schlichtkrull2018modeling}, where each graph convolution step is defined as
\begin{equation}
\label{eq:rgcn}
    C^{\left(l+1\right)}=\sigma\left(\sum_{r \in R} A_{r} C^{(l)} W_{r}^{(l)}\right).
\end{equation}
Here, $W_{r}^{(l)}$ is the trainable weight matrix for layer $l \in \left[0, L\right]$ and $A_r$ is the adjacency matrix for relation $r \in R$, where $R$ denotes our relation set. 
We experiment with both pre- and post-fusion enhancement in this setup and respectively denote them by \modelname-R and \modelname-R$_{post}$. 

\input{supp_sections/tables/relational}
Results in Table \ref{tab:ablation_relational} show that contrary to one's intuition, having a higher-order contextual graph with more relational information does not help localization performance. A multi-relational adjacency matrix is much sparser than its weighted counterpart, where the adjacency matrix is aggregated along the relation dimension. Having a denser adjacency matrix could possibly enhance learning. Overall, this experiment showcases that the association between two given objects is more important in integrating commonsense, rather than the specific relation type.
Results in Table \ref{tab:ablation_relational} show that contrary to one's intuition, having a higher-order contextual graph with more relational information does not help localization performance. A multi-relational adjacency matrix is much sparser than its weighted counterpart, where the adjacency matrix is aggregated along the relation dimension. Having a denser adjacency matrix could possibly enhance learning. Overall, this experiment showcases that the association between two given objects is more important in integrating commonsense, rather than the specific relation type.
Results in Table \ref{tab:ablation_relational} show that contrary to one's intuition, having a higher-order contextual graph with more relational information does not help localization performance. A multi-relational adjacency matrix is much sparser than its weighted counterpart, where the adjacency matrix is aggregated along the relation dimension. Having a denser adjacency matrix could possibly enhance learning. Overall, this experiment showcases that the association between two given objects is more important in integrating commonsense, rather than the specific relation type.
\subsection{Does Auxiliary Commonsense Information Boost Performance?}
\label{ablation:hops}
\subsection{Does Auxiliary Commonsense Information Boost Performance?}
\label{ablation:hops}
\subsection{Does Auxiliary Commonsense Information Boost Performance?}
\label{ablation:hops}
\input{supp_sections/tables/hops}
We analyze the impact of including auxiliary contextual information provided through \(G_{C}\). We examine the performance of \modelname by replacing the proposed seed concept graph \(G_{C}\) with a bigger $1$-hop neighborhood graph. Since including a 1-hop neighborhood leads to an exponential increase in the graph size, we limit $G_C$ to include 1-hop neighborhood only with edge types that add valuable information to the localization setup. Specifically, we include edge types that may involve action information (\ie, verbs) in relation to the objects observed in our video corpus (\eg, $UsedFor$, $CapableOf$, $Causes$, \etc). 
Figure~\ref{fig:ablationHops}
shows the relative performance of this model variant in comparison to the original seed concept (0-hop) graph. Performance consistently worsens across all metrics for both 300 and 250 seed concept sizes, with more drastic drops for 300 concept sizes. We hypothesize that, despite the increased context via a larger graph, the additional information may prove to be noisy, thereby affecting localization accuracy as well as generalization capabilities.
We analyze the impact of including auxiliary contextual information provided through \(G_{C}\). We examine the performance of \modelname by replacing the proposed seed concept graph \(G_{C}\) with a bigger $1$-hop neighborhood graph. Since including a 1-hop neighborhood leads to an exponential increase in the graph size, we limit $G_C$ to include 1-hop neighborhood only with edge types that add valuable information to the localization setup. Specifically, we include edge types that may involve action information (\ie, verbs) in relation to the objects observed in our video corpus (\eg, $UsedFor$, $CapableOf$, $Causes$, \etc). 
Figure~\ref{fig:ablationHops}
shows the relative performance of this model variant in comparison to the original seed concept (0-hop) graph. Performance consistently worsens across all metrics for both 300 and 250 seed concept sizes, with more drastic drops for 300 concept sizes. We hypothesize that, despite the increased context via a larger graph, the additional information may prove to be noisy, thereby affecting localization accuracy as well as generalization capabilities.
We analyze the impact of including auxiliary contextual information provided through \(G_{C}\). We examine the performance of \modelname by replacing the proposed seed concept graph \(G_{C}\) with a bigger $1$-hop neighborhood graph. Since including a 1-hop neighborhood leads to an exponential increase in the graph size, we limit $G_C$ to include 1-hop neighborhood only with edge types that add valuable information to the localization setup. Specifically, we include edge types that may involve action information (\ie, verbs) in relation to the objects observed in our video corpus (\eg, $UsedFor$, $CapableOf$, $Causes$, \etc). 
Figure~\ref{fig:ablationHops}
shows the relative performance of this model variant in comparison to the original seed concept (0-hop) graph. Performance consistently worsens across all metrics for both 300 and 250 seed concept sizes, with more drastic drops for 300 concept sizes. We hypothesize that, despite the increased context via a larger graph, the additional information may prove to be noisy, thereby affecting localization accuracy as well as generalization capabilities.
\begin{table}[t!]
\centering
\resizebox{\columnwidth}{!}{
\begin{tabular}{lccccc}
\toprule
{Encoder} & \textbf{R@0.3} & \textbf{R@0.5} & \textbf{R@0.7} & \textbf{mIoU}  & \textbf{time/epoch} \\ \midrule
{GRU}        	&49.21	&{\textbf{34.60}}&	{\textbf{17.93}} & {\textbf{32.73}} & 74.48s                  \\
{Transformer}  	&{\textbf{53.57}}	&30.67	&13.49  &32.70        & 35.94s                  \\ \bottomrule
\end{tabular}}
\vspace{-0.2cm}
\caption{Performance with recurrent \vs Transformer-based encoders for video and query inputs. Time per epoch is measured in seconds. The best scores are presented in \textbf{bold}.}
\label{tab:encoderAblation}
\end{table}
\subsection{How to Best Encode Inputs?}
\label{ablation:encoder}
We also investigate the impact of adopting a recurrent architecture (GRU/LSTM) \vs Transformers~\cite{vaswani_attention_2017} for generating the video $V$ and pseudo-query $Q$ encodings. Table \ref{tab:encoderAblation} quantitatively compares model performance under such encoding variants for \modelname.
While Transformer-based methods are more than twice as fast as recurrent methods, they surprisingly impede model performance by large margins across most metrics.

\begin{table}[t!]
\centering
\resizebox{\columnwidth}{!}{
\begin{tabular}{lccccc}
\toprule
\textbf{Model} & \textbf{Enhancement} & \textbf{R@0.3} & \textbf{R@0.5} & \textbf{R@0.7} & \textbf{mIou} \\ \midrule
LFVL~\cite{kim2023language}           & None                          & 49.50 & \textbf{34.39} & \textbf{16.95} & 33.19         \\ \midrule
+ CEM          & \multirow{2}{*}{Post}         & \textbf{54.39} & 31.38 & \underline{14.29} & \underline{34.19}         \\
+ CEM$_{250}$  &                               & \underline{53.26} & \underline{33.05} & 13.99 & \textbf{34.30}         \\ \midrule
+ CEM          & \multirow{2}{*}{Pre}          & 49.01 & 28.98 & 12.97 & 31.30         \\
+ CEM$_{250}$  &                               & 49.16 & 29.56 & 13.36 & 31.67         \\ \bottomrule
\end{tabular}}
\vspace{-0.2cm}
\caption{Commonsense enhancement integrated to the LFVL~\cite{kim2023language} method. CEM and CEM$_{250}$ represent enhancement using our commonsense enhancement module with 300 and 250 seed concept graphs ($G_{C}$). Results for both post- and pre-fusion enhancement across CEM and CEM$_{250}$ are displayed. The best and second-best scores are shown in \textbf{bold} and \underline{underline}, respectively.}
\label{tab:lfvl}
\vspace{-0.3cm}
\end{table}


