\section{Implementation Details}
We employ pre-trained I3D~\cite{qian_locate_2022} and C3D~\cite{c3d} models to extract video frame features for Charades-STA and ActivityNet-Captions, respectively. We uniformly sample $T\!=
\!128$ features per video to ensure a fixed length. 
During the pseudo-query generation phase, we employ a Faster R-CNN object detector that is trained on objects enumerated in VisualGenome~\cite{krishna_visual_2017}. We employ a top-$k$ strategy to sample the most probable nouns found in the video segment. We choose a \(k\) value of $5$ based on the experimental analysis by \citet{nam_zero-shot_2021}. 
As for the CEM module, we rely on ConceptNet \cite{speer_conceptnet_2017} for commonsense information and extract the English sub-graph for our experiments. Moreover, we prune the commonsense graph \(G_{C}\) by preserving edge types that convey relevant contextual information, as detailed in Table 1 in the main paper.
We randomly initialize the weights for the GCN-based concept encoder.
Experiments for the balancing hyperparameter $\lambda$, spanning a range of $\lambda \in \{0.75, 0.7, 0.3, 0.25\}$ show that performance is consistently high across all metrics for $\lambda=0.7$, indicating the relative importance of temporal attention-guided loss over the overall localization regression loss.

\section{Ablation Studies}
\label{sec:ablations}
We further present ablation studies that focus on the Commonsense Enhancement Module (CEM) design and the overall efficacy of commonsense integration. Unless specified otherwise, we perform ablations on Charades-STA~\cite{Gao_2017_ICCV} and \modelname with 300 seed concepts.

\subsection{How to Best Inject Commonsense?}
\label{ablation:qcc}
We evaluate the effectiveness of the proposed enhancement mechanism by comparing it against an alternate configuration that omits the enhancement process and instead concatenates the encoded concept vectors with the text query vectors, treating the resultant feature set as the text query features. Figure \ref{fig:ablation_qcc} shows the relative performance of concatenation \vs enhancement across both \modelname configurations with  300 or 250 seed concepts. We observe that enhancement consistently outperforms concatenation, thereby reinforcing the effectiveness of the enhancement flow in injecting necessary commonsense information. Notably, the concatenation configuration for 300 seed concepts still outperforms the PSVL baseline at various recall thresholds \ie, $k=\{0.3, 0.7\}$. This highlights the capacity of commonsense information to enhance localization performance even with a much simpler injection mechanism.
\subsection{How to Best Inject Commonsense?}
\label{ablation:qcc}
We evaluate the effectiveness of the proposed enhancement mechanism by comparing it against an alternate configuration that omits the enhancement process and instead concatenates the encoded concept vectors with the text query vectors, treating the resultant feature set as the text query features. Figure \ref{fig:ablation_qcc} shows the relative performance of concatenation \vs enhancement across both \modelname configurations with  300 or 250 seed concepts. We observe that enhancement consistently outperforms concatenation, thereby reinforcing the effectiveness of the enhancement flow in injecting necessary commonsense information. Notably, the concatenation configuration for 300 seed concepts still outperforms the PSVL baseline at various recall thresholds \ie, $k=\{0.3, 0.7\}$. This highlights the capacity of commonsense information to enhance localization performance even with a much simpler injection mechanism.
\subsection{How to Best Inject Commonsense?}
\label{ablation:qcc}
We evaluate the effectiveness of the proposed enhancement mechanism by comparing it against an alternate configuration that omits the enhancement process and instead concatenates the encoded concept vectors with the text query vectors, treating the resultant feature set as the text query features. Figure \ref{fig:ablation_qcc} shows the relative performance of concatenation \vs enhancement across both \modelname configurations with  300 or 250 seed concepts. We observe that enhancement consistently outperforms concatenation, thereby reinforcing the effectiveness of the enhancement flow in injecting necessary commonsense information. Notably, the concatenation configuration for 300 seed concepts still outperforms the PSVL baseline at various recall thresholds \ie, $k=\{0.3, 0.7\}$. This highlights the capacity of commonsense information to enhance localization performance even with a much simpler injection mechanism.
\input{supp_sections/tables/qcc}
\subsection{Which Modality to Enhance with Commonsense?}
\label{ablation:cem}
To analyze the importance of commonsense enhancement across modalities, we train \modelname with the following configurations: (1) Only query features $Q$ are enhanced (\textbf{Q}), (2) Only video features $V$ are enhanced (\textbf{V}). In addition, we employ two configurations for which both video and query are enhanced. \modelname's CEM makes use of the same concept vectors, but employs separate enhancement steps for video and text query, \ie, we rely on two separate sub-modules $\phi_{C_{\text{vid}}}(V)$ and $\phi_{C_{\text{pq}}}(Q)$. This design choice stems from the hypothesis that the gap between video and query modalities is exacerbated when dealing with less sophisticated queries. Essentially, less sophisticated (or more general) queries may lack specificity or fail to capture the nuances of the desired information accurately. As a result, the gap between the information contained in the video and the intended query widens, making it more challenging to match the two modalities effectively. 
Having separate projection matrices for video and query allows differently enhancing $V$ and $Q$  with the same commonsense information (through the same concept vectors). 
To test this rationale, (3) we train \modelname with shared weights for $V$ and $Q$, \ie, $\phi_{C_{\text{vid}}}(V)$ and $\phi_{C_{\text{pq}}}(Q)$ are identical (\textbf{VQ}). Finally, (4) we represent the original setting of separate enhancement mechanisms for $V$ and $Q$ as \textbf{V+Q}.

\begin{table}[t!]
\centering
\resizebox{\linewidth}{!}{
\begin{tabular}{lcccc}
\toprule
\textbf{Method}            & \textbf{{R@0.3}}       & \textbf{{R@0.5}}       & \textbf{{R@0.7}}       & \textbf{{mIoU}}        \\\midrule
\modelname \textbf{{(Q)}}    & 39.72            & 22.16             & 8.10              & 26.06         \\
\modelname \textbf{{(V)}}    & {\underline{44.11}}  & {31.64}           & 15.20             & {\underline{29.75}}  \\
\modelname \textbf{{(VQ)}}   & 40.21            & \underline{31.78}     & \textbf{18.65}    & 28.45         \\
\modelname \textbf{{(V+Q)}}  & \textbf{49.21}   & \textbf{34.60}    & {\underline{17.93}}   & \textbf{32.73}       \\\bottomrule
\end{tabular}
}
\vspace{-0.2cm}
\caption{\modelname performance with query enhancement only (\textbf{{Q}}), video enhancement only (\textbf{{V}}), shared video and query enhancement (\textbf{{VQ}}) and separate video and query enhancement (\textbf{{V+Q}}). The best and second-best scores are shown in \textbf{bold} and \underline{underline}, respectively.}
\label{tab:weightsCEM}
\end{table}
Table \ref{tab:weightsCEM} presents results for the aforementioned configurations. We observe a significant drop in performance with \textbf{Q} across all metrics, which shows that the localization abilities are negatively impacted by omitting the video feature enhancement. To further support this observation, we see a consistent increase across all metrics for \textbf{V}, where only video features are enhanced and query feature enhancement is omitted. This highlights the positive impact of incorporating important commonsense information in the visual context for boosting model performance. 
Furthermore, we observe a consistent deterioration in model performance across all metrics in \textbf{VQ} except for $R@0.7$ when compared to \textbf{V+Q}. This could be attributed to the fact that a common enhancement flow for $V$ and $Q$ may potentially collapse diverging sources of information into one latent representation. Separating the enhancement for the two modalities allows disentangling the learned latent representations for video and pseudo-query, thereby capturing different relationships, but with the same underlying commonsense knowledge. Finally, \textbf{V+Q} performing the best across all the aforementioned configurations validates our hypothesis of maintaining separate enhancement flows for video and text query features.

\begin{figure}[t!]
    \centering
    \begin{subfigure}
        \centering
        \begin{tikzpicture}[scale=0.5]
          \begin{axis}[
            ybar,
            bar width=15pt,
            ymin=0,
            enlarge x limits={abs=25pt},
            legend style={draw=none,at={(0.5,-0.15)},
            anchor=north,legend columns=2},
            xlabel={Metric},
            ylabel={Value},
            nodes near coords,
            every node near coord/.append style={font=\normalsize,text width=0.5cm,rotate=90,align=center,
            xshift=-20pt,
            yshift=-7pt
            },
            symbolic x coords={$R@0.3$,$R@0.5$,$R@0.7$,$mIoU$},
            point meta=y,  % the displayed number
            xtick=data,
            legend to name={legprepost},
            legend image code/.code={%
                \draw[#1, draw=none] (0cm,-0.1cm) rectangle (0.6cm,0.1cm);
            },  
            legend style={
                draw=none, % ?
                text depth=0pt,
                at={(0.0,-0.15)},
                anchor=north west,
                legend columns=-1,
                % default spacing:
                column sep=5cm,
                % The text "Legend:"
                /tikz/column 2/.style={column sep=15pt},
                %
                % the space between legend image and text:
                /tikz/every odd column/.append style={column sep=0cm},
            },
            cycle list={blueaccent,orangeaccent}
          ]	
            \addplot[fill=blueaccent] coordinates 
            {($mIoU$, 32.73) ($R@0.3$, 49.21) ($R@0.5$,34.60) ($R@0.7$,17.93)}; 
            \addplot[fill=orangeaccent] coordinates 
            {($mIoU$,28.74) ($R@0.3$,44.64) ($R@0.5$,26.50) ($R@0.7$,13.14)}; 
            \legend{Pre-fusion,Post-fusion} 
        \end{axis}
        \end{tikzpicture}
    \end{subfigure}%
    \begin{subfigure}
        \centering
        \begin{tikzpicture}[scale=0.5]
          \begin{axis}[
            ybar,
            bar width=15pt,
            ymin=0,
            enlarge x limits={abs=25pt},
            legend style={draw=none,at={(0.5,-0.15)},
            anchor=north,legend columns=2},
            xlabel={Metric},
            ylabel={Value},
            nodes near coords,
            every node near coord/.append style={
            font=\normalsize,
            text width=0.5cm,
            rotate=90,
            align=center,
            % visualization depends on=y \as \rawy,
            xshift=-20pt,
            yshift=-7pt},
            symbolic x coords={$R@0.3$,$R@0.5$,$R@0.7$,$mIoU$},
            point meta = y,  % the displayed number
            xtick=data,
          ] 			
            \addplot[fill=blueaccent] coordinates 
            {($mIoU$,33.06) ($R@0.3$,50.98) ($R@0.5$,33.18) ($R@0.7$,16.48)}; %Solucao 1
            \addplot[fill=orangeaccent] coordinates 
            {($mIoU$,28.74) ($R@0.3$,43.09) ($R@0.5$,28.63) ($R@0.7$,13.72)}; %Solucao 1
        \end{axis}
        \end{tikzpicture}
    \end{subfigure}
    \ref{legprepost}
    \vspace{-0.2cm}
    \caption{\modelname performance with pre-fusion enhancement \vs post-fusion enhancement for 300 (left) and 250 (right) seed concept sizes.}
    \label{tab:ablation_pre_vs_post}
    \vspace{-0.3cm}
\end{figure}
\subsection{When to Perform Commonsense Enhancement?}
\label{ablation:prevpost}
\modelname separately enhances both video and query features prior to the cross-modal fusion step. However, an  alternative option would be to perform commonsense enhancement on the unified video-query features after cross-modal fusion and interaction. Accordingly, we present results for pre-fusion as well as post-fusion enhancement. 
In Figure \ref{tab:ablation_pre_vs_post}, we observe that our approach of pre-fusion enhancement works significantly better than post-fusion enhancement across both 300 and 250 concept sizes. We believe the underlying reason for this observation is consistent with our previous prior findings, where employing separate enhancement modules for video and query features is best suited to inject necessary information and allowed \modelname to differently approach video and query enhancement.
\begin{table}[t!]
\centering
\resizebox{\linewidth}{!}{
\begin{tabular}{lcccc}
\toprule
\textbf{Model}  &   \textbf{R@0.3}  & \textbf{R@0.5}    &   \textbf{R@0.7}  & \textbf{mIoU}  \\ \midrule
\modelname      &   \textbf{49.21}         & \textbf{34.60}    &   \textbf{17.93}  & \textbf{{32.73}} \\
\modelname-R  & 40.52 & \underline{27.92} & \underline{13.85} & 27.80  \\
\modelname-R$_{post}$ & \underline{46.83} & 25.57 & 12.45 & \underline{30.91} \\ \bottomrule
\end{tabular}
}
\vspace{-0.2cm}
\caption{\modelname performance with multi-relational directed $G_C$ with pre- (\modelname-R) and post-fusion (\modelname-R$_{post}$) enhancement. The best and second-best scores are shown in \textbf{bold} and
\underline{underline}, respectively.}
\label{tab:ablation_relational}
\end{table}

\begin{figure}[t!]
    \centering
    \begin{subfigure}
        \centering
        \begin{tikzpicture}[scale=0.5]
          \begin{axis}[
            ybar,
            bar width=15pt,
            ymin=0,
            enlarge x limits={abs=25pt},
            legend style={draw=none,at={(0.5,-0.15)},
            anchor=north,legend columns=2},
            xlabel={Metric},
            ylabel={Value},
            nodes near coords,
            every node near coord/.append style={font=\normalsize,text width=0.5cm,rotate=90,align=center,
            xshift=-20pt,
            yshift=-7pt},
            symbolic x coords={$R@0.3$,$R@0.5$,$R@0.7$,$mIoU$},
            point meta=y,  % the displayed number
            xtick=data,
            legend to name={leg},
            legend image code/.code={%
                \draw[#1, draw=none] (0cm,-0.1cm) rectangle (0.6cm,0.1cm);
            },  
            legend style={
                draw=none, % ?
                text depth=0pt,
                at={(0.0,-0.15)},
                anchor=north west,
                legend columns=-1,
                % default spacing:
                column sep=5cm,
                % The text "Legend:"
                /tikz/column 2/.style={column sep=15pt},
                %
                % the space between legend image and text:
                /tikz/every odd column/.append style={column sep=0cm},
            },
            cycle list={blueaccent,orangeaccent}
          ]	
            \addplot[fill=blueaccent] coordinates 
            {($R@0.3$, 49.21) ($R@0.5$,34.60) ($R@0.7$, 17.93) ($mIoU$, 32.73)}; %Solucao 1
            \addplot[fill=orangeaccent] coordinates 
            {($R@0.3$, 42.29) ($R@0.5$,28.81) ($R@0.7$,14.68) ($mIoU$, 28.44)}; %Solucao 1
            \legend{0-Hop,1-Hop}
        \end{axis}
        \end{tikzpicture}
    \end{subfigure}%
    \begin{subfigure}
        \centering
        \begin{tikzpicture}[scale=0.5]
          \begin{axis}[
            ybar,
            bar width=15pt,
            ymin=0,
            enlarge x limits={abs=25pt},
            legend style={draw=none,at={(0.5,-0.15)},
            anchor=north,legend columns=2},
            xlabel={Metric},
            ylabel={Value},
            nodes near coords,
            every node near coord/.append style={
            font=\normalsize,
            text width=0.5cm,
            rotate=90,
            align=center,
            % visualization depends on=y \as \rawy,
            xshift=-20pt,
            yshift=-7pt},
            symbolic x coords={$R@0.3$,$R@0.5$,$R@0.7$,$mIoU$},
            point meta = y,  % the displayed number
            xtick=data,
          ] 			
            \addplot[fill=blueaccent] coordinates 
            {($mIoU$, 33.06) ($R@0.3$, 50.98) ($R@0.5$,33.18) ($R@0.7$,16.48)}; %Solucao 1
            \addplot[fill=orangeaccent] coordinates 
            {($mIoU$, 30.65) ($R@0.3$, 45.33) ($R@0.5$,30.99) ($R@0.7$,15.15)}; %Solucao 1
        \end{axis}
        \end{tikzpicture}
    \end{subfigure}
    \ref{leg}
    \vspace{-0.2cm}
    \caption{We compare \modelname performance with 1-hop neighborhood graphs with their 0-hop neighborhood graph counterparts for 300 (left) and 250 (right) seed concepts.}
    \label{fig:ablationHops}
    \vspace{-0.2cm}
\end{figure}
\begin{table}[t!]
\centering
\resizebox{\columnwidth}{!}{
\begin{tabular}{lccccc}
\toprule
{Encoder} & \textbf{R@0.3} & \textbf{R@0.5} & \textbf{R@0.7} & \textbf{mIoU}  & \textbf{time/epoch} \\ \midrule
{GRU}        	&49.21	&{\textbf{34.60}}&	{\textbf{17.93}} & {\textbf{32.73}} & 74.48s                  \\
{Transformer}  	&{\textbf{53.57}}	&30.67	&13.49  &32.70        & 35.94s                  \\ \bottomrule
\end{tabular}}
\vspace{-0.2cm}
\caption{Performance with recurrent \vs Transformer-based encoders for video and query inputs. Time per epoch is measured in seconds. The best scores are presented in \textbf{bold}.}
\label{tab:encoderAblation}
\end{table}
\subsection{How to Best Encode Inputs?}
\label{ablation:encoder}
We also investigate the impact of adopting a recurrent architecture (GRU/LSTM) \vs Transformers~\cite{vaswani_attention_2017} for generating the video $V$ and pseudo-query $Q$ encodings. Table \ref{tab:encoderAblation} quantitatively compares model performance under such encoding variants for \modelname.
While Transformer-based methods are more than twice as fast as recurrent methods, they surprisingly impede model performance by large margins across most metrics.

\subsection{Does Commonsense Help in Language-free Setups?}
\label{ablation:lfvl}
\subsection{Does Commonsense Help in Language-free Setups?}
\label{ablation:lfvl}
\subsection{Does Commonsense Help in Language-free Setups?}
\label{ablation:lfvl}
\input{supp_sections/tables/lfvl}
We also conduct an experiment to test the effectiveness of our CEM approach on a language-free NLVL (LFVL) setting \cite{kim2023language}. LFVL eliminates the need for query annotations by leveraging the cross-modal understanding of CLIP~\cite{radford2021learning} to utilize visual features as textual information. We integrate our commonsense enhancement mechanism into the LFVL pipeline to analyze its impact in this setup. Table \ref{tab:lfvl} compares model performances with two variants, CEM and CEM$_{250}$, which respectively contain 300 and 250 seed concepts. Furthermore, we examine the effectiveness of commonsense enhancement in a post- and pre-fusion setup.
We find that there is a significant increase in the $mIoU$ and $R@0.3$ scores with both CEM and CEM$_{250}$ in post-fusion setup. 
This indicates that the integration of commonsense enhancement positively impacts the overall localization performance. Notably, the comparison between post- and pre-fusion enhancement reveals a striking difference in performance. These findings suggest that enhancing the fused video-query representation with commonsense information is more beneficial compared to enhancing a language-free query representation. The results imply that enhancing the fused representation allows for a more effective alignment between video and query, leading to improved localization performance.
We also conduct an experiment to test the effectiveness of our CEM approach on a language-free NLVL (LFVL) setting \cite{kim2023language}. LFVL eliminates the need for query annotations by leveraging the cross-modal understanding of CLIP~\cite{radford2021learning} to utilize visual features as textual information. We integrate our commonsense enhancement mechanism into the LFVL pipeline to analyze its impact in this setup. Table \ref{tab:lfvl} compares model performances with two variants, CEM and CEM$_{250}$, which respectively contain 300 and 250 seed concepts. Furthermore, we examine the effectiveness of commonsense enhancement in a post- and pre-fusion setup.
We find that there is a significant increase in the $mIoU$ and $R@0.3$ scores with both CEM and CEM$_{250}$ in post-fusion setup. 
This indicates that the integration of commonsense enhancement positively impacts the overall localization performance. Notably, the comparison between post- and pre-fusion enhancement reveals a striking difference in performance. These findings suggest that enhancing the fused video-query representation with commonsense information is more beneficial compared to enhancing a language-free query representation. The results imply that enhancing the fused representation allows for a more effective alignment between video and query, leading to improved localization performance.
We also conduct an experiment to test the effectiveness of our CEM approach on a language-free NLVL (LFVL) setting \cite{kim2023language}. LFVL eliminates the need for query annotations by leveraging the cross-modal understanding of CLIP~\cite{radford2021learning} to utilize visual features as textual information. We integrate our commonsense enhancement mechanism into the LFVL pipeline to analyze its impact in this setup. Table \ref{tab:lfvl} compares model performances with two variants, CEM and CEM$_{250}$, which respectively contain 300 and 250 seed concepts. Furthermore, we examine the effectiveness of commonsense enhancement in a post- and pre-fusion setup.
We find that there is a significant increase in the $mIoU$ and $R@0.3$ scores with both CEM and CEM$_{250}$ in post-fusion setup. 
This indicates that the integration of commonsense enhancement positively impacts the overall localization performance. Notably, the comparison between post- and pre-fusion enhancement reveals a striking difference in performance. These findings suggest that enhancing the fused video-query representation with commonsense information is more beneficial compared to enhancing a language-free query representation. The results imply that enhancing the fused representation allows for a more effective alignment between video and query, leading to improved localization performance.

