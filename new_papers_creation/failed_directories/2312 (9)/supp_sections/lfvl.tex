\subsection{Does Commonsense Help in Language-free Setups?}
\label{ablation:lfvl}
\begin{table}[t!]
\centering
\resizebox{\columnwidth}{!}{
\begin{tabular}{lccccc}
\toprule
\textbf{Model} & \textbf{Enhancement} & \textbf{R@0.3} & \textbf{R@0.5} & \textbf{R@0.7} & \textbf{mIou} \\ \midrule
LFVL~\cite{kim2023language}           & None                          & 49.50 & \textbf{34.39} & \textbf{16.95} & 33.19         \\ \midrule
+ CEM          & \multirow{2}{*}{Post}         & \textbf{54.39} & 31.38 & \underline{14.29} & \underline{34.19}         \\
+ CEM$_{250}$  &                               & \underline{53.26} & \underline{33.05} & 13.99 & \textbf{34.30}         \\ \midrule
+ CEM          & \multirow{2}{*}{Pre}          & 49.01 & 28.98 & 12.97 & 31.30         \\
+ CEM$_{250}$  &                               & 49.16 & 29.56 & 13.36 & 31.67         \\ \bottomrule
\end{tabular}}
\vspace{-0.2cm}
\caption{Commonsense enhancement integrated to the LFVL~\cite{kim2023language} method. CEM and CEM$_{250}$ represent enhancement using our commonsense enhancement module with 300 and 250 seed concept graphs ($G_{C}$). Results for both post- and pre-fusion enhancement across CEM and CEM$_{250}$ are displayed. The best and second-best scores are shown in \textbf{bold} and \underline{underline}, respectively.}
\label{tab:lfvl}
\vspace{-0.3cm}
\end{table}

We also conduct an experiment to test the effectiveness of our CEM approach on a language-free NLVL (LFVL) setting \cite{kim2023language}. LFVL eliminates the need for query annotations by leveraging the cross-modal understanding of CLIP~\cite{radford2021learning} to utilize visual features as textual information. We integrate our commonsense enhancement mechanism into the LFVL pipeline to analyze its impact in this setup. Table \ref{tab:lfvl} compares model performances with two variants, CEM and CEM$_{250}$, which respectively contain 300 and 250 seed concepts. Furthermore, we examine the effectiveness of commonsense enhancement in a post- and pre-fusion setup.
We find that there is a significant increase in the $mIoU$ and $R@0.3$ scores with both CEM and CEM$_{250}$ in post-fusion setup. 
This indicates that the integration of commonsense enhancement positively impacts the overall localization performance. Notably, the comparison between post- and pre-fusion enhancement reveals a striking difference in performance. These findings suggest that enhancing the fused video-query representation with commonsense information is more beneficial compared to enhancing a language-free query representation. The results imply that enhancing the fused representation allows for a more effective alignment between video and query, leading to improved localization performance.