\subsection{Does Retaining Relational Information Boost Localization?}\label{ablation:relatinal_coronet}
\modelname employs a weighted directed graph as the concept graph $G_C$, where the edge weight between two nodes is the total number of relational edges from source to target nodes. In this ablation study, we analyze the impact of retaining multi-relational information in contrast to collapsing to a single weighted edge between two nodes. We replace the original $G_C$ version with a multi-relational directed graph, where each edge belongs to one of the relation types in Table 1 in the main paper, and two nodes may be connected with multiple different edges. To this end, we employ Relational Graph Convolutional Networks (RGCNs) \cite{schlichtkrull2018modeling}, where each graph convolution step is defined as
\begin{equation}
\label{eq:rgcn}
    C^{\left(l+1\right)}=\sigma\left(\sum_{r \in R} A_{r} C^{(l)} W_{r}^{(l)}\right).
\end{equation}
Here, $W_{r}^{(l)}$ is the trainable weight matrix for layer $l \in \left[0, L\right]$ and $A_r$ is the adjacency matrix for relation $r \in R$, where $R$ denotes our relation set. 
We experiment with both pre- and post-fusion enhancement in this setup and respectively denote them by \modelname-R and \modelname-R$_{post}$. 

\begin{table}[t!]
\centering
\resizebox{\linewidth}{!}{
\begin{tabular}{lcccc}
\toprule
\textbf{Model}  &   \textbf{R@0.3}  & \textbf{R@0.5}    &   \textbf{R@0.7}  & \textbf{mIoU}  \\ \midrule
\modelname      &   \textbf{49.21}         & \textbf{34.60}    &   \textbf{17.93}  & \textbf{{32.73}} \\
\modelname-R  & 40.52 & \underline{27.92} & \underline{13.85} & 27.80  \\
\modelname-R$_{post}$ & \underline{46.83} & 25.57 & 12.45 & \underline{30.91} \\ \bottomrule
\end{tabular}
}
\vspace{-0.2cm}
\caption{\modelname performance with multi-relational directed $G_C$ with pre- (\modelname-R) and post-fusion (\modelname-R$_{post}$) enhancement. The best and second-best scores are shown in \textbf{bold} and
\underline{underline}, respectively.}
\label{tab:ablation_relational}
\end{table}

Results in Table \ref{tab:ablation_relational} show that contrary to one's intuition, having a higher-order contextual graph with more relational information does not help localization performance. A multi-relational adjacency matrix is much sparser than its weighted counterpart, where the adjacency matrix is aggregated along the relation dimension. Having a denser adjacency matrix could possibly enhance learning. Overall, this experiment showcases that the association between two given objects is more important in integrating commonsense, rather than the specific relation type.