%File: formatting-instructions-latex-2024.tex
%release 2024.0
\documentclass[letterpaper]{article} % DO NOT CHANGE THIS
\usepackage{aaai24}  % DO NOT CHANGE THIS
\usepackage{times}  % DO NOT CHANGE THIS
\usepackage{helvet}  % DO NOT CHANGE THIS
\usepackage{courier}  % DO NOT CHANGE THIS
\usepackage[hyphens]{url}  % DO NOT CHANGE THIS
\usepackage{graphicx} % DO NOT CHANGE THIS
\urlstyle{rm} % DO NOT CHANGE THIS
\def\UrlFont{\rm}  % DO NOT CHANGE THIS
\usepackage{natbib}  % DO NOT CHANGE THIS AND DO NOT ADD ANY OPTIONS TO IT
\usepackage{caption} % DO NOT CHANGE THIS AND DO NOT ADD ANY OPTIONS TO IT
\frenchspacing  % DO NOT CHANGE THIS
\setlength{\pdfpagewidth}{8.5in}  % DO NOT CHANGE THIS
\setlength{\pdfpageheight}{11in}  % DO NOT CHANGE THIS
%
% These are recommended to typeset algorithms but not required. See the subsubsection on algorithms. Remove them if you don't have algorithms in your paper.
% \usepackage{algorithm}
% \usepackage{algorithmic}
\usepackage[ruled,vlined]{algorithm2e}

%
% These are are recommended to typeset listings but not required. See the subsubsection on listing. Remove this block if you don't have listings in your paper.
% \usepackage{newfloat}
% \usepackage{listings}
% \DeclareCaptionStyle{ruled}{labelfont=normalfont,labelsep=colon,strut=off} % DO NOT CHANGE THIS
% \lstset{%
%       basicstyle={\footnotesize\ttfamily},% footnotesize acceptable for monospace
%       numbers=left,numberstyle=\footnotesize,xleftmargin=2em,% show line numbers, remove this entire line if you don't want the numbers.
%       aboveskip=0pt,belowskip=0pt,%
%       showstringspaces=false,tabsize=2,breaklines=true}
% \floatstyle{ruled}
% \newfloat{listing}{tb}{lst}{}
% \floatname{listing}{Listing}
%
% Keep the \pdfinfo as shown here. There's no need
% for you to add the /Title and /Author tags.

% % TMP
% \usepackage{xcolor}
% \newcommand{\ml}[1]{\textcolor{blue}{[Mathias] #1}}
% \newcommand{\qt}[1]{\textcolor{red}{[QT] #1}}
% \newcommand{\fs}[1]{\textcolor{purple}{[FS] #1}}

\usepackage{amsthm}
\newtheorem{assumption}{Assumption}
\newtheorem{proposition}{Proposition}
\newtheorem{theorem}{Theorem}
\newtheorem*{proposition*}{Proposition}
\newtheorem{lemma}[theorem]{Lemma}
\newtheorem*{lemma*}{Lemma}
\newtheorem{corollary}{Corollary}
\newtheorem*{corollary*}{Corollary}
\newtheorem*{remark}{Remark}
\usepackage{multirow}

\usepackage{caption}
\usepackage{subcaption}
\usepackage{mathtools}
\usepackage{amssymb}
% \usepackage[normalem]{ulem}

\input{math_commands}

\newcommand\abs[1]{\left\lvert#1\right\rvert}
\def\cN{\mathcal{N}}
\def\I{\mathbb{I}}
\def\e{\mathbb{E}}
% \newcommand{\E}{\mathbb{E}}
\def\V{\mathbb{V}}
% \newcommand\norm[1]{\left\lVert#1\right\rVert}

% DISALLOWED PACKAGES
% \usepackage{authblk} -- This package is specifically forbidden
% \usepackage{balance} -- This package is specifically forbidden
% \usepackage{color (if used in text)
% \usepackage{CJK} -- This package is specifically forbidden
% \usepackage{float} -- This package is specifically forbidden
% \usepackage{flushend} -- This package is specifically forbidden
% \usepackage{fontenc} -- This package is specifically forbidden
% \usepackage{fullpage} -- This package is specifically forbidden
% \usepackage{geometry} -- This package is specifically forbidden
% \usepackage{grffile} -- This package is specifically forbidden
% \usepackage{hyperref} -- This package is specifically forbidden
% \usepackage{navigator} -- This package is specifically forbidden
% (or any other package that embeds links such as navigator or hyperref)
% \indentfirst} -- This package is specifically forbidden
% \layout} -- This package is specifically forbidden
% \multicol} -- This package is specifically forbidden
% \nameref} -- This package is specifically forbidden
% \usepackage{savetrees} -- This package is specifically forbidden
% \usepackage{setspace} -- This package is specifically forbidden
% \usepackage{stfloats} -- This package is specifically forbidden
% \usepackage{tabu} -- This package is specifically forbidden
% \usepackage{titlesec} -- This package is specifically forbidden
% \usepackage{tocbibind} -- This package is specifically forbidden
% \usepackage{ulem} -- This package is specifically forbidden
% \usepackage{wrapfig} -- This package is specifically forbidden
% DISALLOWED COMMANDS
% \nocopyright -- Your paper will not be published if you use this command
% \addtolength -- This command may not be used
% \balance -- This command may not be used
% \baselinestretch -- Your paper will not be published if you use this command
% \clearpage -- No page breaks of any kind may be used for the final version of your paper
% \columnsep -- This command may not be used
% % \newpage -- No page breaks of any kind may be used for the final version of your paper
% \pagebreak -- No page breaks of any kind may be used for the final version of your paperr
% \pagestyle -- This command may not be used
% \tiny -- This is not an acceptable font size.
% \vspace{- -- No negative value may be used in proximity of a caption, figure, table, section, subsection, subsubsection, or reference
% \vskip{- -- No negative value may be used to alter spacing above or below a caption, figure, table, section, subsection, subsubsection, or reference

\setcounter{secnumdepth}{1} %May be changed to 1 or 2 if section numbers are desired.

% The file aaai24.sty is the style file for AAAI Press
% proceedings, working notes, and technical reports.
%

% Title

% Your title must be in mixed case, not sentence case.
% That means all verbs (including short verbs like be, is, using,and go),
% nouns, adverbs, adjectives should be capitalized, including both words in hyphenated terms, while
% articles, conjunctions, and prepositions are lower case unless they
% directly follow a colon or long dash
\title{DP-AdamBC: Your DP-Adam Is Actually DP-SGD \\ (Unless You Apply Bias Correction)}
\author{
    %Authors
    % All authors must be in the same font size and format.
    Qiaoyue Tang, Frederick Shpilevskiy, Mathias L\'ecuyer
}
\affiliations{
    %Afiliations
    University of British Columbia \\
    % If you have multiple authors and multiple affiliations
    % use superscripts in text and roman font to identify them.
    % For example,

    % Sunil Issar\textsuperscript{\rm 2},
    % J. Scott Penberthy\textsuperscript{\rm 3},
    % George Ferguson\textsuperscript{\rm 4},
    % Hans Guesgen\textsuperscript{\rm 5}
    % Note that the comma should be placed after the superscript
    % 2366 Main Mall, \\
    Vancouver, British Columbia, Canada\\
    % email address must be in roman text type, not monospace or sans serif
    \{qiaoyuet, fshpil\}@cs.ubc.ca, mathias.lecuyer@ubc.ca
%
% See more examples next
}

%Example, Single Author, ->> remove \iffalse,\fi and place them surrounding AAAI title to use it
\iffalse
\title{My Publication Title --- Single Author}
\author {
    Author Name
}
\affiliations{
    Affiliation\\
    Affiliation Line 2\\
    name@example.com
}
\fi

\iffalse
%Example, Multiple Authors, ->> remove \iffalse,\fi and place them surrounding AAAI title to use it
\title{My Publication Title --- Multiple Authors}
\author {
    % Authors
    First Author Name\textsuperscript{\rm 1,\rm 2},
    Second Author Name\textsuperscript{\rm 2},
    Third Author Name\textsuperscript{\rm 1}
}
\affiliations {
    % Affiliations
    \textsuperscript{\rm 1}Affiliation 1\\
    \textsuperscript{\rm 2}Affiliation 2\\
    firstAuthor@affiliation1.com, secondAuthor@affilation2.com, thirdAuthor@affiliation1.com
}
\fi


% REMOVE THIS: bibentry
% This is only needed to show inline citations in the guidelines document. You should not need it and can safely delete it.
% \usepackage{bibentry}
% END REMOVE bibentry

\begin{document}

\maketitle

\begin{abstract}
    The Adam optimizer is a popular choice in contemporary deep learning, due to its strong empirical performance. However we observe that in privacy sensitive scenarios, the traditional use of Differential Privacy (DP) with the Adam optimizer leads to sub-optimal performance on several tasks. We find that this performance degradation is due to a DP bias in Adam's second moment estimator, introduced by the addition of independent noise in the gradient computation to enforce DP guarantees.
    This DP bias leads to a different scaling for low variance parameter updates, that is inconsistent with the behavior of non-private Adam. %, and Adam's sign descent interpretation.
    We propose DP-AdamBC, an optimization algorithm which removes the bias in the second moment estimation and retrieves the expected behaviour of Adam.
    Empirically, DP-AdamBC significantly improves the optimization performance of DP-Adam by up to 3.5\% in final accuracy in image, text, and graph node classification tasks.
\end{abstract}

\section{Introduction}
\label{sec:intro}
The Adam optimization algorithm \citep{orig_adam} is the default optimizer for several deep learning architectures and tasks, notably in Natural Language Processing (NLP), for which Stochastic Gradient Descent (SGD) tends to struggle. Even in vision tasks where Adam is less prevalent, it typically requires less parameter tuning than SGD to reach good performance.

On all these tasks, deep learning models can leak information about their training set \citep{carlini2019secret,carlini2021extracting,carlini2022membership,balle2022reconstructing}.
We consider settings in which the deep learning model's training data is privacy sensitive, and models are trained with Differential Privacy \citep{dwork2006calibrating,abadi2016deep} to provably prevent training example information leakage \citep{wasserman2010statistical}.
Intuitively, training DP models requires computing each minibatch gradient with DP guarantees by clipping per-example gradients and adding Gaussian noise (\S\ref{sec:motiv}), to bound the maximal influence of any data-point on the final model.
The DP gradients can then feed into any optimization algorithm without modification to update the model's parameters.
Due to its success in the non-private setting, Adam is also prevalent when training  DP models, for NLP \citep{li2021} and GNN \citep{daigavane2022nodelevel} models.
%
However we observe that when combined with DP, Adam does not perform as well as without privacy constraints: Adam suffers a larger degradation of performance compared to SGD on vision tasks, while NLP models perform poorly when training from scratch.

To understand this effect, we go back to the original intuition behind Adam \citep{orig_adam} that relies on exponential moving averages estimating the first and second moments of mini-batch gradients. We show that while DP noise does not affect the first moment, it does add a constant bias to the second.
%
Drawing on a recent empirical investigation that suggests that the performance of Adam may be linked to its update rule performing a smooth version of the sign descent update \citep{kunstner2023heavytailed}, we show that the additive shift in Adam's second moment estimate caused by DP noise moves the Adam update away from that of sign descent, by scaling the gradient dimensions with different magnitudes differently.
Indeed, under typical DP parameters, the DP bias added to the second moment estimates of DP-Adam dominate the second moment estimate, and makes DP-Adam a rescaled version of DP-SGD with momentum.
%
We show how to correct this DP noise induced bias, yielding a variation that we call DP-AdamBC.
%
Empirically, correcting Adam's second moment estimate for DP noise significantly increases test performance for Adam with DP, on tasks for which Adam is well suited.

We make the following contributions:
\begin{enumerate}
    \item We analyze the interaction between DP and the Adam optimizer, and show that DP noise introduces bias in Adam's second moment estimator (\S\ref{sec:motiv-bias}). We show theoretically, and verify empirically, that under typical DP parameters DP-Adam reduces to DP-SGD with momentum (\S\ref{sec:motiv-sgdm}). This behavior violates the sign-descent hypothesis for Adam's performance.
    \item We propose DP-AdamBC, a variation of DP-Adam that corrects for the bias introduced by DP noise. We show that DP-AdamBC is a consistent estimator for the Adam update, under the same simplifying assumptions that justify Adam's update. (\S\ref{sec:method}). %\ml{add something about error bound here if it becomes a big thing in the paper}
    \item We empirically evaluate the effect of DP-AdamBC, and show that it yields significant improvements (up to $3.5$ percentage points of test accuracy) over DP-Adam. (\S\ref{sec:exp}).
\end{enumerate}

\noindent Our implementation is available at: \url{https://github.com/ubc-systopia/DP-AdamBC}. All Appendixes referenced in the paper are available in the long version of the paper \cite{tang2023dpadambc}.

\section{Adam and the Sign-descent Hypothesis}
\label{sec:background}
The Adam update \citep{orig_adam} is defined as follows. Denote the average gradient over a mini-batch of size B with respect to loss function $f$ at step $t$ as:
\[
    g_t = (1/B)\nabla f(\theta_{t-1})
\]
Let $\beta_1$ and $\beta_2$ be Adam's decay coefficients. At each step, Adam updates two estimators:
\[
    m_{t} \leftarrow \beta_{1} m_{t-1} + \left(1-\beta_{1}\right) g_t \ ; \ \ \ \widehat{m}_{t} \leftarrow m_{t} /\left(1-\beta_{1}^{t}\right) ,
\]
\[
    v_{t} \leftarrow \beta_{2} v_{t-1}+\left(1-\beta_{2}\right) g_t^{2} \ ; \ \ \ \widehat{v}_{t} \leftarrow v_{t} /\left(1-\beta_{2}^{t}\right) .
\]
Finally, the Adam update for the model's parameters is:
\[
    \theta_t \leftarrow \theta_{t-1} - \eta \Delta_t \ ; \ \ \ \Delta_t = \hat{m}_t / (\sqrt{\hat{v}_t} + \gamma) ,
\]
with learning rate $\eta$, and $\gamma>0$ a small numerical stability constant.
Intuitively, Adam's $\hat{m}_t$ and $\hat{v}_t$ use an exponential moving average to estimate $\E[g_t]$ and $\E[g_t^2]$, the vector of first and second moment of each parameter's gradient, respectively. The final update is thus approximating $\E[g_t]/\sqrt{\E[g_t^2]}$.

The reasons for Adam's performance are not fully understood. However, recent evidence \citep{kunstner2023heavytailed} supports the hypothesis that Adam derives its empirical performance from being a smoothed out version of sign descent. At a high level, Adam performs well in settings (e.g., NLP) where sign descent also performs well, at least when running with full (or very large) batch.
We next describe Adam's update rule under this sign descent hypothesis, before working out the impact of DP noise on this interpretation. Let $\E{}$ and $\V{}$ denotes the expectation and variance respectively,
\begin{enumerate}
\item If for parameter $i$, $|\E{[g_t]}|_i \gg \sqrt{\V{[g_t]}_i}$, then the update's direction is clear. And since $|\E{[g_t]}|_i \approx \sqrt{\E{[g^2_t]}_i}$, the Adam update is $\E{[g_t]}_i / \sqrt{\E{[g^2_t]}_i} \approx \pm 1$, and Adam is sign descent. Updates are {\em not scaled based on $|\E{[g_t]}|_i$} as in SGD.
\item If for parameter $i$, $|\E{[g_t]}|_i \not\gg \sqrt{\V{[g_t]}_i}$, the sign is less clear and Adam's update is in $[-1, 1]$, scaled closer to $0$ the more uncertain the sign is (smoothing behavior).
\end{enumerate}
Finally, Adam ensures numerical stability when $|\E{[g_t]}|_i \approx 0$ and $\V{[g_t]}_i \approx 0$ using the additive constant $\gamma$ in the denominator of the update. In that case, the update is approximately $\E{[g_t]}_i/\gamma \approx 0$.

To summarize, under the sign descent hypothesis, Adam  updates parameters with low variance gradients using a constant size $\pm 1$ update (or $\pm \eta$ after the learning rate is applied), and rescales the update of parameters with high variance gradients towards $0$. As we describe next, adding DP to gradient computations breaks this interpretation of Adam as sign descent.

\section{Adam Update under Differential Privacy}
\label{sec:motiv}
Most optimization approaches for deep learning models with Differential Privacy follow a common recipe \cite{abadi2016deep}: compute each gradient update over a mini-batch with DP, and leverage DP's post-processing guarantee and composition properties to analyse the whole training procedure.
%
Computing a DP update over a mini-batch involves clipping per-example gradients to control the update's sensitivity, and adding {\em independent} Gaussian noise to the aggregated gradients.
Formally, for each step $t$, let $g_{n} = \nabla f(\theta_t, x_n)$ be the gradient for sample $n$, and let $C$, $\sigma$ be the maximum $L2$-norm clipping value and the noise multiplier, respectively. Given a mini-batch $B$, the DP gradient is:
\begin{gather*}
    \Tilde{g}_t = \overline{g}_t + (1/B) z_t \ ; \ \ \ z_t \sim \cN(0, \sigma^{2}C^{2}\I^{d}) \ ; \\
    \overline{g}_t = (1/B)\sum_{n} g_{n} / {\max{(1, \lVert g_n \rVert_{2}/C})},
\end{gather*}
where $\overline{g}_t$ is the mean of clipped gradients over the minibatch---a biased estimate of $g_t$---and $\Tilde{g}_t$ the DP gradient.

With this recipe, any optimizer that only takes mini-batch updates as input, such as Adam, can be applied to the DP update $\Tilde{g}$ and preserve privacy. This is how existing DP approaches using Adam work (e.g., \cite{li2021}), yielding the following update: let the superscript $p$ denote private version of a quantity, then
\begin{gather*}
    m_{t}^{p} \leftarrow \beta_{1} m_{t-1}^{p} + \left(1-\beta_{1}\right) \Tilde{g}_t, \, \widehat{m}_{t}^{p} \leftarrow m_{t}^{p} /\left(1-\beta_{1}^{t}\right), \\
    v_{t}^{p} \leftarrow \beta_{2} v_{t-1}^{p}+\left(1-\beta_{2}\right) \Tilde{g}_t^{2}, \, \widehat{v}_{t}^{p} \leftarrow v^p_{t} /\left(1-\beta_{2}^{t}\right), \\
    \theta_t \leftarrow \theta_{t-1} - \eta \widehat{m}_{t}^{p} / (\sqrt{\widehat{v}_{t}^{p}} + \gamma).
\end{gather*}

We show next that this DP-Adam algorithm uses a biased estimator for the second moment. This bias dominates the scale of the denominator in Adam's update, thus breaking the sign descent behaviour of Adam (\S\ref{sec:motiv-bias}) and reducing DP-Adam to DP-SGD with momentum and a specific learning rate schedule (\S\ref{sec:motiv-sgdm}).

\subsection{DP noise biases second moment estimates, breaking the sign descent behavior}
\label{sec:motiv-bias}
Under DP, Adam estimates the first and second moments as $m_t^{p}$ and $v_t^{p}$, and rescaled versions $\hat{m}_t^{p}$ and $\hat{v}_t^{p}$, using $\Tilde{g}_t$ in order to preserve privacy. Since the noise added for DP is independent of the gradient update, there is no impact on the first moment in expectation:
\begin{align}
\label{eq:mt}
\begin{split}
    \E{[m_t^{p}]} &= \E{\bigg[ (1-\beta_{1}) \sum_{\tau=1}^{t} \beta_{1}^{t-\tau} \Tilde{g}_\tau \bigg]} \\
    &= (1-\beta_{1}) \sum_{\tau=1}^{t} \beta_{1}^{t-\tau} \bigg( \E{[ \overline{g}_\tau ]} +  \underbrace{\frac{1}{B}\E{[z_\tau]}}_{\text{0}} \bigg) \triangleq \E{[m_t^{c}]}.
\end{split}
\end{align}
However, $v_t^{p}$ is now a biased estimate of the second moment of the mini-batch's update $\bar{g}_t$, as it incurs a constant shift due to DP noise \citep{tang2023dpadam}. By independence of the DP noise $z_t$ and $\bar{g}_t$, we have that:
\begin{align}
\label{eq:vt}
\begin{split}
    \E{[v_t^{p}]} &= \E{\bigg[ (1-\beta_{2}) \sum_{\tau=1}^{t} \beta_{2}^{t-\tau} {\Tilde{g}_\tau}^{2} \bigg]} \\
    &= \underbrace{(1-\beta_{2}) \sum_{\tau=1}^{t} \beta_{2}^{t-\tau} \E{ [ \overline{g}_\tau^{2}]}}_{\triangleq \ \E{[v_t^{c}]}} + (1-\beta_{2}^{t})\underbrace{\bigg(\frac{\sigma C}{B} \bigg)^{2}}_{\Phi} .
\end{split}
\end{align}
In these equations, $\E{[m_t^{c}]}$ and $\E{[v_t^{c}]}$ are the quantities that would be estimated under regular Adam (without DP noise), computed with respect to $\bar{g}_t$ (clipped gradients for DP).

\begin{figure}[htb]
\centering
\includegraphics[width=0.90\linewidth]{figs/type1_mt_vt_hist_motiv_10001_hat.pdf}
\caption{Histogram of \textbf{Left:} un-noised ($\hat{m}_t^c$) and private ($\hat{m}_t^p$) first moment estimates, \textbf{Right:} un-noised ($\hat{v}_t^c$) and private ($\hat{v}_t^p$) second moment estimates near end of training at $t=10000$, using the SNLI dataset with $B=256, C=0.1, \sigma=0.4, \beta_2=0.999$, $\Phi \approx \textnormal{2.441e-8}$ for large $t$.}
\label{fig:motiv_10001}
\end{figure}

We use a text classification dataset (SNLI) to demonstrate the effect of DP noise on first and second moment estimates, with $B=256, C=0.1, \sigma=0.4, \beta_2=0.999$, $\Phi = \textnormal{2.441e-8}$ with large $t$.

Figure \ref{fig:motiv_10001} (Left) shows the histogram of values of the first moment estimates $\hat{m}_t^c$ (clipped gradients, no noise) for each dimension, and private $\hat{m}_t^p$ (clipped and noised gradients), at the end of training. We observe that the center of the distributions align, suggesting that $\E{[\hat{m}_t^p]} = \E{[\hat{m}_t^c]}$ as in Equation \ref{eq:mt}. The private first moment distribution has larger variance compared to the clean distribution as a result of DP noise. Figure \ref{fig:motiv_10001} (Right) shows the histogram of $\hat{v}_t^c$ (clipped, no noise) and private $\hat{v}_t^p$ (clipped and noised) second moment estimates at the end of training. We see that the distributions of $\hat{v}_t^c$ and $\hat{v}_t^p$ are quite different, with a shift in the center approximately equal to $\sqrt{\Phi}$. This suggests that the DP noise variance dominates the scale of $\hat{v}_t^p$ in Equation \ref{eq:vt}.

To understand the implication of DP noise bias $\Phi$, let us follow the original Adam paper \cite{orig_adam} and interpret the update under the following assumption:

\begin{assumption}[Stationarity]\label{assumption:stationarity}
For all $\tau$ in $[0, t]$, the (full) gradient is constant, $\nabla f(\theta_\tau) \triangleq \nabla f$, and minibatch gradients are i.i.d samples such that $\E[g_t] = \nabla f$.
\end{assumption}

\begin{remark}
    Note that Assumption \ref{assumption:stationarity} is not required for convergence (see Appendix F), nor is it used in empirical experiments. It is useful though, to reason about the behavior of DP-Adam and compare it to the intended behavior of Adam without DP, as we do next. The same assumption was used in Adam's original work for the same purpose, to reason about the quality of Adam's moment estimates [\cite{orig_adam}, \S3].
\end{remark}

Under Assumption \ref{assumption:stationarity}, with $\beta_1 \rightarrow 1$, $\beta_2$ such that $(1-\beta_1^{t})/{\sqrt{1-\beta_2^{t}}} = 1$, and for large enough $t$, we have that $\E{[\hat{m}_t^{c}]} \approx \E{[\bar{g}_t]}$ and $\E{[\hat{v}_t^{c}]} \approx \E{[\bar{g}_t^{2}]}$, and $\Delta_t =
\E{[\tilde{g}_t]} / \sqrt{\E{[\tilde{g}_t^2]}} = \E{[\bar{g}_t]} / \sqrt{\V{[\bar{g}_t]+\E(\bar{g}_t)^2}+\Phi}$.
Due to the extra DP bias $\Phi$ in the denominator of Adam's estimator, DP-Adam no longer follows the sign descent interpretation seen in \S\ref{sec:background}.


Focusing on the sign descent regime---when a parameter $i$ in the model has a large signal and small variance, such that $|\E{[\bar{g}_t]}|_i \approx \sqrt{\E{[\bar{g}^2_t]}_i}$---the Adam update becomes $\pm (|\E{[\bar{g}_t]}|_i / \sqrt{\E{[\bar{g}^2_t]}_{i} + \Phi})$ instead of $\pm 1$.
%
For example: if $|\E[\bar{g}_t]|_i=\sqrt{0.1\Phi}$, the update will be $\approx \pm 0.1$, whereas it will be $\approx \pm 1$ if $|\E[\bar{g}_t]|_i=\sqrt{10\Phi}$. In each case, without DP noise Adam would result in a $\pm 1$ update.

Importantly, re-scaling the learning rate $\eta$ is not sufficient to correct for this effect. Indeed, consider two parameters of the model indexed by $i$ and $j$ that, at step $t$, both have updates of small variance but different magnitude, say $|\E[\bar{g}_t]|_i=\sqrt{0.1\Phi}$ and $|\E[\bar{g}_t]|_j=\sqrt{10\Phi}$.  Then the Adam update for $i$ will be $\approx \pm 0.1$ and that of $j$ $\approx \pm 1$, and no uniform learning rate change can enforce a behavior close to sign descent for both $i$ and $j$ in this step. Indeed, under typical DP parameters, DP-Adam is closer for DP-SGD with momentum, as we show next.

\subsection{DP-Adam is DP-SGD with momentum}
\label{sec:motiv-sgdm}
As we saw on Figure \ref{fig:motiv_10001}, under typical DP parameters the DP noise bias $\Phi$ dominates $\hat{v}^p_t$. That is, $\Phi \gg \E[\bar{v}^c_t]$, and we have $\Delta_t \approx \hat{m}^p_t / \sqrt{\Phi}$. Intuitively in this setting, the denominator of DP-Adam's update leads to a constant rescaling, instead of a sign descent \citep{kunstner2023heavytailed} or inverse variance conditioning \citep{balles2020dissecting}. Compensating by properly scaling the learning rate yields an update proportional to $m^p_t$, which is the update of DP-SGD with momentum.

More precisely, using the private gradients $\Tilde{g}_t$ in DP-SGD with Momentum (DP-SGDM) yields the following update:
\[
    b_t^{p} \leftarrow \beta b_{t-1}^{p} + \Tilde{g}_t \ ; \ \ \ \theta_t \leftarrow \theta_{t-1} + \eta b_t^{p} ,
\]
where $\beta \in [0, 1]$ is a momentum decay coefficient. Note the slightly different semantics for $\beta$ compared to Adam, as we follow the typical formulation of DP-SGDM.
%
We thus have $b_t^{p} = \sum_{\tau \leq t} \beta^{t-\tau} \Tilde{g}_t$ and $\hat{m}_t^{p} = \frac{1-\beta_1}{1-\beta_1^t} \sum_{\tau \leq t} \beta_1^{t-\tau} \Tilde{g}_t$.
%
Setting $\beta^{\textnormal{DP-SGDM}} = \beta_1^{\textnormal{DP-Adam}}$, and using the same updates $\Tilde{g}_t$ leads to $b_t^{p} = \frac{1-\beta}{1-\beta^t} \hat{m}_t^{p}$. In the DP regime where $\Phi \gg \E[\bar{v}^c_t]$, and thus $\hat{v}_t^{p} \approx \Phi$, the DP-Adam update is $\Delta_t \approx \hat{m}_t^{p} / \sqrt{\Phi} = \frac{1-\beta}{(1-\beta^t)\sqrt{\Phi}} b_t^{p}$. Hence, DP-Adam is DP-SGDM with the following learning rate schedule:
\begin{equation}
\label{eq:lr_convert}
    \eta^{\textnormal{DP-SGDM}} = \eta^{\textnormal{DP-Adam}} \bigg( \frac{1-\beta}{(1-\beta^t)\sqrt{\Phi}} \bigg).
\end{equation}

\begin{figure*}[t]
\centering
\includegraphics[width=0.7\textwidth]{figs/type1_motiv_sgdm.pdf}
\caption{DP-Adam behaves similarly to DP-SGDM with a specific learning rate (lr) schedule. \textbf{Left: }The implied lr schedule of DP-SGDM. \textbf{Middle: } DP-Adam and DP-SGDM with the specific lr schedule has similar training performance with a mean squared difference of 0.015 in training loss. \textbf{Right: }Performance degrades when adding a larger constant bias to un-noised (clipping-only) DP-Adam as it transitions to behave more like DP-SGDM.}
\label{fig:motiv_sgdm}
\end{figure*}

Figure \ref{fig:motiv_sgdm} empirically confirms this analysis in typical DP regimes.
Figure \ref{fig:motiv_sgdm} (Left) shows the learning rate schedule over step $t$ for the $\Phi$ values of two DP settings (`small' $\Phi$ with $B=256, C=0.1, \sigma=0.4$, `large' $\Phi$ with $B=256, C=1.0, \sigma=1.0$) when $\beta=0.9$ and $\eta^{\textnormal{DP-Adam}}=0.001$. We see that DP-Adam emulates DP-SGDM with an exponentially decreasing learning rate schedule, with an asymptotic value that depends on $\Phi$ ($\approx0.645$ for $\Phi\approx$2.4e-8, $\approx0.026$ for $\Phi\approx$1.5e-5).

Figure \ref{fig:motiv_sgdm} (Middle) shows the training loss over steps for DP-Adam ($B=256, C=0.1, \sigma=0.4, \eta=0.001$) and DP-SGDM (same $B, C, \sigma$, $\eta$ follows Eq. \ref{eq:lr_convert}, converging to $\approx 6.4$), on the SNLI dataset. We observe that the two algorithms have almost identical training performance:  their respective loss over steps closely aligns, with a mean squared difference of $\approx0.015$ over the entire training.

Figure \ref{fig:motiv_sgdm} (Right) shows the effect of adding a constant bias ($\Phi'$) to Adam's update denominator, without noise, on the SNLI dataset. That is, we update parameters with $\Delta_t = m_t^c/\sqrt{v_t^c + \Phi'}$, where $\Phi'=0$ implies un-noised DP-Adam (gradients are clipped, but no noise is added).
We tune $\eta$ for test accuracy at the end of training.
This experiment thus isolates the effect of second moment bias from DP noise.
We observe that on this text classification task, on which Adam performs better than SGD without DP, the performance of DP-Adam degrades as it transitions to DP-SGDM (more bias is added to the denominator). We conclude that DP-Adam's performance likely degrades due to the DP bias $\Phi$. Appendix D shows more performance comparisons between DP-SGDM and DP-Adam.

Prior work made similar observations on the effect of DP noise on DP-Adam's second moment estimator \cite{mohapatra2021role}. Their approach is to remove second moment scaling, which as we showed produces DP-SGDM. Instead, we show how to correct DP noise bias, yielding the DP-AdamBC variant that follows Adam's behavior without DP, despite the addition of noise.

\section{DP-Adam, Bias Corrected (DP-AdamBC)}
\label{sec:method}
Since we can compute the bias in $v_t^{p}$ due to DP noise (see Eq. (\ref{eq:vt})), we propose to correct for this bias by changing the Adam update $\Delta_t$ as follows:
\begin{equation}
\label{eq:adam_corr}
    \Delta_t = \frac{\hat{m}_t}{\sqrt{\max{\big(\hat{v}_t - \Phi}, \gamma' \big)}} = \frac{\hat{m}_t}{\sqrt{\max{\big(\hat{v}_t - (\sigma C / B)^{2}}, \gamma' \big)}}.
\end{equation}

\begin{algorithm}[h]
\DontPrintSemicolon
\SetAlgoLined
% \footnotesize
\scriptsize
\KwOut{Model parameters $\theta$}
\KwIn{Data $D=\{x_i\}_{i=1}^N$, $\eta$, $\sigma$, $B$, $C$, $\beta_1, \beta_2$, $\gamma'$, $\epsilon$-DP, $\delta$-DP; initialize $\theta_0$ randomly; $m_0 = 0, v_0 = 0$; total number of steps $T = f(\epsilon\textnormal{-DP}, \delta\textnormal{-DP}, B, N, \sigma)$}
 \For{$t = 1 \ldots, T$} {
    Take a random batch with sampling probability $B/N$ \;
    $g_i = \nabla \gL(\theta_{t-1}, x_i)$ \;
    $\Tilde{g}_t = \frac{1}{B} \big( \sum_{i} g_{i} / {\max{\big(1, \frac{\lVert g_i \rVert_{2}}{ C}} \big)} + z_t \big), z_t \sim \cN\big(0, \sigma^{2}C^{2}\I^{d}\big)$ \;
    $m_{t} \leftarrow \beta_{1} \cdot m_{t-1} + \left(1-\beta_{1}\right) \cdot \Tilde{g}_t$,\, $\widehat{m}_{t} \leftarrow m_{t} /\left(1-\beta_{1}^{t}\right)$ \;
    $v_{t} \leftarrow \beta_{2} \cdot v_{t-1}+\left(1-\beta_{2}\right) \cdot \Tilde{g}_t^{2}$,\, $\widehat{v}_{t} \leftarrow v_{t} /\left(1-\beta_{2}^{t}\right)$ \;
    $\theta_{t} \leftarrow \theta_{t-1} - \eta \cdot \hat{m}_t / \sqrt{\max{\big(\hat{v}_t - (\sigma C / B)^{2}}, \gamma' \big)}$
  }
 \caption{DP-AdamBC (with corrected DP bias in second moment estimation)}
 \label{alg:dp_adam}
\end{algorithm}

Algorithm \ref{alg:dp_adam} shows the overall DP-AdamBC optimization procedure, including the moment estimates from Adam. The main differences are the bias correction to the second moment estimate, and a different numerical stability constant, which we come back to later in this section, after discussing several important properties of DP-AdamBC.

\paragraph{Privacy Analysis.} Our bias corrected DP-AdamBC follows the same DP analysis as that of DP-Adam, and that of DP-SGD. Since both $\hat{m}_t$ and $\hat{v}_t$ are computed from the privatized gradient $\Tilde{g_t}$, the post-processing property of DP and composition over training iterations ensure privacy. The correction is based only on public parameters of DP-Adam: $\beta_{2}$, step $t$, batch size $B$, and the DP noise variance $(\sigma C)^{2}$. We prove the following proposition in Appendix E. In experiments (\S\ref{sec:exp}) we use R\'enyi DP for composition, though other techniques would also apply.

\begin{proposition}[Privacy guarantee of DP-AdamBC]
\label{prop:privacy_guarantee}
    Let the optimization algorithm DP-SGD($\theta, X, y, C, \sigma, B$) (Algorithm 1 in \citet{abadi2016deep}),
    % DP-Adam($\theta, X, y, C, \sigma$) (Algorithm 1 in [2]),
    with privacy analysis \textit{Compose}($T$, $\theta_{1,\ldots,T}$), be $(\epsilon, \delta)$-DP, then DP-AdamBC($\theta, X, y, C, \sigma, B$) with the same privacy analysis \textit{Compose}($T$, $\theta_{1,\ldots,T}$) is also $(\epsilon, \delta)$-DP.
\end{proposition}


\paragraph{Consistency of DP-AdamBC.}
Remember from \S\ref{sec:background} and \S\ref{sec:motiv-bias} that Adam seeks to approximate $\E[g_t]/\sqrt{\E[g_t^2]}$, and does under Stationarity (Assumption \ref{assumption:stationarity}).
Similarly, under Assumption \ref{assumption:stationarity}, DP-AdamBC is a consistent estimator of $\E[\bar{g}_t]/\sqrt{\E[\bar{g}_t^2]}$ as $\beta_1, \beta_2 \rightarrow 1$, and $t \rightarrow \infty$. Formally, calling $\hat{v}_t^{\textnormal{corr}} = \frac{(1-\beta_2)\sum_{\tau=1}^{t}\beta_2^{t-\tau}\Tilde{g}_t^2}{1-\beta_2^t} - \big( \frac{\sigma C}{B} \big)^2$, we have the following result, proven in Appendix B:
\begin{proposition}\label{proposition:consistent}
Under Assumption \ref{assumption:stationarity}, the DP-AdamBC update (without numerical stability constant) $\frac{\hat{m}_t^p}{\sqrt{\max(\hat{v}_t^{\textnormal{corr}}, 0)}}$ is a consistent estimator of $\frac{\E[\bar{g}_t]}{\sqrt{\E[\bar{g}_t^2]}}$ as $\beta_1, \beta_2 \rightarrow 1$, and $t \rightarrow \infty$.
\end{proposition}
Intuitively, under the stationarity assumption, DP-AdamBC estimates the Adam target update in the limit of averaging over a large number of steps. In practice, $\beta_1$ and $\beta_2$ trade-off the freshness of gradients used in the running estimates with the effect of averaging out DP noise. The DP-Adam update is not a consistent estimate of $\E[\bar{g}_t]/\sqrt{\E[\bar{g}_t^2]}$, but converges to $\E[\bar{g}_t]/\sqrt{\E[\bar{g}_t^2] + \Phi}$. Making $\Phi$ smaller would require increasing $B$ or decreasing $\sigma C$, resulting in a higher privacy cost per optimization step.

\paragraph{DP-AdamBC and sign-descent.} Thanks to its consistency property, the DP-AdamBC update on Equation \ref{eq:adam_corr} re-enables the sign descent interpretation for DP-Adam which closely tracks that of Adam. Ignoring the stochasticity introduced by measurements with DP noise for now:
\begin{enumerate}
    \item If for parameter $i$, $|\E{[\Bar{g}_t]}|_i \gg \sqrt{\V{[\Bar{g}_t]}_i + \Phi}$ , then $|\E{[\Bar{g}_t]}|_i \approx \sqrt{\E{[\Bar{g}^2_t]}_i}$, and $\Delta_t \approx \pm 1$. The update would be similar even without of our bias correction.
    \item If for parameter $i$, $|\E{[\Bar{g}_t]}|_i \gg \sqrt{\V{[\Bar{g}_t]}_i}$ but $|\E{[\Bar{g}_t]}|_i \ll \Phi$, then correcting for $\Phi$ ensures that $|\E{[\Bar{g}_t]}|_i \approx \sqrt{\E{[\Bar{g}^2_t]}_i}$, and $\Delta_t \approx \pm 1$, the expected behavior under Adam and the sign descent hypothesis. Without the correction, the update would be scaled as $\E{[\bar{g}_t]}/\Phi$ instead, and proportional to the gradient size, which is not the Adam or sign descent behavior.
    \item If for parameter $i$, $|\E{[\Bar{g}_t]}|_i \not\gg \sqrt{\V{[\Bar{g}_t]}_i + \Phi}$ (large gradient variance), $\Delta_t \in [-1, 1]$, performing a smooth (variance scaled) version of sign descent (not correcting for $\Phi$ would make the update closer to $0$, especially if $\Phi$ is large compared to $\V{[\Bar{g}_t]}_i$).
\end{enumerate}
In practice we cannot ignore the effect of DP noise of course. The first moment estimate $m_t^p$ is unbiased and adds variance to the optimization. We discuss the impact of stochastic measurements on the second moment next, while \S\ref{eval:empirical-effect-correction} details the empirical effects of our correction.

\paragraph{The numerical stability constant.}
The exponential moving average over DP quantities introduces measurement errors due to DP noise. It is thus possible that $\hat{v}_{i,t} - \Phi < \V{[\bar{g}_t]}_i$, and even that $\hat{v}_{i,t} - \Phi <0$. Our stability correction, $\textrm{max}( . , \gamma')$, deals with these cases similarly to Adam's $\gamma$. We expect that $\sqrt{\gamma'} \gg \gamma$ since the DP noise is typically larger than the gradients' variance.
%
To quantify this effect, we first analyze the error introduced by DP noise to $\hat{v}_t^{\textnormal{corr}}$ when considering a fixed sequence of clipped gradients. That is, the sequence of parameters $\theta_t$ and mini-batches is fixed. This measures the deviation of $\hat{v}_t^{\textnormal{corr}}$ from $\hat{v}_t^c$ due to DP noise, a measurement error from the quantity we are trying to estimate on a fixed sequence of parameters. In this case:

\begin{proposition}
\label{prop:bound_fixed}
Consider a fixed-in-advance sequence of model parameters $\theta_t$ and mini-batches.
For $0 < \alpha < 1$, for each dimension $i$, we have $\sP[\lvert
 \hat{v}_t^{\textnormal{corr}} - \hat{v}_t^c \rvert_i \geq \xi ] \leq \alpha$ with:
\begin{align*}
    \xi \geq
    \begin{cases}
        (\frac{1-\beta_2}{1-\beta_2^t}) \sqrt{\ln{(1/\frac{\alpha}{2})}(2v^2)}
        & 0 \leq \frac{\xi(1-\beta_2^t)}{1-\beta_2}  \leq \frac{\nu^2}{b} \\
        (\frac{1-\beta_2}{1-\beta_2^t}) \ln{(1/\frac{\alpha}{2})}2b
        & \frac{\xi(1-\beta_2^t)}{1-\beta_2}    \geq \frac{\nu^2}{b},
    \end{cases}
\end{align*}
where $\nu = (\frac{4\sigma^2C^2}{B^2})\sqrt{\frac{1-\beta_2^{2t}}{1-\beta_2^2}}, b=\frac{4\sigma^2C^2}{B^2}$.
\end{proposition}
The proof is in Appendix C.
For our SNLI example, this yields a bound of $5.933$e-09 at probability 0.05 at $t=10000$. We show in Appendix C, using empirical measurements, that this bound is accurate.
In practice, the values of $\hat{v}_t^{\textnormal{corr}}$ error are concentrated around their mean $\hat{v}_t^c$, with $\hat{v}_t^{\textnormal{corr}} - \hat{v}_t^c$ smaller than large values of $\hat{v}_t^c$, making bias correction practical.

While it can still happen that $|\Delta_{i, t}| \geq 1$, we show in \S\ref{sec:exp} that debiasing the second moment to follow the sign descent interpretation yields an improvement in model accuracy.
Finally, Appendix C also shows a Martingale analysis that does not assume a fixed sequence of parameters $\theta_t$, which are treated as random variables dependent on the noise at previous steps.
\begin{proposition}
\label{prop:bound_martingale}
    For $0 < \alpha < 1$, for each dimension $i$, we have $\sP[\lvert \hat{v}_t^p - \E{[\hat{v}_t^p]} \rvert_i \geq \xi] \leq \alpha$ with:
    \begin{align*}
    \xi \geq
    \begin{cases}
        (\frac{1-\beta_2}{1-\beta_2^t}) \sqrt{\ln{(1/\frac{\alpha}{2})}(2v^2)}
        & 0 \leq \frac{\xi(1-\beta_2^t)}{1-\beta_2}  \leq \frac{\nu^2}{b} \\
        (\frac{1-\beta_2}{1-\beta_2^t}) \ln{(1/\frac{\alpha}{2})}2b
        & \frac{\xi(1-\beta_2^t)}{1-\beta_2}  \geq \frac{\nu^2}{b},
    \end{cases}
\end{align*}
where $\nu = 2\sqrt{\frac{1-\beta_2^{2t}}{1-\beta_2^2}}( \frac{\sigma^2C^2}{B^2} + \frac{\sigma C^2}{B} )$, $b = \frac{4\sigma^2C^2}{B^2}$.
\end{proposition}
The error bound to $\E[\hat{v}_t^{\textnormal{corr}}]$ is much larger in this case, and not as useful in practice since we want to scale $\gamma'$ based on the realized trajectory.

\paragraph{Convergence of DP-AdamBC.}
To show Assumption \ref{assumption:stationarity} is not required for convergence, we study DP-AdamBC and DP-Adam under the setting of \citet{défossez2022simple}, adding the bounded gradient assumption from \citet{li2023dp2} to adapt it to the DP setting.
The main difference is we derive a high probability bound using techniques similar to that of Proposition \ref{prop:bound_martingale}. This allows us to deal with tecnically unbounded DP noise sampled from a Normal distribution. Note that both the theoretical convergence result and empirical results do not rely on Assumption \ref{assumption:stationarity}, which is only useful for matching the intuition to that of Adam and sign descent (and informs our algorithm). The detailed convergence rates and proofs, as well as a discussion, are in Appendix F.

\section{Empirical effect of Correcting for DP bias}
\label{sec:exp}

% \subsection{Evaluating Performance DP-AdamBC}
% \label{sec:exp-performance}
We compare the performance of DP-SGD, DP-Adam, and DP-AdamBC on image, text and graph node classification tasks with CIFAR10 \citep{cifar10_data}, SNLI \citep{bowman-etal-2015-large}, QNLI \citep{wang2019glue} and ogbn-arxiv \citep{hu2021open} datasets. We evaluate the training-from-scratch setting: for image classification, we use a 5-layer CNN model and all of the model parameters are initialized randomly; for text classification, only the last encoder and the classifier blocks are initialized randomly and the other layers inherit weights from pre-trained BERT-base model \citep{bert_paper}; for node classification, we train a DP-GCN model \citep{daigavane2022nodelevel} from scratch without per-layer clipping. For each optimizer, we tune the learning rate, as well as $\gamma$ or $\gamma'$, to maximize test accuracy at different values of $\epsilon$ for $\delta=1$e-5: $\epsilon \in \{1, 3, 7\}$ for CIFAR10, SNLI and QNLI, and $\epsilon \in \{3, 6, 12\}$ for ogbn-arxiv.
Appendix A includes the detailed dataset and model information, experiment setups and hyperparameters.

Table \ref{tab:res_multi_eps} shows the performance of different optimizers. DP-AdamBC often outperforms both DP-Adam and DP-SGD on NLP datasets (SNLI and QNLI), generally by 1 percentage point and up to 3.5 percentage points on SNLI for large $\epsilon=7$. DP-AdamBC retains a similar performance to DP-Adam on CIFAR10 while DP-SGD outperforms both, and even has an advantage over both DP-Adam and DP-SGD on obgn-arxiv for smaller $\epsilon$ values (4 percentage point at $\epsilon=3$, and 1.5 at $\epsilon=6$).
%
In Appendix D, we include full training trajectory plots (Figure 8), graphical comparison of optimizers' performances (Figure 6), and further examine the generalizability of our method by comparing to baselines with larger dataset and models (Figure 9).

{\bf Discussion.} Based on the experiment results and Adam's sign descent behaviour \citep{kunstner2023heavytailed}, we hypothesize that DP-AdamBC has a larger advantage on tasks and architectures for which Adam and sign descent outperform SGD in the non-private case. The hypothesis follows from DP-Adam's similarity to DP-SGD-with-Momentum (\S\ref{sec:motiv}), showing that DP-SGD and DP-Adam are closer to SGD-style algorithms, whereas DP-AdamBC is closer to the intended behavior of Adam under DP. Our experiments provide some evidence to support this reasoning: DP-AdamBC outperforms other approaches on tasks where Adam outperforms in the non-private case (WikiText-2 Transformer-XL experiment, Figure 3 \citet{kunstner2023heavytailed}); in the two cases in which DP-SGD or DP-Adam perform similarly to DP-AdamBC (CIFAR10 and obgn-arxiv in Figure 3), SGD is well documented to perform better without privacy (\citet{wilson2018marginal}, Table 1 in \citet{daigavane2022nodelevel}, respectively). Therefore, we would recommend using DP-AdamBC for DP training on tasks and model architectures on which Adam is expected (or has often been documented) to perform better than SGD without privacy. This includes modern NLP tasks with transformer-based models where Adam has been used extensively for its strong empirical performances.

{\bf Comparisons to previous work.} We compare the performance of DP-AdamBC to that of a recent Adam-like adaptive optimizer specially developped for DP, named $\textnormal{DP}^{2}$ \citep{lidp2}. $\textnormal{DP}^{2}$ uses delayed pre-conditioners to better realize the benefits of adaptivity. However, the algorithm was only evaluated on simple models, and we show that it doesn not work on the deep learning models we consider. Figure \ref{fig:compare_dp2_snli_only} (Left) shows the comparison between $\textnormal{DP}^{2}$-
RMSProp, DP-AdamBC and DP-SGD on CIFAR10 on SNLI dataset with Bert-base model. We observe that $\textnormal{DP}^{2}$-RMSProp first follows DP-SGD (since the first steps use this optimizer), and then struggles to converge on deep learning tasks, leading to poor performance. Indeed switching between two optimizers seems to make $\textnormal{DP}^{2}$ unstable: Figure \ref{fig:compare_dp2_snli_only} (Right) shows the performance of $\textnormal{DP}^{2}$ with different $s$ (switching frequency). We observe that during training, $\textnormal{DP}^{2}$'s performance either has large turbulence or drops significantly at switching points between optimizers. DP-AdamBC does not suffer from this issue. More analysis and experiments are in Appendix D.

\begin{table}[htb]
\centering
\resizebox{0.44\textwidth}{!}{%
\begin{tabular}{ccccc}
\hline\hline
 &  & $\epsilon \approx 1$ & $\epsilon \approx 3$ & $\epsilon \approx 7$ \\ \hline
\multirow{3}{*}{SNLI} & DP-SGD & \textbf{48.03 (1.25)} & 45.11 (1.84) & 51.04 (0.52) \\
 & DP-Adam & 44.72 (1.26) & 47.52 (1.75) & 52.63 (1.91) \\
 & DP-AdamBC & 45.17 (1.04) & \textbf{50.08 (1.57)} & \textbf{56.08 (0.99)} \\ \hline
\multirow{3}{*}{QNLI} & DP-SGD & 57.10 (1.59) & 58.85 (1.20) & 58.29 (0.92) \\
 & DP-Adam & 58.00 (2.05) & 60.72 (1.12) & 61.23 (1.30) \\
 & DP-AdamBC & \textbf{58.32 (1.90)} & \textbf{61.42 (0.99)} & \textbf{62.83 (1.60)} \\ \hline
\multirow{3}{*}{CIFAR10} & DP-SGD & \textbf{52.37 (0.50)} & \textbf{57.30 (0.76)} & \textbf{65.30 (0.33)} \\
 & DP-Adam & 51.89 (0.69) & 54.08 (0.41) & 62.24 (0.10) \\
 & DP-AdamBC & 49.75 (0.56) & 54.27 (0.23) & 63.43 (0.43) \\ \hline
 &  & $\epsilon \approx 3$ & $\epsilon \approx 6$ & $\epsilon \approx 12$ \\ \hline
\multirow{3}{*}{obgn-arxiv} & DP-SGD & 45.35 (1.38) & 49.12 (1.90) & \textbf{54.20 (0.62)} \\
 & DP-Adam & 46.55 (0.54) & 51.98 (0.48) & 54.02 (0.18) \\
 & DP-AdamBC & \textbf{50.51 (0.56)} & \textbf{53.40 (0.28)} & 53.81 (0.34) \\ \hline\hline
\end{tabular}%
}
\caption{Accuracy under different optimizers, for several privacy budgets. Hyper-parameters are tuned for each target $\epsilon$ and optimizer. Mean (standard deviation) over 5 runs for the best hyper-parameters.}
\label{tab:res_multi_eps}
% %removedVspace
\end{table}

\begin{figure}[t]
\centering
\includegraphics[width=0.85\linewidth]{figs/type1_dp2_snli_only_modified_xaxis.pdf}
\caption{\textbf{Left:} Comparison between DP2RMSProp, DP-AdamBC, DP-Adam and DP-SGD and \textbf{Right:} the performance of DP2RMSProp with different phase switching frequency $s$ on SNLI with Bert-base.}
\label{fig:compare_dp2_snli_only}
% %removedVspace
\end{figure}

\subsection{Empirical Effect of Bias Correction}
\label{eval:empirical-effect-correction}

\paragraph{First and second moment estimates of un-noised and private gradients.}
We numerically compare the scale of the first and second moment estimates based on un-noised and private gradients, $\hat{m}_t^c, \hat{m}_t^p, \hat{v}_t^c, \hat{v}_t^p$ respectively, at different training step $t$. The corresponding un-noised, noised and corrected updates are $\Delta_t^{c} = \frac{\hat{m}_t^p}{\sqrt{\hat{v}_t^c}}, \Delta_t^{p} = \frac{\hat{m}_t^p}{\sqrt{\hat{v}_t^p}}$ and $\Delta_t^{\textnormal{corr}} = \frac{\hat{m}_t^p}{\sqrt{\hat{v}_t^\textnormal{corr}}}$.
Table \ref{tab:exp2} shows the summary statistics of these variables near end of training, computed with the SNLI dataset with $B=256, C=0.1, \sigma=0.4, \Phi \approx 2.441$e-8 in the limit of $t$.
We observe that the difference between $\hat{m}_t^{c}$ and $\hat{m}_t^{p}$ is much smaller than that of $\hat{v}_t^{c}$ and $\hat{v}_t^{p}$, especially in the mean values (the empirical measures of the expectation). In particular, the mean of $\hat{v}_t^{p}$ is approximately $\Phi$, which suggests that the DP bias $\Phi$ dominates over the un-noised estimates of second moment $\hat{v}_t^{c}$.
We also observe that the scale of $\hat{v}_t^{p}$ is generally close to $\Phi$, which suggests the private estimate of the second moments are largely affected by the DP noise.
The scale of the corrected second moment estimates, $\hat{v}_t^{\textnormal{corr}} = \max{(\hat{v}_t - \Phi , \gamma')}$ is closer to the scale of $\hat{v}_t^{c}$, with the numerical stability constant ($\gamma'=3$e-10) preventing tiny denominator values.
If no correction is imposed, $\Phi$ dominates in $\E{[\tilde{g}_t]}$ making the update smaller. The tuned learning rate is larger to compensate, but the update $\Delta_t$ is still proportional to the first moment $\E[\bar{g}_t]$. This is not compatible with the behavior of sign descent (\S\ref{sec:method}).
\begin{table}[t]
\centering
\renewcommand{\arraystretch}{1.1}
\resizebox{0.45\textwidth}{!}{%
\begin{tabular}{c|cccccc}
\hline
 & \textbf{Min} & \textbf{Q1} & \textbf{Median} & \textbf{Q3} & \textbf{Max} & \textbf{Mean} \\ \hline
$m_t^c$ & -7.505e-05 & -7.051e-08 & -2.170e-18 & 7.056e-08 & 7.516e-05 & 4.194e-10 \\
$\hat{m}_t^p$ & -1.879e-04 & -2.428e-05 & 1.204e-08 & 2.427e-05 & 1.833e-04 & 6.120e-09 \\ \hline
$\hat{v}_t^{c}$ & 4.119e-24 & 8.297e-14 & 4.090e-13 & 9.819e-13 & 2.729e-08 & 4.032e-12 \\
$\hat{v}_t^{p}$ & 2.068e-08 & 2.408e-08 & 2.460e-08 & 2.513e-08 & 5.524e-08 & 2.461e-08 \\
$\hat{v}_t^{\textnormal{corr}}$ & 3.000e-10 & 3.000e-10 & 3.000e-10 & 7.137e-10 & 3.082e-08 & 5.633e-10 \\ \hline
$\Delta_t^{c}$ & -2.732e+07 & -3.933e+01 & 1.507e-02 & 3.938e+01 & 1.938e+07 & -2.663e+01 \\
$\Delta_t^{p}$ & -1.218 & -1.548e-01 & 7.680e-05 & 1.548e-01 & 1.159 & 3.921e-05 \\
$\Delta_t^{\textnormal{corr}}$ & -1.085e+01 & -1.116 & 5.473e-04 & 1.116 & 1.026e+01 & 3.650e-04 \\ \hline
\end{tabular}%
}
\caption{Moment estimates with un-noised and noised gradient, w/ and w/o bias correction, at step $t=10000$.}
\label{tab:exp2}
% %removedVspace
\end{table}

To further study the effect of DP noise and of our bias correction, we compare the distribution of the private, un-noised, and corrected variables. The same dataset and hyperparameters are used for demonstration.
%
Figure \ref{fig:mt_vt_hist} (Left) and (Middle) shows the histogram of private ($\sqrt{\hat{v}_t^p}$), un-noised ($\sqrt{\hat{v}_t^c}$) and corrected ($\sqrt{\hat{v}_t^{\textnormal{corr}}}$) second moment estimates, when $\gamma'=3e$-12 and $3e$-10 respectively.
We see that the distributions of $\hat{v}_t^c$ and $\hat{v}_t^p$ are quite different, with a shift in the center approximately equal to $\sqrt{\Phi}$. This suggests that the DP noise variance dominates the scale of $v_t^p$ in Equation \ref{eq:vt}. The corrected second moment estimates are much closer in scale to the clean estimates, with the gap near 0 due to the effect of the numerical stability constant $\gamma'$.
Figure \ref{fig:mt_vt_hist} (Right) shows the distribution of the noised ($\Delta_t^p$) and corrected ($\Delta_t^{\textnormal{corr}}$) Adam updates with respect to the noised first moment $\hat{m}_t^{p}$, rescaled to $[-1, 1]$. We observe that the private distribution is heavily concentrated around 0. The bias correction alleviates the concentration around 0 in the distribution, which is consistent with the interpretation in \S \ref{sec:method}.

\begin{figure}[t]
\centering
\includegraphics[width=0.95\linewidth]{figs/type1_hist_corr_new.pdf}
\caption{Histogram of private ($\hat{v}_t^p$), un-noised ($\hat{v}_t^c$) and corrected ($\hat{v}_t^{\textnormal{corr}}$) second moment estimates with \textbf{Left: } $\gamma'=3e$-12, \textbf{Middle: }$\gamma'=3e$-10. \textbf{Right: } private ($\Delta_t^p$) and corrected ($\Delta_t^{\textnormal{corr}}$) Adam updates with respect to $m_t^{p}$.
}
\label{fig:mt_vt_hist}
% %removedVspace
\end{figure}

\paragraph{Correcting second moment with different values.}
We test whether the noise variance $\Phi$ is indeed the correct value to subtract from the noisily estimated $v_t^{p}$, by subtracting other values $\Phi'$ at different scales instead. In Figure \ref{fig:exp3_4} (Upper Left) we compare the performance of correcting $v_t^{p}$ with the true $\Phi$=2.4e-8 versus $\Phi'$. The experiments of DP-Adam($\Phi'$=1e-7) and DP-Adam($\Phi'$=1e-9) are trained using the same DP hyperparameters except changing value of $\Phi$ to $\Phi'$ and with coarsely tuned learning rates. We observe that both values of $\Phi'>\Phi$ or $\Phi'<\Phi$ lead to weaker performance. It suggests that the DP noise bias in the second moment estimate may be responsible for the degraded performance, and correcting for a different value does not provide a good estimate for $\E{[\overline{g}^2_t]}$.

\subsection{Hyperparameter Analysis}

\paragraph{Effect of the numerical stability constant.}
The numerical stability constant $\gamma$ is known to affect the performance of Adam in the non-private setting, and $\gamma$ is often tuned as a hyperparameter \citep{Reddi2019}. Following the same logic, we test the effect of $\gamma'$ and $\gamma$ on the performance of DP-AdamBC and DP-Adam. Figure \ref{fig:exp3_4} (Upper Right) shows that $\gamma'$ indeed impacts the performance of DP-Adam: values of $v_t^{p}$ are small, and changing $\gamma'$ can avoid magnifying a large number of parameters with tiny estimates of $v_t^{c}$.
Figure \ref{fig:exp3_4} (Lower Left) shows the effect of tuning $\gamma$ in DP-Adam. We observe that DP-AdamBC's numerical stability constant does have an impact on performance, but smaller than DP-Adam's equivalent. This is because the large scale of $\Phi$ makes estimates of $v_t^{p}$ relatively large and similar among parameters. We also observe that tuning $\gamma$ with DP-Adam is not a substitute for correcting for DP noise bias $\Phi$, and DP-AdamBC achieves higher accuracy.

\begin{figure}[htb]
\centering
\includegraphics[width=0.9\linewidth]{figs/type1_exp_hyper.pdf}
\caption{Performance when \textbf{Upper Left: }subtracting different (fake) values of $\Phi$,
\textbf{Upper Right: }tuning $\gamma'$ in DP-AdamBC, \textbf{Lower Left: }tuning $\gamma$ in DP-Adam, \textbf{Lower Right: }tuning $\beta$s in DP-Adam. Tuning hyperparameters in DP-Adam cannot replace DP-AdamBC's bias correction.
}
\label{fig:exp3_4}
% %removedVspace
\end{figure}

\paragraph{Effect of the moving average coefficients.}
The $\beta$ coefficients control the effective length of the moving average window in Adam's estimates of the moments. It thus balances the effect of averaging out the noise, versus estimating moments with older gradients. A larger $\beta$ implies averaging over a longer sequence of past gradients, which potentially benefits performance by decreasing the effect of noise. Figure \ref{fig:exp3_4} (Lower Right) shows the effect of choosing different $\beta$ in DP-Adam, with the learning rate $\eta$ coarsely tuned from 1e-4 to 1e-2. As suggested in \citet{orig_adam}, we set $\beta_1$ and choose $\beta_2$ such that $(1-\beta_1) = \sqrt{1-\beta_2}$. We observe that setting $\beta$s too large or too small is worse than choosing the default values ($\beta_1=0.9, \beta_2=0.99$). Setting $\beta$ smaller shows a clear disadvantage as the performance is both worse and more volatile due to less smoothing over noise. Setting a larger $\beta$ results in similar performance at the end of training. However, lowering the effect of noise this way does not yield similar improvements as correcting for DP noise bias in the second moments.

\clearpage\clearpage

\section*{Acknowledgment}
We are grateful for the support of the Natural Sciences and Engineering Research Council of Canada (NSERC) [reference number RGPIN-2022-04469], as well as a Google Research Scholar award. This research was enabled by computational support provided by the Digital Research Alliance of Canada (alliancecan.ca), and by the University of British Columbia's Advanced Research Computing (UBC ARC).


Voluptate totam ratione error reiciendis aut quos itaque alias nisi inventore voluptatum, delectus consequuntur beatae aliquid itaque nesciunt laboriosam, molestias inventore ipsa quas nisi velit molestiae dolore.Obcaecati dolore amet suscipit exercitationem, excepturi dolores corporis ut ipsa quasi officia veritatis neque voluptas dolorem eligendi.Consequuntur delectus exercitationem minus beatae dolorem natus, pariatur necessitatibus odit nulla vitae deleniti, quod explicabo quis quisquam molestias ducimus, necessitatibus itaque magnam sapiente?Fugiat iure laborum nesciunt, nemo officiis nam nostrum incidunt odio, ducimus nostrum maxime?A natus quia expedita nesciunt harum facere, voluptate pariatur mollitia provident obcaecati rerum magni neque necessitatibus, dolorem voluptas molestiae accusamus ullam et quis commodi similique?Ducimus soluta consequuntur quidem fugiat praesentium numquam voluptatum, reiciendis odit facilis, minus eos vel iusto distinctio deserunt unde nostrum necessitatibus sed, numquam in dignissimos.Error asperiores assumenda sit aperiam impedit molestias maiores excepturi, sit rem quasi nobis, dolore vero necessitatibus molestiae voluptatem earum incidunt excepturi facere maiores sunt veniam, illum mollitia culpa, eius iste accusamus at?Labore cum nam non expedita, inventore repudiandae alias commodi nesciunt aliquid amet, dicta ea corporis omnis corrupti praesentium odio tenetur ab, omnis recusandae similique error nihil reprehenderit, a delectus veritatis similique quam sapiente aliquid?In officia asperiores, possimus reprehenderit in veritatis, sit dolorem quo accusantium?Perspiciatis adipisci atque fuga natus ratione harum maiores ut commodi iste, eaque et incidunt velit?Officiis quos ab nobis iure quia aperiam, suscipit laboriosam recusandae provident in ea natus?Quae error sequi dolore perferendis officiis consequuntur, excepturi qui nesciunt molestiae impedit, amet nihil quis pariatur unde, in ad libero quas rem odio odit at necessitatibus, quia atque exercitationem architecto soluta obcaecati repellendus consequatur tempore minima?Sit adipisci consequatur quisquam esse eius, voluptate quibusdam dolor libero autem quis laudantium, nostrum sint vitae dignissimos, consequuntur perspiciatis tenetur in eos praesentium quisquam voluptas velit magni dignissimos placeat?Ducimus deleniti in debitis, mollitia saepe quam ducimus alias ipsam aliquam molestias.Praesentium tenetur maiores, saepe suscipit hic eligendi libero alias nostrum provident excepturi fuga, asperiores unde soluta porro ea similique sequi debitis quasi beatae, reiciendis ab aliquam natus sapiente molestias vel reprehenderit quam laudantium?Nihil labore deserunt officiis necessitatibus corporis reiciendis odio ipsa aut, suscipit eaque praesentium quia harum non aspernatur sunt architecto tempore ab, mollitia nostrum numquam?Distinctio amet quidem officiis illo optio consequatur perferendis aliquid necessitatibus consectetur temporibus, dolorem distinctio officiis ipsam autem quia dolore deserunt possimus recusandae nemo, vel animi eum nobis vero tempore explicabo fugiat, corrupti blanditiis sunt hic assumenda iste expedita quaerat, fuga corrupti minima eveniet veniam incidunt ut nobis autem in quasi totam?Qui cum adipisci hic, magni libero consectetur corrupti, accusantium esse animi libero dolorem possimus ullam recusandae omnis ea.Consectetur aspernatur neque obcaecati ut doloribus unde totam consequuntur iure, recusandae expedita facere.A laudantium est temporibus aspernatur perferendis eius fugit aliquid fugiat, cum reprehenderit ullam, sed minus reprehenderit nisi inventore qui voluptatem?Quas consectetur molestias sapiente at debitis, recusandae enim quos accusamus hic id libero tenetur possimus ipsa facere voluptates.Cumque nobis nisi tenetur iure praesentium veniam odio cupiditate deleniti impedit fugiat, enim ducimus eaque, est dolorem fuga cumque natus.Fugiat dicta dignissimos exercitationem libero earum inventore consectetur eius ipsum sapiente ducimus, maiores a nam voluptatem saepe maxime aspernatur perspiciatis voluptates, eos consequuntur modi non neque odit voluptatum minus maxime a amet, est officia laboriosam exercitationem quibusdam illum fuga?Delectus ad praesentium a blanditiis accusantium iusto natus, repellat nam facere excepturi architecto, aliquid qui tempora officiis?Sed ipsam dolorem vel corporis, libero excepturi commodi esse rem rerum repellat officia expedita modi?Sed debitis fugiat hic doloribus fugit, repellendus ipsam similique debitis minus delectus deleniti dolorem mollitia consequatur, tempora debitis laboriosam possimus ratione cupiditate natus vero mollitia hic ea adipisci, quod quam deserunt est consectetur asperiores at, at debitis itaque maxime quas.Porro optio odio deleniti enim excepturi dicta fugiat corporis repellat cupiditate recusandae, non dolor quaerat ea inventore praesentium tempora veniam suscipit laudantium, dolores officia distinctio nostrum ducimus a quae nemo aliquam quisquam, illo deleniti repudiandae aliquid iste corrupti nam sit repellendus velit unde, magnam accusamus nemo adipisci exercitationem odio ipsa?Accusantium maxime quo doloremque nisi saepe nobis, rerum omnis sequi eum impedit tempore tempora aperiam ut harum odio, ut nobis repellendus illo magni perferendis fugit explicabo ullam doloribus optio?Adipisci assumenda dignissimos nesciunt nobis ullam quis rem quas ipsam laboriosam laudantium, modi excepturi aperiam optio sed provident assumenda sapiente nihil.Temporibus quisquam rerum itaque ullam pariatur vel culpa iusto, corporis sed voluptate quisquam pariatur ullam, dolores tenetur nesciunt at corrupti quasi sunt dolorem.Debitis molestias voluptatibus eaque voluptas est repudiandae amet enim dolores sit eum, sit pariatur accusamus tempora nam aut, sit velit libero quam quibusdam distinctio consequatur laborum tenetur voluptas.Pariatur ratione optio porro iste illum esse debitis minima aspernatur in, mollitia dolorem iusto numquam culpa deserunt dicta ducimus, tenetur sequi repellat corrupti nesciunt accusamus sit, quam sunt qui id, error nulla quam facilis delectus?Debitis sint nobis repudiandae asperiores corporis, provident modi odio possimus sit commodi eveniet minima animi dicta ad, ab voluptatum optio omnis recusandae minus quas alias delectus quasi, quidem deserunt at delectus quis obcaecati ipsum impedit dolores nisi, doloremque optio maxime laboriosam voluptates ipsa assumenda?Itaque dolore quae fugiat delectus in maxime quidem fugit sunt dolor, suscipit natus quos voluptatum reiciendis voluptates a id aliquid perspiciatis laudantium, deserunt numquam sunt dolor impedit eveniet architecto sed, sed deserunt enim labore tempore aliquam praesentium eum libero magnam sit consequatur?Laborum a magni aut nostrum ipsa debitis, minus nemo doloribus veniam, et eveniet sequi praesentium blanditiis asperiores enim, sequi optio quaerat ipsam, incidunt in totam aspernatur iure sunt adipisci autem?Aliquid exercitationem dicta, minus modi labore dolorem quidem quibusdam et maxime, nostrum enim ut unde, facere at architecto?Vel quas alias, fugit consequuntur sit quia nulla eaque sunt inventore blanditiis asperiores distinctio, expedita iusto mollitia ea quibusdam aperiam obcaecati inventore libero vitae?Praesentium est sapiente dolorem velit, unde iste fugiat vitae porro beatae repellendus dignissimos assumenda corrupti placeat adipisci, laboriosam placeat nemo a nesciunt facilis velit accusantium.Sed cumque magni, amet fugiat deleniti est repellat sunt saepe fugit, accusantium facilis repudiandae deleniti eius dolore fugiat natus officia, voluptatibus itaque assumenda voluptate quia eligendi ex tenetur fugit earum eos labore, aliquid officia laudantium.Asperiores ducimus quae adipisci quibusdam sint consequatur rem aspernatur expedita, dolor corrupti architecto provident earum veritatis numquam, culpa quos debitis deleniti odio placeat autem perferendis cupiditate iure?Quaerat illo cum unde, nobis commodi nam laborum in illo ea praesentium, distinctio recusandae vero perspiciatis pariatur excepturi quasi quae perferendis nihil velit, facilis aliquam ex repudiandae deserunt totam.At maxime ut ratione, placeat quam hic eum, nesciunt cum quis vel deleniti perferendis natus, facere veritatis id, nihil nostrum praesentium labore illum voluptates debitis?Delectus maxime enim qui nulla rerum ipsum, exercitationem voluptates esse expedita?Alias voluptatibus eum ipsa architecto labore, illo iure cupiditate accusantium ab eligendi autem rerum eos alias sint.Rem enim autem praesentium aliquam dolores, atque suscipit laboriosam quaerat fugit fugiat distinctio voluptatem illum, asperiores commodi dicta optio expedita nihil ut cumque odit repellendus?Officia optio facere earum, quam deleniti soluta ipsa corporis assumenda voluptatem odit deserunt numquam?Quo ipsam provident repellendus doloremque est laboriosam, quod adipisci voluptatibus modi accusamus placeat sapiente ullam quas suscipit?Quaerat et at ipsam consequuntur odit hic, quod aliquam nostrum.Enim dicta corrupti deserunt soluta sed ratione, eum veniam iure obcaecati consequuntur eligendi consequatur, sapiente doloremque quibusdam minus error, corporis a quo iusto enim explicabo iste ab tempore cumque sit, eaque quisquam eos voluptatibus quasi excepturi?Explicabo sit saepe laboriosam, eaque ipsam totam quasi perferendis quas adipisci optio odio, reiciendis illo libero nam expedita quasi neque odit consequuntur.Magnam enim corrupti iusto facere fuga a, accusamus sed modi incidunt praesentium accusantium, expedita ducimus tenetur facere cum minus obcaecati minima, rerum pariatur dolore sapiente magnam cupiditate nam distinctio maiores dolor necessitatibus, itaque ducimus sed voluptate facilis veritatis quaerat perspiciatis facere ipsa eius labore?Adipisci officia doloremque, ullam vero voluptatum quas placeat, omnis eveniet perferendis dolores similique aperiam eaque?Ducimus assumenda nostrum omnis autem beatae dolorem minus velit eum, quo inventore reprehenderit reiciendis aspernatur eius, eligendi asperiores laudantium alias iste harum numquam, sint pariatur temporibus eveniet?Cupiditate doloremque explicabo nihil ipsam quasi perspiciatis laudantium voluptates laboriosam quia, facere doloribus perspiciatis tempore modi dolorem.Optio delectus enim laborum temporibus similique illum dignissimos id sequi obcaecati officiis, quos quod ipsa quibusdam ea, aut quia non ipsum.Mollitia labore accusamus in maiores aperiam laboriosam modi quis nam voluptatem, odit incidunt error voluptas laudantium, magnam dolor ea autem enim officia in tempore nam, laborum iste officiis illo facere debitis harum?Repellat optio ab perferendis cupiditate debitis natus, voluptatem veniam perspiciatis atque sed corporis tempore hic, nobis ea sit aperiam cumque sequi a esse nemo officia.Recusandae repellat deleniti ab impedit nam voluptatibus quo tempora at cum repudiandae, consectetur voluptatem omnis, perferendis nisi veritatis tempora ex maiores, reprehenderit aliquid consectetur vel?Ab nemo iste esse deleniti provident rerum nulla enim adipisci assumenda, quod non id sit laudantium nesciunt quisquam consequatur?Similique placeat minima obcaecati dolore reiciendis deserunt nostrum, iure consequatur quaerat accusantium, reiciendis deserunt a esse impedit quae doloremque illo eligendi corrupti amet quia, minus rem incidunt odio et nihil quod, atque aut fugit aspernatur?Tenetur ducimus nihil unde esse, recusandae ipsa rerum tenetur assumenda quod minima delectus nisi eos, quas animi vitae.Consequuntur dolorem quia suscipit, labore suscipit fugiat harum illum mollitia architecto esse minima ea assumenda dicta, magni quas animi suscipit delectus fugit, eos expedita unde ipsa autem fugit vero alias culpa amet, quis saepe qui nulla fugiat est quos beatae?Vel incidunt ullam provident beatae perspiciatis eveniet iusto quibusdam quia nisi doloremque, minus explicabo quas voluptatem aut architecto eos necessitatibus nobis magnam exercitationem?Obcaecati asperiores minima aliquid dolorem cum tempore, rerum amet fuga eligendi pariatur molestiae adipisci?Soluta impedit blanditiis vitae ab hic magnam, similique suscipit itaque ducimus quis, itaque odio aliquam optio, ipsum natus provident fuga animi recusandae quidem alias nemo optio qui explicabo, iure ipsum at.Quibusdam odio dolorum eveniet nobis, ullam minima odio, non excepturi labore cumque eos ipsum, dolores consequuntur molestiae quidem laboriosam recusandae aut?Possimus libero sunt odit quos alias doloribus atque, accusantium totam hic optio sequi omnis repellendus voluptatem velit iure nisi, doloremque atque veniam velit ab eaque quas consectetur tempora, tempore veritatis placeat deserunt autem?Quam accusamus esse iste dolorem assumenda consequuntur labore, molestias officiis distinctio corporis officia earum eius labore voluptas ut ipsum rerum, ea sapiente alias incidunt cumque delectus quo velit tempora dicta, dolor quibusdam id quaerat deserunt beatae ipsa ea fugiat?Recusandae maxime possimus veritatis voluptatibus ullam vero ducimus aperiam vitae iure consequuntur, ea vel omnis, ipsum voluptas est.Vero tempora vitae, quas possimus hic expedita deleniti illo est optio molestias, maiores consequuntur cumque reprehenderit praesentium harum reiciendis fugit officiis eos ab, officia dolor repellendus eius recusandae, similique illo porro praesentium beatae.Dolores possimus iusto ratione dignissimos quos, exercitationem a veritatis inventore iusto maxime ratione nihil rem qui voluptatem fugit?Beatae laborum voluptatum porro totam possimus saepe, ullam nulla saepe officia, repellat labore dignissimos autem delectus natus beatae, dolorem sunt numquam reiciendis enim repellendus sed eligendi facilis adipisci.Totam molestiae itaque soluta autem porro fugiat assumenda recusandae, earum similique ullam animi ratione dicta quaerat, repudiandae eveniet asperiores unde perferendis culpa exercitationem illum dolorem magni deleniti cum?Eveniet magnam veniam quam ab quis magni est, nulla eveniet minus est atque sint, laborum nihil molestiae ratione accusamus accusantium maiores vel inventore?Omnis impedit reprehenderit sed neque consequatur eligendi, delectus reiciendis aliquam omnis quia suscipit, distinctio est enim.Necessitatibus aperiam culpa, aperiam error laborum?Aperiam sed veritatis eaque ad cupiditate sapiente nisi unde accusamus consectetur, laudantium aut tenetur perferendis dicta a eius, quae ducimus magni, eveniet natus sit ab error a magnam fugit laborum voluptatibus quaerat reiciendis, animi non tempora harum vero odio officiis?Eum impedit soluta deserunt quibusdam velit vitae qui commodi fuga, amet quos ipsum minus odio, harum sunt quia, voluptatem ullam dignissimos atque sequi quisquam libero veritatis, totam ipsum accusamus doloribus.Provident expedita alias porro commodi aliquam, vel praesentium aliquid tempora quaerat iure minus itaque totam necessitatibus eligendi?Inventore obcaecati minima, rem nulla animi totam repellendus mollitia sint quidem aspernatur, ipsam laboriosam eligendi eius, alias dolorem pariatur incidunt quibusdam molestiae quos, ducimus quam tenetur similique dolorem enim?Ad modi officia iure dolorem fugit animi facere, totam eaque dolorem praesentium repudiandae tempora, consectetur necessitatibus quod laboriosam expedita tempore facilis ab amet, repellat aliquid labore nobis pariatur animi et totam, est mollitia debitis dicta quis amet voluptatem.Eius repudiandae dolor quasi, dolorem beatae neque nulla tempora ducimus doloribus totam eum blanditiis debitis.Molestiae omnis aspernatur necessitatibus vitae quisquam aperiam dicta culpa laboriosam odit, nemo corrupti sapiente odio reprehenderit dolorem aperiam unde quaerat ut nulla voluptatibus, possimus consequuntur nihil cum suscipit corporis quae maiores explicabo, minus dolor sit repellat iste dolorum ducimus, quas ad quod veritatis quis voluptate repellendus sequi placeat?Odio quis in inventore, placeat nobis architecto harum consequuntur accusamus ratione porro odio magnam, nulla quo iste consectetur, sunt quibusdam ullam illo facere eos blanditiis illum.Nesciunt dignissimos minima similique nam, soluta et earum sed deserunt reprehenderit?Iure omnis similique rem distinctio, reprehenderit officiis repellat maiores aperiam veritatis dolore quaerat, eius nostrum dignissimos dolorum voluptates, nobis mollitia qui maxime?Illo ea ut excepturi fugit autem aliquam quibusdam animi voluptas modi magni, alias natus adipisci provident, libero molestias id explicabo, facilis ab repudiandae earum alias adipisci, ducimus dignissimos consequuntur officiis qui voluptatibus beatae non quod quam vero similique.Et error molestias magnam facere temporibus maiores dignissimos, quia obcaecati consectetur vero harum nisi adipisci perferendis, voluptates commodi velit praesentium, aspernatur sequi quas laudantium optio molestias delectus voluptates repudiandae nobis iste dolor, totam quae neque consequuntur eius.Dolores rerum consectetur quibusdam enim quo eius eveniet molestias quae mollitia error, iusto magnam odio doloremque tempore fugit voluptas saepe aspernatur nihil eveniet in, laudantium doloribus placeat aliquam nihil ipsum nobis pariatur nesciunt magni enim ad, aspernatur accusantium soluta minus beatae amet, inventore laboriosam aliquid error tempora voluptas quo earum sint.Molestias minus necessitatibus illo dicta omnis inventore harum aperiam nisi nostrum, vero dolore illo esse ratione aspernatur debitis, esse voluptatem voluptas ex quos quo nostrum, ad vero numquam aperiam fugit, nobis doloremque veritatis nihil.Sit corrupti dicta ratione eum repellat omnis veniam assumenda, et dignissimos iusto, quaerat totam consequatur suscipit tempora quos quibusdam, sint dolorum nihil similique placeat nesciunt error est illo, iure dolores dolore quidem doloremque rem doloribus id.Voluptatem quas incidunt dolores officia commodi maxime, deserunt porro culpa laudantium nam totam quod magnam quam, dicta culpa ipsa quibusdam blanditiis eos totam quisquam.Dolorum doloribus quidem deleniti debitis consequatur ab hic, fuga doloremque quaerat quas voluptate nam natus, quisquam nihil sapiente impedit non laborum velit iure vero asperiores magnam.Laudantium at harum temporibus labore similique debitis vel quibusdam iste, suscipit magnam libero qui fugiat eum, voluptate veniam natus cupiditate consequatur reprehenderit explicabo officiis corrupti, labore ad velit tenetur voluptatem cupiditate dicta saepe accusantium temporibus totam possimus, in cupiditate corrupti tempore nemo voluptas non.Eos dolores ratione, obcaecati beatae aspernatur non dicta nobis minima perferendis cumque eos, maxime voluptates facere fugiat cum officia id quos aspernatur.Quibusdam aliquid quisquam corrupti mollitia deleniti quos, molestias asperiores quia quis id quae labore aut.\clearpage
\bibliography{main}

\clearpage

\appendix

% \pdfpagewidth=8.5in
\pdfpageheight=11in
% The file ijcai21.sty is NOT the same than previous years'
\usepackage{ijcai21}

% Use the postscript times font!
\usepackage{times}
\usepackage{soul}
\usepackage{url}
\usepackage[hidelinks]{hyperref}
\usepackage[utf8]{inputenc}
\usepackage[small]{caption}
\usepackage{graphicx}
\usepackage{helvet}
\usepackage{amsthm}
\usepackage{booktabs}
\usepackage{algorithm}
\usepackage{algorithmic}
\urlstyle{same}
%
% PDF Info Is REQUIRED.
% For /Author, add all authors within the parentheses,
% separated by commas. No accents or commands.
% For /Title, add Title in Mixed Case.
% No accents or commands. Retain the parentheses.
\usepackage{amsmath}
\usepackage{amsthm}
\usepackage{booktabs}
\usepackage{algorithm}
\usepackage{algorithmic}
\usepackage{subcaption}
\usepackage[dvipsnames]{xcolor}
\usepackage{latexsym}
\newcommand{\ra}[1]{\renewcommand{\arraystretch}{#1}}
\newtheorem{theorem}{Theorem}
\newtheorem{proposition}{Proposition}
\newtheorem{conjecture}{Conjecture}
\newtheorem{definition}{Definition}
\newtheorem{lemma}{Lemma}
\def\tuple#1{( #1 )}




\begin{document}
\maketitle

\clearpage
\section*{Appendix}




\subsection*{Supplementary proofs}
In the appendix we use the term \emph{LB- (UB)-unsafe} to refer to a manipulation that lowers the lower (upper) bound of the manipulator. Also, recall that a \emph{solution} refers to any CS that satisfies the objective.

\subsection*{Proof of Theorem~\ref{thrm:egal_dir_add}}
Let 
$
u_0=\underset{P\in O(G)}{\min}(\{u(m,P)\}),
u_1=\underset{P\in O(G)}{\max}(\{u(m,P)\})
$.
We will refer to the CS yielding $u_0$ as $P_0$. Note that for every $P\in O(G)$ it holds that $Eg(P,G) \leq u_0 $
Assume by contradiction that Max-Egal is subject to UB-improvement.
That is, there exists a manipulation $r_{m}$ and a CS $P^m\in O(G^{m})$ such that $u(m,P^m) > u_1$. That is, $P^m\notin O(G)$.

Since the manipulator can only add edges it holds that $Eg(P^{m},G^m)\geq Eg(P_0,G)$. Moreover, if $Eg(P^{m},G^m)=Eg(P_0,G)$ then $P_{0}\in O(G^{m})$, which is not possible. Therefore, $Eg(P^{m},G^m)>Eg(P_0,G)$. Recall that in directed networks the utility of the other agents does not change. Therefore $\forall a \in A \setminus \{m\}$, \[
u(a,P^{m})  \geq Eg(P^{m},G^m) > Eg(P_0,G).
\]
In addition, $u(m,P^{m})>u_1\geq Eg(P_0,G)$. 
Overall, $\forall a \in A, u(a,P^{m}) \geq Eg(P_0,G)$. That is, $Eg(P^{m},G) \geq Eg(P_0,G)$, and thus $P^m\in O(G)$, which is a contradiction.


Now, assume by contradiction that Max-Egal is subject to LB-improvement.
That is, there exists a manipulation $r_{m}$ such that
\begin{equation} \label{ineq:egal_dir}
\forall \  P \in O(G^m), u(m,P) > u_0.
\end{equation} That is $P_0\notin O(G^m)$.
Denote an arbitrary CS in $O(G^m)$ as $P^m$. 
It holds that $Eg(P^m,G^m)>Eg(P_0,G^m)$ and $Eg(P_0,G)\geq E(P^m,G)$.

Again, in directed networks the utility of the other agents does not change. Therefore if after the manipulation $Eg(P^m,G^m)>Eg(P^m,G)$ it can only change by the utility of $m$. But $u(a,P^m)>u(a,P)$, hence even before the manipulation $Eg(P^m,G^m)>Eg(P^m,G)$, in contradiction.



\subsection*{Proof of Theorem~\ref{thm:all_distance1}}

\subsubsection*{Max-Util}
If $0<|N(m)|<n-1$, by looking at possible networks like the ones in Figures \ref{fig:distance1_util_unsafe_add_LB},\ref{fig:distance1_util_unsafe_add_UB} we can see that adding any edge is LB- and UB-unsafe respectively.
These examples can be tweaked to fit any number of agents and any number of desired agents the manipulator has.
If $|N(m)|=n-1$ the manipulator cannot add edges.


\subsubsection*{Max-Egal}
Obviously adding or removing in undirected networks is both LB- and UB-unsafe. Adding an edge towards agent $a_0$ can lead to utility $0$ when the maximum egalitarian SW is $1$, if $m$ ends up in a coalition alone with $a_0$.
Removing an edge can lead to utility $0$. If one $m$'s neighbours has no other agents, removing that edge results in maximum egalitarian SW $0$, hence the lower bound is also $0$. So both add and remove are LB-unsafe.
For the cases of $|N(m)|<n-3, |N(m)|=n-2$ and $|N(m)|=n-1$ see Figures~\ref{fig:egalremove1},\ref{fig:egalremove2} and \ref{fig:egalremove3} for proofs that removing is UB-unsafe, respectively. If $m$ has more neighbours can just add nodes on the top, and more non-neighbours in bottom of the figures.
Figure~\ref{fig:egalremove4} provides proof that adding UB-unsafe. For higher numbers both neighbours and non-neighbours are added at the bottom. Note that this example requires at least neighbours ($|N(m)|\geq2$. However, if $m$ has only one neighbour it is easy to see no manipulation is beneficial against Max-Egal.


\subsubsection*{At-Least-1}
Adding any edge to another agent $a_0$ might result in a LB- of $0$. That is because the CS where $a_0$ and $m$ form one coalition could be a solution now.
Any partial network in distance 1 can have a possible network with a solution where $m$'s original LB- is higher than $0$, hence adding is LB-unsafe.

% In the case with the equal size constraint, the same could be said. However, in this scenario the difference would be that $m$'s LB is not necessarily $0$. If $|N(m)| < n/2$, it could be lower to $0$.
% If $n/2 \leq |N(m)|$ neighbours, there is a possible network where $m$ is guaranteed utility of $n/2 -1 $. Therefore, adding the edge comes at the cost of a true friend, lowering the LB to $n/2-2$.


For At-Least-1, all that is left to prove is that against manipulator $m^-$ the objective is LB-proof. Of course, over undirected networks it is true, as any removal might lead to an infeasible instance.

Over directed networks, it is also easy to see the any removal may lead to an infeasible instance; If $m$ has at least one neighbour $a_0$ for which both $(a_0,m),(m,a_0)\in E $, it could be the case that the only solution is where $m$ and $a_0$ form a coalition alone. (If all of the other agents form a directed circle, and have no other edges but ones going to or from $m$). In that case removing is LB-unsafe.
If $m$ does not have a neighbour like that, the same example still applies when the only solution is $m$ alongside one of his out-neighbours and another agent. i.e. $a_0,a_1$ where $(m,a_0),(a_0,a_1),(a_1,a_0)\in E$.


\clearpage
\subsection*{Equal Size}
An important constraint that arises in many real world scenarios is that all of the coalitions should be of the same size. For example, when dividing students into classes it is common to ask that all of the classes consist of roughly the same number of students.
Therefore we also consider the setting in which all of the coalitions are of the size $n/k$, if $n/k$ is an integer. Otherwise, some coalitions are of size $\left \lceil{n/k}\right \rceil$ and some are of size $\left \lfloor{n/k}\right \rfloor$. We call this restriction the Equal Size constraint, and through the paper we analyze the different objectives with or without it.

Due to space constraints, the definition and results of the equal size constraints have been moved to the appendix. All of the resistance results from Section~\ref{sec:full_info} still stand with the equal size constraint. The susceptibility results are summarized in Table~\ref{tbl:distnce2_equalsize}.




\begin{table}[htp]

\ra{1}
\begin{tabular}{@{}lcccr@{}}\toprule 
$Objective$ & $M$ & $Network$ & $Type$ & $Figure$\\\midrule
Max-Util & A & Both & Strict & \ref{fig:util_size_add}\\
Max-Util & R & Both & Strict & \ref{fig:Util_size_remove}\\
Max-Egal & A & Undirected & LB & \ref{fig:Egal_size_undirected_add_LB}\\
Max-Egal & A & Undirected & UB & \ref{fig:Egal_size_undirected_add_UB}\\
At-Least-1 & A & Undirected & Strict & \ref{fig:Least1_size_undirected_add}\\
At-Least-1 & R & Both & LB & \ref{fig:least1_size_remove}\\
\bottomrule
\end{tabular}
\caption{Summary of susceptibility results for distance 2 with the equal size constraint.
Key: A = add, R = Remove, LB/UB/Strict = the objective is subject to LB/UB/Strict-improvement.
}
\label{tbl:distnce2_equalsize}
\end{table}

\begin{proposition}
\label{thm:util_distance1}
Max-Util with the equal size constraint is subject to 1-safe LB- and UB-improvement and is 1-safe weak-proof for a manipulator $m^-$ over directed and undirected networks. 
\end{proposition}

\begin{proof}
See Figures~\ref{fig:distance1_util/egal_lb} and \ref{fig:distance1_util_ub} for the 1-safe LB-improvement and 1-safe UB-improvement, respectively. We move on to prove it is 1-safe weak-proof.

First, we prove that if the manipulator has less than $n-1$ neighbours, no UB-safe manipulation exists.
If $0<|N(m)|\leq \lfloor n/2 \rfloor$ then Figures~\ref{fig:thrm5_1} and \ref{fig:thrm5_2} provide examples for UB-unsafe scenarios for even and odd $n$, respectively.
If $\lfloor n/2 \rfloor<|N(m)| < n/2$ then Figures~\ref{fig:thrm5_3} and \ref{fig:thrm5_4} provide examples for UB-unsafe scenarios for even and odd $n$, respectively.

If $|N(m)| = n-1$, then if $n$ is even, the manipulator always gets the same utility and no manipulation is possible.
If $n$ is odd we have shown it is subject to LB and UB 1-safe improvement.
If the manipulator removes more than 1 edge, Figure~\ref{fig:thrm5_5} provides an example that this is UB-unsafe.

By removing only 1 edge, the manipulation cannot be weak-improvement. 
Denote the utilitarian SW of coalition structure $P$ in $G$ as $Ut(P,G)$.
Let 
$
u_0=\underset{P\in O(G)}{\min}(\{u(m,P)\}),
u_1=\underset{P\in O(G)}{\max}(\{u(m,P)\})
$.
We will refer to the CS yielding $u_0$ as $P_0$.
Assume by contradiction that Max-Util is subject to weak-improvement when removing only 1 edge.
That is, there exists a manipulation $r_{m}$ and a CS $P^m\in O(G^{m})$ such that $u(m,P^m) > u_1$. That is, $P^m\notin O(G)$. Also, $P_0\notin O(G^m)$.

Therefore, we can see that $Ut(P_0,G) > Ut(P^m,G)$ and $Ut(P^m,G^m)>Ut(P_0,G^m)$.
Since the manipulator is only able to remove edges it holds that $Ut(P^m,G^m)\leq Ut(P^m,G)$. So we get that $Ut(P_0,G) > Ut(P^m,G^m) \xrightarrow{} Ut(P_0,G) - 1 \geq Ut(P^m,G^m)$
Since only 1 edge was removed, the utilitarian SW could drop at most by $1$ and it holds that $Ut(P^0,G) - 1 \leq Ut(P^0,G^m)$.
Combining the last two inequalities we get $ Ut(P_0,G^m) \geq Ut(P_m,G^m) $ in contradiction.
\end{proof}


\begin{proposition}
\label{thm:Egal_distance1}
Max-Egal with the equal size constraint is subject to 1-safe LB-improvement and is 1-safe UB-proof for a manipulator $m^-$ over directed networks.
\end{proposition}

\begin{proof}
See Figure~\ref{fig:distance1_util/egal_lb} for the 1-safe LB-improvement.

For the cases of $|N(m)|<n-3, |N(m)|=n-2$ and $|N(m)|=n-1$ Figures~\ref{fig:egalremove1},\ref{fig:egalremove2} and \ref{fig:egalremove3} show the that removing any edge is UB-unsafe, respectively.
\end{proof}

\begin{proposition}
\label{thm:least1_distance1_es}
At-Least-1 with the equal size constraint is subject to 1-safe UB-improvement and is 1-safe LB-proof for a manipulator $m^+$ over undirected networks.
\end{proposition}

\begin{proof}
See Figure~\ref{fig:distance1_least1} for the 1-safe UB-improvement.
The resistance proof is the same as without the equal size constraint.
\end{proof}


\begin{theorem}
\label{thm:all_distance1_es}
Except for the situations in Theorems~\ref{thm:util_distance1},\ref{thm:Egal_distance1} and \ref{thm:least1_distance1_es}, all of our objectives with the equal size constraint are 1-safe strategyproof. 
\end{theorem}
Here we separate the proof between the objectives as well:
\subsubsection*{Max-Util}
To show the manipulation is LB-unsafe;
If $0<|N(m)|\leq \lfloor n/2 \rfloor$ then Figures~\ref{fig:unsafeutil3} and \ref{fig:unsafeutil4} provide examples for even and odd $n$, respectively.
If $\lfloor n/2 \rfloor<|N(m)| < n/2$ then Figures~\ref{fig:unsafeutil7} and \ref{fig:unsafeutil8} provide examples for even and odd $n$, respectively.

To show the manipulation is UB-unsafe;
If $0<|N(m)|\leq \lfloor n/2 \rfloor$ then Figures~\ref{fig:unsafeutil1} and \ref{fig:unsafeutil2} provide examples for even and odd $n$, respectively.
If $\lfloor n/2 \rfloor<|N(m)| < n/2$ then Figures~\ref{fig:unsafeutil5} and \ref{fig:unsafeutil6} provide examples for even and odd $n$, respectively.


\subsubsection*{Max-Egal}
Manipulator $m^-$:
If $|N(m)| \leq n/2 -1$, a possible network which is an $n-1$ clique except for $m$ yields an egalitarian SW of exactly $|N(m)|$. Removing any of her neighbours is LB-unsafe as she is guaranteed to get all of the neighbours she reports.
Now look at a possible network where $|N(m)|-1$ neighbours of $m$ are connected only to her and all non-neighbours of $m$ are connected to another node $a_0$. Out of the non-neighbours, $n/2 -2 $ nodes have no more edges. The others are neighbours to neighbours of $m$. The last neighbour of $m$ is connected also to $a_0$. Of course, the maximum possible egalitarian SW is $1$, and the two possible outcomes for $m$ are either getting all the neighbours, or not getting the single neighbours which is connected to $a_0$. By removing an edge it might be the case that this single neighbour is the one being disconnected, hence removing an edge is UB-unsafe.
If $n/2 -1 < |N(m) < n-1$, look at a possible network where $n/2-1$ of neighbours of $m$ form a path $a_1, a_2, .. ,a_{n/2-1}$. All the other nodes form a clique of size $n/2$ and one of them $a_0$ is connected to $a_1$. The highest egalitarian SW achievable is $2$, by putting $m$ alongside the path. By removing the edge towards $a_1$ or $a_2$ the highest egalitarian SW possible is now 1, and it can also be achieved by swapping $m$ and $a_0$. Hence removing is LB-unsafe.
In a possible network where the $n/2-1$ neighbours are connected only to $m$ and $a_0$ (but not between themselves) the highest egalitarian SW possible is $1$, either by $m$ or $a_0$ being with the $n/2-1$ neighbours. By removing an edge towards one of the neighbours, $m$ is guaranteed to not be with them, but in the other coalition with only one neighbour, hence removing is UB-unsafe.
If $|N(m)=n-1$, if $n$ is even no manipulation is beneficial as the manipulator always gets her highest utility. 
If $n$ is odd, the possible network could be made of two cliques $L_1$ and $L_2$ of sizes $n-1/2$ and $n-1/2 -1$, and $L_1$ is missing 1 edge between two nodes $a_1, a_2$. The last node $a_0$ is connected to the whole network just like $m$. The two possible CSs yielding an egalitarian SW of $n-1/2 -1$ are where $L_1$ form a coalition alongside either $m$ or $a_0$ and $L_2$ is with the other. Hence the upper bound for $m$ is $n-1/2$. By removing an edge towards either $a_1$ or $a_2$, $m$ is guaranteed to be with $L_2$, so removing is UB-unsafe.

Manipulator $m^+$:
If $|N(m)| \leq n/2 -1 $, look at a network that is made of a star of size $n/2$ that includes all of $m$'s neighbour, another star of size $n/2-2$, and a node $a_0$ with no edges. Adding an edge towards $a_0$ is guaranteeing utility $0$ for the manipulator, as him being  being with $a_0$ and the smaller star is the only CS yielding egalitarian SW of $1$. Hence it is both LB and UB-unsafe.
If $n/2-1 < |N(m)| < n-1$, it is possible the graph is a clique guaranteeing $m$ a full coalition of neighbours. By adding any edge this might make some of the neighbours fake, therefore UB-unsafe. Lastly, look at a graph where $n/2 -1$ neighbours of $m$ form a circle $C_m$. Also, another $n/2-1$ nodes form a circle $C_0$, and call the last node $a_0$. $a_0$ is connected to two nodes $C_0$ and to one in $C_m$. Also, there is a node $a_1$ in $C_m$ that is connected to two nodes in $C_0$. Adding an edge to $a_0$ is LB-unsafe, as now a CS where $m$ is with $a_0$ instead of $a_1$ is a solution.

\subsubsection*{At-Leas-1}
If $|N(m)|<\lfloor n/2 \rfloor$, removing can still lead to an infeasible instance. If $|N(m)|<\lfloor n/2 \rfloor$ then removing edges while still having more than $\lfloor n/2 \rfloor$ does not change the outcome, as any previous solution is still satisfying At-Least-1. This is because the manipulator is still guaranteed at least $1$ out neighbour in every CS, and the other agents' utility has not changed. Removing edge until having less than $\lfloor n/2 \rfloor$ can again result in infeasible instance.

\begin{figure*}[t]
    \centering
    \begin{subfigure}{0.6\textwidth}
    \centering  
        \begin{subfigure}{0.15\textwidth}
        \renewcommand\thesubfigure{\alph{subfigure}1}
            \centering
        \includegraphics[page=5,width=\textwidth]{Graphs/graphs.pdf}
        \caption{\\U/S/ES}
        \label{fig:util_size_add}
        \end{subfigure}
        \hfill
        \begin{subfigure}{0.15\textwidth}
            \addtocounter{subfigure}{-1}
            \renewcommand\thesubfigure{\alph{subfigure}2}
            \centering
        \includegraphics[page=23,width=\textwidth]{Graphs/graphs.pdf}
        \caption{\\E/LB/ES}
        \label{fig:Egal_size_undirected_add_LB}
        \end{subfigure}
        \hfill
        \begin{subfigure}{0.15\textwidth}
            \addtocounter{subfigure}{-1}
            \renewcommand\thesubfigure{\alph{subfigure}3}
            \centering
        \includegraphics[page=25,width=\textwidth]{Graphs/graphs.pdf}
        \caption{\\E/UB/ES}
        \label{fig:Egal_size_undirected_add_UB}
        \end{subfigure}
        \hfill
        \begin{subfigure}{0.15\textwidth}
                    \addtocounter{subfigure}{-1}
            \renewcommand\thesubfigure{\alph{subfigure}4}
            \centering
        \includegraphics[page=33,width=\textwidth]{Graphs/graphs.pdf}
        \caption{\\1/S/ES}
        \label{fig:Least1_size_undirected_add}
        \end{subfigure}
    \addtocounter{subfigure}{-1}
    \caption{$m^+$}
    \label{fig:add_es_subfig} 
    \end{subfigure}
    \hfill
    \begin{subfigure}{0.3\textwidth}
    \centering  
        \begin{subfigure}{0.3\textwidth}
            \renewcommand\thesubfigure{\alph{subfigure}1}
            \centering
        \includegraphics[page=45,width=\textwidth]{Graphs/graphs.pdf}
        \caption{\\U/S/ES}
        \label{fig:Util_size_remove}
        \end{subfigure}    
        \hfill
        \begin{subfigure}{0.3\textwidth}
        \addtocounter{subfigure}{-1}
        \renewcommand\thesubfigure{\alph{subfigure}2}
            \centering
        \includegraphics[page=21,width=\textwidth]{Graphs/graphs.pdf}
        \caption{\\E/S/ES}
        \label{fig:Egal_size_remove}
        \end{subfigure}
        \hfill
        \begin{subfigure}{0.3\textwidth}
                    \addtocounter{subfigure}{-1}
        \renewcommand\thesubfigure{\alph{subfigure}3}
        \centering
        \includegraphics[page=27,width=\textwidth]{Graphs/graphs.pdf}
        \caption{\\1/LB/ES}
        \label{fig:least1_size_remove}
        \end{subfigure}    
        \addtocounter{subfigure}{-1}
    \caption{$m^-$}
    \label{fig:remove_es_subfig} 
    \end{subfigure}
%    \hfill
%    \begin{subfigure}{0.1\textwidth}

%        \renewcommand\thesubfigure{\alph{subfigure}1}
%    \centering
%        \begin{subfigure}{0.4\textwidth}
%        \centering
%        \includegraphics[page=9,width=\textwidth]{Graphs/Distance 1 manipulation.pdf}
%        \caption{\\E/UB}
%        \label{fig:distance1_egal_ub_any}
%    \end{subfigure} \hfill
%        \begin{subfigure}{0.4\textwidth}
%        \addtocounter{subfigure}{-1}
%        \renewcommand\thesubfigure{\alph{subfigure}2}
%        \centering
%        \includegraphics[page=7,width=\textwidth]{Graphs/Distance 1 manipulation.pdf}
%        \caption{\\1/UB}
%        \label{fig:distance1_least1}
%    \end{subfigure}
%    \addtocounter{subfigure}{-1}
%    \caption{Distance 1}
    %%graphs showing manipulation of different objectives by adding edges. Dotted edges are added by the manipulator.
%    \label{fig:distance1_manipulation}
%    \end{subfigure}
    
    
    \caption{Figures providing examples of susceptibility to manipulation with the equal size constraint. Key: U=Max-Util, E=Max-Egal, 1=At-least-1, S=strict-improvement, UB=UB-improvement, LB=LB-improvement}
    %graphs showing manipulation of different objectives by adding edges. Dotted edges are added by the manipulator.
    \label{fig:es_graphs}
\end{figure*}

\begin{figure}
    \centering
    \begin{subfigure}{0.07\textwidth}
        \centering
        \includegraphics[page=1,width=\textwidth]{Graphs/Distance 1 special cases.pdf}
        \caption{R,UB}
        \label{fig:thrm5_1}
    \end{subfigure}
    \hfill
    \begin{subfigure}{0.07\textwidth}
        \centering
        \includegraphics[page=2,width=\textwidth]{Graphs/Distance 1 special cases.pdf}
        \caption{R,UB}
        \label{fig:thrm5_2}
    \end{subfigure}
    \hfill
    \begin{subfigure}{0.07\textwidth}
        \centering
        \includegraphics[page=3,width=\textwidth]{Graphs/Distance 1 special cases.pdf}
        \caption{R,UB}
        \label{fig:thrm5_3}
    \end{subfigure}
    \hfill
    \begin{subfigure}{0.07\textwidth}
        \centering
        \includegraphics[page=4,width=\textwidth]{Graphs/Distance 1 special cases.pdf}
        \caption{R,UB}
        \label{fig:thrm5_4}
    \end{subfigure}
    \hfill
    \begin{subfigure}{0.07\textwidth}
        \centering
        \includegraphics[page=5,width=\textwidth]{Graphs/Distance 1 special cases.pdf}
        \caption{R,UB}
        \label{fig:thrm5_5}
    \end{subfigure}
    \\
    \begin{subfigure}{0.07\textwidth}
        \centering
        \includegraphics[page=6,width=\textwidth]{Graphs/Distance 1 special cases.pdf}
        \caption{A,UB}
        \label{fig:unsafeutil1}
    \end{subfigure}
    \hfill
    \begin{subfigure}{0.07\textwidth}
        \centering
        \includegraphics[page=7,width=\textwidth]{Graphs/Distance 1 special cases.pdf}
        \caption{A,UB}
        \label{fig:unsafeutil2}
    \end{subfigure}
    \hfill
    \begin{subfigure}{0.07\textwidth}
        \centering
        \includegraphics[page=8,width=\textwidth]{Graphs/Distance 1 special cases.pdf}
        \caption{A,LB}
        \label{fig:unsafeutil3}
    \end{subfigure}
    \hfill
    \begin{subfigure}{0.07\textwidth}
        \centering
        \includegraphics[page=9,width=\textwidth]{Graphs/Distance 1 special cases.pdf}
        \caption{A,LB}
        \label{fig:unsafeutil4}
    \end{subfigure}
    \\
    \begin{subfigure}{0.07\textwidth}
        \centering
        \includegraphics[page=10,width=\textwidth]{Graphs/Distance 1 special cases.pdf}
        \caption{A,UB}
        \label{fig:unsafeutil5}
    \end{subfigure}
    \hfill
    \begin{subfigure}{0.07\textwidth}
        \centering
        \includegraphics[page=11,width=\textwidth]{Graphs/Distance 1 special cases.pdf}
        \caption{A,UB}
        \label{fig:unsafeutil6}
    \end{subfigure}
    \hfill
    \begin{subfigure}{0.07\textwidth}
        \centering
        \includegraphics[page=12,width=\textwidth]{Graphs/Distance 1 special cases.pdf}
        \caption{A,LB}
        \label{fig:unsafeutil7}
    \end{subfigure}
    \hfill
    \begin{subfigure}{0.07\textwidth}
        \centering
        \includegraphics[page=13,width=\textwidth]{Graphs/Distance 1 special cases.pdf}
        \caption{A,LB}
        \label{fig:unsafeutil8}
    \end{subfigure}
    \caption{Unsafe manipulations, Max-Util, with the equal size constraint. Legend: A - add, R - remove, LB/UB - LB/UB-unsafe. In the figures $n$ stands for the total number of agents and $x$ for any number between $0$ and the clique's size}
    %graphs showing manipulation of different objectives by adding edges. Dotted edges are added by the manipulator.
    \label{fig:unsafe_util_equalsize}
\end{figure}


\begin{figure}
    \centering
\begin{subfigure}{0.07\textwidth}
        \centering
        \includegraphics[page=1,width=\textwidth]{Graphs/Distance 1 util unsafe.pdf}
        \caption{R/LB}
        \label{fig:distance1_util_unsafe_remove_LB_appendix}
    \end{subfigure}
    \hfill
    \begin{subfigure}{0.07\textwidth}
        \centering
        \includegraphics[page=2,width=\textwidth]{Graphs/Distance 1 util unsafe.pdf}
        \caption{R/UB}
        \label{fig:distance1_util_unsafe_remove_UB_appendix}
    \end{subfigure}
    \hfill
    \begin{subfigure}{0.07\textwidth}
        \centering
        \includegraphics[page=3,width=\textwidth]{Graphs/Distance 1 util unsafe.pdf}
        \caption{A/LB}
        \label{fig:distance1_util_unsafe_add_LB}
    \end{subfigure}
    \hfill
    \begin{subfigure}{0.07\textwidth}
        \centering
        \includegraphics[page=4,width=\textwidth]{Graphs/Distance 1 util unsafe.pdf}
        \caption{A/UB}
        \label{fig:distance1_util_unsafe_add_UB}
    \end{subfigure}

    \caption{Unsafe manipulations, Max-Util. Legend: A - add, R - remove, LB/UB - LB/UB-unsafe}
    %graphs showing manipulation of different objectives by adding edges. Dotted edges are added by the manipulator.
    \label{fig:distance1_unsafe_util_appendix}
\end{figure}

\begin{figure}
    \centering
    \begin{subfigure}{0.07\textwidth}
        \centering
        \includegraphics[page=1,width=\textwidth]{Graphs/egal unsafe distance1.pdf}
        \caption{R/UB}
        \label{fig:egalremove1}
    \end{subfigure}
    \hfill
    \begin{subfigure}{0.07\textwidth}
        \centering
        \includegraphics[page=2,width=\textwidth]{Graphs/egal unsafe distance1.pdf}
        \caption{R/UB}
        \label{fig:egalremove2}
    \end{subfigure}
    \hfill
    \begin{subfigure}{0.07\textwidth}
        \centering
        \includegraphics[page=3,width=\textwidth]{Graphs/egal unsafe distance1.pdf}
        \caption{R/UB}
        \label{fig:egalremove3}
    \end{subfigure}
    \hfill
    \begin{subfigure}{0.07\textwidth}
        \centering
        \includegraphics[page=4,width=\textwidth]{Graphs/egal unsafe distance1.pdf}
        \caption{A/UB}
        \label{fig:egalremove4}
    \end{subfigure}

    \caption{Unsafe manipulations, Max-Egal. Legend: A - add, R - remove, LB/UB - LB/UB-unsafe}
    %graphs showing manipulation of different objectives by adding edges. Dotted edges are added by the manipulator.
    \label{fig:distance1_unsafe_egal}
\end{figure}

\begin{figure}
    \centering
    \begin{subfigure}{0.07\textwidth}
        \centering
        \includegraphics[page=2,width=\textwidth]{Graphs/Distance 1 manipulation.pdf}
        \caption{\\U,E/LB}
        \label{fig:distance1_util/egal_lb}
    \end{subfigure} \hspace{20pt}
    \begin{subfigure}{0.07\textwidth}
        \centering
        \includegraphics[page=4,width=\textwidth]{Graphs/Distance 1 manipulation.pdf}
        \caption{\\U/UB}
        \label{fig:distance1_util_ub}
    \end{subfigure} \hspace{20pt}
    \begin{subfigure}{0.07\textwidth}
        \centering
        \includegraphics[page=7,width=\textwidth]{Graphs/Distance 1 manipulation.pdf}
        \caption{\\1/UB}
        \label{fig:distance1_least1}
    \end{subfigure}
    \caption{Manipulations by removing edges, for distance 1. Legend: Objective/Manipulation~Type. Key: U=Max-Util, E=Max-Egal, 1=At-Least-1, UB=UB-improvement, LB=LB-improvement.}
    %graphs showing manipulation of different objectives by adding edges. Dotted edges are added by the manipulator.
    \label{fig:distance1_manipulation}
\end{figure}


\clearpage

\subsection*{Max-Util Distance 2 Conjecture}
\begin{conjecture}
\label{conj:util}
Max-Util is 2-safe weak-proof against manipulator $m^-$ over undirected networks.
\end{conjecture}

We lay out some of the advances we have made trying to prove the conjecture:

Let $G_2=(A,E)$ be a partial network and $m^-$ a manipulator. Assume by negation a 2-safe weak-improvement $r^m$ exists over $G_2$. Hence there exists a possible game $\overline{G}$ of $G_2$ for which $r^m$ is a weak-improvement. We want to show that either the manipulation does not really improve the upper bound, or there exists another supplement $\overline{G'}$ of $G$ for which $r^m$ lowers the manipulator's upper bound.

Denote $\overline{G}^m$ as the network $\overline{G}$ after the manipulation $r^m$. Let $P^m=\{B_1,B_2\}$ be (one of) the CSs with higher utility for $m$ than all CSs in $\overline{G}$, and denote by $B_1$ the coalition in which $m$ is. Formally:
\begin{equation}
\label{ineq:util_ub_safe}
u(m,P_m) > \underset{P\in O_{obj}(\overline{G})}{\min}(u(m,P)).    
\end{equation}
Denote by $c_0$ and $c_m$ the minimum cut size of $\overline{G}$ and $\overline{G}^m$ respectively.
We will define $c(P,G)$ as the cut size of coalition structure $P$ in network $G$.


\begin{lemma}
$B_2$ contains at least $2$ node $a_1,a_2$ such that $a_1,a_2\in N$ and $a_1,a_2\notin N^m$.
\end{lemma}
\begin{proof}
Since $r^m$ is a LB-improvement there is a solution for $\overline{G}$ which is not a solution in $\overline{G}^m$. Since we only removed edges, the only reason it would happen is that $c_0 > c_m$.
Since $P^m$ is not a solution in $\overline{G}$ (as it is better than all solutions in it) we get that $c(P^m,\overline{G}) > c_0$. Combining both outcomes we get that $c(P^m,\overline{G}) \geq c_m+2$. Because we know that $c(P^m,\overline{G}^m) = c_m$ and the only changes between $\overline{G}$ and $\overline{G}^m$ are edges that the manipulator removed, it has to be the case that he removed at least $2$ edges going out to $B_2$.
\end{proof}
We get that the maximum utility $m$ can have after the manipulation is $|N(m)|-2$. Because of that, the minimum degree $\delta(\overline{G}) = min(\{deg(v) | v\in V\ and v != m \})$ of all the nodes but $m$ is at least $c_0+1$. Otherwise the CS where a node with at most $c_0$ neighbours is alone is a minimal cut and yields $m$ a utility of $|N(m)|-1$ or $|N(m)|$ in contradiction to inequality~\ref{ineq:util_ub_safe}.

We distinguish between multiple cases. 
Case one: $B_2\subseteq N$, i.e. contains only true neighbours of $m$.
Since $P^m$ has higher utility for the manipulator than all previous solutions, in all previous solutions there are at least $|B_2|+1$ neighbours of her not in her coalition. Hence $c_0 \geq |B_2|+1$. By the previous statement we get that $\delta(\overline{G}) \geq |B_2| + 2$. Since the manipulator might have removed one edge in $\overline{G}^m$ , we get that $\delta(\overline{G}^m) \geq |B_2| + 1$. Because of that, in $P^m$ every node in $B_2$ has at least $2$ neighbours outside $B_2$, which we concludes that $c_m \geq 2|B_2| \rightarrow c_0 > 2|B_2| \rightarrow \delta(\overline{G}) > 2|B_2| + 1 \rightarrow \delta(\overline{G}^m) \geq 2|B_2|+1$. Hence all nodes in $B_2$ has at least $|B_2|+2$ edges going out from $B_2$, so we get that $c_m > |B_2|*(|B_2|+2)$. By repeating this process we get that the cut is unbounded, in contradiction since the network is finite.

Case two: $|B_2\cap N| = |B_2| -1$, i.e. $B_2$ contains exactly one non-neighbour of $m$, denote her by $a$.
By the same logic from above we get that $c_0 \geq |B_2|$ (as in $P^m$ the manipulator only has $|B_2| -1$ neighbours. So $\delta(\overline{G}) \geq |B_2| + 1 \rightarrow \delta(\overline{G}^m) \geq |B_2|$. Since $a$ is not a neighbour of $m$, her degree in $\overline{G}^m$ has not changed and it is still at least $|B_2| +1$. In conclusion in $B_2$ we have $|B_2|-1$ nodes of degree at least $|B_2|$ and one with at least $|B_2| +1$. Hence $c_m \geq |B_2| +1 \rightarrow c_0 \geq |B_2| + 2$, and yet again by repeating the process we get that the cut is unbounded.

Case three: $B_2$ contains at least 2 non-neighbours of $m$.

This is the case we could not prove. Note that in any solution in Max-Util any agent is guaranteed to be in the coalition where she has more neighbours. Therefore if $B_2$ contains at least $2$ neighbours of $m$, then $B_1$ contains at least $3$. However, if the manipulator had only $5$ neighbours originally, he would have had at least $3$ neighbours before the manipulation. So he has at least $4$ neighbours in $B_1$.


To sum it up, if a weak-improvement did exist, it has to be that $B_1$ contains at least $4$ neighbours of $m$, $B_2$ contains at least $2$ neighbours and $2$ non-neighbours of $m$, and the minimum degree of the graph is at least $4$.



\end{document}
\section{Experiment Setups}
\label{apdix:exp_setup}

\paragraph{Dataset.}
For image classification we use CIFAR10 \citep{cifar10_data} which has 50000 training images and 10000 test images. We use the standard train/test split and preprocessing steps as with \texttt{torchvision}.
For text classification we use the SNLI dataset \citep{bowman-etal-2015-large} and the QNLI dataset \citep{wang2019glue}. We use the same train/test split and preprocessing steps as in \texttt{Opacus}'s text classification with DP tutorial.
For node classification, we use the graph dataset, ogbn-arxiv \citep{hu2021open}. In this graph, nodes represent arXiv Computer Science papers and directed edges represents paper cites paper. This graph dataset has 169,343 nodes, average degree 13.7, 128 features, 40 classes, and 0.54/0.18/0.28 train/val/test split.

\paragraph{Model.}
For image classification on CIFAR10, we use a 5-layered CNN model as described in \citep{papernot2020tempered}.
For text classification on SNLI, we use a BERT-base model \citep{bert_paper} as in \texttt{Opacus}'s text classification tutorial.
For node classification, wee use a DP-GCN from \citet{daigavane2022nodelevel}. For DP-Adam and DP-AdamBC, this model has one encoder layer, one message passing layer, and two decoder layers. For DP-SGD, this model has two encoder layers and one decoder layer instead. In both cases, we use a latent size of 100 for the encoder, GNN, and decoder due to memory constraints.

\paragraph{Hyperparameters.}
For image classification on CIFAR10, the DP hyperparameters $C=1.0, \sigma=2.0$, batch size $=8192$, target $\epsilon = 7.1$ with R\'enyi DP for privacy accounting from Opacus \citep{opacus}. The learning rate for DP-AdamBC, DP-Adam and DP-SGD are 0.005, 0.007 and 2.5 respectively. The numerical stability constant is $\gamma'=5e$-8 and $\gamma=1$e-8 for DP-AdamBC, DP-Adam respectively. $\beta_1=0.9, \beta_2=0.999$ in both DP-AdamBC and DP-Adam. We use the Adam and SGD implementation from \texttt{optax} \citep{deepmind2020jax}.

For text classification on SNLI, the DP hyperparameters are $C=0.1, \sigma=0.4$, batch size $=256$, target $\epsilon = 7.0$ with R\'enyi DP for privacy accounting from Opacus \citep{opacus}. The learning rate for DP-AdamBC, DP-Adam and DP-SGD are 0.001, 0.01 and 45.0 respectively. The numerical stability constant is $\gamma'=1e$-10 and $\gamma=1$e-8 for DP-AdamBC, DP-Adam respectively. For text classification on QNLI, the DP hyperparameters are $C=0.1, \sigma=0.4$, $B=256$, target $\epsilon = 7.3$ with R\'enyi DP for privacy accounting from \texttt{Opacus}. The learning rate for DP-AdamBC, DP-Adam and DP-SGD are 0.003, 0.01 and 40.0 respectively. The numerical stability constant is $\gamma'=3e$-9 and $\gamma=1$e-8 for DP-AdamBC, DP-Adam respectively. We use the Adam and SGD implementation from \texttt{PyTorch} \citep{pytorch}.

For node classification in Figure \ref{fig:exp1}, we use a batch size of 10,000 and target $\epsilon = 12.0$ with Rényi DP privacy accounting from \citet{daigavane2022nodelevel}. We use noise multiplier $\lambda = 2$ and maximum degree $K = 7$ as they are defined in \citet{daigavane2022nodelevel}. In their paper, they use per-layer clipping; for each layer, their clipping thresholds are chosen as the 75th percentile of the gradient norms for that layer. We do not use per-layer clipping, we choose the clipping threshold $C$ as the median of their per-layer 75th percentile clipping thresholds. The learning rate for DP-AdamBC, DP-Adam, and DP-SGD are 0.003, 0.008, and 0.7 respectively. The numerical stability constant is $\gamma' = 2e$-6 and $\gamma = 1e$-12 for DP-AdamBC and DP-Adam respectively.

\paragraph{Hardware information.} We run experiments on local machine with Intel 11th 2.5GHz CPU and one Nvidia GeForce RTX 3090 GPU. The typical training time is about 15min, 2.5h and 10min on our image, text and node classification tasks respectively.

\section{Proof of Proposition \ref{proposition:consistent}}
\label{apdix:consistent}

\begin{proposition*}[\ref{proposition:consistent}]
Under Assumption \ref{assumption:stationarity}, the DP-AdamBC update (without numerical stability constant) $\frac{\hat{m}_t^p}{\sqrt{\max(\hat{v}_t^{\textnormal{corr}}, 0)}}$ is a consistent estimator of $\frac{\E[\bar{g}_t]}{\sqrt{\E[\bar{g}_t^2]}}$ as $\beta_1, \beta_2 \rightarrow 1$, and $t \rightarrow \infty$.
\end{proposition*}

\begin{proof}
Let $\hat{m}_t^{p}$, $\hat{v}_t^{p}$ and $\hat{v}_t^{\textnormal{corr}}$ be the following,
\begin{gather*}
    \hat{m}_t^{p} = \frac{(1-\beta_1)\sum_{\tau=1}^{t}\beta_1^{t-\tau}\Tilde{g}_t}{1-\beta_1^t}, \; \\
    \hat{v}_t^{p} = \frac{(1-\beta_2)\sum_{\tau=1}^{t}\beta_2^{t-\tau}\Tilde{g}_t^2}{1-\beta_2^t}, \; \\% \hat{v}_t^{\textnormal{corr}} = \frac{(1-\beta_2)\sum_{\tau=1}^{t}\beta_2^{t-\tau}\Tilde{g}_t^2}{1-\beta_2^t} - \big( \frac{\sigma C}{B} \big)^2.
    \hat{v}_t^{\textnormal{corr}} = \hat{v}_t^{p} - \big( \frac{\sigma C}{B} \big)^2,
\end{gather*}
We first show that $\hat{m}_t^{p} \xrightarrow{p} \E{[\Bar{g}_t]}$, and that $\hat{v}_t^{p} \nrightarrow \E{[\Bar{g}_t^2]}$ (though $\hat{v}_t^{\textnormal{corr}} \xrightarrow{p} \E{[\Bar{g}_t^2]}$), and study the full update at the end of the proof.

We start by showing that $\hat{m}_t^{p} \xrightarrow{p} \E{[\Bar{g}_t]}$. Using Chebychev's inequality, we have:
\begin{equation*}
    \sP\big( |\hat{m}_t^p - \E{[\Bar{g}_t]}| > \delta \big) \leq \frac{\E{[(\hat{m}_t^p - \E{[\Bar{g}_t])^2}]}}{\delta^2},
\end{equation*}
\begin{align*}
    \E{[(\hat{m}_t^p - \E{[\Bar{g}_t])^2}]} &= \V{[\hat{m}_t^p - \E{[\Bar{g}_t]}]} + \big(\E{[\hat{m}_t^p - \E{[\Bar{g}_t]}]}\big)^2 \\
    % &= \V{[\hat{m}_t^p]} + 0
    % &= \big(\frac{1-\beta_1}{1-\beta_1^t}\big)^2 \sum_{\tau=1}^t \beta_1^{t-\tau} \V{[\Tilde{g}_\tau]}
    &= \big(\frac{1-\beta_1}{1-\beta_1^t}\big)^2 \sum_{\tau=1}^t \beta_1^{t-\tau} \big( \V{[\Bar{g}_\tau]} + \V{[z_\tau]} \big) \\
    &= \frac{(1-\beta_1)^2}{(1-\beta_1^t)} \big(\V{[\Bar{g}_t]} + \big(\frac{\sigma C}{B}\big)^2\big) \\
    &\leq \frac{(1-\beta_1)^2}{(1-\beta_1^t)} \big(4C^2 + \big(\frac{\sigma C}{B}\big)^2\big) \\
    &\rightarrow 0 \; \textnormal{when } t \rightarrow \infty, \beta_1 \rightarrow 1.
\end{align*}
The last inequality follows from the fact that $\Bar{g}_t$ is the clipped gradient (for DP), and hence $\V{[\Bar{g}_t]}$ is upper bounded by $4C^2$, the square of L2-norm clipping value.

Next we show that $\hat{v}_t^{p} \nrightarrow \E{[\Bar{g}_t^2]}$ but $\hat{v}_t^{\textnormal{corr}} \xrightarrow{p} \E{[\Bar{g}_t^2]}$. Using Chebychev's inequality again:
\begin{equation*}
\begin{gathered}
    \sP\big( |\hat{v}_t^p - \E{[\Bar{g}_t^2]}| > \delta \big) \leq \frac{\E{[(\hat{v}_t^p - \E{[\Bar{g}_t^2])^2]}}}{\delta^2}, \\
    \E{[(\hat{v}_t^p - \E{[\Bar{g}_t^2])^2]}} = \V{[\hat{v}_t^p - \E{[\Bar{g}_t^2]}]} + \big(\E{[\hat{v}_t^p - \E{[\Bar{g}_t^2]}]}\big)^2,
\end{gathered}
\end{equation*}
Moreover, $\V{[\hat{v}_t^p - \E{[\Bar{g}_t^2]}]} \rightarrow 0$ when $t\rightarrow \infty, \beta_2 \rightarrow 1$ since,
\begin{align*}
    &\V{[\hat{v}_t^p - \E{[\Bar{g}_t^2]}]} = \V{[\hat{v}_t^p]} \\
    &= \big( \frac{1-\beta_2}{1-\beta_2^t} \big)^2 \sum_{\tau=1}^{t} \beta_2^{2(t-\tau)} \V{[\Tilde{g}_{\tau}^2]} \\
    &= \big( \frac{1-\beta_2}{1-\beta_2^t} \big)^2 (\V{[\Bar{g}_t^2]} + \V{[z_t^2]} + 4\V{[\Bar{g}_t]}\V{[z_t]}) \sum_{\tau=1}^{t} \beta_2^{2(t-\tau)}  \\
    % &= \big( \frac{1-\beta_2}{1-\beta_2^t} \big)^2 \bigg( 2\big( \frac{\sigma C}{B} \big)^2 + (2\Bar{g}_t)^2 \frac{\sigma C}{B} \big)^2\bigg) \sum_{\tau=1}^{t} \beta_2^{2(t-\tau)} \\
    &= \big( \frac{1-\beta_2}{1-\beta_2^t} \big)^2  \big( \frac{1-\beta_2^{2t}}{1-\beta_2^2} \big) \bigg( \V{[\Bar{g}_t^2]} + 2\big( \frac{\sigma C}{B} \big)^2 + 4\V{[\Bar{g}_t]} \big( \frac{\sigma C}{B} \big)^2\bigg) \\
    &\rightarrow 0 \; \textnormal{as } t\rightarrow \infty, \beta_2 \rightarrow 1.
\end{align*}
And since $\E{\big[ \hat{v}_t^{p} - \E{[\Bar{g}_t^2]} \big]} = \E{[\Bar{g}_t^2]} + \big(\frac{\sigma C}{B} \big)^2 - \E{[\Bar{g}_t^2]} = \big(\frac{\sigma C}{B} \big)^2 \nrightarrow 0$, but $\E{\big[ \hat{v}_t^{\textnormal{corr}} - \E{[\Bar{g}_t^2]} \big]} = (\E{[\Bar{g}_t^2]} - \big(\frac{\sigma C}{B} \big)^2) + \big(\frac{\sigma C}{B} \big)^2 - \E{[\Bar{g}_t^2]} = 0$, we have $\hat{v}_t^{p} \nrightarrow \E{[\Bar{g}_t^2]}$ but $\hat{v}_t^{\textnormal{corr}} \xrightarrow{p} \E{[\Bar{g}_t^2]}$.

Let $g(x)=1/\sqrt{x}, x\geq 0$. By the continuous mapping theorem, $\frac{1}{\sqrt{\hat{v}_t^{\textnormal{corr}}}} \xrightarrow{p} \frac{1}{\sqrt{\E{[\Bar{g}_t^2]}}}$. Then since $\hat{m}_t^{p} \xrightarrow{p} \E{[\Bar{g}_t]}$ and $\frac{1}{\sqrt{\hat{v}_t^{\textnormal{corr}}}} \xrightarrow{p} \frac{1}{\sqrt{\E{[\Bar{g}_t^2]}}}$, by joint convergence in probability, $(\hat{m}_t^{p}, \frac{1}{\sqrt{\hat{v}_t^{\textnormal{corr}}}}) \xrightarrow{p} (\E{[\Bar{g}_t]}, \frac{1}{\sqrt{\E{[\Bar{g}_t^2]}}})$.
Let $g(x,y)=xy$. Applying the continuous mapping theorem again yields $\frac{\hat{m}_t^p}{\sqrt{\hat{v}_t^{\textnormal{corr}}}} \xrightarrow{p} \frac{\E{[\Bar{g}_t]}}{\sqrt{\E{[\Bar{g}_t^2]}}}$.
\end{proof}


% Therefore $\hat{v}_t^{p}$ is not a consistent estimator of $\Bar{g}_t^2$. Even though the gap could be minimize by decreasing $\sigma C$ or increasing $B$, but it would lead to higher privacy cost.

% By a similar argument, for $\hat{v}_t^{\textnormal{corr}}$ we have,
% \begin{equation*}
%     \sP\big( |\hat{v}_t^{\textnormal{corr}} - \Bar{g}_t^2| > \delta \big) \leq \frac{\E{[(\hat{v}_t^{\textnormal{corr}} - \Bar{g}_t^2)^2]}}{\delta^2}, \; \E{[(\hat{v}_t^{\textnormal{corr}} - \Bar{g}_t^2)^2]} = \V{[\hat{v}_t^{\textnormal{corr}} - \Bar{g}_t^2]} + \big(\E{[\hat{v}_t^{\textnormal{corr}} - \Bar{g}_t^2]}\big)^2,
% \end{equation*}
% which $\V{[\hat{v}_t^{\textnormal{corr}} - \Bar{g}_t^2]} = \V{[\hat{v}_t^p]} \rightarrow 0$ when $t\rightarrow \infty, \beta_2 \rightarrow 1$, and $\E{\big[ \hat{v}_t^{\textnormal{corr}} - \Bar{g}_t^2 \big]} = (\Bar{g}_t^2 - \big(\frac{\sigma C}{B} \big)^2) + \big(\frac{\sigma C}{B} \big)^2 - \Bar{g}_t^2 = 0$, thus $\hat{v}_t^{\textnormal{corr}}$ is a consistent estimator of $\Bar{g}_t^2$.

% \begin{equation*}
% \begin{gathered}
%     \hat{v}_t^{p} = \frac{(1-\beta_2)\sum_{\tau=1}^{t}\beta_2^{t-\tau}\Tilde{g}_t^2}{1-\beta_2^t},
%     \hat{v}_t^{\textnormal{corr}} = \frac{(1-\beta_2)\sum_{\tau=1}^{t}\beta_2^{t-\tau}\Tilde{g}_t^2}{1-\beta_2^t} - \bigg( \frac{\sigma C}{B} \bigg)^2 \\
%     % \lim_{t\rightarrow \infty} \E{\big[ \hat{v}_t^{p} - \Bar{g}_t^2 \big]} = \Bar{g}_t^2 + \big(\frac{\sigma C}{B} \big)^2 - \Bar{g}_t^2 = \big(\frac{\sigma C}{B} \big)^2 \nrightarrow 0\\
%     % \lim_{t\rightarrow \infty} \E{\big[ \hat{v}_t^{\textnormal{corr}} - \Bar{g}_t \big]} = \Bar{g}_t^2 - \Bar{g}_t^2 = 0
% \end{gathered}
% \end{equation*}


\section{Concentration of $\hat{v}_t^p$}
\label{apdix:bound}

First we introduce the following lemma which is used in both proofs of Proposition \ref{prop:bound_fixed} and \ref{prop:bound_martingale}.
\begin{lemma}
\label{lemma:zk2_sub_exponential}
    Let $Z \sim \cN(0, (1/B^2)\sigma^2 C^2)$, $\beta \in [0,1]$ be a constant and $B, C$ be constants such that $B>0$ and $C>0$, then $\beta Z^2$ is sub-exponential with $\nu=2\beta(1/B^2)\sigma^2C^2, b=4\beta(1/B^2)\sigma^2C^2$.
\end{lemma}
\begin{proof}
    Since $Z \sim \cN(0, (1/B^2)\sigma^2 C^2)$, let $X = \frac{Z}{(1/B)\sigma C} \sim \cN(0,1)$ and $a=\beta(1/B^2)\sigma^2C^2$ be a temporary constant,
    \begin{align*}
        &\E{\Big[ \exp{\bigl\{ \lambda (\beta Z^2 - \E{[\beta Z^2]}) \bigl\} }\Big]} \\
        &= \E{\Big[ \exp{\bigl\{ \lambda (a X^2  - a) \bigl\}}\Big]} \\
        &= \frac{1}{\sqrt{2\pi}} \int_{-\infty}^{\infty} \exp{ \bigl\{ \lambda (a z^2 - a) \bigl\}} \exp{\bigl\{-z^2/2\bigl\} } dz \\
        &= \frac{e^{-\lambda a}}{\sqrt{2\pi}} \int_{-\infty}^{\infty} \exp{\bigl\{ -\frac{z^2}{2}(1-2\lambda a)\bigl\} } dz \\
        &= \frac{e^{-\lambda a}}{\sqrt{1-2\lambda a}} \frac{1}{\sqrt{2\pi}} \int_{-\infty}^{\infty} e^{-y^2/2} dy, \; y=\sqrt{1-2\lambda a}z \\
        &= \frac{e^{-\lambda a}}{\sqrt{1-2\lambda a}}, \; \lambda < \frac{1}{2a} \\
        & \leq e^{\lambda^2 \nu^2/2}, \; \textnormal{for } \nu \geq \sqrt{2}a.
    \end{align*}
    The constant $\sqrt{2}$ comes from taking Taylor expansion around 0 of $\frac{e^{-\lambda a}}{\sqrt{1-2\lambda a}}$ and $e^{\lambda^2 \nu^2/2}$,
    \begin{equation*}
        \frac{e^{-\lambda a}}{\sqrt{1-2\lambda a}} = 1 + a^2\lambda^2 + o(\lambda^2), \; e^{\lambda^2 \nu^2/2} = 1 + \frac{1}{2}\lambda^2 \nu^2 + o(\lambda^2).
    \end{equation*}
    For any $v \geq \sqrt{2}a$ the last inequality would hold as $\frac{1}{2}\lambda^2 \nu^2 \geq a^2\lambda^2$, we pick $\nu = 2a$ and $b=4a$.
\end{proof}

\subsection{Proof of Proposition \ref{prop:bound_fixed}}\label{appendix:bound_fixed}
\begin{proposition*}[\ref{prop:bound_fixed}]
Consider a fixed-in-advance sequence of model parameters $\theta_t$ and mini-batches.
For $0 < \alpha < 1$, for each dimension $i$, we have $\sP[\lvert
 \hat{v}_t^{\textnormal{corr}} - \hat{v}_t^c \rvert_i \geq \xi ] \leq \alpha$ with:
\begin{align*}
    \xi \geq
    \begin{cases}
        (\frac{1-\beta_2}{1-\beta_2^t}) \sqrt{\ln{(1/\frac{\alpha}{2})}(2v^2)}
        & 0 \leq \frac{\xi(1-\beta_2^t)}{1-\beta_2}  \leq \frac{\nu^2}{b} \\
        (\frac{1-\beta_2}{1-\beta_2^t}) \ln{(1/\frac{\alpha}{2})}2b
        & \frac{\xi(1-\beta_2^t)}{1-\beta_2}    \geq \frac{\nu^2}{b},
    \end{cases}
\end{align*}
where $\nu = (\frac{4\sigma^2C^2}{B^2})\sqrt{\frac{1-\beta_2^{2t}}{1-\beta_2^2}}, b=\frac{4\sigma^2C^2}{B^2}$.
\end{proposition*}

\begin{proof}
Let $Z_{\tau} \sim \cN(0, (1/B^2)\sigma^2 C^2)$ be the independent DP noise drawn from a Normal distribution,
$Z_{\tau} \sim \cN(0, (1/B^2)\sigma^2 C^2) \implies X_{\tau} = \frac{Z_{\tau}}{(1/B)\sigma C} \sim \cN(0,1)$.
Let $Z_t$ be the random variable of the step $t$ moving average of independent squared noise, $Z_t \coloneqq (\frac{1-\beta_2}{1-\beta_2^t})\sum_{\tau=1}^{t} \beta_2^{t-\tau} Z_\tau^2$.
%Assuming fixed-in-advance $\theta_t$, we derive a bound for how the observed bias ($Z_t$) deviates from its expected value ($\Phi$), or how $\hat{v}_t^{\textnormal{corr}}$ (without the numerical stability constant) deviates from $\hat{v}_t^c$.

By Lemma \ref{lemma:zk2_sub_exponential}, let $a=\beta_{2}^{t-\tau}(1/B^2)\sigma^2C^2$ be a temporary constant, $\beta_2^{t-\tau}Z_\tau^2$ is sub-exponential with $\nu_{\tau}=2a, b_{\tau}=4a$.
By the additive property of sub-exponential and since $\{Z_\tau^2, \tau={1, ..., t}\}$ are independent, $\sum_{\tau=1}^{t}\beta_2^{t-\tau}Z_{\tau}^2$ is sub-exponential with
\begin{gather*}
    \nu^{*} =\sqrt{\sum_{\tau=1}^{t} \nu_{\tau}^2} = \sqrt{\sum_{\tau=1}^{t} 4a^2} = \bigg(\frac{2\sigma^2C^2}{B^2}\bigg)\sqrt{\frac{1-\beta_2^{2t}}{1-\beta_2^2}}, \\
    b^{*} = \max_{\tau=1,...,t}b_\tau = \frac{4\sigma^2C^2}{B^2}.
\end{gather*}
By Proposition 2.9 in \citet{wainwright2019high},
\begin{align*}
    \sP&[\lvert Z_t - \E{[Z_t]} \rvert  \geq (\frac{1-\beta_2}{1-\beta_2^t}) \delta] \\
    &\leq
    \begin{cases}
      2\exp{ \bigl\{ -\delta^2/(2\nu^{*2}) \bigl\} } & 0 \leq \delta \leq \frac{\nu^{*2}}{b^{*}} \\
      2\exp{ \bigl\{ -\delta/(2b^{*}) \bigl\} } & \delta > \frac{\nu^{*2}}{b^{*}}.
   \end{cases}
\end{align*}
For any tolerance level $\alpha < 1$, and with $\xi = (\frac{1-\beta_2}{1-\beta_2^t}) \delta$, we have $\sP[\lvert
 \hat{v}_t^{\textnormal{corr}} - \hat{v}_t^c \rvert_i \geq \xi ] \leq \alpha$ with:
\begin{align*}
    \xi \geq
    \begin{cases}
        (\frac{1-\beta_2}{1-\beta_2^t}) \sqrt{(-\ln{\frac{\alpha}{2}})(2\nu^{*2})}
        & 0 \leq \frac{\xi(1-\beta_2^t)}{1-\beta_2}  \leq \frac{\nu^{*2}}{b^{*}} \\
        (\frac{1-\beta_2}{1-\beta_2^t}) (-\ln{\frac{\alpha}{2}})2b^{*}
        & \frac{\xi(1-\beta_2^t)}{1-\beta_2}    \geq \frac{\nu^{*2}}{b^{*}}.
    \end{cases}
\end{align*}
\end{proof}

\subsection{Proof of Proposition \ref{prop:bound_martingale}}\label{appendix:bound_martingale}

We now provide high probability bounds for the error incurred when estimating $\hat{v}_t$.
In the DP optimization procedures we consider, the sequence of batches is drawn independently from everything else. We can thus understand the procedure as first drawing a sequence of batches, and then proceeding with DP optimization on this sequence. Without loss of generality, we fix the sequence of batches, and analyze the behavior of $\hat{v}_t$ on a fixed (unknown) sequence of batches $b_t$. The main reason is to ensure that the source of randomness in $\theta_t$ comes from the DP noise draw only, i.e. given step $t$ and a sequence of realized sample draw of DP noise, the parameters $\theta_t$ and thus $\overline{g}_t = (1/B)\sum_{n \in b_t} g_{n}(\theta_t) / {\max{(1, \lVert g_n(\theta_t) \rVert_{2}/C})}$ are deterministic. Then, we have that:

\begin{proposition*}[\ref{prop:bound_martingale}]
    % For $0 < \alpha < 1$, for each dimension $i$, we have $\sP[\lvert \hat{v}_t^p - \E{[\hat{v}_t^p]} \rvert_i \geq (\frac{1-\beta_2}{1-\beta_2^t})\xi] \leq \alpha$ with:
        For $0 < \alpha < 1$, for each dimension $i$, we have $\sP[\lvert \hat{v}_t^p - \E{[\hat{v}_t^p]} \rvert_i \geq \xi] \leq \alpha$ with:
%     \begin{align*}
%     \xi \geq
%     \begin{cases}
%         \sqrt{(-\ln{\frac{\alpha}{2}})(2v^2)}
%         % = \sqrt{(-\ln{\frac{\alpha}{2}})(\frac{8\sigma^2C^4}{B^2})(\frac{\sigma^2 C^2}{B^2} + 2)(\frac{1-\beta_2^t}{1-\beta_2})}
%         & 0 \leq \xi \leq \frac{\nu^2}{b^{*}} \\
%         (-\ln{\frac{\alpha}{2}})2b^{*}
%         %(-\ln{\frac{\alpha}{2}})(\frac{8\sigma^2C^2}{B^2})
%         & \xi \geq \frac{\nu^2}{b^{*}},
%     \end{cases}
% \end{align*}
    \begin{align*}
    \xi \geq
    \begin{cases}
        (\frac{1-\beta_2}{1-\beta_2^t}) \sqrt{\ln{(1/\frac{\alpha}{2})}(2v^2)}
        % = \sqrt{(-\ln{\frac{\alpha}{2}})(\frac{8\sigma^2C^4}{B^2})(\frac{\sigma^2 C^2}{B^2} + 2)(\frac{1-\beta_2^t}{1-\beta_2})}
        & 0 \leq \frac{\xi(1-\beta_2^t)}{1-\beta_2}  \leq \frac{\nu^2}{b} \\
        (\frac{1-\beta_2}{1-\beta_2^t}) \ln{(1/\frac{\alpha}{2})}2b
        %(-\ln{\frac{\alpha}{2}})(\frac{8\sigma^2C^2}{B^2})
        & \frac{\xi(1-\beta_2^t)}{1-\beta_2}  \geq \frac{\nu^2}{b},
    \end{cases}
\end{align*}
where $\nu = 2\sqrt{\frac{1-\beta_2^{2t}}{1-\beta_2^2}}( \frac{\sigma^2C^2}{B^2} + \frac{\sigma C^2}{B} )$, $b = \frac{4\sigma^2C^2}{B^2}$.
\end{proposition*}

\begin{proof}
Remember that $\hat{v}_t^{p} = (\frac{1-\beta_2}{1-\beta_2^t})\sum_{\tau=1}^{t} \beta_{2}^{t-\tau} {\Tilde{g}_\tau}^{2} = (\frac{1-\beta_2}{1-\beta_2^t})\sum_{\tau=1}^{t} \beta_{2}^{t-\tau} \big(\bar{g}_\tau + z_\tau\big)^{2}$.
% $v_t^{p} = (1-\beta_2)\sum_{\tau=1}^{t} \beta_{2}^{t-\tau} {\Tilde{g}_\tau}^{2} = (1-\beta_2)\sum_{\tau=1}^{t} \beta_{2}^{t-\tau} \big(\bar{g}_\tau + z_\tau\big)^{2}$.
We use the Doob martingale construction to analyze $\hat{v}_t^{p}$'s deviation from its mean. The sequence $Z \triangleq \{Z_\tau\}_{\tau=1}^t$ is a sequence of independent random variables (the DP noise draws). Define $f(Z) = \sum_{\tau=1}^{t} \beta_{2}^{t-\tau} \big(\bar{g}_\tau + Z_\tau\big)^{2}$,
%\ml{it might be cleaner to define it as $v_t - \Phi$ depending on what we want to show, not sure...}
$Y_0 = \E[f(Z)]$, $Y_t = f(Z)$, and
% $Y_k = \E[f(Z) | z_1, \ldots, z_k]$
$Y_k = \E[f(Z) | Z_1, \ldots, Z_k]$. $D_k \triangleq Y_k - Y_{k-1}$ is a martingale difference sequence w.r.t.
% $Z_k \triangleq \{z_\tau\}_{\tau=1}^k$ (see Example 2.8 in \cite{wainwright2019high} \ml{all cites of this taken from \url{https://www.stat.berkeley.edu/~mjwain/stat210b/Chap2_TailBounds_Jan22_2015.pdf}, to check on the real book---it's in my office...}).
$Z_k \triangleq \{Z_\tau\}_{\tau=1}^k$. Let $z_1, \ldots, z_{k}$ be a sequence of realized samples of $Z_{k}$.
% Given the definition, we have that \ml{this needs better notations to distinguish the RV and draw}:
% \begin{align*}
% D_k &= \E[f(Z) | z_1, \ldots, z_k] - \E[f(Z) | z_1, \ldots, z_{k-1}] \\
%     &= \E[\E[f(Z) | z_k] - f(Z) | z_1, \ldots, z_{k-1}] \\
%     &= \E\Big[\beta_{2}^{t-k} \big(\bar{g}_k + z_k\big)^{2} - \beta_{2}^{t-k} \big(\bar{g}_k + Z_k\big)^{2} \Big] \\
%     &= \beta_{2}^{t-k}\big(2\bar{g}_k z_k + z_k^2 - \V(Z_k)\big)
% \end{align*}
Given the definition, we have that:
\begin{align*}
D_k &= \E[f(Z) | Z_1, \ldots, Z_k] - \E[f(Z) | Z_1, \ldots, Z_{k-1}] \\
    &= \bigg(\sum_{\tau=1}^{k} \beta_2^{t-\tau}(\Bar{g}_\tau + z_\tau)^2 \\ & \ \ \ + \E\bigg[\sum_{\tau=k+1}^{t} \beta_2^{t-\tau}(\Bar{g}_\tau + Z_\tau)^2 \Big\vert Z_1, \ldots, Z_k \bigg] \bigg) \\
    & \ \ \ - \bigg(\sum_{\tau=1}^{k-1} \beta_2^{t-\tau}(\Bar{g}_\tau + z_\tau)^2 \\ & \ \ \ + \E\bigg[\sum_{\tau=k}^{t} \beta_2^{t-\tau}(\Bar{g}_\tau + Z_\tau)^2 \Big\vert Z_1, \ldots, Z_{k-1} \bigg] \bigg) \\
    &= \beta_{2}^{t-k} \big(\bar{g}_k + z_k\big)^{2} - \beta_2^{t-k}\E\Big[\big(\bar{g}_k + Z_k\big)^{2} \big\vert Z_1, \ldots, Z_{k-1}\Big] \\
    % &= \beta_{2}^{t-k}\big((\bar{g}_k^2- \E[\bar{g}_k^2]) + z_k^2 + 2\bar{g}_k z_k - \V(Z_k)\big).
    &= \beta_{2}^{t-k}\big(z_k^2 + 2\bar{g}_k z_k - \V(Z_k)\big),
\end{align*}
where $\E[\bar{g}_k^2 | Z_1, \ldots, Z_{k-1}] = \bar{g}_k^2$ since $\bar{g}_k$ is deterministic given previous DP noise draws and a fixed batch $b_k$.
Next we show that $D_k$ is sub-exponential, i.e. $(*) = \E{\big[ e^{\lambda (D_k - \E{[D_k]})} | Z_1, \ldots, Z_{k-1}\big]} \leq e^{\lambda^2 v_k^2/2}$ for some $v_k$.
By Lemma \ref{lemma:zk2_sub_exponential}, let $a=\beta_{2}^{t-k}(1/B^2)\sigma^2C^2$ be a temporary constant, $\beta_2^{t-\tau}Z_\tau^2$ is sub-exponential with $\nu=2a, b=4a$.
For the second part, let $X_k = 2 \beta_{2}^{t-k}\bar{g}_k Z_k \sim (0, (2 \beta_{2}^{t-k}\bar{g}_k)^2 (1/B^2) \sigma^2 C^2)$, therefore $X_k$ is sub-exponential with $\nu = 2 \beta_{2}^{t-k}\bar{g}_k (1/B) \sigma C \leq 2 \beta_{2}^{t-k} (1/B) \sigma C^2$ (since $\Bar{g}_k$ is bounded by $C$ by clipping) and $b=0$, i.e.
\begin{gather*}
    \E{\Big[ \exp{\bigl\{ 2\lambda \beta_{2}^{t-k}\bar{g}_k Z_k \bigl\} }\Big]} \leq e^{\lambda^2 \nu^2 / 2}, \\
    \nu = 2 \beta_{2}^{t-k}(1/B)\sigma C^2, \; \forall \, |\lambda| \leq \infty.
\end{gather*}
By the additive property of sub-exponential we get $D_k$ is sub-exponential with $\nu_k = \nu_{1k} + \nu_{2k}$ where $\nu_{1k} = 2\beta_{2}^{t-k}(1/B^2)\sigma^2C^2$, $\nu_{2k} = 2 \beta_{2}^{t-k} (1/B)\sigma C^2$, $b = 4a = 4\beta_{2}^{t-k}(1/B^2)\sigma^2C^2$, $|\lambda| < 1/b$.
%
By Theorem 2.3 in \cite{wainwright2019high}, Chapter 2 , we have that $\sum_{k=1}^{t} D_k$ is sub-exponential with
\begin{gather*}
    \nu^{*} = \sqrt{\sum_{k=1}^{t} (\nu_{1k} + \nu_{2k})^2} = 2\sqrt{\frac{1-\beta_2^{2t}}{1-\beta_2^2}}\bigg( \frac{\sigma^2C^2}{B^2} + \frac{\sigma C^2}{B} \bigg), \\
    b^{*} = \max_{k=1, \ldots, t} b_k = \frac{4\sigma^2C^2}{B^2}.
\end{gather*}
%\ml{any reason you do $\beta-1$ in the ratio instead of flipping both to $1-\beta$? It confused me a bit}
For all $\delta \geq 0$,
\begin{align*}
    \sP[\lvert \sum_{k=1}^{t} D_k &\rvert \geq \delta] = \sP[\lvert f(Z) - \E{[f(Z)]} \rvert \geq \delta] \\ &\leq
    \begin{cases}
      2\exp{ \bigl\{ -\delta^2/(2\nu^{*2}) \bigl\} } & 0 \leq \delta \leq \frac{\nu^{*2}}{b^{*}} \\
      2\exp{ \bigl\{ -\delta/(2b^{*}) \bigl\} } & \delta > \frac{\nu^{*2}}{b^{*}},
   \end{cases}
\end{align*}
%
i.e. For $\hat{v}_t^p = (\frac{1-\beta_2}{1-\beta_2^t})f(Z)$ and $\E{[\hat{v}_t^p]} = (\frac{1-\beta_2}{1-\beta_2^t})\E{[f(Z)]}$, $\forall$ tolerance level $\alpha \leq 1$, and with $\xi = (\frac{1-\beta_2}{1-\beta_2^t}) \delta$, we have $\sP[\lvert \hat{v}_t^p - \E{[\hat{v}_t^p]} \rvert_i \geq \xi] \leq \alpha$ with:
\begin{align*}
    \xi \geq
    \begin{cases}
        (\frac{1-\beta_2}{1-\beta_2^t}) \sqrt{(-\ln{\frac{\alpha}{2}})(2\nu^{*2})}
        % = \sqrt{(-\ln{\frac{\alpha}{2}})(\frac{8\sigma^2C^4}{B^2})(\frac{\sigma^2 C^2}{B^2} + 2)(\frac{1-\beta_2^t}{1-\beta_2})}
        & 0 \leq \frac{\xi(1-\beta_2^t)}{1-\beta_2}  \leq \frac{\nu^{*2}}{b^{*}} \\
        (\frac{1-\beta_2}{1-\beta_2^t}) (-\ln{\frac{\alpha}{2}})2b^{*}
        %(-\ln{\frac{\alpha}{2}})(\frac{8\sigma^2C^2}{B^2})
        & \frac{\xi(1-\beta_2^t)}{1-\beta_2}  \geq \frac{\nu^{*2}}{b^{*}},
    \end{cases}
\end{align*}
\end{proof}

\subsection{Numerical Analysis}\label{appendix:num-analysis}
We compute the numerical values of the bound in Proposition \ref{prop:bound_fixed} and \ref{prop:bound_martingale} under different tolerance levels $\alpha$. We also empirically measure the deviance of the observed DP bias to its expected value $\Phi$ by measuring the absolute difference between $\hat{v}_t^p$ and $\hat{v}_t^c + \Phi$. Table \ref{tab:num_bound_fixed_grad}, \ref{tab:num_bound} and \ref{tab:empirical_bound} summarizes the corresponding values at different step $t$ on the SNLI dataset. We observe that the empirical values in Table \ref{tab:empirical_bound} are smaller comparing to $\Phi$, suggesting that the observed DP bias are quite concentrated around its mean, and subtracting $\Phi$ from $\hat{v}_t^p$ should be relatively accurate. We also observe that the value of the bound in Proposition \ref{prop:bound_fixed} are much closer to the empirical values than in Proposition \ref{prop:bound_martingale}. It suggests that the bound derived under the fixed sequence of model parameters (Proposition \ref{prop:bound_fixed}) might be empirically more practical than the bound derived under the martingale assumptions (Proposition \ref{prop:bound_martingale}).

\begin{table*}[tb]
\centering
\renewcommand{\arraystretch}{1.1}
\resizebox{0.80\textwidth}{!}{%
\begin{tabular}{c|c|cccc}
\hline \hline
\textbf{Experiment} & \textbf{Quantity} & \bm{$t=10$} & \bm{$t=100$} & \bm{$t=1000$} & \bm{$t=10000$} \\ \hline
\multirow{3}{*}{\begin{tabular}[c]{@{}c@{}}$B=256, C=0.1$, \\ $\sigma=0.4, \beta_2=0.999$\end{tabular}}
 & $\sP[ \lvert \hat{v}_t^{\textnormal{corr}} - \hat{v}_t^c \rvert \geq ?] \leq 0.01$ & 1.005e-07 & 3.180e-08  & 1.046e-08 & 7.110e-09 \\
 & $\sP[ \lvert \hat{v}_t^{\textnormal{corr}} - \hat{v}_t^c \rvert \geq ?] \leq 0.05$ & 8.388e-08 & 2.654e-08 & 8.725e-09 & 5.933e-09  \\
 & $\sP[\lvert \hat{v}_t^{\textnormal{corr}} - \hat{v}_t^c \rvert \geq ?] \leq 0.10$ & 7.559e-08 & 2.391e-08 & 7.863e-09 & 5.347e-09 \\ \hline
\multirow{3}{*}{\begin{tabular}[c]{@{}c@{}}$B=256, C=1.0$, \\ $\sigma=0.4, \beta_2=0.999$\end{tabular}}
 & $\sP[\lvert \hat{v}_t^{\textnormal{corr}} - \hat{v}_t^c \rvert \geq ?] \leq 0.01$ & 1.005e-06 & 3.180e-06  & 1.046e-06 & 7.110e-07  \\
 & $\sP[\lvert \hat{v}_t^{\textnormal{corr}} - \hat{v}_t^c \rvert \geq ?] \leq 0.05$ & 8.388e-06 & 2.654e-06 & 8.725e-07 & 5.933e-07 \\
 & $\sP[\lvert \hat{v}_t^{\textnormal{corr}} - \hat{v}_t^c \rvert \geq ?] \leq 0.10$ & 7.559e-06 & 2.391e-06 & 7.863e-07 & 5.347e-07  \\ \hline \hline
\end{tabular}%
}
\caption{Numerical values for relevant quantities of the bound as in Proposition \ref{prop:bound_fixed}.}
\label{tab:num_bound_fixed_grad}
\end{table*}

\begin{table*}[tb]
\centering
\renewcommand{\arraystretch}{1.1}
\resizebox{0.80\textwidth}{!}{%
\begin{tabular}{c|c|cccc}
\hline \hline
\textbf{Experiment} & \textbf{Quantity} & \bm{$t=10$} & \bm{$t=100$} & \bm{$t=1000$} & \bm{$t=10000$} \\ \hline
\multirow{3}{*}{\begin{tabular}[c]{@{}c@{}}$B=256, C=0.1$, \\ $\sigma=0.4, \beta_2=0.999$\end{tabular}}
 & $\sP[\lvert \hat{v}_t^p - \E{[\hat{v}_t^p]} \rvert \geq ?] \leq 0.01$ & 3.222e-05 & 1.019e-05 & 3.351e-06 & 2.279e-06 \\
 & $\sP[\lvert \hat{v}_t^p - \E{[\hat{v}_t^p]} \rvert \geq ?] \leq 0.05$ & 2.688e-05 & 8.505e-06 & 2.797e-06 & 1.902e-06 \\
 & $\sP[\lvert \hat{v}_t^p - \E{[\hat{v}_t^p]} \rvert \geq ?] \leq 0.10$ & 2.423e-05 & 7.664e-06 & 2.520e-06 & 1.714e-06 \\ \hline
\multirow{3}{*}{\begin{tabular}[c]{@{}c@{}}$B=256, C=1.0$, \\ $\sigma=0.4, \beta_2=0.999$\end{tabular}}
 & $\sP[\lvert v_t^p - \E{[v_t^p]} \rvert \geq ?] \leq 0.01$ & 3.222e-03 & 1.019e-03 & 3.351e-04 & 2.279e-04 \\
 & $\sP[\lvert v_t^p - \E{[v_t^p]} \rvert \geq ?] \leq 0.05$ & 2.688e-03 & 8.505e-04 & 2.797e-04 & 1.902e-04 \\
 & $\sP[\lvert v_t^p - \E{[v_t^p]} \rvert \geq ?] \leq 0.10$ & 2.423e-03 & 7.664e-04 & 2.520e-04 & 1.714e-04 \\ \hline \hline
\end{tabular}%
}
\caption{Numerical values for relevant quantities of the bound as in Proposition \ref{prop:bound_martingale}.}
\label{tab:num_bound}
\end{table*}

\begin{table*}[tb]
\centering
\renewcommand{\arraystretch}{1.1}
\resizebox{0.80\textwidth}{!}{%
\begin{tabular}{c|c|cccc}
\hline \hline
\textbf{Experiment} & \textbf{Quantity} & \bm{$t=10$} & \bm{$t=100$} & \bm{$t=1000$} & \bm{$t=10000$} \\ \hline
\multirow{3}{*}{\begin{tabular}[c]{@{}c@{}}$B=256, C=0.1$, \\ $\sigma=0.4, \beta_2=0.999$ \end{tabular}}
 & $\sP[|v_t^p - \E{[v_t^p]}| \geq ?] = 0.01$ & 3.266e-08 & 9.004e-09 & 2.940e-09 & 2.052e-09 \\
 & $\sP[|v_t^p - \E{[v_t^p]}| \geq ?] = 0.05$ & 2.063e-08 & 6.181e-09 & 2.107e-09 & 1.492e-09 \\
 & $\sP[|v_t^p - \E{[v_t^p]}| \geq ?] = 0.10$ & 1.493e-08 & 4.746e-09 & 1.670e-09 & 1.197e-09 \\ \hline
\multirow{3}{*}{\begin{tabular}[c]{@{}c@{}}$B=256, C=1.0$, \\ $\sigma=0.4, \beta_2=0.999$ \end{tabular}}
 & $\sP[|v_t^p - \E{[v_t^p]}| \geq ?] = 0.01$ & 3.266e-06 & 9.004e-07 & 2.940e-07 & 2.052e-07 \\
 & $\sP[|v_t^p - \E{[v_t^p]}| \geq ?] = 0.05$ & 2.064e-06 & 6.181e-07 & 2.107e-07 & 1.492e-07 \\
 & $\sP[|v_t^p - \E{[v_t^p]}| \geq ?] = 0.10$ & 1.493e-06 & 4.746e-07 & 1.670e-07 & 1.197e-07 \\ \hline \hline
\end{tabular}%
}
\caption{Empirically measured values for deviance of the observed DP bias from $\Phi$.}
\label{tab:empirical_bound}
\end{table*}

\section{More Results}
\label{apdix:more_res}

Figure \ref{fig:exp1} shows the mean $\pm$ standard deviation of test accuracy over the privacy budget ($\epsilon$-DP) on three datasets, repeated over 5 runs with different random seeds.
On SNLI, DP-AdamBC performs better than DP-Adam: the accuracy improves from 52.63\% to 56.08\% ($3.5$ percentage points). Both perform much better than DP-SGD (51.04\% on SNLI).
On QNLI, we observe a similar behaviour which the accuracy improves from 61.23\% to 62.83\% ($1.6$ percentage points) with the bias correction, and both perform better than DP-SGD (58.29\%).
For CIFAR10, on which Adam often performs worse than SGD in non-private settings, DP-Adam and DP-AdamBC (62.24\% vs 63.43\% accuracy) performs similarly and are both worse than DP-SGD (65.30\% accuracy).
On ogbn-arxiv, DP-Adam and DP-AdamBC perform similarly (54.02\% vs 53.81\% accuracy) and are outperformed by DP-SGD (54.20\%).

\paragraph{DP-Adam has similar performance to DP-SGDM.}
We provide more details about the comparison between DP-Adam and DP-SGDM with a converted learning rate schedule as in Equation \ref{eq:lr_convert}. Figure \ref{fig:exp_rel_sgdm_more} shows the train loss, test loss and test accuracy between the two algorithms on SNLI (top row) and two experiment setups with different $\Phi$ on CIFAR10 (middle and bottom row). The solid line and the shaded area show the mean and standard deviation over 5 repeated runs. For SNLI (top row), it was run with $\eta=0.01$ for DP-Adam and $\eta \approx 6.4$ for DP-SGDM with $B=256, C=0.1, \sigma=0.4, \beta_1=0.9$. For CIFAR10 with relatively small $\Phi$ (middle row), it was run with $\eta=0.001$ for DP-Adam and $\eta\approx0.2048$ for DP-SGDM with $B=2048, C=1.0, \sigma=1.0, \beta_1=0.9$; for CIFAR10 with relatively large $\Phi$ (bottom row), it was run with $\eta=0.001$ for DP-Adam and $\eta \approx0.0256$ for DP-SGDM with $B=256, C=1.0, \sigma=1.0, \beta_1=0.9$. We observe that the performances are close between the two algorithms in train loss, test loss and test accuracy. Some discrepancy still exists since the observed value of DP bias is concentrated around but not exactly equal to $\Phi$. When $\Phi$ is relatively large as in the second case on CIFAR10, $\Phi$ is more likely to dominate the denominator of DP-Adam's update, and we observe an even closer behaviour between DP-Adam and DP-SGDM in all three aspects. There could also be possible different generalization behaviour between the two algorithm, but the training behaviour is almost identical.

\begin{figure*}[t]
\centering
\includegraphics[width=0.9\linewidth]{figs/multi_eps_lineplot.pdf}
\caption{(From left to right) Comparing the performance of DP-Adam, DP-AdamBC and DP-SGD on QNLI and SNLI dataset (nlp), CIFAR10 (images) and obgn-arxiv (node classification) at different target pribacy budget ($\epsilon$). Each result is tuned separately. We report the mean (standard deviation) over 5 runs for the best parameters.}
\label{fig:res_multi_eps}
\end{figure*}

\begin{figure*}[tb]
\centering
\includegraphics[width=0.7\linewidth]{figs/exp_appendix_sgdm_story_error_bars3.pdf}
\caption{Compare train loss, test loss and test accuracy between DP-Adam and the DP-SGDM with the converted learning rate schedule as in Equation \ref{eq:lr_convert} on SNLI (top row) and CIFAR10 with relatively small $\Phi$ (middle row) and large $\Phi$ (bottom row) respectively.
}
\label{fig:exp_rel_sgdm_more}
\end{figure*}

\paragraph{Additional Experiments.}
\label{apdix:additional_exp}
We repeat the comparison between DP-AdamBC, DP-Adam and DP-SGD on SNLI dataset with target $\epsilon=3$. The hyperparameters are $lr_{\textnormal{DP-Adam}} = 0.005$, $\gamma'_{\textnormal{DP-Adam}} = 1\textnormal{e-8}$, $lr_{\textnormal{DP-AdamBC}} = 0.005$ , $\gamma_{\textnormal{DP-AdamBC}} = 5\textnormal{e-9}$, $B=256, C=0.1, \sigma=0.5$. For each algorithm, we report the mean and standard deviation over 5 repeated runs with different seeds. Figure \ref{fig:additional_exp}(Left) shows that DP-AdamBC is approximately $2.6\%$ better in final mean test accuracy than DP-Adam with this privacy budget. The final mean(standard deviation) test accuracy for DP-AdamBC and DP-Adam are $50.1\%(1.6\%)$, $47.5\%(1.8\%)$ respectively.

We conduct additional experiments to compare the performances between DP-AdamBC, DP-Adam, and DP-SGD for smaller privacy budgets for all datasets. For CIFAR10, we tune the relevant hyperparameters (learning rate, $\gamma$, $\gamma'$, C, B, and number of steps) independently for each experiment. For SNLI and QNLI, we tune the same learning rate except for $B=256$ is fixed at its maximum capacity allowed on our machine.
%
For ogbn-arxiv, we tune the model hyperparameters for DP-Adam/DP-AdamBC and DP-SGD over 50 runs and then tune the optimizer hyperparameters for each algorithm (with fixed model hyperparameters). For DP-Adam/DP-AdamBC, we use noise multiplier $\lambda = 2.7$, maximum degree $K = 6$, batch size $B = 10,000$, clipping norm $C = 0.0003$, one encoder layer, and two decoding layers. For DP-SGD, we use noise multiplier $\lambda = 2.3$, maximum degree $K = 4$, batch size $B = 2000$, clipping norm $C = 0.0015$, one encoder layer, and one decoder layer. The optimizer hyperparameters are $lr_{\textnormal{DP-Adam}}~=~0.1$, $\gamma'_{\textnormal{DP-Adam}}~=~1\textnormal{e-8}$, $lr_{\textnormal{DP-AdamBC}}~=~0.0001$, $\gamma_{\textnormal{DP-AdamBC}}~=~4.3\textnormal{e-9}$, $lr_{\textnormal{DP-SGD}}~=~0.24$. We report the mean and standard deviation of 5 repeated runs with different seeds for the tuned algorithms. Table \ref{tab:res_multi_eps} shows that DP-AdamBC has the best mean test accuracy at this privacy budget.

To examine the generalizability of the results we test the algorithm on larger dataset and model. We repeat the comparison between DP-AdamBC and DP-Adam on SST2 dataset and Bert-Large model, with the last encoder block and the classifier head randomly initialized and trained. The hyperparameters are $lr_{\textnormal{DP-Adam}} = 0.005$, $\gamma'_{\textnormal{DP-Adam}} = 1\textnormal{e-8}$, $lr_{\textnormal{DP-AdamBC}} = 0.003$ , $\gamma_{\textnormal{DP-AdamBC}} = 3\textnormal{e-9}$, $B=256, C=0.1, \sigma=0.4$. Figure \ref{fig:additional_exp}(Right) shows that DP-AdamBC has around 2.4\% advantage in final mean test accuracy than DP-Adam.

\begin{figure*}[htb]
\centering
\includegraphics[width=\linewidth]{figs/main_res_reorganized_wide_modified_xaxis.pdf}
\caption{(From left to right) Comparing the performance of DP-Adam, DP-AdamBC and DP-SGD on QNLI and SNLI dataset (nlp), CIFAR10 (images) and obgn-arxiv (node classification). At the end of training $\epsilon\textnormal{-DP} \approx 7$ for CIFAR10, QNLI and SNLI and $\epsilon\textnormal{-DP} \approx 12$ for obgn-arxiv. Each optimizer is tuned separately. The x-axis is the step over a single training trajectory converted to privacy budget $\epsilon$ to make results comparable for different optimizers.}
\label{fig:exp1}
\end{figure*}

\begin{figure*}[htb]
     \centering
     \begin{subfigure}[b]{0.35\textwidth}
         \centering
         \includegraphics[width=\textwidth]{figs/addexp_eps3_snli.pdf}
     \end{subfigure}
     % \hfill
     \begin{subfigure}[b]{0.35\textwidth}
         \centering
         \includegraphics[width=\textwidth]{figs/bert_large_repeated.pdf}
     \end{subfigure}
    \caption{(\textbf{Left: }) Comparison between DP-AdamBC and DP-Adam when target $\epsilon=3$ on the SNLI dataset. DP-AdamBC shows $2.6\%$ advantage in final mean test accuracy. The final mean(standard deviation) test accurary for DP-AdamBC and DP-Adam are $50.1\%(1.6\%)$, $47.5\%(1.8\%)$ respectively. (\textbf{Right: }) Comparison between DP-AdamBC and DP-Adam on the SST2 dataset with Bert-Large. DP-AdamBC shows 2.4\% advantage in final test accuracy compared to DP-Adam (68.80\% vs 66.4\%).}
    \label{fig:additional_exp}
\end{figure*}

\paragraph{Comparison to DP-Adam variant.}
\label{apdix:compare_dp2}

We performed an empirical comparison of $\textrm{DP}^2$ with DP-AdamBC, and discuss the differences between the two algorithms below (we refer to Algorithm 1 in \citet{lidp2} as $\textrm{DP}^2$). There are three key differences between $\textrm{DP}^2$ and DP-AdamBC:
\begin{enumerate}
    \item Adam does not use pre-conditioned gradients in its update, since the moments are estimated from non-scaled gradients, whereas RMSprop (the base for $\textrm{DP}^2$ in \citet{lidp2}) uses scaled (thus pre-conditioned) gradients in the update. Figure 10 in \citet{lidp2} shows that the gains obeseved in $\textrm{DP}^2$ come in large part from reducing the amount of clipping and noising by using pre-conditioned gradients, but such advantage cannot directly transfer to Adam.
    \item The pre-conditioning term ($D_t$) is computed on much larger batches of gradients, since it is accumulated over multiple iterations. This means that $D_t$ is computed on a different distribution than the actual $t$-step gradients. For Adam, it would imply that the first moments (expectation of $t$-step gradient) and second moments (variance of $t$-step gradient) are estimated with different sampling distributions which could break Adam's sign descent behaviour.
    \item $\textrm{DP}^2$ does not use momentum on the gradients (as the first moment in Adam) potentially because it alternates between two optimizers. Momentum is typically important on tasks for which Adam works well \citep{kunstner2023heavytailed}.
\end{enumerate}

Empirically, \citet{lidp2} evaluates $\textrm{DP}^2$ on linear models and matrix factorization, and not on deep learning tasks. Figure \ref{fig:compare_dp2_snli_only} (Left) and \ref{fig:compare_dp2_cifar_only} (Left)  shows the comparison between DP2-RMSProp, DP-AdamBC and DP-SGD on SNLI with Bert-base and CIFAR10 with CNN respectively. DP2-RMSProp introduces several new hyperparameters (learning rates, clipping thresholds and delay parameters in both phases of SGD and RMSProp) whereas DP-AdamBC adds no additional hyperparameters compared to DP-Adam. We tuned $\textrm{DP}^2$ with grid search over: learning rate-\{0.001, 0.01, 0.1, 1.0, 3.0, 5.0, 7.0, 10.0\} for both datasets in both phases of $\textrm{DP}^2$, (CIFAR10) $s1=s2=s$ as suggested in \citet{lidp2}, s-\{25, 65, 130, 250\} (roughly 4, 10, 20, 40 out of 50 epochs), and C-\{0.1, 1, 3\} in both phases of $\textrm{DP}^2$; (SNLI) s-\{1000, 2000, 4500\} (roughly 0.5, 1, 2 out of 3 epochs) and C-\{0.01, 0.1, 1\} in both phases of $\textrm{DP}^2$, we fixed batch size $B$ and noise multiplier $\sigma$ to be the same as with the other three algorithms. Figure \ref{fig:compare_dp2_snli_only} (Left) and \ref{fig:compare_dp2_cifar_only} (Left) shows the mean and standard deviation of the test accuracy over 5 runs with on the two datasets. We observe that $\textrm{DP}^2$ first follows DP-SGD (since the first steps use this optimizer), and then struggles to converge on deep learning tasks, leading to poor performance. Indeed switching between two optimizers seem to make $\textrm{DP}^2$ unstable: Figure \ref{fig:compare_dp2_snli_only} (Right) and \ref{fig:compare_dp2_cifar_only} (Right) shows $\textrm{DP}^2$ on these two tasks with different $s$ (switching frequency). We observe the performance either has large turbulence or drops significantly when switching optimizers during training.

% \begin{figure*}
%     \centering
%     \includegraphics[width=0.7\textwidth]{figs/dp2_cifar_snli_new.pdf}
%     \caption{Comparison between DP2RMSProp, DP-AdamBC, DP-Adam and DP-SGD on CIFAR10 with CNN (left) and SNLI with Bert-base (right).}
%     \label{fig:compare_dp2_res}
% \end{figure*}

% \begin{figure*}
%     \centering
%     \includegraphics[width=0.7\textwidth]{figs/dp2_cifar_snli_effect_of_s.pdf}
%     \caption{The performance of DP2RMSProp with different phase switching frequency $s$ on CIFAR10 with CNN (left) and SNLI with Bert-base (right).}
%     \label{fig:compare_dp2_effect_s}
% \end{figure*}

\begin{figure*}[htb]
    \centering
    \includegraphics[width=0.7\textwidth]{figs/dp2_cifar_only.pdf}
    \caption{\textbf{Left:} Comparison between DP2RMSProp, DP-AdamBC, DP-Adam and DP-SGD on CIFAR10 with CNN, \textbf{Right:} The performance of DP2RMSProp with different phase switching frequency $s$ on CIFAR10 with CNN.}
    \label{fig:compare_dp2_cifar_only}
\end{figure*}


\section{Privacy Analysis}
\label{apdix:privacy_proof}
Since DP-AdamBC uses the privitized gradient to update first and second moment estimates, and the DP bias $\Phi$ can be calculated from public hyperparameters $B, \sigma, C$. By the post-processing property of DP, for a given privacy accounting method, the same privacy guarantee holds for DP-AdamBC as with DP-SGD or DP-Adam. We formalize the proof as follows.

\begin{theorem}[Privacy guarantee of DP-SGD]
There exist constants $c_1$ and $c_2$ so that given the sampling probability $q=L/N$ and the number of steps $T$, for any $\epsilon < c_1 q^2 T$ , Algorithm 1 in \citet{abadi2016deep} is $(\epsilon, \delta)$-differentially private for any $\delta>0$ if we choose $\sigma \geq (c_2 q \sqrt{T \log{(1/\delta)}}) / \epsilon$.
\end{theorem}

\begin{proposition}[Privacy guarantee of DP-AdamBC]
    Let the optimization algorithm DP-SGD($\theta, X, y, C, \sigma, B$) (Algorithm 1 in \citet{abadi2016deep}),
    % DP-Adam($\theta, X, y, C, \sigma$) (Algorithm 1 in [2]),
    with privacy analysis \textit{Compose}($T$, $\theta_{1,\ldots,T}$), be $(\epsilon, \delta)$-DP, then DP-AdamBC($\theta, X, y, C, \sigma, B$) with the same privacy analysis \textit{Compose}($T$, $\theta_{1,\ldots,T}$) is also $(\epsilon, \delta)$-DP.
\end{proposition}

\begin{proof}
    Let \textit{PrivitizeGradient}($\theta, X, y, C, \sigma, B$) be the key step providing DP guarantee in DP-SGD, DP-Adam and DP-AdamBC:
    \begin{center}
    (Compute gradient) $g_t(x_i) \xleftarrow{} \nabla_\theta \mathcal{L} (\theta, x_i), \forall i \in B$, \\
    (Clip gradient) $\Bar{g}_t(x_i) \xleftarrow{} g_t(x_i) / \max{(1, \lVert g_t(x_i) \rVert /C)}$, \\
    (Noise gradient) $\Tilde{g_t}\xleftarrow{} (1/B) (\sum_i \Bar{g}_t(x_i) + \mathcal{N}(0, \sigma^2 C^2))$.
    \end{center}
    When the DP noise is sampled from a Gaussian distribution, by standard arguments of the Gaussian mechanism in \citet{dwork2006calibrating} (Appendix A), the procedure is $(\epsilon', \delta')$-DP with $\sigma \geq (2\ln{(1.25/\delta')}C)/\epsilon'$. By the privacy amplification theorem, \textit{PrivitizeGradient}($\theta, X, y, C, \sigma, B$) is $(O(q\epsilon'), q\delta')$-DP with sampling probability $q=B/N$ for batch size $B$ and total sample size $N$. Let \textit{Compose}($T$, $\theta_{1,\ldots,T}$) computes the overall privacy cost over $T$ training iterations with a privacy accountant (e.g. strong composition in \citet{dwork2006calibrating}, moment accountant in \citet{abadi2016deep}, RDP accountant in \citet{Mironov_2019}), \textit{PrivitizeGradient}($\theta, X, y, C, \sigma$) over $T$ iterations with \textit{DP-SGD} update $\theta_{t+1} \xleftarrow{} \theta_{t} - \eta \nabla_{\theta_t} f$, where $\eta$ is a hyperparameter representing learning rate, $\nabla_{\theta_t} f$ is the output privitized gradient from \textit{PrivitizeGradient}, is $(\epsilon, \delta)$-DP. Since DP-AdamBC's update (Equation (4)) does not inquire additional private information as (1) $\Phi$ and $\gamma'$ are determined from user-defined hyperparameters and (2) moment estimates are calculated from privitized gradients. By the post-processing property of DP \citep{dwork2006calibrating}, \textit{PrivitizeGradient}($\theta, X, y, C, \sigma$) over $T$ iterations with \textit{DP-AdamBC} update is $(\epsilon, \delta)$-DP. If \textit{Compose}($T$, $\theta_{1,\ldots,T}$) is the moment accountant in \citet{abadi2016deep}, DP-AdamBC has the same privacy guarantees as in Theorem 1.
\end{proof}


\section{Convergence analysis}
\label{apdix:convergence_analysis}
In a recent work, \citet{défossez2022simple} shows dropping the correction term in the first moment estimation (Section 2.2) has no observable effect and proves the convergence of Adam with this mild modification. We use the same setup and extend the analysis from \citet{défossez2022simple}.

Let $F: \sR^d \rightarrow \sR$ be the objective function which $d \in \sN$ is the problem dimension (number of parameters of the function to optimize), $\theta$ be the model parameters, $f: \sR^d \rightarrow \sR$ be the stochastic function such that $\forall \theta \in \sR^{d}, \E{[\nabla f(\theta)]} = \nabla F(\theta)$, $\eta$ be the learning rate, the subscript $t$ be the step index, the subscript $i$ be the dimension index. We make the following assumptions.

\begin{assumption}[$F$ is bounded below]
\label{assmption:f_bounded_below}
$F$ is bounded below by $F_*$, i.e. $\forall \theta \in \sR^{d}, F(\theta) \geq F_*$.
\end{assumption}

\begin{assumption}[Bounded stochastic gradient]
\label{assmption:bounded_stochastic_grad}
The $L2$-norm of the stochastic gradient is uniformly almost surely bounded, i.e. $\exists C \geq 0$ such that $\forall t$, $\forall \theta \in \sR^{d}, \norm{\nabla f_t(\theta)} \leq C$ a.s..
% uniformly bounded implies the sequence of f over step t is all bounded, and C does not depend on theta.
\end{assumption}
Note that Assumption \ref{assmption:bounded_stochastic_grad} implies that for every update step the gradient clipping operation in DP-SGD, DP-Adam or DP-AdamBC has no effect. It is usually assumed for the ease of theoretical analysis such as in \citet{li2023dp2}, but is often violated empirically.

At step $t$, we call the noise sample $z_t \sim Z$. The privatized gradient is $\Tilde{g_t} = \nabla f_t(\theta) + z_t$, and the private second moment estimate is $v_t^p=\sum_{j=1}^{t}\beta_2^{j}\Tilde{g_t}^2$. We introduce the following Lemma, which is the main technical change we need to adapt the proof of \citet{défossez2022simple} to our setting with DP noise.
% \begin{lemma}
% \label{lemma:gaussian_concentration_on_z}
%     Let $z \sim Z$ denote a sample from the DP noise random variable, following a zero-mean Normal distribution $\cN(0, \Phi=(\sigma C/B)^2)$. Let $v_t^p$ be the private second moment estimate defined above. For any $0< \alpha < 1 $, with probability of at least $(1-\alpha)$ we have the: $\Pr\Big[|Z| \geq \sqrt{\ln{(1/\frac{\alpha}{2})}(2\Phi)}\Big] \leq \alpha$; and that $\Pr\Big[|v_{t, i}| \geq \sum_{j=0}^{t-1}\beta_2^{j}(C + \sqrt{\ln{(1/\frac{\alpha}{2})}(2\Phi)})^2\Big] \leq \alpha$.
% \end{lemma}
% \begin{proof}
%     The result is an application of Gaussian concentration inequality \citep{wainwright2019high}, which we have $\Pr[|Z| \geq \delta] \leq 2\exp{(-\delta^2/(2\Phi))}$, or equivalently, with probability of at least $(1-\alpha)$ where $0< \alpha < 1 $ we have $\Pr\Big[|Z| \geq \sqrt{\ln{(1/\frac{\alpha}{2})}(2\Phi)}\Big] \leq \alpha$. Since $v_{t,i}=\sum_{j=0}^{t-1} \beta_2^{j}\Tilde{g_t}^2$ and since by Assumption \ref{assmption:bounded_stochastic_grad} we have $\nabla f_t(\theta) \leq C$, so that $\Pr\Big[|v_{t, i}| \geq \sum_{j=0}^{t-1}\beta_2^{j}(C + \sqrt{\ln{(1/\frac{\alpha}{2})}(2\Phi)})^2\Big] \leq \alpha$.
%     %$\Pr[|v_{t, i}| \geq \delta] = \Pr[|v_{t, i}| \geq (t)(\beta_2^j)(\delta+C)^2] \leq 2\exp{(-\delta^2/(2\Phi))}$, or equivalently, $\Pr[|v_{t, i}| \geq ]$
% \end{proof}

% \ml{You don't need to re-prove union bounds. I suggest the following (you need to call it corollary later when you use it).}
\begin{lemma}
\label{lemma:gaussian_concentration_on_z}
Let $\mu^{*} = (\beta_2(\beta_2^t-1)/(\beta_2-1))((\Phi-\frac{2\Phi}{\pi})+(C+\sqrt{\frac{2\Phi}{\pi}})^2)$, $\nu^{*} = (2\beta_2^2\Phi\sqrt{(\beta_2^t-1)/ (\beta_2^2-1)})$, $b^{*}=4\beta_2\Phi$ we have,
    $\Pr[v_t^p \geq \delta] \leq \alpha$ with,
    % \ml{$\nu$ is not defined at this point. I also wonder if you could express the constraints between $\delta$ and $\frac{\nu^{*2}}{b^*}$ as a constraint of quantities like $\Phi$ and $\beta_2$, so that you can be in only one case? If not that's ok, but it would make everything easier! (hopefully the sqrt case :))}
    \begin{align*}
        \delta \geq
        \begin{cases}
            \mu^{*} + \sqrt{2\nu^{*2}\ln{(\frac{2}{\alpha})}}
            & 0 \leq \delta  \leq \frac{\nu^{*2}}{b^{*}} \\
            \mu^{*} + 2b^{*}\ln{(\frac{2}{\alpha})}
            & \delta   \geq \frac{\nu^{*2}}{b^{*}}.
        \end{cases}
    \end{align*}
\end{lemma}
\begin{proof}
    By Assumption \ref{assmption:bounded_stochastic_grad} we have $|\nabla f_t(\theta)| \leq C, \, \forall t$, so that $\beta_2^t(\nabla f_t(\theta) + z_t)^2 \leq \beta_2^t(|\nabla f_t(\theta)| + |z_t|)^2 \leq \beta_2^t(C + |z_t|)^2$.
    % \ml{You can't remove the absolute value around $z_t$ here, I don't think that's correct. I think you need to upper bound $v_t^p$ with $\E[v_t^p] + \sqrt{\ln{(1/\frac{\alpha}{2})}(2\nu^{*2})}$, and then use $\E[v_t^p] \leq \mu$. That would work, no? I don't know if you need the martingale stuff for that or not? Or is there another way? At least the current one has a broken step.}
    Since $z_t$ is sampled from $\cN(0, \Phi=(\sigma C/B)^2)$, by the similar argument as in Proposition \ref{prop:bound_fixed}, $\beta_2^t(C + z_t)^2$ is sub-exponential with $\nu = 2\beta_2^t\Phi$, $b=4\beta_2^t\Phi$. Since each $z_t$ is drawn independently, by the additive property of sub-exponential, $\sum_{j=1}^{t}\beta_2^{j}(C + z_t)^2$ is sub-exponential with $\nu^{*} = (2\beta_2^2\Phi\sqrt{(\beta_2^t-1)/ (\beta_2^2-1)})$, $b^{*}=4\beta_2\Phi$.
    % and $\E{[\sum_{j=1}^t\beta_2^j(C + Z_j)^2]} = \sum_{j=1}^t \E{[\beta_2^j(C + Z_j)^2]} = \sum_{j=1}^t (\beta_2^j(\Var(C+Z_j)+(\E(C+Z_j))^2))=(\Phi+C^2)(\beta_2(\beta_2^t-1)/(\beta_2-1)) \coloneqq \mu$.
    In addition we have $\E{[v_t^p]} \leq \E{[\sum_{j=1}^{t}\beta_2^j(C+|Z_t|)^2]}=\sum_{j=1}^{t}\beta_2^j\E{[(C+|Z_t|)^2]} = \sum_{j=1}^{t}\beta_2^j (\Var[|Z_j|]+(C+\E{[|Z_j|]})^2)$. Given $Z_j\sim N(0,\Phi)$, $|Z_j|$ has a truncated Normal distribution where $\E{[|Z_j|]} = \sqrt{\frac{2\Phi}{\pi}}$ and $\Var{[|Z_j|]}=\Phi-\frac{2\Phi}{\pi}$. Let $\mu^{*} = (\beta_2(\beta_2^t-1)/(\beta_2-1))((\Phi-\frac{2\Phi}{\pi})+(C+\sqrt{\frac{2\Phi}{\pi}})^2)$ we have $E{[v_t^p]}\leq \mu^{*}$.
    By Proposition 2.9 in \citet{wainwright2019high} we have the result.
    % for any $0 < \alpha < 1$, we have $\Pr[v_t^p \geq \delta] \leq \alpha$ with:
    % \begin{align*}
    %     \delta \geq
    %     \begin{cases}
    %         \mu + \sqrt{(\ln{1/\frac{\alpha}{2}})(2\nu^{*2})}
    %         & 0 \leq \delta  \leq \frac{\nu^{*2}}{b^{*}} \\
    %         \mu + (\ln{1/\frac{\alpha}{2}})2b^{*}
    %         & \delta   \geq \frac{\nu^{*2}}{b^{*}}.
    %     \end{cases}
    % \end{align*}
\end{proof}

We can apply a union bound over Lemma \ref{lemma:gaussian_concentration_on_z} to directly obtain the following result:
\begin{corollary}
\label{lemma:union_bound_on_v}
    Let $\mu^{*}$ be defined as in Lemma \ref{lemma:gaussian_concentration_on_z}. We have that $\Pr[\sup_t v_t^p \geq \delta] \leq \alpha$ with,
    \begin{align*}
        \delta \geq
        \begin{cases}
            \mu^{*} + \sqrt{2\nu^{*2}\ln{(\frac{2T}{\alpha})}}
            & 0 \leq \delta  \leq \frac{\nu^{*2}}{b^{*}} \\
            \mu^{*} + 2b^{*}\ln{(\frac{2T}{\alpha})}
            & \delta   \geq \frac{\nu^{*2}}{b^{*}}.
        \end{cases}
    \end{align*}
\end{corollary}

% \begin{lemma}
% \label{lemma:union_bound_on_v}
%     The union bound on $v_t^p$ for all $t \in {1, \ldots, T}$, $T \in \sR^+$ is $\Pr[\bigcup_{t=1}^{T} |v_{t, i}| \geq \sum_{j=0}^{t-1}\beta_2^{j}(C + \sqrt{\ln{(1/\frac{\alpha}{2T})}(2\Phi)})^2] \leq \alpha$.
% \end{lemma}
% \begin{proof}
%     For every index $i$, let $E_t$ be the event of $\Pr[|v_{t, i}| \geq \delta]$, $t \in [1, \ldots, T]$ for some $T \in \sR^+$, and we want to bound the total probability for all $t$. By Boole's inequality we have,
%     \begin{align*}
%         \Pr\bigg(\bigcup_{t=1}^{T} E_t \bigg) \leq \sum_{t=1}^{T} \Pr(E_t) = \sum_{t=1}^{T} \Pr(|v_{t, i}| \geq \delta) \leq T\alpha.
%     \end{align*}
%     Reorganizing the terms gives the result $\Pr[\bigcup_{t=1}^{T} |v_{t, i}| \geq \sum_{j=0}^{t-1}\beta_2^{j}(C + \sqrt{\ln{(1/\frac{\alpha}{2T})}(2\Phi)})^2] \leq \alpha$.
% \end{proof}


\begin{assumption}[Smoothness of $F$]
\label{assmption:f_smooth}
The gradient of $F$ is $L$-Liptchitz-continuous such that $\forall \theta, \theta' \in \sR^{d}, \norm{\nabla F(\theta) - \nabla F(\theta')} \leq L \norm{\theta - \theta'}$.
\end{assumption}

Let $\theta_0$ be the randomly initialized starting point, $\forall t \in [1, \ldots, T]$, the update rule of DP-AdamBC is $\theta_{t} \leftarrow \theta_{t-1} - \eta \cdot \hat{m}_t / \sqrt{\max{\big(\hat{v}_t - \Phi}, \gamma' \big)}$, and the update rule of DP-Adam is $\theta_{t} \leftarrow \theta_{t-1} - \eta \cdot \hat{m}_t / \sqrt{\hat{v}_t + \gamma}$, where $\hat{m}_t$, $\hat{v}_t$ and $\Phi$ are the private first, second moment estimates and the DP bias in private second moment estimates (Section \ref{sec:motiv-bias}). We prove the following propositions.

\begin{proposition}[Convergence of DP-AdamBC and DP-Adam Without momentum]
    Given the above assumptions and the defined update rule with $\beta_1 = 0, 0 < \beta_2 < 1$, $\eta_t = \eta \sqrt{\frac{1-\beta_2^t}{1-\beta_2}}, \eta>0$, $\forall \alpha$ s.t. $0<\alpha<1$, let $\nu^{*} = (2\beta_2^2\Phi\sqrt{(\beta_2^t-1)/ (\beta_2^2-1)})$, $b^{*}=4\beta_2\Phi$, we have $\forall T \in \sN^*$, \\
    \\
    (DP-Adam)
    % \begin{align*}
    %     \E{[\norm{\nabla F(\theta_{t})}^2]} &\leq \frac{2(C + \sqrt{\ln{(1/\frac{\alpha}{2T})}(2\Phi)})(F(\theta_0)-F_{*})}{\eta T} \\ &+ \bigg( \frac{4d(C^2+\Phi)}{\sqrt{1-\beta_2}} + \frac{\eta d L \sqrt{C^2+\Phi}}{1-\beta_2} \bigg) \\ & \times \bigg(\frac{1}{T}\ln{\big(1+\frac{C^2+\Phi}{(1-\beta_2)\epsilon}) - \ln{(\beta_2)}} \bigg),
    % \end{align*}
    \begin{align*}
        \E{[\norm{\nabla F(\theta_{t})}^2]} &\leq \frac{2\delta(F(\theta_0)-F_{*})}{\eta T} \\ &+ \bigg( \frac{4d(C^2+\Phi)}{\sqrt{1-\beta_2}} + \frac{\eta d L \sqrt{C^2+\Phi}}{1-\beta_2} \bigg) \\ & \times \bigg(\frac{1}{T}\ln{\big(1+\frac{C^2+\Phi}{(1-\beta_2)\epsilon}) - \ln{(\beta_2)}} \bigg),
    \end{align*}
    (DP-AdamBC)
    \begin{align*}
        \E{[\norm{\nabla F(\theta_{t})}^2]} &\leq \frac{2\sqrt{\delta^2 - \Phi}(F(\theta_0)-F_{*})}{\eta T} \\ &+ \bigg( \frac{4dC^2)}{\sqrt{1-\beta_2}} + \frac{\eta d L C}{1-\beta_2} \bigg) \\ & \times \bigg(\frac{1}{T}\ln{\big(1-\frac{C^2+\Phi}{(1-\beta_2)\Phi}) - \ln{(\beta_2)}} \bigg),
    \end{align*}
        \begin{align*}
    \delta \geq
    \begin{cases}
        \mu^{*} + \sqrt{\ln{(1/\frac{\alpha}{2T})}(2\nu^{*2})}
        & 0 \leq \delta  \leq \frac{\nu^{*2}}{b^{*}} \\
        \mu^{*} + \ln{(1/\frac{\alpha}{2T})}2b^{*}
        & \delta   \geq \frac{\nu^{*2}}{b^{*}}.
    \end{cases}
\end{align*}
\end{proposition}

\paragraph{DP-Adam.}
\begin{proof}
    By dropping the correction term in $m_t$ we have $\forall t, \eta_t = \eta\sqrt{\frac{1-\beta_2^t}{1-\beta_2}}$. Since Assumption \ref{assmption:bounded_stochastic_grad} assumes away the effect of gradient clipping, and since the DP noises are sampled from a zero-mean Normal distribution, proving convergence for DP-Adam (without Momentum) is very similar to the original proof of Theorem 2 in \citep{défossez2022simple}.
    %except that we have $\E{[(\nabla_i f_t^{p}(\theta_{t-1}))^2]} = \E{[(\nabla_i f_t^{c}(\theta_{t-1}) + z_t)^2]} \leq C^2  + \Phi$, $\E{[v_{t}^{p}]} \leq \sqrt{(C^2 + \Phi)\frac{1-\beta_2^t}{1-\beta_2}}$.
    We show the key steps below. Using Assumption \ref{assmption:f_smooth} we have,
    \begin{gather*}
        F(\theta_t) \leq F(\theta_{t-1}) - \eta_t \nabla F(\theta_{t-1})^T u_t + (\eta_t^2 L/2) \lVert u_t \rVert^2, \; \\
        u_t = \frac{\nabla_i f_t^{p}(\theta_{t-1})}{\sqrt{v_t^{p} + \epsilon}},
    \end{gather*}
    where $u_t$ is the update of DP-Adam without momentum.
    Taking the complete expectation with respect to all past steps before $t$ and $t$ step noise distribution we get,
    \begin{align*}
        \E{[F(\theta_t)]} \leq F(\theta_{t-1}) &- \eta_t \E{\bigg[\frac{\nabla_i F(\theta_{t-1}) \nabla_i f_t^{p}(\theta_{t-1})}{\sqrt{v_{t, i}^{p} + \epsilon }}\bigg]} \\ &+ \frac{\eta_t^2 L}{2}\E{[\lVert u_t \rVert^2]}.
    \end{align*}
    Given $\epsilon \ll v_t^p$, $\forall i \in [d]$  we can bound the first expectation term on the right side using Lemma \ref{lemma:gaussian_concentration_on_z}
    % With $\Tilde{v}_{t,i} = \sum_{j=0}^{t-1}(\nabla_i f_t^{p}(\theta_{t-1}))^2 = \sum_{j=0}^{t-1}(\nabla_i f_t^{c}(\theta_{t-1}) + z_t)^2$, $\max{|g_{i}|} \leq \norm{g} \leq C$ and by Lemma \ref{lemma:gaussian_concentration_on_z}, $\forall i \in [d]$ we have,
    % \begin{align*}
    %     \sqrt{\epsilon + \Tilde{v}_{t,i}} &\leq \Big(C + \sqrt{-\ln{\frac{\alpha}{2}}(2\Phi)}\Big)\sqrt{\sum_{j=0}^{t-1} \beta_2^{j}} \\ &= \Big(C + \sqrt{-\ln{\frac{\alpha}{2}}(2\Phi)}\Big)\sqrt{\frac{1-\beta_2^t}{1-\beta_2}},
    % \end{align*}
    to get a high probability bound.
    %with probability of at least $(1-\alpha)$, $0< \alpha < 1 $.
    By the similar steps as in Lemma 5.1 \citep{défossez2022simple}, except that we have $\E{[(\nabla_i f_t^{p}(\theta_{t-1}))^2]} = \E{[(\nabla_i f_t^{c}(\theta_{t-1}) + z_t)^2]} \leq C^2  + \Phi$, substituting the result of Lemma \ref{lemma:gaussian_concentration_on_z} gives the inequality on $\E{\bigg[\frac{\nabla_i F(\theta_{t-1}) \nabla_i f_t^{p}(\theta_{t-1})}{\sqrt{v_{t, i}^{p} + \epsilon }}\bigg]}$.
    % substituting in the inequality above we get,
    % \begin{align*}
    %     \E{\bigg[\frac{\nabla_i F(\theta_{t-1}) \nabla_i f_t^{p}(\theta_{t-1})}{\sqrt{v_{t, i}^{p} + \epsilon }}\bigg]}
    %     % &\geq \frac{(\nabla_i F(\theta_{t-1}))^2}{2\sqrt{\Tilde{v}_{t,i}^{p}+ \epsilon}} - 2\sqrt{C^2+\Phi}\E{\bigg[ \frac{(\nabla_i f_t^{p}(\theta_{t-1}))^2}{v_{t,i}^{p}+ \epsilon} \bigg]} \\
    %     &\geq \frac{(\nabla_i F(\theta_{t-1}))^2}{2\Big(C + \sqrt{-\ln{\frac{\alpha}{2}}(2\Phi)}\Big)\sqrt{\frac{1-\beta_2^t}{1-\beta_2}}} \\ &- 2\sqrt{C^2+\Phi}\E{\bigg[ \frac{(\nabla_i f_t^{p}(\theta_{t-1}))^2}{v_{t,i}^{p}+ \epsilon} \bigg]}.
    % \end{align*}
    Substituting in the result and since $\eta \leq \eta_t$ we get,
    % $\eta=\eta_t/\sqrt{\frac{1-\beta_2^t}{1-\beta_2}}$
    \begin{align*}
        \E{[F(\theta_t)]} \leq F(\theta_{t-1}) &- \frac{\eta}{2\delta} \norm{\nabla F(\theta_{t-1})}^2 \\ &+ \bigg(2\eta_t \sqrt{C^2+\Phi} + \frac{\eta_t^2 L}{2}\bigg) \E{[\lVert u_{t} \rVert^2]},
    \end{align*}
    where $\delta$ has the form as in Lemma \ref{lemma:gaussian_concentration_on_z}.
    Summing over all steps $t \in [T]$ and taking the complete expectation, and since $\eta_t \leq \eta / \sqrt{1-\beta_2}$ we have,
    \begin{align*}
        \E{[F(\theta_T)]} &\leq F(\theta_0) - \frac{\eta \sum_{t=0}^{T-1} \E{[\norm{\nabla F(\theta_{t-1})}^2]}}{2\delta} \\ &+ \bigg(\frac{2\eta \sqrt{C^2+\Phi}}{\sqrt{1-\beta_2}} + \frac{\eta^2 L}{2(1-\beta_2)}\bigg) \sum_{t=0}^{T-1} \E{[\lVert u_{t} \rVert^2]},
    \end{align*}
    where $\delta$ has the form as in Corollary \ref{lemma:union_bound_on_v}.
    Since $y=ln(x)$ is a concave function for $x\in \mathbb{R}^{+}$, by Jensen's inequality we have $\E{(\ln{(x)})} \leq \ln{(\E{(x)})}$. By the similar steps as in Lemma 5.2 \citep{défossez2022simple} we have,
    \begin{align*}
        \sum_{t=0}^{T-1} \E{[\lVert u_{t, i} \rVert^2]} &= \sum_{i=1}^{d} \E{\bigg[ \sum_{t=0}^{T-1} u_{t, i}^2 \bigg]} \\ &\leq \sum_{i=1}^{d} \E{[\ln{(1+v_{T}^p/\epsilon)} - T\ln{(\beta_2)}]} \\
        &\leq \sum_{i=1}^{d} \big(\ln{(1+\frac{(C^2+\Phi)(1-\beta_2^{T})}{(1-\beta_2)\epsilon}) - T\ln{(\beta_2)}}\big)\\
        &\leq d\big(\ln{(1+\frac{C^2+\Phi}{(1-\beta_2)\epsilon}) - T\ln{(\beta_2)}}\big).
    \end{align*}
    Substituting in the result and rearrange the terms gives the final result.
    % \begin{align*}
    %     \E{[\norm{\nabla F(\theta_{t})}^2]} &\leq \frac{2(C + \sqrt{-\ln{\frac{\alpha}{2}}(2\Phi)})(F(\theta_0)-F_{*})}{\eta T} \\ &+ \bigg( \frac{4d(C^2+\Phi)}{\sqrt{1-\beta_2}} + \frac{\eta d L \sqrt{C^2+\Phi}}{1-\beta_2} \bigg) \bigg(\frac{1}{T}\ln{\big(1+\frac{C^2+\Phi}{(1-\beta_2)\epsilon}) - \ln{(\beta_2)}} \bigg).
    % \end{align*}

\end{proof}

\paragraph{DP-AdamBC.}

\begin{proof}
    Similar to the proof steps above, by Assumption \ref{assmption:f_smooth} we have,
    \begin{gather*}
        F(\theta_t) \leq F(\theta_{t-1}) - \eta_t \nabla F(\theta_{t-1})^T u_t + (\eta_t^2 L/2) \lVert u_t \rVert^2, \; \\
        u_t = \frac{\nabla_i f_t^{p}(\theta_{t-1})}{\sqrt{v_t^{p} - \Phi}},
    \end{gather*}
    where $u_t$ is the update of DP-AdamBC without momentum.
    Taking the complete expectation we get,
    \begin{align*}
        \E{[F(\theta_t)]} \leq F(\theta_{t-1}) &- \eta_t \E{\bigg[\frac{\nabla_i F(\theta_{t-1}) \nabla_i f_t^{p}(\theta_{t-1})}{\sqrt{v_{t, i}^{p} - \Phi }}\bigg]}\\ &+ \frac{\eta_t^2 L}{2}\E{[\lVert u_t \rVert^2]}.
    \end{align*}
    We derive a high probability bound for the first expectation term on the right side.
    Let $G=\nabla_i F(\theta_{t-1}), g=\nabla_i f_t^{p}(\theta_{t-1}), \Bar{g}=\nabla_i f_t^{c}(\theta_{t-1})$, let $\Tilde{v}_t=\beta_2 v_{t-1}+\E{[g^2]}$ denote $v_t$ with the last gradient replaced by its conditional expectation with respect to all past steps, let $z, \Tilde{v}, v$ abbreviate for $z_t, \Tilde{v}_{t, i}^{p}, v_{t, i}^{p}$, we rewrite the first expectation term as follows,
    \begin{align*}
        \E{\bigg[ \frac{Gg}{\sqrt{v - \Phi}} \bigg]} &= \E{\bigg[ \frac{G(\Bar{g}+z)}{\sqrt{v - \Phi}} \bigg]} \\ &= \E{\bigg[ \frac{G\Bar{g}}{\sqrt{v - \Phi}} \bigg]} + \E{\bigg[ \frac{Gz}{\sqrt{v - \Phi}} \bigg]} \\ &= \E{\bigg[ \frac{G\Bar{g}}{\sqrt{v - \Phi}} \bigg]},
    \end{align*}
    since $G$ and $z$ are independent and $z$ has expectation equals zero. In addition, we have that $E{[g^2]} = \E{[(\Bar{g}+z)^2]} = \E{[\Bar{g}^2]} + \Phi$, and by definition $\E{[g^2]} \leq \Tilde{v}$, so that $\E{[\Bar{g}^2]} \leq \Tilde{v} - \Phi$ and $\E{[\Bar{g}^2]} \leq C^2$. With these conditions, by the same steps as in Lemma 5.1 \citep{défossez2022simple} we have,
    \begin{align*}
        \E{\bigg[\frac{\nabla_i F(\theta_{t-1}) \nabla_i f_t^{p}(\theta_{t-1})}{\sqrt{v_{t, i}^{p} -\Phi }}\bigg]} &\geq \frac{(\nabla_i F(\theta_{t-1}))^2}{2\sqrt{\Tilde{v}_{t,i}^{p}-\Phi}} \\ &- 2C\E{\bigg[ \frac{(\nabla_i f_t^{p}(\theta_{t-1}))^2}{v_{t,i}^{p}-\Phi} \bigg]}.
    \end{align*}
    % Similarly with Lemma \ref{lemma:gaussian_concentration_on_z}, $\forall i \in [d]$ we have,
    % \begin{align*}
    %     \sqrt{\Tilde{v}_{t,i}-\Phi} \leq \sqrt{\sum_{j=0}^{t-1} \beta_2^{j}\Big(C + \sqrt{-\ln{\frac{\alpha}{2}}(2\Phi)}\Big)^2 - \Phi},
    % \end{align*}
    % with probability of at least $(1-\alpha)$, $0< \alpha < 1 $.
    Substituting the result from Lemma \ref{lemma:gaussian_concentration_on_z} we get with probability of at least $(1-\alpha)$, $0< \alpha < 1 $,
    % \begin{align*}
    %     &\E{\bigg[\frac{\nabla_i F(\theta_{t-1}) \nabla_i f_t^{p}(\theta_{t-1})}{\sqrt{v_{t, i}^{p} + \epsilon }}\bigg]} \\
    %     &\geq \frac{(\nabla_i F(\theta_{t-1}))^2}{2\sqrt{\sum_{j=0}^{t-1} \beta_2^{j}\Big(C + \sqrt{-\ln{\frac{\alpha}{2}}(2\Phi)}\Big)^2 - \Phi}} \\ &- 2C\E{\bigg[ \frac{(\nabla_i f_t^{p}(\theta_{t-1}))^2}{v_{t,i}^{p}+ \epsilon} \bigg]}.
    % \end{align*}
    \begin{align*}
        &\E{\bigg[\frac{\nabla_i F(\theta_{t-1}) \nabla_i f_t^{p}(\theta_{t-1})}{\sqrt{v_{t, i}^{p} + \epsilon }}\bigg]} \\
        &\geq \frac{(\nabla_i F(\theta_{t-1}))^2}{2\sqrt{\sum_{j=0}^{t-1} \beta_2^{j}\delta^2 - \Phi}} - 2C\E{\bigg[ \frac{(\nabla_i f_t^{p}(\theta_{t-1}))^2}{v_{t,i}^{p}+ \epsilon} \bigg]},
    \end{align*}
    where $\delta$ is in the form as in Lemma \ref{lemma:gaussian_concentration_on_z}.
    Summing over all steps $t \in [T]$ and taking the complete expectation we have,
    % \begin{align*}
    %     \E{[F(\theta_T)]} &\leq F(\theta_0) - \frac{\eta \sum_{t=0}^{T-1} \E{[\norm{\nabla F(\theta_{t-1})}^2]}}{2\sqrt{\Big(C + \sqrt{-\ln{\frac{\alpha}{2T}}(2\Phi)}\Big)^2 - \Phi}} \\ &+ \bigg(\frac{2\eta C}{\sqrt{1-\beta_2}} + \frac{\eta^2 L}{2(1-\beta_2)}\bigg) \sum_{t=0}^{T-1} \E{[\lVert u_{t} \rVert^2]}.
    % \end{align*}
    \begin{align*}
        \E{[F(\theta_T)]} &\leq F(\theta_0) - \frac{\eta \sum_{t=0}^{T-1} \E{[\norm{\nabla F(\theta_{t-1})}^2]}}{2\sqrt{\delta^2 - \Phi}} \\ &+ \bigg(\frac{2\eta C}{\sqrt{1-\beta_2}} + \frac{\eta^2 L}{2(1-\beta_2)}\bigg) \sum_{t=0}^{T-1} \E{[\lVert u_{t} \rVert^2]},
    \end{align*}
    where $\delta$ is in the form as in Corollary \ref{lemma:union_bound_on_v}.
    We bound $\sum_{t=0}^{T-1} \E{[\lVert u_t \rVert^2]}$ using the similar approach as in Lemma 5.2 \citep{défossez2022simple}. Let $a_t=(\nabla f_t^{p})^2$ and $b_t = \sum_{j=1}^{t}\beta_2^{t-j}a_j$, then $\sum_{t=1}^{T} \E{[\lVert u_t \rVert^2]} = \sum_{i=1}^{d} \E{[\sum_{t=1}^{T} \frac{a_{t, i}}{b_{t,i} - \Phi}]}$. Given $\ln$ is increasing and $b_t > a_t > 0$,
    \begin{align*}
        \frac{a_{t, i}}{b_{t,i} - \Phi} &\leq \ln{\big(1/(1-\frac{a_{t,i}}{b_{t,i}-\Phi})\big)} \\
        % &= \ln{(b_{t,i}-\Phi)} - \ln{(b_{t,i}-\Phi-a_{t,i})} \\
        % &= \ln{(b_{t,i}-\Phi)} - \ln{(\beta_2b_{t-1,i}-\Phi)} \\
        &= \ln{\big( \frac{b_{t,i}-\Phi}{b_{t-1,i}-\Phi} \big)} + \ln{\big( \frac{b_{t-1,i}-\Phi}{\beta_2 b_{t-1,i}-\Phi} \big)}
    \end{align*}
    \begin{align*}
        \sum_{t=1}^{T} \frac{a_{t, i}}{b_{t,i} - \Phi} &=  \sum_{t=1}^{T} \ln{\big( \frac{b_{t,i}-\Phi}{b_{t-1,i}-\Phi} \big)} +  \sum_{t=1}^{T} \ln{\big( \frac{b_{t-1,i}-\Phi}{\beta_2 b_{t-1,i}-\Phi} \big)} \\
        &\leq \ln{\big( 1-\frac{b_{T,i}}{\Phi} \big)} - T\ln{(\beta_2)},
    \end{align*}
    where the last inequality is because $b_{0}=0$ and $b_t > \Phi \; \forall t$.
    Substituting the result in we get,
    \begin{align*}
        \sum_{t=0}^{T-1} \E{[\lVert u_{t, i} \rVert^2]} &= \sum_{i=1}^{d} \E{\bigg[ \sum_{t=0}^{T-1} u_{t, i}^2 \bigg]} \\ &\leq \sum_{i=1}^{d} \E{[\ln{(1-v_{T}^p/\Phi)} - T\ln{(\beta_2)}]} \\
        &\leq \sum_{i=1}^{d} \big(\ln{(1-\frac{(C^2+\Phi)(1-\beta_2^{T})}{(1-\beta_2)\Phi}) - T\ln{(\beta_2)}}\big)\\
        &\leq d\big(\ln{(1-\frac{C^2+\Phi}{(1-\beta_2)\Phi}) - T\ln{(\beta_2)}}\big).
    \end{align*}
    Substituting in the result and rearrange the terms gives the final result.
    % \begin{align*}
    %     \E{[\norm{\nabla F(\theta_{t})}^2]} &\leq \frac{2\sqrt{\Big(C + \sqrt{-\ln{\frac{\alpha}{2}}(2\Phi)}\Big)^2 - \Phi}(F(\theta_0)-F_{*})}{\eta T} \\ &+ \bigg( \frac{4dC^2)}{\sqrt{1-\beta_2}} + \frac{\eta d L C}{1-\beta_2} \bigg) \\ & \times \bigg(\frac{1}{T}\ln{\big(1-\frac{C^2+\Phi}{(1-\beta_2)\Phi}) - \ln{(\beta_2)}} \bigg).
    % \end{align*}
\end{proof}

\paragraph{Discussion on the convergence bound.}
Comparing the convergence rate between the two algorithm is not straight-forward giving that one has a slight advantage in $\E{[(\nabla F) u_t]}$, the expectation of the update direction deviating from the true descent direction, and a disadvantage in $\E{\norm{u_t}^2}$, the expectation of the update size. We discuss an approximate comparison between $\E{\norm{u_t}^2}$. Let $u_t^{\textnormal{DP-Adam}}=\frac{g}{\sqrt{v+\epsilon}}$, $u_t^{\textnormal{DP-AdamBC}}=\frac{g}{\sqrt{v-\Phi}}$, with the data and noise distribution we can consider the numerator and denominator as two random variables such that for each dimension $i$, for DP-Adam we have $\E{[u_{t, i}^2]} = \E{[A/B]}$, for DP-AdamBC we have $\E{[u_{t, i}^2]} = \E{[A/B']}$, where $A$ is the random variable for $t$-step gradient $g^2$, $B$ is the random variable for $v+\epsilon$, $B'$ is the random variable for $v-\Phi$. It would be difficult to derive closed form results for expectation on ratios of random variables without specific assumption, but taking a second-order Taylor expansion approximately gives $\E{[\frac{X}{Y}]} \approx \frac{\E{[X]}}{\E{[Y]}}-\frac{Cov(X,Y)}{\E{[Y]}^2}+\frac{\Var{(Y)}}{\E{[Y]}^3}$. Since $\Var{(B)}=\Var{(B')}$, the differences between the approximated $\E{[A/B]}$ and $\E{[A/B']}$ is only in the denominators which only differs by constant $\Phi$. Therefore, giving that $\E{[(\nabla F) u_t]}$ only differs by constant terms, and $\E{\norm{u_t}^2}$ are approximately similar, we believe qualitatively there is no large difference in the convergence rate between the two algorithms under such analysis settings.


\begin{proposition}[Convergence of DP-AdamBC and DP-Adam With momentum]
    Given the above assumptions and the defined update rule with $0 < \beta_2 < 1$, $0 \leq \beta_1 < \beta_2 $, $\eta_t = \eta (1-\beta_1) \sqrt{\frac{1-\beta_2^t}{1-\beta_2}}, \eta>0$, $\forall \alpha$ s.t. $0<\alpha<1$, we have $\forall T \in \sN^*$ such that $T>\frac{\beta_1}{1-\beta_1}$ and with $\Tilde{T}=T-\frac{\beta_1}{1-\beta_1}$, let $\nu^{*} = (2\beta_2^2\Phi\sqrt{(\beta_2^t-1)/ (\beta_2^2-1)})$, $b^{*}=4\beta_2\Phi$, \\
    \\
    (DP-Adam)
    \begin{align*}
        &\E{[\norm{\nabla F(\theta_t)}^2]} \leq \frac{2\delta(F_0-F_*)}{\eta \Tilde{T}} \\ &+ E\bigg( \ln{\bigg( 1+\frac{\delta^2}{\epsilon(1-\beta_2)} \bigg)} -T\log{(\beta_2)}\bigg),
    \end{align*}
    \begin{align*}
        E &= \frac{\eta d L(1-\beta_1)\delta}{(1-\beta_1/\beta_2)(1-\beta_2)} + \frac{2\eta^2 d L^2 \beta_1}{(1-\beta_1/\beta_2)(1-\beta_2)^{3/2}} \\&+ \frac{12d\delta^2\sqrt{1-\beta_1}}{(1-\beta_1/\beta_2)^{3/2}\sqrt{1-\beta_2}},
    \end{align*}
    \\
    (DP-AdamBC)
    \begin{align*}
        &\E{[\norm{\nabla F(\theta_t)}^2]} \leq \frac{2\sqrt{\delta^2 - \Phi}(F_0-F_*)}{\eta\Tilde{T}}\\ &+ E\bigg( \ln{\bigg( 1-\frac{\delta^2}{\Phi(1-\beta_2)} \bigg)} -T\log{(\beta_2)}\bigg),
    \end{align*}
    \begin{align*}
        E &= \frac{\eta d L(1-\beta_1)\sqrt{\delta^2 - \Phi}}{(1-\beta_1/\beta_2)(1-\beta_2)} + \frac{2\eta^2 d L^2 \beta_1}{(1-\beta_1/\beta_2)(1-\beta_2)^{3/2}} \\&+ \frac{12d(\delta^2 - \Phi)\sqrt{1-\beta_1}}{(1-\beta_1/\beta_2)^{3/2}\sqrt{1-\beta_2}},
    \end{align*}
    \begin{align*}
        \delta \geq
        \begin{cases}
            \mu^{*} + \sqrt{\ln{(1/\frac{\alpha}{2T})}(2\nu^{*2})}
            & 0 \leq \delta  \leq \frac{\nu^{*2}}{b^{*}} \\
            \mu^{*} + \ln{(1/\frac{\alpha}{2T})}2b^{*}
            & \delta   \geq \frac{\nu^{*2}}{b^{*}}.
        \end{cases}
    \end{align*}
\end{proposition}

\paragraph{DP-Adam.}
\begin{proof}
    Let $G_t = \nabla F(\theta_{t-1})$, $g_t = \nabla f_t^{p}(\theta_t-1)$, by Assumption \ref{assmption:f_smooth} we have,
    \[
        F(\theta_t) \leq F(\theta_{t-1}) - \eta_t G_t^T u_t + \frac{\eta_t^2 L}{2} \norm{u_t}^2, \; u_t = \frac{m_t^{p}}{\sqrt{v_t^{p}+\epsilon}},
    \]
    where $m_t^{p}$ and $v_t^{p}$ are the first and second moment estimated from privatized gradient which makes $u_t$ the update of DP-Adam with momentum. Taking the expectation over past steps we get,
    \[
        \E{[F(\theta_t)]} \leq \E{[F(\theta_{t-1})]} - \eta_t \E{[G_t^T u_t]} + \frac{\eta_t^2 L}{2} \E{[\norm{u_t}^2]}.
    \]
    We bound $\E{[G_t^T u_t]}$ using a similar approach as in Lemma A.1 \citep{défossez2022simple} with the following key steps. For index $0 \leq k \leq t-1$, we first decompose $G_t^T u_t$ as,
    \begin{align*}
        \sum_{i\in[d]}G_{t,i}\frac{m_{t,i}^{p}}{\sqrt{v_{t,i}^{p}+\epsilon}} &= \underbrace{\sum_{i\in[d]} \sum_{k=0}^{t-1} \beta_1^{k}G_{t-k, i}\frac{g_{t-k,i}}{\sqrt{v_{t,i}^{p}+\epsilon}}}_{A} \\ &+ \underbrace{\sum_{i\in[d]} \sum_{k=0}^{t-1} \beta_1^{k}(G_{t, i}-G_{t-k, i})\frac{g_{t-k,i}}{\sqrt{v_{t,i}^{p}+\epsilon}}}_{B}.
    \end{align*}
    We first bound $B$ with the Gaussian concentration bound on $v_{t,i}^{p}$ using the similar approach as in Equation (A.13) in \citet{défossez2022simple}. Let $\lambda = \frac{\sqrt{1-\beta_1}}{2\delta}$ where $\delta$ is in the form as in Lemma \ref{lemma:gaussian_concentration_on_z}, $x = |G_{t,i}-G_{t-k, i}|$, $y=\frac{g_{t-k,i}}{\sqrt{v_{t,i}^{p}+\epsilon}}$, following the same steps we get,
    \begin{align*}
        |B| &\leq \frac{\eta_t^2 L^2\sqrt{1-\beta_1}}{4\delta}\Big( \sum_{l=1}^{t-1} \norm{u_{t-l}}^2 \sum_{k=l}^{t-1} \beta_1^k \sqrt{k} \Big)\\ &+ \frac{\delta}{\sqrt{1-\beta_1}}\Big( \sum_{k=0}^{t-1} (\frac{\beta_1}{\beta_2})^k \norm{U_{t-k}}^2 \Big).
    \end{align*}
   %  Since the noises are sampled from an unbounded Normal distribution, i.e. $v_t^{p}$ is only bounded under expectation, we need to take the iterated expectation on the variable conditioned on noise first then take the expectation with respect to noise, i.e. $\E{[B]} = \E_Z{[\E_{B|Z}{[B|Z]}]}$. Applying Jensen's inequality with $f(x)=1/\sqrt{x}$ we have,
   %  \[
   %      \E{\bigg[\sum_{i\in[d]} \sum_{k=0}^{t-1} \beta_1^{k}(G_{t, i}-G_{t-k, i})\frac{g_{t-k,i}}{\sqrt{v_{t,i}^{p}+\epsilon}}\bigg]} \geq \sum_{i\in[d]} \sum_{k=0}^{t-1} \beta_1^{k}(G_{t, i}-G_{t-k, i})\frac{g_{t-k,i}}{\E{[\sqrt{v_{t,i}^{p}+\epsilon}]}}.
   %  \]
   %  % Let $U_{t}=\frac{g_t}{\sqrt{v_t^p+\epsilon}}$, with $\E{[g_t^2]} \leq C^2  + \Phi$ and $\epsilon + v_{t, i}^{p} \geq \epsilon + \beta_2^k v_{t-k, i}^{p} \geq \beta_2^k (\epsilon+v_{t-k, i}^{p})$ such that $U_{t-k,i}^2 \leq \frac{1}{\beta_2^k} U_{t-k,i}^2$, following the same steps as in \citep{défossez2022simple} we have,
   %  % \[
   %  %     |B| \leq \sqrt{1-\beta_1}\frac{\eta_t^2 L^2}{4\sqrt{C^2+\Phi}} \bigg( \sum_{l=1}^{t-1} \norm{u_{t-l}}^2 \sum_{k=l}^{t-1}\beta_1^{k}\sqrt{k} \bigg) + \sqrt{\frac{C^2+\Phi}{1-\beta_1}} \bigg( \sum_{k=0}^{t-1}\big(\frac{\beta_1}{\beta_2}\big)^k \sqrt{k+1} \norm{U_{t-k}}^2 \bigg).
   %  % \]
   %  Since $\E{[\sqrt{v_{t,i}^{p}+\epsilon}]} \leq \sqrt{(t+1)(C^2+\Phi)}$, let $\lambda = \frac{\sqrt{1-\beta_1}}{2\sqrt{(t+1)(C^2+\Phi)}}$, $x=|G_{t,i}-G_{t-k,i}|$, $y=\frac{|g_{t-k,i}|}{\sqrt{v_{t,i}^{p}+\epsilon}}$, following the same steps as in \citep{défossez2022simple} we have,
   % \begin{align*}
   %      |B| \leq \frac{\sqrt{1-\beta_1}\eta_t^2 L^2}{4\sqrt{(t+1)(C^2+\Phi)}} \bigg( \sum_{l=1}^{t-1} \norm{u_{t-l}}^2 \sum_{k=l}^{t-1}\beta_1^{k}\sqrt{k} \bigg) + \sqrt{\frac{(t+1)C^2+\Phi}{1-\beta_1}} \bigg( \sum_{k=0}^{t-1}\big(\frac{\beta_1}{\beta_2}\big)^k\norm{U_{t-k}}^2 \bigg).
   % \end{align*}
    Let $\Tilde{v}_{t, k}^{p} = \beta_2^{k} v_{t-k} + \E{[\sum_{j=t-k+1}^{t} \beta_2^{t-j} g_{j}^2]}$ be the second moment estimate with last $k$ gradients replaced by their expected value, since by definition $\Tilde{v}_{t, k+1}^{p} + \epsilon \geq \E{[\sum_{j=t-k}^{t} \beta_2^{t-j}g_{j}^2]}$ and $v_{t}^{p} + \epsilon \geq \sum_{j=t-k}^{t} \beta_2^{t-j}g_{j}^2$ and with the result of Lemma \ref{lemma:gaussian_concentration_on_z}, following the same steps as in \citep{défossez2022simple},
    % and $\sum_{j=t-k}^{t} \beta_2^{t-j} g_{j}^2\leq (k+1)(C + \sqrt{-\ln{\frac{\alpha}{2}}(2\Phi)})^2 $,
    % \begin{align*}
    %     &\E{[A]} \geq \frac{1}{2} \bigg( \sum_{i\in[d]} \sum_{k=0}^{t-1} \beta_1^k \E{\bigg[ \frac{G_{t-k,i}^2}{\sqrt{\Tilde{v}_{t, k+1,i}+\epsilon}} \bigg]}  \bigg)\\ &- 2\sqrt{\frac{C + \sqrt{-\ln{\frac{\alpha}{2}}(2\Phi)}}{1-\beta_1}} \bigg( \sum_{i\in[d]} \sum_{k=0}^{t-1} \big(\frac{\beta_1}{\beta_2}\big)^k \sqrt{k+1}\E{[\norm{U_{t-k}}^2]} \bigg).
    % \end{align*}
    \begin{align*}
        &\E{[A]} \geq \frac{1}{2} \bigg( \sum_{i\in[d]} \sum_{k=0}^{t-1} \beta_1^k \E{\bigg[ \frac{G_{t-k,i}^2}{\sqrt{\Tilde{v}_{t, k+1,i}+\epsilon}} \bigg]}  \bigg)\\ &- 2\sqrt{\frac{\delta}{1-\beta_1}} \bigg( \sum_{i\in[d]} \sum_{k=0}^{t-1} \big(\frac{\beta_1}{\beta_2}\big)^k\E{[\norm{U_{t-k}}^2]} \bigg).
    \end{align*}
    Then combing the results for $A$ and $B$ gives the bound on $\E{[G_t^T u_t]}$. Let $\Omega_t=\sqrt{\sum_{j=0}^{t-1}\beta_2^{j}}$, by Assumption \ref{assmption:f_bounded_below}, summing over all steps $t$ and reorganizing the terms we get,
    % \begin{align*}
    %     &\frac{\sum_{t=1}^{T}\frac{\eta_t}{\Omega_t} \sum_{k=0}^{t-1}\beta_1^k \E{[\norm{G_{t-k}}^2]}}{2(C + \sqrt{-\ln{\frac{\alpha}{2T}}(2\Phi)})} \leq F(\theta_0) - F_{*} \\ &+ \frac{\eta_N^2L}{2}\sum_{t=1}^{T} \E{[\norm{u_t}^2]} + \frac{\eta_{T}^3L^2 \sqrt{1-\beta_1}}{4(C + \sqrt{-\ln{\frac{\alpha}{2T}}(2\Phi)})}\sum_{t=1}^{T} \sum_{l=1}^{t-1} \beta_1^k\sqrt{k} \\& + \frac{3\eta_{T}(C + \sqrt{-\ln{\frac{\alpha}{2T}}(2\Phi)})}{\sqrt{1-\beta_1}}\sum_{t=1}^{T}\sum_{k=0}^{t-1}\big(\frac{\beta_1}{\beta_2}\big)^k \sqrt{k+1} \E{[\norm{U_{t-k}}^2]}.
    % \end{align*}
    \begin{align*}
        &\frac{\sum_{t=1}^{T}\frac{\eta_t}{\Omega_t} \sum_{k=0}^{t-1}\beta_1^k \E{[\norm{G_{t-k}}^2]}}{2\delta} \leq F(\theta_0) - F_{*} \\ &+ \frac{\eta_N^2L}{2}\sum_{t=1}^{T} \E{[\norm{u_t}^2]} + \frac{\eta_{T}^3L^2 \sqrt{1-\beta_1}}{4\delta}\sum_{t=1}^{T} \sum_{l=1}^{t-1} \beta_1^k\sqrt{k} \\& + \frac{3\eta_{T}\delta}{\sqrt{1-\beta_1}}\sum_{t=1}^{T}\sum_{k=0}^{t-1}\big(\frac{\beta_1}{\beta_2}\big)^k \sqrt{k+1} \E{[\norm{U_{t-k}}^2]},
    \end{align*}
    where $\delta$ is in the form as in Corollary \ref{lemma:union_bound_on_v}.
    Bounding $\E{[\norm{u_t}^2]}$ is similar to the steps in Lemma A.2 \citep{défossez2022simple} which we have,
    \begin{gather*}
        \sum_{t=1}^{T} \E{[\norm{u_t}^2]} \leq \frac{\sum_{i \in [d]} \ln{\big( 1+\frac{v_{T,i}}{\epsilon} \big) - T \log{(\beta_2)}}}{(1-\beta_1)(1-\beta_1/\beta_2)} ,\\
        v_{T,i} \leq \frac{(C + \sqrt{-\ln{\frac{\alpha}{2T}}(2\Phi)})^2}{1-\beta_2}.
    \end{gather*}
    The rest of the proof rearranges the other terms with techniques including changing index and order of summation and is exactly the same as in \citep{défossez2022simple} which leads to the final result.
    %, with $\Tilde{T}=T-\frac{\beta_1}{1-\beta_1}$.
    % \begin{align*}
    %     &\E{[\norm{\nabla F(\theta_t)}^2]} \leq \frac{2(C + \sqrt{-\ln{\frac{\alpha}{2}}(2\Phi)})(F_0-F_*)}{\eta\Tilde{T}} \\ &+ E\bigg( \ln{\bigg( 1+\frac{(C + \sqrt{-\ln{\frac{\alpha}{2}}(2\Phi)})^2}{\epsilon(1-\beta_2)} \bigg)} -T\log{(\beta_2)}\bigg),
    % \end{align*}
    % \begin{align*}
    %     E &= \frac{\eta d L(1-\beta_1)(C + \sqrt{-\ln{\frac{\alpha}{2}}(2\Phi)})}{(1-\beta_1/\beta_2)(1-\beta_2)} \\&+ \frac{2\eta^2 d L^2 \beta_1}{(1-\beta_1/\beta_2)(1-\beta_2)^{3/2}} \\&+ \frac{12d(C + \sqrt{-\ln{\frac{\alpha}{2}}(2\Phi)})^2\sqrt{1-\beta_1}}{(1-\beta_1/\beta_2)^{3/2}\sqrt{1-\beta_2}}.
    % \end{align*}
\end{proof}

\paragraph{DP-AdamBC.}
\begin{proof}
    We start with,
    \begin{gather*}
        \E{[F(\theta_t)]} \leq \E{[F(\theta_{t-1})]} - \eta_t \E{[G_t^T u_t]} + \frac{\eta_t^2 L}{2} \E{[\norm{u_t}^2]}, \\
        u_t = \frac{m_t^{p}}{\sqrt{v_t^{p}-\Phi}},
    \end{gather*}
    where $u_t$ is the update of DP-AdamBC with momentum. To bound $\E{[G_t^T u_t]}$ we first decompose the quantity as follows,
    \begin{align*}
        \E{[G_t^T u_t]} &= \sum_{i\in[d]} \E{\bigg[ \frac{G_{t,i}m_{t,i}^p}{\sqrt{v_{t,i}^p - \Phi}} \bigg]} \\&= \sum_{i\in [d] } \E{\bigg[ \frac{G_{t,i}m_{t,i}^c}{\sqrt{v_{t,i}^p - \Phi}} \bigg]} + \sum_{i\in[d]} \underbrace{\E{\bigg[ \frac{G_{t,i}\sum_{j=1}^{t}\beta_1^{t-i}z_{t,i}}{\sqrt{v_{t,i}^p - \Phi}} \bigg]}}_{0} \\
        &= \underbrace{\sum_{i\in [d] } \sum_{k=0}^{t-1} \beta_1^{k} G_{t-k,i} \E{\bigg[ \frac{\bar{g}_{t-k,i}}{\sqrt{v_{t,i}^p - \Phi}} \bigg]}}_{A} \\&+ \underbrace{\sum_{i\in [d] } \sum_{k=0}^{t-1} \beta_1^{k} (G_{t, i}-G_{t-k,i}) \E{\bigg[ \frac{\bar{g}_{t-k,i}}{\sqrt{v_{t,i}^p - \Phi}} \bigg]}}_{B}.
    \end{align*}
    Let $\Tilde{v}_{t, k}^{p} = \beta_2^{k} v_{t-k} + \E{[\sum_{j=t-k+1}^{t} \beta_2^{t-j} g_{j}^2]}$ be the second moment estimate with last $k$ gradients replaced by their expected value, and substituting result from Lemma \ref{lemma:gaussian_concentration_on_z},
    % since by definition $\Tilde{v}_{t, k+1}^{p} -\Phi \leq (k+1)(C + \sqrt{-\ln{\frac{\alpha}{2}}(2\Phi)})^2 - \Phi$, bounding $A$ and $B$ are similar as above except for substituting this condition,
    % \begin{align*}
    %     &\E{[G_t^T u_t]} \geq \frac{1}{2}\bigg( \sum_{i \in [d]} \sum_{k=0}^{t-1} \beta_1^k \E{\bigg[ \frac{G_{t-k,i}}{\sqrt{\Tilde{v}_{t,k+1,i}-\Phi}}\bigg]} \bigg) \\ &- \frac{\sqrt{1-\beta_1}\eta_t^2L^2}{4\sqrt{(C + \sqrt{-\ln{\frac{\alpha}{2}}(2\Phi)})^2 - \Phi}}\bigg( \sum_{l=1}\norm{u_{t-l}}^2\sum_{k=1}^{l-1}\beta_1^k\sqrt{k} \bigg) \\&- \frac{3\sqrt{(C + \sqrt{-\ln{\frac{\alpha}{2}}(2\Phi)})^2 - \Phi}}{\sqrt{1-\beta_1}} \bigg( \sum_{k=0}^{t-1} \big(\frac{\beta_1}{\beta_2} \big)^2 \sqrt{k+1} \norm{U_{t-k}}^2\bigg).
    % \end{align*}
    \begin{align*}
        &\E{[G_t^T u_t]} \geq \frac{1}{2}\bigg( \sum_{i \in [d]} \sum_{k=0}^{t-1} \beta_1^k \E{\bigg[ \frac{G_{t-k,i}}{\sqrt{\Tilde{v}_{t,k+1,i}-\Phi}}\bigg]} \bigg) \\ &- \frac{\sqrt{1-\beta_1}\eta_t^2L^2}{4\sqrt{\delta^2 - \Phi}}\bigg( \sum_{l=1}\norm{u_{t-l}}^2\sum_{k=1}^{l-1}\beta_1^k\sqrt{k} \bigg) \\&- \frac{3\sqrt{\delta^2 - \Phi}}{\sqrt{1-\beta_1}} \bigg( \sum_{k=0}^{t-1} \big(\frac{\beta_1}{\beta_2} \big)^2 \norm{U_{t-k}}^2\bigg),
    \end{align*}
    where $\delta$ is in the form as in Lemma \ref{lemma:gaussian_concentration_on_z}.
    Let $\Omega_t=\sqrt{\sum_{j=0}^{t-1}\beta_2^{j}}$, by Assumption \ref{assmption:f_bounded_below}, summing over all steps $t$ and reorganizing the terms we get,
    % \begin{align*}
    %     &\frac{\sum_{t=1}^{T}\frac{\eta_t}{\Omega_t} \sum_{k=0}^{t-1}\beta_1^k \E{[\norm{G_{t-k}}^2]}}{2\sqrt{(C + \sqrt{-\ln{\frac{\alpha}{2T}}(2\Phi)})^2 - \Phi}} \leq F(\theta_0) - F_{*} \\&+ \frac{\eta_N^2L}{2}\sum_{t=1}^{T} \E{[\norm{u_t}^2]} \\&+ \frac{\eta_{T}^3L^2\sqrt{1-\beta_1}}{4\sqrt{(C + \sqrt{-\ln{\frac{\alpha}{2T}}(2\Phi)})^2 - \Phi}}\sum_{t=1}^{T} \sum_{l=1}^{t-1} \beta_1^k\sqrt{k} \\
    %     &+ \frac{3\eta_{T}\sqrt{(C + \sqrt{-\ln{\frac{\alpha}{2T}}(2\Phi)})^2 - \Phi}}{\sqrt{1-\beta_1}}\sum_{t=1}^{T}\sum_{k=0}^{t-1}\big(\frac{\beta_1}{\beta_2}\big)^k \sqrt{k+1} \E{[\norm{U_{t-k}}^2]}.
    % \end{align*}
    \begin{align*}
        &\frac{\sum_{t=1}^{T}\frac{\eta_t}{\Omega_t} \sum_{k=0}^{t-1}\beta_1^k \E{[\norm{G_{t-k}}^2]}}{2\sqrt{\delta^2 - \Phi}} \leq F(\theta_0) - F_{*} \\&+ \frac{\eta_N^2L}{2}\sum_{t=1}^{T} \E{[\norm{u_t}^2]} + \frac{\eta_{T}^3L^2\sqrt{1-\beta_1}}{4\sqrt{\delta^2 - \Phi}}\sum_{t=1}^{T} \sum_{l=1}^{t-1} \beta_1^k\sqrt{k} \\
        &+ \frac{3\eta_{T}\sqrt{\delta^2 - \Phi}}{\sqrt{1-\beta_1}}\sum_{t=1}^{T}\sum_{k=0}^{t-1}\big(\frac{\beta_1}{\beta_2}\big)^k \E{[\norm{U_{t-k}}^2]},
    \end{align*}
    where $\delta$ is in the form as in Corollary \ref{lemma:union_bound_on_v}.
    Bounding $\E{[\norm{u_t}^2]}$ is similar to the steps in Lemma A.2 \citep{défossez2022simple} which we have,
    \begin{gather*}
        \sum_{t=1}^{T} \E{[\norm{u_t}^2]} \leq \frac{\sum_{i \in [d]} \ln{\big( 1-\frac{v_{T,i}}{\Phi} \big) - T \log{(\beta_2)}}}{(1-\beta_1)(1-\beta_1/\beta_2)} , \\
        v_{T,i} \leq \frac{(C + \sqrt{-\ln{\frac{\alpha}{2T}}(2\Phi)})^2}{1-\beta_2}.
    \end{gather*}
    The rest of the proof rearranges the other terms with techniques including changing index and order of summation and is exactly the same as in \citep{défossez2022simple} which leads to the final result.
    %, with $\Tilde{T}=T-\frac{\beta_1}{1-\beta_1}$,
    % \begin{align*}
    %     &\E{[\norm{\nabla F(\theta_t)}^2]} \leq \frac{2\sqrt{(C + \sqrt{-\ln{\frac{\alpha}{2}}(2\Phi)})^2 - \Phi}(F_0-F_*)}{\eta\Tilde{T}}\\ &+ E\bigg( \ln{\bigg( 1-\frac{(C + \sqrt{-\ln{\frac{\alpha}{2}}(2\Phi)})^2}{\Phi(1-\beta_2)} \bigg)} -T\log{(\beta_2)}\bigg),
    % \end{align*}
    % \begin{align*}
    %     E &= \frac{\eta d L(1-\beta_1)\sqrt{(C + \sqrt{-\ln{\frac{\alpha}{2}}(2\Phi)})^2 - \Phi}}{(1-\beta_1/\beta_2)(1-\beta_2)} \\ &+ \frac{2\eta^2 d L^2 \beta_1}{(1-\beta_1/\beta_2)(1-\beta_2)^{3/2}} \\&+ \frac{12d((C + \sqrt{-\ln{\frac{\alpha}{2}}(2\Phi)})^2 - \Phi)\sqrt{1-\beta_1}}{(1-\beta_1/\beta_2)^{3/2}\sqrt{1-\beta_2}}.
    % \end{align*}

\end{proof}


\section{Limitations}
We observe that DP-AdamBC improves performance of DP-Adam in the cases where both algorithms outperform DP-SGD, such as in text classification tasks with SNLI and QNLI. In cases where DP-SGD outperforms DP-Adam, such as in image classification with CIFAR10 (Figure \ref{fig:exp1}) and in node classification with obgn-arxiv (as reported in \citet{daigavane2022nodelevel}), DP-AdamBC tends to perform similarly to DP-Adam, with minor advantages.
%
Although the observed DP bias is quite concentrated around its mean $\Phi$, we note that $\gamma'$ in DP-AdamBC is an important hyperparameter that affects the choice of learning rate $\eta$ and affects the final performance. As such, our results are dependent on our efforts tuning parameters for each algorithm.
%
However, this also opens avenues for improvement. Since $\gamma'$ concentrates over steps $t$, we could apply a decreasing schedule for $\gamma'$ (and $\eta$, since a smaller learning rate is typically needed for smaller $\gamma'$) following the bound of Propositions \ref{prop:bound_fixed} and \ref{prop:bound_martingale} (and confirmed in the numerical analysis in \S \ref{apdix:bound}).

\end{document}