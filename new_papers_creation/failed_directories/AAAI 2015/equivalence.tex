\section{A Related Discrete Game}
\label{sec:discretization}

The game in its current formulation may seem somewhat hard to reason about, due
to large (continuous) strategy spaces.\footnote{In particular, the game is
clearly not normal form. Hence, we cannot directly apply Nash's theorem
about existence of a mixed equilibrium. We defer treatment of such
equilibria to future study.} To simplify our analysis, we use the
following discrete game, which can be thought of as imposing 
a specific pricing mechanism given vendors' selection of
items.

\begin{definition}[The price-moderated VC game]
\label{def:pm-vc}
Given a buyer valuation function over the vendors' items, consider the 
following two-round process:
\begin{enumerate}
\item Each vendor $i \in [k]$ commits to offering
a subset of $S_i \subseteq A_i$ of items; this is its (discrete) strategy;
\item Given strategy vector $\mathbf{S}=(S_1,\ldots,S_k)$, item prices are set to their marginal values. I.e., if we set $S^* = \bigcup_{i=1}^kS_i$, then for each $a \in S^*$, the mechanism will set $\tilde p (a)=m_a(S^* \setminus \{ a \})$. For each item $a' \notin S^*$ the mechanism sets $\tilde p(a') = v(A^*)+1$. Let $\mathbf{\tilde p}$ be the resulting price vector.
\end{enumerate}
The consumer then buys the set $X(v;\mathbf{\tilde p})$, as before. 
We call the resulting game a \emph{price-moderated vendor competition} game, or more succinctly, a \emph{PMVC} game.
\end{definition}
By analogy to our definitions for the original game, let $X'(v;\mathbf{S})$ denote the set of items sold, given the strategy profile $\mathbf{S}$; i.e, given the price vector $\mathbf{\tilde p}$ imposed by the pricing mechanism in the second round, $X'(v;\mathbf{S}) = X(v;\mathbf{\tilde p})$. Similarly, define a vendor's utility to be $u_i'(S_i,\mathbf{S_{-i}})$, for $i \in [k]$.

Note that the specified pricing ($v(A^{*}+1)$) of items not offered (i.e., not in $S^*$) ensures that the consumer will never buy them (i.e., $X(v;\mathbf{\tilde p}) \subseteq S^*$).
Further observe that the set of price vectors $\mathbf{\tilde p}$ that correspond to the discrete strategy profiles $\mathbf{S}$ in the PMVC game is a strict subset of the strategy space in the original VC game.
We justify our use of this game in our analysis by establishing the relationship between the original VC game and the proposed PMVC game, using a number of straightforward results.

\commentout{
Our following observation trivially follows from the fact that the induced strategy space of the PMVC game, resulting from translating the discrete strategy profiles to their corresponding price vectors as described above, can be viewed as subset of the strategy space of the original game.
\begin{observation}
Every resulting price vector $\mathbf{\tilde p}$ in the PMVC game, corresponding to a strategy profile $\mathbf{S}=(S_1,\ldots,S_k)$, can be translated to a strategy profile in the original VC game by setting the price of each sold item $a \in S^*$ to $m_{a}(S^*\setminus \{a\})$, and the price of every unsold item $a ' \notin S^*$ to be $v(A^*)+1$.
\end{observation}
}

\paragraph{Assumption} 

For ease of exposition, we assume that the buyer is maximal.
As we shall see, this implies that $X'(v;\mathbf{S})=X(v;\mathbf{p})$.
However, we can adapt the pricing mechanism by judiciously setting the prices
to be slightly below the marginal contributions to ensure maximality
(we leave the details of such a
modification to an expanded version of the paper).  

We now
describe an important relationship between the VC and PMVC games that
simplifies our subsequent analysis by relating our original model to a simpler discrete game:
\begin{proposition}
\label{prop:discrete}
For every strategy profile $\mathbf{p}$ in the VC game and valuation
$v$, there is a strategy
profile $\mathbf{S}$ in the PMVC game such that $X'(v;\mathbf{S}) =
X(v;\mathbf{p})$, and $u'_i(S_i,\mathbf{S_{-i}}) \geq
u_i(p_i,\mathbf{p_{-i}})$ for each vendor $i$.
\end{proposition}
\begin{proof}
%\jo{Begin Joel's alternative proof.}
Let $\mathbf{p}$ be a strategy profile in the VC game, and let $T=X(v;\mathbf{p})$. Consider the strategy profile $\mathbf{S}$ where $S_i = X(v;\mathbf{p}) \cap A_i$, for $i=1,\ldots,k$, and let $\mathbf{\tilde p}$ be the resulting price vector imposed by the pricing mechanism. Furthermore, we let $\tilde T=X'(v;\mathbf{S})$. We begin by showing that $T=\tilde T$. First, notice that, as for all $a \notin T$, $\tilde p(a)=v(A^*)+1$, and hence item $a$ is not sold, and $\tilde T \subseteq T$. Next, suppose for the sake of contradiction that $\tilde T \subsetneq T$, and let $a \in T \setminus \tilde T$. By the submodularity of the function $v(\cdot)$, we have that $m_a(\tilde T) \geq m_a(T) \geq 0$. This implies that 
\small
\begin{align*}
u_b(\tilde T \cup a, \mathbf{\tilde p})
&= v(\tilde T \cup a) - \sum_{a' \in \tilde T}\tilde p(a') - m_a(T\setminus a) \\
&\geq v(\tilde T \cup a) - \sum_{a' \in \tilde T}\tilde p(a') - m_a(\tilde T)= u_b(\tilde T,\mathbf{\tilde p}).
\end{align*}
\normalsize
By maximality, the buyer would rather buy item $a$ as well, resulting in a contradiction.

We now claim that $u'_i(S_i,\mathbf{S_{-i}}) \geq u_i(p_i,\mathbf{p_{-i}})$. This follows from the fact that marginal contributions are the maximal prices at which the buyer still buys $X'(v;\mathbf{\tilde p})$. That is, any increase in the price would result in the buyer not buying the product: %we have that (again by maximality)
\small
\begin{align*}
 u_b(\tilde T, \mathbf{\tilde p}) 
&= v(\tilde T) - \sum_{a' \in \tilde T \setminus a}\tilde p(a') - m_a( T \setminus a)\\
&= v(\tilde T \setminus a ) - \sum_{a' \in \tilde T \setminus a}\tilde p(a') = u_b(\tilde T \setminus a, \mathbf{\tilde p})
\end{align*}
%\jo{Begin Omer's original proof.-- commented-out.}
\commentout{Every state in the general game induces a set of sold items $B\subseteq A$. 
We claim pricing all elements $b\in B$, $m_{b}(B)$, which is what happens in the discrete case, will only increase each players profit, so that the discrete game, where every player sells $B_{i}=B\cap A_{i}$ is a discrete game state ensuring each player at least as much utility as it got in its general game state. 
Obviously, every unsold item $a\in A\setminus B$ has $p_{a}>m_{a}(B\cup \{a\})$, otherwise it would have been sold (as it would have meant $v(B\cup \{a\})-p(B)-p_{a}>v(B)-p(B)$), contradicting the set $B$ being that of the sold items. However, if the price of items $b\in B$ were such that $p_{b}(B)<m_{b}(B)$, these prices could be increased to $m_{b}(B)$ and the items would still be sold (as $v(B)-p(B)-(m_{b}(B)-p_{b}(B))\geq v(B\setminus \{b\})-p(B\setminus \{b\})$), and the profit for the seller would increase.}
\normalsize
\end{proof}
Similarly to the previous proposition, which offered a mapping of strategy profiles in a way that does not cause the vendor's utilities to deteriorate, we now show that the same mapping also preserves Nash equilibria in cases where such equilibria exist. We note that the following result uses similar arguments to those given by Babaioff et al. for proving a related characterization of pure Nash equilibria.
\begin{theorem}
\label{thm:NE}
For every pure Nash equilibrium $\mathbf{p}$ of a VC game there is a pure Nash equilibrium $\mathbf{S}=(S_1,\ldots,S_k)$ in the corresponding PMVC game, such that: (1) $X'(v;\mathbf{S})=X(v;\mathbf{p})$; and (2) for all $a \in X(v;\mathbf{p})$, $\tilde p(a)=p(a)$, where $\mathbf{\tilde p}$ is the induced price vector for $\mathbf{S}$. %, and (3) every price vector induced by a Nash equilibrium of the PMVC game is a Nash equilibrium in the original VC game.
\end{theorem}
\begin{proof}
For convenience, let $B=X(v;\mathbf{p})$. As before, we let the strategy profile in the corresponding PMVC game be $\mathbf{S}=(S_1,\ldots,S_k)$, where $S_i = B \cap A_i$, for $i=1,\ldots,k$.

We begin by proving part (2) of the theorem. Suppose that there is an item $a\in B$ such that $p(a) \neq m_{a}(B \setminus a)$. If $p(a) > m_{a}(B) = v(B)-v(B\setminus a)$, then $v(B\setminus a)- p(B\setminus a)>v(B)-p(B)$, implying that the buyer would not buy item $a$, contradicting our assumption that $a\in B$.

Assume now that $p(a) < m_{a}(B)$. Letting $\mathbf{p'}$ denote the vector resulting by replacing $p(a)$ in $\mathbf{p}$ with $m_a(B \setminus a)$, we clearly have that $u_b(B, \mathbf{p'}) = u_b(B \setminus a, \mathbf{p'})$. We now prove the following claim:
\begin{claim}
 $u_b(B\setminus a,\mathbf{p'}) \geq u_b(T,\mathbf{p'})$, for all $T \subseteq B \setminus a$.
\end{claim}
(Proof omitted due to space constraints.)
\commentout{
\begin{proof}
Let $(B \setminus a ) \setminus T = \{c_1,\ldots,c_m\}$, and let $P_t=(B \setminus a) \setminus \{c_1,\ldots,c_t\}$ for all $t=1,\ldots,m$. Then $u_b(B \setminus a,\mathbf{p'}) \geq u_b(P_t,\mathbf{p'})$.
We prove the claim inductively. Suppose the claim is true for $t < m$: $u_b(B \setminus a,\mathbf{p'}) \geq u_b(P_t,\mathbf{p'})$. We now show that the same inequality holds for $t+1$ as well:
\small
\begin{align*}
 u_b(P_{t+1},\mathbf{p'}) &= %v(P_{t+1}) - p'(P_{t+1}) =
 v(P_{t}) - p'(P_t) - (m_{c_{t+1}}(P_{t+1}) - p(c_{t+1})) \\
&\leq u_b(P_{t},\mathbf{p'}) - (m_{c_{t+1}}(B) - p(c_{t+1})) \leq u_b(P_t, \mathbf{p'})
\end{align*}
\normalsize
where the first inequality follows from submodularity, and the second inequality follows from the fact that $p(c_{t+1}) \leq m_{c_{t+1}}(B)$ as we have previously shown.
\end{proof}
}

Therefore, the vendor who owns $a$ can increase his payoff by setting the price of item $a$ to any value between $p(a)$ and $m_a(B)$, contradicting the equilibrium state.

%If we change $p_{b}$ to $m_{b}(B)$, buyers would still buy it, as it is still worthwhile for them (i.e., $v(B)-p(B)\geq v(B\setminus \{b\})-p(B\setminus \{b\})$), and as the profit for the seller would increase (as other sold items have a price below or equal to $m_{i}(B)$, they will still be sold), contradicting that $(p_{1},\ldots,p_{n})$ is a Nash equilibrium.

What is left to prove is that $\mathbf{S}$ is a Nash equilibrium in the PMVC game. Note that we can assume w.l.o.g.\ that the price of all products which are not sold is $v(A^*)+1$, as they remain unsold and continue to contribute nothing to the buyer or seller. Now, suppose $\mathbf{S}$ is not a Nash equilibrium, and that there is a player $i$, which can benefit from changing his set of sold items from $S_i$ to $S'_i$, which would result in a different vector of induced prices $\mathbf{\tilde p'}=(\tilde p'_i, \mathbf{\tilde p'_{-i}})$. We now argue that vendor $i$ can make an identical improvement in his revenue by changing his price vector from $p_i$ to $\tilde p'_i$, contradicting $p$ being a Nash equilibrium. For convenience, we let $B'=(B \setminus S_i) \cup S'_i$, and $B'' = X(v;\tilde p'_i,\mathbf{p_{-i}})$.

To show this, first notice that no other vendor would sell any previously unsold items as a result; that is, $X(v;\tilde p'_i,\mathbf{p_{-i}}) \setminus A_i \subseteq X(v;p_i,\mathbf{p_{-i}}) \setminus A_i$ (since prices of items in $(A^*\setminus A_{i})\cap  X(v;p_i,\mathbf{p_{-i}})$ are still $v(A^*)+1$). So $B''=X(v;\tilde p'_i,\mathbf{p_{-i}}) \subseteq X'(v;S'_i,\mathbf{S'_{-i}})=B'$. Thanks to submodularity, we have that for every $a \in S'_i$, $m_{a}(B')<m_{a}(B'')$. Arguments similar to the ones given above (on $p(a)=m_{a}(B)$) imply that player $i$ would sell all the items in $S'_i$, and as the prices are unchanged from the PMVC game, will make the same profit as in the PMVC game. As this increases the player's profit in the PMVC game, it would increase its profit in the VC game as well, in contradiction to $p$ being a Nash equilibrium. 
%Now, assume there is a Nash equilibrium in the discrete game, for a set of products $B$ being sold. If it isn't a Nash equilibrium in the general game, that means, according to the previous proposition, that there is a player $i$ that can change its strategy (by that inducing a game with sold items $B'$), and we know from the previous proposition that there is a state in the discrete game selling $B'$ and giving at least the same profit to player $i$. But since the change from $B$ to $B'
\end{proof}

\paragraph{Discussion}
Note that we have not shown an exact equivalence between the two games: the set of Nash equilibria in the VC game is a subset of the equilibria in the PMVC game.
However, Proposition~\ref{prop:discrete} and Theorem~\ref{thm:NE} allow us to reason about our original game to a considerable extent. 

In contrast to the original model of Babaioff et al.~in which $n_i=1$ for all $i=1,\ldots,k$, we can show that in our more general game, there may not always be a pure Nash equilibrium. In order to do so, we provide an example of a VC game in the next section with two vendors who each control two items. We show that this game does not admit any pure Nash equilibrium by relating to its corresponding PMVC game, using Theorem~\ref{thm:NE}.
%\jo{Flesh out the complete explanation to the PoA/PoS results.}
Moreover, if we restrict ourselves to VC games that do admit pure Nash equilibria, we can provide quantitative bounds on their quality. Specifically, when restricting ourselves to VC games that have pure Nash equilibria, we provide a
lower bound on the price of stability of the PMVC game by analyzing an instance of the game. As the optimal objective value (the valuation of the set that is bought by the buyer) is always $v(A^*)$, Theorem~\ref{thm:NE} immediately implies that the same lower bound applies to the VC game. To complement lower bound, we provide an upper bound for the price of anarchy, also ensuring tightness of bounds.

%%% Local Variables: 
%%% mode: latex
%%% TeX-master: "vendor_competition"
%%% End: 



