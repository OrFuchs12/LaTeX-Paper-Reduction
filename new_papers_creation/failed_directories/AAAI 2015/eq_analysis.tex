\section{Equilibrium Analysis}

Previously we outlined
several properties of the discrete PMVC game. We now describe
how the PMVC game can serve as a surrogate to help analyze
the stability of the VC game, and its quality of equilibria when they exist.

\subsection{Existence of pure Nash equilibria}
We begin by showing that, as opposed to the special case where each vendor owns a single item, some instances of our game may not actually admit pure Nash equilibria. 
\begin{proposition}
  \label{prop:no_PNE}
There exists an instance of the VC game with two vendors, where $n_1=n_2=2$, that does not admit a pure Nash equilibrium.
\end{proposition}
%\jo{Consider moving the (rather tedious) proof to the appendix and just provide an overview in the paper itself}
\begin{proof}
  Let $A_1=\{a,b\}$ and $A_2=\{c,d\}$. We define the buyer's
  valuation function $v$ according to Table~\ref{tab:ce-valuation} (the
  value in each cell is the valuation of the union of the
  sets given at the head of the entry's row and column).
\commentout{
  $v(\emptyset ) = 0,
   v(\{b\})=2.503,
   v(\{d\})=2.703,
  v(\{c\})=2.803,
  v(\{a\})=3.203,
  v(\{c,d\})=4.1045,
  v(\{a,b\})=4.4045,
  v(\{b,d\})=5.204,
  v(\{a,d\})=v(\{b,c\})=5.304,
  v(\{a,c,d\})=v(\{a,b,d\})=6.5045,
  v(\{a,b,c\})=v(\{b,c,d\})=6.5045,$ and $v(\{a,b,c,d\})=7.6045$.
}


\begin{table}[ht]
%\begin{minipage}[t]{.3\linewidth}
\begin{tabular}{@{}lllll@{}}
\toprule
            & $\emptyset$ & $\{c\}$ & $\{d\}$ & $\{c,d\}$ \\ \midrule
$\emptyset$ & 0           & 2.8   & 2.7   & 4.1    \\
$\{a\}$     & 3.2       & 5.4   & 5.3   & 6.5    \\
$\{b\}$     & 2.5       & 5.3   & 5.2   & 6.6    \\
$\{a,b\}$   & 4.4      & 6.6  & 6.5  & 7.6    \\ \bottomrule
\end{tabular}
\caption{The buyer's valuation function}
\label{tab:ce-valuation}
%\end{minipage}
\commentout{
\qquad\quad\quad
\begin{minipage}[t]{.3\linewidth}
%\centering
\begin{tabular}{@{}lllll@{}}
\toprule
            & $\emptyset$ & $\{c\}$       & $\{d\}$       & $\{c,d\}$   \\ \midrule
$\emptyset$ & (0,0)       & (0,2.8)     & (0,2.7)    & (0,2.7)   \\
$\{a\}$     & (3.2,0)   & (2.6,2.201) & (2.6,2.1) & (2.4,2.3) \\
$\{b\}$     & (2.5,0)   & (2.5,2.8) & (2.5,2.7) & (2.5,2.7) \\
$\{a,b\}$   & (3.1,0)   & (2.5,2.2)   & (2.5,2.1)   & (2.1,2.1)   \\ \bottomrule
\end{tabular}
\caption{The payoffs for each vendor.}
\label{ce-payoffs}
\end{minipage}
}
\end{table}
It is easy to verify that $v$ is (strictly) non-decreasing and
submodular. Now, consider the PMVC game with the same item sets and
valuation function $v$. For each strategy profile $(S_1,S_2)$,
the mechanism prices items according to their marginal
contributions (Definition~\ref{def:pm-vc}). Therefore,
vendor payoffs are the sum of the prices of their offered items. 
The vendors' payoffs for each strategy profile are easily calculable from the table (omitted due to space constraints), and it is evident from them that there is no pure Nash equilibrium in the PMVC
game. Theorem~\ref{thm:NE} then implies our proposition.
\commentout{
The vendors' payoffs for each strategy profile are given in
Table~\ref{ce-payoffs} (the first entry corresponds to the row
player, Vendor~1, and the second to the column player, Vendor~2). As is evident from
Table~\ref{ce-payoffs}, there is no pure Nash equilibrium in the PMVC
game. Theorem~\ref{thm:NE} then implies our proposition.}
\end{proof}

\subsection{How bad can equilibria be?}

Given the negative nature of Proposition~\ref{prop:no_PNE},
we now 
% one way to extend our study of the game, would be to take the following
% approach. Suppose we 
restrict attention to the subclass of VC
games that \emph{do} admit pure Nash equilibria,
and ask whether reasonable
guarantees on social welfare in such equilibria can be derived

More formally, let $\mathcal{G}=\{G=(v,(A_1,\ldots,A_k)): \exists
\text{ a pure Nash equilibrium in G} \}$ be the set of VC
games which admit a pure Nash equilibrium. 
Define the \emph{price of anarchy (PoA)}
as follows:
\small
\[ PoA_{\mathcal{G}} = \max_{G \in
  \mathcal{G}}\frac{\max_{\mathbf{p^*}}f(\mathbf{p^*})}{\min_{\mathbf{p}:\mathbf{p}\text{
    is a pure Nash equilibrium}}f(\mathbf{p})}\]
    \normalsize
PoA is a commonly used worst-case measure of
the efficiency of the equilibria, and in our case reflects
the efficiency loss in
$\mathcal{G}$ resulting from the introduction of strategic
pricing, as opposed to using a ``centrally coordinated'' pricing policy.

\begin{theorem}
  Define the set of VC games $\mathcal{G}_{m}$, such
  that $G \in \mathcal{G}_m$ iff (1)
  $G$ has a pure Nash equilibrium, and
  (2) $max_{i=1}^k|A_i|=m$. Then the PoA of $\mathcal{G}_m$ is at most
  $H_m+1$, where $H_m$ is the $m$'th harmonic number.
\end{theorem}
\begin{proof}
  Consider a game $G=(v, \mathbf{A}=(A_1,\ldots,A_k)\}$ in  $\mathcal{G}_m$. It is enough to provide a lower bound on the minimal social
  welfare of a pure Nash equilibrium in the corresponding PMVC game: by
  Theorem~\ref{thm:NE}, this will establish a lower bound on the
  social welfare of a pure Nash equilibrium in $G$ as well. So let
  $\mathbf{S}=(S_1,\ldots,S_k)$ be a pure Nash equilibrium of the
  PMVC game. As $v(\cdot)$ is non-decreasing, we can assume w.l.o.g. that $|S_i|=\{a_i\}$, for some $a_i \in A_i$.

  Again by the assumption that $v(\cdot)$ is non-decreasing, we know
  that optimal social welfare is obtained when all of $A^*$ is
  sold, so it is enough to upper bound $v(\mathbf{A})$ in terms of $v(\mathbf{S})$.

  We now show the following straightforward bound on the social
  welfare resulting from switching from $S_i$ to $A_i$:
  \begin{lemma}
\label{lem:poa_bound1}
\small
    $v(A_i,\mathbf{S_{-i}}) \leq v(\emptyset,\mathbf{S_{-i}}) + H_{n_i} (v(S_i,\mathbf{S_{-i}}) -
    v(\emptyset,\mathbf{S_{-i}}))$, for all $i=1,\ldots,k$.
    \normalsize
  \end{lemma}
  \begin{proof}
  As $\mathbf{S}$ is a Nash equilibrium, the profit from selling $A_{i}$ is higher than selling any set $B\subseteq A_{i}$. Using the definition of the pricing mechanism of the PMVC game, we know
\small  \[ v(S_i,\mathbf{S_{-i}}) - v(\emptyset, \mathbf{S_{-i}}) \geq
  \sum_{b \in B}m_b(B \setminus b,\mathbf{S_{-i}}), \quad\quad
  \text {for all } B \subseteq A_i \]
\normalsize  
  By an averaging argument, this means that for all $B \subseteq A_i$, there exists an item $b \in
B$ such that
\small
\begin{align}
\label{eq:bound1}
\frac{1}{|B|} (v(S_i,\mathbf{S_{-i}}) - v(\emptyset,
\mathbf{S_{-i}})) \geq m_b(B \setminus b,\mathbf{S_{-i}})
\end{align}
\normalsize
The above implies that there is a relabelling of the items in $A_i$,
so that: (1) $A_i=\{b_1,\ldots,b_{n_i}\}$, (2) $b_1 = a_1$, and (3) if
set $P_t = \{b_1,\ldots,b_t\}\cup \mathbf{S_{-i}}$ and $P_0=\mathbf{S_{-i}}$, the following holds:
\small
\begin{align*}
  v(A_i,\mathbf{S_{-i}}) &% v(\emptyset,\mathbf{S_{-i}})+\sum_{t=1}^{n_i}m_{b_t}(P_{t-1})
  \leq v(\emptyset,\mathbf{S_{-i}})+\sum_{i=1}^{n_i}\frac{1}{t}(v(S_i,\mathbf{S_{-i}}) - v(\emptyset,\mathbf{S_{-i}}))\\
& = v(\emptyset,\mathbf{S_{-i}})+H_{n_i}(v(S_i,\mathbf{S_{-i}}) - v(\emptyset,\mathbf{S_{-i}}))
\end{align*}
\normalsize
where the first equality follows from a simple telescopic series, and
the first inequality follows from Eq.~\ref{eq:bound1}.
  \end{proof} %% Lemma ends
Next, we show the following useful bound:
\begin{lemma}
\label{lem:poa_bound2}
  $\sum_{i=1}^k v(A_i,\mathbf{S_{-i}}) \geq v(A_i, \mathbf{A_{-i}}) +
  (k-1)v(S_i, \mathbf{S_{-i}}) $
\end{lemma}
\begin{proof}
  $L^{(t)}=(A_1,\ldots,A_t,S_{t+1},\ldots,S_k)$, for $t=1,\ldots,k$,
  and $L^{(0})=\mathbf{S}$. That is, $L^{(t)}$ is the strategy profile
  resulting from replacing the length-$t$ prefix of $\mathbf{S}$ with
  that of $\mathbf{A}$. 
  We prove by induction that
\small  \[\sum_{i=1}^t v(A_i,\mathbf{S_{-i}}) \geq v(L^{(t)}) +
  (t-1)v(\mathbf{S})\]\normalsize
and the lemma would follow by setting $t=k$.
  
  The inequality clearly holds for $t=1$, due to the monotonicity of
  $v(\cdot)$. Assume  that the inequality holds for $t<k$. Thus, for $t+1$,
  we have:
\small  \[ \sum_{i=1}^{t+1} v(A_i,\mathbf{S_{-i}}) \geq v(L^{(t)}) +
  (t-1)v(\mathbf{S}) + v(A_{t+1},\mathbf{S_{-(t+1)}})\]\normalsize
By the second definition of submodularity, we know $v(L^{(t)}) +
v(A_{t+1},\mathbf{S_{-(t+1)}}) \geq v(L^{(t+1)}) +v(\mathbf{S})$. Putting this in the preceding inequality concludes the proof.
\end{proof}
We can also prove an upper bound on the optimal social welfare in terms
of the social welfare of $\mathbf{S}$. By the above two lemmas, we
get:
\small
\begin{align*}
 & v(\mathbf{A}) \leq \sum_{i=1}^k v(A_i,\mathbf{S_{-i}})- (k-1)v(\mathbf{S})\\
%  &\leq \sum_{i=1}^kv(\emptyset,\mathbf{S_{-i}}) +
 % \sum_{i=1}^k H_{n_i}(v(S_i,\mathbf{S_{-i}}) -
  %v(\emptyset,\mathbf{S_{-i}})) - (k-1)v(\mathbf{S})\\
& \leq \sum_{i=1}^kv(S_i,\mathbf{S_{-i}}) +
  \sum_{i=1}^k H_{n_i}(v(S_i,\mathbf{S_{-i}}) -
  v(\emptyset,\mathbf{S_{-i}})) - (k-1)v(\mathbf{S})\\ 
&= v(\mathbf{S})
  +\sum_{i=1}^k H_{n_i}(v(S_i,\mathbf{S_{-i}}) - v(\emptyset,\mathbf{S_{-i}})),
\end{align*}
\normalsize
where third inequality follows from monotonicity of
$v(\cdot)$.

%Letting $Q^{(i)}=\{a_1,\ldots,a_i\}$ and $Q^{(0)}=\emptyset$, we get
By submodularity and that $n_i\leq m$ for $i=1,\ldots,k$, 
\small
\begin{align*}
%  v(\mathbf{A}) &\leq v(\mathbf{S}) + H_{m} \sum_{i=1}^{k}m_{a_i}(Q^{(i)}) \\
  v(\mathbf{A}) &\leq v(\mathbf{S}) + H_{m}\sum_{i=1}^{k}(v(S_i,\mathbf{S_{-i}}) - v(\emptyset,\mathbf{S_{-i}})) \\
&\leq v(\mathbf{S}) + H_m v(\mathbf{S})= v(\mathbf{S})(H_m +1),
\end{align*}
\normalsize
which establishes our upper bound on the PoA. 
\end{proof}
We also give an example of a game with
a pure Nash equilibrium that matches the above bound.
\begin{theorem}
\label{prop:POA-lb}
There exists a game in $\mathcal{G}_{m}$ with a price of anarchy
  of $H_m$.
\end{theorem}
\begin{proof}
Our counter-example is obtained by making the bound of
Lemma~\ref{lem:poa_bound1} tight. Consider a game $G =
(v,\mathbf{A}=(A_1,\ldots,A_k))$, in which $|A_i|=m$, for
$i=1,\ldots,k$.

We define the valuation function as follows. For a strategy
profile $\mathbf{T}=(T_1,\ldots,T_k)$, we set
$v(\mathbf{T})=\sum_{i=1}^k\ell(T_i)$, where $\ell(T_i)=0$ if
$|T_i|=0$, and otherwise we set $\ell(T_i)=H_{|T_i|}$. Observe that the vendors are all symmetric, and that
furthermore, the payoffs only depend on their own prices.

We now consider the following strategy profile $\mathbf{p}$.
Pick an arbitrary item
$a_i$ from each $A_i$, for $i=1,\ldots,k$, and set $p(a_i)=1$. Price the
remaining items at $v(A^*)+1$. Note that the payoff
of each vendor is precisely $1$ (for non-maximal buyers, $a_i$ prices can be decreased by a small $\epsilon$).
%\footnote{If the buyer is not maximal, we can decrease the prices of the $a_i$'s by some small $\epsilon$.}

It is easy to see that $\mathbf{p}$ is a pure Nash
equilibrium. Indeed, suppose that it is not, and let $i$ be an
arbitrary vendor. Then he has an
alternative pricing $p'_i \neq p_i$, such that deviating to it would
improve his payoff of $1$. Suppose that the set of items being bought under a deviation
to $p'_i$ is $B=X(v;p'_i,\mathbf{p'_{-i}})$, such that $\sum_{a \in B}p(a)>1$. Then
there exists an item $b \in B$, such that $p(b) > 1/|B|$. But then by the
definition of the valuation function we have:
\small
\begin{align*}
  \ell(B) - p(B) = H_{|B|} - p(B \setminus b) - p(b) <
  H_{|B|-1} - p(B \setminus b)
\end{align*}
\normalsize
contradicting the assumption that the buyer buys the set $B$.
\end{proof}
Note that the above construction can be extended to show that the
price of stability (PoS) is identical:
\begin{corollary}
The price of stability of $\mathcal{G}_m$ is $\Omega(H_m)$.
\end{corollary}
\begin{proof-sketch}
  Use the construction in the proof for Proposition~\ref{prop:POA-lb},
  but set $\ell(T_i)=1$ if $|T_i|=1$, and
  $\ell(T_i)=H_{|T_i|}-\epsilon$, if $|T_i|>1$, for sufficiently small
  $\epsilon$. It is not hard to show that the aforementioned pure Nash
  equilibrium is the \emph{only} Nash equilibrium. A similar bound follows.
\end{proof-sketch}
\commentout{
\jo{Begin Omer's old proof...}
Suppose we have $m$ players, each player $i$ selling items $C_{i}=\{a^{i}_{1},\ldots,a^{i}_{n_{i}}\}$. Suppose the worst valuation comes from each player selling the set $A_{i}\subset C_{i}$. As the valuation function is monotonic, we will assume $|A_{i}|=1$, containing just item $a_{i}$, and while there can be situation where $|A_{i}|>1$, a similar solution method works for them as well.

First, note that for every $B_{i}\subseteq C_{i}$ such that $a_{i}\in B_{i}$, thanks to $a_{i}$'s being a Nash equilibrium, we know:
\begin{equation*}
\begin{split}
&v(a_{1},\ldots,a_{i-1},a_{i},a_{i+1},\ldots,a_{m})-v(a1,\ldots,a_{i-1},a_{i+1},\ldots,a_{m})\geq \\ \geq &\sum_{b\in B_{i}}v(a1,\ldots,a_{i-1},B_{i},a_{i+1},\ldots,a_{m})-v(a1,\ldots,a_{i-1},B_{i}\setminus \{b\},a_{i+1},\ldots,a_{m})
\end{split}
\end{equation*}

I.e., for every $B_{i}\subseteq C_{i}$, there is a $b\in B_{i}$ such that
\begin{equation*}
\begin{split}
&v(a1,\ldots,a_{i-1},B_{i},a_{i+1},\ldots,a_{m})-v(a1,\ldots,a_{i-1},B_{i}\setminus \{b\},a_{i+1},\ldots,a_{m})\leq \\ \leq &\frac{1}{|B_{i}|}(v(a_{1},\ldots,a_{i-1},a_{i},a_{i+1},\ldots,a_{m})-v(a1,\ldots,a_{i-1},a_{i+1},\ldots,a_{m}))
\end{split}
\end{equation*}

Hence, there are items $c^{i}_{1},\ldots, c^{i}_{|C_{i}\setminus\{a_{i}\}|}\in C_{i}\setminus\{a_{i}\}$ such that
\begin{equation*}
\begin{split}
&v(a1,\ldots,a_{i-1},C_{i},a_{i+1},\ldots,a_{m})=-v(a1,\ldots,a_{i-1},B_{i}\setminus \{b\},a_{i+1},\ldots,a_{m})=\\=&v(a_{1},\ldots,a_{i-1},C_{i},a_{i+1},\ldots,a_{m})-v(a1,\ldots,a_{i-1},C_{i}\setminus\{c^{i}_{1}\},a_{i+1},\ldots,a_{m})) + \\ +& v(a1,\ldots,a_{i-1},C_{i}\setminus\{c^{i}_{1}\},a_{i+1},\ldots,a_{m}))-v(a1,\ldots,a_{i-1},C_{i}\setminus\{c^{i}_{1},c^{i}_{2}\},a_{i+1},\ldots,a_{m}))+\ldots +\\ + & v(a1,\ldots,a_{i-1},C_{i}\setminus\{c^{i}_{1},\ldots, c^{i}_{|C_{i}\setminus\{a_{i}\}|}\},a_{i+1},\ldots,a_{m}))-v(a1,\ldots,a_{m}))+v(a1,\ldots,a_{m})) \leq \\ \leq & (\frac{1}{n_{i}}+\frac{1}{n_{i}-1}+\ldots + \frac{1}{2})(v(a_{1},\ldots,a_{i-1},a_{i},a_{i+1},\ldots,a_{m})-v(a1,\ldots,a_{i-1},a_{i+1},\ldots,a_{m})) +v(a1,\ldots,a_{m}))\approx\\ \approx& \log(n_{i})(v(a_{1},\ldots,a_{i-1},a_{i},a_{i+1},\ldots,a_{m})-v(a1,\ldots,a_{i-1},a_{i+1},\ldots,a_{m}))+v(a1,\ldots,a_{m}))
\end{split}
\end{equation*}

Thanks to sub-modularity we know that
$$
\sum_{i=1}^{m}v(a1,\ldots,a_{i-1},C_{i},a_{i+1},\ldots,a_{m})\geq v(C_{1},\ldots,C_{m})+ (m-1)v(a1,\ldots,a_{m})
$$

and therefore:
\begin{equation*}
\begin{split}
&v(C_{1},\ldots,C_{m})+ (m-1)v(a1,\ldots,a_{m})\leq \sum_{i=1}^{m}v(a1,\ldots,a_{i-1},C_{i},a_{i+1},\ldots,a_{m})\leq \\ \leq&\sum_{i=1}^{m}(\log(n_{1})(v(a_{1},\ldots,a_{i-1},a_{i},a_{i+1},\ldots,a_{m})-v(a1,\ldots,a_{i-1},a_{i+1},\ldots,a_{m}))+v(a1,\ldots,a_{m}))\\
\Rightarrow & v(C_{1},\ldots,C_{m})\leq (1+\sum_{i=1}^{m}\log(n_{i}))v(a1,\ldots,a_{m}) -\sum_{i=1}^{m}(\log(n_{i})v(a1,\ldots,a_{i-1},a_{i+1},\ldots,a_{m}))\leq \\\leq& (1+m\max\log(n_{i}))v(a1,\ldots,a_{m}) -\max\log(n_{i})\sum_{i=1}^{m}v(a1,\ldots,a_{i-1},a_{i+1},\ldots,a_{m})
\end{split}
\end{equation*}

Once again, thanks to sub-modularity:
$$
\sum_{i=1}^{m}v(a1,\ldots,a_{i-1},a_{i+1},\ldots,a_{m})\geq (m-1)v(a1,\ldots,a_{m})
$$
(this is easily seen for $m=2^{t}$ for some $t\in\mathbb{N}$, as we continuously pair elements, creating one copy of  $v(a1,\ldots,a_{m})$ and a set containing 2 fewer elements than we paired. When $m$ is different, we can pair elements to be a complement of the ``left-off'' set.)

We thus can write
\begin{equation*}
\begin{split}
&v(C_{1},\ldots,C_{m})\leq 1+m\max\log(n_{i}))v(a1,\ldots,a_{m}) -\max\log(n_{i})\sum_{i=1}^{m}v(a1,\ldots,a_{i-1},a_{i+1},\ldots,a_{m}) \leq \\ \leq &1+m\max\log(n_{i}))v(a1,\ldots,a_{m}) -\max\log(n_{i})\sum_{i=1}^{m}(m-1)v(a1,\ldots,a_{m})=\\&=1+\max\log(n_{i}))v(a1,\ldots,a_{m})
\end{split}
\end{equation*}

This is a tight bound --- consider the following case:
\begin{itemize}
\item For $B_{i}\subseteq C_{i}$, $|B_{i}|\in \{0,1\}$, $v(B_{1},\ldots,B_{m})=|\{B_{i} | |B_{i}|=1\}|$.
\item For $B_{i}\subseteq C_{i}$, $|B_{i}|\in \{0,1\}$ and $D_{k}\subseteq C_{k}$ with some $d\in D_{k}$, $v(B_{1},\ldots,D_{k},\ldots,B_{m})=|\{B_{i} | |B_{i}|=1\}|+\frac{1}{|D_{k}|}(v(B_{1},\ldots,d,\ldots,B_{m})-v(B_{1},\ldots,B_{k-1},B_{k+1},\ldots,B_{m}))$.
\item For $D_{i}\in C_{i}$, we choose $d_{i}\in D_{i}$ and $v(D_{1},\ldots,D_{m})=\sum_{|D_{i}|>0}^{m}(v(d_{1},\ldots,d_{i-1},D_{i},d_{i+1},\ldots,d_{m}) - v(d_{1},\ldots,d_{m})) + v(d_{1},\ldots,d_{m})$.
\end{itemize}

This means taking $a_{i}\in C_{i}$, $v(a_{1},\ldots,a_{m})=m$. If every seller offers some items, profits are the same --- each seller has $1$. Selling everything results in $\sum_{i=1}^{m}\log(n_{i})+m$.

This is the same price of stability -- using the same example, we tweak the second item to read For $B_{i}\subseteq C_{i}$, $|B_{i}|\in \{0,1\}$ and $D_{k}\subseteq C_{k}$ with some $d\in D_{k}$, $v(B_{1},\ldots,D_{k},\ldots,B_{m})=|\{B_{i} | |B_{i}|=1\}|+\frac{1}{|D_{k}|}(v(B_{1},\ldots,d,\ldots,B_{m})-v(B_{1},\ldots,B_{k-1},B_{k+1},\ldots,B_{m}))-\epsilon$. This means $v(a_{1},\ldots,a_{m})$ is the sole Nash equilibrium, while the overall value of the grand coalition barely changes for small enough $\epsilon$.
}

%%% Local Variables: 
%%% mode: latex
%%% TeX-master: "vendor_competition"
%%% End: 
