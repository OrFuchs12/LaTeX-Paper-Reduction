\begin{abstract}
We study a game with \emph{strategic} vendors (the agents) who own multiple items and a single buyer with a submodular valuation function. The goal of the vendors is to maximize their revenue via pricing of the items, given that the buyer will buy the set of items that maximizes his net payoff.% (valuation minus the prices).

 We show this game may not always have a pure Nash equilibrium, 
 in contrast to previous results for the special case where each
 vendor owns a single item. We do so by relating our game to an intermediate, discrete game
 in which the vendors only choose the available items, and their
 prices are set exogenously afterwards.

 We further make use of the intermediate game to provide tight bounds
 on the price of anarchy for the subset games that have pure Nash
 equilibria; we find that the optimal PoA reached in the previous special cases does not hold, but only a logarithmic one. Finally, we show that for a special case of submodular functions,
 efficient pure Nash equilibria always exist.
\end{abstract}
%%% Local Variables: 
%%% mode: latex
%%% TeX-master: "vendor_competition"
%%% End: 
