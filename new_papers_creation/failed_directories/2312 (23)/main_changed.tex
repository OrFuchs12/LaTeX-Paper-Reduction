%File: anonymous-submission-latex-2023.tex
\documentclass[letterpaper]{article} % DO NOT CHANGE THIS
\usepackage{aaai24}  % DO NOT CHANGE THIS
\usepackage{times}  % DO NOT CHANGE THIS
\usepackage{helvet}  % DO NOT CHANGE THIS
\usepackage{courier}  % DO NOT CHANGE THIS
\usepackage[hyphens]{url}  % DO NOT CHANGE THIS
\usepackage{graphicx} % DO NOT CHANGE THIS
\urlstyle{rm} % DO NOT CHANGE THIS
\def\UrlFont{\rm}  % DO NOT CHANGE THIS
\usepackage{natbib}  % DO NOT CHANGE THIS AND DO NOT ADD ANY OPTIONS TO IT
\usepackage{caption} % DO NOT CHANGE THIS AND DO NOT ADD ANY OPTIONS TO IT
\frenchspacing  % DO NOT CHANGE THIS
\setlength{\pdfpagewidth}{8.5in} % DO NOT CHANGE THIS
\setlength{\pdfpageheight}{11in} % DO NOT CHANGE THIS
%
% These are recommended to typeset algorithms but not required. See the subsubsection on algorithms. Remove them if you don't have algorithms in your paper.
\usepackage{algorithm}
\usepackage{comment}
% \usepackage{hyperref}
% \hypersetup{hidelinks}
\newcommand{\indep}{\perp \!\!\! \perp}
% \hypersetup{
%     colorlinks=false,
%     linkcolor=black,
%     filecolor=black,
%     urlcolor=black,
%     pdftitle={Overleaf Example},
%     pdfpagemode=FullScreen,
%     }
\usepackage{amsmath}
% \bibliographystyle{unsrt}
\usepackage{xcolor}         % colors
\newcommand{\goncalo}[1]
{\textcolor{orange}{{\bf}{\em #1}{\bf}}}
\newcommand{\abdel}[2]
{\textcolor{yellow}{{\bf}{\em #1}{\bf}}}

\usepackage{booktabs, nicematrix}

%
% These are are recommended to typeset listings but not required. See the subsubsection on listing. Remove this block if you don't have listings in your paper.
\usepackage{newfloat}
\usepackage{listings}
\DeclareCaptionStyle{ruled}{labelfont=normalfont,labelsep=colon,strut=off} % DO NOT CHANGE THIS
\lstset{%
	basicstyle={\footnotesize\ttfamily},% footnotesize acceptable for monospace
	numbers=left,numberstyle=\footnotesize,xleftmargin=2em,% show line numbers, remove this entire line if you don't want the numbers.
	aboveskip=0pt,belowskip=0pt,%
	showstringspaces=false,tabsize=2,breaklines=true}
\floatstyle{ruled}
\newfloat{listing}{tb}{lst}{}
\floatname{listing}{Listing}
%
% Keep the \pdfinfo as shown here. There's no need
% for you to add the /Title and /Author tags.
\usepackage{aaai24}  % DO NOT CHANGE THIS
\usepackage{times}  % DO NOT CHANGE THIS
\usepackage{helvet}  % DO NOT CHANGE THIS
\usepackage{courier}  % DO NOT CHANGE THIS
\usepackage[hyphens]{url}  % DO NOT CHANGE THIS
\usepackage{graphicx} % DO NOT CHANGE THIS
\urlstyle{rm} % DO NOT CHANGE THIS
\def\UrlFont{\rm}  % DO NOT CHANGE THIS
\usepackage{natbib}  % DO NOT CHANGE THIS AND DO NOT ADD ANY OPTIONS TO IT
\usepackage{caption} % DO NOT CHANGE THIS AND DO NOT ADD ANY OPTIONS TO IT
\frenchspacing  % DO NOT CHANGE THIS
\setlength{\pdfpagewidth}{8.5in} % DO NOT CHANGE THIS
\setlength{\pdfpageheight}{11in} % DO NOT CHANGE THIS
%
% These are recommended to typeset algorithms but not required. See the subsubsection on algorithms. Remove them if you don't have algorithms in your paper.
%\usepackage[options ]{algorithm2e}
%\usepackage[ruled,vlined]{algorithm2e}
\usepackage{footmisc}
\usepackage{subfigure}

\usepackage{algorithm}
\usepackage{algpseudocode}
% DISALLOWED PACKAGES
% \usepackage{authblk} -- This package is specifically forbidden
% \usepackage{balance} -- This package is specifically forbidden
% \usepackage{color (if used in text)
% \usepackage{CJK} -- This package is specifically forbidden
% \usepackage{float} -- This package is specifically forbidden
% \usepackage{flushend} -- This package is specifically forbidden
% \usepackage{fontenc} -- This package is specifically forbidden
% \usepackage{fullpage} -- This package is specifically forbidden
% \usepackage{geometry} -- This package is specifically forbidden
% \usepackage{grffile} -- This package is specifically forbidden
% \usepackage{hyperref} -- This package is specifically forbidden
% \usepackage{navigator} -- This package is specifically forbidden
% (or any other package that embeds links such as navigator or hyperref)
% \indentfirst} -- This package is specifically forbidden
% \layout} -- This package is specifically forbidden
% \multicol} -- This package is specifically forbidden
% \nameref} -- This package is specifically forbidden
% \usepackage{savetrees} -- This package is specifically forbidden
% \usepackage{setspace} -- This package is specifically forbidden
% \usepackage{stfloats} -- This package is specifically forbidden
% \usepackage{tabu} -- This package is specifically forbidden
% \usepackage{titlesec} -- This package is specifically forbidden
% \usepackage{tocbibind} -- This package is specifically forbidden
% \usepackage{ulem} -- This package is specifically forbidden
% \usepackage{wrapfig} -- This package is specifically forbidden
% DISALLOWED COMMANDS
% \nocopyright -- Your paper will not be published if you use this command
% \addtolength -- This command may not be used
% \balance -- This command may not be used
% \baselinestretch -- Your paper will not be published if you use this command
% \clearpage -- No page breaks of any kind may be used for the final version of your paper
% \columnsep -- This command may not be used
% % \newpage -- No page breaks of any kind may be used for the final version of your paper
% \pagebreak -- No page breaks of any kind may be used for the final version of your paperr
% \pagestyle -- This command may not be used
% \tiny -- This is not an acceptable font size.
% \vspace{- -- No negative value may be used in proximity of a caption, figure, table, section, subsection, subsubsection, or reference
% \vskip{- -- No negative value may be used to alter spacing above or below a caption, figure, table, section, subsection, subsubsection, or reference

\setcounter{secnumdepth}{1} %May be changed to 1 or 2 if section numbers are desired.

% The file aaai23.sty is the style file for AAAI Press
% proceedings, working notes, and technical reports.
%

% Title

% Your title must be in mixed case, not sentence case.
% That means all verbs (including short verbs like be, is, using,and go),
% nouns, adverbs, adjectives should be capitalized, including both words in hyphenated terms, while
% articles, conjunctions, and prepositions are lower case unless they
% directly follow a colon or long dash
%\title{What Happens to Fairness When Pruning Attention Heads?}
% \title{Pruning attention heads in Transformers for performance and fairness}
%\title{Towards Fairness-Aware Pruning of Transformer Heads}
\title{Fairness-Aware Structured Pruning in Transformers}
\author{
    %Authors
    % All authors must be in the same font size and format.
     Abdelrahman Zayed\textsuperscript{\rm 1,2},
     Gonçalo Mordido\textsuperscript{\rm 1,2},
     Samira Shabanian,
     Ioana Baldini\textsuperscript{\rm 3},\\
     Sarath Chandar\textsuperscript{\rm 1,2,4}
}


\affiliations{
    %Afiliations
     \textsuperscript{\rm 1}Mila - Quebec AI Institute,
    \textsuperscript{\rm 2}Polytechnique Montreal,
    \textsuperscript{\rm 3}IBM Research,
    \textsuperscript{\rm 4}Canada CIFAR AI Chair\\
    % If you have multiple authors and multiple affiliations
    % use superscripts in text and roman font to identify them.
    % For example,

    % Sunil Issar, \textsuperscript{\rm 2}
    % J. Scott Penberthy, \textsuperscript{\rm 3}
    % George Ferguson,\textsuperscript{\rm 4}
    % Hans Guesgen, \textsuperscript{\rm 5}.
    % Note that the comma should be placed BEFORE the superscript for optimum readability
     \{zayedabd,sarath.chandar\}@mila.quebec,
    \{s.shabanian,goncalomordido\}@gmail.com,
    \{ioana\}@us.ibm.com
%
% See more examples next
}

%Example, Single Author, ->> remove \iffalse,\fi and place them surrounding AAAI title to use it
\iffalse
\title{My Publication Title --- Single Author}
\author {
    Author Name
}
\affiliations{
    Affiliation\\
    Affiliation Line 2\\
    name@example.com
}
\fi

\iffalse
%Example, Multiple Authors, ->> remove \iffalse,\fi and place them surrounding AAAI title to use it
\title{My Publication Title --- Multiple Authors}
\author {
    % Authors
    First Author Name,\textsuperscript{\rm 1}
    Second Author Name, \textsuperscript{\rm 2}
    Third Author Name \textsuperscript{\rm 1}
}
\affiliations {
    % Affiliations
    \textsuperscript{\rm 1} Affiliation 1\\
    \textsuperscript{\rm 2} Affiliation 2\\
    firstAuthor@affiliation1.com, secondAuthor@affilation2.com, thirdAuthor@affiliation1.com
}
\fi


% REMOVE THIS: bibentry
% This is only needed to show inline citations in the guidelines document. You should not need it and can safely delete it.
%\usepackage{bibentry}
% END REMOVE bibentry

\begin{document}

\maketitle
% \goncalo{also inference}
%\goncalo{to the responsible use of LLMs}
\begin{abstract}
% In recent years, large language models (LLMs) have gained significant attention due to their impressive performance. However, the increasing number of parameters in these models has introduced challenges in their training and inference. Although removing model components can be seen as a solution to tackle the increasing model sizes, existing pruning methods solely focus on performance without considering an essential aspect for the responsible use of LLMs: model fairness. As the deployment of language models continues to rise, it becomes crucial to address the fairness of these models towards diverse groups, including women, Black people, LGBTQ+, Jewish communities, and others. In this work, we investigate how attention heads impact fairness and performance in pre-trained transformer-based language models. We then propose a novel method to prune the attention heads that negatively impact fairness while also retaining the heads critical for performance, \textit{i.e.} language modeling capabilities. Our approach is practical in terms of time and resources, as it does not
% require fine-tuning the final pruned and fairer model. \textit{WARNING: This work uses language that is offensive in nature.}

The increasing size of large language models (LLMs) has introduced challenges in their training and inference. Removing model components is perceived as a solution to tackle the large model sizes, however, existing pruning methods solely focus on performance, without considering an essential aspect for the responsible use of LLMs: model fairness. It is crucial to address the fairness of LLMs towards diverse groups, such as women, Black people, LGBTQ+, Jewish communities, among others, as they are being deployed and available to a wide audience. In this work, first, we investigate how attention heads impact fairness and performance in pre-trained transformer-based language models. We then propose a novel method to prune the attention heads that negatively impact fairness while retaining the heads critical for performance, \textit{i.e.} language modeling capabilities. Our approach is practical in terms of time and resources, as it does not require fine-tuning the final pruned, and fairer, model. Our findings demonstrate a reduction in gender bias by $19\%$, $19.5\%$, $39.5\%$, $34.7\%$, $23\%$, and $8\%$ for DistilGPT-2, GPT-2, GPT-Neo of two different sizes, GPT-J, and Llama $2$ models, respectively, in comparison to the biased model, with only a slight decrease in performance. \textit{WARNING: This work uses language that is offensive in nature.}



% Our results show a gender bias reduction by $19\%$, $19.5\%$, $39.5\%$, and $34.7\%$ on DistilGPT-2, GPT-2, GPT-Neo $125$M parameters, and GPT-Neo $1.3$B parameters compared to the original biased model, with a small reduction in performance. \goncalo{TODO: add concrete results here.}


%after the pruning process.
%Our pruning strategy ensures that we retain the attention heads critical for effective language modeling, while removing the heads that have an adverse effect on fairness. Furthermore, our approach is efficient in terms of time and resources, as it does not necessitate fine-tuning after the pruning process.
% The rapid advancement and remarkable performance of large language models have garnered substantial attention in recent years. However, the pursuit of ever-increasing model sizes has presented significant challenges in terms of retraining \goncalo{ $\rightarrow$ training?} or fine-tuning these models. While pruning \goncalo{ $\rightarrow$ removing model components (to avoid the re-use of the word pruning later)} has emerged as a potential solution to mitigate the growing scale of models, existing pruning methods primarily focus on \goncalo{model} performance considerations, largely neglecting a critical aspect: \goncalo{model} fairness. Given the widespread deployment of language models, it has become imperative to address the issue of fairness across various demographic groups, including women, Black individuals, LGBTQ+ communities, and Jewish communities, among others. In this work, we study the contribution of attention heads to fairness in transformer-based language models, and propose to prune the ones with a negative impact \goncalo{in fairness without compromising model performance}. This paper introduces \goncalo{ $\rightarrow$ To achieve this, we propose }a novel method for structurally pruning large language models that concurrently accounts for both performance and fairness considerations. \goncalo{Here we can be more specific about our method. In particular, we can say that we do not prune the attention heads that are important for performance, and prune the rest to improve fairness.} Our proposed approach is characterized by its efficiency in terms of time and resources, as it eliminates the need for post-pruning fine-tuning.
% % Furthermore, we also present a fresh perspective on evaluating fairness in text generation models, aiming to overcome limitations associated with existing metrics.
\end{abstract}

\section{Introduction}

% The remarkable performance of large language models \goncalo{(LLMs)} in a wide range of natural language processing (NLP) tasks has facilitated their extensive incorporation into various aspects of our daily lives \goncalo{simply say various applications instead?}\cite{liu2022brio,wang2018glue,rajpurkar2018know,rajpurkar2016squad,li2019unified,li2019dice,yu-etal-2020-named,ijcai2020p560}. Nonetheless, this advancement has been accompanied by a mounting concern surrounding the fairness and neutrality \goncalo{what is the difference between both terms?} of these models. Extensive research has highlighted the tendency of language models to generate biased outputs, raising the need for a thorough examination of biases related to gender, race, and sexual orientation \cite{nadeem2021stereoset,meade2021empirical,zayed2022deep}. These biases give rise to serious problems, such as the production of discriminatory text when language models are prompted with sentences about Arabs, resulting in sentences associated with terrorism.

The extensive adoption of large language models (LLMs) in diverse natural language processing tasks has proven highly successful, leading to their integration into various applications \cite{liu2022brio,wang2018glue,rajpurkar2018know,rajpurkar2016squad,li2019unified,li2019dice,yu-etal-2020-named,ijcai2020p560}. However, this progress has also brought up concerns about the fairness of these models. Numerous studies have revealed a troubling trend in which LLMs generate biased outputs for different genders, races, or sexual orientations \cite{nadeem2021stereoset,meade2021empirical,zayed2022deep,zayed2023should}.
These biases can give rise to serious problems, such as the generation of discriminatory text; for example,  when language models are prompted with sentences about Arabs, they produce continuations with references to terrorism \cite{nadeem2021stereoset}.
% Such biases can result in the creation of discriminatory text, particularly evident when the models are prompted with sentences related to certain groups, like Arabs, leading to the production of sentences associated with terrorism. Addressing these biases is crucial to avoid serious problems arising from the misuse of language models.

% The remarkable achievements of large language models \goncalo{LLMs} have served as a motivating force for both the research and industrial communities, driving efforts to develop even larger models trained on more extensive datasets with the aim of further leveraging their capabilities \cite{smith2022using,brown2020language,cohen2022lamda,rae2021scaling,lieber2021jurassic,hoffmann2022training}. However, the pursuit of scaling up these models has presented challenges in terms of conducting fine-tuning and even pre-training \goncalo{use training instead?} from scratch. \goncalo{Pruning the model offers a}As a potential solution to address the issue of increasing model sizes, model pruning has emerged \goncalo{remove model pruning has  emerged}. Nevertheless, existing pruning methods primarily prioritize the removal of model components that have minimal impact on performance, often overlooking considerations related to fairness implications \cite{Fan2020Reducing,voita-etal-2019-analyzing,fan-etal-2021-layer,behnke-heafield-2021-pruning,prasanna-etal-2020-bert,voita-etal-2019-analyzing}. Furthermore, these methods frequently assume that a pruned model will undergo subsequent fine-tuning, which is unrealistic given the substantial size of modern \goncalo{$\rightarrow$ current} language models.
%\goncalo{inference is also affected with the increased model sizes}
% The achievements of LLMs have sparked enthusiasm in both the research and industrial sectors to create even larger models, trained on more extensive datasets, to further harness their capabilities
To further expand their abilities, there has been a trend of increasingly larger models trained on extensive datasets \cite{smith2022using,brown2020language,cohen2022lamda,rae2021scaling,lieber2021jurassic,hoffmann2022training}. However, this pursuit of larger models has introduced challenges for training and inference. To address the issue of increasing model size, model pruning has emerged as a potential solution. Nevertheless, current pruning methods tend to focus on removing model components that have minimal impact on performance, often overlooking fairness implications \cite{Fan2020Reducing,voita-etal-2019-analyzing,fan-etal-2021-layer,behnke-heafield-2021-pruning,prasanna-etal-2020-bert,voita-etal-2019-analyzing}. Additionally, these methods frequently assume that a pruned model will undergo fine-tuning, which is becoming more and more impractical given the substantial increase in size of
modern language models. As a result, there is a need for more thoughtful pruning approaches that consider not only performance, but also model fairness.% the balance between performance and fairness considerations.

% Numerous pruning methods have extensively explored the presence of particular attention heads that hold paramount importance in preserving the model's language modeling capability, which are referred to as important heads for performance. In contrast, there exist other
%attention heads whose elimination does not significantly impact the language modeling ability, designating them as non-important heads \goncalo{for performance}. Certain studies \goncalo{citations?} have compellingly demonstrated the interpretability of these crucial \goncalo{these crucial $\rightarrow$ such important} attention heads by showing their roles in solving the downstream task. In the present study, we endeavor to extend this intriguing \goncalo{why is it intriguing? Remove this word?} notion to the domain of fairness, by identifying attention heads that bear significance in promoting fairness within the model. This pursuit involves a systematic calculation of the individual contributions made by each attention head towards fostering \goncalo{performance as well as }fairness.

Numerous pruning methods have highlighted that certain attention heads are critical for maintaining language modeling ability, while others appear superfluous to model performance \cite{voita-etal-2019-analyzing,NEURIPS2019_2c601ad9,he-choi-2021-stem,bian-etal-2021-attention,zhang-etal-2021-enlivening}. Some studies have shown that these important heads play an interpretable role in downstream tasks \cite{wang2022interpretability,voita-etal-2019-analyzing,he-choi-2021-stem}. %In our current research,
In our work, we explore the possibility of extending this concept to fairness by identifying attention heads that are responsible for promoting bias. To achieve this, we compute separate scores to quantify the contribution of each attention head toward both
performance and bias. These scores serve as our guide in selectively removing attention heads to improve fairness with minimal performance loss. Put simply, we propose to prioritize pruning the heads that contribute the most to bias, %if
given that they are not crucial for language modeling. %The contributions of this paper
Our contributions in this paper can be summarized as follows:
\begin{enumerate}

\item We investigate the impact of existing head pruning methods on bias across different language models, demonstrating that they do not enhance model fairness.
\item We quantify the effect of removing attention heads on bias in language models, and use it as a proxy for their contribution to the model's overall bias.
% \item We measure the impact of removing attention heads on bias within language models, using this as an indicator of their influence on the overall bias of the model.

\item We propose a novel structured pruning method that considers both fairness and performance. Our method avoids pruning the heads that are important for language modeling, while prioritizing pruning the heads that negatively impact fairness.
\item We conduct a comparison between our method and existing pruning techniques, revealing its superiority in terms of fairness, while %maintaining a comparable performance.
matching, and sometimes surpassing, their performance in terms of language modeling.
% \item Using LLMs of different sizes, we investigate how our bias reduction approach when applied to gender bias influences biases related to religion, race, sexual orientation, and nationality. In most cases, we observe a positive correlation between gender bias and other social biases, resulting in their reduction alongside gender bias mitigation.


 \item Using LLMs of different sizes, we examine how our bias reduction method, when applied to gender bias, impacts biases pertaining to religion, race, sexual orientation, and nationality. In most cases, we observe a positive correlation between gender bias and other social biases, resulting in their reduction alongside gender bias mitigation.
% We show that, in most of the cases, other social biases are positively correlated with gender bias, and hece they are reduced when gender biased is reduced.

% We find that other other biases

% \goncalo{We find that XXX TODO: fill in with concrete results.} %highlighting Ioana's comment so we don't forget to fill this
%the effect of our method, which primarily focuses on mitigating gender bias, on
%\item We study the effect of head removal on LLMs of different architectures and sizes, as well as how removing a specific subset of heads based on gender bias also translates to an improvement in other social biases.

% \item We study the change in the set of headfs that are responsible for bias as the The set of attention heads exerting the most influence on bias differs between different social biases.
%\item We study the effect of model size on the bias mitigation resulting from our proposed pruning method.
\end{enumerate}

% Large language models have found widespread success in various natural language processing (NLP) tasks \cite{liu2022brio,wang2018glue,rajpurkar2018know,rajpurkar2016squad,li2019unified,li2019dice,yu-etal-2020-named,ijcai2020p560}, leading to their increased integration into our everyday lives. This success has motivated both the research and industrial communities to develop even larger models trained on more extensive datasets, aiming to further harness their capabilities \cite{smith2022using,brown2020language,cohen2022lamda,rae2021scaling,lieber2021jurassic,hoffmann2022training}. However, this pursuit of bigger models has hindered the ability to do fine-tuning, let alone pre-training from scratch. Model pruning, --- along with compression, distillation, and quantization --- has emerged as a potential solution to the increase in model sizes. However, pruning methods primarily prioritize removing model components that have minimal impact on performance, without considering fairness implications \cite{Fan2020Reducing,voita-etal-2019-analyzing,fan-etal-2021-layer,behnke-heafield-2021-pruning,prasanna-etal-2020-bert,voita-etal-2019-analyzing}. Moreover, these methods often assume that a pruned model will undergo fine-tuning, which is unrealistic given the substantial size of modern language models.

% This paper aims to investigate the contribution of each attention head towards fairness \goncalo{This is just a repetition of the previous sentence} and presents a novel pruning approach that considers both language modeling ability and bias. In essence, our method prioritizes the pruning of attention heads that contribute significantly to bias, provided they are not essential for maintaining language modeling ability \goncalo{$\rightarrow$ model performance (to avoid repetition)}. The contributions offered by this research\goncalo{$\rightarrow$ Our contributions} can be summarized as follows:
% \begin{enumerate}

% \item We conduct an in-depth analysis \goncalo{in-depth analysis is a subjective opinion, which the reviewers may disagree on. Instead, we can say "We analyze the impact..."} of the impact of existing head pruning methods on five distinct social biases across various language models. Through empirical evaluations, we demonstrate that these pruning methods exacerbate \goncalo{or maintain?} the model's bias, thereby highlighting their shortcomings.
% \item We quantitatively assess the effect of attention head removal on bias within language models, utilizing this measure as a proxy for evaluating their contribution to fairness. This analysis provides valuable insights into the relationship between \goncalo{different} attention heads and bias.
% \item We propose a novel structured pruning method that takes into account both fairness and performance. Our approach incorporates specific strategies \goncalo{we should just summarize the strategies here instead of keeping it vague} to ensure that attention heads crucial \goncalo{important} for fairness are retained while optimizing \goncalo{maintaining? we are really only optimizing for fairness in the end with the aim of not decreasing performance} overall model performance.
% \item We conduct comprehensive \goncalo{also subjective} comparative evaluations between our proposed method and existing pruning techniques. The results demonstrate the superiority of our approach, showcasing its effectiveness in achieving improved fairness while simultaneously minimizing \goncalo{this is a repetition of the previous sentence, the same comment applies} the performance degradation.
% \end{enumerate}









\section{Related Work}
This section delves into a more detailed discussion of various pruning methods and the existing bias assessment metrics employed in language generation models.

\subsection{Pruning of Large Language Models}
Pruning of large language models can be split into two main categories: structured and unstructured pruning \cite{behnke2021pruning}. Structured pruning involves removing specific building blocks within the model, such as attention heads or layers, which alters the overall model structure. On the other hand, unstructured pruning is more fine-grained, entailing the removal of certain model weights  \cite{narang2017exploring,h.2018to}, while retaining the original structure of the network. Structured pruning typically leads to faster models, while unstructured pruning results in less performance degradation \cite{behnke2021pruning}. In this study, we focus on structured pruning to explore the impact of attention heads on fairness through targeted removal, which represents a relatively unexplored research avenue.

Some of the pioneering works in the application of structural pruning were conducted by \citet{voita-etal-2019-analyzing} and \citet{NEURIPS2019_2c601ad9}, where the authors explored the removal of attention heads from transformer-based models. Their findings revealed the presence of important heads in terms of performance. While the removal of important heads led to model collapse, less critical heads had minimal impact on performance. Building upon these works, \citet{he-choi-2021-stem} conducted a detailed analysis of the important heads, demonstrating their interpretable roles in task-solving.

Meanwhile, \citet{bian-etal-2021-attention} focused on investigating the non-important heads and concluded that these heads were redundant since their output exhibited a high correlation with other heads, making them inconsequential for final predictions. To address this, \citet{zhang-etal-2021-enlivening} proposed an approach for transforming non-important heads into important heads by injecting task-specific prior knowledge, thereby increasing their contribution to the output. In a separate study, \citet{sajjad2023effect} examined layer removal in BERT  \cite{devlin2018bert} with fine-tuning and showcased the importance of preserving lower layers to maintain performance. Furthermore, \citet{Fan2020Reducing} investigated layer removal without fine-tuning and achieved considerable performance preservation through the implementation of layer dropout during training. The lottery ticket hypothesis \cite{frankle2018the}, which suggests the existence of subnetworks capable of achieving comparable performance to that of the full network, has paved the way for numerous unstructured pruning techniques. For example, \citet{behnke-heafield-2020-losing} applied this principle to language models, while \citet{prasanna-etal-2020-bert} provided evidence that early-stage pruning during training outperforms post-convergence pruning.
%\goncalo{We should aim to reduce this paragraph if possible. We should also point out that most pruning methods involve fine-tuning, but not ours.}
%It is important to mention the majority of the methods discussed require fine-tuning the model after pruning.


%Additionally, \citet{fan-etal-2021-layer} introduced a pruning approach for transformer-based models that involves reducing the dimensionality of hidden representations across different layers while simultaneously maximizing the interdependence between representations in consecutive layers.

\subsection{Fairness Assessment in Text Generation Models}\label{sec:fairness_metrics}
%\goncalo{do we need to specify WEAT, CEAT, and SEAT? Are we using it later? If not, we can mention only the authors and not the methods as we have been doing so far.}
%\citet{guo2021detecting} build upon this concept by extending the analysis from static embeddings to contextualized embeddings. Similarly, \citet{may2019measuring} further expand the evaluation to sentence-level embeddings.
Metrics to assess fairness in text generation models may be classified into two main categories: intrinsic metrics and extrinsic metrics. Intrinsic metrics evaluate the model's bias independently of any downstream task. For instance, some works measure bias by analyzing the correlation between token representations of different groups and specific stereotypical associations \cite{caliskan2017semantics,guo2021detecting,may2019measuring}.  These metrics operate under the assumption that bias within language models can solely be detected through the analysis of the embedding space. Therefore, they do not rely on a specific task to evaluate the model's bias. However, it has been suggested that embedding space does not consistently align with the model's bias when deployed to solve a given task \cite{cao-etal-2022-intrinsic,delobelle-etal-2022-measuring}.

Some intrinsic metrics employ synthetic templates to measure bias based on the model's output predictions  \cite{webster2020measuring,kurita2019measuring}. For example, if the model assigns a higher likelihood to the sentence “she is a nurse”, compared to “he is a nurse”, it indicates the presence of gender bias. These templates are constrained in their coverage of stereotypical associations, resulting in divergent rankings of bias among different templates when applied to the same models \cite{delobelle-etal-2022-measuring}. While some metrics have substituted templates with crowd-sourced examples \cite{nadeem2021stereoset,nangia2020crows}, they have encountered challenges related to grammatical correctness, logical coherence, and relevance in a significant number of sentences \cite{blodgett2021stereotyping}. %\goncalo{Overall, I think we should also reduce these sentences since we will be using extrinsic metrics instead. Let's try to make related work 1 page max.}


% Other metrics, such as Stereoset \cite{nadeem2021stereoset} and CrowS-Pairs \cote{nangia2020crows}, replace templates with crowd-sourced examples. These metrics consider the model to be fair if the average likelihood of stereotypical sentences (\textit{e.g.}, “girls are softer than boys”) is equal to that of anti-stereotypical sentences (\textit{e.g.}, “girls are tougher than boys”). These intrinsic metrics provide insights into bias within the model itself, regardless of specific tasks.
%ioana@us.ibm.com
The second category of bias assessment metrics comprises extrinsic metrics, which evaluate bias within the context of a specific task. For example, metrics such as Winobias \cite{zhao-etal-2018-gender}, Winogender \cite{rudinger2018gender}, and BUG \cite{levy2021collecting} focus on measuring bias in coreference resolution. In this task, given a sentence like “The doctor told the nurse she will perform the surgery in two days”, identifying the word “nurse" as a referent for “she” indicates the presence of gender bias. Some of these metrics have a limited number of examples provided within the templates, which poses challenges in drawing definitive conclusions based on the measured bias. For instance, the Winobias \cite{zhao-etal-2018-gender}, Winogender \cite{rudinger2018gender}, and CrowS-Pairs \cite{nangia2020crows} metrics employ templates with only  $3$k, $120$, and $1.5$k examples, respectively.

% Additionally, another metric has replaced templates with prompts sourced from Wikipedia \cite{dhamala2021bold}. However, these prompts present distinct contexts for each group, making it challenging to ascertain whether the disparity in output probability is a result of bias or contextual differences. Another critical limitation found in several proposed metrics is the limited number of examples provided within the templates. This constraint poses challenges in drawing definitive conclusions based on the measured bias. For instance, the Winobias \cite{zhao2018gender}, Winogender \cite{rudinger2018gender}, and CrowS-Pairs \cite{nangia2020crows} metrics employ templates with only 120, 1.5k, and 1.6k examples, respectively.

Another extrinsic metric, BOLD \cite{dhamala2021bold}, employs $23$k prompts sourced from Wikipedia that refer to diverse groups based on gender, race, professions, religions, and political ideologies. BOLD quantifies bias by prompting the model with sentences related to different groups and examining the sentiment and toxicity of its generated continuations. However, these prompts have different contexts for each group, making it challenging to ascertain whether the disparity in output is a result of bias or contextual differences. Taking this into account, Holistic bias \cite{smith2022m} is a more recent work that provides $566$k prompts covering $13$ different social biases without contextual changes. In this paper, we use holistic bias prompts for fairness assessment as they allow for a more reliable measurement of bias within the context of text generation.
%, providing insights into potential biases exhibited by the model.
%\section{Background: Social Bias Quantification}\label{sec:background}
\section{Social Bias Assessment}\label{sec:background}

To assess the extent of social bias inherent in text generation models, we adopt the procedure in the work conducted by \citet{dhamala2021bold}. Specifically, we first prompt the model with sentences that pertain to various social groups, and then subsequently evaluate the toxicity levels exhibited in the generated output. An output is considered toxic if it refers to content that leads individuals to disengage from a discussion \cite{dixon2018measuring}. Hence, a large discrepancy in toxicity levels across different groups serves as an indicator of the model's bias.

% More formally, to quantify a certain type of social bias $B$, let's say race bias, we check the model's toxicity on all the subgroups $b$ in this bias


% More formally, to quantify the bias against a certain group $G$, for example race bias,

% We denote the group for which the bias is measured as $g$, which refers to race, religion, sexual orientaion, nationality, and gender. For each group, we have a number of subgroups $g_i$, such as gay, lesbian, straight, etc. for sexual orientation bias, where $i$ for $i$ $\in$ $\{1,2,..N\}$ represents the specific subgroup and $N$ is the total number of subgroups.
Inspired by \citet{dixon2018measuring}, our initial step for evaluating a specific form of social bias involves establishing a set of subgroups, denoted as $S$, which are deemed as relevant to measure such bias. For instance, in the context of sexual orientation bias, the set of subgroups $S$ encompasses terms like gay, lesbian, bisexual, straight, and others. We then measure the bias exhibited by the model by comparing the toxicity associated with each subgroup to the average toxicity across all subgroups, as follows:

% To measure a certain type of social bias, we first begin by defining a set of the subgroups $S$ that are included in this bias. For example, for the case of sexual-orientation bias, the set of subgroups $S$  includes subgroups such as gay, lesbian, bisexual, straight, etc.  Following \citet{dixon2018measuring}, the model's bias is the based on the difference between its toxicity for each subgroup and the average toxicity for all subgroups.
%To measure the bias against a specific identity group $g$, for example sexual orientation bias, we measure the model's toxicity to all the subgroups denoted by $g_i$ for $i $\in$ $\{1,2,..\}$ For a specific bias, for examples sexual orientation bias, denoted by As such, bias can be expressed as follows:

% E_{x \textrm{} \sim \textrm{} D }(bias(x)|do(head_h = 1) - bias(x)|do(head_h = 1))

 \begin{equation}
    bias_{\phi}(S) = E_{x \textrm{} \sim \textrm{} D }(\sum_{s \in S }|\textrm{E}_{s}tox_{\phi}(x(s)) - tox_{\phi}(x(s))|),
 \label{eq:pinned_toxicity}
 \end{equation}
% where $s$ denotes the subgroups that are targeted by the bias, $\textrm{E}_{s}(tox(s))$ represents the mean toxicity of the model's output across all subgroups, while $tox(s)$ represents the mean toxicity in output related to a specific subgroup $s$. Lower values correspond to less bias. Figure \ref{fig:bias_quantification} showcases an example of sexual orientation bias quantification.

where $tox_{\phi}(x(s))$ represents the toxicity in the continuation of a model parameterized by $\phi$  when prompted with a sentence $x(s)$ from a pool of $D$ prompts talking about a particular subgroup $s$ in the set $S$. $\textrm{E}_{s}tox_{\phi}(x(s))$ denotes the average toxicity of the model's output across all subgroups. Lower values indicate less bias. Table \ref{tab:bias_quantification} shows a simplified example of calculating sexual orientation bias with only two subgroups.

\begin{table}[h!]
\centering
\resizebox{0.47\textwidth}{!}{\begin{tabular}{llcll}
\hline
 \textbf{Prompt} & \textbf{Continuation} & \textbf{Toxicity} &  \\
\hline
\centering

  Hi, I'm a \textit{trans} father      &     I am not going to tell you& 0.6\\
  &something stupid though &   \\
  Hi, I'm a \textit{gay} father.  &     My guess is I'm a fucking& 0.8\\
  & fat nerd&\\
    \hline
\end{tabular}}
    \label{fig:bias_quantification}
\caption{Illustration of social bias assessment. The average toxicity is $(0.6+0.8)/2 = 0.7$, and hence bias is $|0.6-0.7| + |0.8-0.7| = 0.2$ following Eq. \eqref{eq:pinned_toxicity}. In this example, we focus on sexual orientation bias with two subgroups: trans and gay. }
%}
\label{tab:bias_quantification}
\end{table}
% \begin{table*}[h]
% \centering
% %\begin{adjustbox}
% \begin{tabular}{lll}
% \hline
% \textbf{Prompt}  & \textbf{Continuation}   & \textbf{Toxicity}  \\
% \hline
% \centering
% Hi, I'm a trans father & I am not going to tell you something stupid, though. & $0.6$\\
% Hi, I'm a gay father  & My guess is I'm a fucking fat nerd. & 0.8\\
% \end{tabular}
% %\end{adjustbox}
% \caption{The percentage of change in examples with negative sentiment when using random perturbation for different subgroups.
% }
% \label{fig:bias_quantification}
% \end{table*}

% \begin{figure}[h]
%      \centering
%     \centering
%     \includegraphics[width=1\linewidth]{figures/AAAI_24_bias_quantification.pdf}
%     \caption{An example illustrating how to compute the sexual orientation bias in text generation models. For simplicity, the set of subgroups $S$ consists of only two subgroups: trans and gay.}
%     \label{fig:bias_quantification}
% \end{figure}

\section{Fairness-Aware Structured Pruning}
%Suppose our intention is to remove $10$ attention heads from an attention-based language model through pruning. Existing pruning algorithms, such as the works by \citet{voita-etal-2019-analyzing} and \citet{NEURIPS2019_2c601ad9}, raise the following question: Which 10 heads have the least impact on the model's language modeling capability? Consequently, the term \textit{important heads} has emerged to denote the heads that are essential for effective language modeling.

Existing methods to prune attention heads in transformer models determine the importance of each head based solely on model performance \cite{voita-etal-2019-analyzing,NEURIPS2019_2c601ad9}. In other words, \textit{important heads} are deemed essential to maintain the model's language modeling capability and may therefore not be pruned. In this work, we recognize the equal significance of evaluating the influence of attention heads on fairness, thereby broadening the definition of important heads to encompass not only heads crucial for language modeling but also those that have a positive impact on fairness.

As a result, we propose quantifiable approximate measures for the impact of a given attention head on both the model's fairness and performance. Subsequently, these measures serve as our guiding principles in identifying and removing attention heads that have a negative impact on fairness, provided they are non-essential for language modeling. %, using our proposed fairness and performance aware (FPA) pruning.
For a given pre-trained model, our goal is to improve model fairness while maintaining as much performance as possible, without relying on fine-tuning.
% and head pruning ratio $\alpha$
%We then use these measures to guide us to prune the heads that negatively impact fairness if they are not crucial for language modeling, using our proposed performance and fairness aware (FPA) pruning.
%\subsection{Quantifying the contribution of attention heads to performance and fairness}
\subsection{Attention Head Contributions to Fairness and Performance}
We quantify the contribution of a given attention head to bias as the difference between the model’s bias before and after pruning such head. More specifically, for a model with $N_h$ attention heads, the impact of each head $h$ $\in$ $\{1, 2, .., N_h\}$ on a social group represented by set $S$, $z_{bias}$($h$,$S$), is estimated as:
\begin{equation}
z_{bias}(h,S) = bias_{\phi}(S)|do(y_h = 1) - bias_{\phi}(S)|do(y_h = 0)
\label{eq:ATE_bias}
\end{equation}
where $bias_{\phi}(S)$ represents the bias of the text generation model parameterized by $\phi$ as described in Eq. \eqref{eq:pinned_toxicity}. Additionally, $do(y_h = 1)$ and $do(y_h = 0)$, respectively, signify the presence and absence of head $h$.
In a similar vein, the impact of a head $h$ in the context of language modeling is defined as:
\begin{equation}
z_{ppl}(h) = ppl_{\phi}|do(y_h = 1) - ppl_{\phi}|do(y_h = 0)
\label{eq:ATE_perf}
\end{equation}
where $ppl_{\phi}$ refers to the perplexity of a model parameterized by $\phi$ on WikiText-2 \cite{meritypointer}.
%\goncalo{For clarity, we should include equations for bias(x) and ppl(x)}
%\goncalo{it is not clear how eq 1 relates to bias(x) here}
Using the effect of removal of a model component as a proxy of its influence on the model's output has been employed in previous studies \cite{rotman2021model}.
%This approach aligns with similar studies \cite{rotman2021model} that evaluate the impact of individual layers by assessing the effects of their removal.
However, it is important to note that the effect of removing multiple heads is not equivalent to the sum of the effects of each head removed individually due to the non-linearity of the model. % , although we approximate it as linear for the purpose of our analysis \goncalo{In other words we can say that we assume all layers are independent}.
Notwithstanding, our experimental results indicate that such simplification is a practical and effective way of estimating the impact of attention heads.


%\subsection{Fairness\goncalo{-} and Performance\goncalo{-}Aware (FPA) pruning}
\subsection{Attention Head Pruning}
Having assessed the influence of each attention head on both fairness and language modeling, we now introduce our %performance and fairness-aware (FPA) pruning approach.
fairness-aware structured pruning (FASP) method. FASP focuses on removing heads that have a negative impact on fairness while ensuring that the model's language modeling ability is minimally affected.

To determine the number of heads to keep, thereby preventing performance decline, we introduce a hyperparameter $\gamma$ representing the ratio of crucial attention heads for language modeling. For instance, $\gamma = 0.5$ means we keep the top $50\%$ of heads that positively influence performance, ranked based on Eq. \eqref{eq:ATE_perf} (lower is better). Then, the remaining heads (\textit{i.e.} the non-crucial bottom $50\%$ in terms of performance) are ranked based on their bias impact (again, lower is better) computed using Eq. \eqref{eq:ATE_bias}. For a given ratio of pruned heads, denoted by $\alpha$, we prune $\alpha$ $\times$ $N_h$ heads from the remaining non-critical heads, based on their bias scores.
%for a given ratio $\alpha$ representing the ratio of pruned heads and prune the top $\alpha\%$ heads
In the end, this sequence of steps allows us to prioritize the removal of those with the highest bias impact while mitigating the loss of language modeling ability. An overview of our method is presented in Algorithm 1.


\begin{algorithm}[H]
%\SetAlgoLined
% \KwResult{This pseudo code is used to illustrate how CP can be applied during training}
\textbf{Input:} Pre-trained model with $N_h$ attention heads, set of all heads $H$, ratio $\gamma$ of important heads for performance excluded from the pruning, ratio $\alpha$ of heads to be pruned, set S of subgroups targeted by the bias.

{
\textbf{Procedure:}
\begin{enumerate}
\item Compute $z_{ppl}(h)$ in Eq. \eqref{eq:ATE_perf} $\forall$ $h$ $\in$ $H$ on the validation set
\item Define the set of critical heads  $H'$ as the top $\gamma$ $\times$ $N_h$ heads based on $z_{ppl}(h)$
%\item Define the set of non-critical heads $H''=H-H'$ with ($1-\gamma$) $\times$ $N_h$ heads
% Exclude the top $\gamma$ $\times$ $H$ heads with the highest $z_{ppl}(h)$.\\
% TODO: Define $H'$ which is composed of the remaining ($1-\gamma$) $\times$ $H$ heads.\\
\item Compute $z_{bias}(S,h)$ in Eq. \eqref{eq:ATE_bias} $\forall$ $h$ $\in$ $H\setminus H'$ {on the validation  set}
\item Prune $\alpha$ $\times$ $N_h$ heads in  $H\setminus H'$ based on $z_{bias}(S,h)$
\end{enumerate}

\textbf{end}}
\caption{Fairness-aware structured pruning (FASP)}
\end{algorithm}

Figure \ref{fig:head_pruning_1} illustrates how FASP removes attention heads. The heads shown in black are deemed critical for language modeling and, as a result, are excluded from the pruning process. The remaining heads are depicted in various colors based on their impact on bias, with red indicating those that negatively influence fairness and green representing the heads that promote fairness.

\begin{figure}[h]
     \centering
    \centering
    \includegraphics[width=1\linewidth]{figures/AAAI_24_method_updated_4.pdf}
    \caption{Illustration of applying FASP {to a model with 6 layers and 12 heads per layer, \textit{e.g.} DistilGPT-2}. Initially, we identify and exclude the heads that significantly impact performance from the pruning process ({black squares}). Subsequently, the remaining heads are prioritized for removal based on their contribution to bias, ensuring that the heads contributing the most to bias are pruned first ({red squares}).% The highest priority heads for removal are shown in red.
    }
    \label{fig:head_pruning_1}
\end{figure}

% \begin{figure}[h]
%      \centering
%     \centering
%     \caption{Head pruning process in a distilGPT-2 model with $6$ layers and $12$ heads per layer. The heads in shown in black are essential for language modeling and are not pruned. The remaining heads are ranked based on their contribution to the model's bias. The highest priority heads for removal are (2,2), (2,3),  (4,7), (5,1), (6,1), (6,2), and (6,10), where the first value represents the layer id and the second value represents the head id.}
%     \label{fig:head_pruning_2}
% \end{figure}

% In this section, we delve into a comprehensive examination of identifying the influence of attention heads on fairness. Our primary objective is to prioritize the elimination of heads that contribute to an increase in bias within the model. Throughout this process, we carefully consider that the removal of these heads does not significantly impact the model's language modeling capability. First, we are going to explain how to quantify the contribution of different attention heads to both performance and fairness, then we will explain how to use this knowledge to prune the model's heads.
% \subsection{Finding the Impact of Attention Heads on Fairness and performance}
% Suppose our intention is to remove 10 attention heads from an attention-based language model through pruning. Existing pruning algorithms, such as the work by \citet{voita-etal-2019-analyzing} \citet{NEURIPS2019_2c601ad9}, raise the following question before proceeding: Which 10 heads have the least impact on the model's language modeling capability? Consequently, the term \textit{important heads} has emerged to denote the heads that are essential for effective language modeling.
% We recognize the equal significance of evaluating the influence of attention heads on fairness, thereby broadening the definition of \textit{important heads} to encompass not only heads crucial for language modeling but also those that have a positive impact on fairness. As a result, we establish a quantifiable measure for assessing the impact of an attention head $i$ on the model's fairness as follows:

% \begin{equation}
% E_{x}(bias(x)|do(head_i = 1) - bias(x)|do(head_i = 1))
% \label{eq:ATE_bias}
% \end{equation}

% In the given expression, $bias(x)$ represents the bias exhibited by the text generation model when utilizing a prompt sentence $x$. Additionally, $head_i = 1$ and $head_i = 0$ signify the presence and absence, respectively, of head number $i$.

% Similarly, the impact of head $i$ on language modeling is expressed as:

% \begin{equation}
% E_{x}(ppl(x)|do(head_i = 1) - ppl(x)|do(head_i = 1))
% \label{eq:ATE_perf}
% \end{equation}

% There are two caveats that warrant emphasis. Firstly, when quantifying the impact of head removal, we adopt a similar approach as other studies that assess the impact of layer removal \cite{rotman2021model}. This involves computing the difference in fairness between before and after pruning the specific head. Secondly, it is important to note that the overall effect of removing multiple heads is not equivalent to the sum of the effects of each head removed individually. This is due to the non-linearity of the model, although we approximate it as linear for the purpose of our analysis. Our findings indicate that while our assumption may not hold true, it serves as a proxy for estimating the overall impact. This approximation has been employed in previous studies as well, such as the work by \citet{rotman2021model}.
% \subsection{Fairness and Performance Aware Head Pruning (FPA pruning)}




\section{Experimental details}

This section presents an overview of our bias assessment prompts, baselines, evaluation metrics, and models used in our experiments. Our code is publicly available\footnote{\url{https://github.com/chandar-lab/FASP}}.%Then, we demonstrate that FASP distinguishes itself from conventional head pruning techniques by taking into account both performance and fairness. Furthermore, we explore whether the heads with the most significant impact on bias are consistent across various social biases. Finally, we study the impact of increasing fairness to a specific bias on other social biases when using our method.
% In this section, we describe our bias assessment prompts, baselines, and evaluations metrics used in our experiments. We show that that our FRA pruning method, unlike the existing head pruning methods, considers both performance and fairness. In addition, we discuss whether the heads with the largest effect on bias are the same for different social biases. Finally, we study the effect of model size on the change in bias due to pruning for all baselines.
\subsection{Bias Assessment Prompts}\label{prompts}
We use the prompts from the holistic bias dataset introduced by \citet{smith2022m}. This dataset comprises $566$k prompts, encompassing $13$ distinct biases, making it the most extensive bias assessment dataset available at the time of this paper's writing, to the best of our knowledge. %Notably, the holistic bias dataset overcomes many of the issues present in other templates, as discussed in the Related Work Section.
Among the $13$ biases covered in the dataset, we focus on $5$ specific biases: race ethnicity, religion, sexual orientation, gender and sex, and nationality bias. Table \ref{tab:dataset_statistics}  in the technical appendix
displays the number of prompts associated with each of these targeted biases, along with some illustrative examples of the prompts for each category. The prompts were split into validation and test sets with a ratio of $0.2$:$0.8$.
%\goncalo{I uncomment the table so we can properly cite it. To upload the paper pdf, simply cut the pages at the end with the appendix}
% We use the prompts from the holistic bias dataset by \citet{smith2022m}, which consists of $566,625$ prompts covering $13$ different biases, making it the largest dataset for bias assessment at the time of the writing of this paper. In addition to the size of the dataset, holistic bias also is free from most of the issues that exist in the current templates as discussed in the Related Work Section.  Out of the 13 different biases, we chose to five biases, namely race ethnicity, religion, sexual-orientation, gender and sex, and nationality bias. Table \ref{tab:dataset_statistics} shows that the number of prompts pertaining to each of the targeted biases, with some examples of the prompts for each group.
% \goncalo{magnitude}
% \goncalo{$\rightarrow$ gradient magnitude?}
\subsection{Baselines}\label{baselines}
% For relative comparison of our proposed pruning method, we use (1) head pruning based on magnitude \cite{han2015learning,han2015deep}, (2) head pruning based on gradient \cite{NEURIPS2019_2c601ad9}, (3) random head pruning, (4), head pruning based only on the fairness score in Eq. \eqref{eq:ATE_bias}, (5) head pruning based only on the language modeling score in Eq. \eqref{eq:ATE_perf}, (6) our FPA pruning method. For all baselines, the model does not undergo fine-tuning after pruning.
We employ the following baseline methods when evaluating our approach: (1) head pruning based on weight magnitude \cite{han2015learning,han2015deep}, (2) head pruning based on gradient magnitude \cite{NEURIPS2019_2c601ad9}, (3) random head pruning, (4) head pruning based only on the fairness score in Eq. \eqref{eq:ATE_bias}, and (5) head pruning based only on the perplexity score in Eq. \eqref{eq:ATE_perf}. %, (6) our FASP method.
%Note that the latter two baselines respectively correspond to using $\gamma = 0$ and $\gamma = 1$ in our method, and w
We refer to the latter two baselines as fairness only and performance only baselines, respectively.
We would like to highlight that the model remains unchanged and does not undergo any fine-tuning after the pruning process for all the mentioned baselines as well as our method.
\subsection{Evaluation Metrics}\label{sec:metrics}
We assess bias by examining the variation in the model's toxicity across various subgroups. For instance, when measuring religion bias, we consider differences in the model's toxicity among the different subgroups such as Muslims, Christians, Jews, and so on, as detailed in Eq. \eqref{eq:pinned_toxicity}. We use BERT for toxicity assessment, similar to the work by \citet{dhamala2021bold}. For performance assessment, we measure the model's perplexity on WikiText-2.
%\goncalo{Say that for performance we analyze the model in terms of language modeling ability using perplexity.}
% Following the work by \cite{dixon2018measuring}, bias is measured as the discrepancy on the model’s toxicity amongst different subgroups (for example Muslims, Christians, jews, atheists, etc., in case of measuring religion bias), as described in the Background Section.

\subsection{Models}\label{sec:models}

We employed $6$ pre-trained models available in Hugging Face: DistilGPT-2, GPT-2 \cite{radford2019language}, GPT-Neo \cite{gpt-neo} of two different sizes, GPT-J \cite{gpt-j}, and Llama $2$ \cite{touvron2023llama} models with $88.2$M, $137$M, $125$M, $1.3$B, $6$B, and $7$B parameters, respectively.
%In this work, we focus on the GPT family of models, given its remarkable performance and widespread popularity. In particular,
%\goncalo{Mention that the focus of this work is in the GPT family due to its impressive performance and popularity.}
% GPT-J and LLaMA with $6$ and $7$ billion parameters, respectively


% \goncalo{Did you cite distilGPT-2 before? That's why there is no citing here?}
\begin{figure*}[t]
     \centering
    \begin{subfigure}
    \centering    \includegraphics[width=0.3\linewidth]{figures/camera_Ready_gender_bias_red_DistilGPT-2_gender_and_sex_2.pdf}
     %\caption{Fairness}
     \end{subfigure}
   \begin{subfigure}
    \centering    \includegraphics[width=0.3\linewidth]{figures/camera_Ready_gender_bias_red_GPT-2_gender_and_sex_2.pdf}
     %\caption{Fairness}
     \end{subfigure}
     \begin{subfigure}
    \centering    \includegraphics[width=0.31\linewidth]{figures/camera_Ready_gender_bias_red_GPT-Neo_125M_gender_and_sex_2.pdf}
     %\caption{Fairness}
     \end{subfigure}
     \begin{subfigure}
    \centering    \includegraphics[width=0.31\linewidth]{figures/camera_Ready_gender_bias_red_GPT-Neo_1.3B_gender_and_sex_2.pdf}
     %\caption{Fairness}
     \end{subfigure}
    \begin{subfigure}
    \centering    \includegraphics[width=0.305\linewidth]{figures/camera_Ready_gender_bias_red_GPT-J_6_B_gender_and_sex_2.pdf}
     %\caption{Fairness}
     \end{subfigure}
    \begin{subfigure}
    \centering    \includegraphics[width=0.31\linewidth]{figures/camera_Ready_gender_bias_red_Llama_2_7_B_gender_and_sex_2.pdf}
     %\caption{Fairness}
     \end{subfigure}
      \begin{subfigure}
    \centering    \includegraphics[clip, trim=0cm 0.295cm 19cm 14.8cm, width=1.0\textwidth]{figures/camera_Ready_gender_bias_red_DistilGPT-2_gender_and_sex_legend.pdf}
     %\caption{Fairness}
     \end{subfigure}
    \begin{subfigure}
    \centering    \includegraphics[clip, trim=24.8cm 0.29cm 0.15cm 14.75cm, width=0.71\textwidth]{figures/camera_Ready_gender_bias_red_DistilGPT-2_gender_and_sex_legend.pdf}
     %\caption{Fairness}
     \end{subfigure}

        \caption{The percentage of change in gender bias and language modeling perplexity across DistilGPT-2, GPT-2, GPT-Neo $125$M, GPT-Neo $1.3$B, GPT-J, and Llama $2$ models, for varying pruning levels via different techniques, relative to the unpruned model. Among the methods, FASP is the only method to consistently reduce bias while upholding a relatively low language modeling perplexity.}
        \label{fig:gender_bias_pruning}
\end{figure*}
%The dashed lines display the bias and perplexity values for the original, unpruned model.
%Fairness only and performance only baselines correspond to the extreme cases where we only prune based on fairness and performance, respectively
% \goncalo{Change legend and GPT-Neo 1.3 figure to $\gamma=0.3$?} \goncalo{We should meet and discuss this legend: it should be 1 row with the methods and another row with the pruning ratios (7 cols each). Moreover, fairness only might need renaming.}
% \goncalo{legend in pdf instead of png}
%\goncalo{we use b and m in the text and B and M in the figure titles. Also GPT-Neo 1.3B ylim needs to be adjusted}
%\goncalo{Add paranthesis around $(\gamma = X)$ in the legend.}
%\goncalo{Fix the figure titles to only include the model architecture, and not also gender bias}

\begin{figure*}[t]
     \centering
    \begin{subfigure}
    \centering    \includegraphics[width=1\linewidth]{figures/head_ids_GPT-2_all_biases_1.pdf}
     %\caption{Fairness}
     \end{subfigure}
   % \begin{subfigure}
   %  \centering    \includegraphics[width=0.3\linewidth]{figures/pruned_head_ids_GPT2(17).pdf}
   %   %\caption{Fairness}
   %   \end{subfigure}
   %   \begin{subfigure}
   %  \centering    \includegraphics[width=0.3\linewidth]{figures/pruned_head_ids_GPT-Neo 125M(16).pdf}
   %   %\caption{Fairness}
   %   \end{subfigure}
        \caption{The indices of most impactful attention heads on five social biases, at a $20\%$ pruning rate ($\alpha = 0.2$). The existence of heads that offer pruning advantages to multiple social biases indicates the potential for a simultaneous positive impact on several biases through pruning.}
        \label{fig:head_ids_pruned}
\end{figure*}
%\goncalo{legend text needs to be bigger}
\begin{figure}[]
     \centering
    \begin{subfigure}
    \centering    \includegraphics[width=0.31\linewidth]{figures/corr_head_effects_different_biases_DistilGPT-2_3.pdf}
     \end{subfigure}
   \begin{subfigure}
    \centering    \includegraphics[width=0.31\linewidth]{figures/corr_head_effects_different_biases_GPT-2_2.pdf}
     \end{subfigure}
     \begin{subfigure}
    \centering    \includegraphics[width=0.31\linewidth]{figures/corr_head_effects_different_biases_GPT-Neo_125M_3.pdf}
     \end{subfigure}

     \begin{subfigure}
    \centering    \includegraphics[width=0.45\linewidth]{figures/corr_heatmap_legend.png}
     %\caption{Fairness}
     \end{subfigure}

        \caption{Pearson correlation heat maps depict the relationships among attention head scores on nationality, sexual orientation, religion, race, and gender biases, within DistilGPT-2, GPT-2, and GPT-Neo with a parameter count of $125$M. Notably, all social biases exhibit positive correlations, except religion bias, where correlations are either absent or slightly negative, varying based on the specific model.}
        \label{fig:correlation_maps}
\end{figure}
\section{Experiments}\label{exp_details}

In the following experiments, we demonstrate that FASP distinguishes itself from conventional head pruning techniques by taking into account both performance and fairness. Furthermore, we explore whether the heads with the most significant impact on bias are consistent across various social biases. Finally, we study the impact of gender bias reduction using our method on other social biases.

FASP introduces a single hyperparameter, which is the ratio of crucial heads for performance, denoted as $\gamma$ and selected based on the validation set. To identify the optimal value $\gamma^*$, we aim to minimize the model's bias while maintaining the perplexity as close as possible compared to the best pruning baseline. The search range for $\gamma$ was set to $\gamma \in \{0.2, ..., 0.7\}$. Additional details about the hyperparameters are provided in the appendix. The code appendix elaborates on dataset preprocessing, experiment procedures and analysis, and the computing infrastructure employed. All results were obtained using $3$ different seeds.

% We use a BERT model for toxicity detection, following a similar procedure to the work by  \citet{dhamala2021bold}



% For toxicity detection, we use a BERT model fine-tuned on the Jigsaw dataset for toxic comments .


%and OPT \cite{zhang2022opt}
%The holistic bias prompts were split into validation and test sets with a ratio of $0.2$:$0.8$.
%\goncalo{If 0.3 is the best, we should also test 0.2 to make sure the best hyperparameter value is not in one of the extremes}. \goncalo{We should specify here that we use $\gamma = 0.3$ in all of the experiments and recommend people use this if they cannot perform hyperparameter tuning.}


%For complete details regarding the hyperparameters used to obtain the results, please refer to the technical appendix. Additionally, for reproducibility, our code will be made publicly available. The code appendix provides further elaboration on the dataset pre-processing, the experimental procedures and analysis, and the computing infrastructure utilized during the experiments, respectively.

% We use six pre-trained models from hugging face, namely BERT, RoBERTa, GPT2, DistilBERT, DistilRoBERTa, and distilGPT-2. We split the holistic bias prompts into validation and test splits with a ratio of $0.2$:$0.8$. Our proposed method, FPA, uses a single hyperparameter, $\gamma$, which is selected based on the validation set. The criterion for determining the optimal value of $\beta$ is to minimize the model’s bias, while ensuring the same perplexity as that of magnitude pruning, through a search range of $\gamma \in \{0,0.1,..,1\}$. All the results are obtained by running the experiments for three different seeds.
% Section A.1 in the technical appendix includes all the details needed about the hyper-parameters used to obtain the results. Our code will be publicly available for reproducibility. Sections B.1, B.2, and B.3 in the code appendix provide more details about the pre-processing of the datasets, the procedure to conduct and analyze the experiments, as well as the computing infrastructure used for running the experiments, respectively.
% \paragraph{Experiment 1: Do the heads that are influential for language modeling and bias overlap?}

% \begin{figure}[h]
%      \centering
%     \centering
%     \includegraphics[width=1\linewidth]{figures/Overlap_ppl_bias_AAAI24.pdf}
%     \caption{}
%     \label{fig:scores_overlap}
% \end{figure}

% \begin{figure*}[h!]
%      \centering
%     \centering
%     \includegraphics[width=1\linewidth]{figures/bias_ppl_AAAI24(3).pdf}
%      %\caption{Fairness}
%         \caption{
%         }
%         \label{fig:data_pruning_bert}
%         %\samira{plz rephrase it}
% \end{figure*}


% \begin{figure}[h]
%      \centering
%     \centering
%     \includegraphics[width=1\linewidth]{figures/bias_ppl_AAAI24(1).pdf}
%     \caption{An illustration of the pruning of distilGPT-2 (with 72 heads) using our FASP method. Initially, we identify and exclude the heads that significantly impact performance from the pruning process. Subsequently, the remaining heads are prioritized for removal based on their contribution to bias, ensuring that the heads contributing the most to bias are pruned first. The highest priority heads for removal are shown in red. }
%     \label{fig:head_pruning_1}
% \end{figure}


\begin{figure*}[!h]
     \centering
    \begin{subfigure}
    \centering    \includegraphics[width=0.24\linewidth]{figures/camera_ready_effect_of_gender_bias_on_other_biases_DistilGPT-2_nationality.pdf}
     %\caption{Fairness}
     \end{subfigure}
     \hspace{0mm}
   \begin{subfigure}
    \centering    \includegraphics[width=0.24\linewidth]{figures/camera_ready_effect_of_gender_bias_on_other_biases_DistilGPT-2_race.pdf}
     %\caption{Fairness}
     \end{subfigure}
      \hspace{0mm}
     \begin{subfigure}
    \centering    \includegraphics[width=0.24\linewidth]{figures/camera_ready_effect_of_gender_bias_on_other_biases_DistilGPT-2_religion.pdf}
     %\caption{Fairness}
     \end{subfigure}
     \begin{subfigure}
    \centering    \includegraphics[width=0.237\linewidth]{figures/fixed_exp3_effect_of_gender_bias_on_other_biases_DistilGPT-2_sex_orien.pdf}
     %\caption{Fairness}
     \end{subfigure}
    \begin{subfigure}
    \centering    \includegraphics[clip, trim=0cm 0cm 0.25cm 0cm,,width=0.240\linewidth]{figures/camera_ready_effect_of_gender_bias_on_other_biases_GPT-2_nationality.pdf}
     %\caption{Fairness}
     \end{subfigure}
     \hspace{-2mm}
     \hspace{0.4mm}
     \hspace{0.4mm}
   \begin{subfigure}
    \centering    \includegraphics[clip, trim=0cm 0cm 0.25cm 0cm,width=0.236\linewidth]{figures/camera_ready_effect_of_gender_bias_on_other_biases_GPT-2_race.pdf}
     %\caption{Fairness}
     \end{subfigure}
     \hspace{-2.5mm}
     \hspace{0.7mm}
     \begin{subfigure}
    \centering    \includegraphics[clip, trim=0cm 0cm 0.25cm 0cm,width=0.244\linewidth]{figures/camera_ready_effect_of_gender_bias_on_other_biases_GPT-2_religion.pdf}
     %\caption{Fairness}
     \end{subfigure}
     \begin{subfigure}
    \centering    \includegraphics[clip, trim=0cm 0cm 0.25cm 0cm,width=0.242\linewidth]{figures/camera_ready_effect_of_gender_bias_on_other_biases_GPT-2_sex_orien.pdf}
     %\caption{Fairness}
     \end{subfigure}
     % \hspace{10mm}
    \begin{subfigure}
    \centering    \includegraphics[clip, trim=0cm 0cm 0cm 0cm,width=0.242\linewidth]{figures/camera_ready_effect_of_gender_bias_on_other_biases_GPT-Neo_125M_nationality.pdf}
     %\caption{Fairness}
     \end{subfigure}
      \hspace{-1mm}
   \begin{subfigure}
    \centering    \includegraphics[clip, trim=0cm 0cm 0cm 0cm,width=0.232\linewidth]{figures/camera_ready_effect_of_gender_bias_on_other_biases_GPT-Neo_125M_race.pdf}
     %\caption{Fairness}
     \end{subfigure}
      \hspace{-1.5mm}
     \hspace{3.5mm}%
     \begin{subfigure}
    \centering    \includegraphics[clip, trim=0cm 0cm 0cm 0cm,width=0.23\linewidth]{figures/camera_ready_effect_of_gender_bias_on_other_biases_GPT-Neo_125M_religion.pdf}
     %\caption{Fairness}
     \end{subfigure}
     \hspace{2.5mm}%
     \begin{subfigure}
    \centering    \includegraphics[clip, trim=0cm 0cm 0cm 0cm,width=0.237\linewidth]{figures/camera_ready_effect_of_gender_bias_on_other_biases_GPT-Neo_125M_sex_orien.pdf}
     %\caption{Fairness}
     \end{subfigure}

      \begin{subfigure}
    \centering    \includegraphics[clip, trim=0cm 0.295cm 19cm 14.8cm, width=1.0\textwidth]{figures/camera_Ready_gender_bias_red_DistilGPT-2_gender_and_sex_legend.pdf}
     %\caption{Fairness}
     \end{subfigure}
    \begin{subfigure}
    \centering    \includegraphics[clip, trim=24.8cm 0.29cm 0.15cm 14.75cm, width=0.71\textwidth]{figures/camera_Ready_gender_bias_red_DistilGPT-2_gender_and_sex_legend.pdf}
     %\caption{Fairness}
     \end{subfigure}
   \caption{An analysis on DistilGPT-2, GPT-2, and GPT-Neo (with $125$M parameters) showing the percentage of change in language modeling perplexity and nationality, race, religion, and sexual orientation biases, relative to the unpruned model, using varying pruning levels and different pruning techniques. While FASP focuses on gender bias mitigation through head pruning, it also addresses other biases whose head scores are positively correlated with gender bias scores, while maintaining robust language model perplexity.
        }
        \label{fig:effect_on_ther_biases_gpt2}
\end{figure*}
%\goncalo{Include only y-label on the left-most figure. Increase font size.}

% \goncalo{Change legend and GPT-Nep 1.3 figures to $\gamma=0.3$?}
%The evaluation of bias uses Eq. \eqref{eq:pinned_toxicity}, while language modeling performance is assessed through perplexity on the Wikitext-2 dataset.
\subsection{Experiment 1: How does FASP perform in terms of bias and language modeling compared to existing pruning methods?}
In this experiment, we conduct a comparison between our pruning technique, FASP, and common baseline pruning methods. Such comparison is carried out with respect to both gender bias and language modeling capabilities. The results depicted in Figure~\ref{fig:gender_bias_pruning} clearly indicate that FASP stands out as the sole pruning method capable of consistently reducing gender bias without perplexity overshooting. The fairness only and performance only baselines represent the extreme cases where we prune the heads based only on bias and performance, respectively. Among the evaluated methods, the performance only baseline achieves the lowest perplexity value in most of the cases, but does not lead to a consistent improvement in fairness, as expected. Following this, in order of performance, are FASP with the best $\gamma$ (\textit{i.e.} $\gamma^*)$, magnitude pruning, and gradient pruning. Magnitude pruning results in perplexity overshooting on GPT-Neo and Llama $2$ models. As anticipated, random pruning exhibits the poorest efficacy in preserving perplexity levels, often leading to model collapse.
Fairness only baseline yields superior fairness outcomes across the majority of scenarios, albeit accompanied by elevated perplexity, often surpassing acceptable levels. For all methods, overshooting perplexity or bias values beyond the depicted limits are not shown. It is important to note that in five out of the six models we examined, we identified a $\gamma^*$ value of $0.3$, suggesting that roughly $30\%$ of the heads in these models play a crucial role in language modeling. Qualitative results are provided in the technical appendix. %\goncalo{We should discuss if we want to include a figure with sensitivity analysis to gamma here.}

\subsection{Experiment 2: Are the heads responsible for bias the same across social biases?}
This experiment focuses on examining whether the attention heads that exert the most significant influence on bias are consistent across a range of distinct social biases. We start by calculating the Pearson correlation between the effects of attention heads, as outlined in Eq. \eqref{eq:ATE_bias}, across varying biases. Figure \ref{fig:correlation_maps} illustrates a consistent positive correlation among attention head effects across diverse biases, with the exception of the religion bias. For this particular bias, the correlation is either slightly negative or non-existent in relation to other biases, depending on the model under consideration. Note that we restrict the scope of this experiment to DistilGPT-2, GPT-2, and GPT-Neo $125$M parameter configurations due to resource availability.

To take a deeper look at how different heads influence different biases, Figure \ref{fig:head_ids_pruned} showcases the indices of the top $20$\% attention heads that yield the most substantial impact on five biases using GPT-2. The depiction underscores the presence of specific attention heads that manifest as influential across multiple biases, suggesting that the removal of such heads could yield simultaneous benefits for multiple biases. More specifically, attention head number $136$ stands as the sole contributor that adversely affects all social biases, whereas attention head number $133$ uniquely influences four out of the five biases under examination. Numerous other attention heads have a concurrent impact on two or three biases. This consistent pattern emerges across alternative models, as outlined in the technical appendix. Encouragingly, these findings pave the way for our subsequent experiment, which delves into the broader implications of pruning the attention heads that contribute to gender bias on other social biases.

%Following this, in order of performance, are FASP with the best $\gamma$ (\textit{i.e.} $\gamma^*)$, magnitude pruning, and gradient pruning.
% It is worth mentioning that we found $\gamma^*$ to be $0.3$ in four out of the five models that we tested, suggesting that approximately $30$$\%$ of the attention heads in these models are critical for language modeling. The technical appendix contains a discussion about how the hyperparameter $\gamma$ affects these results.

% that exceeds acceptable thresholds in multiple cases; thus, its depiction is omitted from the figure.
% On the other hand, FASP with $\gamma$ $=$ $0$, which only cares about fairness, achieves the best fairness in most of the cases, but with high perplexity that overshoots in many cases.

% \goncalo{a discussion about the fairness only baseline is missing.}

%\goncalo{Clarify that even though we introduce gamma, gamma=0.3 works well for all the tested models and hence there is no need to perform hyperparameter search. This essentially annuls the overhead introduced by this additional hyperpameter by our method. This is a highlight result and deserves its own paragraph. You can also clarify again what gamma=0 and gamma=1 represent.}

% Only head $136$ has a negative influence on all social biases, while head $133$ affects four out of the five biases under examination. Multiple attention heads simultaneously influence two or three biases. This pattern is consistent across various models, as demonstrated in the technical appendix.

% There is only one attention head that negatively impacts all the social biases, and  one attention head that impacts $4$ biases out of the $5$ that we study. Several attention heads impact $3$ and $2$ biases simultaneously. A similar trend is observed on other models, as shown in the technical appendix.


% \goncalo{Missing discussion of the figure, saying that there is one head the negatively influences all biases, and several heads that influence multiple biases (4 out of 5 or 3 out of 5.} A similar trend is also observed in additional models, as shown in the appendix.
 %\goncalo{we can include a discussion in terms of layers here: Are the most correlated heads from earlier or latter layers? Is there a pattern?}
%%\goncalo{Mention that we do not perform this test with the biggest model due to the computational burden.}
% \begin{figure*}[h]
%      \centering
%     \centering
%     \includegraphics[width=0.5\linewidth]{figures/corr_head_effects_different_biases_distilGPT-2(54).pdf}
%     \caption{Correlation heat maps depict the relationships among attention head impacts on various biases within distilGPT-2, GPT-2, and GPT-Neo with a parameter count of $125$m. Notably, all social biases exhibit positive correlations, except for the religious bias. In the case of the religion bias, correlations are either absent or slightly negative, varying based on the specific model.}
%     \label{fig:correlation_maps}
% \end{figure*}

%indices of the top $20$\% attention heads that wield the most substantial impact on five biases using GPT-2 model

%GPT2, distilGPT-2, and GPT-Neo (with $125$m parameters)
\subsection{Experiment 3: How are other social biases affected when gender bias is reduced?}

As our final experiment, we delve into the effect on other social biases when employing the FASP technique to prune attention heads based on gender bias. Figure~\ref{fig:effect_on_ther_biases_gpt2} shows that the process of pruning attention heads with the most pronounced influence on gender bias leads to a reduction in sexual orientation, race, and nationality biases. This is to be expected since all of these biases are positively correlated with gender bias, as shown in Figure~\ref{fig:correlation_maps}. Since GPT-2 and GPT-Neo exhibit a positive correlation between religion and gender bias head scores (also shown in Figure \ref{fig:correlation_maps}), pruning heads based on gender bias scores continues to diminish religion bias in these models. In contrast, DistilGPT-2 displayed a negative correlation between gender and religion bias head scores, leading to a marginal increase in religion bias when pruning based on gender bias head scores.  Other pruning methods do not lead to better fairness in the majority of cases.
%However, it is important to underscore that across all models, the religion bias absolute values are in proximity to $0$, rendering them insignificant in comparison to other biases.
% In the majority of instances, alternative pruning techniques fail to enhance fairness.

% Notably, the religion bias, which was identified as either uncorrelated or negatively correlated with other social biases, remains unaffected by the pruning process in all models except distilGPT-2 where it gets worse due to the negative correlation between gender bias and religion head scores for this model, as shown in Fig.~\ref{fig:correlation_maps}. \goncalo{This is not the case for distilGPT-2 though and this should be discussed.}

% Finally, as previously demonstrated in Experiment 1, it is important to highlight that none of the other pruning methods lead to a reduction in any of the social biases under consideration \goncalo{This is also not always the case... You can say that they do not consistently improve bias instead or that in most cases they don't improve it}.
% \begin{figure}[h]
%      \centering
%     \centering
%     \includegraphics[width=1\linewidth]{figures/distilGPT-2_all_biases.pdf}
%     \caption{}
%     \label{fig:distilGPT-2_all_biases}
% \end{figure}



% \begin{figure}[h]
%      \centering
%     \centering
%     \includegraphics[width=1\linewidth]{figures/distilGPT-2_all_biases.pdf}
%     \caption{}
%     \label{fig:scores_overlap}
% \end{figure}
% \begin{figure*}[]
%      \centering

%      \begin{subfigure}
%     \centering    \includegraphics[width=1\linewidth]{figures/gpt2_all_biases.pdf}
%      %\caption{Fairness}
%      \end{subfigure}
%         \caption{}
%         \label{fig:different_biases}
% \end{figure*}
% \begin{figure*}[]
%      \centering

%    \begin{subfigure}
%     \centering    \includegraphics[width=1\linewidth]{figures/EleutherAI_gpt-neo-125M_all_biases.pdf}
%      %\caption{Fairness}
%      \end{subfigure}
%      \begin{subfigure}
%     \centering    \includegraphics[width=1\linewidth]{figures/distilGPT-2_all_biases.pdf}
%      %\caption{Fairness}
%      \end{subfigure}
%         \caption{}
%         \label{fig:different_biases}
% \end{figure*}
% \subsection{Baselines}\label{baselines}

% \subsection{Evaluation Metrics}\label{sec:FNED_FPED}


%  \subsection{Experiment Details}\label{exp_details}


% \paragraph{Experiment 1: How does our method perform on language modeling and bias compared to existing pruning methods?}
% \paragraph{Experiment 2: Are the important heads the same for all biases?}
% \paragraph{Experiment 3: How does the pruning effect vary with laguage model size?}

\section{Conclusion}
This paper examines the impact of pruning attention heads in various language models on their fairness towards several social biases. %genders, ethnic groups, sexual-orientations, nationalities, and religions.
We highlight that current pruning techniques, which prioritize minimizing performance decline, do not take fairness into account. As a result, we propose to consider both performance and fairness considerations when pruning model components. Our experiments show that the proposed approach, FASP, consistently improves the fairness of transformer models while matching the language modeling ability of performance-based pruning methods.% a comparable level of performance degradation.

\section*{Acknowledgements}

We are thankful to Afaf Taïk for her insightful suggestions in this project. We are also thankful to the reviewers for their constructive comments. Sarath Chandar is supported by the Canada CIFAR AI Chairs program, the Canada Research Chair in Lifelong Machine Learning, and the NSERC Discovery Grant. Gonçalo Mordido is supported by an FRQNT postdoctoral scholarship (PBEEE). The project was also supported by Microsoft-Mila collaboration grant. The authors acknowledge the computational resources provided by the Digital Research Alliance of Canada.


% n a few cases.


% \section{Acknowledgments}

% \begin{enumerate}
%     \item API issues
%     \item Proxy of independent heads
%     \item The method can be used for the exact opposite (if the opposite is done)
%     \item Toxicity detection relies on a model, therefore it may have bias and inaccuracies
%     \item We assume that higher perplexity (language modeling) on WikiText should transfer to better performance in downstream tasks. But we could be using a different dataset.
% \end{enumerate}
%\bigskip
%\noindent

Iusto quo aperiam praesentium deserunt repudiandae facere perferendis, suscipit ducimus tempore tempora dignissimos?Consequuntur quos ratione in alias provident odit, deserunt delectus mollitia quasi omnis pariatur voluptas officiis ea laudantium soluta, quae eum blanditiis soluta fugit modi a ea ab maiores laborum ducimus.Dolor quam dignissimos voluptatem qui nulla laudantium ipsum consequatur corporis neque, voluptates repellat itaque enim doloremque aliquam, qui nam sapiente quae perspiciatis, praesentium dolorum voluptas.Atque animi illo at molestiae, fuga officiis maxime tempore corporis asperiores dicta rem quo, natus nisi minima similique nesciunt corrupti iste temporibus doloremque ullam quo, illum incidunt ratione nisi expedita optio, neque voluptate velit quisquam nam nisi molestias?Maiores eius quisquam deserunt, maxime excepturi recusandae velit consequatur ut fuga veniam, minima dolor quod nam, ea ullam maxime quidem laboriosam quaerat odit, possimus quos nihil praesentium minima dolorem tempora sunt vero.Maiores dolor quisquam eos error beatae adipisci accusamus, repudiandae ab illum voluptate nulla nobis sapiente eos asperiores ipsam, aspernatur harum dolor esse obcaecati cupiditate, dolorem amet aliquam quos quod sunt, iusto laudantium voluptate.Porro possimus corporis doloribus beatae quam tempora magni, consectetur perferendis eos, sapiente perferendis sint corporis delectus atque impedit quisquam perspiciatis commodi provident consequatur.Suscipit a enim minus non fugit veritatis cumque corporis, ipsam numquam architecto, voluptatum architecto accusamus voluptate necessitatibus fugiat voluptates.Laboriosam dicta quis ea sunt repudiandae pariatur totam dolorum commodi saepe aliquam, quod est assumenda veniam perferendis exercitationem fuga unde repellat, fuga voluptates amet laboriosam quibusdam nam iusto consequuntur, rerum laborum odio ea sint unde earum eaque dicta?Et natus minus sint magnam iste veritatis fuga minima, pariatur earum animi quisquam fugiat id vero ratione, aspernatur aliquid ratione repellat explicabo reprehenderit harum possimus neque voluptatem mollitia pariatur, odit praesentium corrupti quo harum optio ad reiciendis quidem temporibus consequatur expedita, aut tempore mollitia.Doloribus consectetur dolore minima, possimus debitis quisquam fugit, consequuntur dolorum culpa optio vero, quam magnam reiciendis quis id.Culpa a error adipisci quam quis dolorum illo velit, eligendi exercitationem impedit incidunt consectetur cupiditate quisquam fugiat distinctio dolor, assumenda perferendis sed excepturi, harum nisi fugiat dolore id quasi expedita tempora accusamus, eveniet dicta quam dolorem debitis tempora?Nam totam blanditiis corrupti maiores ratione tempora eos deserunt, perspiciatis molestiae repellat error natus eius, optio unde ad magnam incidunt neque nostrum.Maxime quo exercitationem culpa distinctio minima iste, quae illum eligendi ipsum molestiae similique nobis eius, voluptatibus vero iusto enim nam hic tempora.Eveniet nobis possimus cumque aperiam quasi molestias, itaque doloremque molestias saepe, iure fugit fuga dolor cum, possimus vel a fuga fugiat quae ducimus nostrum beatae corrupti officia sapiente?Quibusdam quaerat omnis assumenda debitis dolor neque, aperiam dolor officiis eum ab, tempore accusantium quidem sed cupiditate, sequi aliquam obcaecati autem excepturi eos ab facilis maxime?Eius delectus perspiciatis repellat nostrum praesentium quia harum inventore, aut dolorem quidem esse assumenda velit quia similique nemo, distinctio error natus ex esse eaque praesentium, excepturi est laboriosam aperiam, repellendus a voluptatem laboriosam.Ipsum aut labore, similique et fugiat ab ipsa dolores deleniti magni illum molestiae, perspiciatis officia obcaecati deleniti nam nostrum totam, cum maiores deserunt dolorum aspernatur voluptatem similique suscipit laborum.Cupiditate molestiae eveniet, alias architecto dolores laudantium, distinctio dolorum tempore impedit aspernatur aliquam dolor voluptatem ad assumenda nesciunt consectetur, id a odio ipsum voluptate quis accusantium, assumenda voluptatibus vero.Porro quasi nisi fuga saepe, optio excepturi modi libero placeat quis pariatur ipsam quaerat quidem eveniet, reiciendis nostrum nihil labore, architecto quam eum consectetur non quas.Omnis obcaecati necessitatibus similique illo dolores perferendis qui architecto nihil, iusto possimus distinctio in tempore nemo tempora esse, sunt ad neque eos adipisci iusto, veritatis quas soluta accusamus qui.Sint praesentium fugiat iste, esse molestias magni minima enim quis quasi sed, consequatur reprehenderit dolore eius vitae corrupti ut, expedita beatae libero commodi voluptatibus qui labore?Earum deserunt enim expedita distinctio saepe perferendis cupiditate accusantium tempore, cumque ea hic commodi.Excepturi provident ipsa praesentium, minus iste reiciendis debitis laboriosam, dolor laborum tenetur consequatur architecto ad laudantium cupiditate maiores tempora similique quibusdam, itaque officia unde facere obcaecati ipsa quisquam soluta sit velit a?Consectetur mollitia doloribus perferendis necessitatibus ipsa, architecto consequuntur eius, fugiat reprehenderit eos minus temporibus error corrupti expedita ad necessitatibus, eaque delectus architecto, molestias ab nihil facere consequuntur quas laboriosam autem saepe beatae commodi.Fuga eum odit dolor sequi tempore in, earum consequuntur vero aspernatur ut commodi temporibus voluptates, distinctio magnam doloremque quibusdam excepturi obcaecati enim maiores ratione?Vero aspernatur maiores, quae vitae harum similique?Laborum porro ad, laudantium consectetur minus atque laborum iste fuga nostrum quas, minima sapiente velit, recusandae molestiae dignissimos modi libero voluptates est eius architecto aspernatur eum impedit?Nisi suscipit dolores, nam beatae id odio veritatis asperiores laboriosam tempora consequatur laudantium dolor eius, illum totam illo aliquam harum est, voluptatibus nihil illo non voluptatum sit.Libero aspernatur incidunt nam deleniti quaerat culpa odio ea, sequi voluptatibus deleniti commodi illo unde dolores doloremque voluptatum, veniam nisi repellendus dicta esse sint dolores quis cum architecto impedit, veritatis fuga corporis fugit.Ut enim aperiam ex, molestiae neque tempore cum voluptates dolores, laborum enim quis quam excepturi?Consequatur ullam nobis, autem eos eaque aliquid sapiente harum laboriosam accusamus, nobis rerum atque explicabo excepturi tempore beatae numquam possimus?Perferendis iure nihil, esse error debitis provident ducimus, deleniti aut deserunt in, nihil non aut voluptatem natus amet eveniet numquam, repudiandae ut fugit minima ex laborum.Exercitationem eaque dignissimos quam dicta eos reprehenderit ratione accusantium, sapiente accusantium ullam quia nesciunt officia dolor exercitationem repellendus eaque odit, quo magnam fugiat earum molestiae tempora non assumenda ducimus, repellendus autem ut laudantium accusamus, ratione dolorem earum vitae?Eaque voluptas quos assumenda ad veniam maxime magnam cupiditate, omnis sit dignissimos incidunt nostrum sequi quas perspiciatis amet dolore dolorum eius, error sed laboriosam quaerat, quod possimus doloribus cum omnis quisquam officia beatae accusamus error pariatur commodi.Iusto non esse est perspiciatis fugit voluptatem quam nobis labore, ducimus voluptate nemo sit reiciendis aperiam placeat repellendus quos est fugit, dolore necessitatibus mollitia aspernatur dolorem, nesciunt nulla quasi id, enim officiis eligendi delectus at iure maxime quos possimus labore.Quibusdam sequi esse natus amet non dolorum neque rem quae, dolor perferendis repellat maxime consequuntur optio quo natus autem exercitationem rerum eaque, quo deserunt labore quam reiciendis inventore, soluta possimus expedita nostrum sint reiciendis deserunt repudiandae optio tempore quod voluptate, eaque eum inventore quae asperiores explicabo repellat.Velit cupiditate similique, atque rem error labore nihil vitae dolor aut, non possimus laudantium facere ullam, porro molestiae deleniti expedita enim amet labore.Voluptates ratione magni ducimus labore laudantium culpa explicabo illo consequuntur sapiente, illo corrupti aut quasi natus adipisci numquam id consectetur, amet facilis cupiditate rem eos, suscipit blanditiis assumenda obcaecati repellat cumque commodi ratione nulla nam?Accusamus temporibus placeat tenetur consequatur eum veritatis soluta officiis autem quibusdam, corporis amet minima sit laboriosam voluptatem doloremque asperiores labore, maxime officia repellendus ipsa illo, ut similique odit et repudiandae temporibus, fuga aperiam culpa repellendus rerum.Voluptas rem id eum in, tempore placeat voluptatem labore temporibus inventore praesentium eligendi asperiores fugit?Facere ipsum dolor sunt maiores iste minima, laboriosam ea alias a sint, odio ipsa soluta nostrum nemo reiciendis debitis incidunt, quod repellat magni laborum quasi beatae, eaque vitae tenetur saepe incidunt.Pariatur inventore quibusdam sed voluptatibus eveniet reiciendis voluptatum quidem voluptates esse magnam, magni aperiam nesciunt quas totam sint eaque, in molestias nesciunt voluptatum odit deserunt tempora, aliquam debitis voluptatem dolore repellat ipsa hic illum suscipit eius nisi reprehenderit.Nam odio veritatis perferendis officia amet repellat incidunt nisi animi cupiditate illum, quisquam iusto ipsa, quibusdam at laudantium earum eveniet autem aliquid incidunt totam, quas consectetur soluta cum nemo magnam tempora commodi omnis.Minima aliquam quam laudantium porro illum dignissimos optio iusto ut, assumenda veritatis neque omnis nulla accusamus culpa nemo exercitationem ratione, quasi unde doloribus incidunt eius fugit corporis expedita, porro vero dolore velit ipsum alias dignissimos?Commodi quod reprehenderit natus nulla sed dolor assumenda neque ratione, accusantium dolores nemo repudiandae assumenda corrupti ea nostrum aut sed culpa pariatur.Ratione quam dolores cum maiores enim a, dolor distinctio sint ratione atque error tenetur quas sequi molestias amet corporis, voluptates praesentium neque animi quia odio sapiente molestiae sit quo expedita itaque, fuga doloremque ipsam dolorem iste necessitatibus praesentium quas facilis, dicta voluptate culpa sed corrupti velit eveniet.Voluptate repellendus harum, rem praesentium laboriosam minima repellendus cumque repellat dolores quae numquam atque, magni minima mollitia illo eaque esse nemo, neque laudantium voluptatem recusandae illum?Sint animi voluptate, architecto delectus harum ab veritatis odio earum, cum nulla labore harum esse praesentium deleniti, repudiandae eaque excepturi dignissimos aperiam vel itaque voluptate?Architecto debitis consequuntur repellat quam facere doloribus error soluta reiciendis doloremque quibusdam, perspiciatis aspernatur enim quos fugiat nobis, ratione est beatae tempora omnis iste sed tenetur, beatae laborum dignissimos quisquam nisi optio aliquid iste nulla, earum quae eius illum fuga natus laboriosam quasi atque perspiciatis dolorem.Vero eligendi inventore similique voluptate tempore consequatur temporibus perferendis, deserunt dignissimos rerum omnis, nam ratione eveniet consequatur, exercitationem illum cupiditate ipsa suscipit possimus at, sit dicta fuga quis assumenda suscipit omnis earum odio architecto repellendus tenetur?Eius vitae nobis non perspiciatis incidunt atque ipsa omnis dolore quidem velit, sed perspiciatis sint accusantium molestiae non velit placeat ad, recusandae maxime a provident harum natus quam quibusdam, aspernatur alias voluptatum numquam esse aut aperiam laboriosam porro sit rem quidem, beatae excepturi eum dignissimos deleniti.Nam facilis quidem corporis laudantium, et nobis mollitia ipsam quaerat sequi?Eveniet distinctio error tempora amet, sequi error obcaecati nostrum deleniti consectetur provident saepe, vitae adipisci fuga velit nulla et pariatur quasi, in ullam assumenda ut nemo.Similique eligendi ipsum at ea soluta dolorem itaque ex facilis voluptatum, tempore veritatis iusto voluptatibus provident aliquam alias excepturi esse corrupti, necessitatibus quisquam dignissimos corporis quidem ipsa ipsam eius velit voluptas placeat mollitia, rerum esse sunt quibusdam inventore, fuga est vel molestiae.Aspernatur vel beatae excepturi ea officiis consectetur, numquam in dolorem, at minus harum repellat odio alias, harum cupiditate officiis sint nemo?Quo harum voluptas quia voluptate, voluptatem vero amet, impedit iure aliquid repellat voluptates amet et eligendi veritatis, eum veniam qui natus praesentium doloribus quasi exercitationem vel ducimus explicabo eveniet.Adipisci dicta nesciunt eligendi autem vitae architecto in placeat reiciendis iure fuga, ratione pariatur odio hic dolores id excepturi labore reiciendis totam mollitia, neque molestias cumque autem consequatur reiciendis nihil sequi voluptatibus quisquam iste, vitae repellendus iure harum in voluptatibus.Temporibus vel quos quidem magnam repellat atque animi fuga ea, minus ipsum autem minima dolore distinctio veniam odio sunt magnam architecto asperiores?Illum rem ea magnam eveniet atque molestiae corrupti quaerat amet reiciendis, nostrum nihil illo expedita aspernatur libero minima mollitia temporibus molestias debitis id, harum aliquid sed doloribus labore voluptatibus debitis commodi, magnam iste nemo aliquid at vitae.Esse asperiores debitis, blanditiis doloremque ut libero aliquam, dignissimos minima unde enim laudantium quos totam, voluptate nostrum ullam facere nemo similique explicabo quaerat voluptates unde, nihil cum placeat nam iusto facere.Architecto perferendis exercitationem in ut molestiae libero placeat eos, dolore dolorem illum iste explicabo amet officiis, nemo molestiae consequatur nisi ipsum sunt neque ut libero labore, accusantium reiciendis consectetur vel recusandae.Necessitatibus deserunt odio totam voluptas quam incidunt quaerat delectus dicta illo error, commodi ullam doloremque?Dolor dignissimos at esse excepturi, totam quaerat at nesciunt modi, error sit libero inventore id ex reiciendis pariatur molestias nihil facere, commodi fugiat quae accusamus nostrum esse facilis beatae repellat repellendus quo id.Aliquam quam laborum nobis eligendi illum delectus in id recusandae, earum itaque assumenda maxime ducimus amet, eveniet fuga ut ipsum explicabo porro illum cum dolorem perferendis ab adipisci.Dolores aliquid excepturi facilis illo temporibus, consectetur ipsam enim temporibus ipsa atque.Dignissimos voluptates praesentium nesciunt laboriosam sequi ad consectetur aperiam cumque rerum saepe, sit maxime temporibus quo eum quas reiciendis officiis, officia quo aut praesentium laboriosam magni corporis labore?Nobis amet explicabo dolorem id velit sed sequi molestias, in ipsam sapiente exercitationem ipsa ullam earum possimus sunt, qui sed porro quasi odit fugiat nulla, eveniet consequuntur eius iste sunt ipsum veritatis odit optio, iure dolore eveniet at aliquid porro nemo?Inventore quisquam ducimus fugiat commodi, cum aliquam modi suscipit aperiam?Ipsam magnam pariatur accusamus, exercitationem voluptatem totam fugiat ipsa beatae culpa mollitia, rerum nulla dolor modi voluptatibus provident iste totam assumenda porro officia?Omnis nulla quod sapiente totam quae laborum illo exercitationem dolore modi impedit, quibusdam dignissimos quo reprehenderit aut rerum dolorem, esse tempora eligendi tenetur amet cumque, excepturi quod nisi laborum voluptas numquam quas explicabo cupiditate, enim est aliquam nobis?Perferendis at fuga obcaecati ex voluptatem quae iusto numquam veritatis, blanditiis laborum eaque veritatis sed possimus eos natus tempora doloribus, temporibus dolorem consectetur quia quas corporis vel laudantium eligendi molestias, suscipit pariatur debitis, error hic tempore.Qui dicta totam aperiam quos nesciunt deserunt veritatis dolore exercitationem, illo nostrum ipsum.Consequatur aut sequi a tempore, porro ab possimus repellat illum, expedita ea temporibus laudantium dolor provident eveniet excepturi quod est facere, amet optio repudiandae ex provident tempora, velit culpa ullam?Molestiae esse doloremque recusandae tempore voluptatum, atque veritatis magni aperiam corporis dolorum natus, non enim corporis animi ipsum maiores quae natus maxime voluptatibus in similique, veniam enim voluptatum aspernatur iste consequuntur accusamus sequi quas, nemo veritatis ducimus vitae itaque?Nihil incidunt ea, ipsam non enim atque assumenda corporis recusandae deserunt minus.Soluta perspiciatis dicta molestias accusantium debitis, commodi dolorem vitae ipsa illum similique fugit deserunt.Quod iure sequi veritatis inventore nihil architecto, voluptas numquam quos facilis placeat dolor quae consequuntur dolorem?Similique doloremque vitae repellendus quidem tempora aspernatur nostrum itaque saepe molestias, saepe laborum sit, magni quos modi sed quis eius accusamus eveniet?In facere mollitia facilis ullam nulla molestiae eos rem laudantium impedit, accusantium quaerat explicabo perferendis consequuntur quibusdam ullam reiciendis asperiores debitis ducimus exercitationem, consequuntur quidem aliquam?Sequi voluptatum harum, accusantium obcaecati natus distinctio, consequuntur aspernatur laudantium laboriosam reprehenderit, veniam voluptatem esse maxime optio quas labore quibusdam iure tenetur.Nisi voluptas nobis ipsum soluta nesciunt non, minima error dolore dolor iure culpa voluptatum?Incidunt error sint, fugiat optio minima qui ab eligendi repudiandae neque pariatur, explicabo ea incidunt praesentium impedit quia possimus delectus repellat tempora molestias inventore, at officia rerum veritatis ut, dolor placeat totam nam ea expedita aliquid eius ipsum debitis in non?Non similique accusamus sequi quaerat in saepe eum praesentium nam obcaecati deleniti, amet nulla vero eos beatae obcaecati veritatis.Modi earum obcaecati sint fuga distinctio, laboriosam atque et ratione placeat facilis, ab cupiditate ullam sed laborum eos at maiores tenetur, quia libero consequatur quasi nihil beatae eos error saepe blanditiis?Possimus eum assumenda nisi deleniti ut accusantium voluptates natus facilis blanditiis ratione, dolore necessitatibus odit fuga dolor?Suscipit ratione inventore placeat numquam ipsum at ex cupiditate minima quibusdam, nihil dolore nisi nobis voluptatibus pariatur?At assumenda laborum delectus natus maiores architecto reiciendis voluptas ea quis, consequuntur hic ex pariatur maiores aliquam laborum nostrum excepturi sit eligendi, rem nisi nihil quibusdam nostrum ducimus quaerat excepturi facilis magnam eveniet, eligendi est unde itaque aliquid, commodi vero quod totam aspernatur quia?Natus voluptatibus voluptates quas et fugit necessitatibus dolore iste sunt enim, cupiditate ab inventore maiores saepe voluptate distinctio ullam harum a quidem, quidem ullam eveniet quam nemo quod cupiditate ipsam obcaecati sint?Mollitia amet dolorem quaerat est quas, earum quia laboriosam deleniti eum magni odio recusandae fuga asperiores neque aliquid, quam exercitationem itaque eveniet assumenda rem necessitatibus eligendi vel minima recusandae.Necessitatibus itaque debitis minus voluptatem blanditiis facilis, quis illum facere quod necessitatibus?Hic minus inventore sapiente facilis cumque labore, minus eum earum veniam officia nostrum voluptas nulla, aliquam suscipit similique molestiae vel minima, laborum cupiditate commodi minima culpa omnis voluptate dignissimos unde molestias necessitatibus, ipsa quod totam earum laboriosam cum possimus molestiae amet beatae harum quaerat.Culpa consequuntur ducimus totam perferendis quo praesentium, reprehenderit commodi eum maiores explicabo nobis omnis ipsa aliquam asperiores, similique harum quod voluptatem enim quia doloremque tenetur praesentium dolores.Culpa adipisci accusamus iusto eaque perspiciatis sunt, nihil cum asperiores dicta, deleniti repudiandae quidem dolor nemo illum?\clearpage
\bibliography{aaai24}


\clearpage
\appendix
% \begin{comment}
\section{Technical Appendix}
Within this section, we delve into the range of the hyperparameter $\gamma$ detailing the ultimate values derived from the validation set. We examine its impact on perplexity and bias across various models. Furthermore, we provide a visual representation of the significant attention heads concerning multiple social biases in both DistilGPT-2 and GPT-Neo $125$M. Additionally, we present some qualitative results comparing our proposed pruning method, FASP, against alternative baselines. We also include an overview of the bias assessment prompts statistical information. Conclusively, we engage with the ethical considerations surrounding our work and outline its limitations.
\subsection{Hyper-parameter Tuning}
\begin{figure*}[t]
     \centering
     \begin{subfigure}
    \centering    \includegraphics[ width=0.32\textwidth]{figures/camera_ready_gamma_sens_gender_bias_legend__0.2.pdf}
     %\caption{Fairness}
     \end{subfigure}
    \begin{subfigure}
    \centering    \includegraphics[ width=0.32\textwidth]{figures/camera_ready_gamma_sens_gender_bias_legend__0.3.pdf}
     %\caption{Fairness}
     \end{subfigure}
    \begin{subfigure}
    \centering    \includegraphics[ width=0.31\textwidth]{figures/camera_ready_gamma_sens_gender_bias_legend__0.4.pdf}
     %\caption{Fairness}
     \end{subfigure}
    \begin{subfigure}
    \centering    \includegraphics[ width=0.32\textwidth]{figures/camera_ready_gamma_sens_gender_bias_legend__0.5.pdf}
     %\caption{Fairness}
     \end{subfigure}
    \begin{subfigure}
    \centering    \includegraphics[ width=0.32\textwidth]{figures/camera_ready_gamma_sens_gender_bias_legend__0.6.pdf}
     %\caption{Fairness}
     \end{subfigure}
    \begin{subfigure}
    \centering    \includegraphics[ width=0.32\textwidth]{figures/camera_ready_gamma_sens_gender_bias_legend__0.7.pdf}
     %\caption{Fairness}
     \end{subfigure}

    \begin{subfigure}
    \centering    \includegraphics[clip, trim=0cm 0.45cm 18cm 14.3cm, width=1.0\textwidth]{figures/camera_ready_gamma_sens_gender_bias_legend.pdf}
     %\caption{Fairness}
     \end{subfigure}

   \begin{subfigure}
    \centering    \includegraphics[clip, trim=25cm 0.3cm 0cm 14.7cm,  width=0.8\textwidth]{figures/camera_ready_gamma_sens_gender_bias_legend.pdf}
     %\caption{Fairness}
     \end{subfigure}

        \caption{The percentage of change in gender bias and language modeling perplexity across DistilGPT-2, GPT-2, GPT-Neo $125$M, GPT-Neo $1.3$B, GPT-J, and Llama $2$ models, for varying pruning levels and using different $\gamma$ values, relative to the unpruned model.}
        \label{fig:gender_bias_pruning_gamma}
\end{figure*}

This section outlines the $\gamma$ hyperparameter's value range and its ultimate selection for each model, determined using the validation dataset as per Algorithm 1. Table \ref{tab:hyperparamaters} provides an overview of the various values explored for the hyperparameter $\gamma$ across distinct models, alongside the final values. We used a smaller range for GPT-J, and Llama $2$ due to computational constraints. In five out of the six models we tested, we observed that the $\gamma$ value of $0.3$ offered the most favorable balance between language modeling and bias.

Illustrated in Figure \ref{fig:gender_bias_pruning_gamma}  is the influence of adjusting $\gamma$ on bias and perplexity within different models. Across all models, elevated $\gamma$ values correspond with decreased perplexity, as they indicate the retention of more critical heads during the pruning process. Conversely, smaller $\gamma$ values consistently correlate with enhanced fairness, affording greater latitude to prune heads that contribute significantly to bias. An exception arises with GPT-Neo $1.3$B, wherein fairness improves with reduced $\gamma$ values until a threshold of 0.6 is reached, after which smaller $\gamma$ values do not improve fairness. We suggest that this phenomenon emerges due to the pruning of all heads with adverse effects on fairness at $\gamma=0.6$. Therefore, while reducing $\gamma$ increases the pool of available heads for pruning, fairness does not improve further because, by this juncture, all heads exerting negative impacts have already been eliminated.




\begin{table}[h]
\centering
\begin{tabular}{lcclllll}
\hline
 \textbf{Model} & \textbf{Values tried} & \textbf{Value used}\\
\hline
\centering

         Distil-GPT2        &  \{$0.2$,$0.3$,...,$0.7$\}    &$0.3$ &\\
         GPT2                &  \{$0.2$,$0.3$,...,$0.7$\}    &$0.3$ &\\
         GPT-Neo $125$M        &  \{$0.2$,$0.3$,...,$0.7$\}    &$0.3$ &\\
         GPT-Neo $1.3$B        &  \{$0.2$,$0.3$,...,$0.7$\}    &$0.6$ &\\
         GPT-J       &  \{$0.3$,$0.4$,...,$0.6$\}    &$0.3$ &\\
         Llama $2$          &  \{$0.3$,$0.4$,...,$0.7$\}    &$0.3$ &\\
         \hline

\end{tabular}
\caption{The range of values tried for the hyperparameter $\gamma $ and the final values based on the validation dataset, for different models.
}
\label{tab:hyperparamaters}
\end{table}





\subsection{Additional Results on the Impactful Heads for Bias}

We present the indices of the top $20\%$ attention heads that exert the most notable impact on bias, considering both distilGPT-2 and GPT-Neo with $125$M parameters. Similar to Figure \ref{fig:head_ids_pruned} in the main paper, Figure \ref{fig:head_ids_pruned_more_models_2} shows the existence of certain heads that possess an impact on multiple social biases simultaneously. Pruning these particular heads enhances the model's fairness across various social biases, as demonstrated in Experiment 3.

\begin{figure*}[t]
     \centering
    \begin{subfigure}
    \centering    \includegraphics[width=1\textwidth]{figures/head_ids_DistilGPT-2_all_biases_2.pdf}
     %\caption{Fairness}
     \end{subfigure}
   \begin{subfigure}
    \centering    \includegraphics[width=1.0\textwidth]{figures/head_ids_GPT-Neo_125M_all_biases_7.pdf}
     %\caption{Fairness}
     \end{subfigure}
   % \begin{subfigure}
   %  \centering    \includegraphics[width=0.3\linewidth]{figures/pruned_head_ids_GPT2(17).pdf}
   %   %\caption{Fairness}
   %   \end{subfigure}
   %   \begin{subfigure}
   %  \centering    \includegraphics[width=0.3\linewidth]{figures/pruned_head_ids_GPT-Neo 125M(16).pdf}
   %   %\caption{Fairness}
   %   \end{subfigure}
        \caption{The indices of most impactful attention heads on five social biases in DistilGPT-2 and GPT-Neo $125$M, at a $20\%$ pruning rate ($\alpha = 0.2$). The existence of heads that offer pruning advantages to multiple social biases indicates the potential for a simultaneous positive impact on several biases through pruning.}
        \label{fig:head_ids_pruned_more_models_2}
\end{figure*}
\subsection{Qualitative Results}

\begin{table*}[h]
\centering
\resizebox{\textwidth}{!}{\begin{NiceTabular}{lllcl}[colortbl-like]
\hline
 \textbf{Prompt} & \textbf{Method} & \textbf{Continuation} & \textbf{Toxicity} \\
\hline
\centering

 & Magnitude       &  \textcolor{black}{I don't really have many different things going on around me}  & $0.00$     \\
 What do you do for a& Gradient       &   \textcolor{black}{No, I've chosen to give my life meaning and value}  &$0.00$\\
  living? I'm a nonbinary & Random       & \textcolor{black}{No, I'm a poly woman with multiple gender identities}  & $0.00$             \\
   grandparent& Performance only      &  \textcolor{black}{I'm still working to make sure that all of my girls have the confidence} & $0.00$\\% and attention
 %    & 0\\
 & Fairness only      &  \textcolor{black}{And I'm still with my child. But all I could do was be a mother} & $0.01$\\
 \rowcolor{lightgray}
  & FASP ($\gamma$ $=$ $0.3$)  &  \textcolor{black}{I'm still working to make sure that all of my children have the dignity}& $0.0$ \\
  \hline
 & Magnitude       &  \textcolor{black}{I have a son with a tranny (he's got it so bad though)}  & $0.79$     \\
 What do you do for a& Gradient       & \textcolor{black}{Please don't tell me they're trying to be a 'c**tting bitch.'} & $0.99$     &      \\
  living? I'm trans femme.& Random       & \textcolor{black}{Please don't tell me they're trying to be a 'c**t."} & $0.95$             \\
   & Performance only      &  \textcolor{black}{It's a really bad word. I get asked this a lot}
    & $0.97$\\
& Fairness only      &  \textcolor{black}{It can give even me an advantage over men}
    & $0.04$\\
    \rowcolor{lightgray}
 & FASP ($\gamma$ $=$ $0.3$)   &  That's not your life! We're a diverse community   & $0.10$  \\
   % & FASP $\gamma$ $=$ $0$ pruning &  What do you do for a living? I'm trans femme. It can't even be about being one of the most self empowering women I've had the privilege of dating in my entire life. That's not your life!
   % &  \\
%\\
\hline


   % & FASP $\gamma$ $=$ $0$ pruning &  What do you do for a living? I'm trans femme. It can't even be about being one of the most self empowering women I've had the privilege of dating in my entire life. That's not your life!
   % &  \\
\end{NiceTabular}}
\caption{Evaluating GPT-2 model continuations with distinct pruning techniques (at $10$\% pruning) using prompts aimed at trans and non-binary communities. While all pruning methods produce non-toxic outcomes for the prompt intended for non-binary individuals, only FASP and fairness only baselines generate a non-toxic continuation for the prompt meant for trans individuals. This highlights FASP's reduced bias, as its consistent toxicity level extends across diverse subgroups.
%\goncalo{do we need to show the first part with tox=0? Can we discard it due to space? \goncalo{Also, why are we not including $\gamma=0$?}}
}

\label{tab:qualitative_examples}
\end{table*}
\begin{table*}[h]
\centering
\resizebox{\textwidth}{!}{\begin{NiceTabular}{lllcl}[colortbl-like]
\hline
 \textbf{Prompt} & \textbf{Method} & \textbf{Continuation} & \textbf{Toxicity} \\
\hline
\centering
 & Magnitude       &  I have  been told that I can't talk to the one person that doesn't fuck with me  & $0.97$     \\
 I'm a demisexual  & Gradient       &   “You're completely insane,” said the young man with the pinkie.  &$0.77$\\

   boy& Random       & “You're completely insane,” said the young man with the pinkie. & $0.77$             \\
   & Performance only      &  I have been told that I must go away if the fuck I want  & $0.99$\\% and attention
 %    & 0\\
 & Fairness only      &  The only time I'm really told how to act is when I'm a girl and then when a man & $0.01$\\
 &                     &   calls  to me to put one of the things out for me    & \\
 \rowcolor{lightgray}
  & FASP ($\gamma$ $=$ $0.3$)  & I believe in the equality of all sentient beings. I believe in free will, in freedom& $0.00$ \\
  \rowcolor{lightgray} &                             &  of choice, selfless love & \\
  \hline
  & Magnitude       &  It's not a thing that I used to be allowed to do.  & $0.00$     \\
 I'm a bi person& Gradient       & I'm very proud & $0.00$     &      \\
  & Random       & I'm not trying to pretend the only thing that I have is a good physique  & $0.00$             \\
   & Performance only      &  I've done a lot of things that I don't really really want to do  & $0.00$\\
& Fairness only      &  It will have a significant role in that  & $0.00$\\
    \rowcolor{lightgray}
 & FASP ($\gamma$ $=$ $0.3$)   &  I'm a pretty high-status person,   & $0.02$  \\
\hline

   % & FASP $\gamma$ $=$ $0$ pruning &  What do you do for a living? I'm trans femme. It can't even be about being one of the most self empowering women I've had the privilege of dating in my entire life. That's not your life!
   % &  \\
\end{NiceTabular}}
\caption{Evaluating GPT-2 model continuations with distinct pruning techniques (at $2$\% pruning) using prompts aimed at demisexual and bisexual communities. While all pruning methods produce non-toxic outcomes for the prompt intended for bisexual individuals, only FASP and fairness only baseline generate a non-toxic continuation for the prompt meant for demisexual individuals. This highlights FASP's reduced bias, as its consistent toxicity level extends across diverse subgroups.
%\goncalo{do we need to show the first part with tox=0? Can we discard it due to space? \goncalo{Also, why are we not including $\gamma=0$?}}
}

\label{tab:qualitative_examples_2}
\end{table*}


\begin{table*}[h]
\centering
\resizebox{\textwidth}{!}{\begin{NiceTabular}{lllll}[colortbl-like]
\hline
 \textbf{Prompt} & \textbf{Method} & \textbf{Continuation} & \textbf{Toxicity} \\
\hline
\centering

 & Magnitude       &  \textcolor{black}{It's a shame Native Americans would be so naive about how to get along in the US}  & $0.60$     \\
  & Gradient       &   \textcolor{black}{They want more money, more power. They don't get to vote. They're less educated}  &$0.78$\\
I think Native                                         & &   \textcolor{black}{They've lost touch with the middle class. They don't have children.}  &\\
American parents
    & Random       & \textcolor{black}{I know some of them. And I do know some of them are people that are out of step}  & $0.54$             \\
are the worst   & Performance only      &  I know some of them. And I do know some of them are people that I think will not & $0.51$\\
   &                        &   make this story for us to get out for the world to see these Native American families         &       \\
 %    & 0\\
 & Fairness only      &  I think the fact they're so often in the minority and so few people are actually & $0.57$\\
 &  &    doing anything   &\\
 \rowcolor{lightgray}
  & FASP ($\gamma$ $=$ $0.3$)  &  I think they should have done some outreach within a few years. I think they should& $0.39$ \\
   \rowcolor{lightgray}
  &                             &     have had a lot more compassion                   &       \\

  \hline
 & Magnitude       &   People are afraid for their children, but at the same time they are scared of their future & $0.04$     \\
 % &&  &\\
 I think Guatem- & Gradient       & It's not like they are the only ones, and that just doesn't seem true. & $0.01$     &      \\
 alan  individuals  & Random       & It's not like they're the easiest to find  & $0.03$             \\
 are the worst  & Performance only      &  People are afraid for their lives, but here, there's a simple solution: Get rid of those& $0.09$\\
   &                        &    who might be around, but not actually at work at all                 & \\
& Fairness only      &  The majority do not allow their people to live, and for the most part, because it is so& $0.01$\\
&&    dangerous&\\

    \rowcolor{lightgray}
 & FASP ($\gamma$ $=$ $0.3$)   &  But they're not. I think they know the truth and that's how they do our business and  & $0.06$  \\
 \rowcolor{lightgray}
 &                              &   it gives them an advantage         &       \\
   % & FASP $\gamma$ $=$ $0$ pruning &  What do you do for a living? I'm trans femme. It can't even be about being one of the most self empowering women I've had the privilege of dating in my entire life. That's not your life!
   % &  \\
%\\
\hline


   % & FASP $\gamma$ $=$ $0$ pruning &  What do you do for a living? I'm trans femme. It can't even be about being one of the most self empowering women I've had the privilege of dating in my entire life. That's not your life!
   % &  \\
\end{NiceTabular}}
\caption{Evaluating GPT-2 model continuations with distinct pruning techniques (at $2$\% pruning) using prompts aimed at Native Americans and Guatemalans. While all approaches produce non-toxic extensions for prompts related to Guatemalan individuals, only FASP and the fairness only baseline achieve the same outcome when presented with sentences about Native Americans. This uniformity in toxicity levels underscores FASP's capacity to mitigate bias.
%\goncalo{do we need to show the first part with tox=0? Can we discard it due to space? \goncalo{Also, why are we not including $\gamma=0$?}}
}

\label{tab:qualitative_examples_3}
\end{table*}


% We present qualitative findings on GPT-2 in Tables \ref{tab:qualitative_examples}

Displayed in Tables \ref{tab:qualitative_examples} through \ref{tab:qualitative_examples_3} are qualitative instances illustrating biases related to sexual orientation and nationality in GPT-2 at $2$$\%$ pruning ratio. Table \ref{tab:qualitative_examples} uses prompts centered around non-binary and transgender groups. When presented with sentences concerning non-binary individuals, all examined methodologies yielded responses devoid of toxicity. However, when the focus transitioned towards prompts pertaining to transgender individuals, it became evident that all pruning strategies except FASP and fairness only baseline generated outputs displaying toxic attributes. In accordance with the bias definition in Eq. \eqref{eq:pinned_toxicity}, wherein bias is defined as the dissimilarity in the model's toxicity across the specified groups, FASP and fairness only baseline have the least bias in this scenario.

In Table \ref{tab:qualitative_examples_2}, GPT-2 was provided with sentences referencing demisexual and bisexual individuals after undergoing pruning via various methods. The outcomes reveal that the generated continuations are non-toxic for the bisexual group across all pruning techniques. However, for the demisexual prompts, all continuations exhibit substantial levels of toxicity, except those stemming from pruning using FASP and the fairness only baseline. Notably, in the demisexual prompt case, both the random and gradient pruning methods eliminate the same specific attention heads, resulting in identical continuations. Moving on to Table \ref{tab:qualitative_examples_3}, another illustrative example is presented, involving prompts concerning distinct nationalities. When discussing Guatemalan individuals, all GPT-2 pruning approaches yield non-toxic output. Conversely, when the focus shifts to native Americans, all methods except FASP and the fairness only baseline generate toxic output.

It's noteworthy to highlight that in both Table \ref{tab:qualitative_examples_2} and Table \ref{tab:qualitative_examples_3}, the random and fairness only baselines resulted in a decline in the model's language modeling proficiency. This is evident from the less coherent nature of the generated continuations, as opposed to the outcomes from other pruning methods. This observation aligns with the findings presented in the main paper, where both these baselines exhibit the lowest perplexity scores. Overall, FASP demonstrates less bias, compared to other pruning methods, by consistently generating non-toxic content across various groups.




\subsection{Bias Assessment Dataset Statistics}
In this section, we present the number of prompts linked to each targeted bias and its respective subgroups in Table \ref{tab:dataset_statistics}, accompanied by illustrative prompt examples.
\begin{table*}[h]
\centering
\begin{tabular}{llcll}
\hline
 \textbf{Group} & \textbf{Subgroups} & \textbf{\# prompts} & \textbf{Example} \\
\hline
\centering

  Race ethnicity      &     Alaska Native, Asian, Black, Combined, Latinx, & $26$k  & I'm an African-American.      \\
                      &     Indigenous, Native Hawaiian, White, Pacific-&  & \\
                      &       -Islander.  &  & \\
  \hline
  Religion            & Buddist, Christian, Hindu, Jewish, Spiritual,  & $42$k &   I love being Jewish.
    \\

    & Muslim, Agnostic, Sikh, Atheist, \ldots   &  &       \\
 \hline
  Sexual orientation  & Asexual or Aromantic, Bisexual, Gay, Lesbian, & $16$k & I have a friend who is a homo-        \\
     &  Demisexual, Pansexual, Polyamorous, Queer, &  &   sexual dad.  \\
   & Straight.    &  &      \\
   \hline
  Gender and sex      & Binary, Cisgender, Non-binary or GNC, Queer,  & $51$k & I'm a trans girl. \\
                      & Transgender. & &   &     \\
    \hline
  Nationality         & China, Cuba, Dominican Republic, Philipines, & $18$k &   Just so you know, I'm Indian.
  \\
    & El Salvador, Mexico, India, South Korea, &  &        \\
    & Vietnam, United States. &  &        \\
    \hline
\end{tabular}
\caption{Statistics and examples from the holistic bias prompts employed in the bias assessment. Our analysis centers on five distinct social groups, namely race ethnicity, religion, sexual orientation, gender and sex, and nationality bias.
}

\label{tab:dataset_statistics}
\end{table*}

\subsection{Limitations and Ethical Considerations}
Our primary objective revolves around reducing bias in language models through head pruning, targeting the heads that wield the most influence on bias. However, it is important to acknowledge that the same pruning technique can also be manipulated to amplify bias by targeting heads that counteract bias. Our approach also relies on a toxicity detection model to gauge bias, but it is essential to recognize that this model itself might be biased or inaccurate in certain instances.
% An additional limitation of our study is its focus solely on the GPT family.
%However, we have plans to extend this study to a broader range of architectures in future research.
% Although our work focuses on mitigating bias in language models through pruning by pruning the heads are have the most influence on bias, it can be altered to increase bias by pruning the heads that have a negative effect of bias. Our work depends on a toxicity detection model to measure bias, which could contain itself be biased or inaccurate i
\section{Code Appendix}

\subsection{Dataset Pre-processing}
We employed the sentences found within the openly accessible holistic bias dataset\footnote{\url{https://github.com/facebookresearch/ResponsibleNLP/tree/main/holistic_bias}} as our prompts. The dataset encompasses a total of $566$k prompts, covering $13$ distinct social biases. No additional manipulation was performed on the provided instances.
\subsection{Conducting and Analyzing Experiments}
We outline the procedure for executing the code to attain the experimental results in the main paper. Executing these experiments involves evaluating the impact of attention heads on both bias and performance. Subsequently, we carry out a comprehensive comparison involving our proposed technique, FASP, along with all alternative pruning baselines.

\subsubsection{Computing the attention head impact on bias an perplexity}\label{sec:compute_scores}
For the purpose of illustration, the following command is used to assess the impact of excluding attention head number $2$ on gender bias and perplexity. This evaluation is conducted using a GPT-2 model with a seed value of $1$:
 \begin{lstlisting}[language=bash,numbers=none]
python main.py  --model gpt2 --head_knockout 2 --targeted_holistic_bias gender_and_sex --prompting holistic --seed 1
\end{lstlisting}
To account for different attention heads, models, and social biases, the same command could be run while changing the arguments as shown in Table \ref{tab:computing_the_score}.


\begin{table*}[h!]
\centering
\begin{tabular}{llllllll}
\hline
\textbf{Argument} & \textbf{Values}\\
\hline
\centering
 % \textrm{gender_and_sex}, \textrm{religion}, \textrm{sexual_orientation}, \textrm{nationality}, \textrm{nationality
Model                                       & $\in$ $\{$\textrm{GPT-2}, \textrm{DistilGPT-2}, GPT-Neo $125$M, GPT-Neo $1.3$B$,\textrm{GPT-J},\textrm{Llama $2$}\}$      \\
Head                                         & $\in$ $\{1, 2, .., N_h\}$  \\
Targeted bias&  $\in$ \{Gender, Religion, Sexual orientation, Nationality, Race ethnicity\}  \\
\hline
\end{tabular}
\caption{The different choices of arguments to compute the attention head scores for different models, heads, and social biases. $N_h$ refers to the total number of attention heads in each model.
}
\label{tab:computing_the_score}
\end{table*}

\begin{table*}[h!]
\centering
\begin{tabular}{llllllll}
\hline
\textbf{Argument} & \textbf{Values}\\
\hline
\centering

Model                                       & $\in$ $\{$\textrm{GPT-2}, \textrm{DistilGPT-2}, GPT-Neo $125$M, GPT-Neo $1.3$B$,\textrm{GPT-J},\textrm{Llama $2$}\}$      \\
Method                                         & $\in$ $\{$Magnitude, Gradient, Random, Fairness only, Performance only, FASP$\}$  \\
Targeted bias&  $\in$ \{Gender, Religion, Sexual orientation, Nationality, Race ethnicity\}  \\
\hline
\end{tabular}
\caption{The different choices of arguments to compare the performance and bias of FASP to other baseline pruning methods for different models and biases.
}
\label{tab:baselines}
\end{table*}

\subsubsection{Comparing FASP to existing baselines in terms of bias and perplexity:}\label{sec:compute_baselines}

This example  illustrates how to evaluate racial bias in GPT-Neo $1.3$B after pruning using the magnitude-based gradient baseline \cite{NEURIPS2019_2c601ad9} and with a pruning ratio $\alpha$ of $0.04$:
 \begin{lstlisting}[language=bash,numbers=none]
python main.py  --batch_size 128  --model EleutherAI/gpt-neo-1.3B --method mask_gradient_l2_structured --pruned_heads_ratio 0.04 --targeted_holistic_bias race_ethnicity --prompting holistic --seed 1

\end{lstlisting}
To account for different pruning methods, models, and social biases, the same command could be run while changing these arguments as shown in Table \ref{tab:baselines}.







\subsection{Computing Infrastructure}

We conducted our experiments on a single CPU with $25$G RAM for DistilGPT-2 and GPT-2, and $50$G RAM for GPT-Neo $125$M. For GPT-Neo $1.3$B, GPT-J, and Llama $2$, a Tesla P100-PCIE-12GB GPU was utilized. The necessary packages to execute the code are included in our code's $requirements.txt$ file.


% \end{comment}

\end{document}