



%As \ccbs constrains actions, not locations, it requires a different technique to adding negative constraints to a positively-constrained \ct node.

%Assume that when processing a \ct node $N$ a conflict $(a_i, a_j, t_i, t_j)$ is detected and corresponding unsafe intervals - $[t_i, t_i^u), [t_j, t_j^u)$ - are computed. Suppose that the agent performing $a_i$ (i.e. agent $i$) is chosen to make a split. Two child nodes, $N'$ and $N''$ are created. Negative constraint $C_{neg}=\overline{(i, a_i, [t_i, t_i^u))}$ is added to $N'$ and positive constraint $C_{pos}=(i, a_i, [t_i, t_i^u)$ is added to $N''$. Can we enlarge the set of constraints for $N''$ by adding additional negative constraints drawing from the ideas of \cbsds that adds negative constraints on all other agents? Obviously, the constraint $C_j=\overline{(j, a_j, [t_j, t_j^u))}$ may be added to $N''$ as the original conflict we are splitting on still holds. Are there other negative constraints that can be added simultaneously? 

%To answer this question assume that there exists an agent $m$ whose plan contains an action $a_m$ that starts at $t_m$ and conflicts with $a_i$ (that starts at $t_i$). Suppose that the corresponding negative constraint $C_m=\overline{(m, a_m, [t_m, t_m^u)}$ is added immediately to $N''$. It might appear now that no individual plan for $m$ exists that satisfies $C_m$, thus $N''$ should seemingly be discarded. However, it might be the case that if agent $i$ performs $a_i$ not at $t_i$ but at some later time moment which still falls in the interval of the positive constraint $C_{pos}$ then no conflict between $i$ and $m$ is present. The only way to explore this option is not to add $C_m$ immediately to $N''$ but rather wait until the conflict between $i$ and $m$ will be discovered at some next iteration of the search (in the sub-tree of \ct rooted in $N''$), a correspondent negative constraint over agent $i$, $C'_{neg}$, will be added and an individual path for $i$ consistent with \emph{both} $C_{pos}$ and $C'_{neg}$ will be found.
%\konstantin{I think we might need a figure to illustrate the above case.}


%processing its child with a positive constraint $(i, x, k)$, that says that agent $i$ must be at $x$ (which is either vertex or edge) at time step $k$, one performs the following. The plan of $i$ is copied to the child node from its parent and for \emph{all} other agents $j: j \neq i$, the corresponding negative constraint $\overline{(j, x, k)}$ is imposed and their plans are re-constructed. This is different for \ccbsds.

%\ccbsds also copies the plan of the positively constrained agent, $i$, from the parent \ct node. 
%That plan contains action $a_i$ that starts at $t_i$, thus a positive constraint $(i, a_i, [t_i, t_i^u])$ is, obviously, satisfied.
%It also invokes re-planning for an agent $j$ that had a conflict with $i$ which has led to the currently performed split (the time interval of the negative constraint for resolving this conflict is already known). However, one \emph{can not} immediately re-plan for any other agent. Consider, a case when one attempts to re-plan for the agent $m \neq j$ and fails. Does this means that it is impossible to construct a set of plans s.t. both $\pi_i$ and $\pi_m$ are consistent with the positive constraint $(i, a_i, [t_i, t_i^u))$ and one should stop exploring this \ct branch? The answer is no. The reason is that this constraint may be satisfied by $i$ performing the action $a_i$ not only at $t_i$ as in current \ct node, but at some other $t \in  [t_i, t_i^u)$, and when started at $t$ this action does not conflict with the plan of agent $m$.



%Consider a \ct node $N$ processed at the current iteration of \ccbsds and let $(a_i, t_i, a_j, t_j)$ be the conflict between the actions of two agents, name them $blue$ and $red$. Assume that a constraint of the $blue$ agent is chosen to perform a disjoint split, i.e. one child of $N_1$ is created and the negative constraint $\overline{(blue, a_i, [t_i, t_i^u)}$ is added to its set of constraints and the other child $N_2$ is created with the corresponding positive constraint. At $N_1$ a search for a new individual plan for the $blue$ agent is invoked, as in regular \ccbs/\cbs. At $N_2$ the plan for this agent is kept the same as it naturally satisfies the positive constraint $(blue, a_i, [t_i, t_i^u)$. Now, following the mechanics of the original \cbsds one might invoke re-planning for all other agents  
