\section{Introduction}
\label{sec:introduction}
Autonomous driving and advanced driving assistance system (ADAS) rely on highly accurate road scene analysis. As a fundamental step to achieving a reliable road scene understanding, object detection has received great attention in recent years and has been significantly boosted with the development of deep neural networks (DNNs) \cite{lin2017focal, redmon2018yolov3, sun2021sparse}. In practical road scenes, cars appear to be one of the most frequently observed yet dangerous objects, and car detection is still a challenging problem due to the large structural and appearance variations of cars in different scenes.

Although existing state-of-the-art methods explore rich contexts in multiple modalities including RGB, LiDAR, and infrared to improve the detection accuracy, these methods typically assume the imaging quality is optimal. When it comes to adverse conditions such as low light, rain, and fog, a significant accuracy drop would occur due to the poor and limited sensing scene information fed into the algorithms. For example, an RGB camera may fail to capture important visual cues under low-light conditions \cite{song2019vision, arora2022automatic}, a LiDAR sensor may struggle to distinguish targets in complex environments due to its limited range and resolution \cite{Qian_2021_CVPR, chen2020pseudo}, and, similarly, an infrared sensor may produce blurry images with low contrast when exposed to extreme weather conditions \cite{du2021weak, sun2022drone}. Instead, polarization, characteristic of the light wave, can robustly reveal the intrinsic physical properties of cars and their surrounding environment (\textit{e.g.}, the surface geometric structure, roughness, and material) in various view/lighting/weather conditions. This inspires us to exploit the \textbf{\textit{reliable}} and \textbf{\textit{discriminative}} features provided by polarization to complement traditional RGB features for robust car detection.

Linear polarization cues, described by the angle of polarization (AoLP) and the degree of linear polarization (DoLP), might not be equally obvious/informative over different scenes and image regions, or even confound valid RGB cues. To address these challenges, we design a novel RGB-Polarization Car Detection Network (PCDNet) with RGB intensities, trichromatic AoLP and DoLP as input.
PCDNet is built on three key modules: (i) Polarization Integration (PI) module that fuses AoLP and DoLP to generate a comprehensive and semantically meaningful polarization representation; (ii) Material Perception (MP) module to explore the polarization/material properties of cars across different learning samples for enhancing the polarization cues in each scene; and (iii) Cross-Domain Demand Query (CDDQ) module to dynamically integrate the informative polarization cues into RGB features based on the spatial demand map generated from RGB domain.

To train PCDNet, we introduce an RGB-Polarization car detection dataset, dubbed RGBP-Car, which consists of 1,611 RGB images and pixel-aligned trichromatic (\textit{i.e.}, red, green and blue channels) AoLP and DoLP images, as well as corresponding annotated 31,234 bounding boxes of cars. To ensure diversity, the images in RGB-P Car are captured from various real-world traffic scenes with different view/weather/lighting conditions and vehicle densities.

We perform extensive experiments to demonstrate the superiority of our method over competing approaches and show the importance of polarization cues for robust car detection in challenging scenes (\textit{e.g.}, Fig. \ref{fig:teaser}). In summary, our contributions are:
\begin{itemize}
\item the first solution to exploit both RGB and trichromatic angle/degree of linear polarization (AoLP/DoLP) cues for robust car detection;
\item a new pixel-aligned RGB-P car detection dataset covering challenging dense cars and low light scenarios;
\item a novel multimodal fusion network that dynamically integrates RGB and polarization features in a request-and-complement manner;
\item a novel polarization cues perception strategy to implicitly explore the intrinsic material properties of cars across the whole learning samples.
\end{itemize}
