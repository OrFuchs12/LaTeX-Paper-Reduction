\begin{algorithm}[H]
    \caption{\textsc{Round-POT}}
    \label{alg:rounding}
    \begin{algorithmic} [1]
        \REQUIRE{ $\vx = (\vecflatten(\vX)^\top, \vp^\top, \vq^\top)^\top$; marginals $\vr$, $\vc$; mass $s$.}
        %\label{inequa2}
        % \STATE \( \vp'' = \alpha \vp' \)
        % \STATE \( \vq'' = \beta \vq' \)
        \STATE \( \bar{\vp} = \mathtt{EP}(\vr, s, \vp) \) 
        \STATE \( \bar{\vq} = \mathtt{EP}(\vc, s, \vq) \)
        \STATE \(\vg = \min\{\ones, (\vr-\bar{\vp}) \oslash \vX \ones \}\)
        \STATE\(\vh = \min\{\ones, (\vc-\bar{\vq}) \oslash \vX^\top \ones \} \)
        \STATE \( \vX' = \diag(\vg) \vX \diag(\vh)\)
        \STATE \( \ve_1 = (\vr - \bar{\vp}) - \vX' \one, \ve_2 = (\vc - \bar{\vq}) - \vX'^\top \one\)
        \STATE \( \bar{\vX} = \vX' + \ve_1 \ve_2^\top / \norm{\ve_1}_1 \)
        \ENSURE{$ \Bar{\vx} = (\Bar{\vX}$, $\Bar{\vp}$,  $\Bar{\vq})$}
    \end{algorithmic}
\end{algorithm}
\section{Rounding Algorithm}
% \label{sec:rounding_algorithm}
All efficient algorithms for standard OT
\citep{Dvurechensky-2018-Computational, lin2019efficient, guminov2021accelerated}
only output an infeasible approximation of the optimum value, and leverage the well-known rounding algorithm \citep[Algorithm 2]{altschuler2017near} to project it back to the set of admissible couplings. Nevertheless, its ad-hoc design tailored to the OT's marginal constraints makes  generalization to the case of POT with more intricate structural constraints non-trivial. In fact, we attribute the rather limited literature on efficient POT solvers to such lack of a rounding algorithm for POT. Specifically, previous works rely on imposing additional assumptions on the input masses to permit reformulation of POT into standard OT with an additional computational burden \cite{Chapel-nips2020, nhatho-mmpot}. Deviating from the vast literature, we address this fundamental challenge by proposing a novel rounding procedure for POT, termed \textsc{Round-POT} (Algorithm \ref{alg:rounding}), to efficiently round any approximate solution to a feasible solution of \eqref{prob:pot_with_pq}.
%as formally summarized in Theorem \ref{prop:rounding}.
%\lipsum[1]
  %\lipsum
% \begin{algorithm}
%     \caption{\textsc{Round-POT} \textcolor{red}{Hoang: also use wrapfigure for this one}}
%     \label{alg:rounding}
%     \begin{algorithmic} [1]
%         \REQUIRE{ $\vx = (\vecflatten(\vX)^\top, \vp^\top, \vq^\top)^\top$; marginals $\vr$, $\vc$; mass $s$.}
%         \STATE \( \vp' = \min\{\vp, \vr\}, \vq' = \min\{\vq, \vc\}\) %\label{inequa1}
%         \STATE \( \alpha = \min\left\{ 1, \dfrac{\norm{\vr}_1 - s}{\norm{\vp'}_1} \right\}, \beta = \min\left\{ 1, \dfrac{\norm{\vc}_1 - s}{\norm{\vq'}_1} \right\} \)  %\label{inequa2}
%         % \STATE \( \vp'' = \alpha \vp' \)
%         % \STATE \( \vq'' = \beta \vq' \)
%         \STATE \( \bar{\vp} = \mathtt{EP}(\vr, s, \alpha \vp') \) \textcolor{red}{Hoang: shouldn't this have 4 inputs?}
%         \STATE \( \bar{\vq} = \mathtt{EP}(\vc, s, \beta \vq') \)
%         \STATE \(\vg = \min\{\ones, (\vr-\bar{\vp}) \oslash \vX \ones \}, \vh = \min\{\ones, (\vc-\bar{\vq}) \oslash \vX^\top \ones \}\)
%         \STATE \( \vX' = \diag(\vg) \vX \diag(\vh)\)
%         \STATE \( \ve_1 = (\vr - \bar{\vp}) - \vX' \one, \ve_2 = (\vc - \bar{\vq}) - \vX'^\top \one\)
%         \STATE \( \bar{\vX} = \vX' + \dfrac{1}{\norm{\ve_1}_1} \ve_1 \ve_2^\top \)
%         \ENSURE{$ \Bar{\vx} = (\Bar{\vX}$, $\Bar{\vp}$,  $\Bar{\vq})$}
%     \end{algorithmic}
% \end{algorithm}
% The common output of the three algorithms in this work is an approximate solution $\vx \geq 0$ such that the marginal constraints are violated by a predefined error: $\norm{\vA \vx - \vb}_1 \leq \delta$ for some $\delta$. A more naive approach of rounding POT is to directly follow a well-known rounding algorithm for OT \citep[Algorithm 2]{altschuler2017near} with an additional step of scaling down the input matrix due to an additional constraint $\ones^\top \vX \ones = s$. However, this approach might not produce positive slack variables $\vp$ and $\vq$, violating the feasible set. Hence, we propose a novel rounding procedure, summarized in Algorithm \ref{alg:rounding}, to ``round'' $\vx$ into $\Bar{\vx}$ in the feasible set in $\mathcal{O}(n^2)$ time such that $\vx$ and $\Bar{\vx}$ are close together. Note that the operation $\min$ between two or more vectors are element-wise. 
Given an approximate solution $\xX = (\vX, \vp, \vq) \geq 0$ violating the POT constraints of \eqref{prob:pot_with_pq} by a predefined error, $\norm{\vA \vx - \vb}_1 \leq \delta$ for some $\delta$,  \textsc{Round-POT} returns $\Bar{\vx} = (\Bar{\vX}, \Bar{\vp}, \Bar{\vq}) \geq \zeros$ strictly in the feasible set, i.e., $\vA \Bar{\vx} = \vb$, and close to $\vx$ in $\ell_1$ distance. 

The Enforcing Procedure ($\mathtt{EP}$) (Algorithm \ref{alg:EP}) is a novel subroutine to ensure $\zeros \leq \bar{\vp} \leq \vr$ and $\norm{\bar{\vp}}_1 = \norm{\vr}_1 - s$ (Lemma \ref{Guarantees_for_EP}). Equivalently, a similar procedure is applied to ($\vc, s, \vq$) in step 2 of Algorithm \ref{alg:rounding} with similar guarantees for $\bar{\vq}$. Step 1 transforms $\vp$ (or $\vq$) to $\vp'$ (or $\vq'$) so that $\zeros \leq \vp' \leq \vr$ (or $\zeros \leq \vq' \leq \vc$). The transformation in steps 2 and 3 ensures that $\norm{\vp''}_1 \leq \norm{\vr}_1 - s$ (or $\norm{\vq''}_1 \leq \norm{\vc}_1 - s$). The rest of the $\mathtt{EP}$ steps ensure the other guarantee $\norm{\bar{\vp}}_1 = \norm{\vr}_1 - s$.
The proof is included in Rounding Algorithm section in the Appendix. 
\begin{lemma}
    \label{Guarantees_for_EP}
    (Guarantees for $\mathtt{EP}$) We obtain in $\mathcal{O}(n)$ time $\zeros \leq \bar{\vp} \leq \vr$ and $\norm{\bar{\vp}}_1 = \norm{\vr}_1 - s$. 
\end{lemma}
For \textsc{Round-POT}, steps 3 through 7 check whether the solutions $\vX$ violate each of the two equality constraints $\vX \ones = \vr - \bar{\vp}$ and $\vX^\top \ones = \vc - \bar{\vq}$; if so, the algorithm projects $\vX$ into the feasible set. It is noteworthy that these two constraints directly implies the last needed constraint $\ones^\top \vX \ones = s$. Finally, \textsc{round-POT} returns an output that satisfies the required constraints in Equation \eqref{prob:pot_with_pq}.
The following Theorem \ref{prop:rounding} characterizes the error guarantee of the rounded output $\Bar{\vx}$. Its detailed proof can be found in Rounding Algorithm section in the Appendix.

\begin{algorithm}[H]
            \caption{Enforcing Procedure $\mathtt{EP}$}
            \label{alg:EP}
            \begin{algorithmic}[1]
                \REQUIRE{marginal $\vr$; mass $s$; slack variable $\vp$.}
                \STATE \( \vp' = \min\{\vp, \vr\}, \alpha = \min\left\{ 1, (\norm{\vr}_1 - s) / \norm{\vp'}_1 \right\} \)
                \STATE \( \vp'' = \alpha \vp' \)
                \IF{$1 > (\norm{\vr}_1-s) / \norm{\vp'}_1$}
                    \STATE \(\bar{\vp} = \vp''\)
                \ELSE %\COMMENT{$\vp' = \vp''$}
                    \STATE{$i = 0$}
                    \WHILE{$\norm{\vp''}_1 \leq \norm{\vr}_1 -s$}
                        \STATE $i = i + 1$ \\
                        \STATE \( p''_i = r_i \)
                    \ENDWHILE \\
                    \STATE \( p''_i = p''_i - (\norm{\vp''}_1 - \norm{\vr}_1 +s)\) %, k is the last index in the while loop \\
                    \STATE $\bar{\vp} = \vp''$
                \ENDIF 
                \ENSURE{$\bar{\vp}$}.
            \end{algorithmic}
\end{algorithm}

\begin{theorem}
    (Guarantees for $\textsc{Round-POT}$)
    \label{prop:rounding}
    Let $\vA$, $\vx$ (consisting of $\vX$, $\vp$ and $\vq$) and $\vb$ be defined as in the preliminaries. If $\vx$ satisfies that $\vx \geq 0$ and $\norm{\vA \vx - \vb}_1 \leq \delta$ for some $\delta \geq 0$, Algorithm \ref{alg:rounding} outputs $\Bar{\vx} \geq \zeros$ (consisting of $\Bar{\vX}, \Bar{\vp}$ and $\Bar{\vq}$) in $\mathcal{O}(n^2)$ time such that $\vA \Bar{\vx} = \vb$ and $\norm{\vx - \Bar{\vx}}_1 \leq 23 \delta$.
\end{theorem}
% \begin{remark}
%     We apply \textsc{Round-POT} to ensure Sinkhorn's solution $\bar{\vX}$ lies in the feasible solution set $\mathcal{U}(\vr, \vc, s)$. As Figure \ref{fig:pot_apdagd_sinkhorn_compare} (e) shows, we still do not get the convergence of the primal optimality gap.
% \end{remark}
% \begin{algorithm}[H]
%             \caption{Enforcing Procedure $\mathtt{EP}$}
%             \label{alg:EP}
%             \begin{algorithmic}[1]
%                 \REQUIRE{$\vr; s; \vp$.}
%                 \STATE \( \vp' = \min\{\vp, \vr\}\) %\label{inequa1}
%                 \STATE \( \alpha = \min\left\{ 1, (\norm{\vr}_1 - s) / \norm{\vp'}_1 \right\} \)
%                 \IF{$1 > \dfrac{\norm{\vr}_1-s}{\norm{\vp'}_1}$}
%                     \STATE \(\bar{\vp} = \vp''\)
%                 \ELSE %\COMMENT{$\vp' = \vp''$}
%                     \STATE{$i = 0$}
%                     \WHILE{$\norm{\vp''}_1 \leq \norm{\vr}_1 -s$}
%                         \STATE $i = i + 1$ \\
%                         \STATE \( p''_i = r_i \)
%                     \ENDWHILE \\
%                     \STATE \( p''_i = p''_i - (\norm{\vp''}_1 - \norm{\vr}_1 +s)\) %, k is the last index in the while loop \\
%                     \STATE $\bar{\vp} = \vp''$
%                 \ENDIF 
%                 \ENSURE{$\bar{\vp}$}.
%             \end{algorithmic}
% \end{algorithm}
