\section{Further Experiment Setup Details}
% \label{sec:experiment_details}
In this section, we will provide the more detailed experimental setup for the several applications in the main text. 
\subsection{Computational Resources}

For all numerical experiments in this paper, we use Python 3.9 with all algorithms implemented using \texttt{numpy} version 1.23.2. We run all experiments on a Linux machine with an Intel Xeon Silver 4114 CPU with 40 threads and 128GB of RAM. No GPU is used.

\subsection{APDAGD vs DE}
Each RGB image of size $32 \times 32 \times 3$ 
is transformed into grayscale, downscaled to $10 \times 10$, and flattened into a histogram of $100$ bins, all using the \texttt{scikit-image} library. 
We add $10^{-6}$ to every bin to avoid zeros. For every chosen pair of marginals, we divide each histogram by the maximum mass of the two, giving one marginal with a total mass of $1$ and other possibly less than $1$. For the cost matrix $\vC$, we use the squared Euclidean distance between pixel locations and normalize so $\norm{\vC}_\text{max} = 1$. The total transported mass is $0.8$ times the minimum total mass between the marginals. 

\subsection{Color Transfer}
% \label{sec:experiment_details_color_transfer}
Each RGB image can be considered a point cloud of pixels, and applying the color palette of one image into another is equivalent to applying the OT transport map to convert one color distribution to another. We choose the source image (Figure \ref{fig:color_transfer_comparison}a, \url{https://flic.kr/p/Lm6gFA}) and the target image (Figure \ref{fig:color_transfer_comparison}b, \url{https://flic.kr/p/c89LBU}) images, taken from Flickr. These images are chosen because they have quite distinct color palettes and different sizes. We follow the setup by \citep{Blondel-2018-Smooth} and implement the experiments as follows. Each image of size $h \times w \times 3$ is considered a collection of $hw$ pixels in 3 dimensions. Each image is quantized using $k$-means with $n = 100$ centroids. We apply $k$-means clustering to find $n$ centroids and assign to each pixel the centroid it is closest to. The image now becomes a color histogram with $n$ bins representing the centroids. For a pair of source and target images, similar to the previous subsection, we divide each histogram by the maximum total mass, yielding $\max\{\norm{\vr}_1, \norm{\vc}_1\} = 1$. Let $\va_1, \ldots, \va_n$ and $\vb_1, \ldots, \vb_n$ be the collections of centroids for the source and target images. The cost matrix is defined as $C_{i, j} = \norm{\va_i - \vb_j}_2^2$. With a partial transport map $\vX \in \RR_{+}^{n \times n}$, every centroid $\va_i$ in the source image is transformed to 
% $\hat{\va}_i = \dfrac{\sum_{j=1}^{n} X_{i, j} \vb_j}{\sum_{j=1}^{n} X_{i, j}}.$ 
$\hat{\va}_i = (\sum_{j=1}^{n} X_{i, j} \vb_j) / (\sum_{j=1}^{n} X_{i, j})$. All pixels in the source image previously assigned to $\va_i$ are now assigned to $\hat{\va}_i$. The total transported mass is set to $s = \alpha \min \{\norm{\vr}_1, \norm{\vc}_1\}$, where $\alpha \in [0, 1]$. To approximate the POT solution, we set $\varepsilon = 10^{-2}$ and run both Sinkhorn \cite{nhatho-mmpot} and APDAGD for 1,000 iterations. We set $\alpha = 0.1$, corresponding to transporting exactly 10\% of the allowed mass.
We plot the optimality gap $\inner{\vC}{\vX} - f^*$ for each transport map after every iteration in Figure \ref{fig:color_transfer_comparison} (c). As explained earlier, Sinkhorn with the accompanying rounding algorithm by \citep{altschuler2017near} does not produce a primal feasible solution, and its primal gap does not satisfy an error of $\varepsilon = 10^{-2}$ (indicated by the red line).

\subsection{Point Cloud Registration}
Previously, \cite{qin2022rigid} proposed a procedure to find such a transformation using partial optimal transport. Let the two marginal distributions be $\vr = \frac{1}{m} \ones_m$ and $\vc = \frac{1}{n} \ones_n$ and the cost matrix be $C_{i, j} = \norm{\vx_i - \vy_j}_2^2$. Given an optimal transport matrix $\vT \in \RR^{m \times n}$ between these marginals, the rotation matrix $\vR$ and translation vector $\vt$ are obtained by minimizing the energy: 
\begin{align*}
    \min_{\vR, \vt} ~ \sum_{i=1}^{m} \sum_{j=1}^{n} T_{i, j} \norm{\vx_i - (\vR \vy_j + \vt)}_2^2.
\end{align*}
This problem admits the closed-form solution
\begin{align*}
    \vR = \mathbf{V} \vS \mathbf{U}^\top, \quad \vt = \vu_x - \vR \vu_y,
\end{align*}
where $\vu_x = \frac{1}{m} \sum_{i=1}^{m} \vx_i$, $\vu_y = \frac{1}{n} \sum_{j=1}^{n} \vy_j$. The matrices $\mathbf{U}$, $\vS$ and $\mathbf{V}$ are obtained as follows. Let $\hat{\vX} \in \RR^{m \times 3}$ whose $i$th row is $(\vx_i - \vu_x)^\top$. Similar for $\hat{\vY} \in \RR^{n \times 3}$. Then, obtain the singular value decomposition of the matrix $\hat{\vX} \vT^\top \hat{\vX}^\top$ as $\mathbf{U} \mathbf{\Lambda} \mathbf{V}^\top$. Finally, $\vS = \diag(1, 1, \det(\mathbf{V} \mathbf{U}^\top))$.

To find the the matrix $\vT$, we follow the iterative procedure in \cite[Algorithm 1]{qin2022rigid} in which the point cloud $Q$ is updated gradually until convergence. In our implementation, we set the initial value for $\gamma$ (strength of entropic regularization) to 0.004 and the annealing rate of $\gamma$ to $\lambda = 0.99$. With each $\gamma$, we find the OT matrix using Sinkhorn, and the POT matrix using two methods: Sinkhorn and APDAGD. The iterative process ends when the change of $\vR$ in Frobenius norm falls below $10^{-6}$.