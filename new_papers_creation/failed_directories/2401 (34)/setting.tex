%!TEX root =  main.tex
\section{Setting}\label{sec:setting}

 


% The player set is denoted as  and the arm set is denoted as . 

% , and arms have combinatorial preferences over groups of players

The two-sided market consists of $N$ players and $K$ arms. Denote the player and the arm set as $\cN = \set{p_1,p_2,\ldots,p_N}$ and $\cK=\set{a_1,a_2,\ldots,a_K}$, respectively. 
Just as in common applications such as the online labor market, players have preferences over individual arms.  
The relative preference of player $p_i$ for arm $a_j$ can be quantified by a real value $\mu_{i,j}\in (0,1]$, which is unknown and needs to be learned during interactions with arms.
For each player $p_i$, we assume $\mu_{i,j}\neq\mu_{i,j'}$ for distinct arms $a_j,a_{j'}$ as in previous works \cite{kelso1982job,roth1984stability,liu2020competing,liu2021bandit,kong2023player,wang2022bandit}. 
And $\mu_{i,j}>\mu_{i,j'}$ implies that player $p_i$ prefers $a_j$ to $a_{j'}$. 
For the other side of participants, arms are usually certain of their preferences for players based on some known utilities, e.g., the profiles of workers in the online labor markets scenario \cite{liu2020competing,liu2021bandit,zhang2022matching,kong2023player,wang2022bandit}. In many-to-one markets, when faced with a set $P \subseteq \cN$ of players, the arm can determine which subset of $P$ it prefers most. Denote $\ch_j(P)$ as this choice of arm $j$ when faced with $P$. Then for any $P' \subseteq P$, arm $a_j$ prefers $\ch_j(P)$ to $P'$. 
% It is worth noting that not a larger subset (with more players) is preferred, as accepting a player may require some cost. 




At each round $t=1,2,\ldots$, each player $p_i \in \cN$ proposes to an arm $A_i(t) \in \cK$. Let $A^{-1}_j(t) = \set{p_i: A_i(t)=a_j}$ be the set of players who propose to $a_j$. When faced with the player set $A^{-1}_j(t)$, arm $a_j$ only accepts its most preferred subset $\ch_j(A^{-1}_j(t))$ and would reject others. 
Once $p_i$ is successfully accepted by arm $A_i(t)$, it receives a utility gain $X_{i,A_i(t)}(t)$, which is a $1$-subgaussian random variable with expectation $\mu_{i,A_i(t)}$. 
Otherwise, it receives  $X_{i,A_i(t)}(t)=0$. 
We further denote $\bar{A}_i(t)$ as $p_i$'s matched arm at round $t$. Specifically, $\bar{A}_i(t)=A_i(t)$ if $p_i$ is successfully matched and $\bar{A}_{i}(t) = \emptyset$ otherwise. 
Inspired by real applications such as labor market where workers usually update their working experience on their profiles, we also assume each player can observe the successfully matched players $\ch_j(A^{-1}_j(t)) = \bar{A}_{j}^{-1}(t) = \set{p_i: \bar{A}_i(t)=a_j}$ with each arm $a_j \in \cK$ as previous works \cite{liu2021bandit,kong2022thompson,ghosh2022nonstationary,kong2023player,wang2022bandit}.


The matching $\bar{A}(t)$ at round $t$ is the set of all pairs $(p_i,\bar{A}_i(t))$. 
Stability of matchings is a key concept that describes the state in which any player or arm has no incentive to abandon the current partner \citep{gale1962college,roth1992two}. 
Formally, a matching is stable if it cannot be improved by any arm or player-arm pair.
Specifically, an arm $a_j$ improves $\bar{A}(t)$ if $\ch_j( \bar{A}^{-1}_j(t)) \neq \bar{A}_j^{-1}(t)$. That's to say, arm $a_j$ would not accept all members in $\bar{A}^{-1}_j(t)$ when faced with this set. A pair $(p_i,a_j)$ improves the matching $\bar{A}(t)$ if $p_i$ prefers $a_j$ to $\bar{A}_i(t)$ and $a_j$ would accept $p_i$ when faced with $\bar{A}^{-1}_j(t)\cup \set{p_i}$, i.e., $p_i \in \ch_j( \bar{A}^{-1}_j(t)\cup \set{p_i} )$.
That's to say, $p_i$ prefers arm $a_j$ than its current partner and $a_j$ would also accept $p_i$ if $p_i$ apply for $a_j$ together with $a_j$'s current partners
\citep{kelso1982job,abdulkadirouglu2005college,roth1992two}. 




Responsive preferences are widely studied in many-to-one markets which guarantee the existence of a stable matching \citep{roth1992two,wang2022bandit}. 
Under this setting, each arm $a_j$ has a preference ranking over individual players and a capacity $C_j>0$. When a set of players propose to $a_j$, it accepts $C_j$ of them with the highest preference ranking. 
% Applications include college admission where a college (arm) admits students (players) with the highest score within the quota. 
This case recovers the one-to-one matching when $C_j=1$. 
For convenience, define $C=\sum_{j\in[K]} C_j$ as the total capacities of all arms. 
Apart from responsiveness, we also consider a more general substitutability setting in Section \ref{sec:decen}. 


% \fang{start: decide put where}
% One of the most common and general conditions that ensure the existence of a stable matching is \textit{substitutability}. 

% \begin{definition}{(Substitutability)}\label{def:substi}
% The preference of arm $a_j$ satisfy substitutability if for any player set $P\subseteq \cN$ that contains $p_i$ and $p_{i'}$, $p_i \in \ch_j(P\setminus \set{p_{i'}})$when $p_i \in \ch_j(P)$.
% \end{definition}

% The above property states that arm $a_j$ keeps accepting player $p_i$ when other players become unavailable. This is the sense that $a_j$ does not regard players as complementary individuals in a team (in which case the arm may give up accepting the player when others become unavailable) but as substitutes. 
% Such a phenomenon appears in many real applications and covers one-to-one and many-to-one with responsive preferences studied in previous works \citep{liu2020competing,liu2021bandit,kong2022thompson,zhang2022matching,kong2023player,wang2022bandit}. 
% For completeness, we provide the descriptions of these two settings and then give a short proof. 

% \textbf{a. One-to-one matching \cite{roth1992two}:}
%     Each arm has individual preferences over players and would accept the most preferred player among those who propose to it. Applications include enterprise bidding where a demand-side company (arm) selects only one suitable supplier (player). 
    
    
%       \textbf{b. Responsive preferences \cite{roth1992two,wang2022bandit}:} Each arm $a_j$ has a budget $C_j$ and has individual preferences over players. When a set of players propose to it, $a_j$ accepts the $C_j$ highest-ranked players. Applications include college admission where a college (arm) admits students (players) with the highest score within the quota. When $C_j=1$, this case recovers the one-to-one matching. 
%     \fang{$N\le \sum_{j\in[K]}C_j$}

% \begin{remark} 
% Since case b includes case a as a special case, we only prove why case b satisfies substitutability. 
% Select a player set $P\subseteq \cN$ which contains $p_{i}$ and $p_{i'}$. 
% % Denote $\tau$ and $\tau'$ as the type of $p_i$ and $p_{i'}$.
% Suppose $p_i \in \ch_j(P)$, i.e., $p_i$ is one of the $C_j$ highest-ranked players in $P$.  
% Then when the available set becomes $P\setminus\set{p_{i'}}$, 
% $p_i$ is still one of the $C_j$ highest-ranked players, i.e., $p_i \in \ch_j(P\setminus\set{p_{i'}})$.
% \end{remark}


% The substitutability property is more general than responsiveness as arms' preferences can have combinatorial structures. The following is an example that satisfies substitutability but not responsiveness \cite{roth1992two}. 

% \begin{example}
% There are $3$ players and $2$ arms, i.e., $\cN=\set{p_1,p_2,p_3}, \cK=\set{a_1, a_2}$. 
% % All players prefer $a_1$ most and $a_2$ least. 
% The arms' preference rankings over subsets of players are
% \begin{itemize}
%     \item $a_1: \set{p_1,p_2},\set{p_1,p_3},\set{p_2,p_3},\set{p_3},\set{p_2},\set{p_1}$.
%     \item $a_2: \set{p_3},\emptyset$.
% \end{itemize}
% That is to say, $\ch_j(P)$ is the subset that ranks highest among all subsets listed above that only contain players in $P$. Taking the preferences of $a_2$ as an example, when $p_3 \in P$, then $\ch_j(P)=\set{p_3}$; otherwise, $\ch_j(P)=\emptyset$. 
% \end{example}
% \fang{end}

In this paper, we study the bandit problem in many-to-one matching markets with responsive and substitutable preferences. Under both properties, the set $M^*$ of stable matchings between $\cN$ and $\cK$ is non-empty \citep{roth1992two,kelso1982job}. 
For each player $p_i$, let $\overline{m}_i\in[K]$ and $\underline{m}_i\in[K]$ be the index of $p_i$'s most and least favorite arm among all arms that can be matched with $p_i$ in a stable matching, respectively.
The objective of each player $p_i$ is to minimize the cumulative stable regret defined as the cumulative difference between the reward of the stable arm and that the player receives during the horizon. The player-optimal and pessimal stable regret are defined as
\begin{align}
\overline{R}_i(T) = \EE{\sum_{t=1}^T \mu_{i,\overline{m}_i} - X_{i,A_i(t)}(t)} \,,\\
\underline{R}_i(T) = \EE{\sum_{t=1}^T \mu_{i,\underline{m}_i} - X_{i,A_i(t)}(t)} \,,
\end{align}
respectively \citep{liu2020competing,liu2021bandit,kong2022thompson,wang2022bandit,zhang2022matching,kong2023player}. The expectation is taken over by the randomness in reward gains and the players' policies. 

For convenience, we define the corresponding gaps to measure the hardness of the problem. 
% , and the formal theoretical guarantee is shown in Theorem \ref{thm:decen}.

\begin{definition}\label{def:gap}
    For each player $p_i$ and arm $a_j \neq a_{j'}$, define $\Delta_{i,j,j'} = \abs{\mu_{i,j}-\mu_{i,j'}}$ as the preference gap of $p_i$ between $a_j$ and $a_{j'}$. Let $\Delta = \min_{i,j,j':j\neq j'}\Delta_{i,j,j'}$ be the minimum preference gap among all players and arms, which is non-zero since players have distinct preferences. 
    % \fang{remove}
    % Further, for each player $p_i$, define \shuai{why need this rewritten}$\overline{\Delta}_{i,\max} = \mu_{i,\overline{m}_i}$ and $\underline{\Delta}_{i,\max} = \mu_{i,\underline{m}_i}$ as the maximum player-optimal and pessimal stable regret suffered by $p_i$ at a time, respectively.
\end{definition}
