\section{ID homomorphisms} \label{sec:CID-homomorphisms}
\subsection{Properties preserved given an ID homomorphism} \label{sec:properties-preserved-proofs}


We now prove two
properties that are preserved by any ID homomorphism:\footnote{An ID homomorphism is analogous to the notion of graph homomorphism from graph theory, which essentially requires that edges are preserved along the map. 
In fact, if we would consider every node in an ID graph as a decision and as having an edge to itself, then any ID homomorphism is also a graph homomorphism when considering the two ID graphs as ordinary graphs (ignoring node types).}
solubility, and VoI.~\looseness=-1

\lecidhomsufficientrecall*

\begin{proof}
Since $\sG$ is soluble, there is a total ordering of decisions $<$  such that for all $D^1<D^2$, $\Pi^{D} \dsep U(D^2) \mid \Fa(D^2)$. 
To show that $\sG'$ is also soluble, we use $<$ to construct an ordering on the decisions of $\sG'$ that has the same property. 
We define $<'$ for decisions in $\sG'$: as $D^1 <' D^2$ when:~\looseness=-1
\begin{itemize}
    \item $h(D^1) \neq h(D^2)$ and $h(D^1) < h({D^2})$; or
    \item $h(D^2) = h(D^2)$ and $\sG'$ contains $D^1 \to D^2$.
\end{itemize}
This is a total order since whenever $h(D^1) \neq h(D^2)$ then either $h(D^1) < h(D^2)$ or $h(D^1) > h(D^2)$ by the total order on $\sG$, and whenever $h(D^1) = h(D^2)$ then $\sG'$ contains $D^1 \to D^2$ or $D^1 \leftarrow D^2$ by (\combinesonlylinkeddecisions).

\newcommand{\porig}{p_\sorig}
\newcommand{\porigwalk}{w}
Now we show that for any two decisions $D^1,D^2$ in $\sG'$, with $D^1<'D^2$, that any path $p\colon \Pi^{D^1} \to D^1 \upathto U$ for some $U \in \Desc (D^2)$ cannot be active given $\Fa(D^2)$. Consider two cases:

	\textit{Case (1) }: Assume $h(D^2)\!=\!h(D^1)$. Then $D^1\!<\!D^2$ so $\calG'$ contains $D^1 \to D^2$ by definition of $<'$, 
	and so any path $p\colon \Pi^{D^1} \to D^1 \upathto U$ that starts with the link $\Pi^{D^1} \to D^1 \to Y$ is blocked at $D^1$ given $\Pa({D^2})$.
	Any path that begins as $\Pi^{D^1} \to D^1 \leftarrow Y$ is blocked at the non-collider $Y$:
	the presence of $D^1 \leftarrow Y$ implies 
	that $\sG$ contains $h(D^1) \leftarrow h(Y)$ (\preserveslinks),
	so that $\sG'$ contains $D^2 \leftarrow Y$ (\coversallinfolinks), 
	and $Y \in \Pa(D^2)$.
	
	
	\textit{Case (2)} : Assume $h(D^2)\neq h(D^1)$. 
	We will prove the contrapositive:
	if $\Pi^{D^1} \not\dsep U(D^2) \mid \Fa(D^2)$ in $\sG'$
	then $\Pi^{h(D^1)} \not\dsep U(h(D^2)) \mid \Fa(h(D^2))$ in $\sG$ where $h(D^1)<h(D^2)$.
	If $p\colon\Pi^{D^1} \to D^1 \upathto U(D^2)$ is active given $\Fa(D^2)$, then consider the walk with node repetition $\porigwalk\colon\Pi^{h(D^1)} \to h(D^1) \upathto h(U(D'))$ consisting of 
	$f(V)$ for each node $V$ in $p$. 
	We know that each $V$ in $p$ is a (chain/fork/collider) if and only if 
	$h(V)$ is a (chain/fork/collider) in $\porigwalk$, since if there is a link $N\to V$ or $N\gets V$ then there must be a link $h(N)\to h(V)$ or $h(N) \gets h(V)$ respectively by the (\preserveslinks) assumption of ID homomorphisms. 
	And that each node $V$ contains a descendant in $\Fa(D^2)$
	if and only if $h(V)$ contains a descendant in $\Fa(h(D^2))$.
	So every collider in $w$ has a descendant in $\Fa(h(D^2))$ while every non-collider does not.
	This implies that $\Pi^{h(D^1)} \not \dsep U(h(D^2)) \mid \Fa(h(D^2))$ by \cref{21jan21.2-excising-loops-from-active-walks-to-get-active-paths}, 
	and we know that $h(D^1)<h(D^2)$ by the definition of $<'$
	so the result follows.
\end{proof}



We can now define how a homomorphism allows us to define a procedure for 
transporting IDs between the two graphs, such that corresponding 
IDs and policies lead to equivalent outcomes.






\lecidhomequivalence*

\begin{proof}
We define the \emph{ported ID} $\calM=(\sG,\dom,P)$ as follows: Each non-utility node $N$ in $\sG$ has $\dom(N) =  \prod_{N^i\in h^{-1}(N)} \dom(N^i)$, and each utility node $U$ in $\sG$ has $\dom (U) = \mathbb R$. Each non-decision node $N$ has as $P^N(n|\pa)$ the joint conditional distribution of each $\tilde P^{N^i}(n^i|\pa^i)$. We define the \emph{ported policy} $\pi$ so that each decision $D$ has as $\pi^D(d|\pa)$ the joint conditional distribution of each $\pi^{D^i}(d^i|\pa)$. These in fact factor over $\sG$  by property (b) of ID homomorphisms.


We show the result by induction on the graph of nodes $N^i$ in $\Gsplit$. Let $N=\sorig(N^i)$. 

\sloppy \textit{base step} : \textit{Assume $N^i$ has no parents in $\Gsplit$}. Then $P(N\!=\!(n^1,...,n^k)) =P^N((n^1,...,n^k))=\prod_{i=1}^k \tilde P^{N^i}(n^i|n^1,...,n^{i-1})
=\tilde P(N^1\!=\!n^1,...,N^k=n^k)$.

\textit{inductive step} : \textit{Assume that for all parents $Y^i$ of $N^i$, letting $Y=\sorig(Y^i)$, we have that $P(Y\!=\!(y^1,...,y^k))=\tilde P(Y^1\!=\!y^1,...,Y^k=y^k)$}. Then
\begin{alignat*}{2}
    &P(N\!=\!(n^1,...,n^k))\\
    &=  \sum_{y^1,...,y^k}P(Y\!=\!(y^1,...,y^k))\cdot P^N((n^1,...,n^k)|y^1,...,y^k)
    \quad\quad &&{}\\
    &=\sum_{y^1,...,y^k}\tilde P(Y^1\!=\!y^1,...,Y^k\!=\!y^k))\cdot \prod_{i=1}^k \tilde P^{N^i}(n^i|n^1,...,n^{i-1}, y^1,...,y^k)
    \quad\quad && \text{}\\
    &=\tilde P(N^1\!=\!n^1,...,N^k=n^k) \quad &&\text{} 
\end{alignat*}
Which shows the result.
\end{proof}

We will write 
the ``transported ID'' and policy from \cref{le:cidhom1-equivalence} as
$h(M)$ and $h(\pi)$. This means that we also treat a homomorphism $h$ as a function between IDs and policies. In fact, in order to show that ID homomorphisms preserve VoI %
, we show that on policies, $h$ is a bijection, which relies on the properties (c) and (d) of ID homomorphisms,  and is the primary reason why (c,d) are included:
\begin{restatable}{lemma}{lecidhompolicybijection} \label{le:bijection-between-deterministic-policies-on-split-graph}
Any ID homomorphism $h$ is a bijection from (optimal) policies on $\tilde \calM$ to (optimal) policies on $\calM=h(\tilde \calM)$. 
\end{restatable}


\begin{proof}
We define an inverse for the map as follows: Take a policy $\pi_D$ on $M$.
This gives a joint distribution $\tilde \pi^{D^i}=(\pi^{D})^i$ over $\dom(D^i)$ for $D^i\in \mathcal D=h^{-1}(D)$. Moreover, for any $X^j\in \Pa(D)$ and for any $X^{j,k}\in h^{-1}(X^j)$,  each decision $D^i$ has $X^{j,k}\in \Pa(D^i)$  by (\coversallinfolinks), and since these decisions $D^i$ form a complete graph (each $D^i$ is linked to each $D^j$) 
by condition (\combinesonlylinkeddecisions), this distribution $\tilde \pi^{D^i}$ also factors over $\tilde \sG$ and hence is a policy $\tilde M$. But this is precisely the definition of $\pi^D$ being the transported policy of $\tilde \pi^{\mathcal D}$, so that $\pi^D \mapsto \pi^{\mathcal D}$ is indeed the desired inverse. The optimal version of this lemma then follows from \cref{le:cidhom1-equivalence}.
\end{proof}









\lecidhompreservesmateriality*

\begin{proof}
\newcommand{\MatValue}[2]{\EE^{#2}_{#1}(\totutilvar)}
\newcommand{\maxMatValueExpanded}[3]{\max\limits_{#1}\MatValue{#2}{#3}}
\newcommand{\maxMatValueInlineExpanded}[3]{\max_{#1}\MatValue{#2}{#3}}
\newcommand{\maxMatValue}[2]{\maxMatValueExpanded{#1}{#1}{#2}}
\newcommand{\maxMatValueInline}[2]{\maxMatValueInlineExpanded{#1}{#1}{#2}}
Let $\mathcal X$ be the set of nodes $X^j$ in $\sG'$ such that $h(X^j)=X$, and $\mathcal D$ the set of nodes $D^j$ such that $\sorig(D^j)=D$. 

Firstly, note that $\sorig$ is also a homomorphism from  $\sG'_{\mathcal X\to \mathcal D}$ to $\sG_{X\to D}$ and from  $\sG'_{\mathcal X\not\to \mathcal D}$ to $\sG_{X\not\to D}$ (since in both cases, there is still an edge $X^i\to D^i$ iff there is an edge $X\to D$). Hence, for any policy $\pi'$ on $M'$ and letting $\pi=h(\pi')$ be the corresponding policy on $M$, apply \cref{le:cidhom1-equivalence} twice to conclude that 
$\MatValue{\pi}{M_{X\to D}}=\MatValue{\pi'}{M'_{\mathcal X\to \mathcal D}}$
and 
$\MatValue{\pi'}{M'_{X\not\to D}}=\MatValue{\pi}{M_{\mathcal X\not\to \mathcal D}}$.


Since the map that maps a policy on $M'$ to the corresponding policy on $M$ (see \cref{le:cidhom1-equivalence}) is a bijection by  \cref{le:bijection-between-deterministic-policies-on-split-graph}, this implies that 
$\maxMatValueInline{\pi}{M_{X\to D}}=\maxMatValueInline{\pi'}{M'_{\mathcal X\to \mathcal D}}$
and 
$\maxMatValueInline{\pi}{M_{X\not\to D}}=\maxMatValueInline{\pi'}{M'_{\mathcal X\not\to \mathcal D}}$.
These imply:
\begin{alignat*}{2}
    \maxMatValue{\pi}{M_{X\to D}} 
    &= \maxMatValue{\pi'}{M'_{\mathcal X\to \mathcal D}} \quad\quad &&: \text{by the argument above}\\
    &\geq \maxMatValue{\pi'}{M'_{X^i\to D^i}} \quad\quad &&: \text{more infolinks cannot decrease max utility}\\
    &> \maxMatValue{\pi'}{M'_{X^i\not\to D^i}} \quad &&:\text{by assumption: $X^i$ is material for $D^i$} \\
    &\geq 
    \maxMatValue{\pi'}{M'_{\mathcal X\not\to \mathcal D}} \quad &&:\text{more infolinks cannot decrease max utility} \\
    &= 
    \maxMatValue{\pi}{M_{X\not\to D}} \quad &&:\text{by the argument above}
\end{alignat*}
which shows the result.
\end{proof}































\subsection{Transformations that ensure a homomorphism} \label{app:ensure-cidhom}
\lecidhomcomposition*%

\begin{proof} We show that each of the four properties is preserved under composition:

(a) If $h$ and $h’$ preserve node types, then clearly so does $h \circ h'$. 

(b) If $\sG''$ contains $A \to B$ then by (b) for $h'$, 
$\sG'$ contains $h’(A)\to h'(B)$ or $h'(A)=h'(B)$. 
In either case, (b) for $h$ implies that
$\sG$ contains $h\circ h'(A) \to h\circ h'(B)$ or $h\circ h'(A) = h\circ h'(B)$.

(c) If $\sG$ contains $h\circ h'(N) \to h\circ h'(D)$, then by (c) for $h$, $\sG'$ contains $h’(N) \to h'(D)$ and by the same argument 
$\sG$ contains $N\to D$.

(d) Assume $h\circ h'(D^1)=h\circ h'(D^2)$ and $D^1\!\neq\! D^2$ in $\sG''$. 
Then if $h’(D^1)\!=\!h'(D^2)$, by (d) for $h'$, 
$\sG''$ contains
$D^1\!\to\! D^2$ or $D^2\!\to\! D^1$ showing the result. 
If $h’(D^1)\neq h'(D^2)$ then by (d) for $h$, 
$\sG'$ contains $h’(D^1)\to h'(D^2)$ or $h’(D^2) \to h'(D^1)$, and hence by (c) for $h'$, $\sG''$ contains $D^1 \to D^2$ or $D^2\to D^1$.
\end{proof}































\lecopyingcidhom*
\ryan{I've simplified/shortened this a bit further. Feel free to revert any changes is preferred.}

\begin{proof}
ID homomorphism condition (a) follows by definition. (b) follows from the definition of $E'$. (c,d) follow since they hold for all nodes $N$ by definition, including the decisions.
\end{proof}




\lepruningcidhom*

\begin{proof}
The homomorphism properties follow:
(a) by definition. (b) from $E'\!\subseteq\! E$, (c) from every $N\!\to\! D \in \sD$ 
being in $E$, (d) from $h$ being the identity map so every $D^1\!\neq\! D^2$ has $h(D^1)\!\neq\! h(D^2)$.~\looseness=-1
\end{proof}