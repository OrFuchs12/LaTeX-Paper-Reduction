






\section{Discussion and Conclusion} \label{sec:discussion}


This paper has described techniques for analyzing soluble influence diagrams.
In particular, we introduced ID homomorphisms, a method for transforming ID graphs while preserving key properties, and showed how these can be used to establish equivalent ID graphs with conveniently parameterizable ``trees of systems''.
These techniques enabled us to derive the first completeness result for a graphical criterion for value of information in the multi-decision setting.

Given the promise of reinforcement learning methods, it is essential that we obtain a formal understanding of how multi-decision behavior is shaped.
The graphical perspective taken in this paper has both advantages and disadvantages. On the one hand, some properties cannot be 
distinguished from a graphical perspective alone. On the other hand, it means our results are applicable even when the precise relationships are unspecified or unknown. 
There are a range of ways that this work could be beneficial.
For example, analogous results for the single-decision setting have contributed to safety and fairness analyses
\citep{Armstrong2020pitfalls,cohen2020unambitious,everitt2019tampering,Everitt2019modeling,langlois2021rl,Everitt2021agent}.~\looseness=-1

Future work could include applying the tools developed in this paper to other incentive concepts such as value of control \citep{shachter1986evaluating}, instrumental control incentives, and response incentives \citep{Everitt2021agent}, 
to further analyse the value of remembering past decisions
\citep{Shachter2016,lee2020characterizing},
and to generalize the analysis to multi-agent influence diagrams \citep{Hammond2021equilibrium,Koller2003}.









