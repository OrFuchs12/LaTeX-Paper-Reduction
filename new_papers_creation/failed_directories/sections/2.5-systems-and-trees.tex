








\section{Systems and trees of systems in an ID graph} \label{sec:preliminaries-systems-and-trees}


\subsection{Systems}

Before detailing the properties of systems, we first recap the elements of a system.
We call $D^s$, $U^s$, $X^s$, and $\sinfolink^s\colon X^s \to D^s$ the \emph{decision node}, \emph{utility node}, \emph{info node}, and \emph{infolink} of $s$, respectively, and refer to $\scontrol^s$, $\sinfo^s$ and $\sobspaths^s(C)$ for each collider $C$ in $\sinfo^s$ as the \emph{paths of $s$}.

\begin{definition}[Elements of a system] \label{def:back-&front-sections-of-system}
For a system $s$:
    \begin{itemize}
        \item An \emph{obs node} $O$ of $s$ is the penultimate node of each obs path $\sobspaths^s(C)$.\footnote{For ``observation node". But note that though $D^i$ does ``observe" $X^i$, it is not an obsnode, since it is not the penultimate node of an $\sobspaths^s(C)$, but is the first node of $\sinfo^s$.}

        \item The \emph{question node}  $Q^s$, if $\sinfo^s$ contains at least one fork node, is the closest-to-$U^s$ fork node on $\sinfo^s$.\footnote{This implies that the segment of the info path from $Q^s$ to $U^s$ is a directed path $Q^s\pathto U^s$, since there are no fork nodes on that path, and it must begin and end with an arrow towards $U^s$.}
        
        \item The \emph{back section}, if $\sinfo^s$ contains a fork, is the set of nodes in $X^s \upathto Q^s$ in $\sinfo^s$ 
        (including $X^s$ and $Q^s$)
        and in each $\sobspaths(C^i)$, except for $D^s$. Otherwise, the back section is empty.
        
        \item The \emph{front section} consists of the nodes in any path in $s$
        that are not in the back section.
    \end{itemize}
\end{definition}


\begin{definition}[Within-system links and paths] \label{sep12.4-within-system-within-tree-links}
A link $A\to B$ that is in $\sinfo^s$, $\scontrol^s$, or any $\sobspaths^s(C^i)$, or the link $X^s\to D^s$ for some system $s$ is called a \emph{within-system-$s$} or \emph{within-system link}. A \emph{within-system path} is a path that contains only within-system links.
\end{definition}

We will now prove a number of fundamental properties of systems.


\begin{lemma}[Basic properties of a system in a soluble ID graph] \label{le:20Nov7.1-Basic-properties-of-systems-with-sufficient-recall}
Any system $s$ in a soluble ID graph has the properties: 
\begin{enumerate}[label=(\alph*)]
    \item (No infolinks in the back-section) \label{le:20Nov7.1a-no-decisions-in-the-back-section} 
    The back section of $s$ can only contain a decision $D' \in \sD$ if $D'=X^s$, and the infopath $\sinfo^s$ is front-door. Moreover then $D'$ is not in any $\sobspaths^s(C^i)$.
    \chris{Above can be phrased more neatly. but not a priority.}
    
    \item (Infolinks in $s$ are descendants of $D^s$) \label{le:20Nov7.1b-info-path-decisions-observe-the-control-path}
    An information link $N \to D'$ for $D'\neq D^s$ can only be contained in a path in system $s$ if the control path $\scontrol^s$ contains a parent of $D'$, so that $D' \in \Desc(D^s)$.
    
    \item (Parents of ancestor decisions are parents of $D^s$) \label{le:20Nov7.1c-parents-of-ancestors-are-parents}
    A node $N$ in system $s$ can only be a parent of an ancestor decision $D'$ of $D^s$ if $N$ is also a parent of $D^s$.\footnote{Note that in a normal form tree (see below), this link $N \to D'$ is an out-of-tree link}
\end{enumerate}
\end{lemma}



\begin{proof} We prove each property in succession:

(a) (No infolinks in the back section) 
We will prove what restrictions are implies by considering sequentially the cases where 
$D'$ is in either the infopath, or in the observation path.
To begin with, let us state what we know in both cases:
$D'$ must be an ancestor of $D^s$. 
As such, $D'<D^s$ in any topological ordering, 
so solubility requires that $D' \dsep U^s \mid \Fa(D^s)$.

If $D' \in \sinfo^s$, then the path
$p\colon \Pi'\to D' \oset[0.5ex]{\sinfo^s}{\upathto} U^s$ may be
formed from by truncating the infopath $\sinfo^s$.
By solubility, $p$ must be blocked given $\Fa(D^s)$.
We know, however, that
$\sinfo\colon X^s \oset[0.5ex]{\sinfo^s}{\upathto} U^s$
is active given $\Fa({D^s}) \setminus \{X^s\}$.
If $D' \neq X^s$ then $p$ does not contain $X^s$, and so it is is active given $\Fa({D^s})$, 
violating solubility.
Moreover, if $D'=X^s$ and $\sinfo$ is a backdoor path, then $p$ will have a collider at $D'$, 
and solubility is violated once again.
So $\sinfo^s$ can only contain a decision $D'$ if $D'=X^s$ and $\sinfo^s$ is frontdoor.

Now we will prove that $D' \not \in \sobspaths(C)$, by contradiction.
Suppose that $D' \in \sobspaths(C)$.
Then, consider 
the path $q\colon \Pi'\to D' \oset[0.5ex]{\sobspaths^s(C)}{\dashleftarrow} C \oset[0.5ex]{\sinfo^s}{\upathto} U^s$,
constructed by truncating the observation path and infopath.
By assumption, the path $C\oset[0.5ex]{\sobspaths^s(C)}{\pathto} Y\to D$ is minimal-length, 
so no node $W\neq Y$ on the path can be a parent of $D^s$, 
and so $D^s \oset[0.5ex]{\sobspaths^s(C)}{\dashleftarrow} C$ is active given $\Fa(D^s)$.
The segment $C \oset[0.5ex]{\sinfo^s}{\upathto} U^s$ is active given $\Fa(D^s) \setminus \{X^s\}$.
Since $\sinfo^s$ is a path, the segment $C \oset[0.5ex]{\sinfo^s}{\upathto} U^s$ cannot contain $X^s$, 
and thus is active given $\Fa(D^s)$.
So the path $q$ is active given $\Fa(D^s)$, violating solubility.

Together, these two cases prove the result.

(b) (Infolinks in $s$ are descendants of $D^s$) 
We know from sublemma \ref{le:20Nov7.1a-no-decisions-in-the-back-section} that
the back section cannot contain any link $N \to D'$.
So $D'$ must lie in the front-section of $s$:
either in $Q^s \oset[0.5ex]{\sinfo^s}{\upathto}U^s$, or in $\scontrol^s$.
In either case, we have $U^s \in \Desc(D^s)$ and $U^s \in \Desc(D')$.
So in order for the ID graph to be soluble, 
we must have either $\Pi^{D'} \dsep U^s \mid \Fa(D^s)$ 
or $\Pi^{D^s} \dsep U^s \mid \Fa(D')$.


We can show that the first case $\Pi^{D'} \dsep U^s \mid \Fa(D^s)$ cannot hold.
If $D'$ is in $\scontrol^s$, note that $\scontrol^s$ consists of only descendants of $D^s$. 
If $D'$ is in $q:Q^s \oset[0.5ex]{\sinfo^s}{\upathto}U^s$ then note that $q$ is assumed to be active given $\Fa(D^s) \setminus \{X^s\}$, 
and cannot contain $X^s$.
In either case, $\Pi^{D'} \not \dsep U^s \mid \Fa(D^s)$.
Hence we must have $\Pi^{D^s} \dsep U^s \mid \Fa(D')$, from which
it follows that every directed path from $D^s$ to $U^s$ (including $\scontrol^s$) must contain a parent of $D'$.

(c) (Parents of ancestor decisions are parents of $D^s$) Assume $N$ is a parent of $D'$ in a path of $s$. 
It cannot be in $\scontrol^s$, because then $D'$ would be a descendant of $D^s$.
So $N$ must be in $\sinfo^s$ or one of $\sobspaths^s(C)$.
If $N$ is in $\sinfo^s$, consider the path
$p\colon \Pi^{D’} \to D’ \leftarrow N \oset[0.5ex]{\sinfo^s}{\upathto} U^s$.
We know $\sinfo^s$ is active given $\Fa({D^s}) \setminus \{X^s\}$.
Hence if $N \notin \Pa(D^s)$, then $p$ is active given $\Pa({D^s})$ and since it doesn't contain $D^s$ also active given $\Fa(D^s)$, violating solubility. Hence $N\in \Pa(D^s)$.

Similarly, if $N$ is in $\sobspaths^s(C)$, then consider the path
$q\colon \Pi^{D'}\to D'\leftarrow N\oset[0.5ex]{\sobspaths^s(C)}{\dashleftarrow} C \oset[0.5ex]{\sinfo^s}{\upathto} U^s$.
Hence if $N \notin \Pa(D^s)$, then since $\sobspaths^s(C)$ is minimal-length, it holds that $N\oset[0.5ex]{\sobspaths^s(C)}{\dashleftarrow} C$ is active given $\Fa(D^s)$, as in the proof of \ref{le:20Nov7.1a-no-decisions-in-the-back-section}.
Moreover, the segment
$C\oset[0.5ex]{\sinfo^s}{\upathto}U^s$ is active given $\Fa(D^s)\setminus \{X^s\}$ by assumption, and hence given $\Fa(D^s)$.
Since $D' \in \Anc(D^s)$, $q$ is active given $\Fa(D^s)$, again violating solubility. Hence again $N\in \Pa(D^s)$.
\end{proof}






\subsection{Trees of systems}
First, let us recap the definition of a tree of systems.

\deftree*
We define the \emph{predecessor system} 
and \emph{predecessor path} of system $s^i$ 
as $(\predsys{s^i},\predpath{s^i}):=\pred{s^i}$.
Moreover, we will sometimes say ``An ID graph with tree" to refer to an ID graph, together with a tree on that ID graph.

\begin{terminology*}
If $s^i=\predsys{s^j}$ then we say that $s^j$ is a child system of $s^i$. 
We will similarly apply the standard terminology of trees and graphs: Ancestor system, descendant system.
\end{terminology*}









\begin{lemma} [A tree of systems has a tree structure] \label{le:21Jan21.1-tree-of-systems-has-tree-structure}
Given a tree of systems $T=(\calS, \spred)$, the pair $(\calS,\spredsys)$ is a tree structure, i.e. it satisfies:
\begin{itemize}
    \item There is a unique node $\ssRoot$ that has $\spredsys(\ssRoot)=\ssRoot$; and
    
    \item For any node $s$, there is some number $n\in \mathbb{N}$ such that $\spredsys^n(s)=\ssRoot$.
\end{itemize}
\end{lemma}

\begin{proof}
The first condition is satisfied directly by definition of $\ssRoot$. For the second condition, we only need to show that for any $s^i$, there is a sequence of systems $(s^1,...,s^n)$ such that $s^1=\ssRoot$ and $s^n=s^i$, and $\predsys{s^j}=s^{j-1}$ for all $1<j\leq n$. Assume by contradiction that there is a system that doesn't satisfy this, and let $\calS^*$ be the set of all such systems. Then since the restrition of $\spredsys$ to $\calS^*$ has no fixed points ($\ssRoot$ is the only fixed point and is not in $\calS^*$ by definition), it must have some sequence $(\tilde s^1,...,\tilde s^k)$ with $\spredsys(\tilde s^1)=\tilde s^k$ and $\spredsys(\tilde s^j)=\tilde s^{j-1}$ for all $1<j\leq k$ (i.e. a cycle). But this would imply that there is at least one pair of systems $(\tilde s^k,\tilde s^{m})$ with $\spredsys(\tilde s^k)=\tilde s^m$ but where $D^{\tilde s^k}$ is a later decision than $D^{\tilde s^m}$, contradicting \autosubref{le:20Nov7.1-Basic-properties-of-systems-with-sufficient-recall}{le:20Nov7.1b-info-path-decisions-observe-the-control-path}.
\end{proof}













\newcommand{\decisionsindescendantsystemsaredescendantsstatement}{If $s'$ is a descendant system of $s$, then $D^{s'}$ is a descendant node of $D^s$.}

\begin{lemma}[Basic properties of a tree in a soluble ID]
\label{le:20dec14.1-basic-properties-of-trees-with-SR}
Let $T$ be a tree on a soluble ID graph. Then:
\begin{enumerate}[label=(\alph*)]
    \item (Decisions in descendant systems are descendants) \label{le:20dec14.1a-decisions-in-descendant-systems-are-descendants}
    \decisionsindescendantsystemsaredescendantsstatement

    \item (Info links to ancestor decisions only from obsnodes) \label{le:20dec14.1b-only-info-links-from-observation-nodes-to-ancestor-decisions}
    Let $s'$ be a descendant system of $s$. If there is a link from a node $V$ in 
    any path in
    $s'$ 
    to any node in $\sD \cap \Anc(D^s)$ (including $D^s$), then either: 
    i) $V$ is an obsnode in $s'$, 
    or ii) $V=X^{s'}=X^s$.
\end{enumerate}
\end{lemma}




\begin{proof} We prove each property in succession:

\sublemmaproof{a}{Decisions in descendant systems are descendants}
If $s'$ is a child system of $s$, 
then $D^{s'}$ is a descendant of $D^s$ by 
\autosubref{le:20Nov7.1-Basic-properties-of-systems-with-sufficient-recall}{le:20Nov7.1b-info-path-decisions-observe-the-control-path} 
(since it cannot lie in the back section by 
\autosubref{le:20Nov7.1-Basic-properties-of-systems-with-sufficient-recall}{le:20Nov7.1a-no-decisions-in-the-back-section}).
By induction the result follows: If any system $s'$ with child system $s''$ is a descendant system of $s$, then $D^{s''} \in \Desc(D^{s'})$, and by the induction assumption we know $D^{s'} \in \Desc(D^{s})$, so that $D^{s''} \in \Desc(D^{s})$.~\looseness=-1


\sublemmaproof{b}{Only info links from obsnodes to ancestor decisions}
Since $\sinfo^s$ is active, and each $\sobspaths^s(C)$ is a minimal length path,
the only parents of $D^{s'}$ within system $s'$ (i.e. the only nodes in $\Pa(D^{s'})\cap \sV^{s'}$) are $X^{s'}$ and the obsnodes of $s'$. 
Therefore, by \autosubref{le:20Nov7.1-Basic-properties-of-systems-with-sufficient-recall}{le:20Nov7.1c-parents-of-ancestors-are-parents} and using Sublemma \ref{le:20dec14.1a-decisions-in-descendant-systems-are-descendants} that $D^s$ is an ancestor of $D^{s'}$ (since $s$ is an ancestor of $s'$), these are the only nodes in $s’$ that can be parents of $D^s$ or of ancestor decisions of $D^s$. 

To show the result we show that $X^{s'}$ cannot be such a parent when {$X^{s'} \neq X^s$}:
Let $s^*$ be the closest-to-$s'$ ancestor of $s'$ in the tree such that {$X^{s*}\neq X^{s'}$}. Assume such $s^*$ exists and either equals $s$ or is a descendant of $s$ since otherwise $X^{s}$ would equal $X^{s'}$, which would show the result.
We know that $X^{s'}$  is in the system $s^*$, since it is the closest-to-$s'$ system such that {$X^{s*}\neq X^{s'}$}, so that there is a child system of $s^*$ whose info node equals $X^{s'}$ {} and hence must be part of an info-link in $s^*$. 
Hence $X^{s'}$ cannot be a parent of $D^{s^*}$ since the only parents within a system of that system's decision other than its info node are its obsnodes, but $X^{s'}$  cannot be one of the obsnodes since then $D^{s'}$ would have to be in the back section, which would violate \autosubref{le:20Nov7.1-Basic-properties-of-systems-with-sufficient-recall}{le:20Nov7.1a-no-decisions-in-the-back-section}{}.
But we assumed $s^*$ is a descendant system of $s$, and hence $D^{s^*}$ is a descendant decision of $D^s$ (by Sublemma \ref{le:20dec14.1a-decisions-in-descendant-systems-are-descendants}), which implies that $D^s$ and its ancestor decisions also don't have $X^{s'}$ as a parent (due to the result shown in the previous paragraph).~\looseness=-1
\end{proof}













\subsection{Normal form trees of systems} \label{sec:normal-form-trees}

In this section, we will prove that 
in a \emph{normal form tree},
a system can only get information from 
its own parents, and obsnodes of descendant systems.

\defnormalform*

We will also use the components of the definition of normal form tree separately. 


\begin{lemma}[Concrete properties of position-in-tree-uniqueness]\label{le:21-may-9-position-in-tree-uniqueness-properties}
A tree $T$ satisfied position-in-tree-uniqueness if and only if every node $N$ that is in some path of some system in $T$ lies in precisely one path $p$ of one system $s$, with four exceptions:
    \begin{itemize}
        \item If $N$ is a collider node in 
        path $p=\sinfo^s$ 
        then it is also the first node in 
        $\sobspaths^s(N)$.
        
        \item If $N=U^s$ then it lies in both $\sinfo^s$ and $\scontrol^s$.
        
        \item If $N$ is in an infolink $X^{s'} \to D^{s'}$ (with $s' \neq s$) on path $p$, 
        then $N$ is also in $\sinfo^{s'}$ (if $N=X^{s'}$), or also in $\scontrol^{s'}$ and in $\sobspaths^{s'}(C^i)$ for each collider $C^i$ in $\sinfo^{s'}$ (if $N=D^{s'}$). In both cases 
        $N$ may also be the info node for exactly one of its child systems $s^1$, of exactly one child system $s^2$ of $s^1$, and so on.
        Formally, $N=X^{s^1}=...=X^{s^n}$ where each $s^{i}$ is a child system of $s^{i-1}$.
    \end{itemize}
\end{lemma}
\begin{proof}
First we show that if a tree $T$ satisfies position-in-tree-uniqueness, then the result is true. Assume that a tree $T$ does not satisfy the required property, i.e. there is at least one node $N$ that is in multiple paths, but without satisfying one of the exceptions. Then by \cref{le:21jan25.6-first-split-preserves-tree} a different tree $T'$ can be obtained by applying graph transformation 1 (\cref{def:21jan25.2-first-split-to-ensure-systems-and-paths-uniqueness}), where $N$ is replaced with different nodes in those paths.~\looseness=-1

Now we show the other direction. Assume that the property holds.
Assume $N$ is part of two paths $p^1$ and $p^2$. Then one of the three exceptions must apply. If the first exception applies, then $N$ is a collider, and $p^1=\sinfo^s$ and $p^2=\sobspaths^s(N)$, so that replacing $N$ with two separate nodes on $p^1$ and $p^2$ would make that the obspath of $N$ no longer starts with a collider on $\sinfo^s$. If the second one applies, then replacing $N=U^s$ with two nodes would mean that the control and info path no longer end at the same utility node. If the third case applies and $N=X^{s'}$ of a  descendant system $s'$ of $s$, then $p^1, p^2$ equal $\sinfo^{s'}$ and $\spredpath^{s'}$. Replacing $N$ with two nodes on the two paths would break the required property on $\spred$ for $T$ to be a tree. If the third case applies and $N=D^{s'}$, then $p^1,p^2$ equal two of: $\scontrol^{s'}$, $\spredpath^{s'}$ or one of $\sobspaths^{s'}$. If one of them equals $\spredpath^{s'}$, then replacing $N$ with two nodes would again break the required property on $\spred$ for $T$ to be a tree. Otherwise, it would mean that at least one of the $\sobspaths^{s'}$ no longer ends at $D^{s'}$, so that $s'$ would no longer be a system.~\looseness=-1
\end{proof}




\begin{definition} [Base system and path of a node; chain of systems] \label{def:21feb17.2-base-sytem-and-path-of-node}
If $T$ is a normal form tree of systems, then we refer to the system $s$ and the path $p$ from
\cref{le:21-may-9-position-in-tree-uniqueness-properties}
(including in the exceptions) respectively as the \emph{base system} and \emph{base path} of node $N$. 
\end{definition}
Note that this implies that a utility node $U^s$ has no base path. We refer to the sequence of systems of which a node $N$ is the info node (in the third exception) as the \emph{chain of systems} of $N$ (which is possibly empty).

\begin{definition}
A within-tree-$T$ path for a normal form tree $T$ on an ID graph is a path that contains only within-system links for the systems in $T$.
\end{definition}

Note that we define the notion of within-tree path only for normal form trees, since it is not sensible for trees that don't satisfy \systemsAndPathsUniqueness: If a node $N$ occurs in two unrelated systems, then a sequence of within-tree links may jump between nodes in the tree that are not linked.


\subsection{Properties of normal form trees of systems} 
\label{sec:graph-knowledge-lemma-proof}


In this subsection, we will prove \cref{le:2v2-graph-knowledge-lemma} --- that 
the only information that $D^s$ receives that is relevant within system $s$ is information that it receives from its parents and obsnodes of descendant systems.
To reach this result, we first need to state some more fundamental properties of normal form trees.




\newcommand{\atmostonebackdoorsystempernodestatement}{For any node there is at most one backdoor-info system $s$ of which it is one of the nodes in the link $X^s\to D^s$. }
\newcommand{\psys}{p_\mathrm{systems}}
\newcommand{\pwalk}{p_\mathrm{walk}}
\begin{lemma}[Properties of soluble ID graphs with trees that have \systemsAndPathsUniqueness]
\label{le:20Nov24.1-basic-properties-of-trees-with-sufficient-recall-and-unique-systems-and-paths}
Any soluble ID graph $\sG$ with a tree that has \systemsAndPathsUniqueness
has the following properties.
\begin{enumerate}[label=(\alph*)]


    \item (A within-tree path corresponds to a walk with node repetition in the tree of systems) \label{le:20Nov24.1a1-Within-tree-paths-correspond-to-paths-in-the-tree-of-systems}
    For any within-tree path $p\colon  N^1 \upathto N^n$, there is a walk with node repetition in the tree of systems $\psys\colon  s^1 \upathto s^m$, with $m\geq n$, together with a walk with node repetitions $\pwalk\colon V^1 \upathto V^m$ in $\sG$ such that each $V^i$ is in some path in system $s^i$ and if we remove from $\pwalk$ every node that equals its predecessor we obtain $p$.\footnote{Hence in particular, there can only be a within-tree path between a node $N^1$ in system $s$ and node $N^n$ in system $s'$ if there is a path between $s$ and $s’$ in the tree of systems.}

    
    \item (Within-tree links between systems only via $X^s$, $D^s$) \label{le:20Nov24.1a2-Within-tree-links-between-systems-only-via-Xs-Ds}
    If $N - N'$ is a within-tree link, where $N$ and $N'$ are in nodes in paths of systems $s$ and $s'$ respectively, and $s \neq s'$,
    then $N - N'$ must contain $X^s$ or $D^s$.
    
\end{enumerate}
\end{lemma}



\begin{proof} We prove each sublemma in succession:


\sublemmaproof {a} {Within-tree paths correspond to walks with node repetition in the tree of systems}.
We construct this walk with node repetition $\psys$ recursively as follows, by iterating from $N^1$ to $N^n$, using the fact that each link in $p$ is within-system for some system (see definition of within-tree paths). For the base case, let $s^1$ equal any of the systems that $N^1$ is a node in. Let $s^{k+1}$ and $V^{k+1}$ be defined mutually based on $s^{k}$ and $V^{k}$: If the node $N’$ that is next to $V^k$ on $p$ is also in system $s^k$, then let $s^{k+1}=s^k$ and let $V^{k+1}=N’$, in which case the desired result follows that $s^k=s^{k+1}$ and that $V^{k} - V^{k+1}$ is a link in $p$. If it is not also in system $s^k$, then by definition of within-tree path, $N’$ and $V^k$ are both in some system $s’ \neq s^k$, where $s’$ is part of the chain of systems of $V^k$. Then let $s^{k+1}$ be the next system from $s^k$ in that chain, and let $V^{k+1}=V^k$, from which the desired result follows that there is a link $s^k-s^{k+1}$ and $V^{k+1}=V^k$. Together with the base case this shows the result by induction.~\looseness=-1

\sublemmaproof{b} {Within-tree links between systems only via $X^s$, $D^s$}. 
Take any link $A - B$ with $A$ a part of $s$ and $B$ a part of some other system $s’$.
Then we must either have that $A$ is in both $s$ and in $\predsys{s}$, or that $B$ is in both $s$ and $\predsys{s}$. Whichever it is, by the \systemsAndPathsUniqueness assumption, this can only be if that node equals $X^s$ or $D^s$, since any node that is in multiple systems $s’$ must equal either $X^{s'}$ or $D^{s'}$ for all systems $s'$ except its base system.
\end{proof}








We now show graphically that in an ID graph with normal form tree a decision $D^s$ cannot get relevant information about system $s$ from any paths via nodes outside system $s$
and descendant systems. 
This will imply the following:


\graphKnowledgeLemma*
\ryan{graph knowledge lemma numbered as 43 here, but as 20 in the main paper}



\newcommand{\OutsideParents}{\Pa^{-s} \setminus \ObsDesc^s}
\newcommand{\InsideParents}{\Pa^{s} \cup \ObsDesc^s}
\Chris{I should try to simplify the proof below further.}
\begin{proof} Take any path from a node in $\Back^s$ to a node in $\OutsideParents$. We will show that it is inactive given $\InsideParents$.

We first assume that the path starts from a node in the back section, so that it is an ancestor of $D^s$.
First note that since the decision of $s$ and those of its descendant systems cannot be ancestors of $D^s$ (\autosubref{le:20dec14.1-basic-properties-of-trees-with-SR}{le:20dec14.1a-decisions-in-descendant-systems-are-descendants}), this implies that if the path contains any of these it is necessarily inactive given {$\InsideParents$}, since active paths between ancestors given ancestors contain only ancestors (\cref{le:2.12-active-paths-between-ancestors-contain-only-ancestors}).

So assume that the path does not contain the decision of $s$ (i.e. $D^s$) nor those of its descendant systems. We will consider the initial within-tree segment of the path. 

By \autosubref{le:20Nov24.1-basic-properties-of-trees-with-sufficient-recall-and-unique-systems-and-paths}{le:20Nov24.1a1-Within-tree-paths-correspond-to-paths-in-the-tree-of-systems}, this initial within-tree path corresponds to a walk with node repetition in the tree of systems, and since the latter has a tree structure by construction, this initial within-tree path either has to exit system $s$ via a node in its predecessor system, or stay within $s$ itself and its descendant systems. 
The former can only happen via one of the links via $X^s$ and $D^s$ by \autosubref{le:20Nov24.1-basic-properties-of-trees-with-sufficient-recall-and-unique-systems-and-paths}{le:20Nov24.1a2-Within-tree-links-between-systems-only-via-Xs-Ds}.


We first show that in this case, the path is blocked given $\InsideParents$. We already assumed that the path doesn’t contain $D^s$, so assume that the link contains $X^s$. Since $\sinfo^s$ is front-door by the appropriateness assumption of normal form tree, $X^s$ blocks the path, since $X^s\in \InsideParents$. 


So we now assume that the initial within-tree segment does not exit into the predecessor of $s$, and hence is contained within system $s$ and its descendant systems. Consider the first link of the path that is out-of-tree. At the start of this proof we assumed that the path doesn't contain the decisions $D^s$, nor $D^{s'}$ of any of its ancestors $s'$. Hence by \autosubref{le:20Nov7.1-Basic-properties-of-systems-with-sufficient-recall}{le:20Nov7.1a-no-decisions-in-the-back-section}{}, the only decision that the initial within-tree segment could contain is $X^s$ if that is a decision, but we just assumed that the path doesn't contain this.~\looseness=-1

So we assume now that the initial within-tree segment doesn't contain any decisions, so that the first out-of-tree link would have to be of the form $N \to D$ for some decision $D$ (by the no redundant links assumption of normal form trees, and using the fact that $N$ is inside the tree). $N$ can be either in system $s$ or in a descendant system, and can be either an observation node or some other node. Consider two exhaustive cases:

\begin{enumerate}[label=(\alph*)]
    \item Assume $N$ is neither an obsnode in $s$ nor in a descendant system $s’$. Then $D$ cannot be an ancestor of $D^s$, since the only info links from nodes in $s$ or its descendant systems to $D^s$ or to ancestor decisions are from obsnodes by \autosubref{le:20dec14.1-basic-properties-of-trees-with-SR}{le:20dec14.1b-only-info-links-from-observation-nodes-to-ancestor-decisions}, and hence the path cannot be active (active paths between ancestors given ancestors contain only ancestors by \cref{le:2.12-active-paths-between-ancestors-contain-only-ancestors}).
    
    \item Assume $N$ is an observation node of $s$  or of some descendant system of $s$. Then $N$ blocks the path, since it is in $\Pa^s\cup \ObsDesc^s$.
    
\end{enumerate}
This shows the result.
\end{proof}



