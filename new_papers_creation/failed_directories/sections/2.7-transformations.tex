
\subsection{Obtaining a (homomorphically) transformed ID graph with a normal form tree} \label{sec:cidhom-graph-splitting} 

We will prove that
if an infolink is in the minimal $d$-reduction, then
there exists a transformed ID graph with a normal form tree and homomorphism to the original.
We will show that a series of three homomorphic transformations %
yields a graph $\sG^3$ with tree $T^3$ is in normal form, and
root infolink corresponding to that of $T$.
Since each transformation is homomorphic, 
their composition is a
homomorphism from $\calG^3$ to $\calG$.
The transformations are:~\looseness=-1

\begin{itemize}
    \item First, we \emph{obtain a full tree} on $\calG$.
    \item \emph{Transformation 1} obtains
    $(\sG^1,T^1)$, where $T^1$ has property (a).
    This splits nodes other than $X^s$ and $D^s$, to ensure that 
    they do not appear in multiple positions in the tree.~\looseness=-1
    
    \item \emph{Transformation 2} obtains 
    $(\sG^2,T^2)$, where $T^2$ has the properties (a, b).
    This is done by modifying any backdoor infopath to be front-door.
    
    \item \emph{Transformation 3} obtains 
    $(\sG^3,T^3)$, where $T^3$ has the properties (a, b, c).
    This consists of removing edges other than the within-tree links.
\end{itemize}
We will not use the intermediate graphs $\sG^1,\sG^2$, except to define $\sG^3$.

\subsubsection{Obtain a full tree on $\calG$}
We will construct an arbitrary full tree using only infolinks in the minimal $d$-reduction.

\begin{lemma}[Existence of full tree]
\label{le:m3.1-existence-of-complete-tree-of-systems}
Let $\calG$ be a soluble ID graph whose minimal $d$-reduction $\calG^*$ contains the link $X \to D$.
Then there exists a full tree of systems $T$ on $\sG^*$ with root system on $X \to D$.
\end{lemma}

\begin{proof}
We construct a tree iteratively. 
Since $X \to D$ is in $\sG^*$, 
there exists a directed path $p$ from $X$ to some $U \in \Desc(D)$ active given $\Fa(D^i) \setminus \{X\}$.
Let the infopath be any such $p$, let the control path be any directed path from $D$ to $U$ and 
let the obspaths be the shortest directed paths to $D$ from each collider in $p$.
Then, choose any infolink $X' \to D'$ 
in a path $q$ of any system $s$ 
that lacks an associated system.
Since $X' \to D'$ is in $\calG^*$, 
we can choose paths in the same fashion
and repeat this procedure 
until every infolink that is traversed 
has its own system.
This process halts, because a
path in a system $s$ (whose decision is $D^s$)
can only contain an infolink $X' \to D'$ if $D' \in \Desc(D^s)$.
This is because:
i) $\scontrol^s$ is directed,
ii) $\sinfo^s$ only contains infolinks in $\Desc(D^s)$
(\autosubref{le:20Nov7.1-Basic-properties-of-systems-with-sufficient-recall}{le:20Nov7.1b-info-path-decisions-observe-the-control-path}),
and iii) $\sobspaths^s$ cannot contain any infolinks.
So a full tree has been constructed.~\looseness=-1
\end{proof}




\subsubsection{Transformation 1 (split): ensuring \systemsAndPathsUniqueness} \label{subsubsec:split-1}


For Transformation 1, a node $N$ is copied into a different node
(of unchanged type)
for each position that $N$ occupies in the tree. More precisely, we replace each node $N$ that is in path $p$ in system $s$, with the new node $\NewNode(N,s,p)$. This function is defined such that each node $\NewNode(N,s,p)$ has a unique position in the tree, which basically means that it is only a part of path $p$ in system $s$, except that we need to make sure that certain nodes are in multiple paths (e.g. a collider node $C$ in a path $\sinfo^s$ must be in both $\sinfo^s$ and in $\sobspaths^s(C)$). We don't delete the original occurrences of each node $N$, so that the original graph is a subgraph of the transformed graph.


\ryan{Seems we should be able to remove the name $Nsplit$ and just 
substitute in New()}

    
    



\begin{definition}[Graph transformation 1]
\label{def:21jan25.2-first-split-to-ensure-systems-and-paths-uniqueness}
Let $T^0=(\calS^0,\spred^0)$ be a tree on an ID graph $\sG^0=(\sV^0,E^0)$. Then define $\splitfirst(\sG^0,T^0)=(\sG^1, T^1)$
where $\sG^1=(\sV^1,E^1)$, together with homomorphism $\hfirst\colon\sV^1 \to \sV^0$ as

\begin{itemize}
    \item Obtain any $\sG^1$ and $\hfirst$ from \autoref{le:21may19.2-CID-hom-from-node-copying-and-deleting}, by adding for each node $N$ a set of copies
    \[\sCopies(N) = \{\sNsplit \mid \textnormal{$\exists s\in \calS^0, \exists$ path $p \in s$ such that $N\in p$, and  $\sNsplit=\NewNode(N,s,p)$ } \},\]
    where by tree recursion on $T^0$ (which has tree structure: \autoref{le:21Jan21.1-tree-of-systems-has-tree-structure}) we define $\NewNode(N,s,p)$ \[= \begin{cases}
        \NewNode(N, \predsys{s}, \predpath{s})	\casesif {$s\neq \rootsys{T^0}$ and $N\in \{ X^s, D^s \}$}\\
        (N, s, \{\sinfo^s, \scontrol^s\}) 		\casesif {$N = U^s$}\\
        (N, s, \sinfo^s)	\casesif {$p = \sobspaths^s(N)$}\\ 
        N \casesif {$s=\rootsys{T^0}$ and $N\in\{X^s,D^s$\}}\\
        (N, s, p)			 	\casesotherwise
        \end{cases}.
    \]
    
    
    \item $T^1$ is the tree $(\calS^1,\spred^1)$, where 
    the system $\sssplit^i\in \calS^1$ is defined as the system $(\ssplit(s^i,\sinfo^{T,s^i}), \ssplit(s,\scontrol^{T,s^i}), \ssplit(s,\sobspaths^{T,s}))$ for $s^i\in \calS^T$, where 
    \(\ssplit(s^i,p)^j = \NewNode(p^j,s^i,p),\)
    and where $p^j$ denotes the $j$'th node of a path $p$. (this indeed gives a path, since there is an edge between $\ssplit(s^i,p)^j=\NewNode(p^j,s^i,p)$ and $\ssplit(s^i,p)^{j+1}=\NewNode(p^{j+1},s^i,p)$ because there is an edge between $p^j$ and $p^{j+1}$ and by definition of $E^1$ using $p^j\neq p^{j+1}$). Moreover $\spred^1$ is the same as $\spred^0$ except that each $s$ in $T^0$ is replaced with its transformed $\sssplit$.~\looseness=-1
\end{itemize}
\end{definition}
















\begin{lemma}[Transformation 1 preserves tree] \label{le:21jan25.6-first-split-preserves-tree}
Let $(\sG^1, T^1)=\splitfirst(\sG^0,T^0)$. If $T^0$ is a tree of systems  on $\sG^0$ with root link $X\to D$, then $T^1$ is a tree of systems on $\sG^1$ with root link $X'\to D'$ with $\hfirst(X')=X$ and $\hfirst(D')=D$.
\end{lemma}


\begin{proof}
First we show that $T^1$ satisfies the three conditions of a tree of systems: 
(1) We will show below that each indexed element of $\calS^1$ is indeed a system.
(2) since $\spred^0 = \spred^1$, and since $T^0$ is a tree of systems, the required condition on $\spred^1$ is satisfied. (3) We show that each system's infolink is an infolink on its predecessor path: The nodes $X^{\sssplit}$ and $D^{\sssplit}$ in $\sG^1$ equal $\NewNode(X^s,s,\sinfo^s)$ and $\NewNode(D^s, s, \scontrol^s)$ for $X^s$ and $D^s$ in $\sG^0$. By definition of $\NewNode(N,s,p)$, this is indeed an infolink on $\predpath{s}$. 

It remains to be shown that $\sssplit$ indeed is a system for each $s \in \calS^0$: 

    \proofstepinf{1} {We show that $\scontrol^{\sssplit}$ is a directed path to a utility node}. $\scontrol^{\sssplit}$ is a path from  $\NewNode(D^{s}, {s}, \scontrol^{s})$ to $\NewNode(U^{s}, {s}, \scontrol^{s})$ and by definition of $E^1$ this is indeed a directed path (since $\scontrol^{s}$ is directed in $T$); 
    
    \proofstepinf{2} {We show that $\sinfo^{\sssplit}$ is an active path to the same utility node}. Firstly, $\sinfo^{\sssplit}$ is a path from $\NewNode(X^{s}, {s}, \sinfo^{s})$ to $\NewNode(U^{s}, {s}, \sinfo^{s})$, and since $\NewNode(U^{s}, {s}, \sinfo^{s}) = \NewNode(U^{s}, {s}, \scontrol^{s})$ by definition of $\NewNode$ for utility nodes, therefore the control and info path indeed end at the same utility node. Secondly, it follows easily from the definition of $E^1$, that a node $\NewNode(N,s,\sinfo^{s})$ blocks the path if and only if $N$ blocks $\sinfo^{s}$, and the latter is active by assumption, so that $\sinfo^{\sssplit}$ is active as well;

    \proofstepinf{3}{Finally, we show that the $\sobspaths^{\sssplit}$ are minimal length paths from collider nodes on $\sinfo^{\sssplit}$ to $D^{\sssplit}$}. 
    Firstly, $\sobspaths^{\sssplit}(\NewNode(C,s,\sinfo^{s}))$ is a path from $\NewNode(C, s,\sobspaths^{s}(C))$ to $\NewNode(D^s,s,\sobspaths^{s}(C))$. By definition of $L$, the former equals $(C,(s,\sinfo^{s}))$ and the latter equals $\NewNode(D^s,\predsys{s},\predpath{s})$ if $s\neq \rootsys{T}$ and $(D^s,(s,\scontrol^s))$ if $s=\rootsys{T}$, which in both cases equals $\NewNode(D^s, s,\scontrol^s)=D^{\sssplit}$, so that this is indeed a valid obspath. To show that it's minimal length, assume by contradiction that there is a shorter path and denote its $j$'th node by $(N^j,(s^j,p^j))$, so that there are links $(N^j,s^j,p^j)\to (N^{j+1},s^{j+1},p^{j+1})$. Then by definition of $E^1$, $\sG^0$ contains an edge $N^j\to N^{j+1}$, and hence this path in $\sG^0$ must also be shorter than $\sobspaths^s(C)$, contradicting the assumption that $s$ is a system.
    
This shows that $T^1$ is a tree of systems. Finally, The root infolink of $T^1$ is $X\to D$ by definition of $\NewNode$, and $\hfirst(X)=X$ and $\hfirst(D)=D$, so it is mapped to the root info link of $T$.
\end{proof}



















\begin{lemma}[Transformation 1 ensures \systemsAndPathsUniqueness]
\label{le:21jan26.1-first-split-ensures-systems-and-paths-uniqueness}
Let $(\sG^0,T^0)$ be any soluble ID graph with complete tree. Then $(\sG^1,T^1)=\splitfirst(\sG^0,T^0)$ is an ID graph with complete tree that satisfies (a) \systemsAndPathsUniqueness.
\end{lemma}

\begin{proof} We first show that the split preserves fullness, then that it ensures \systemsAndPathsUniqueness.

    (i) {(full tree)}. 
    \chris{my sense is this subproof can be simplified, it seems very long for what it's doing. Basically it's a "by definition" kind of proof.}
    We show that if ${T^0}$ is a full tree, then so is $T^1$: 
    Let $(X,s^{*,1},p^{*,1}) \to (D,s^{*,2},p^{*,2})$ be an infolink in $\sG^1$ on the path $\psplit$ in system $\sssplit$. We need to show that there is a system $\sssplit'$ such that $X^{\sssplit'}=(X,s^{*,1},p^{*,1})$ and $D^{\sssplit'}=(D,s^{*,2},p^{*,2})$. 
    
    Note that this link in $\psplit$ implies that there is a corresponding link in the original path $p$. By the definition of $T^1$, the split path $\psplit$ was constructed from the original path $p$ (that has the same path type in system $s$ as $\psplit$ does in $\sssplit$) where if $(X,s^{*,1},p^{*,1})$ is the $i$'th node on $\psplit$, it corresponds to the $i$'th node on $p$ by $(X,s^{*,1},p^{*,1})=\psplit^i=\NewNode(p^i,s,p)$ and similarly $(D,s^{*,2},p^{*,2})=\psplit^{i+1}=\NewNode(p^{i+1},s,p)$. Hence $X=p^i$ and $D=p^{i+1}$, so that $X\to D$ is also an infolink on $p$ in $s$ in ${T^0}$.
    And since by assumption ${T^0}$ is full, there is a system $s'$ with $\spred^0(s')= (s,p)$ such that $X^{s'} \to D^{s'}$ equals $X \to D$. This implies also that the desired system in $T^1$ exists: Since by definition $\spred^1$ is equivalent to $\spred^0$ it implies that $\spred^1(\sssplit')= (\sssplit,\psplit)$, and by construction of $\sssplit$,   $X^{\sssplit'}$ and $D^{\sssplit'}$ equal $\NewNode(X^{s'},s',\sinfo^{s'})=\NewNode(X,s',\sinfo^{s'})$ and $\NewNode(D^{s'},s',\scontrol^{s'})=\NewNode(D,s',\scontrol^{s'})$ respectively, which by definition of $\NewNode(X,s,p)$ implies that they equal $\NewNode(X,s,p)$ and $\NewNode(D,s,p)$ respectively, showing the result. 
    
    {(ii)} {(\systemsAndPathsUniqueness)}.
    \newcommand{\uspindprop}{\mathcal P}
    Any node $\sNsplit$ in the tree either equals one of $X,D$, or is a node of the form $\sNsplit=(N,(s^*,p^*))$. In the former case, let $s^*=\rootsys{T^1}$ and let $p^*=\sinfo^{\sRoot}$ if the node equals $X$ and $p^*=\scontrol^{\sRoot}$ if it equals $D$. We will show that $\sssplit^*$ and $\psplit^*$ are the node's base system and base path respectively, by taking any path $\psplit$ in any system $\sssplit$ such that the node is on $\psplit$, and showing that for the original path $p$ and system $s$, either $p=p^*$ and $s=s^*$ or that one of the exceptions applies.
    
    We will show this by induction on the tree: Assume that it holds for any $\psplit'$ in system $\sssplit'$ that is an ancestor system of $\sssplit$. Note that if $\sNsplit=(N,(s^*,p^*))$ is on path $\psplit$ in system $\sssplit$, then $\sNsplit=\NewNode(N,s,p)$, so consider two cases of the definition of $\NewNode(N,s,p)$ separately:
    
    \proofcaseinf{1} {Assume $s\neq \rootsys{T^0}$ and $N\in \{ X^s, D^s \}$.} Then $\sNsplit = (N, s^*, p^*) = \NewNode(N, s, p) = \NewNode(N, \predsys{s}, \predpath{s})$, and we will use the induction assumption on $\predsys{\sssplit}, \predpath{\sssplit}$: We know that $(N, s^*, p^*)$ lies on $\predpath{\sssplit}$ (since $T^1$ is a tree), and by the induction assumption, either $\predsys{\sssplit}=\sssplit^*$ and $\predpath{\sssplit}= \psplit^*$ (in which case the third or fourth exception applies to $\sssplit$ and $\psplit$, showing the result), or one of the exceptions applies. Since $X^s$ and $D^s$ aren't utility nodes, and can't be colliders on the info path of $\predsys{s}$ (\autosubref{le:20Nov7.1-Basic-properties-of-systems-with-sufficient-recall}{le:20Nov7.1a-no-decisions-in-the-back-section}), and $\spredpath$ always is either an info or control path, only the third exception can apply to $\predsys{\sssplit}$ and $\predpath{\sssplit}$, i.e. $(N, s^*, p^*)$ is the info or decision node of $\predsys{\sssplit}$, where the latter is a child system of $\sssplit^*$ or it is the info node of an unbroken chain of descendant systems of $\sssplit^*$. In both cases, $N$ cannot equal $D^s$, since the decision node of a system ($\predsys{\sssplit}$ in this case) is neither the decision of an info link on its info path nor on its control path and hence cannot equal the decision node of one of its child systems ($\sssplit$ in this case), and hence $(N, s^*, p^*)$ must be the info node of $\sssplit$ and an unbroken chain of predecessor systems between $\sssplit$ and $\sssplit^*$, so that it satisfies the third exception.
    
    
    \proofcaseinf{2} {Assume $s= \rootsys{T^0}$ or $X \notin \{X^s, D^s\}$.} Then if $s=\rootsys{T^0}$ and $N\in \{X^s, D^s\}$, then $\sNsplit = N$, and the result follows easily (where $\psplit$ may be an obs path in which case the final exception applies). So assume otherwise, so that  $\sNsplit = (N, s^*, p^*)= \NewNode(N, s, p)$, where $s^*=s$, and we can easily match each of the cases of $\NewNode(N, s, p)$ to the exceptions, showing the result.~\looseness=-1
\end{proof}







\subsubsection{Transformation 2 (split): ensuring no backdoor info-paths}
In the second transformation, we turn any backdoor-info paths into frontdoor infopaths.



\begin{definition}[Transformation 2]
\label{def:21jan28.1-third-split-to-ensure-new-appropriateness}
Let $\sG^1=(\sV^1,E^1)$ be an ID graph with tree $T^1=(\calS^1,\spred^1)$. Then $\splitsecond(\sG^1,T^1)=(\sG^2, T^2)$, where $\sG^2=(\sV^2,E^2)$ and $\hsecond\colon\sV^2 \to \sV^1$ are defined as follows:~\looseness=-1
\begin{itemize}
    \item Obtain any $\sG^2$ and $\hsecond$ from \autoref{le:21may19.2-CID-hom-from-node-copying-and-deleting}, by adding for each node $N$ a set of copies
    \[\sCopies(N) = \begin{cases} 
        \{(N,``\scopy", s)\} \casesif {$N=X^{s}$ for some backdoor-info system $s$ in ${T^1}$}\\
        \emptyset \casesotherwise
    \end{cases}\]
    
    
    
    \item \sloppy $T^2$ is the tree $(\calS^2,\spred^2)$, where each system $\sssplit\in \calS^2$ is obtained from $s$ by replacing the first link $X^s\gets N$ in $\sinfo^s$ with the links $X^s\to (X^s,``\scopy")\gets N$, and extending $\sobspaths^s$ with $\sobspaths^s((X^s,``\scopy"))$ to be the path consisting of the single link $(X^s,``\scopy", s)\to D^s$. Moreover $\spred^2$ is the same as $\spred^1$ except that each $s$ in $T^1$ is replaced with its transformed $\sssplit$. 
\end{itemize}
\end{definition}


    
    
    







\begin{lemma}[Transformation 2 preserves tree]\label{le:21jan28.3-third-split-preserves-tree}
Let $(\sG^2, T^2)=\splitsecond(\sG^1,T^1)$. If $T^1$ is a tree of systems on $\sG^1$ with root link $X\to D$, then $T^2$ is a tree of systems on $\sG^2$ with root link $X'\to D'$ with $\hsecond(X')=X$ and $\hsecond(D')=D$.
\end{lemma}

\begin{proof}
First we show that $T^2$ satisfies the three conditions of a tree of systems: (1)
We will show below that each indexed element of $\calS^2$ is indeed a system.
(2) since $\spred^1$ is equivalent to $\spred^2$, and since ${T^1}$ is a tree of systems, the required conditions on $\spred^2$ are satisfied.
(3) For any system $s$, $X^{\sssplit}=X^s$ and $D^{\sssplit}=D^s$ (i.e. they are unchanged under the split), and the front-section of $\predsys{\sssplit}$ is identical to that of $\predsys{s}$, and since infolinks can only be in the front section (by \autosubref{le:20Nov7.1-Basic-properties-of-systems-with-sufficient-recall}{le:20Nov7.1a-no-decisions-in-the-back-section}), so that the fact that ${T^1}$ is a tree of systems and hence has $X^s\to D^s$ as an infolink on $\predpath{s}$, this implies that $X^{\sssplit}\to D^{\sssplit}$ is an infolink on $\predpath{\sssplit}$.

It remains to be shown that $\sssplit \in \calS^2$ indeed is a system for each $s \in \calS^1$:
    
    \proofstepinf{1} {$\scontrol^{\sssplit}$ is a directed path to a utility node}. 
    $\scontrol^{\sssplit}$ is identical to $\scontrol^s$.
    
    \proofstepinf{2} {$\sinfo^{\sssplit}$ is an active path to the same utility node}.
    Since $\sinfo^s$ is active given $\Pa(D^s)$ by assumption, and $\sinfo^{\sssplit}$ is identical to $\sinfo^s$ except that if $\sinfo^s$ is backdoor from $X^s$ then the first link $X^s\gets N$ is replaced by $X^s\to (X^s, ``\scopy")\gets N$, hence by definition of $E^2$, a node on $\sinfo^{\sssplit}$ is a parent of $D^{\sssplit}$ in $\sG^2$ iff it is a parent of $D^s$ in $\sG^1$ or if it equals $(X^s, ``\scopy")$. Therefore, $(X^s, ``\scopy")$ doesn't block because it's a collider, and the other nodes don't block $\sinfo^{\sssplit}$ because by assumption they didn't block $\sinfo^s$.
    
    \proofstepinf{3}{Finally, $\sobspaths^{\sssplit}$ are minimal length paths from collider nodes on $\sinfo^{\sssplit}$ to $D^{\sssplit}$}: For $\sobspaths^\sssplit((X^s,``\scopy"))$, it is a single link to $D^s$ and hence trivially minimal-length, so consider the other colliders $C$ which are also on $\sinfo^s$. Firstly, each $\sobspaths^{\sssplit}(C)$ is identical to $\sobspaths^s(C)$. Secondly, the split doesn't introduce shorter-length such paths, since any path from $C$ to $D^{\sssplit}$ via some newly added $(X, ``\scopy")$ would correspond to a path via $X^s$ in $\sG^1$ that is at least as short, using the fact that this ID transformation is homomorphic and hence doesn't introduce extra edges. 

This shows that $T^2$ is a tree of systems. Finally, we show that the root infolinks are equivalent:
The if $s^i$ is the root system of ${T^1}$ then $\sssplit^i$ is the root system of $T^2$, and since the infolink of each system in $T^2$ equals that of the corresponding system in ${T^1}$ by definition, we have $\hsecond(X^{\sssplitroot})=X^{\ssRoot}$ and $\hsecond(D^{\sssplitroot})=D^{\ssRoot}$, showing the result.
\end{proof}



\begin{lemma}[Transformation 2 ensures \newAppropriateness]
\label{le:21jan28.4-third-split-ensures-new-appropriateness}
Let $(\sG^1,T^1)$ be any soluble ID graph with complete tree satisfying property (a) \systemsAndPathsUniqueness. Then $(\sG^2,T^2)=\splitsecond(\sG^1,T^1)$ is an ID graph with complete tree satisfying (a) and also (b) \newAppropriateness.
\end{lemma}

\begin{proof} 
We first show that the split preserves fullness, \systemsAndPathsUniqueness and \noOverlappingQX, then we show that it ensures \newAppropriateness:

    {(i)} {(full tree)}. We show that if ${T^1}$ is a full tree, so is $T^2$: For any link $X\to D$ on a path in a system $\sssplit$, that infolink was also on the same path in system $\sssplit$, since $\sssplit$ is identical to $s$ except for the first link on $\sinfo^{\sssplit}$ but that link is in the back section and hence cannot contain $X$ or $D$ by \autosubref{le:20Nov7.1-Basic-properties-of-systems-with-sufficient-recall}{le:20Nov7.1a-no-decisions-in-the-back-section}. Hence there is a system $s'$ in ${T^1}$ with $X\to D$ as its infolink. Hence since by \autoref{def:21jan28.1-third-split-to-ensure-new-appropriateness} the infolink of $\sssplit'$ is the same as that of $s'$, there is a system in $T^2$ that has $X\to D$ as its infolink, namely $\sssplit'$.~\looseness=-1
    
    {(ii)} {(a-\systemsAndPathsUniqueness)}.
    We show that if ${T^1}$ satisfies \systemsAndPathsUniqueness, then so does $T^2$. We state the argument informally: The nodes in $T^2$ are identical to those in ${T^1}$, except for sometimes a split of $X^s$. In that case, $(X^s, ``\scopy")$ is a new node that only appears in system $\sssplit$. Moreover, any other nodes are precisely in system $\sssplit$ if they were in system $s$, so since the node satisfied the required property in ${T^1}$, it also does so in $T^2$.
    
    
    
    {
    (iii) {(b-no-backdoor-infopaths)} Take any system $\sssplit\in \calS^2$. 
    If $s$ is frontdoor info, then $\sssplit$ is identical, so is also frontdoor info.
    If $s$ is backdoor-info, then $\sssplit$ is modified to be frontdoor-info.
    }
\end{proof} 






\subsubsection{Transformation 3 (pruning): ensuring \noRedundantLinks}


\begin{definition}[Transformation 3]
\label{def:transformation-4}
Let $\sG^2\!=\!(\sV^2,E^2)$ be an ID graph with tree $T^2$.
Then $\splitthird(\sG^2,T^2)=(\sG^3, T^3)$ 
where $\sG^3=(\sV^3,E^3)$ and the identity homomorphism $\hthird$ are obtained from $\sG^2$ using \autoref{le:21may19.2-CID-hom-from-edge-pruning} by removing 
all \autosubref{def:sep5.3-normal-form-graph-with-tree}{def:sep5.3e-no-redundant-links} (\noRedundantLinks)
links are removed (which are all into non-decision nodes), and where
$T^3=T^2$.~\looseness=-1
\end{definition}

\begin{lemma}[Transformation 3 is homomorphic]
\label{le:transformation-4-is-homomorphic}
$\hthird$ from \autoref{def:transformation-4} is an ID homomorphism from $\sG^3$ to $\sG^2$.
\end{lemma}


\begin{lemma}[Transformation 3 preserves tree]\label{le:transformation-4-preserves-tree}
Let $(\sG^3, T^3)=\splitthird(\sG^2,T^2)$. If $T^2$ is a tree of systems on $\sG^2$ with root link $X\to D$, then $T^3$ is a tree of systems on $\sG^3$ with root link $X'\to D'$ with $\hthird(X')=X$ and $\hthird(D')=D$.
\end{lemma}

\begin{proof}
The tree $T^3$ is rooted at $X \to D$ 
such that $\hthird(X) \to \hthird(D)$
because it is unchanged from $T^2$.
$T^3$ is 
a tree of systems because it is unchanged from $T^2$, 
while $\sG^3$ retains every edge in any path of every system of $T^2$
--- only redundant links 
\autosubref{def:sep5.3-normal-form-graph-with-tree}{def:sep5.3e-no-redundant-links} (\noRedundantLinks)
are removed.~\looseness=-1
\end{proof}



\begin{lemma}[Transformation 3 preserves (a,b) and ensures (c)]
\label{le:transformation-4-ensures-no-redundant-links}
Let $(\sG^2,T^2)$ be any soluble ID graph with complete tree satisfying properties (a,b) of normal form trees. Then $(\sG^3,T^3)=\splitthird(\sG^2,T^2)$ is an ID graph with complete tree satisfying (a-c).
\end{lemma}

\begin{proof} 
The tree $T^3$ on ID graph $\sG^3$ satisfies (b) 
because $T^3\!=\!T^2$ and $T^2$ satisfies (b).
It satisfies (a) because $T^3\!=\!T^2$ and $\sG^3$ has the same set of nodes, and a subset of the edges of $\sG^2$. \ryan{Explain this.}
It satisfies (c) by definition.~\looseness=-1
\end{proof}




\subsubsection{Composing the transformations to obtain an ID graph with normal form tree}

We will now perform these three transformations in order 
to obtain a normal form tree.


\lenormalformtransform*


\begin{proof}

Given that the minimal $d$-reduction $\sG^*$ of $\sG$ contains $X \to D$, we can first pick an arbitrary full tree from \autoref{le:m3.1-existence-of-complete-tree-of-systems} to obtain a tree $T^0$ satisfying (a) \systemsAndPathsUniqueness.

Then, let $(\sG', T')\!=(\calG^3,T^3)=\!\splitthird \circ \splitsecond \circ \splitfirst (\sG^0,T^0)$
and let $h=\!\hfirst \circ \hsecond\circ \hthird$
using \autoref{def:21jan25.2-first-split-to-ensure-systems-and-paths-uniqueness},  \autoref{def:21jan28.1-third-split-to-ensure-new-appropriateness}
and \autoref{def:transformation-4}.
We show that these have each of the desired properties.%

Firstly, $T'$ is normal form:  Each transformation results in a tree with one more property of normal form trees, and preserves the properties of the previous transformations (\autoref{le:21jan26.1-first-split-ensures-systems-and-paths-uniqueness},  \autoref{le:21jan28.4-third-split-ensures-new-appropriateness}, \autoref{le:transformation-4-ensures-no-redundant-links}).~\looseness=-1

Secondly, $h$ is a homomorphism from $\sG'$ to $\sG^0$ since ID homomorphism is preserved under composition (\autoref{le:20dec7.1-composition-of-CID-splits}). 

Thirdly, $\sG'$ is soluble since that is preserved under ID homomorphisms (\autoref{20nov25.1-CID-homomorphism-preserves-sufficient-recall-SR}{}).

Fourthly, each transformation outputs a tree $T^i$ where $h^{(i-1) \gets i}$ maps nodes in the root infolink 
to nodes of infolink of $T^{i-1}$
(\autoref{le:21jan25.6-first-split-preserves-tree},  \autoref{le:21jan28.3-third-split-preserves-tree}
\autoref{le:transformation-4-preserves-tree}),
so the composition has $h(X')=X^0$ and $h(D')=D^0$.


Finally, Transformation 1 results in an ID graph with tree where the nodes in the tree that are also in the original ID graph $\sG^0$ are precisely $X$ and $D$. And transformations 2-4 only remove and add nodes that are not in $\sG^0$, so the property also holds for $G'$ and $T'$, showing the result.
\end{proof}


