\section{Value of Information} \label{sec:voi-part-1}
The VoI of a variable
\ryan{Add ``is a widely studied property that [cite cite]''?}
indicates how much the attainable expected utility increases when a variable is observed compared to when it is not:










\begin{restatable}[VoI Criterion]{theorem}{thmvoi} \label{thm:voi}
Let $\calG$ be a soluble ID graph containing an edge $X \to D$ from
chance node $X \in \sX$ to decision $D \in \sD$.
There exists an ID $\calM$ compatible with $\calG$ such that $X$ has strictly positive VoI for $D$ if and only if 
the minimal $d$-reduction contains $X \to D$.
\end{restatable}



This is closely related to the concept of \emph{materiality}; an observation $X \in \Pa(D)$ is called 
material if its VoI is positive.

The graphical criterion for VoI that
we will use iteratively removes information links that cannot contain useful information,
based on a condition called \emph{nonrequisiteness}.
If $X\dsep \sU(D^i)\mid \Fa(D^i)\setminus\{X\}$, then both $X$ and the information link $X\to D^i$ 
are called \emph{nonrequisite}, otherwise, they are \emph{requisite}.
Intuitively, nonrequisite links contain no information about 
influencable utility nodes, so the attainable expected utility is not 
decreased by their removal.
Removing one nonrequisite observation link can make a previously requisite information link nonrequisite, 
so the criterion involves iterative removal of nonrequisite links.
The criterion was first proposed by \citet{nilsson2000evaluating}, who also proved that it is sound. 
Formally, it is captured by what we calll a $d$-reduction:

\begin{definition}[$d$-reduction]
\label{def:d-reduction}
  The ID graph $\calG'$ is a \emph{$d$-reduction} of $\calG$ if $\calG'$ can be obtained from
  $\calG$ via a sequence $\calG=\calG^1,...,\calG^k=\calG'$
  where each $\calG^i,i>1$ differs from its predecessor $\calG^{i-1}$
  by the removal of one nonrequisite information link.
  A $d$-reduction is called \emph{minimal} if it lacks any nonrequisite information links.~\looseness=-1 
\end{definition}

For any ID graph $\calG$, there is only one minimal $d$-reduction \citep{nilsson2000evaluating},
i.e.\ the minimal $d$ reduction is independent of the order in which edges are removed.
We can therefore denote \emph{the \minimaldred} of $\calG$
as $\calG^*$.
Thus, \citet[Theorem 3]{nilsson2000evaluating} states that \emph{if} an ID graph $\calG$ contains $X \to D$
but %
$\calG^*$ does not, then $X$ has zero VoI in
every ID compatible with $\calG$.
Our completeness result replaces this with an \emph{if and only if} statement.
~\looseness=-1












\begin{restatable}[VoI Criterion]{theorem}{thmvoi} \label{thm:voi}
Let $\calG$ be a soluble ID graph containing an edge $X \to D$ from
chance node $X \in \sX$ to decision $D \in \sD$.
There exists an ID $\calM$ compatible with $\calG$ such that $X$ has strictly positive VoI for $D$ if and only if 
the minimal $d$-reduction contains $X \to D$.
\end{restatable}



The VoI criterion is posed in terms of a graph $\calG$ that contains $X \to D$.
To analyse a graph that does not, %
one can simply add the edge $X \to D$ then apply the 
same criterion
as long as the new ID graph is soluble \citep{Shachter2016}.
\ryan{Is this the correct shachter cite?}

The proof will be given in \cref{sec:voi-completeness-main}, with details in \cref{sec:preliminaries-systems-and-trees,sec:model-definition}.
We note that this excludes the case of remembering a past decision $X \in \sD$, 
because 
Nilsson's criterion
is incomplete for this case.
For example, the simple ID graph with the edges $D\to D'\to U$ and $D\to U$, $D$ satisfies the graphical criterion of being requisite for $D'$, but $D'$ has zero VoI because it is possible for the decision $D$ to be deterministically assigned some optimal value. This means that there is no need for $D'$ to observe $D$.~\looseness=-1
