\begin{abstract}
% Effective responses to emergencies and non-emergencies are essential for disaster management and can help reduce potential life-threatening consequences. However, dispatching centers often do not have the resources to handle all calls, regardless of when or where they occur. In this study, we analyzed audio recordings from the Nashville Metropolitan Government Dispatching Center, developed an assistant system to help the dispatchers quickly and accurately classify non-emergency calls and highlight important context information from the ongoing call to help the dispatchers quickly and easily file a report. We also evaluated our system under real-world conditions to assess its effectiveness. From the experimental results, we find our system can accurately give out classification results with an average F-1 score of 92.54\%. With the help of the confidence-aware features in key information identification, our system is tested to successfully highlight the most essential context to file a report with an average consistency score of 0.9329 when compared to the ground truth. From our emulated results, our system helps reduce the conversation by 1.01 turns on average only based on the context information in the first few rounds and also offers call dispatching suggestions with an average accuracy of 94.49\%.

% Emergency and non-emergency services perform as critical gateways between residents and first responders. Effective emergency and non-emergency response proves essential for disaster management. However, despite responders' and governments' best efforts, the sheer volume of 911 emergency and 311 non-emergency calls combined with severe labor shortages have heavily burdened existing infrastructure in the USA. 
Emergency and non-emergency response systems are essential services provided by local governments and critical to
protecting lives, the environment, and property. The effective handling of (non-)emergency calls is critical for public safety and well-being. 
% Emergency and non-emergency services play a vital role as connections between residents and first responders. 
% It is crucial to have efficient responses for both situations to manage disasters effectively. 
By reducing the
burden through non-emergency callers, residents in critical need of assistance through 911 will receive fast
and effective response. 
Collaborating with the Department of Emergency Communications (DEC) in Nashville, we analyzed 11,796 non-emergency call recordings and developed Auto311\footnote{Code and Demo: https://github.com/AICPS-Lab/Auto311}, the first
automated system to handle 311 non-emergency calls, which (1) effectively and dynamically predicts ongoing non-emergency incident types to generate tailored case reports during the call; (2) itemizes essential information from dialogue contexts to complete the generated reports; and (3) strategically structures system-caller dialogues with optimized confidence. We used real-world data to evaluate the system's effectiveness and deployability. The experimental results indicate that the system effectively predicts incident type with an average F-1 score of 92.54\%. Moreover, the system successfully itemizes critical information from relevant contexts to complete reports, evincing a 0.93 average consistency score compared to the ground truth. Additionally, emulations demonstrate that the system effectively decreases conversation turns as the utterance size gets more extensive and categorizes the ongoing call with 94.49\% mean accuracy. 

% Emergency and non-emergency services act as critical gateways between residents and first responders. Effective responses to emergencies and non-emergencies prove essential for disaster management. However, despite the best efforts by responders and governments, the sheer volume of emergency (911) and non-emergency (311) calls, coupled with a severe labor shortage, have put immense strain on the existing infrastructure in the USA. This study analyzes audio recordings from the Nashville Metropolitan Government Dispatch Center. We develop an automated system to (1) accurately and dynamically predict the ongoing incident type(s) of non-emergency calls to generate specific case reports, (2) itemize essential information to complete the generated case report, (3) and strategically organize the dialogue between our system and the caller with optimized system confidence scores. Furthermore, this study evaluates the system using real-world data to assess effectiveness and potential adaptability. The experimental results indicate the system can provide accurate classification with an average F-1 score of 92.54\%. By implementing confidence-aware features in key information identification, the system successfully itemizes essential information from the most vital context to complete the reports with an average consistency score of 0.9329 compared to the ground truth. Additionally, emulated results demonstrate that the system decreases conversation turns by 1.01 on average solely based on early contextual information while offering call dispatch suggestions with 94.49\% average accuracy.

% highlight salient contextual information from ongoing calls to expedite dispatcher report filing. 

% dispatch centers across cities frequently lack sufficient resources to handle all calls irrespective of timing or location. 

% Emergency and non-emergency services act as critical gateways between residents and first responders. Despite the best efforts by responders and governments, the sheer volume of emergency 


% (2) developed a tool to help dispatchers quickly and accurately identify the key information from callers, and (3) evaluated the tool under real-world conditions to assess its effectiveness, adaptability, and robustness.
\end{abstract}