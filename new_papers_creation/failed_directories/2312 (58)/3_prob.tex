\section{Problem Definition}
\label{sec:prob}



We consider an undirected and unweighted graph $G = (V, E)$,
where $V=\{v_1,v_2,v_3,...,v_n\}$ is the set of $n$ vertices/nodes, and
$E = \{e_{ij}=(v_i,v_j)\}$ is the set of $m$ edges. The adjacency matrix of $G$ can be represented as a non-negative symmetric
matrix $A = [A_{ij} ] \in R_+^{n \times n}$ where $A_{ij} = 1$ if there is an edge between vertices $i$ and $j$, and $A_{ij} = 0$. %
The degree of vertex $i$ is defined as $d_i = \sum_j a_{ij}$. 
In addition, we have features (or attributes) associated with each node in the graph, $X^0 \in \mathbb{R}^{n\times r}$ where $r$ is the size of the feature vector on nodes.  %


\paragraph{Graph clustering. }Our objective is to have a disjoint clustering of the nodes in the graph. More specifically, the problem of \textit{graph clustering} is to partition the set of nodes into $k$ clusters or groups $\{V_i \}^k_{i=1}$ ($V_i \cap V_j =\emptyset$  for $i \ne j$), %
such that the nodes within a cluster are more densely connected than nodes belonging to different clusters.  Furthermore, in this work, we aim to incorporate node attributes in addition to the graph topology for graph clustering. %

While there exist several metrics to measure the quality of clustering such as conductance \cite{yang2012defining} and normalized cut-ratio \cite{shi2000normalized}, modularity remains the most popular and widely used metric for graph clustering in the literature \cite{fortunato2016community}. 

\paragraph{Modularity. }The approach of graph clustering based on maximizing the modularity of the graph has been introduced by Newman \cite{newman2006modularity}. As a graph topology-based measure, modularity \cite{newman2006modularity} quantifies the difference between the fraction of the edges that fall within the clusters and the expected fraction assuming the edges have been distributed randomly. Formally, modularity ($Q$) is defined as follows:
\begin{equation}
     Q=\frac{1}{2m}\sum_{ij}(A_{ij}-\frac{d_id_j}{2m})\delta(c_i,c_j)
\end{equation}
where $\delta(c_i ,c_j)$ is the Kronecker delta, i.e., $\delta(c_i,c_j)=1$ if $c_i=c_j$ and $0$ otherwise,
and $c_i$ is the community to which node $i$ is
assigned. 

The value of modularity for unweighted and undirected graphs lies in the range $[-1/2,1]$. %
The $Q$ value close to $0$ implies
that the fraction of edges inside communities is no better than a random distribution, and higher values usually correspond to a stronger cluster structure.
The modularity $Q$ can also be expressed in the matrix form as follows:
\begin{equation}
\label{eq:q_matrix}
    Q= \frac{1}{2m} \sum_{ij} B \odot M = \frac{1}{2m} Tr(B M^T)=\frac{1}{2m} Tr(B M)    
\end{equation}
where $M$ is a $n\times n$ symmetric matrix with $M_{ij} = \delta (c_i, c_j)$ and $B_{ij}= (A_{ij}-\frac{d_id_j}{2m})$ is called the modularity matrix.\\

\textbf{Modularity maximization. } %
Since a larger $Q$ implies a prominent cluster structure, optimizing the modularity is a popular way of finding good clusters. While modularity optimization is known to be NP-Hard \cite{brandes2006maximizing}, there exist techniques such as spectral relaxation and greedy algorithms, which permit efficient solutions \cite{newman2006finding,blondel2008fast}.

\textbf{Our goal.} We achieve graph clustering via modularity maximization. The definition of modularity brings the idea of computing pairwise memberships allowing a natural interpretation without knowing the number of clusters. We aim to take advantage of that and the power of graph representation learning techniques that can exploit both structural and non-structural information from graphs. In the experiments, we show the efficacy of our method on four different objectives that quantify the quality of the clusters: modularity \cite{newman2006modularity}, conductance \cite{yang2012defining}, Normalized mutual information (NMI), and F1 score. %










