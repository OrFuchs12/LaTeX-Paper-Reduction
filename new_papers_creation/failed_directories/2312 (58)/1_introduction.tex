\section{Introduction and Related Work}

Graph clustering is a fundamental problem in network analysis and plays an important role in uncovering structures and relationships between the nodes or entities in a graph. It has numerous applications in several domains such as community detection in social networks \cite{newman2006finding}, identifying functional modules in biological systems \cite{wang2010fast}, image segmentation in computer vision~\cite{felzenszwalb2004efficient}, and recommender systems \cite{moradi2015effective}. The primary goal of graph clustering is to group nodes with similar characteristics or functions while maintaining a clear distinction between different clusters. 



 \paragraph{Node attributes. }While traditional graph clustering methods primarily rely on graph topology such as modularity maximization \cite{newman2006modularity,newman2003mixing}, recent research \cite{wang2017mgae} has recognized the importance of incorporating node attributes in the clustering process, offering a more comprehensive approach for grouping nodes. Node attributes provide additional information associated with each node and provide contextual insights that can improve the accuracy in the clustering process. %

 \paragraph{GNN-based clustering. }Recently, there have been several attempts that use the power of deep learning in the form of Graph neural networks (GNNs) \cite{kipf2016semi, hamilton2017inductive, velickovic2017graph} for graph clustering. GNNs provide a powerful tool in graph-based machine learning that is successful in many diverse prediction tasks \cite{zhang2018link,ying2018hierarchical,ying2018graph},
by incorporating the graph topology node features or attributes. For GNN-based graph clustering, MGAE \cite{wang2017mgae} marginalizes the corrupted node features to learn representations via a graph encoder and applies spectral clustering. Another graph autoencoder based approach has been proposed in \cite{park2019symmetric}. To improve the efficiency of clustering, contrastive learning methods have been used recently as well \cite{liu2023simple,kulatilleke2022scgc,xia2021self}. For more neural methods on deep graph clustering, we refer the readers to this recent survey \cite{yue2022survey}. %










\paragraph{Neural modularity maximization.} One of the initial methods to optimize modularity via deep learning for graph clustering is proposed by \citet{yang2016modularity}. They design a nonlinear reconstruction method based on graph autoencoders, which also incorporate constraints among node pairs. \citet{wu2020deep} propose a method that obtains a spatial proximity matrix by using the adjacency matrix and the opinion leaders in the social network. The spatial eigenvectors of the proximity matrix are applied subsequently to optimize modularity. \citet{mandaglio2018consensus} is another study that incorporates the modularity metric for community detection and graph clustering. \citet{choong2018learning, bhatia2018dfuzzy} try to find communities without predefined community structure. \citet{choong2018learning} propose a generative model for community detection using a variational autoencoder. \citet{bhatia2018dfuzzy} firstly analyze the possible number of communities in the graph, then fine-tune it using modularity. Later on, \citet{sun2021graph} use a graph neural network that optimizes modularity and attributes similarity objectives. Another related work in this domain is the method DMoN by \citet{muller2023graph}. This method designs an architecture to encode cluster assignments and then formulate a modularity-based objective function for optimizing these assignments.

\paragraph{Our contributions. } A vast majority of these methods have the limitation that they require the number of clusters to be given as an input or they do not take full advantage of the associated node attributes along with the additional node level information like partial availability of labels or samples of known pairwise memberships. To overcome these challenges, we propose a framework named Deep Graph Cluster ({\model}) that eliminates the need for a predefined number of clusters and harnesses graph representation learning methods that can leverage node attributes along with other available auxiliary information. Our major contributions are as follows.\\


\begin{itemize}
    \item \textbf{\model:} We develop a novel framework that uses pairwise (soft) memberships between nodes to solve the graph clustering problem via modularity maximization. The complexity of our framework scales linearly with the size of the graph. 

    \item \textbf{Handling Unknown Number of Clusters and Auxiliary Information:} Our proposed methodology can generalize well to cases when the number of clusters is not known beforehand. Our designed loss function is also flexible towards accommodating the additional local or node-level information. These are the major strengths of our approach. 
 
    \item \textbf{Extensive Empirical Evaluation:} We conduct extensive experiments on seven real-world datasets of different sizes on four different objectives that quantify the quality of clusters. Our method shows significant performance gain against state-of-the-art methods in most of the settings.
    
\end{itemize}


