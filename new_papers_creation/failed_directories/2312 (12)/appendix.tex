% %File: formatting-instruction.tex
% \documentclass[letterpaper]{article} % DO NOT CHANGE THIS
% \usepackage{aaai24}  % DO NOT CHANGE THIS
% \usepackage{times}  % DO NOT CHANGE THIS
% \usepackage{helvet}  % DO NOT CHANGE THIS
% \usepackage{courier}  % DO NOT CHANGE THIS
% \usepackage[hyphens]{url}  % DO NOT CHANGE THIS
% \usepackage{graphicx} % DO NOT CHANGE THIS
% \urlstyle{rm} % DO NOT CHANGE THIS
% \def\UrlFont{\rm}  % DO NOT CHANGE THIS
% \usepackage{natbib}  % DO NOT CHANGE THIS AND DO NOT ADD ANY OPTIONS TO IT
% \usepackage{caption} % DO NOT CHANGE THIS AND DO NOT ADD ANY OPTIONS TO IT
% \frenchspacing  % DO NOT CHANGE THIS
% \setlength{\pdfpagewidth}{8.5in}  % DO NOT CHANGE THIS
% \setlength{\pdfpageheight}{11in}  % DO NOT CHANGE THIS
% \usepackage{algorithm}
% % \usepackage{algorithmic}
% \usepackage{multirow}
% \usepackage{makecell}
% \usepackage{pifont}
% \usepackage{bbding}
% \usepackage{amsmath}
% \usepackage{amssymb}
% \usepackage{algpseudocode}
% \usepackage{booktabs}
% \usepackage{amstext}
% \usepackage{bm}
% \usepackage{subfigure}
% \newcommand{\eg}{\textit{e}.\textit{g}.}
% \usepackage{enumitem}
% \usepackage[table]{xcolor}

% \usepackage{url}            % simple URL typesetting
% \usepackage{booktabs}       % professional-quality tables
% \usepackage{mathtools,amssymb}
% \usepackage{amsfonts}       % blackboard math symbols
% \usepackage{nicefrac}       % compact symbols for 1/2, etc.
% \usepackage{microtype}      % microtypography
% \usepackage{pgfplots,pgfplotstable}
% \pgfplotsset{compat=1.14}
% \usepackage{array,colortbl}
% \usepackage{xcolor}
% \usepackage{algorithm,algorithmicx,algpseudocode}
% \usepackage[capitalise]{cleveref}
% \usepackage{caption}
% \usepackage{graphbox}
% \usepackage{placeins}
% % \usepackage{wrapfig}
% \usepackage{subcaption}
% \usepackage{etoolbox}

% \newcommand{\bzero}{\mathbf{0}}
% \newcommand{\bone}{\mathbf{1}}
% \newcommand{\bb}{\mathbf{b}}
% \newcommand{\bu}{\mathbf{u}}
% \newcommand{\bv}{\mathbf{v}}
% \newcommand{\bw}{\mathbf{w}}
% \newcommand{\bx}{\mathbf{x}}
% \newcommand{\by}{\mathbf{y}}
% \newcommand{\bz}{\mathbf{z}}
% \newcommand{\bxh}{\hat{\mathbf{x}}}
% \newcommand{\btheta}{{\boldsymbol{\theta}}}
% \newcommand{\bphi}{{\boldsymbol{\phi}}}
% \newcommand{\bepsilon}{{\boldsymbol{\epsilon}}}
% \newcommand{\bmu}{{\boldsymbol{\mu}}}
% \newcommand{\bnu}{{\boldsymbol{\nu}}}
% \newcommand{\bSigma}{{\boldsymbol{\Sigma}}}
% \newcommand{\vardbtilde}[1]{\tilde{\raisebox{0pt}[0.85\height]{$\tilde{#1}$}}}
% \newcommand{\defeq}{\coloneqq}
% \newcommand{\grad}{\nabla}
% \newcommand{\E}{\mathbb{E}}
% \newcommand{\Var}{\mathrm{Var}}
% \newcommand{\Cov}{\mathrm{Cov}}
% \newcommand{\Ea}[1]{\E\left[#1\right]}
% \newcommand{\Eb}[2]{\E_{#1}\!\left[#2\right]}
% \newcommand{\Vara}[1]{\Var\left[#1\right]}
% \newcommand{\Varb}[2]{\Var_{#1}\left[#2\right]}
% \newcommand{\kl}[2]{D_{\mathrm{KL}}\!\left(#1 ~ \| ~ #2\right)}
% \newcommand{\pdata}{{p_\mathrm{data}}}
% \newcommand{\bA}{\mathbf{A}}
% \newcommand{\bI}{\mathbf{I}}
% \newcommand{\bJ}{\mathbf{J}}
% \newcommand{\bH}{\mathbf{H}}
% \newcommand{\bL}{\mathbf{L}}
% \newcommand{\bM}{\mathbf{M}}
% \newcommand{\bQ}{\mathbf{Q}}
% \newcommand{\bR}{\mathbf{R}}

% \begin{document}
\onecolumn
\section{Appendix}

\subsection{Settings of PatchDiff}
The denoising step $T$ of our PatchDiff is $1000$, and the values of images and positional tensors are normalized into a range of $[-1, 1]$. We use AdamW optimizers with a initial learning rate of 1e-3 and a one-cycle learning rate scheduler. The weight decay strength is set to $0.0001$. The global noise $\epsilon_g$ is sampled from a global Gaussian distribution $mathcal{N}_g$ which has a standard deviation of $\sigma_g=0.02$ and every pixel has the identical noise vector. 

To better demonstrate the training and sampling differences between the PatchDiff and DDPM~\cite{DDPM}, we further present the algorithm details 
of training and sampling in Algorithm 1 and Algorithm 2. In particular, $\bepsilon_\theta$ is a function approximator intended to predict $\epsilon_1$ from $\bx_t$, $\epsilon_g$ is the global noise we introduced, and other variables are the same as in DDPM. Actually the sampling process is completely identical to the DDPM and without affected by the introducing of $\epsilon_g$ in the training stage.
\begin{figure}[!htpb]
\begin{minipage}[t]{0.495\textwidth}
\begin{algorithm}[H]
  \caption{Training} \label{alg:training}
  \small
  \begin{algorithmic}[1]
    \Repeat
      \State $\bx_0 \sim q(\bx_0)$
      \State $t \sim \mathrm{Uniform}(\{1, \dotsc, T\})$
      \State $\bepsilon_{1}\sim\mathcal{N}(\mathbf{0}, \mathbf{I}), \bepsilon_{g}\sim\mathcal{N}_g(\mathbf{0}, \sigma_g\mathbf{I})$
      \State Take gradient descent step on
      \State $\grad_\theta \left\| \bepsilon_1 - \bepsilon_\theta\bigl(\sqrt{\bar{\alpha}_t} \mathbf{x}_0 + \sqrt{1-\bar{\alpha}_t} \bepsilon_1 + \bepsilon_g, t\bigr) \right\|^2$
    \Until{converged}
  \end{algorithmic}
\end{algorithm}
\end{minipage}
\hfill
\begin{minipage}[t]{0.495\textwidth}
\begin{algorithm}[H]
  \caption{Sampling} \label{alg:sampling}
  \small
  \begin{algorithmic}[1]
    \vspace{.04in}
    \State $\bx_T \sim \mathcal{N}(\mathbf{0}, \mathbf{I})$
    \For{$t=T, \dotsc, 1$}
      \State $\bz \sim \mathcal{N}(\mathbf{0}, \mathbf{I})$ if $t > 1$, else $\bz = \mathbf{0}$
      \State $\bx_{t-1} = \frac{1}{\sqrt{\alpha_t}}\left( \bx_t - \frac{1-\alpha_t}{\sqrt{1-\bar\alpha_t}} \bepsilon_\theta(\bx_t, t) \right) + \sigma_t \bz$
    \EndFor
    \State \textbf{return} $\bx_0$
    \vspace{.04in}
  \end{algorithmic}
\end{algorithm}
\end{minipage}
% \vspace{-1em}
\end{figure}
% \vspace{0.2cm}

\subsection{Settings of Patch-based Detectors}
\subsubsection{Data augmentation}
We first present the data augmentation details applied during the training of detectors in MVTec AD and MVTec LOCO, as respectively illustrated in Table~\ref{tab:augmentation_mvtec} and Table~\ref{tab:augmentation_loco}, where $p$ denotes the probability of the images being with color jitter. It is worthy note that a larger color jitter range is applied for each generated set, which is expected to be helpful for learning color-level anomalies without training additional generators and detectors. (In principle, GRAD should expose the color-level structures by directly reduce the receptive field size of PatchDiff to 0 and generate pure noise images, then learn the color-level anomalies by level-1 detector)

\begin{table}[!htbp]
\centering
\renewcommand{\arraystretch}{1.}
\footnotesize
\resizebox{0.9\textwidth}{!}{
\begin{tabular}{lccccc}
\toprule
  &  &  &  & \multicolumn{2}{c}{Color jitter ($p=0.2$)} \\
\cmidrule{5-6}
Category  & Vertical flip & Horizontal flip & Random rotation ($\pm5^\circ$) & Normal data & Generated data\\
\midrule
Bottle  & \checkmark & \checkmark & \checkmark & $0.05$ & $0.5$\\
Cable  & \ding{55} & \ding{55} & \checkmark & $0.05$ & $0.5$\\
Capsule  & \ding{55} & \ding{55} & \checkmark &$0.05$ & $0.5$\\
Carpet  & \checkmark & \checkmark & \checkmark &$0.05$ & $0.5$\\
Grid  & \checkmark & \checkmark & \checkmark & $0.05$ & $0.5$\\
Hazelnut  & \checkmark & \checkmark & \checkmark & $0.05$ & $0.5$\\
Leather  & \checkmark & \checkmark & \checkmark & $0.05$ & $0.5$\\
Metal Nut  & \ding{55} & \ding{55} & \checkmark & $0.05$ & $0.5$\\
Pill  & \ding{55} & \ding{55} & \checkmark & $0.05$ & $0.5$\\
Screw  & \checkmark & \checkmark & \checkmark & $0.05$ & $0.5$\\
Tile  & \checkmark & \checkmark & \checkmark & $0.05$ & $0.5$\\
Toothbrush  & \ding{55} & \checkmark & \checkmark & $0.05$ & $0.5$\\
Transistor  & \ding{55} & \checkmark & \checkmark & $0.05$ & $0.5$\\
Wood  & \checkmark & \checkmark & \checkmark & $0.05$ & $0.5$\\
Zipper  & \checkmark & \checkmark & \checkmark & $0.05$ & $0.5$\\
\bottomrule
\end{tabular}}
\caption{Overview of the dataset augmentation techniques applied during training to each of the sub-dataset present in MVTec AD, similar to the setting as relative works~\cite{MVloco}.}
\label{tab:augmentation_mvtec}
\end{table}

\begin{table}[!htbp]
\centering
\footnotesize
\resizebox{1\textwidth}{!}
{
\begin{tabular}{lccccc}
\toprule
  &  &  &  & \multicolumn{2}{c}{Color jitter ($p=0.2$)} \\
\cmidrule{5-6}
Category  & Vertical flip & Horizontal flip & Random rotation ($\pm5^\circ$) & Normal data & Generated data\\
\midrule
Breakfast Box  & \ding{55} & \ding{55} & \checkmark & $0.05$ & $0.5$\\
Screw Bag & \checkmark & \checkmark & \checkmark & $0.05$  & $0.5$\\
Pushpins & \checkmark & \checkmark & \checkmark & $0.05$  & $0.5$\\
Splicing Connectors & \checkmark & \checkmark & \checkmark & $0.05$  & $0.5$\\
Juice Bottle & \ding{55} & \ding{55} & \checkmark & $0.05$  & $0.5$\\
\bottomrule
\end{tabular}}
\caption{Overview of the dataset augmentation techniques applied during training to each of the sub-dataset present in MVTec LOCO, similar to the setting as relative works~\cite{MVloco}.}
\label{tab:augmentation_loco}
\end{table}

\begin{table}[!htpb]
\centering
\footnotesize
\begin{tabular}{lcccc}
\toprule
Dataset     &Detector level & \makecell{Practical size of\\receptive field} & input size & PatchDiff level\\
\midrule
MVTec AD                       & 34    & $34\times34$  & $256\times256$   & 5, 9, 13   \\
\midrule
\multirow{3}{*}{MVTec LoCo AD} & 34    & $34\times34$  & $256\times256$   & 5, 9, 13   \\
                               & 68    & $34\times34$  & $128\times128$   & 5, 9, 13   \\
                               & 136   & $34\times34$  & $64\times64$     & 9, 13, 17 \\
\bottomrule    
\end{tabular}
\caption{The level configures list for patch-level detectors.}
\label{tab: GRad_level_configs}
% \vspace{-0.2cm}
\end{table}

\subsubsection{Training Detail} We then present the configuration details of the patch-level detectors across various levels, as outlined in Table \ref{tab: GRad_level_configs}. In the case of the MVTec AD dataset, we exclusively train a level-$34$ patch-level detector for each sub-dataset. In addition, for MVTec LOCO, we developed three detectors — each corresponding to level-$34$, $68$, and $136$ within their respective sub-datasets. Concerning these levels, the images are resized to dimensions of $256\times256$, $128\times128$, and $64\times64$ respectively. This resizing strategy allows us to maintain the practical receptive field size of each detector at $34\times34$, while the effective receptive field relatively expands to $34\times34$, $68\times68$, and $136\times136$ for the level-$34$, $68$, and $136$ detectors respectively. Moreover, the level-$34$ and $68$ patch-level detectors employ contrastive images generated by level-$5$, $9$, and $13$ PatchDiffs, whereas the level-$136$ patch-level detector employs contrastive images generated by level-$9$, $13$, and $17$ PatchDiffs. This meticulous selection of contrastive images from varying levels PatchDiff further contributes to the detectors' adeptness in capturing diverse local anomaly patterns. Moreover, for our reweighting mechanism, we introduce a memory bank size of 512 for storing the normal features during the training phase. We train the detector for 2000 epochs using AdamW with a one-cycle learning rate scheduler and an initial learning rate of 1e-3.

\begin{table}[!htpb]
\centering
\footnotesize
\begin{tabular}{cccccc}
\toprule
Layer Name & Stride & Kernel Size & Number of Kernels & Padding & Activation \\
\midrule
Conv-1 & 2$\times$2 & 4$\times$4 & 64 & 0 & ReLU \\
Conv-2 & 2$\times$2 & 4$\times$4 & 128 & 0 & ReLU \\
Conv-3 & 1$\times$1 & 3$\times$3 & 256 & 0 & ReLU \\
Conv-4 & 1$\times$1 & 3$\times$3 & 512 & 0 & ReLU \\
Conv-5 & 1$\times$1 & 3$\times$3 & 256 & 0 & ReLU \\
Conv-6 & 1$\times$1 & 1$\times$1 & 256 & 0 & ReLU \\
Conv-7 & 1$\times$1 & 1$\times$1 & 256 & 0 & ReLU \\
Conv-8 & 1$\times$1 & 1$\times$1 & 1 & 0 & - \\
\bottomrule
\end{tabular}
\caption{Network architecture of our patch-level detector.}
\label{tab: arch_detector}
\end{table}

\begin{table}[!htpb]
\centering
\footnotesize
% \resizebox{0.75\textwidth}{!}{
\begin{tabular}{cccccc}
\toprule
Layer Name & Stride & Kernel Size & Number of Kernels & Padding & Activation \\
\midrule
Conv-1 & 1$\times$1 & 1$\times$1 & 256 & 0 & ReLU \\
Conv-2 & 1$\times$1 & 1$\times$1 & 256 & 0 & ReLU \\
Conv-3 & 1$\times$1 & 1$\times$1 & 256 & 0 & ReLU \\
Conv-4 & 1$\times$1 & 1$\times$1 & 5780 & 0 & - \\
\bottomrule
\end{tabular}
\caption{Network architecture of MLP-based decoder for our regularization on features.}
\label{tab: arch_decoder}
\end{table}

\subsubsection{Network Architecture} Additionally, we illustrate the network architecture of our patch-level detector in Table~\ref{tab: arch_detector}, which outputs $1\times1$ anomaly score for an input patch size of $5\times34\times34$ pixels. Consequently, each individual patch-level detector encompasses around 2.9 million parameters, equipped with only 8 fully convolutional layers. Even though we integrate the comprehensive performance of three detectors for MVTec LOCO, the whole number of parameters is still only 8.7 million parameters, highlighting its lightweight structure when compared to prevailing backbone architectures as shown in first figure of our paper. Moreover, in Table~\ref{tab: arch_decoder}, we further illustrate the network architecture of MLP-based network for our regularization on features. We resize the output size $5780$ into $5\times34\times34$ pixels to achieve the reconstruction for the features encoded by our detectors. 

\subsection{Additional Experiment Results}
\subsection{Anomaly Generation}
Furthermore, as shown in Fig.~\ref{fig: mvtec_generation}, we present the samples of contrastive images generated by level-13 PatchDiff for MVTec AD.

\begin{figure*}[!h]
    \centering
\includegraphics[width=0.7\linewidth]{images/mvtec_generation_results.pdf}
    \caption{Contrastive images generated by level-13 PatchDiff for MVTec AD~\cite{MVTecAD}. } 
    \label{fig: mvtec_generation}
\end{figure*}

\begin{table*}[!h]
    \centering
    % \footnotesize
    % \setlength{\belowcaptionskip}{0.2cm}
    % \setlength{\abovecaptionskip}{0.0cm}
    \renewcommand{\arraystretch}{1.2}
    \resizebox{\textwidth}{!}
    {
\begin{tabular}{cl|c|c|c|c|c|c|c|c}
\toprule
% \multicolumn{2}{c|}{Category} & \makecell[c]{IGD\\\tiny{\citealp{IGD}}} & \makecell[c]{PSVDD\\\tiny{\citealp{PSVDD}}} & \makecell[c]{FCDD\\\tiny{\citealp{FCDD}}} & \makecell[c]{CutPaste\\\tiny{\citealp{CutPaste}}} & \makecell[c]{NSA\\\tiny{\citealp{NSA}}} & \makecell[c]{DRAEM\\\tiny{\citealp{DRAEM}}} & \makecell[c]{DSR\\\tiny{\citealp{DSR}}} & \makecell[c]{GRAD\\\tiny{Ours}} \\ \midrule
\multicolumn{2}{c|}{Category}     & IGD & PSVDD & FCDD & CutPaste &NSA & DRAEM & DSR & GRAD \\ \midrule
\multirow{5}{*}{Texture} 
& carpet & (94.7, 82.8 ) & (92.9, 92.6) & (96.0, - ) & (93.1, 98.3) & (95.5, 95.6) & (95.5,97.0) & (-, \textbf{100.}) & (\textbf{96.5}, 98.2) \\
& grid & (97.7, 97.8 ) & (94.6, 100.) & (91.0, - ) & (\textbf{99.9}, 97.5) & (99.2, 99.9) & (99.7, 99.9) & (-, \textbf{100.}) & (97.2, \textbf{100.}) \\
& leather & (99.5, 95.8) & (90.9, 98.6) & (98.0, - ) & (\textbf{100.}, 99.5) & (99.5, 99.9) & (98.6, \textbf{100.}) & (-, \textbf{100.}) & (98.8, \textbf{100.}) \\
& tile & (78.0, 99.1) & (97.8, 91.4) & (91.0, - ) & (93.4, 90.5) & (\textbf{99.3}, \textbf{100.}) & (99.2, 99.6) & (-, \textbf{100.}) & (95.4, \textbf{100.}) \\
& wood & (89.1, 94.6) & (96.5, 90.8) & (88.0, - ) & (\textbf{98.6}, 95.5) & (90.7, 97.5) & (96.4, \textbf{99.1}) & (-, 96.3) & (87.2, 98.3) \\
\midrule
\multirow{10}{*}{Object} 
& bottle & (92.2, \textbf{100.}) & (98.6, 98.1) & (97.0, - ) & (98.3, 97.6) & (98.3, 97.7) & (\textbf{99.1}, 99.2) & (-, \textbf{100.}) & (96.5, \textbf{100.}) \\
& cable & (84.7, 90.6) & (90.3, 96.8) & (90.0, - ) & (80.6, 90.0) & (96.0, 94.5) & (94.7, 91.8) & (-, 93.8) & (\textbf{98.4}, \textbf{99.3}) \\
& capsule & (\textbf{97.7}, 91.5) & (76.7, 95.8) & (93.0, - ) & (96.2, 97.4) & (97.6, 95.2) & (94.3, \textbf{98.5}) & (-, 98.1) & (97.1, 96.4) \\
& hazelnut & (98.0, 99.7) & (92.0, 97.5) & (95.0, - ) & (97.3, 97.3) & (97.6, 94.7) & (\textbf{99.7}, \textbf{100.}) & (-, 95.6) & (96.6, 98.1) \\
& metal nut & (92.6, 91.3) & (94.0, 98.0) & (94.0, - ) & (99.3, 93.1) & (98.4, 98.7) & (\textbf{99.5}, 98.7) & (-, 98.5) & (93.7, \textbf{100.}) \\
& pill & (97.3, 87.3) & (86.1, 95.1) & (81.0, - ) & (92.4, 95.7) & (\textbf{98.5}, \textbf{99.2}) & (97.6, 98.9) & (-, 97.5) & (98.1, 95.7) \\
& screw & (97.0, 82.5) & (81.3, 95.7) & (86.0, - ) & (86.3, \textbf{96.7}) & (96.5, 90.2) & (97.6, 93.9) & (-, 96.2) & (\textbf{99.2}, 96.0) \\
& toothbrush & (97.7, 99.7) & (\textbf{100.}, 98.1) & (94.0, - ) & (98.3, 98.1) & (94.9, \textbf{100.}) & (98.1, \textbf{100.}) & (-, 99.7) & (98.0, 99.7) \\
& transistor & (84.4, 90.6) & (91.5, 97.0) & (88.0, - ) & (95.5, 93.0) & (88.0, 95.1) & (90.9, 93.1) & (-, 97.8) & (\textbf{97.8}, \textbf{100.}) \\
& zipper & (96.7, 97.0) & (97.9, 95.1) & (92.0, - ) & (\textbf{99.4}, 99.3) & (94.2, 99.8) & (98.9, \textbf{100.}) & (-, \textbf{100.}) & (98.3, 99.7) \\
\midrule
\multicolumn{2}{c|}{Average} & (93.1, 93.4) & (92.5, 93.2 ) & (92.1, 95.7) & (95.2, 96.0) & (96.3, 97.2) & (\textbf{97.3}, 98.0) & (-, 98.2) & (96.8, \textbf{98.7}) \\
\bottomrule
\end{tabular}}
\caption{Anomaly detection performance on MVTec AD dataset~\cite{MVTecAD}. Both pixel-level (left) and image-level (right) AUROC results are shown in each column. The best results are in bold.}
\label{tab: mvtec_main_detail}
\end{table*}

\subsection{Anomaly Detection and Localization}
In the main body, we exclusively present the averaged performance comparison on MVTec AD. In this section, we extend our analysis to provide a detailed result of the anomaly detection and localization performance across each individual sub-dataset within MVTec AD, and display anomaly maps on MVTec AD in Fig.~\ref{fig: main_mvtec_ad_results}. As shown in Table~\ref{tab: mvtec_main_detail}, we compare GRAD to IGD~\cite{IGD}, PSVDD~\cite{PSVDD}, FCDD~\cite{FCDD}, CutPaste~\cite{CutPaste}, NSA~\cite{NSA}, DRAEM~\cite{DRAEM}, and DSR~\cite{DSR}, all of which are independent of pretrained feature extractors. It is easy to find GRAD achieves a strong detection and localization of anomalies.  

\begin{figure*}[!h]
    \centering
\includegraphics[width=0.9\linewidth]{images/mvtec_results.pdf}
    \caption{Defect localization results of GRAD on MVTec AD~\cite{MVTecAD}. } 
    \label{fig: main_mvtec_ad_results}
    % \vspace{-0.2cm}
\end{figure*}


% In addition, as shown in Table~\ref{tab:ablation_GRad_level}, we conduct an ablations study on selecting levels of patch-level detectors. It is easy to find that when integrating all three different levels of detectors, GRAD can achieve a strong performance for the detection of structural as well as logical anomalies.

% \begin{table}[!htbp]
% \centering
% \footnotesize
% \resizebox{0.3\textwidth}{!}{
% \begin{tabular}{ccc|c}
% \toprule
% \multicolumn{3}{c|}{Level Settings}& Image-level \\
% 136 & 68 & 34  & AUROC\\
% \midrule
% \checkmark &   \ding{55} & \ding{55}   &  85.2   \\
% \ding{55}& \checkmark & \ding{55} & 85.4 \\
% \ding{55}&\ding{55} & \checkmark & 75.1\\
% \checkmark & \checkmark & \ding{55}   &  86.8     \\ %
% \checkmark &   \ding{55} & \checkmark   &  86.4   \\
% \ding{55} & \checkmark & \checkmark  & 86.2 \\
% \checkmark & \checkmark & \checkmark &  \textbf{87.5}  \\
% \bottomrule
% \end{tabular}}
% \caption{Ablation study on detector levels. Detection AUROC results on MVTec LOCO dataset. The best results are in bold.}
% \label{tab:ablation_GRad_level}
% \end{table}



% \bibliography{aaai24}
% \bibliographystyle{aaai24}

% \end{document}

