\begin{abstract}
Wearable-based Human Activity Recognition (HAR) is a key task in human-centric machine learning due to its fundamental understanding of human behaviours. Due to the dynamic nature of human behaviours, continual learning promises HAR systems that are tailored to users' needs. However, because of the difficulty in collecting labelled data with wearable sensors, existing approaches that focus on supervised continual learning have limited applicability, while unsupervised continual learning methods only handle representation learning while delaying classifier training to a later stage. This work explores the adoption and adaptation of CaSSLe, a continual self-supervised learning model, and Kaizen, a semi-supervised continual learning model that balances representation learning and down-stream classification, for the task of wearable-based HAR. These schemes re-purpose contrastive learning for knowledge retention and, Kaizen combines that with self-training in a unified scheme that can leverage unlabelled and labelled data for continual learning. In addition to comparing state-of-the-art self-supervised continual learning schemes, we further investigated the importance of different loss terms and explored the trade-off between knowledge retention and learning from new tasks. In particular, our extensive evaluation demonstrated that the use of a weighting factor that reflects the ratio between learned and new classes achieves the best overall trade-off in continual learning.
\end{abstract}