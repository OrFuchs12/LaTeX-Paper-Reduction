\section{Conclusion}
In this work, we adapted two state-of-the-art CSSL frameworks, CaSSLe and Kaizen, from visual representation learning to human activity recognition, one of the fundamental tasks in human-centric computing. Our evaluation indicates that a unified training scheme handling both representation learning and classification learning, as proposed in Kaizen, can perform better under realistic data assumptions, with the advantage of being deployable at any point during the process, which is particularly vital for HAR and other human-centric applications. Additional experiments indicate that the use of a progressive importance coefficient which adaptively adjusts the importance of knowledge retention and classification learning can allow us to explore the trade-off between different learning objectives, reaching higher levels of performance compared to a fixed loss function. This work demonstrated the potential of utilising self-supervised learning for developing human activity recognition models that can adapt to changes in user behaviours.
