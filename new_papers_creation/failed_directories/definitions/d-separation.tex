










\begin{definition}[d-separation; \citet{Verma1988soundness}]%
\label{def:d-separation}
    A path $p$ is \emph{blocked} by a set of nodes $\sZ$ if $p$ contains a collider $X \to W \gets Y$, such that neither $W$ nor any of its descendants are in $\sZ$, 
    or $p$ contains a chain $X \to W \to Y$ or fork $X \gets W \to Y$ where $W$ is in $\sZ$.
    If $p$ is not blocked, then it is \emph{active}.
    For disjoint sets $\sX$, $\sY$, $\sZ$, the set $\sZ$ is said to \emph{d-separate} $\sX$ from $\sY$,
    ${(\sX \dsep \sY \mid \sZ)}$ if $\sZ$ blocks every path
    from a node in $\sX$ to a node in $\sY$. Sets that are not d-separated are
    called \emph{d-connected}.~\looseness=-1
\end{definition}


