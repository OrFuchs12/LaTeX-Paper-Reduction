%File: anonymous-submission-latex-2024.tex
\documentclass[letterpaper]{article} % DO NOT CHANGE THIS
\usepackage[submission]{aaai24}  % DO NOT CHANGE THIS
\usepackage{times}  % DO NOT CHANGE THIS
\usepackage{helvet}  % DO NOT CHANGE THIS
\usepackage{courier}  % DO NOT CHANGE THIS
\usepackage[hyphens]{url}  % DO NOT CHANGE THIS
\usepackage{graphicx} % DO NOT CHANGE THIS
\urlstyle{rm} % DO NOT CHANGE THIS
\def\UrlFont{\rm}  % DO NOT CHANGE THIS
\usepackage{natbib}  % DO NOT CHANGE THIS AND DO NOT ADD ANY OPTIONS TO IT
\usepackage{caption} % DO NOT CHANGE THIS AND DO NOT ADD ANY OPTIONS TO IT
\frenchspacing  % DO NOT CHANGE THIS
\setlength{\pdfpagewidth}{8.5in} % DO NOT CHANGE THIS
\setlength{\pdfpageheight}{11in} % DO NOT CHANGE THIS
%
% These are recommended to typeset algorithms but not required. See the subsubsection on algorithms. Remove them if you don't have algorithms in your paper.
\usepackage{algorithm}
\usepackage{algorithmic}
\usepackage{booktabs} % for professional tables
\usepackage{multirow}
\usepackage[algo2e]{algorithm2e}
\LinesNumbered

%
% These are are recommended to typeset listings but not required. See the subsubsection on listing. Remove this block if you don't have listings in your paper.
\usepackage{newfloat}
\usepackage{listings}
\DeclareCaptionStyle{ruled}{labelfont=normalfont,labelsep=colon,strut=off} % DO NOT CHANGE THIS
\lstset{%
	basicstyle={\footnotesize\ttfamily},% footnotesize acceptable for monospace
	numbers=left,numberstyle=\footnotesize,xleftmargin=2em,% show line numbers, remove this entire line if you don't want the numbers.
	aboveskip=0pt,belowskip=0pt,%
	showstringspaces=false,tabsize=2,breaklines=true}
\floatstyle{ruled}
\newfloat{listing}{tb}{lst}{}
\floatname{listing}{Listing}
%
% Keep the \pdfinfo as shown here. There's no need
% for you to add the /Title and /Author tags.
\pdfinfo{
/TemplateVersion (2024.1)
}

% DISALLOWED PACKAGES
% \usepackage{authblk} -- This package is specifically forbidden
% \usepackage{balance} -- This package is specifically forbidden
% \usepackage{color (if used in text)
% \usepackage{CJK} -- This package is specifically forbidden
% \usepackage{float} -- This package is specifically forbidden
% \usepackage{flushend} -- This package is specifically forbidden
% \usepackage{fontenc} -- This package is specifically forbidden
% \usepackage{fullpage} -- This package is specifically forbidden
% \usepackage{geometry} -- This package is specifically forbidden
% \usepackage{grffile} -- This package is specifically forbidden
% \usepackage{hyperref} -- This package is specifically forbidden
% \usepackage{navigator} -- This package is specifically forbidden
% (or any other package that embeds links such as navigator or hyperref)
% \indentfirst} -- This package is specifically forbidden
% \layout} -- This package is specifically forbidden
% \multicol} -- This package is specifically forbidden
% \nameref} -- This package is specifically forbidden
% \usepackage{savetrees} -- This package is specifically forbidden
% \usepackage{setspace} -- This package is specifically forbidden
% \usepackage{stfloats} -- This package is specifically forbidden
% \usepackage{tabu} -- This package is specifically forbidden
% \usepackage{titlesec} -- This package is specifically forbidden
% \usepackage{tocbibind} -- This package is specifically forbidden
% \usepackage{ulem} -- This package is specifically forbidden
% \usepackage{wrapfig} -- This package is specifically forbidden
% DISALLOWED COMMANDS
% \nocopyright -- Your paper will not be published if you use this command
% \addtolength -- This command may not be used
% \balance -- This command may not be used
% \baselinestretch -- Your paper will not be published if you use this command
% \clearpage -- No page breaks of any kind may be used for the final version of your paper
% \columnsep -- This command may not be used
% \newpage -- No page breaks of any kind may be used for the final version of your paper
% \pagebreak -- No page breaks of any kind may be used for the final version of your paperr
% \pagestyle -- This command may not be used
% \tiny -- This is not an acceptable font size.
% \vspace{- -- No negative value may be used in proximity of a caption, figure, table, section, subsection, subsubsection, or reference
% \vskip{- -- No negative value may be used to alter spacing above or below a caption, figure, table, section, subsection, subsubsection, or reference

\usepackage{amsthm}
\usepackage{amsmath}


\usepackage{paralist}
\usepackage{xspace}
\newcommand{\tuple}[1]{\ensuremath{\left \langle #1 \right \rangle }}
\newcommand{\pre}{\textit{pre}}
\newcommand{\params}{\textit{params}}
\newcommand{\eff}{\textit{eff}}
\newcommand{\name}{\textit{name}}
\newcommand{\type}{\textit{type}}
\newcommand{\cnf}{\textit{CNF}}
\newcommand{\conj}{\textit{Conj}}
\newcommand{\true}{\textit{true}}
\newcommand{\false}{\textit{false}}
\newcommand{\unobserved}{\textit{?}}
\newcommand{\realm}{\ensuremath{M^*}\xspace}
\newcommand{\liftf}{F}
\newcommand{\liftl}{L}
\newcommand{\lifta}{A}
\newcommand{\pisam}{\textit{PI-SAM}\xspace}
\newcommand{\sam}{\textit{SAM}\xspace}
\newcommand{\sgam}{\textit{SGAM}\xspace}
\newcommand{\bindings}{\textit{bindings}}
\newcommand{\iseff}{\textit{IsEff}}
\newcommand{\ispre}{\textit{IsPre}}
\newcommand{\state}{\textit{State}}
\usepackage{xcolor}
\newcommand{\brendan}[1]{{\textcolor{red}{[Brendan: #1]}}}
\newcommand{\hai}[1]{{\textcolor{orange}{[Hai: #1]}}}
% \newcommand{\roni}[1]{{\textcolor{green}{[Roni: #1]}}}
\newcommand{\roni}[1]{ }



\newtheorem{theorem}{Theorem}
\newtheorem{proposition}{Proposition}
\newtheorem{lemma}{Lemma}
\newtheorem{corollary}[theorem]{Corollary}
% \theoremstyle{definition}
% \newtheorem{definition}[theorem]{Definition}
% \newtheorem{assumption}[theorem]{Assumption}
\newtheorem{definition}{Definition}
% \newtheorem{assumption}[theorem]{Assumption}

\newtheorem{remark}[theorem]{Remark}
% \theoremstyle{observation}
\newtheorem{observation}[theorem]{Observation}

\setcounter{secnumdepth}{0} %May be changed to 1 or 2 if section numbers are desired.

\usepackage{paralist}
\usepackage{xspace}



% The file aaai24.sty is the style file for AAAI Press
% proceedings, working notes, and technical reports.
%

% Title

% Your title must be in mixed case, not sentence case.
% That means all verbs (including short verbs like be, is, using,and go),
% nouns, adverbs, adjectives should be capitalized, including both words in hyphenated terms, while
% articles, conjunctions, and prepositions are lower case unless they
% directly follow a colon or long dash
\title{Learning Safe Action Models with Partial Observability\\ Supplementary Material}
\author{
    %Authors
    % All authors must be in the same font size and format.
    Written by AAAI Press Staff\textsuperscript{\rm 1}\thanks{With help from the AAAI Publications Committee.}\\
    AAAI Style Contributions by Pater Patel Schneider,
    Sunil Issar,\\
    J. Scott Penberthy,
    George Ferguson,
    Hans Guesgen,
    Francisco Cruz\equalcontrib,
    Marc Pujol-Gonzalez\equalcontrib
}
\affiliations{
    %Afiliations
    \textsuperscript{\rm 1}Association for the Advancement of Artificial Intelligence\\
    % If you have multiple authors and multiple affiliations
    % use superscripts in text and roman font to identify them.
    % For example,

    % Sunil Issar\textsuperscript{\rm 2},
    % J. Scott Penberthy\textsuperscript{\rm 3},
    % George Ferguson\textsuperscript{\rm 4},
    % Hans Guesgen\textsuperscript{\rm 5}
    % Note that the comma should be placed after the superscript

    1900 Embarcadero Road, Suite 101\\
    Palo Alto, California 94303-3310 USA\\
    % email address must be in roman text type, not monospace or sans serif
    proceedings-questions@aaai.org
%
% See more examples next
}

%Example, Single Author, ->> remove \iffalse,\fi and place them surrounding AAAI title to use it
\iffalse
\title{Learning Safe Action Models with Partial Observability\\ Supplementary Material}
\author {
    Author Name
}
\affiliations{
    Affiliation\\
    Affiliation Line 2\\
    name@example.com
}
\fi

\iffalse
%Example, Multiple Authors, ->> remove \iffalse,\fi and place them surrounding AAAI title to use it
\title{Learning Safe Action Models with Partial Observability\\ Supplementary Material}
\author {
    % Authors
    First Author Name\textsuperscript{\rm 1},
    Second Author Name\textsuperscript{\rm 2},
    Third Author Name\textsuperscript{\rm 1}
}
\affiliations {
    % Affiliations
    \textsuperscript{\rm 1}Affiliation 1\\
    \textsuperscript{\rm 2}Affiliation 2\\
    firstAuthor@affiliation1.com, secondAuthor@affilation2.com, thirdAuthor@affiliation1.com
}
\fi


% REMOVE THIS: bibentry
% This is only needed to show inline citations in the guidelines document. You should not need it and can safely delete it.
\usepackage{bibentry}
% END REMOVE bibentry

\begin{document}

\maketitle

% \begin{abstract}
% A common approach for solving planning problems is to model them in a formal language such as the Planning Domain Definition Language (PDDL), and then use an appropriate PDDL planner. 
% Several algorithms for learning PDDL models from observations have been proposed but plans created with these learned models may not be sound. 
% We propose two algorithms for learning PDDL models that are guaranteed to be safe to use even when given observations that include partially observable states. 
% We analyze these algorithms theoretically, characterizing the sample complexity each algorithm requires to guarantee probabilistic completeness. 
% We also show experimentally that our algorithms are often better than FAMA, a state-of-the-art PDDL learning algorithm. 
% \end{abstract}



\section{PI-SAM Sample Complexity Analysis} 

\begin{algorithm}[t]
\small
\DontPrintSemicolon
\SetKwInOut{Input}{Input}\SetKwInOut{Output}{Output}
\Input{Partially Observed Trajectories $\mathcal{T}$}
\Output{($\pre$, $\eff$) for a safe action model}
\BlankLine
    \ForEach{action $a$}{
        $\eff(a)\gets\emptyset$ \\ 
        $\pre(a)\gets $ all parameter-bound literals \nllabel{line:init_pre} \\
        \ForEach{transition $\tuple{s, a, s'}$}{
            \ForEach{literal $l \in \pre(a)$}{
                \uIf{$\neg l$ is unmasked and $\neg l \in s$}{
                 Remove $l$ from $\pre(a)$
                }
            }
            \ForEach{literal $l\in s' \setminus s$ that is unmasked in $s$ and in $s'$}{
                Add $l$ to $\eff(a)$
            }
            
        }
    }
    Return $\tuple{\pre, \eff}$

\caption{Partial Information SAM Learning Algorithm (PI-SAM)}\label{alg:pisam}
\end{algorithm}

% \begin{observation}[PI-SAM Rules]\label{obs:pi-sam-learning-rules}
% For any action triplet $\tuple{s, a, s'}$ and literal $\ell$% it holds that
% \begin{compactitem}
%     \item Rule 1 [not a precondition]. If $\left(\ell\in s\right)\wedge
%     \left(s[\ell]\neq\unobserved\right)$ 
%     then $\neg \ell$ is not a precondition of $a$. 
%     \item Rule 2 [an effect]. If 
%     $\left(\ell\notin s\right)\wedge
%      \left(\ell\in s'\right)\wedge
%     \left(s[\ell]\neq\unobserved\right)\wedge
%      \left(s'[\ell]\neq\unobserved\right)$ then $\ell$ is an effect of $a$.
%     \item Rule 3 [not an effect]. If 
%     $\left(\ell\notin s'\right)\wedge
%     \left(s'[\ell]\neq\unobserved\right)$ then $\ell$ is not an effect of $a$.

% \end{compactitem}
% \end{observation}

\begin{definition}[Bounded Concealment Assumption]
A masking function satisfies the {\em $\eta$-bounded concealment assumption} in an environment if for every literal that is not a precondition of an action, when that action is taken and the literal is false, then the corresponding fluent is observed in both the pre- and post-states with probability at least $\eta$. 

% The {\em $\eta$-bounded concealment assumption} is the following property of a masking function and environment:
% For every literal that is not a precondition of an action, when that action is taken and the literal is false, then the corresponding fluent is observed in both the pre- and post-states with probability at least $\eta$. 
%\roni{Is this sufficient also for learning effects?} 
\end{definition}

\setcounter{theorem}{1}
\begin{theorem}\label{complexity-pisam-thm-appendix}
Under $\eta$-bounded concealment, given 
% \begin{equation*}
%     \small
% m \geq \frac{1}{\epsilon \cdot\eta} (2\ln 3\sum_{\substack{f\in\mathcal{F}\\a\in\mathcal{A}}}\prod_{t\in T}arity(a,t)^{arity(f,t)} + \ln \frac{1}{\delta})
% \end{equation*}
\begin{equation*}
    \small
m \geq \frac{1}{\epsilon \cdot\eta} (2\ln 3 |A|.{|\mathcal{F}|} + \ln \frac{1}{\delta})
\end{equation*}
trajectories sampled from $\mathcal{T}_D$, with probability at least $1-\delta $, 
PI-SAM Learning Algorithm (Algorithm~\ref{alg:pisam}) returns a safe action model $M_\pisam$ such that a problem drawn from $\mathcal{P}_D$ is not solvable with $M_\pisam$ with probability at most $\epsilon$.
\end{theorem}

To prove the theorem, we use the following definition of an {\em adequate} action model:
\begin{definition}[Adequate]
An action model $M$ is {\em $\epsilon$-adequate} if, with probability at most $\epsilon$, a trajectory $T$ sampled from $\mathcal{T}_D$ contains an action triplet $\tuple{s,a,s'}$ where either
\begin{compactenum}
\item $s$ does not satisfy $\pre_M(a)$ or
\item there is a literal in $s'\setminus s$ but not in $\eff_M(a)$.
\end{compactenum}
\end{definition}

\begin{proof}[Proof of Theorem~\ref{complexity-pisam-thm-appendix}]
We first argue that PI-SAM returns an $\epsilon$-adequate action model with probability $1-\delta$: indeed, consider any action model $\tilde{M}$ that is {\em not} $\epsilon$-adequate: then either
\begin{compactenum}
\item with probability at least $\epsilon$, trajectories sampled from $\mathcal{T}_D$ contain a triplet $\tuple{s,a,s'}$ for which $s$ does not satisfy $\pre_{\tilde{M}}(a)$, or
\item with probability at least $\epsilon$, trajectories sampled from $\mathcal{T}_D$ contain a triplet $\tuple{s,a,s'}$ for which there is a literal in $s'\setminus s$ but not in $\eff_{\tilde{M}}(a)$.
\end{compactenum}
In the first case, note that since $\tuple{s,a,s'}$ is a valid transition under the true action model $M^*$, the literal for which $\pre_{\tilde{M}}(a)$ is violated cannot be in $\pre_{M^*}(a)$. Therefore, by $\eta$-bounded concealment, the violated precondition literal in $\pre_{\tilde{M}}(a)$ is observed with probability at least $\eta$ when such a transition occurs; thus, with probability at least $\eta\cdot\epsilon$ overall, the literal is observed and deleted from $\pre_{M_{\pisam}}(a)$. Since PI-SAM never adds precondition literals back, this ensures that $M_{\pisam}\neq \tilde{M}$.

Similarly, in the second case, if $l\in s'\setminus s$, $l\in\eff_{M^*}(a)$. Thus, $\eta$-bounded concealment ensures that $l$ is observed in both $s$ and $s'$ with probability at least $\eta$ when such a transition occurs. So, overall with probability $\eta\cdot\epsilon$, the trajectory contains a triple $\tuple{s,a,s'}$ where $l$ is observed and $l\in s'\setminus s$. When this happens, $l$ is added to $\eff_{\pisam}(a)$, and we again get $M_\pisam\neq \tilde{M}$ since PI-SAM never removes literals from the effects.

Thus, in either case, the probability of obtaining a trajectory that ensures that $\tilde{M}$ is not output is at least $\eta\cdot\epsilon$ on each example. Since the examples are drawn independently, the probability that we do not obtain such an example after $m$ draws is at most $(1-\eta\cdot\epsilon)^m\leq e^{-\eta\cdot\epsilon\cdot m}$. For $m$ as stated in the claim, this is at most
\begin{align*}
&e^{-2\ln 3 |A|.{|\mathcal{F}|} - \ln \frac{1}{\delta}}\\
&= \frac{\delta}{3^{2|A|.{|\mathcal{F}|}}}.
\end{align*}
Note that there are only $3^{2 |A|.{|\mathcal{F}|}}$ possible consistent sets of fluents for the action $a$ (for each fluent, each precondition or effect will either contain that fluent, or its negation, or neither of them), and hence $3^{2|A|.{|\mathcal{F}|}}$ possible action models, given by effects and preconditions for each action. There are, in particular, at most this many action models that are not $\epsilon$-adequate. So, by a union bound over all such action models, the probability that PI-SAM returns any of them is at most $\delta$.

We note that PI-SAM only deletes a literal from $\pre(a)$ when a triplet $\tuple{s,a,s'}$ is observed where $l$ is false in $s$, and hence cannot be a precondition of $a$ in $M^*$. Thus, whenever action $a$ can be taken in some state under $M_\pisam$, it can also be taken in $M^*$. Conversely, when $M_\pisam$ is $\epsilon$-adequate, we have that with probability at least $1-\epsilon$ the sequence of actions appearing in the trajectory associated with a draw from $\mathcal{T}_D$ is a valid plan in $M_\pisam$: the first condition ensures that the preconditions of $M_\pisam$ allow the given action to be executed, and the second condition guarantees that $M_\pisam$ obtains the same states on each transition. Thus, with probability $1-\epsilon$, the goal is achievable under $M_\pisam$ using the plan.
\end{proof}

% \begin{algorithm}[t]
% \small
% \DontPrintSemicolon
% \SetKwInOut{Input}{Input}\SetKwInOut{Output}{Output}
% \Input{Partially Observed Trajectories $\mathcal{T}$}
% \Output{CNF representation of effects $CNF_{\eff}(l)$ for each literal $l$ and precondition $\pre(a)$ for each action $a$}
% \BlankLine
%     \ForEach{ literal $l$}{
%         $CNF_{\eff}(l) \gets \emptyset$\\
%         % \ForEach{trajectory $\{s_0, a_1, s_1, ..., a_k, s_k\}\in\mathcal{T}$ where $s_0(l) = false$ and $s_k(l) = true$}{
%         \ForEach{sequence $\{s_0, a_1, s_1, ..., a_k, s_k\}\subseteq \mathcal{T}_i\in\mathcal{T}$ 
%         where $s_k(l) = true$}{
%             \uIf{$s_0(l) = false $}{
%                 $CNF_{\eff}(l) \gets CNF_{\eff}(l) \wedge (\iseff(l, a_1)  \vee   \iseff(l, a_2) \vee .. \vee \iseff(l, a_k))$
%             }
        
%             \ForEach{$j=1$ to $k-1$}{
%                 $CNF_{\eff}(l) \gets CNF_{\eff}(l) \wedge (\neg\iseff(\neg l, a_j) \vee \iseff(l, a_{j+1}) \vee\ldots\vee \iseff(l, a_k)) $
%             }
%         }
%         \ForEach{action $a$}{
%             $CNF_{eff}(l) \gets CNF_{eff}(l) \wedge (\neg \iseff(l, a) \vee \neg \iseff(\neg l, a))$
%         }
%     } 
%     \ForEach{ literal $l$, action $a$}{
%         $\pre(a)\gets $ all parameter-bound literals \nllabel{line:init_pre} \\
%         \ForEach{trajectory $\{s_0, a_1, s_1, ..., a_k, s_k\}\in\mathcal{T}$ where $a_k = a$}{
%             $i \gets k$\\
%             $\varphi\gets \emptyset$\\
%             \While{$i>0$}{
%                 \uIf{$l$ is observed in $s_{i-1}$}{
%                     $\varphi \gets \varphi \lor (\neg \iseff(\neg l,a_i) \wedge \neg \iseff(\neg l, a_{i+1}) \wedge\cdots\wedge \neg \iseff(\neg l, a_{k-1}))$\\
%                     break\\
%                 }
%                 \uElseIf{$\neg l$ is observed in $s_{i-1}$}{
%                     break
%                 }
%                 \Else{
%                     $\varphi \gets \varphi \lor (\iseff( l,a_i) \wedge \neg \iseff(\neg l, a_{i+1}) \wedge\cdots\wedge \neg \iseff(\neg l, a_{k-1}))$\\
%                     $i \gets i-1 $\\
%                 }
%             }
%         }
%         \uIf{ $\neg \varphi \wedge CNF_{\eff}(l) $ is not satisfiable}{
%             Remove $\neg l$ from $\pre(a)$
%         }
%     }

% \caption{Extended PI-SAM Learning Algorithm (EPI-SAM)}\label{alg:episam}
% \end{algorithm}



% \begin{lemma}\label{lem:cnf-char}
% At line 10 in Algorithm \ref{alg:episam}, for every action model consistent with the set of partially observed trajectories $\mathcal{T}$, the assignment obtained by setting $\iseff(l,a)$ to true if $l$ is an effect of $a$ for each literal $l$ and action $a$ is a satisfying assignment to $CNF_{\eff}(l)$, and conversely, for any satisfying assignment to $CNF_{\eff}(l)$, the corresponding action model is an action model consistent with the trajectories $\mathcal{T}$.
% \end{lemma}
% \begin{proof}
% We first consider the following CNF encoding of the possible trajectories that could yield the observations appearing in $\mathcal{T}$: let $CNF_{\mathcal{T}}$ be a formula with variables $\iseff(l,a)$ for each literal $l$ and action $a$, $\ispre(l,a)$ for each literal $l$ and action $a$, and $State(l,t,h)$ for each literal $l$, trajectory $h\in\mathcal{T}$, and $t=1,\ldots,k$ where $k$ is the length of $h$. The clauses of $CNF_{\mathcal{T}}$ are obtained from clausal encodings of the STRIPS axioms as follows: For each $l$ and $a_t$ at step $t$ of some trajectory $h\in\mathcal{T}$, include
% \begin{compactenum}
% \item $\neg \ispre(l,a_t)\vee State(l,t-1,h)$ if $l$ is not observed at step $t-1$ or $\neg \ispre(l,a_t)$ if $l$ is observed false.
% \item $\neg \iseff(l,a_t)\vee State(l,t,h)$, if $l$ is not observed at step $t$, or $\neg \iseff(l,a_t)$ if $l$ is observed false at step $t$, and
% \item $\neg State(l,t-1,h)\vee \iseff(\neg l,a_t)\vee State(l,t,h)$ if $l$ is not observed at either step $t$ or $t-1$, $\neg State(l,t-1,h)\vee \iseff(\neg l,a_t)$ if $l$ is not observed at $t-1$ and observed false at $t$, and $\iseff(\neg l,a_t)\vee State(l,t,h)$ if $l$ is observed true at $t-1$ and not observed at $t$.
% \end{compactenum}
% We also include the mutual exclusion axioms $\neg \iseff(l,a)\vee \neg \iseff(\neg l,a)$ and $\neg State(l,0,h)\vee \neg State(\neg l,0,h)$ for each unobserved literal $l$.
% Observe that the claim holds for $CNF_{\mathcal{T}}$ if we additionally set the variables $State(l,t,h)$ to true if $l$ would be true in the corresponding fully-observed trajectory at step $t$ (and conversely): indeed, these are encodings of precisely the action models that obey the STRIPS axioms and corresponding trajectories. We also note that the clauses of $CNF_{\mathcal{T}}$ each only use variables corresponding to a single fluent, so the set of satisfying assignments correspond to products of assignments to the formulas $CNF_{\mathcal{T}}(l)$ that only include the clauses using the same fluent as $l$. (Note that the STRIPS axioms ensure that the post-state is determined uniquely given an assignment to the pre-state.)

% We now show that the satisfying assignments to each $CNF_{\eff}(l)$ have the corresponding property for the literal $l$. Recall that resolution is refutation complete, so it suffices to show that for any resolution refutation of a CNF formula $\varphi\wedge CNF_{\mathcal{T}}(l)$, where $\varphi$ (like $CNF_{\eff}(l)$) only contains variables $\iseff(l,a)$, a corresponding refutation of $\varphi\wedge CNF_{\eff}(l)$ exists. 

% We first observe that $\ispre(l,a)$ only appears negatively in all clauses, and hence clauses containing these variables cannot be used in any refutation. Similarly, the initial state variables $State(l,0,h)$ only appear negatively and hence we cannot use the mutual exclusion clauses for the state variables either.

% Second, observe that to obtain a refutation using the clauses of $CNF_{\mathcal{T}}(l)$, the variables $State(l,t,h)$ must be eliminated, but this variable only appears in a negative literal in the third type of clause (these do not appear in $\varphi$ by assumption), created for step $t+1$ of trajectory $h\in\mathcal{T}$. (And in the mutual exclusion axiom for unobserved variables in the initial state.) The variable only appears positively in the clauses created for step $t$ of the second and third type. Observe that we can rewrite the proof (possibly increasing its size) so that these applications of the resolution rule occur first.

% We now claim that the clauses resulting from the final application of the resolution rule on a variable $State(l,t,h)$ appear in $CNF_{\eff}(l)$, which will prove the lemma. Indeed, we can only apply the resolution rule to some run of consecutive clauses of the third type in which the literal $l$ was not observed, ending with a step $t+\Delta$ where $l$ was observed, and beginning with either eliminating $State(l,t-1,h)$ using the clause $\neg \iseff(l,a_{t-1})\vee State(l,t-1,h)$ (created on lines 6--7), or with a clause corresponding to a step $t$ where $l$ was observed in step $t-1$ (created on lines 4--5). 
% \end{proof}

\setcounter{AlgoLine}{0}
\section{EPI-SAM Theoretical Properties with Proofs}


\begin{observation}[EPI-SAM Rules]
For any sub-trajectory $T'$ of a trajectory in $\mathcal{T}$ that ends in a state where literal $l$ is not masked, i.e., where $T'.s_{-1}[l]\neq\unobserved$, then
% $\left(T'[-1].s(l)\neq\unobserved\right)\wedge \left(T'[-1].s(l)=\true\right)$. 
    \begin{compactitem}
        \item Rule 1 [an effect]. 
        If $l\in T'.s_{-1}$ and $l\notin T'.s_0$ 
        then $\exists a\in T'.a$ that has $l$ as an effect. 
        \item Rule 2 [not an effect]. 
        If $l\in T.s_{-1}$ then $\neg l$ is not an effect of $T'.a_{-1}$
        \item Rule 3 [must not delete]. 
        If $l\in T'.s_{-1}$ and $\neg l$ is an effect of some action $T'.a_i$ then $\exists i'>i$ that has $l$ as an effect. 
    \end{compactitem}
\label{obs:epi-sam-learning-rules-appendix}
\end{observation}
\begin{algorithm}[t]
\small
\DontPrintSemicolon
\SetKwInOut{Input}{Input}\SetKwInOut{Output}{Output}
\SetKwBlock{Main}{Main}{end}
\Input{Partially observed trajectories $\mathcal{T}$}
\Output{$CNF_{\eff}(\ell)$ for each literal $l$}
    \ForEach{ literal $\ell$}{
        $\cnf_{\eff}(\ell) \gets \emptyset$\\
        \ForEach{action $a$}{
            Add to $\cnf_{\eff}(\ell)$: $\left\{\neg \iseff(\ell, a) \vee \neg \iseff(\neg \ell, a)\right\}$ \nllabel{epi:not-mutex-appendix}
        }
        % \ForEach{trajectory $T\in\mathcal{T}$ and index $i$ where $l\in T.s_i$}{        
        \ForEach{trajectory $T\in\mathcal{T}$}{        
            \ForEach{index $i\in\{1,\ldots,|T|\}$ where $\ell\in T.s_i$}{        
            % \ForEach{$\tuple{s,a,s'}\in T\in\mathcal{T}$ where $l\in s'$}{
            $T'\gets$ max. prefix of $T.s_i$ where $\ell$ is masked \nllabel{epi:max-prefix-appendix}\\
            % EPI-SAM Rule 1: if l changed from false to true, it must be an effect of one of the actions
            \uIf{$\ell\notin T'.s_0$}{
                Add to $\cnf_{\eff}(\ell)$:  $\left\{\iseff(\ell, T'.a_1)  \vee   \cdots \vee \iseff(\ell, T'.a_{|T'|})\right\}$ \nllabel{epi:refute1-appendix}
            }
            % EPI-SAM Rule 2: \neg l cannot be an effect
            Add to $\cnf_{\eff}(\ell)$:  $\{\neg \iseff(\neg\ell, T'.a_{|T'|})\}$ \nllabel{epi:add-not-an-effect-appendix}\\
            % EPI-SAM Rule 3: Ensure if l is deleted it will be added afterwards
            \ForEach{$j=1$ to $|T'|-1$}{
                Add to $\cnf_{\eff}(\ell)$: 
                $\{\neg\iseff(\neg\ell, T'.a_j) \vee \iseff(\ell, T'.a_{j+1}) \vee 
                \cdots \vee \iseff(\ell, T'.a_{|T'|})\}$  \nllabel{epi:refute2-appendix}
            }
        }
        }
    }
    \Return $\{\cnf_\eff(\ell)\}_\ell$
\caption{EPI-SAM: Learning Effects}
\label{alg:episam-effects-appendix}
\end{algorithm}

\begin{algorithm}[t]
\small
\DontPrintSemicolon
\SetKwInOut{Input}{Input}\SetKwInOut{Output}{Output}
\SetKwBlock{Irrelevant}{Irrelevant}{end}
\SetKwBlock{AssumePrecondition}{AssumePrecondition}{end}
\SetKwBlock{LearnPreconditions}{LearnPreconditions}{end}
\SetKwBlock{Main}{Main}{end}
\Input{Partially observed trajectories $\mathcal{T}$}
\Output{Precondition $\pre(a)$ for each action $a$}
% \Irrelevant{
%     % \Input{action $a$, literal $l$, set $\cnf_\eff(l)$}
%     \Input{action $a$, literal $\ell$, set of trajectories $\mathcal{T}$}
%     \Output{$\true$ ~iff $a$ does not affect the value of $l$}
%     \ForEach{$\tuple{s_1,a,s_1'}, \tuple{s_2,a,s_2'}$ in $T\in\mathcal{T}$}{
%         \If{$s'_1[\ell]\neq\unobserved \wedge s'_2[\ell]\neq\unobserved\wedge s_1'[\ell]\neq s_2'[\ell]$}{
%         % \If{$l$ is unmasked in $s'_1$ and $s'_2$ $\wedge$ $\left(s_1'[l]\neq s_2'[l]\right)$}{
%             \Return $\true$
%         }
%     }
%     \Return $\false$
% }
% \AssumePrecondition{
%     \Input{action $a$, literal $\ell$, set of trajectories $\mathcal{T}$}
%     \Output{a copy of $\mathcal{T}$ where $\ell$ is always true before $a$}
%     $\mathcal{T}_{a,\ell}\gets\emptyset$\\
%     \ForEach{trajectory $T\in\mathcal{T}$}{
%         Initialize $T_{a,\ell}\gets$ a copy of $T$\\
%         \ForEach{$\tuple{s,a,s'}\in T_{a,\ell}$}{
%             $s[\ell]\gets\true$
%         }
%         Add $T_{a,\ell}$ to $\mathcal{T}_{a,\ell}$
%     }
%     \Return $\mathcal{T}_{l,a}$
% }
% \LearnPreconditions{
    \lForEach{action $a$}{$\pre(a)\gets$ all literals}
    \ForEach{action $a$, literal $\ell$}{
        \uIf{$\exists \tuple{s,a,s'}\in T\in\mathcal{T}$ where $\neg \ell\in s$ \nllabel{line:pi-rule-start-appendix}}{
            % SAM Rule 1: if \neg l is in s then it cannot be a precondition of a
            Remove $\ell$ from $\pre(a)$ \nllabel{episam-easy-pre-delete-appendix}\\
            Continue to the next $(a, \ell)$ pair \nllabel{line:pi-rule-end-appendix}
        }
        % Assume l is a precondition of a. 
        % Create a copy of all trajectories under this assumption
        % $\mathcal{T}_{l,a}\gets\emptyset$\\
        % \ForEach{trajectory $\mathcal{T}_i\in\mathcal{T}$}{
        %     Initialize $T_{l,a}\gets \emptyset$\\
        %     \ForEach{$\tuple{s,a,s'}\in \mathcal{T}_i\in\mathcal{T}$}{
        %         Set $s(l)$ to true in $T_{l,a}$
        %     }
        %     Add $T_{l,a}$ to $\mathcal{T}_{l,a}$
        % }
        $\mathcal{T}_{a,\ell}\gets$ \textbf{AssumePrecondition}($a$, $\ell$, $\mathcal{T}$) \nllabel{epi:assume-appendix}\\
        % Remove actions that are known to not have any effect on the value of l
        $A_{irr}\gets\emptyset$ \nllabel{line:epi-propagate-start-appendix}\\
        \While{$\exists a'\notin A_{irr}$ where \textbf{Irrelevant}($a'$,$\ell$,$\mathcal{T}_{a,\ell}$)}{
            \ForEach{$\tuple{s,a',s'}$ in $T\in\mathcal{T}_{a,\ell}$}{
                \eIf{$s[\ell]$ and $s'[\ell]$ are inconsistent \nllabel{epi:inconsistent-appendix}}{
                            Remove $\ell$ from $\pre(a)$ \nllabel{episam-hard-pre-delete-appendix}\\
                            Continue to the next $(a,\ell)$ pair\nllabel{line:propagate-end-appendix}
                }{
                    \lIf{$s[\ell]=\unobserved$}{
                        $s[\ell]\gets s'[\ell]$ \nllabel{epi:propagate-appendix}
                    }
                    Remove $\tuple{s,a',s'}$ from $T$ \nllabel{epi:remove-appendix}
                }
            }
        }
    }
    \Return $\{\pre(a)\}_a$
% }
\caption{EPI-SAM: Learning Preconditions}\label{alg:episam-preconditions-appendix}
\end{algorithm}


% \subsection{Theoretical Properties}

%Here we analyze the safety and runtime of EPI-SAM, showing that it is both safe and tractable. We also show that it is the strongest algorithm for solving safe model-free planning problems, in the sense that any algorithm able to solve a problem that cannot be solved by EPI-SAM cannot also be safe. 

Next, we show that EPI-SAM is safe, runs in polynomial time, and it is the strongest algorithm for solving safe model-free planning problems, in the sense that any algorithm able to solve a problem that cannot be solved by EPI-SAM cannot also be safe. 
Throughout this analysis, we denote by $A^*$ the action model of the underlying problem, and denote by $\pre_A(a)$ and $\eff_A(a)$ the set of preconditions and effects, respectively, of an action $a$ according to an action model $A$. 
Observe that every classical action model $A$ corresponds to an assignment $\sigma_A$ to the formula $\Phi_\eff=\bigwedge_\ell \cnf_\eff(\ell)$, by setting $\iseff(\ell,a)$ to true if $\ell$ is an effect of $a$ for each literal $\ell$ and action $a$. 
Similarly, every satisfying assignment of $\Phi_\eff$ describes the effects of a classical action model. 
\begin{lemma}\label{lem:cnf-char-appendix}
If a classical action model $A$ is consistent with $\mathcal{T}$
then $\sigma_A$ is a satisfying assignment of $\Phi_\eff$. 
Conversely, every satisfying assignment $\sigma$ to $\Phi_\eff$ describes the effects of at least one classical action model that is consistent with $\mathcal{T}$. 
\end{lemma}
\noindent
{\em Sketch of proof.}
% Consider the logical formula created by the STRIPS axioms, instantiated at each step of each trajectory in $\mathcal{T}$. 
% This formula is defined over variables of the form $\iseff(l,a)$, $\ispre(l,a)$, and $\state(l,i,T)$, representing that 
% $l$ is a precondition of $a$, 
% $l$ is an effect of $a$, 
% and $l=\true$ in the $i^{th}$ state of trajectory $T$, respectively.  
% This formula includes the following clauses for every transition $\tuple{s_{i-1},a_i,s_i}$ in every trajectory $T\in\mathcal{T}$:
% \begin{enumerate}
% \item $\ispre(l,a_i)\rightarrow \state(l,i-1,T)$
% \item $\iseff(l,a_i)\rightarrow \state(l,i,T)$
% \item $\neg\iseff(l,a_i)\rightarrow (\state(l,i-1,T)\rightarrow(\state(l,i,T)))$
% \end{enumerate}
% with $\state(l,i,T)$ replaced by true or false when $l$ is observed true or false, respectively, at step $i$ in $T$. 
% The clausal encoding of the above formula to a CNF, denoted $\cnf_{\mathcal{T}}$,
% is as follows: 
% \begin{itemize}
% \item (C1) $\neg \ispre(l,a_i)\vee \state(l,i-1,T)$
% \item (C2) $\neg \iseff(l,a_i)\vee \state(l,i,T)$
% \item (C3) $\iseff(l,a_i)\vee \neg \state(l,i-1,T) \vee \state(l,i,T)$
% \end{itemize}
Consider the clausal encoding of the STRIPS axioms, instantiated at each step of each trajectory in $\mathcal{T}$. 
This CNF, denoted $\cnf_\mathcal{T}$ is defined over variables of the form $\iseff(l,a)$, $\ispre(l,a)$, and $\state(l,i,T)$, representing that 
$l$ is a precondition of $a$, 
$l$ is an effect of $a$, 
and $l=\true$ in the $i^{th}$ state of trajectory $T$, respectively.  
This CNF includes the following clauses for every transition $\tuple{s_{i-1},a_i,s_i}$ in every trajectory $T\in\mathcal{T}$:
\begin{compactitem}
\item (C1) $\neg \ispre(l,a_i)\vee \state(l,i-1,T)$
\item (C2) $\neg \iseff(l,a_i)\vee \state(l,i,T)$
\item (C3) $\iseff(l,a_i)\vee \neg \state(l,i-1,T) \vee \state(l,i,T)$
\end{compactitem}
By construction, a satisfying assignment to $\cnf_{\mathcal{T}}$ corresponds to the effects of an action model and the complete trajectories for this action model, given the values observed in the trajectories of $\mathcal{T}$. Moreover, the action model with these effects and no preconditions is consistent with $\mathcal{T}$. 

Let $\cnf_\mathcal{T}(\ell)$ be the formula containing all the clauses in $\cnf_\mathcal{T}$ containing literals for a single fluent literal $\ell$.
Note that the clauses of $\cnf_\mathcal{T}$ only contain literals for a single fluent literal, so 
$\cnf_\mathcal{T}$ is satisfiable iff for every $\ell$ the formula $\cnf_{\mathcal{T}}(\ell)$ is satisfiable. 
The final part of our proof will show that the CNF returned by EPI-SAM, $\cnf_\eff(\ell)$, is satisfiable iff $\cnf_\mathcal{T}(\ell)$ is satisfiable. 
To this end, we rely on the refutation-completeness of resolution and examine which clauses may appear in a refutation of $\cnf_{\mathcal{T}}(\ell)$. 
The $\ispre(a,\ell)$ literals, appearing only negatively, cannot appear in a refutation. 
Thus, any refutation will be based on clauses of types C2 and C3. 
Two types of proofs can be created from such clauses. 
The first requires observing the value of $\ell$ in enough states such that we have contradicting unit clauses with $\iseff$ literals for some action $a_i$. That is, we have transitions 
$\tuple{s_i,a_i,s'_i}$ and $\tuple{s_j,a_i,s'_j}$
where $l$ is observable in states 
$s'_i$, $s_{j-1}$, and $s_j$ 
with values $\false$, $\true$, and $\false$, respectively.
This option is implemented in line~\ref{episam-easy-pre-delete-appendix} of Algorithm~\ref{alg:episam-preconditions-appendix}. 
The second type of proof requires using resolution to eliminate at least one $\state$ literal. Reordering the applications of the resolution rule on these literals to the beginning of the proof, we see that we must create clauses that correspond to consecutive runs of unobserved literals using the resolution rule on clauses of type C3 for each step, beginning with either an observed literal or with using clauses of type C2 to eliminate the first $\state(\ell, i, T)$ literal. These are, respectively, the clauses of $\cnf_{\eff}(l)$ created on lines~\ref{epi:refute1-appendix} and~\ref{epi:refute2-appendix} in Algorithm~\ref{alg:episam-effects-appendix}.




% \subsection{Roni Version}
% Thus, any refutation must eliminate at least one $\iseff(l,a_i)$ literal. 
% This requires applying resolution between a C2 clause and a C3 clause in a pair of transitions $\tuple{s_{i-1},a_i,s_i}\in T$ and $\tuple{s_{j-1},a_j, s_j}\in T'$ 
% where $a_i=a_j$, resulting in the clause:
% \[ 
% \state(l,i,T)\vee 
% \neg\state(l,j-1,T')\vee\state(l,j,T')
% \]
% There are two ways to proof a contradiction with this clause. 
% The first is by observing that the value of $l$ in the states $s_i$, $s_{j-1}$, and $s_j$ 
% is $\false$, $\true$, and $\false$, respectively.
% This option is implemented in line~\ref{episam-easy-pre-delete} of Algorithm~\ref{alg:episam-preconditions}. 
% The second option corresponds to where  $l$ is masked in at least one of these states, and requires a proof that eliminates at least one $\state$ literal. 
% Every $\state(l,i,T)$ literal appears in exactly two clauses: a C2 and a C3 clause that correspond to consecutive states where the $\ell$ is masked. Therefore, any proof must create clauses that correspond to consecutive runs of state where $\ell$ is masked using the resolution rule on C3 clauses for each step, beginning with either an observed literal or by using a C2 clause to eliminate the first $\state(\ell,i,T)$ literal. 
% These are, respectively, the clauses of $CNF_{\eff}(l)$ created on lines~\ref{epi:refute1} and~\ref{epi:refute2} in Algorithm~\ref{alg:episam-effects}.
% \roni{How is it now?}
% \roni{I am not super happy with this proof. Don't we need a claim about the preconditions?} \brendan{OK, properly the statement should only concern the effects part of the model. But remember that we can always find a consistent set of preconditions by just saying there are no preconditions, so the question about preconditions is not important for the existence of a consistent action model.}\roni{Ok, I think I edited this accordingly. Did I get it right and readable? @Brendan@Hai}



% \subsubsection{Original version of the proof}
% We then use the refutation-completeness of resolution to reduce the problem to identifying the clauses that may appear in resolution refutations. The $\ispre(a,l)$ literals, appearing only negatively, cannot appear in a refutation, and the literals $State(l,i,T)$ must be eliminated, \roni{not necessarily: there are clauses of type C2 and C3 where there is no $\state(l,i,T)$ literals, when the value of $l$ is observed in state $i$ at trajectory $T$} we can reach a contradiction if we observe the values of $\ell$ in states but these only appear in consecutive instances of the second and third type of clauses where the literal $l$ is unobserved; and only the third clause has the negative literal. Reordering the applications of the resolution rule on these literals to the beginning of the proof, \roni{Not obviousy to me why we can do this reordering} we see that we must create clauses that correspond to consecutive runs of unobserved literals using the resolution rule on the third type of clause for each step, beginning with either an observed literal or with using the second type of clause to eliminate the first $State(l,i,T)$ literal. These are, respectively, the clauses of $CNF_{\eff}(l)$ created on lines~\ref{epi:refute1} and~\ref{epi:refute2} in Algorithm~\ref{alg:episam-effects}.


% Any refutation must eliminate $\iseff(l,a_i)$ literals. 
% This requires applying resolution between a C2 clause and a C3 clause in a pair of transitions $\tuple{s_{i-1},a_i,s_i}$ and $\tuple{s_{j-1},a_j, s_j}$ 
% where $a_i=a_j$. 
% If $\ell$ is unmasked in all these states, then the corresponding clause exists in the CNF returned by EPI-SAM. 
% Otherwise, i.e., if $\ell$ is masked in at least one of these states, we must eliminate a $\state(\ell,i,T)$ literal. 
% Such a literal only exists in instances of clauses C2 and C3 that correspond to consecutive states where the $\ell$ is masked. 
% Therefore, any proof must create clauses that correspond to consecutive runs of state where $\ell$ is masked using the resolution rule on C3 clauses for each step, beginning with either an observed literal or by using a C2 clause to eliminate the first $\state(\ell,i,T)$ literal. 
% These are, respectively, the clauses of $CNF_{\eff}(l)$ created on lines~\ref{epi:refute1} and~\ref{epi:refute2} in Algorithm~\ref{alg:episam-effects}
% \roni{I am not super happy with this proof. Don't we need a claim about the preconditions?} \brendan{OK, properly the statement should only concern the effects part of the model. But remember that we can always find a consistent set of preconditions by just saying there are no preconditions, so the question about preconditions is not important for the existence of a consistent action model.}\roni{Ok, I think I edited this accordingly. Did I get it right and readable? @Brendan@Hai}


\begin{lemma}\label{lem:pre-strong-appendix}
For every action $a$ in $A_\sam$ and literal $\ell$, it holds that 
$\ell\in\pre_{A_\sam}(a)$ if and only if there exists an action model $A$ consistent with $\mathcal{T}$ where $\ell\in\pre_A(a)$. 
\end{lemma}
\begin{proof}
We first prove that if EPI-SAM removes a literal $\ell$ from $\pre(a)$, then there exists a transition $\tuple{s,a,s'}$ in $\mathcal{T}$ where $\ell$ is false, and hence cannot be in $\pre_{A*}(a)$. 
EPI-SAM removes $\ell$ from $\pre(a)$ in two places in Algorithm~\ref{alg:episam-preconditions-appendix}: line~\ref{episam-easy-pre-delete-appendix} and line~\ref{episam-hard-pre-delete-appendix}. 
The correctness of line~\ref{episam-easy-pre-delete-appendix} is immediate: if $\ell$ is observed to be false in a state where $a$ has been applied then it cannot be a precondition of $a$ (PI-SAM Rule 1). 
Before removing a precondition due to line~\ref{episam-hard-pre-delete-appendix}, 
EPI-SAM creates a set of trajectories $\mathcal{T}_{\ell,a}$ that assumes $\ell$ was true whenever $a$ was taken, 
and detects the set of actions $A_{irr}$ that cannot affect the value of $\ell$ in any action model consistent with $\mathcal{T}_{\ell,a}$. 
Because of the frame axioms, the value of $\ell$ gets propagated in any transition that includes an action in $A_{irr}$. $\ell$ is only removed in line \ref{episam-hard-pre-delete-appendix} if this propagation results in a state where $\ell$ has contradicting values. As this occurs for any action model consistent with $\mathcal{T}_{\ell,a}$, this implies that $\ell$ cannot be true in every state where $a$ was applied, and thus cannot be a precondition of $a$ in any action model consistent with $\mathcal{T}$. 



Next, we prove that if $\ell$ has not been deleted from $\pre(a)$ by EPI-SAM, then there exists an action model $A$ consistent with $\mathcal{T}$ where $\ell\in\pre_A(a)$. 
% Since $\ell$ has not been deleted from $\pre_{A_\sam}$, then 
% for any action $a\notin A_{irr}$ we have that the value of $\ell$ after $a$ is either masked or takes at most one value (true of false). 
Consider the subset of $A_{irr}$ that includes only actions that have been in a transition where the value of $\ell$ is not masked. 
For each action $a'$ in this set, we are guaranteed that this value of $\ell$ is always the same, denoted $v(a',\ell)$. Otherwise $a'$ would have been added to $A_{irr}$. 
The action model created by assigning $v(a',\ell)$ as an effect of $a'$ for each of these actions is consistent with $\mathcal{T}_{l,a}$. 
Therefore, there exists an action model where $\ell$ is a precondition of $a$ that is consistent with $\mathcal{T}$. 
\end{proof}

% \begin{proof}
% We first prove that if EPI-SAM removes a literal $\ell$ from $\pre(a)$, then there exists a transition $\tuple{s,a,s'}$ in $\mathcal{T}$ where $\ell$ is false, and hence cannot be in $\pre_{A*}(a)$. 
% EPI-SAM removes $\ell$ from $\pre(a)$ in two places in Algorithm~\ref{alg:episam-preconditions}: line~\ref{episam-easy-pre-delete} and line~\ref{episam-hard-pre-delete}. 
% The correctness of line~\ref{episam-easy-pre-delete} is immediate: if $\ell$ is observed to be false in a state where $a$ has been applied then it cannot be a precondition of $a$ (PI-SAM Rule 1). 
% Before removing a precondition due to line~\ref{episam-hard-pre-delete}, 
% EPI-SAM creates a set of trajectories $\mathcal{T}_{\ell,a}$ that assumes $\ell$ was true whenever $a$ was taken, 
% and detects the set of actions $A_{irr}$ that cannot affect the value of $\ell$ in any action model consistent with $\mathcal{T}_{\ell,a}$. 
% Because of the frame axioms, the value of $\ell$ gets propagated in any transition that includes an action in $A_{irr}$. $\ell$ is only removed in line \ref{episam-hard-pre-delete} if this propagation results in a state where $\ell$ has contradicting values. As this occurs for any action model consistent with $\mathcal{T}_{\ell,a}$, this implies that $\ell$ cannot be true in every state where $a$ was applied, and thus cannot be a precondition of $a$ in any action model consistent with $\mathcal{T}$. 



% Next, we prove that if $\ell$ has not been deleted from $\pre(a)$ by EPI-SAM, then there exists an action model $A$ consistent with $\mathcal{T}$ where $\ell\in\pre_A(a)$. 
% % Since $\ell$ has not been deleted from $\pre_{A_\sam}$, then 
% % for any action $a\notin A_{irr}$ we have that the value of $\ell$ after $a$ is either masked or takes at most one value (true of false). 
% Consider the subset of $A_{irr}$ that includes only actions that have been in a transition where the value of $\ell$ is not masked. 
% For each action $a'$ in this set, we are guaranteed that this value of $\ell$ is always the same, denoted $v(a',\ell)$. Otherwise $a'$ would have been added to $A_{irr}$. 
% The action model created by assigning $v(a',\ell)$ as an effect of $a'$ for each of these actions is consistent with $\mathcal{T}_{l,a}$. 
% Therefore, there exists an action model where $\ell$ is a precondition of $a$ that is consistent with $\mathcal{T}$. 
% \end{proof}
% The algorithm continues to the next iteration of the loop and does not consider this literal again, thus it remains in $\pre(a)$ upon termination.
% any action at the end of a run of states where the fluent is unobserved must be followed by states where the fluent takes at most one value. Then we could assign the effect to that action that sets the fluent to that value, and thus obtain a consistent action model with $\mathcal{T}_{l,a}$. Thus, in this case, $l$ could be a precondition of $a$. The algorithm continues to the next iteration of the loop and does not consider this literal again, thus it remains in $\pre(a)$ upon termination.

% Finally, we note that the contingent plan must achieve the goal with any setting of the $\iseff$ variables that is consistent with $CNF_{\eff}$. By Lemma~\ref{lem:cnf-char}, we see that this means that in particular the goal is achieved with the assignment corresponding to the real action model. Thus, the EPI-SAM action model is indeed safe.
%%
%% OLD SAT-BASED ANALYSIS
%%
%%Consider the formula $CNF_{\mathcal{T}}(l)$ created in Lemma~\ref{lem:cnf-char}: $\ispre(l,a)$ can be set to true in a satisfying assignment iff for every step $t$ where $a=a_t$, we have $State(l,t-1,h)$ true and $State(\neg l,t-1,h)$ false (equiv., $\neg State(\neg l,t-1,h)$ true).  (Equivalently, these are the clauses that can be derived using the unit clause $\ispre(l,a)$.) Thus, by the refutation completeness of resolution, we can refute this collection of unit clauses with $CNF_{\mathcal{T}}(l)$ iff $l$ cannot be a precondition of $a$.
%%
%%Again, the $State$ variables must be eliminated, and the (other) $\ispre$ variables cannot be used in a refutation because they only appear negatively. Again, recall that the $State$ variables only appear negatively in clauses of the third type in $CNF_{\mathcal{T}}$. Thus, in addition to the clauses constructed in $CNF_{\eff}(l)$, these new literals (only) allow us to obtain clauses for each run of steps in the trajectory where $l$ is not observed, either starting with a clause of the third type where $l$ is observed true (created in lines 16--17), or using a clause of the second type to eliminate the initial state literal (in line 22). By Lemma~\ref{lem:cnf-char}, this resulting formula can be refuted with $CNF_{\eff}(l)$ iff $l$ cannot be a precondition of $a$.
%%
%%Because we only remove literals from the preconditions when this formula is refuted, in any plan constructed with the EPI-SAM action model, whenever an action is taken it must satisfy the preconditions of the real action. Finally, we note that the contingent plan must achieve the goal with any setting of the $\iseff$ variables that is consistent with $CNF_{\eff}$. Again, by Lemma~\ref{lem:cnf-char}, we see that this means that in particular the goal is achieved with the assignment corresponding to the real action model. Thus, the EPI-SAM action model is indeed safe.
% \end{proof}
% (line~\ref{} in Algorithm~\ref{X}), or when assuming $\ell$ is precondition

% in a transition  
% This immediate that this holds for any $\ell$ deleted in line \ref{episam-easy-pre-delete}. 

% % some trajectory in $\mathcal{T}$ must have had $\ell$ false in the pre-state of $a$, and hence $l$ could not have been in $\pre_{M^*}(a)$ for the true action model; it therefore will follow that whenever a state $s$ satisfies $\pre(a)$, it satisfies $\pre_{A^*}(a)$ and is therefore applicable.

% % It is immediate that this holds for any transition where
% % $\ell$ observed to be false just before $a$ (line \ref{epi:line:easy-remove} in Algorithm~\ref{alg:episam-preconditions}). 
% Otherwise, Algorithm \ref{alg:episam} creates a set of trajectories $\mathcal{T}_{\ell,a}$ in which we assume that $\ell$ was true whenever $a$ was taken; 

%\roni{I don't fully follow this condition. Do you mean that $l$ can be deleted iff there is no action model consistent with $\mathcal{T}_{l,a}$?} \brendan{Roughly so -- except that only the portions of the action model that have effects on l's fluent are relevant/considered, and the question is then whether or not there is an action model (restricted to this fluent) for which l could be the precondition. We consider each literal (hence, fluent) separately.}

%\emph{Sketch of proof (incomplete)}
%In the first part (line 1-9 in Algorithm \ref{alg:episam}), EPI-SAM creates for every literal a CNF that captures all possible clauses that we can write about literal $l$. If literal $l$ was an effect at some action in the trajectories, it must be represented by a clause in the CNF. In the second part (line 10-25 in Algorithm \ref{alg:episam}), the algorithm initializes that for every action, its preconditions contain all possible literals. For each action, the algorithm removes from its preconditions only literal such that there is no possible assignment that satisfies the CNF that the algorithm can infer about that literal. Thus, it won't remove any literal that is an actual precondition of an action.  
% \begin{definition}[Strength of Action Models]
% If there exists a trajectory that is consistent with $M'$ but not with $M$, then we say that $M$ is weaker than $M'$.
% If no such trajectory exists then we say that $M$ is at least as strong as $M'$. 
% \label{def:weakness}
% \end{definition}

% \begin{theorem}[The Strength of SGAM Learning]
% Let $M_{SGAM}$ be the action model created by SGAM learning given the set of trajectories $\mathcal{T}$. 
% $M_{SGAM}$ is at least as strong as any action model $M'$ that is safe and consistent with $\mathcal{T}$. 
% \label{thm:sam-learning-complete-grounded}
% \end{theorem}
% \begin{proof}
% Consider an action model $M'$, which is safe and consistent with $\mathcal{T}$. % and safe w.r.t.\ \realm. 
% Let $a$ be an action and $s$ be a state such that $a$ is applicable in $s$ according to $M'$, i.e., $\pre_{M'}(a)\subseteq s$. 

% Since $M'$ is safe w.r.t.\ \realm, then 
% $\pre_{\realm}(a)\subseteq s$

% and $a_{M'}(s)=a_{\realm}(s)$. 
% By construction of $M_\sgam$, if a literal $l$ is a precondition of $a$ according to $M_\sgam$, 
% then it has appeared in the pre-state of all action triplets in $\mathcal{T}(a)$. 
% Thus, there exists a consistent action model in which $l$ is a precondition of $a$ 
% and this action model may be the real model. 
% Therefore, since $M'$ is safe it follows that $\pre_{M_\sgam}(a) \subseteq \pre_{M'}(a)$, 
% and thus $a$ is applicable in $s$ according to $M_\sgam$, 
% i.e., $\pre_{M_\sgam}(a)\subseteq s$. 
% Since $M_\sgam$ is safe, %it follows that
% $a_{M_\sgam}(s)=a_{\realm}(s)=a_{M'}(s)$. %, as required. 
% Thus, every trajectory consistent with  $M'$ will also be consistent with $M_\sgam$.
% \end{proof}


% We now prove the learned action model is safe. 
\begin{theorem}\label{thm:episam-safe-appendix}
EPI-SAM returns a safe action model.
\end{theorem}
\begin{proof}
Let $A^*$ denotes the action model of the underlying planning problem.
Due to Lemma~\ref{lem:pre-strong-appendix}, every action applicable according to $A_\sam$ is also applicable according to $A^*$. 
Consider a goal $G$ and a (strong) plan to achieve it $\pi_\sam$ created by a conformant planner given $A_\sam$. 
This means $\pi_\sam$ achieves $G$ for any action model that satisfies the $\{\cnf_\ell\}_\ell$. 
Due to Lemma~\ref{lem:cnf-char-appendix}, we know that this means $\pi_\sam$ achieves $G$ according to any action model consistent with $\mathcal{T}$. 
Thus, $\pi_\sam$ also achieves $G$ according to $A^*$, as required. 
\end{proof}


% \begin{theorem}[Safety]\label{thm:episam-safe}
% Let $\pi_\sam$ be a plan generated by a conformant planner given the
% action model returned by EPI-SAM. The contingent planning formulation created by EPI-SAM (Algorithm \ref{alg:episam}) is safe.
% \end{theorem}
% \begin{proof}
% We first claim that if Algorithm \ref{alg:episam} omits $l$ from $\pre(a)$, then some trajectory in $\mathcal{T}$ must have had $l$ false in the pre-state of $a$, and hence $l$ could not have been in $\pre_{M^*}(a)$ for the true action model; it therefore will follow that whenever a state $s$ satisfies $\pre(a)$, it satisfies $\pre_{M^*}(a)$ and is therefore applicable.

% It is immediate that this holds for any $l$ that is deleted in line \ref{episam-easy-pre-delete}. 
% Otherwise, Algorithm \ref{alg:episam} creates a set of trajectories $\mathcal{T}_{l,a}$ in which we assume that $l$ was true whenever $a$ was taken; 
% observe that $l$ can be deleted iff there does not exist a set of effects on the fluent for $l$ consistent with $\mathcal{T}_{l,a}$. 
% %\roni{I don't fully follow this condition. Do you mean that $l$ can be deleted iff there is no action model consistent with $\mathcal{T}_{l,a}$?} \brendan{Roughly so -- except that only the portions of the action model that have effects on l's fluent are relevant/considered, and the question is then whether or not there is an action model (restricted to this fluent) for which l could be the precondition. We consider each literal (hence, fluent) separately.}
% Now, note that line \ref{episam-action-delete} only deletes actions that cannot have an effect on this fluent, as any subsequent (deleted) actions also could not have had an effect on the fluent, and the fluent then occurs with two different values at the next observation. 
% %\roni{This is where I get lost a bit. It seems there's an invariant you're claiming here but I don't get it}\brendan{Yes, I omitted a bit here. Strictly speaking what you have is inductively that for all of the deleted actions in the trajectory, none of those actions could have had an effect on the literal. Supposing that this is true, then any action that appears immediately prior to such sequences of actions, if it has an effect, must set the fluent equal to the next observed value in the trajectory since no subsequent action can change it. But now, if an action would need to set the fluent to two different values (in different portions of the trajectories), then since any action effect would only set the fluent to one value, the action could not had an effect on the fluent, so we can add this action to that sequence, i.e., delete/ignore all occurrences of the action in the trajectories, since the action cannot change the fluent we are examining.}\roni{This seems to say that if we run PI-SAM on the resulting trajectories (projected over that single literal), then we know we get a CNF for the effects that is trivially satisfiable  (without running a SAT solver). I can almost see it but not fully. I understand that the satisfying assignment would set the effect of the remaining action to be the constant value we see for $l$ after them. But, can't that conflict with something we observed for $\neg l$? e.g., some sequence of actions? maybe if I look at it from the fluent prespective I'll get it. Let me think about this more} \brendan{I can try to help: As long as the last possible action in the unobserved sequence sets the literal $l$ true, the action model will be consistent with the observed trajectory. At that point, it doesn't matter whether some earlier action in the sequence had $\neg l$ as an effect. (If that's your concern.) The other thing you could be concerned about is that the same action can't have both $l$ and $\neg l$ as an effect, but this is precisely the condition that causes us to omit the occurrences of the action from the trajectory---so by the contrapositive, if you are setting the action's effect to $l$, this didn't occur (it doesn't appear immediately prior to an observation of $\neg l$), and so the action only appears prior to an observation of $l$ being true.}


% Therefore, in particular, if we encounter an action for which this fluent had a different value in the pre-state that cannot have an effect on the fluent, no such action model can exist and so we can delete the literal on line \ref{episam-hard-pre-delete}. 
% %\roni{I got lost here also. How in the pseudo-code do you ensure that the action (I guess $a'$) cannot have an effect on the fluent?}\brendan{See above. We only delete actions that can't have an effect on the fluent. Hopefully it's clearer now.}
% Now, on the other hand, if the loop on line \ref{episam-action-delete-loop} terminates, any action at the end of a run of states where the fluent is unobserved must be followed by states where the fluent takes at most one value. Then we could assign the effect to that action that sets the fluent to that value, and thus obtain a consistent action model with $\mathcal{T}_{l,a}$. Thus, in this case, $l$ could be a precondition of $a$. The algorithm continues to the next iteration of the loop and does not consider this literal again, thus it remains in $\pre(a)$ upon termination.

% Finally, we note that the contingent plan must achieve the goal with any setting of the $\iseff$ variables that is consistent with $CNF_{\eff}$. By Lemma~\ref{lem:cnf-char}, we see that this means that in particular the goal is achieved with the assignment corresponding to the real action model. Thus, the EPI-SAM action model is indeed safe.
% %%
% %% OLD SAT-BASED ANALYSIS
% %%
% %%Consider the formula $CNF_{\mathcal{T}}(l)$ created in Lemma~\ref{lem:cnf-char}: $\ispre(l,a)$ can be set to true in a satisfying assignment iff for every step $t$ where $a=a_t$, we have $State(l,t-1,h)$ true and $State(\neg l,t-1,h)$ false (equiv., $\neg State(\neg l,t-1,h)$ true).  (Equivalently, these are the clauses that can be derived using the unit clause $\ispre(l,a)$.) Thus, by the refutation completeness of resolution, we can refute this collection of unit clauses with $CNF_{\mathcal{T}}(l)$ iff $l$ cannot be a precondition of $a$.
% %%
% %%Again, the $State$ variables must be eliminated, and the (other) $\ispre$ variables cannot be used in a refutation because they only appear negatively. Again, recall that the $State$ variables only appear negatively in clauses of the third type in $CNF_{\mathcal{T}}$. Thus, in addition to the clauses constructed in $CNF_{\eff}(l)$, these new literals (only) allow us to obtain clauses for each run of steps in the trajectory where $l$ is not observed, either starting with a clause of the third type where $l$ is observed true (created in lines 16--17), or using a clause of the second type to eliminate the initial state literal (in line 22). By Lemma~\ref{lem:cnf-char}, this resulting formula can be refuted with $CNF_{\eff}(l)$ iff $l$ cannot be a precondition of $a$.
% %%
% %%Because we only remove literals from the preconditions when this formula is refuted, in any plan constructed with the EPI-SAM action model, whenever an action is taken it must satisfy the preconditions of the real action. Finally, we note that the contingent plan must achieve the goal with any setting of the $\iseff$ variables that is consistent with $CNF_{\eff}$. Again, by Lemma~\ref{lem:cnf-char}, we see that this means that in particular the goal is achieved with the assignment corresponding to the real action model. Thus, the EPI-SAM action model is indeed safe.
% \end{proof}
% %\emph{Sketch of proof (incomplete)}
% %In the first part (line 1-9 in Algorithm \ref{alg:episam}), EPI-SAM creates for every literal a CNF that captures all possible clauses that we can write about literal $l$. If literal $l$ was an effect at some action in the trajectories, it must be represented by a clause in the CNF. In the second part (line 10-25 in Algorithm \ref{alg:episam}), the algorithm initializes that for every action, its preconditions contain all possible literals. For each action, the algorithm removes from its preconditions only literal such that there is no possible assignment that satisfies the CNF that the algorithm can infer about that literal. Thus, it won't remove any literal that is an actual precondition of an action.  
% % \begin{definition}[Strength of Action Models]
% % If there exists a trajectory that is consistent with $M'$ but not with $M$, then we say that $M$ is weaker than $M'$.
% % If no such trajectory exists then we say that $M$ is at least as strong as $M'$. 
% % \label{def:weakness}
% % \end{definition}

% % \begin{theorem}[The Strength of SGAM Learning]
% % Let $M_{SGAM}$ be the action model created by SGAM learning given the set of trajectories $\mathcal{T}$. 
% % $M_{SGAM}$ is at least as strong as any action model $M'$ that is safe and consistent with $\mathcal{T}$. 
% % \label{thm:sam-learning-complete-grounded}
% % \end{theorem}
% % \begin{proof}
% % Consider an action model $M'$, which is safe and consistent with $\mathcal{T}$. % and safe w.r.t.\ \realm. 
% % Let $a$ be an action and $s$ be a state such that $a$ is applicable in $s$ according to $M'$, i.e., $\pre_{M'}(a)\subseteq s$. 

% % Since $M'$ is safe w.r.t.\ \realm, then 
% % $\pre_{\realm}(a)\subseteq s$

% % and $a_{M'}(s)=a_{\realm}(s)$. 
% % By construction of $M_\sgam$, if a literal $l$ is a precondition of $a$ according to $M_\sgam$, 
% % then it has appeared in the pre-state of all action triplets in $\mathcal{T}(a)$. 
% % Thus, there exists a consistent action model in which $l$ is a precondition of $a$ 
% % and this action model may be the real model. 
% % Therefore, since $M'$ is safe it follows that $\pre_{M_\sgam}(a) \subseteq \pre_{M'}(a)$, 
% % and thus $a$ is applicable in $s$ according to $M_\sgam$, 
% % i.e., $\pre_{M_\sgam}(a)\subseteq s$. 
% % Since $M_\sgam$ is safe, %it follows that
% % $a_{M_\sgam}(s)=a_{\realm}(s)=a_{M'}(s)$. %, as required. 
% % Thus, every trajectory consistent with  $M'$ will also be consistent with $M_\sgam$.
% % \end{proof}

% \begin{theorem}[Strength]
% The contingent planning formulation returned by the EPI-SAM learning algorithm (Algorithm \ref{alg:episam}) is the strongest safe action model.
% \end{theorem}
% \begin{proof}
% In the proof of Theorem~\ref{thm:episam-safe}, we noted that literals are deleted from $\pre(a)$
% %the formula we created can be refuted iff $\ispre(l,a)$ can be refuted with $CNF_{\mathcal{T}}$, i.e., 
% iff $l$ cannot be a precondition of $a$ in any action model consistent with the trajectories in $\mathcal{T}$. Thus, suppose that some action model allows action $a$ to be taken in a state $s$ that does not satisfy the preconditions constructed by EPI-SAM. Then there is some $l$ that is a precondition of $a$ for EPI-SAM that is false in $s$. But by the above,
% %$\ispre(l,a)$ may be set to true in some satisfying assignment of $CNF_{\mathcal{T}}$ -- i.e., 
% $l$ may be a precondition of $a$ in the true action model, so the other action model is not safe. 

% Similarly, suppose that there is a plan under the other action model that is allowed by the EPI-SAM action model, but for which EPI-SAM does not achieve the goal. This means (by Lemma~\ref{lem:cnf-char}) that there was some action model consistent with $\mathcal{T}$ under which the goal was not achieved. Again, the other action model is therefore not safe.
% \end{proof}

\begin{theorem}[Strength]
The action model $A_\sam$ returned by EPI-SAM is the strongest safe action model, 
in the sense that if an action model $A$ is not safe with respect to $A_\sam$, 
then there exists an action model $A'$ consistent with $\mathcal{T}$ 
such that $A$ is not safe with respect to $A'$. 
\end{theorem}
\begin{proof}
By contradiction, assume that $A_{bad}$ is an action model that is not safe with respect to $A_{\sam}$, but it is safe with respect to any action model consistent with $\mathcal{T}$.   
This means that either there exists a literal $\ell$ that is in 
$\pre_{A_\sam}$ but not in $\pre_{A_{bad}}$ 
or a plan $\pi_{bad}$ that 
achieves some goal $G$ according to $A_{bad}$ but not according to $A_\sam$. 
The first condition cannot hold due to Lemma~\ref{lem:pre-strong-appendix}: for any precondition assumed by $A_\sam$ there exists an action model consistent with $\mathcal{T}$ that requires it. 
For the second condition, suppose that there is a plan under $A_{bad}$ that is allowed by the EPI-SAM action model, but for which EPI-SAM does not achieve the goal. This means (by Lemma~\ref{lem:cnf-char-appendix}) that there was some action model consistent with $\mathcal{T}$ under which the goal was not achieved. The other action model is therefore not safe.
% but not  its preconditions
% Due to Lemma~\ref{lem:pre-strong}, removing any precondition assumed by $A_\sam$ is not safe, in the sense that there are action models consistent with $\mathcal{T}$
% from any action 
% In the proof of Theorem~\ref{thm:episam-safe}, we noted that literals are deleted from $\pre(a)$
% %the formula we created can be refuted iff $\ispre(l,a)$ can be refuted with $CNF_{\mathcal{T}}$, i.e., 
% iff $l$ cannot be a precondition of $a$ in any action model consistent with the trajectories in $\mathcal{T}$. Thus, suppose that some action model allows action $a$ to be taken in a state $s$ that does not satisfy the preconditions constructed by EPI-SAM. Then there is some $l$ that is a precondition of $a$ for EPI-SAM that is false in $s$. But by the above,
% %$\ispre(l,a)$ may be set to true in some satisfying assignment of $CNF_{\mathcal{T}}$ -- i.e., 
% $l$ may be a precondition of $a$ in the true action model, so the other action model is not safe. 
%
% Similarly, suppose that there is a plan under the other action model that is allowed by the EPI-SAM action model, but for which EPI-SAM does not achieve the goal. This means (by Lemma~\ref{lem:cnf-char}) that there was some action model consistent with $\mathcal{T}$ under which the goal was not achieved. Again, the other action model is therefore not safe.
\end{proof}



\begin{theorem}
Given a set of trajectories $\mathcal{T}$, the EPI-SAM learning runs in time 
\begin{small}
\[\mathcal{O}\Big( |A|\cdot|\mathcal{F}|\cdot \sum_{a\in \mathcal{A}}|\mathcal{T}(a)|    \Big)\]
\end{small}
where $A$ is the set of actions, $\mathcal{F}$ is the set of literals.
\end{theorem}
\begin{proof}
The EPI-SAM algorithm consists of two parts: learning the effects and learning the preconditions. For learning the effects (algorithm \ref{alg:episam-effects-appendix}), the algorithm has to iterate over all literals to create a CNF formula for each literal. The first inner loop (line 3-4) iterates through all actions to add mutually exclusive clauses for each action to the CNF, while the second inner loop (line 5-12) goes through every transition in each trajectory to add clauses to the CNF based on Rule 1, 3 of Observation \ref{obs:epi-sam-learning-rules-appendix}. Thus, the total time complexity of Algorithm \ref{alg:episam-effects-appendix} is $\mathcal{O}\Big(|\mathcal{F}|(|A| + \sum_{a\in \mathcal{A}}|\mathcal{T}(a))|\Big)$.  

The second part, learning the preconditions (Algorithm \ref{alg:episam-preconditions-appendix}), iterates over all actions and literals to build the precondition (in form of conjunction) for each action. In each inner loop, each literal in each transition of each trajectory is examined $O(1)$ times---in the first loop we create one clause for each step, and it is set to true or deleted at most once in the second loop. 
We can perform the bookkeeping in the second loop in linear time overall by suitable data structures: we maintain a linked list over the occurrences of a given action, all with a reference to a common structure for the action that records which settings of $l$ appear in the post-state, and we record each of the unobserved runs of a literal with a linked list. Then checking if an action should be deleted takes $O(1)$ time and deleting the occurrences of an action takes $O(1)$ time per occurrence. The data structures likewise take $O(1)$ time per each occurrence of a literal to initialize. The algorithm has to go through every transition in each trajectory to build the data structures and perform the check/delete operations. Thus, the total time complexity of Algorithm \ref{alg:episam-preconditions-appendix} is $\mathcal{O}\Big(|A|\cdot|\mathcal{F}|\cdot \sum_{a\in \mathcal{A}}|\mathcal{T}(a)|\Big)$.  \end{proof}


%\emph{Sketch of proof (incomplete)}
%Let $CNF_\eff=\bigwedge_l CNF_\eff(l)$. 
%Observation 1: The effects part of every action model that is consistent with the observations is represented by a satisfying assignment to $CNF_\eff$.
%Observation 2: Every satisfying assignment to $CNF_\eff$ represents the effects part of an action model that is consistent with the observations. 
%Proof of observation 1+2:
%Here we need to say something about how it is Ok to learn the effects before the preconditions. 
%Observation 3: If EPI-SAM removed a literal $l$ from $pre(a)$ then $l$ isn't a precondition in any consistent action model.  
%Observation 4: If EPI-SAM did not remove a literal $l$ from $pre(a)$ then there exists a consistent action model in which $l$ is in $pre(a)$.  


% \paragraph{Time Complexity}
%\brendan{Note: while it's linear time for propositional domains so long as we can create a clause in unit time, it might actually be quadratic in the lifted domains since the same parameter bound literal could correspond to different ground literals at different points in the trajectory. We could suppress the difference by just saying ``polynomial time'' and leaving the exact exponent to the reader, and this would also save us from needing to talk about data structures.}

% \begin{theorem}
% Given a set of trajectories $\mathcal{T}$, the EPI-SAM learning runs in time 
% \begin{small}
% \[\mathcal{O}\Big(\sum_{a\in \mathcal{A}}|\mathcal{T}(a)|\sum_{f\in\mathcal{F}}\prod_{t\in T}arity(a,t)^{arity(f,t)}\Big)\]
% \end{small}
% \end{theorem}

% \begin{proof}
% EPI-SAM algorithm consists of two parts: learning the effects and learning the preconditions. For learning the effects (algorithm \ref{alg:episam-effects}), the algorithm has to iterate over all literals to create a CNF formula for each literal. The first inner loop (line 3-4) iterates through all actions to add mutually exclusive clauses for each action to the CNF, while the second inner loop (line 5-12) goes through every transition in each trajectory to add clauses to the CNF based on Rule 1, 3 of Observation \ref{obs:epi-sam-learning-rules}. Thus, the total time complexity of Algorithm \ref{alg:episam-effects} is $\mathcal{O}\Big(|\mathcal{F}|(|A| + \sum_{a\in \mathcal{A}}|\mathcal{T}(a))|\Big)$.  

% The second part, learning the preconditions (Algorithm \ref{alg:episam-preconditions}), iterates over all actions and literals to build the precondition (in form of conjunction) for each action. In each inner loop, each literal in each transition of each trajectory is examined $O(1)$ times---in the first loop we create one clause for each step, and it is set to true or deleted at most once in the second loop. 
% We can perform the bookkeeping in the second loop in linear time overall by suitable data structures: we maintain a linked list over the occurrences of a given action, all with a reference to a common structure for the action that records which settings of $l$ appear in the post-state, and we record each of the unobserved runs of a literal with a linked list. Then checking if an action should be deleted takes $O(1)$ time and deleting the occurrences of an action takes $O(1)$ time per occurrence. The data structures likewise take $O(1)$ time per each occurrence of a literal to initialize. The algorithm has to go through every transition in each trajectory to build the data structures and perform the check/delete operations. Thus, the total time complexity of Algorithm \ref{alg:episam-preconditions} is $\mathcal{O}\Big(|A|\cdot|\mathcal{F}|\cdot \sum_{a\in \mathcal{A}}|\mathcal{T}(a)|\Big)$.  
% \end{proof}
%The data structures described above are reminiscent of those used by SAT solvers for unit propagation. 
% in SAT solvers. Indeed, a SAT encoding of the problem would be solved by standard SAT solvers in linear time.
%The data structures described above are similar to the data structures enabling unit propagation to be performed in linear time; indeed, a SAT encoding of the problem would be solved by standard SAT solvers in linear time.
%EPI-SAM (Algorithm \ref{alg:episam}) consists of two parts: extracting all possible effect clauses for each literal as a CNF, and running the SAT solver to determine the preconditions for each action. The first part runs in polynomial time in the number of literals and total number of state-action triplets. The second part, however, depends on the running time of the SAT solver, which cannot be polynomial time if it is complete (unless P=NP). This is inherent:
%\begin{theorem}
%The following problem is NP-complete: given a set of partially observed trajectories $\mathcal{T}$, action $a$, and literal $l$, decide whether $l$ is a precondition of $a$ in some action model consistent with $\mathcal{T}$.
%\end{theorem}
%\noindent
%{\em Sketch of proof.}
%The problem is clearly in NP. We reduce 3SAT to the problem by creating an action for each literal and a trajectory for each clause in a domain with a single fluent $f$, ending in an additional action. We also create a trajectory for each literal ensuring that one literal in each pair has $f$ as an effect. The additional action does not have $f$ as a precondition iff $f$ is false in one of these trajectories, which occurs only if the formula is unsatisfiable.


% \section{EPI-SAM Safety Proof Details}

% \begin{lemma}\label{lem:pre-strong}
% For every action $a$ in $A_\sam$ and literal $\ell$, it holds that 
% $\ell\in\pre_{A_\sam}(a)$ if and only if there exists an action model $A$ consistent with $\mathcal{T}$ where $\ell\in\pre_A(a)$. 
% \end{lemma}
% \begin{proof}
% We first prove that if EPI-SAM removes a literal $\ell$ from $\pre(a)$, then there exists a transition $\tuple{s,a,s'}$ in $\mathcal{T}$ where $\ell$ is false, and hence cannot be in $\pre_{A*}(a)$. 
% EPI-SAM removes $\ell$ from $\pre(a)$ in two places in Algorithm~\ref{alg:episam-preconditions}: line~\ref{episam-easy-pre-delete} and line~\ref{episam-hard-pre-delete}. 
% The correctness of line~\ref{episam-easy-pre-delete} is immediate: if $\ell$ is observed to be false in a state where $a$ has been applied then it cannot be a precondition of $a$ (PI-SAM Rule 1). 
% Before removing a precondition due to line~\ref{episam-hard-pre-delete}, 
% EPI-SAM creates a set of trajectories $\mathcal{T}_{\ell,a}$ that assumes $\ell$ was true whenever $a$ was taken, 
% and detects the set of actions $A_{irr}$ that cannot affect the value of $\ell$ in any action model consistent with $\mathcal{T}_{\ell,a}$. 
% Because of the frame axioms, the value of $\ell$ gets propagated in any transition that includes an action in $A_{irr}$. $\ell$ is only removed in line \ref{episam-hard-pre-delete} if this propagation results in a state where $\ell$ has contradicting values. As this occurs for any action model consistent with $\mathcal{T}_{\ell,a}$, this implies that $\ell$ cannot be true in every state where $a$ was applied, and thus cannot be a precondition of $a$ in any action model consistent with $\mathcal{T}$. 



% Next, we prove that if $\ell$ has not been deleted from $\pre(a)$ by EPI-SAM, then there exists an action model $A$ consistent with $\mathcal{T}$ where $\ell\in\pre_A(a)$. 
% % Since $\ell$ has not been deleted from $\pre_{A_\sam}$, then 
% % for any action $a\notin A_{irr}$ we have that the value of $\ell$ after $a$ is either masked or takes at most one value (true of false). 
% Consider the subset of $A_{irr}$ that includes only actions that have been in a transition where the value of $\ell$ is not masked. 
% For each action $a'$ in this set, we are guaranteed that this value of $\ell$ is always the same, denoted $v(a',\ell)$. Otherwise $a'$ would have been added to $A_{irr}$. 
% The action model created by assigning $v(a',\ell)$ as an effect of $a'$ for each of these actions is consistent with $\mathcal{T}_{l,a}$. 
% Therefore, there exists an action model where $\ell$ is a precondition of $a$ that is consistent with $\mathcal{T}$. 
% \end{proof}





% \begin{theorem}
% Given a set of trajectories $\mathcal{T}$, the EPI-SAM learning runs in time 
% \begin{small}
% \[\mathcal{O}\Big( |A|\cdot|\mathcal{F}|\cdot \sum_{a\in \mathcal{A}}|\mathcal{T}(a)|    \Big)\]
% \end{small}
% where $A$ is the set of actions, $\mathcal{F}$ is the set of literals.
% \end{theorem}
% \begin{proof}
% EPI-SAM algorithm consists of two parts: learning the effects and learning the preconditions. For learning the effects (algorithm \ref{alg:episam-effects}), the algorithm has to iterate over all literals to create a CNF formula for each literal. The first inner loop (line 3-4) iterates through all actions to add mutually exclusive clauses for each action to the CNF, while the second inner loop (line 5-12) goes through every transition in each trajectory to add clauses to the CNF based on Rule 1, 3 of Observation \ref{obs:epi-sam-learning-rules}. Thus, the total time complexity of Algorithm \ref{alg:episam-effects} is $\mathcal{O}\Big(|\mathcal{F}|(|A| + \sum_{a\in \mathcal{A}}|\mathcal{T}(a))|\Big)$.  

% The second part, learning the preconditions (Algorithm \ref{alg:episam-preconditions}), iterates over all actions and literals to build the precondition (in form of conjunction) for each action. In each inner loop, each literal in each transition of each trajectory is examined $O(1)$ times---in the first loop we create one clause for each step, and it is set to true or deleted at most once in the second loop. 
% We can perform the bookkeeping in the second loop in linear time overall by suitable data structures: we maintain a linked list over the occurrences of a given action, all with a reference to a common structure for the action that records which settings of $l$ appear in the post-state, and we record each of the unobserved runs of a literal with a linked list. Then checking if an action should be deleted takes $O(1)$ time and deleting the occurrences of an action takes $O(1)$ time per occurrence. The data structures likewise take $O(1)$ time per each occurrence of a literal to initialize. The algorithm has to go through every transition in each trajectory to build the data structures and perform the check/delete operations. Thus, the total time complexity of Algorithm \ref{alg:episam-preconditions} is $\mathcal{O}\Big(|A|\cdot|\mathcal{F}|\cdot \sum_{a\in \mathcal{A}}|\mathcal{T}(a)|\Big)$.  

% \end{proof}


% \begin{algorithm}[t]
% \small
% \DontPrintSemicolon
% \SetKwInOut{Input}{Input}\SetKwInOut{Output}{Output}
% \Input{Original problem $\Pi=\tuple{F,A,I,G}$, 
% $Pre(a)$ for every action $a$,
% $CNF_{\eff}(l)$ for every literal $l$}
% \Output{Contingent planning problem $\Pi'$}

% \ForEach{action $a$}{Create $obs_a$}
% \ForEach{literal $l$}{Create $obs_l$}
% \ForEach{pair of action and literal}{Create a fluent \texttt{(is-eff $obj_a$, $obj_l$)}} 

% % Stuff to add to the domain file
% \ForEach{observed action $a$}{
%     Add an action to the domain with $Pre(a)$ as its preconditions\\
%     \ForEach{literal $l$}{
%         Add a conditional effect of the form:
%         \texttt{(when (is-eff $obj_a$ $obj_l$) $l$)}
%     }
% }

% % Stuff to add to the problem file
% %The problem contains all fluents from the original problem and the following fluents:
% \ForEach{literal $l$}{
%     \ForEach{clause $ C=(\sigma_1 \iseff(a_1,l) \vee \sigma_2 \iseff(a_2,l) \vee ... \vee \sigma_k \iseff(a_k,l)) \in \varphi$ ($\sigma_i$ is identity or $\neg$)}{
%     %-- i.e., if it is identity, no symbol appears for $\sigma_i$; if it's negation, ``not'' appears for $\sigma_i$ below
%         Add to problem file:\\
%         $\left(\textit{oneof}~~ \sigma_1 \iseff(a_1,l) 
%         ~\sigma_2 \iseff(a_2,l)
%         \cdots
%         ~\sigma_k \iseff(a_k,l) \right)$
%         }
%     }
% \caption{Action Model to Contingent PDDL}\label{alg:tocontingent}
% \end{algorithm}


\iffalse{
%% Indeed, the following is false..
\setcounter{theorem}{4}
\begin{theorem}
The following problem is NP-complete: given a set of partially observed trajectories $\mathcal{T}$, action $a$, and literal $l$, decide whether $l$ is a precondition of $a$ in some action model consistent with $\mathcal{T}$.
\end{theorem}
\begin{proof}
To see that the problem is in NP, observe that we can take an encoding of an action model and proposed completions of the trajectories of $\mathcal{T}$ as input. In linear time, we can scan the completions to check that they agree with the corresponding trajectories of $\mathcal{T}$ for the observed fluents. Then, in linear time, we can check that the completions are consistent with the action model. Finally, we simply check that $l$ is a precondition of action $a$ in the proposed model, and accept iff all of these checks pass. It is immediate that the given action model witnesses that $l$ may be a precondition of $a$ if it is accepted; conversely, if such an action model exists, it can be provided along with the corresponding complete trajectories it would produce for $\mathcal{T}$.

Now, to show that the problem is complete, we reduce from 3SAT. The reduction is as follows: we are given as input a 3CNF consisting of clauses $\ell_{i,1}\lor\ell_{i,2}\lor\ell_{i,3}$ on $n$ variables. We create a planning domain with $3m+2n+1$ actions, $A_0,A_1,\bar{A}_1,\ldots,A_n,\bar{A}_n,A_{1,1},A_{1,2},A_{1,3},\ldots,A_{m,1},A_{m,2},A_{m,3}$ and one fluent, $f$.  
For each positive literal $\ell_{i,j}=x_k$ in a clause, we create a trajectory where initially $f$ is unobserved, then $A_{i,j}$, with $f$ unobserved, followed by $A_k$, followed by $f$ false.
For each negative literal $\ell_{i,j}=\neg x_k$, we create a similar trajectory, where initially $f$ is unobserved, then $A_{i,j}$, with $f$ unobserved, followed by $\bar{A}_k$, followed by $f$ false.
Next, for $k=1,\ldots,n$, we have a trajectory in which initially $f$ is false, then $A_k$ with $f$ unobserved, then $\bar{A}_k$ with $f$ true.

%For each pair of complementary literals $\ell_{i,p},\ell_{j,q}$, we create two trajectories, we have initially $f$ unobserved, then $A_{i,p}$ with $f$ unobserved, followed by $A_{j,q}$, followed by $f$ false. Then we have $A_{j,q}$ with $f$ unobserved, followed by $A_{i,p}$, followed by $f$ false. This is at most $3m(3m-1)$ trajectories.

In the next $m$ trajectories, we have one for each $i$th clause: we have a trajectory where $f$ is initially observed false, followed by $A_{i,1},A_{i,2},A_{i,3}$ (with $f$ unobserved) followed by $A_0$ (with $f$ unobserved again). Finally, we ask if it is possible that $f$ is a precondition of $A_0$. We observe that the reduction is indeed polynomial time.

For correctness, we first observe that we cannot have both $f$ and $\neg f$ as effects of the same action. 
The first trajectories ensure that if $f$ is an effect of $A_{i,j}$ then for a positive literal $x_k$, $A_k$ has $\neg f$ as an effect, and for a negative literal $\neg x_k$, $\bar{A}_k$ has $\neg f$ as an effect.
The next set of trajectories ensure that both $A_k$ and $\bar{A}_k$ cannot have $\neg f$ as effects.
%The first trajectories ensure that if $f$ was an effect of $A_{i,p}$, $\neg f$ is an effect of $A_{j,q}$, and the next trajectories ensure vice-versa, if $f$ is an effect of $A_{j,q}$, $\neg f$ must be an effect of $A_{i,p}$. Note that therefore, in particular, $f$ cannot be an effect of both $A_{i,p}$ and $A_{j,q}$ if these are a complementary pair (though $f$ might not be an effect of either).

$f$ can only be a precondition of $A_0$ if there is a completion of the final $m$ trajectories in which it is always true. If such a completion exists, then by setting the literals to true that correspond to the actions with $f$ as an effect, we note that we get a satisfying assignment. (Note that the literals corresponding to the negation cannot also have $f$ as an effect, so these must be consistent.) Conversely, given a satisfying assignment, we consider an action model in which the final satisfied literal in each clause has $f$ as an effect.
\end{proof}
}\fi


% \bibliography{aaai24}

\end{document}
