%File: formatting-instruction.tex
\documentclass[letterpaper]{article} % DO NOT CHANGE THIS
\usepackage{aaai24}  % DO NOT CHANGE THIS
\usepackage{times}  % DO NOT CHANGE THIS
\usepackage{helvet}  % DO NOT CHANGE THIS
\usepackage{courier}  % DO NOT CHANGE THIS
\usepackage[hyphens]{url}  % DO NOT CHANGE THIS
\usepackage{graphicx} % DO NOT CHANGE THIS
\urlstyle{rm} % DO NOT CHANGE THIS
\def\UrlFont{\rm}  % DO NOT CHANGE THIS
\usepackage{natbib}  % DO NOT CHANGE THIS AND DO NOT ADD ANY OPTIONS TO IT
\usepackage{caption} % DO NOT CHANGE THIS AND DO NOT ADD ANY OPTIONS TO IT
\frenchspacing  % DO NOT CHANGE THIS
\setlength{\pdfpagewidth}{8.5in}  % DO NOT CHANGE THIS
\setlength{\pdfpageheight}{11in}  % DO NOT CHANGE THIS
\usepackage{algorithm}
% \usepackage{algorithmic}
\usepackage{multirow}
\usepackage{makecell}
\usepackage{pifont}
\usepackage{bbding}
\usepackage{amsmath}
\usepackage{amssymb}
\usepackage{algpseudocode}
\usepackage{booktabs}
\usepackage{amstext}
\usepackage{bm}
% \usepackage{subfigure}
\newcommand{\eg}{\textit{e}.\textit{g}.}
\usepackage{enumitem}
\usepackage[table]{xcolor}

\usepackage{url}            % simple URL typesetting
\usepackage{booktabs}       % professional-quality tables
\usepackage{mathtools,amssymb}
\usepackage{amsfonts}       % blackboard math symbols
\usepackage{nicefrac}       % compact symbols for 1/2, etc.
\usepackage{microtype}      % microtypography
\usepackage{pgfplots,pgfplotstable}
\pgfplotsset{compat=1.14}
\usepackage{array,colortbl}
\usepackage{xcolor}
\usepackage{algorithm,algorithmicx,algpseudocode}
\usepackage[capitalise]{cleveref}
\usepackage{caption}
\usepackage{graphbox}
\usepackage{placeins}
% \usepackage{wrapfig}
\usepackage{subcaption}
\usepackage{etoolbox}


\newcommand{\bzero}{\mathbf{0}}
\newcommand{\bone}{\mathbf{1}}
\newcommand{\bb}{\mathbf{b}}
\newcommand{\bu}{\mathbf{u}}
\newcommand{\bv}{\mathbf{v}}
\newcommand{\bw}{\mathbf{w}}
\newcommand{\bx}{\mathbf{x}}
\newcommand{\by}{\mathbf{y}}
\newcommand{\bz}{\mathbf{z}}
\newcommand{\bxh}{\hat{\mathbf{x}}}
\newcommand{\btheta}{{\boldsymbol{\theta}}}
\newcommand{\bphi}{{\boldsymbol{\phi}}}
\newcommand{\bepsilon}{{\boldsymbol{\epsilon}}}
\newcommand{\bmu}{{\boldsymbol{\mu}}}
\newcommand{\bnu}{{\boldsymbol{\nu}}}
\newcommand{\bSigma}{{\boldsymbol{\Sigma}}}
\newcommand{\vardbtilde}[1]{\tilde{\raisebox{0pt}[0.85\height]{$\tilde{#1}$}}}
\newcommand{\defeq}{\coloneqq}
\newcommand{\grad}{\nabla}
\newcommand{\E}{\mathbb{E}}
\newcommand{\Var}{\mathrm{Var}}
\newcommand{\Cov}{\mathrm{Cov}}
\newcommand{\Ea}[1]{\E\left[#1\right]}
\newcommand{\Eb}[2]{\E_{#1}\!\left[#2\right]}
\newcommand{\Vara}[1]{\Var\left[#1\right]}
\newcommand{\Varb}[2]{\Var_{#1}\left[#2\right]}
\newcommand{\kl}[2]{D_{\mathrm{KL}}\!\left(#1 ~ \| ~ #2\right)}
\newcommand{\pdata}{{p_\mathrm{data}}}
\newcommand{\bA}{\mathbf{A}}
\newcommand{\bI}{\mathbf{I}}
\newcommand{\bJ}{\mathbf{J}}
\newcommand{\bH}{\mathbf{H}}
\newcommand{\bL}{\mathbf{L}}
\newcommand{\bM}{\mathbf{M}}
\newcommand{\bQ}{\mathbf{Q}}
\newcommand{\bR}{\mathbf{R}}
\newtheorem{lemma}{Lemma}
\newtheorem{theorem}{Theorem}


\pdfinfo{
/Title (Generating and Reweighting Dense Contrastive Patterns for Unsupervised Anomaly Detection)
/Author (Put All Your Authors Here, Separated by Commas)}
\setcounter{secnumdepth}{0}  

\begin{document}
\title{Generating and Reweighting Dense Contrastive Patterns\\for Unsupervised Anomaly Detection}
\author{
    Songmin Dai\textsuperscript{\rm 1}\thanks{The first two authors contributed equally to this paper.},
    Yifan Wu\textsuperscript{\rm 1}\footnotemark[1],
    Xiaoqiang Li\textsuperscript{\rm 1}\thanks{Corresponding author.},
    Xiangyang Xue\textsuperscript{\rm 2}
}
\affiliations{
    \textsuperscript{\rm 1}School of Computer Engineering and Science, Shanghai University\\
    \textsuperscript{\rm 2}School of Computer Science, Fudan University\\
    laodar@shu.edu.cn, VictorWu@shu.edu.cn, xqli@shu.edu.cn, xyxue@fudan.edu.cn
}

\maketitle


\begin{abstract}
Recent unsupervised anomaly detection methods often rely on feature extractors pretrained with auxiliary datasets or on well-crafted anomaly-simulated samples. However, this might limit their adaptability to an increasing set of anomaly detection tasks due to the priors in the selection of auxiliary datasets or the strategy of anomaly simulation. To tackle this challenge, we first introduce a prior-less anomaly generation paradigm and subsequently develop an innovative unsupervised anomaly detection framework named GRAD, grounded in this paradigm. GRAD comprises three essential components: (1) a diffusion model (PatchDiff) to generate contrastive patterns by preserving the local structures while disregarding the global structures present in normal images, (2) a self-supervised reweighting mechanism to handle the challenge of long-tailed and unlabeled contrastive patterns generated by PatchDiff, and (3) a lightweight patch-level detector to efficiently distinguish the normal patterns and reweighted contrastive patterns. The generation results of PatchDiff effectively expose various types of anomaly patterns, e.g. structural and logical anomaly patterns. In addition, extensive experiments on both MVTec AD and MVTec LOCO datasets also support the aforementioned observation and demonstrate that GRAD achieves competitive anomaly detection accuracy and superior inference speed.
\end{abstract}

\section{Introduction}
\label{sec:introduction}

\begin{figure}[!t]
    \centering
    \includegraphics[width=1\linewidth]{images/speed.pdf}
    \vspace{-0.8cm}
    \caption{Anomaly detection performance vs. latency per image on an NVIDIA Tesla V100 GPU. Each bubble’s area is proportional to the number of parameters in each detector, and each AU-ROC value is an average of the image-level detection AU-ROC values on MVTec LOCO~\cite{MVloco}.}
    \label{fig: speed}
    \vspace{-0.6cm}
\end{figure}

Image anomaly detection plays a crucial role in various fields, including industrial product defect detection, medical image lesion detection, security screening using X-ray images, and video surveillance~\cite{ComplentaryGAN, MVTecAD, app1, app2, GANomaly}. However, securing real-world anomalous data for training is typically challenging and scarce due to the inability to cover a sufficiently diverse range of potential anomaly patterns. %%%%%ToBeFixed
Consequently, the setting of one-class learning, which employs only normal samples for model training, has proven to be better suited for most industrial anomaly detection tasks~\cite{MVTecAD, MVloco}. 
In recent years, many high-accuracy industrial anomaly detection methods heavily rely on ImageNet~\cite{ImageNet10} pretrained feature extractor. Nevertheless, such reliance may limit their generalization capabilities in scenarios~\cite{MVloco} where ImageNet pretrained features are insufficiently informative, or on other types of image-like data~\cite{mvtec3D, BackTo3dFeatures}. Additionally, some methods have achieved promising results on the MVTec AD~\cite{MVTecAD} without using pretrained feature extractors. These methods utilize manually-selected external out-of-distribution (OOD) datasets~\cite{FCDD} or carefully designed anomaly-simulated data to sample anomaly patterns~\cite{CutPaste, DRAEM, SLSG}. However, previous anomaly acquisition strategies can be considered as ad-hoc solutions that overly rely on priors or visual inspection of test images, such as in MVTec AD, where most anomalies are low-level structure anomalies (\eg, scratches, dents, and contaminations). Such reliance may cause these strategies to fail in detecting other types of anomalies, such as logical anomalies recently proposed in the MVTec LOCO~\cite{MVloco}. These logical anomalies are represented as violations of logical constraints in images, which not only challenges the ananomaly simulation-based methods but also the pretrained representations by auxiliary datasets. Therefore, it becomes necessary to devise image anomaly detection techniques that are independent of both pretrained Imagenet feature extractors and ad-hoc anomaly acquisition strategies.

In this paper, we introduce a novel framework named \textbf{GRAD} (\textbf{G}enerating and \textbf{R}eweighting dense contrastive patterns for unsupervised \textbf{A}nomaly \textbf{D}etection), which achieves SOTA performance in both anomaly detection accuracy and inference runtime, as depicted in Fig.~\ref{fig: speed}. We first put forward a novel anomaly generation paradigm: retaining the structure information within each small patch of the image while disregarding the global structure information of the whole image. Based on this paradigm, we design an anomaly generator called PatchDiff. This generator enforces a constraint on the receptive field size of the diffusion model~\cite{DDPM} and removes the attention layers~\cite{AttentionNotNeed}, thus ensuring that only the local structure within each patch is retained, while the global structure is discarded. As illustrated in Fig.~\ref{fig: loco_generation_results}, with different sizes of the receptive field, PatchDiff can generate diverse dense contrastive patterns that cover a range of anomaly types, \eg, the structural and logical anomalies proposed in MVTec LOCO. Subsequently, we expect to utilize the generated local anomaly patterns to learn a patch-level anomaly detector. However, the contrastive patterns generated by PatchDiff may also be normal and we cannot provide patch-wise ground truth for them. Consequently, the generated contrastive patterns are unlabeled. Furthermore, the local patterns in both normal and generated data could often be long-tailed. Considering the previous two points, we introduce a self-supervised reweighting mechanism to mitigate the negative impacts of fake anomalous patches (patches without effective anomaly patterns) and imbalanced distribution. The mechanism utilizes density information of the features extracted by the detector during the training phase to assign different weights to the contrastive patches. It filters the fake anomalous patches and rebalances the distribution of the contrastive patches. Finally, to obtain high-throughput anomaly detection models better applied in practical industrial scenarios, we design a lightweight Fully Convolutional Network (FCN)-based patch-level detector with a pure encoder architecture. It consists of only 8 convolutional layers but performs on par with larger models in industrial anomaly detection. Furthermore, to deal with tasks that involve mixed-level anomalies, we can also integrate multiple detectors with different receptive fields. We empirically find that a single-level detector is enough to achieve competitive accuracy on MVTec AD dataset, while three detectors can be integrated to handle both structural and logical anomalies in MVTec LOCO.

\begin{figure}[!t]
    \centering
    \includegraphics[width=\linewidth]{images/loco_generation_results.pdf}
    \vspace{-0.4cm}
    \caption{Anomaly contrastive images generated by our PatchDiff on MVTec LOCO. The number $n$ above the images indicates that this column is generated based on the corresponding $n \times n$ receptive field size. We show that employing varying sizes of limited receptive fields effectively enables the PatchDiff to expose anomalies at different levels: generators with smaller sizes tend to expose structural anomalies, while generators with larger sizes tend to expose logical anomalies.}%They provide dense contrastive patterns about how the normal images are combined by the elements at each level.
    \label{fig: loco_generation_results}
    \vspace{-0.4cm}
\end{figure}

The main contributions of this paper can be summarized as follows:
\begin{itemize}
    \item We propose a novel paradigm for generating anomaly patterns without scenario-specific priors. Based on this, we develop PatchDiff which can effectively expose a range of local anomaly patterns.
    \item We introduce a self-supervised reweighting mechanism for the generated contrastive data to rebalance them and filter out the fake anomalous patches. This mechanism enables we can efficiently use the unlabeled and long-tailed contrastive patterns for anomaly detection.
    \item We design a lightweight encoder-based patch-level detector trained with only the normal data and generated contrastive data, which relies on no external dataset, heavy pretrained backbone, or memory-consuming decoder architecture. 
\end{itemize}


\section{Related Works}


\begin{figure*}[!t]
\centering
\includegraphics[width=\linewidth]{images/diffusion.pdf}
    \caption{An illustration of the proposed anomaly generator, PatchDiff. Compared with usual Diffusion models, PatchDiff limits the receptive field of the U-Net used to denoise, which preserves only the local structures rather than the global structures. PatchDiff can effectively produce higher-level novel visual structures coming from the recombinations of specific-level local structures. We can use PatchDiffs with various receptive filed sizes to generate multilevel dense contrastive patterns, which are useful for exposing the multilevel anomalies like the structure anomaly and the logical anomaly in MVTec LOCO.}
    \label{fig: PatchDiff}
\end{figure*}

\noindent\textbf{Reconstruction-based.}
A well-trained autoencoder (AE) on normal data is supposed to produce lower reconstruction errors on the normal data than the anomalous data~\cite{AE1, AE2, VAE}. However, in practice, it may also reconstruct anomalies very well or even better~\cite{Pidhorskyi2018GenerativePN}. To alleviate this problem, recent works developed many advanced variants of AE by using generative priors or novel architectures~\cite{OCGAN, MemAE, DAAD, RIAD, InTra, SSPCAB, UniAD}.

\noindent\textbf{Pretrained feature-based.}
State-of-the-art methods for industrial anomaly detection tend to use features of a deep network pretrained on external datasets (\eg, ImageNet). These methods~\cite{PaDiM, DifferNet, Cflow-ad, PatchCore, hyun2023reconpatch, zhang2023prototypical} effectively utilize the general low-level visual features encoded by the pretrained network to do the anomaly detection and achieve appealing performance on MVTec AD~\cite{MVTecAD}. However, they are hard to directly apply in other image-like domains (e.g. the depth map)~\cite{mvtec3D, BackTo3dFeatures} or to cover the higher-level anomaly type, logical anomalies~\cite{MVloco}.

\noindent\textbf{Anomaly simulation-based.}
% In the pertinent field of OOD~\cite{baseline_ood}, prior studies~\cite{Relu, OE, energyOOD} advocate for the utilization of outliers exposure to engender low confidence in the OOD space. 
To overcome the limitations of pre-trained features and ensure that the model produces well-defined and expected results outside the normal distribution, several anomaly simulation methods~\cite{FCDD, CutPaste, DRAEM, SLSG} are proposed. They employ various ad-hoc strategies to simulate specific types of anomaly patterns tailored to different datasets. Most of them heavily rely on human priors and can only expose specific anomaly patterns, making them also challenging to generalize to different scenarios.
%In particular, our work is similar to the anomaly simulation-based methods but actually is generation-based.


\section{Approaches}

Our method can be primarily divided into two stages: (1) Generating diverse contrastive images based on our novel proposed anomaly generation paradigm to cover the anomaly patterns at interest levels. (2) Training lightweight patch-level detectors with our proposed reweighting mechanism to fully utilize the unlabeled and long-tailed generated contrastive patterns. In the following, we will describe the key parts of GRAD in detail.

\subsection{Generating anomaly Contrastive Images}
\label{sec: generating}

In contrast to previous ad-hoc anomaly acquisition strategies~\cite{CutPaste, DRAEM, SLSG}, we introduce a novel and prior-less anomaly generation paradigm: preserving the structure information within each small image patch while disregarding the global structure information of the entire image. To implement this, we propose a diffusion model~\cite{DDPM} based generator called PatchDiff.
As shown in Fig.~\ref{fig: PatchDiff}, the diffusion and denoise process is very similar to DDPM, the differences mainly come from the U-Net architecture in the following aspects:  

(1) To prevent the U-Net from utilizing long-range information for recovering global structures during denoising, we deliberately remove self-attention used in DDPM~\cite{DDPM}. Self-attention is a powerful tool for capturing long-range contextual information, but for our specific task, it is unnecessary~\cite{AttentionNotNeed}, since local consistency is all we need. 

(2) To further ensure that the U-Net focuses on recovering the local patterns within the corresponding patches during denoising, we directly reduce the depth of both the encoder and decoder of the U-Net. In this way, each latent neuron of the bottleneck has a limited receptive field, and thus it denoises using only the local content and retaining only local structures. %cover the patterns of each patch of corresponding size.

(3) To enable the U-Net to effectively model position-dependent cues, we incorporate a 2-channel coordinate map as additional information alongside the input. This coordinate map is a tensor with dimensions matching that of the input image, where each element represents the coordinate of the corresponding pixel. Noteworthy, the output of our U-Net is still a 3-channel image as same as the original U-Net in DDPM.

Then we modify the training loss of original Diffusion models by introducing a global tiny noise $\bepsilon_{g}$ during the noise injection process. It is motivated by the observation that there is a tendency for overall color deviation in the generated results. Consequently, to avoid the color deviation, the training loss of PatchDiff at each denoising step $t$ becomes 
\begin{equation*}
\begin{aligned}
\mathbb{E}_{\bepsilon_1, \bepsilon_g}
\big\| \bepsilon_1 - \bepsilon_\theta\bigl(\sqrt{\bar{\alpha}_t} \mathbf{x}_0 + \sqrt{1-\bar{\alpha}_t} \bepsilon_1 + \bepsilon_g, t\bigr) \big\|^2,
\end{aligned}
\end{equation*}
where $\epsilon_{1}\sim\mathcal{N}(\mathbf{0}, \mathbf{I})$, $\epsilon_{\theta}$ and $\bar{\alpha}_t$ are the same as in DDPM. 
% $\epsilon_{g}\sim\mathcal{N}(\mathbf{0}, \mathbf{I})$, 
As depicted in Fig.~\ref{fig: loco_generation_results}, the images generated by PatchDiff effectively avoid the presence of low-level anomalous cues that often occur in simulation strategies, easily noticeable edges when tailoring two images together. 
Instead, PatchDiff focuses more on the slightly higher-level anomaly patterns. By setting multiple receptive filed sizes to the U-Net, PatchDiff can efficiently expose both structural and logical anomalies in MVTec LOCO. This enables PatchDiff to produce more comprehensive local abnormalities without using any prior knowledge of test anomalies. Additionally, the reduction in the depth of architecture and the removal of attention layers contribute to a decrease in the model's complexity and calculation cost, leading to improved training and sampling speed. Furthermore, it is worth noting that the training process of PatchDiff uses only fitting loss like DDPM\cite{DDPM}, which is very stable and easy to implement. In summary, PatchDiff is a prior-less, easy-to-implemented, relatively-fast multilevel local anomaly pattern generation method. 

\begin{figure}[!t]
\centering
    \includegraphics[width=1\linewidth]{images/training_detector.pdf}
    \vspace{-0.6cm}
    \caption{Schematic overview of two components during training patch-level detectors. The left portion is the training set which consists of one type of positive patch and two types of negative patches. The right portion is the reweighting mechanism which comprises mechanism (a) to filter the fake anomaly patterns and mechanism (b) to rebalance the long-tailed training data.}
    \label{fig: detector}
    \vspace{-0.4cm}
\end{figure}

\subsection{Training Patch-level Detector}
\label{sec: training}
A naive idea to utilize the contrastive images generated by PatchDiff is directly labeling them as the anomalous class and training image-level detectors. But it does not fully exploit the dense and local anomaly patterns nor provide useful anomaly scores for localization. Instead, we opt to train patch-level anomaly detectors that detect level-specific local anomalies by patch-wisely classifying the normal images and contrastive images. 
Our patch-level detector is implemented with an 8-layer Fully Convolutional Network, FCN~\cite{FCN}, in a pure encoder way. At the training stage, the detector takes input patches of a fixed size, precisely $34 \times 34$ pixels, and produces an output anomaly score corresponding to each individual patch. To address local anomalies of multiple concerned levels (\eg, both structural and logical anomalies in MVTec LOCO), we choose to maintain the detector architecture but resize the original images into lower resolutions, which indirectly achieves the adjustment of the receptive field sizes. This approach enables us to train additional detectors capable of capturing higher-level anomaly patterns without redesigning the detector's architecture and further reduces the computational cost. In the following, we will introduce how to train the patch-level detector.
% We believe that reducing the resolution will not significantly affect the detection performance for higher-level anomaly patterns. 
% To better elucidate the details of training the detector, in the following sections, we will introduce: 1) How to prepare the training set for detector training. 2) How to reweight the unlabeled and long-tailed contrastive patterns generated by PatchDiff. 3) How to further improve the detector's performance by regularization techniques.

\begin{figure}[!t]
    \centering
    \includegraphics[width=1\linewidth]{images/reweighted_map.pdf}
    \caption{The reweighted map to show the effects of reweighting mechanism. (a) and (b) respectively displays the origin images and the generated contrastive images. (c) and (d) respectively depicts the effects when filtering fake anomaly patterns and rebalancing long-tailed training data. Our reweighting mechanism learns to identify patterns to be disregarded, indicated by the blue regions, and patterns to be emphasized, represented by the red regions, through a self-supervised approach} % Intuitively, at the training begins, the weights of patches are assigned with similar weights. 
    \label{fig:reweightmap}
\end{figure}

\subsubsection{Preparing the Training set} 

Similar to the input during the generation phase, we use a 2-channel coordinate map $F$ as an additional input. As illustrated in the left portion of Fig.~\ref{fig: detector}, we prepare three types of 5-channel patches as training inputs, including one type of positive patch and two types of negative patches. Let $I$ denote an image data, and $\mathcal{I}^{+}$ and $\mathcal{I}^{-}$ represent sets of normal samples and generated samples from PatchDiff, respectively. Subsequently, the positive patches set $\mathcal{C}^{+}$ and negative patches set $\mathcal{C}^{-}$ are defined as
\begin{equation*}
    \begin{aligned}
    \mathcal{C}^{+}=&\left\{c \mid c=\operatorname{RandCrop}(I\oplus F), I \in \mathcal{I}^{+}\right\},\\
    \mathcal{C}^{-}=&\left\{c \mid c=\operatorname{RandCrop}(I)\oplus \operatorname{RandCrop}(F), I \in \mathcal{I}^{+}\right\} \\
    &\cup\left\{c \mid c=\operatorname{RandCrop}(I\oplus F), I \in \mathcal{I}^{-}\right\},
    \end{aligned}
\end{equation*}
where $\oplus$ denotes concatenation along the channel axis. The negative patches are constructed in two ways: (1) the patches from generated samples along with their corresponding coordinate maps, and (2) the patches from normal samples with incorrect coordinate maps. Specifically, the patches from the latter way are believed to provide examples that break the dependence between patch content and position. This explicitly enhances the detector's utilization of the auxiliary information from the coordinate maps and improves its ability to capture position-aware cues.

\subsubsection{Reweighting the Contrastive Patches}
There are two potential challenges during training the patch-level detector $D$: (1) The images generated by PatchDiff are pixel-unlabeled, leading to the presence of fake anomaly patterns (e.g. the background region in the generated images) among the negative patches, which will mislead the detector. (2) Some important anomaly patterns may appear more rarely and lie in the low-density regions of the data manifold, causing the detector to overlook such patterns during the training process. To mitigate these challenges, we propose a feature density-based reweighting mechanism that incorporates two reweighting strategies, as shown in the right part of Fig.~\ref{fig: detector}. This mechanism relies on the feature distributions extracted from the last latent layer of our patch-level detector on both positive and negative samples. Let us denote $\mathcal{M}^+$ and $\mathcal{M}^-$ as the feature sets of positive and negative samples, respectively. Then the two reweighting strategies can be performed as follows:

(1) Filtering the fake anomaly patterns. As depicted in Fig.~\ref{fig: detector}(a), we introduce a reweighting factor $w^\text{noisy-}_i$ for each given negative patch $\bm{c}^-_i$, to assign smaller weights to the patches whose features are too close to or even within normal features set $\mathcal{M}^+$. The reweighting factor can be formulated as
\begin{equation}
    w^\text{noisy-}_i = \frac{1}{\sum_{\bm{z}\in\mathcal{M}^+} \exp(\beta_\text{density} \text{sim}(\bm{z},\bm{z}^-_i))} ,
\end{equation}
where $\bm{z}^-_i$ is the feature vector of the negative patch $\bm{c}^-_i$, $\mathrm{sim}(\bm{z}, \bm{z}'):= \bm{z} \cdot \bm{z}'/\lVert{\bm{z}}\rVert\lVert{\bm{z}'}\rVert$ is the density kernel based on the cosine similarity and $\beta_\text{density}>0$ is a hyper-parameter for controlling kernel bandwidth. 

(2) Rebalancing the long-tailed training patches. As depicted in Fig.~\ref{fig: detector}(b), we introduce a reweighting factor $w^\text{tail-}_i$ for each given negative patch $\bm{c}^-_i$ to downweight the patches whose features are in the high-density regions. Empirically, we find introducing a reweighting factor $w^\text{tail+}_j$ for each positive patch $\bm{c}^+_j$ is also helpful. Therefore we have the following two additional reweighting factors for the training patches
\begin{equation}
\begin{aligned}
    w^\text{tail-}_i = \frac{1}{\sum_{\bm{z}\in\mathcal{M}^-} \exp(\beta_\text{density} \text{sim}(\bm{z},\bm{z}^-_i))},\\
    w^\text{tail+}_j = \frac{1}{\sum_{\bm{z}\in\mathcal{M}^+} \exp(\beta_\text{density} \text{sim}(\bm{z},\bm{z}^+_j))}.
\end{aligned}
\end{equation}

The effects of our reweighting mechanism are shown in Fig~\ref{fig:reweightmap}. By incorporating these two kinds of reweighting factors, our reweighted binary classification loss $\mathcal{L}_{\mathrm{RBCE}}$ can be formulated as
\begin{equation}
\label{eq: loss_function_rbce}
\begin{aligned}
    \mathcal{L}_{\mathrm{RBCE}}&=-\frac{1}{\lambda^+}\sum_{j=1}^{|\mathcal{C}^+|} w^\text{tail+}_j\log (1-f(\bm{c}^+_j))\\
    &\quad -\frac{1}{\lambda^-}\sum_{i=1}^{|\mathcal{C}^-|} w^\text{tail-}_i w^\text{noisy-}_i \log(f(\bm{c}^-_i)),
\end{aligned}
\end{equation}
where $\lambda^+$ and $\lambda^-$ are the normalization constants to keep the total weights of each class equal to 1:
\begin{equation}
\label{eq: regularization parameter}
    \lambda^+ = \sum_{j=1}^{|\mathcal{C}^+|} w^\text{tail+}_j,\quad
    \lambda^- = \sum_{i=1}^{|\mathcal{C}^-|} w^\text{tail-}_i w^\text{noisy-}_i.
\end{equation}
In practice, the $\mathcal{M}^+$ and $\mathcal{M}^-$ are both implemented with a memory bank that store the features of previous training steps in a queue.

\subsubsection{Regularization on Features and Gradients}
We further utilize a classical unsupervised representation learning method named denoising autoencoder ~\cite{DAE} to regularize the learned feature by detector $D$. To achieve that, we introduce a simple MLP-based network $R$ that recovers the original input patches from the feature vectors extracted from the last latent layer of $D$. Let $f_{Z}$ denote the function extracting features from input patches, $f_{R}$ denote the function recovering input patches from features, and $\mathcal{C}$ denote the collection of all training patches $\mathcal{C}^+\!\cup\!\mathcal{C}^-$. The feature regularization loss can be formulated as
\begin{equation}
    \begin{aligned}
	    \mathcal{L}_{\mathrm{feat}}= \frac{1}{|\mathcal{C}|}\sum_{\bm{c}\in \mathcal{C}} \left\|f_{R}(f_{Z}(\bm{c}+\bm{\epsilon_c})+\bm{\epsilon_z}) - \bm{c}\right\|^2,
     \end{aligned}
\end{equation}
where $\bm{\epsilon_c}$ and $\bm{\epsilon_z}$ are respectively the noise perturbations added to the feature layer and the input layer. The auxiliary denoising task regularizes the last hidden layer of the detector to extract informative and robust representations. We highlight the auxiliary network $R$ will be dropped in the inference stage so that will not increase the inference runtime.

Additionally, we propose a gradient regularization loss to smooth the learned decision function $f$, which further discourages the detector from learning imperceptible distinctions between normal patterns and fake anomaly patterns. The gradient regularization loss can be formulated as
\begin{equation}
    \mathcal{L}_{\mathrm{grad}}=\frac{1}{|\mathcal{C}^+|}\sum_{\bm{c}\in \mathcal{C}^+} \left\|\nabla_{\bm{c}} f(\bm{c})\right\|^{2}.
    \label{eq_gp}
\end{equation}
It penalizes the gradient norms of the decision scores with respect to the input data, which is often used to improve the Lipschitz smoothness and robustness, and thus the generalization performance of decision functions~\cite{GradPU, arjovsky2017towards,inputGrad}. 
\subsubsection{The Overall Training Loss}

We calculate the overall training loss for the patch-level anomaly detector by aggregating the aforementioned three types of losses as
\begin{equation}
    \mathcal{L}=\mathcal{L}_{\mathrm{RBCE}} + \alpha_1 \mathcal{L}_{\mathrm{feat}} +\alpha_2 \mathcal{L}_{\mathrm{grad}},
\label{eq:final_loss}
\end{equation}
where $\alpha_1$ and $\alpha_2$ are hyper-parameters to adjust the impact of $\mathcal{L}_{\mathrm{feat}}$ and $\mathcal{L}_{\mathrm{grad}}$. 

\begin{table*}[!ht]
\centering
    \label{tab:mvtec_main}
    \resizebox{0.95\textwidth}{!}{
    \begin{tabular}{c|c|c|c|c|c|c|c|c}
    \toprule
    Category & \makecell[c]{SPADE\\\tiny{\citealp{SPADE}}} & \makecell[c]{PaDiM\\\tiny{\citealp{PaDiM}}} & \makecell[c]{S-T\\\tiny{\citealp{S-T}}} & \makecell[c]{PatchCore\\\tiny{\citealp{PatchCore}}} &\makecell[c]{GCAD\\\tiny{\citealp{MVloco}}} & \makecell[c]{DADF\\\tiny{\citealp{DADF}}} & \makecell[c]{SINBAD\\\tiny{\citealp{SINBAD}}} & \makecell[c]{GRAD\\\tiny{Ours}} \\ 
    \midrule
    breakfast box 
    & 78.2 & 65.7 & 68.6 & 81.3 & 83.9 & 75.3 & \textbf{92.0} & 81.2 \\
    juice bottle 
    & 88.3 & 88.9 & 91.0 & 95.6 & \textbf{99.4} & 98.6 & 94.9 & 97.6 \\
    pushpins 
    & 59.3 & 61.2 & 74.9 & 72.3 & 86.2 & 81.0 & 78.8 & \textbf{99.7} \\
    screw bag 
    & 53.2 & 60.9 & 71.2 & 64.9 & 63.2 & 77.3 & \textbf{85.4} & 76.6 \\
    splicing connectors 
    & 65.4 & 67.8 & 81.1 & 82.4 & 83.9 & 86.4 & \textbf{92.0} & 85.4 \\
    \midrule
    average 
    & 68.8 & 68.9 & 77.3 & 79.3 & 83.3 & 83.7 & 86.8 & \textbf{87.5} \\
    \bottomrule
\end{tabular}}
\caption{Image-level AU-ROC performance for anomaly detection of different methods on MVTec LOCO~\cite{MVloco}. The best results are in bold.}
\label{tab: auc_mvtec_loco}
\end{table*}

\begin{figure*}[!htbp]
\setlength{\belowcaptionskip}{0.0cm}
\setlength{\abovecaptionskip}{0.1cm}
    \centering
\includegraphics[width=0.9\linewidth]{images/loco_results.pdf}
    \caption{Defect localization results of GRAD on MVTec LOCO~\cite{MVloco}. } %on cable, hazelnut, metal nut, screw, grid, carpet The last column shows the failure cases. More examples can be found in Appendix.
    \label{fig: main_results}
\end{figure*}

\section{Experiments}

In this section, we first briefly introduce the experimental details (See Appendix for more details). Then we report the anomaly detection accuracies and the ablation study on each component. % Finally, we show the failure cases of GRAD.

\subsection{Dataset}

To validate the effectiveness and generalizability of our approach, we perform experiments on both MVTec AD~\cite{MVTecAD} and MVTec LOCO~\cite{MVloco}. There are 15 sub-datasets in MVTec AD and 5 sub-datasets in MVTec LOCO and each sub-dataset presents a diverse set of anomalies. 
% Both of them contain high-resolution industry images of multiple categories. The images are collected in realistic scenes and cover both aligned and unaligned objects. 
Particularly, the training sets among them contain only normal images, while the test sets contain both normal and various types of industrial defects. Pixel-level annotations are provided in the test set.

\subsection{Training Settings}

We simply define level-$n$ PatchDiff as the PatchDiff with a receptive field of $n \times n$ pixels, and the images generated by it belong to level-$n$. Similarly, we define level-$n$ detector as the patch-level detector with an indirect receptive field of $n \times n$ pixels.

\noindent\textbf{PatchDiff}. For each sub-dataset in MVTec AD, we train 3 levels of PatchDiffs (level-5, 9, 13). For each sub-dataset in MVTec LOCO, we need to train 3 different levels of detectors, and consequently, we train 4 levels of PatchDiffs (level-5, 9, 13, 17). In particular, 2 of them use level-5, 9, and 13 PatchDiffs and another one uses level-9, 13, and 17 PatchDiffs. For all PatchDiffs, we generally train them for a total of 10,000 training steps. For each sub-dataset, we sample 1,000 images for each level-$n$.

\noindent\textbf{Patch-level Detector}. Each sub-dataset in MVTec AD and MVTec LOCO contains limited training images. To train competitive detectors from scratch for each small sub-dataset, we adopt general data augmentations on both normal and generated images like previous works\cite{MVTecAD, MVloco}. For level-$34, 68,$ and $136$ detectors, the images are respectively resized into $256\times256$, $128\times128$, and $64\times64$. We train the detector on batches of size $128\times(k+2)$ for 2,000 epochs and report the accuracy of the final epoch. Each batch contains 128 randomly cropped positive patches from 4 normal images and $128\times(k+1)$ negative patches from 4 normal images and $4k$ contrastive images, where $k$ equals the number of levels of used generated contrastive images. Specifically, we use $k=3$ for all experiments as mentioned before.

\subsection{Evaluation Settings}
The image-level anomaly score directly takes the max value of a score map from the patch-based anomaly detector, and the pixel-level detection result is obtained by up-sampling the score map and then applying a Gaussian blur with a kernel size of 16. Consistent with existing methods~\cite{MVTecAD, MVloco}, we use AU-ROC as the evaluation metric for the evaluation of image-level anomaly detection and pixel-level anomaly localization.

\begin{table}[htbp]
% \footnotesize
\centering
\setlength{\belowcaptionskip}{0.0cm}
\setlength{\abovecaptionskip}{0.1cm}
\resizebox{0.9\linewidth}{!}{
% \setlength{\tabcolsep}{2pt}{
\begin{tabular}{lcc}
\toprule
Method & \makecell[c]{Pixel-level\\AU-ROC} & \makecell[c]{Image-level\\AU-ROC} \\ 
\midrule
IGD\tiny{~\citep{IGD}} & 93.1 & 93.4 \\
PSVDD\tiny{~\citep{PSVDD}} & 92.5 & 93.2 \\
FCDD\tiny{~\citep{FCDD}} & 92.1 & 95.7 \\
CutPaste\tiny{~\citep{CutPaste}} & 95.2 & 96.0 \\
NSA\tiny{~\citep{NSA}} & 96.3 & 97.2 \\
DRAEM\tiny{~\citep{DRAEM}} & \textbf{97.3} & 98.0 \\
DSR\tiny{~\citep{DSR}} & - & 98.2 \\
\rowcolor{gray!20} GRAD \tiny{(Ours)} & 96.8 & \textbf{98.7} \\ 
\bottomrule
\end{tabular}}
\caption{Anomaly detection performance on MVTec AD dataset~\cite{MVTecAD}. The best results are in bold.}
\label{tab: auc_mvtec}
\end{table}

\subsection{Main Results}

\noindent\textbf{The anomaly detection results.} We compare GRAD with different methods on MVTec AD and MVTec LOCO, as shown in Table~\ref{tab: auc_mvtec_loco} and Table~\ref{tab: auc_mvtec}. For both datasets, GAD has the best average image-level AU-ROC score, demonstrating the effectiveness of GRAD in anomaly detection. In table~\ref{tab: auc_mvtec_loco}, it is important to note that the fairness of the comparison might be compromised to some extent, as all the compared methods utilize ImageNet pretrained feature extractors. However, GRAD still achieves superior performance by 0.7\% even without such advantages, which shows that ImageNet pretrained features inadequately address the intricacies of logical anomaly detection within MVTec LOCO, and further demonstrates that our contrastive images generated by PatchDiff do expose both structural and logical anomalies effectively. In particular, GRAD achieves excellent results (+13.5\%) on the sub-dataset of pushpins, which exactly fits our observation that the generated images for pushpins perfectly expose several abnormal logical situations in the testing set, \eg, the additional pushpin in the top left compartment and no pushpins in the top right compartment, as shown in level-17 generated pushpin image of Fig.~\ref{fig: loco_generation_results}. In addition, we show the defect localization results in Fig.~\ref{fig: main_results}. In table~\ref{tab: auc_mvtec}, all the methods we compared do not rely on pretrained features and external data. Although GRAD does not achieve the best result for anomaly localization (pixel-level AUROC), it is still competitive among them.


\noindent\textbf{Inference runtimes.} We compare and report the inference latency and FPS in Table~\ref{tab:GRad_speed}. Obviously, GRAD achieves a remarkable throughput performance due to its extremely lightweight architecture, and thereby, GRAD's inference speed is more than 16 times faster than GCAD's.
    
\begin{table}[!htbp]
\centering
\setlength{\belowcaptionskip}{0.0cm}
\setlength{\abovecaptionskip}{0.1cm}
\resizebox{0.9\linewidth}{!}{
\begin{tabular}{lrr}
    \toprule
    Method & latency (ms$\downarrow$) & FPS$\uparrow$ \\
    \midrule
    S-T\tiny{~\cite{S-T}} & 82.2 & 12.2 \\
    FastFlow\tiny{~\cite{FastFlow}} & 26.1 & 38.3 \\
    DSR\tiny{~\cite{DSR}} & 24.8 &40.3 \\
    GCAD\tiny{~\cite{MVloco}} & 12.9 & 77.5 \\
    PatchCore\tiny{~\cite{PatchCore}} &47.1 &21.2 \\
    \rowcolor{gray!20} GRAD \tiny{(Ours)} & \textbf{0.799} & \textbf{1251.6}\\
    \bottomrule
\end{tabular}}
\caption{Inference speed on NVIDIA Tesla V100. The data of our method is obtained on MVTec LOCO dataset with three patch-level detectors (patch size: 34, input size: 256, 128, and 64).}
\label{tab:GRad_speed}
\end{table}

\begin{table}[!htbp]
\centering
\setlength{\belowcaptionskip}{0.0cm}
\setlength{\abovecaptionskip}{0.1cm}
\resizebox{\linewidth}{!}{
\begin{tabular}{lccc}
\toprule
& & \footnotesize{AUROC} & \\ 
% \cmidrule{2-4}
& Level-34 & Level-68 & Level-136 \\
\midrule
baseline\dag & 78.2 & 77.8 & 64.3\\
+ Regularization & 81.6 & 80.9 & 65.2\\
+ Noisy Reweighting & 82.5 & 82.5 & 72.1 \\
+ Long-tail Reweighting & 85.2 & 85.4 & 75.1\\
\bottomrule
\end{tabular}}
\caption{Ablation study on components. Detection AUROC results on MVTec LOCO dataset of three patch-level detectors are presented. \dag The baseline setting uses no regularization techniques and reweighting strategies.}
\label{tab: ablation_component}
\end{table}

% \begin{table}[!htbp]
% \centering
% \setlength{\belowcaptionskip}{0.0cm}
% \setlength{\abovecaptionskip}{0.1cm}
% % \begin{tabular}{ccc|c}
% % \toprule
% % \multicolumn{3}{c|}{Configurations}& Detection  \\
% % L136 & L68 & L34  & AUROC \\
% % \hline
% % \hline
% % \checkmark &   &    &  unkown   \\
% % & \checkmark &  & 85.4 \\
% % & & \checkmark & 85.2\\
% % \midrule
% % \checkmark & \checkmark &      &  unkown     \\ %
% % \checkmark &    & \checkmark   &  unkown \\
% %  & \checkmark & \checkmark  & unkown \\
% %  \midrule
% % \checkmark & \checkmark & \checkmark &  $\textbf{unkown}$  \\
% % \bottomrule
% % \end{tabular}
% \begin{tabular}{l|ccccccc}
% \toprule
% Level-136 & \checkmark & & & \checkmark & \checkmark & & \checkmark \\ %\cline{3-3} \cline{6-6} \cline{9-9} 
% Level-68 & & \checkmark & & \checkmark & & \checkmark & \checkmark \\ %\cline{3-3} \cline{6-6} \cline{9-9} 
% Level-34 & & & \checkmark & & \checkmark & \checkmark & \checkmark \\ %\cline{1-1} \cline{3-3} \cline{6-6} \cline{9-9} 
%   \midrule 
%   AUROC & - & 85.4 & 85.2 & - & - & - & 88.2\\ %\cline{9-9} 
% \bottomrule
% \end{tabular}
% \caption{Ablation study about detector levels on MVTec LOCO dataset.}
% \label{tab: ablation_GRad_level}
% \end{table}

\subsection{Ablation study}

We first perform an extensive ablation study to validate the effectiveness of two reweighting factors and the regularization technique on MVTec LOCO. The results are shown in Table~\ref{tab: ablation_component}. More details and comprehensive ablation results can be found in Appendix. We utilize the baseline as the beginning and then add regularization, noisy reweighting and long-tail reweighting one by one. 
% This allows us easily to report that the regularization techniques help detectors learn better representations and smooth decision boundaries, and further show that the fake anomaly patches and long-tail distribution do interfere with the performance of the detectors, and it is clear that our reweighting mechanism does effectively mitigate them from these two aspects. 

\textbf{Effects of regularization techniques.} One of the novel contributions presented in this paper is the regularization on features and gradients, which helps our encoder-based detector extract an informative and robust representation and build a smooth decision boundary for the data manifold. As demonstrated in Table~\ref{tab: ablation_component}, the integration of these techniques translates into improvements of +3.4/+3.1/+0.9 on the MVTec LOCO dataset.

\textbf{The effect of reweighting mechanism.} Our reweighting mechanism comprises two essential components: (1) noisy reweighting, which aims to filter fake anomaly patches, and (2) long-tail reweighting, designed to rectify the imbalanced distribution of input data. When integrating the noisy reweighting, our detectors display enhancements of +0.9/+1.6/+6.9 on the MVTec LOCO dataset, as presented in Table~\ref{tab: ablation_component}. Furthermore, with the incorporation of long-tail reweighting, our detectors achieve improvements of +2.7/+2.9/+3.0, as shown in the same table. These outcomes underscore the disruptive influence of fake anomaly patches and the presence of long-tail distributions on detector performance. It is evident that our reweighting mechanism adeptly mitigates these challenges from both fronts, offering substantial advantages to our detectors.

Particularly, in Table~\ref{tab: ablation_component}, Level-136 detectors exhibit relatively poorer performance in anomaly detection. This result can be attributed to their input size, which is merely $64 \times 64$, resulting in insufficient resolution to offer informative structural anomaly details. However, this is in line with our intentions, as the purpose of Level-136 detectors is not to emphasize minute details, but rather to capture the logical relationships among components within the receptive fields of size $136 \times 136$.
    
% \subsection{Failure Cases}

% We show the failure cases at generation stage in Fig~\ref{fig:failure_case_generation} and the failure cases at detection stage in last coloums of Fig~\ref{fig:main_results}. 

\section{Conclusion}
In this paper, we propose a novel unsupervised anomaly detection framework, GRAD, by generating and reweighting dense contrastive patterns. The proposed generation method PatchDiff is able to generate multilevel contrastive patterns which exposes a range of local anomaly patterns. The proposed reweighting strategies fully utilize the unlabeled and long-tailed contrastive patterns and help the patch-level anomaly detector better learn the exposed local anomaly patterns. GRAD requires no scenario-specific prior, external datasets, or heavy pretrained feature extractor. It achieves competitive anomaly detection and localization accuracy with a superior inference speed. 

Further refinements to GRAD can be explored: 1) Advancing the algorithms for handling long-tailed and noisy labeled data, thereby utilizing the potential of the generated dense contrastive patterns more effectively. 2) Investigating the feasibility of integrating multiple patch-level detectors into a single lightweight network. 3) Exploring better network architecture and training settings for PatchDiff.

\section{Acknowledgements}

This work is supported in part by Shanghai science and technology committee under grant No. 21511100600.  We appreciate the High Performance Computing Center of Shanghai University, and Shanghai Engineering Research Center of Intelligent Computing System for providing the computing resources and technical support. Additionally, we extend our heartfelt appreciation to Professor Jincheng Jin and Weizhong Zhang from Fudan University, as well as Associate Professor Bin Li, for their invaluable assistance and insights during the writing process of this paper. Their expert guidance and stead- fast support were instrumental in the successful completion of this research.


\bibliography{aaai24}
% \bibliographystyle{aaai24}

\onecolumn{\pdfpagewidth=8.5in
\pdfpageheight=11in
% The file ijcai21.sty is NOT the same than previous years'
\usepackage{ijcai21}

% Use the postscript times font!
\usepackage{times}
\usepackage{soul}
\usepackage{url}
\usepackage[hidelinks]{hyperref}
\usepackage[utf8]{inputenc}
\usepackage[small]{caption}
\usepackage{graphicx}
\usepackage{helvet}
\usepackage{amsthm}
\usepackage{booktabs}
\usepackage{algorithm}
\usepackage{algorithmic}
\urlstyle{same}
%
% PDF Info Is REQUIRED.
% For /Author, add all authors within the parentheses,
% separated by commas. No accents or commands.
% For /Title, add Title in Mixed Case.
% No accents or commands. Retain the parentheses.
\usepackage{amsmath}
\usepackage{amsthm}
\usepackage{booktabs}
\usepackage{algorithm}
\usepackage{algorithmic}
\usepackage{subcaption}
\usepackage[dvipsnames]{xcolor}
\usepackage{latexsym}
\newcommand{\ra}[1]{\renewcommand{\arraystretch}{#1}}
\newtheorem{theorem}{Theorem}
\newtheorem{proposition}{Proposition}
\newtheorem{conjecture}{Conjecture}
\newtheorem{definition}{Definition}
\newtheorem{lemma}{Lemma}
\def\tuple#1{( #1 )}




\begin{document}
\maketitle

\clearpage
\section*{Appendix}




\subsection*{Supplementary proofs}
In the appendix we use the term \emph{LB- (UB)-unsafe} to refer to a manipulation that lowers the lower (upper) bound of the manipulator. Also, recall that a \emph{solution} refers to any CS that satisfies the objective.

\subsection*{Proof of Theorem~\ref{thrm:egal_dir_add}}
Let 
$
u_0=\underset{P\in O(G)}{\min}(\{u(m,P)\}),
u_1=\underset{P\in O(G)}{\max}(\{u(m,P)\})
$.
We will refer to the CS yielding $u_0$ as $P_0$. Note that for every $P\in O(G)$ it holds that $Eg(P,G) \leq u_0 $
Assume by contradiction that Max-Egal is subject to UB-improvement.
That is, there exists a manipulation $r_{m}$ and a CS $P^m\in O(G^{m})$ such that $u(m,P^m) > u_1$. That is, $P^m\notin O(G)$.

Since the manipulator can only add edges it holds that $Eg(P^{m},G^m)\geq Eg(P_0,G)$. Moreover, if $Eg(P^{m},G^m)=Eg(P_0,G)$ then $P_{0}\in O(G^{m})$, which is not possible. Therefore, $Eg(P^{m},G^m)>Eg(P_0,G)$. Recall that in directed networks the utility of the other agents does not change. Therefore $\forall a \in A \setminus \{m\}$, \[
u(a,P^{m})  \geq Eg(P^{m},G^m) > Eg(P_0,G).
\]
In addition, $u(m,P^{m})>u_1\geq Eg(P_0,G)$. 
Overall, $\forall a \in A, u(a,P^{m}) \geq Eg(P_0,G)$. That is, $Eg(P^{m},G) \geq Eg(P_0,G)$, and thus $P^m\in O(G)$, which is a contradiction.


Now, assume by contradiction that Max-Egal is subject to LB-improvement.
That is, there exists a manipulation $r_{m}$ such that
\begin{equation} \label{ineq:egal_dir}
\forall \  P \in O(G^m), u(m,P) > u_0.
\end{equation} That is $P_0\notin O(G^m)$.
Denote an arbitrary CS in $O(G^m)$ as $P^m$. 
It holds that $Eg(P^m,G^m)>Eg(P_0,G^m)$ and $Eg(P_0,G)\geq E(P^m,G)$.

Again, in directed networks the utility of the other agents does not change. Therefore if after the manipulation $Eg(P^m,G^m)>Eg(P^m,G)$ it can only change by the utility of $m$. But $u(a,P^m)>u(a,P)$, hence even before the manipulation $Eg(P^m,G^m)>Eg(P^m,G)$, in contradiction.



\subsection*{Proof of Theorem~\ref{thm:all_distance1}}

\subsubsection*{Max-Util}
If $0<|N(m)|<n-1$, by looking at possible networks like the ones in Figures \ref{fig:distance1_util_unsafe_add_LB},\ref{fig:distance1_util_unsafe_add_UB} we can see that adding any edge is LB- and UB-unsafe respectively.
These examples can be tweaked to fit any number of agents and any number of desired agents the manipulator has.
If $|N(m)|=n-1$ the manipulator cannot add edges.


\subsubsection*{Max-Egal}
Obviously adding or removing in undirected networks is both LB- and UB-unsafe. Adding an edge towards agent $a_0$ can lead to utility $0$ when the maximum egalitarian SW is $1$, if $m$ ends up in a coalition alone with $a_0$.
Removing an edge can lead to utility $0$. If one $m$'s neighbours has no other agents, removing that edge results in maximum egalitarian SW $0$, hence the lower bound is also $0$. So both add and remove are LB-unsafe.
For the cases of $|N(m)|<n-3, |N(m)|=n-2$ and $|N(m)|=n-1$ see Figures~\ref{fig:egalremove1},\ref{fig:egalremove2} and \ref{fig:egalremove3} for proofs that removing is UB-unsafe, respectively. If $m$ has more neighbours can just add nodes on the top, and more non-neighbours in bottom of the figures.
Figure~\ref{fig:egalremove4} provides proof that adding UB-unsafe. For higher numbers both neighbours and non-neighbours are added at the bottom. Note that this example requires at least neighbours ($|N(m)|\geq2$. However, if $m$ has only one neighbour it is easy to see no manipulation is beneficial against Max-Egal.


\subsubsection*{At-Least-1}
Adding any edge to another agent $a_0$ might result in a LB- of $0$. That is because the CS where $a_0$ and $m$ form one coalition could be a solution now.
Any partial network in distance 1 can have a possible network with a solution where $m$'s original LB- is higher than $0$, hence adding is LB-unsafe.

% In the case with the equal size constraint, the same could be said. However, in this scenario the difference would be that $m$'s LB is not necessarily $0$. If $|N(m)| < n/2$, it could be lower to $0$.
% If $n/2 \leq |N(m)|$ neighbours, there is a possible network where $m$ is guaranteed utility of $n/2 -1 $. Therefore, adding the edge comes at the cost of a true friend, lowering the LB to $n/2-2$.


For At-Least-1, all that is left to prove is that against manipulator $m^-$ the objective is LB-proof. Of course, over undirected networks it is true, as any removal might lead to an infeasible instance.

Over directed networks, it is also easy to see the any removal may lead to an infeasible instance; If $m$ has at least one neighbour $a_0$ for which both $(a_0,m),(m,a_0)\in E $, it could be the case that the only solution is where $m$ and $a_0$ form a coalition alone. (If all of the other agents form a directed circle, and have no other edges but ones going to or from $m$). In that case removing is LB-unsafe.
If $m$ does not have a neighbour like that, the same example still applies when the only solution is $m$ alongside one of his out-neighbours and another agent. i.e. $a_0,a_1$ where $(m,a_0),(a_0,a_1),(a_1,a_0)\in E$.


\clearpage
\subsection*{Equal Size}
An important constraint that arises in many real world scenarios is that all of the coalitions should be of the same size. For example, when dividing students into classes it is common to ask that all of the classes consist of roughly the same number of students.
Therefore we also consider the setting in which all of the coalitions are of the size $n/k$, if $n/k$ is an integer. Otherwise, some coalitions are of size $\left \lceil{n/k}\right \rceil$ and some are of size $\left \lfloor{n/k}\right \rfloor$. We call this restriction the Equal Size constraint, and through the paper we analyze the different objectives with or without it.

Due to space constraints, the definition and results of the equal size constraints have been moved to the appendix. All of the resistance results from Section~\ref{sec:full_info} still stand with the equal size constraint. The susceptibility results are summarized in Table~\ref{tbl:distnce2_equalsize}.




\begin{table}[htp]

\ra{1}
\begin{tabular}{@{}lcccr@{}}\toprule 
$Objective$ & $M$ & $Network$ & $Type$ & $Figure$\\\midrule
Max-Util & A & Both & Strict & \ref{fig:util_size_add}\\
Max-Util & R & Both & Strict & \ref{fig:Util_size_remove}\\
Max-Egal & A & Undirected & LB & \ref{fig:Egal_size_undirected_add_LB}\\
Max-Egal & A & Undirected & UB & \ref{fig:Egal_size_undirected_add_UB}\\
At-Least-1 & A & Undirected & Strict & \ref{fig:Least1_size_undirected_add}\\
At-Least-1 & R & Both & LB & \ref{fig:least1_size_remove}\\
\bottomrule
\end{tabular}
\caption{Summary of susceptibility results for distance 2 with the equal size constraint.
Key: A = add, R = Remove, LB/UB/Strict = the objective is subject to LB/UB/Strict-improvement.
}
\label{tbl:distnce2_equalsize}
\end{table}

\begin{proposition}
\label{thm:util_distance1}
Max-Util with the equal size constraint is subject to 1-safe LB- and UB-improvement and is 1-safe weak-proof for a manipulator $m^-$ over directed and undirected networks. 
\end{proposition}

\begin{proof}
See Figures~\ref{fig:distance1_util/egal_lb} and \ref{fig:distance1_util_ub} for the 1-safe LB-improvement and 1-safe UB-improvement, respectively. We move on to prove it is 1-safe weak-proof.

First, we prove that if the manipulator has less than $n-1$ neighbours, no UB-safe manipulation exists.
If $0<|N(m)|\leq \lfloor n/2 \rfloor$ then Figures~\ref{fig:thrm5_1} and \ref{fig:thrm5_2} provide examples for UB-unsafe scenarios for even and odd $n$, respectively.
If $\lfloor n/2 \rfloor<|N(m)| < n/2$ then Figures~\ref{fig:thrm5_3} and \ref{fig:thrm5_4} provide examples for UB-unsafe scenarios for even and odd $n$, respectively.

If $|N(m)| = n-1$, then if $n$ is even, the manipulator always gets the same utility and no manipulation is possible.
If $n$ is odd we have shown it is subject to LB and UB 1-safe improvement.
If the manipulator removes more than 1 edge, Figure~\ref{fig:thrm5_5} provides an example that this is UB-unsafe.

By removing only 1 edge, the manipulation cannot be weak-improvement. 
Denote the utilitarian SW of coalition structure $P$ in $G$ as $Ut(P,G)$.
Let 
$
u_0=\underset{P\in O(G)}{\min}(\{u(m,P)\}),
u_1=\underset{P\in O(G)}{\max}(\{u(m,P)\})
$.
We will refer to the CS yielding $u_0$ as $P_0$.
Assume by contradiction that Max-Util is subject to weak-improvement when removing only 1 edge.
That is, there exists a manipulation $r_{m}$ and a CS $P^m\in O(G^{m})$ such that $u(m,P^m) > u_1$. That is, $P^m\notin O(G)$. Also, $P_0\notin O(G^m)$.

Therefore, we can see that $Ut(P_0,G) > Ut(P^m,G)$ and $Ut(P^m,G^m)>Ut(P_0,G^m)$.
Since the manipulator is only able to remove edges it holds that $Ut(P^m,G^m)\leq Ut(P^m,G)$. So we get that $Ut(P_0,G) > Ut(P^m,G^m) \xrightarrow{} Ut(P_0,G) - 1 \geq Ut(P^m,G^m)$
Since only 1 edge was removed, the utilitarian SW could drop at most by $1$ and it holds that $Ut(P^0,G) - 1 \leq Ut(P^0,G^m)$.
Combining the last two inequalities we get $ Ut(P_0,G^m) \geq Ut(P_m,G^m) $ in contradiction.
\end{proof}


\begin{proposition}
\label{thm:Egal_distance1}
Max-Egal with the equal size constraint is subject to 1-safe LB-improvement and is 1-safe UB-proof for a manipulator $m^-$ over directed networks.
\end{proposition}

\begin{proof}
See Figure~\ref{fig:distance1_util/egal_lb} for the 1-safe LB-improvement.

For the cases of $|N(m)|<n-3, |N(m)|=n-2$ and $|N(m)|=n-1$ Figures~\ref{fig:egalremove1},\ref{fig:egalremove2} and \ref{fig:egalremove3} show the that removing any edge is UB-unsafe, respectively.
\end{proof}

\begin{proposition}
\label{thm:least1_distance1_es}
At-Least-1 with the equal size constraint is subject to 1-safe UB-improvement and is 1-safe LB-proof for a manipulator $m^+$ over undirected networks.
\end{proposition}

\begin{proof}
See Figure~\ref{fig:distance1_least1} for the 1-safe UB-improvement.
The resistance proof is the same as without the equal size constraint.
\end{proof}


\begin{theorem}
\label{thm:all_distance1_es}
Except for the situations in Theorems~\ref{thm:util_distance1},\ref{thm:Egal_distance1} and \ref{thm:least1_distance1_es}, all of our objectives with the equal size constraint are 1-safe strategyproof. 
\end{theorem}
Here we separate the proof between the objectives as well:
\subsubsection*{Max-Util}
To show the manipulation is LB-unsafe;
If $0<|N(m)|\leq \lfloor n/2 \rfloor$ then Figures~\ref{fig:unsafeutil3} and \ref{fig:unsafeutil4} provide examples for even and odd $n$, respectively.
If $\lfloor n/2 \rfloor<|N(m)| < n/2$ then Figures~\ref{fig:unsafeutil7} and \ref{fig:unsafeutil8} provide examples for even and odd $n$, respectively.

To show the manipulation is UB-unsafe;
If $0<|N(m)|\leq \lfloor n/2 \rfloor$ then Figures~\ref{fig:unsafeutil1} and \ref{fig:unsafeutil2} provide examples for even and odd $n$, respectively.
If $\lfloor n/2 \rfloor<|N(m)| < n/2$ then Figures~\ref{fig:unsafeutil5} and \ref{fig:unsafeutil6} provide examples for even and odd $n$, respectively.


\subsubsection*{Max-Egal}
Manipulator $m^-$:
If $|N(m)| \leq n/2 -1$, a possible network which is an $n-1$ clique except for $m$ yields an egalitarian SW of exactly $|N(m)|$. Removing any of her neighbours is LB-unsafe as she is guaranteed to get all of the neighbours she reports.
Now look at a possible network where $|N(m)|-1$ neighbours of $m$ are connected only to her and all non-neighbours of $m$ are connected to another node $a_0$. Out of the non-neighbours, $n/2 -2 $ nodes have no more edges. The others are neighbours to neighbours of $m$. The last neighbour of $m$ is connected also to $a_0$. Of course, the maximum possible egalitarian SW is $1$, and the two possible outcomes for $m$ are either getting all the neighbours, or not getting the single neighbours which is connected to $a_0$. By removing an edge it might be the case that this single neighbour is the one being disconnected, hence removing an edge is UB-unsafe.
If $n/2 -1 < |N(m) < n-1$, look at a possible network where $n/2-1$ of neighbours of $m$ form a path $a_1, a_2, .. ,a_{n/2-1}$. All the other nodes form a clique of size $n/2$ and one of them $a_0$ is connected to $a_1$. The highest egalitarian SW achievable is $2$, by putting $m$ alongside the path. By removing the edge towards $a_1$ or $a_2$ the highest egalitarian SW possible is now 1, and it can also be achieved by swapping $m$ and $a_0$. Hence removing is LB-unsafe.
In a possible network where the $n/2-1$ neighbours are connected only to $m$ and $a_0$ (but not between themselves) the highest egalitarian SW possible is $1$, either by $m$ or $a_0$ being with the $n/2-1$ neighbours. By removing an edge towards one of the neighbours, $m$ is guaranteed to not be with them, but in the other coalition with only one neighbour, hence removing is UB-unsafe.
If $|N(m)=n-1$, if $n$ is even no manipulation is beneficial as the manipulator always gets her highest utility. 
If $n$ is odd, the possible network could be made of two cliques $L_1$ and $L_2$ of sizes $n-1/2$ and $n-1/2 -1$, and $L_1$ is missing 1 edge between two nodes $a_1, a_2$. The last node $a_0$ is connected to the whole network just like $m$. The two possible CSs yielding an egalitarian SW of $n-1/2 -1$ are where $L_1$ form a coalition alongside either $m$ or $a_0$ and $L_2$ is with the other. Hence the upper bound for $m$ is $n-1/2$. By removing an edge towards either $a_1$ or $a_2$, $m$ is guaranteed to be with $L_2$, so removing is UB-unsafe.

Manipulator $m^+$:
If $|N(m)| \leq n/2 -1 $, look at a network that is made of a star of size $n/2$ that includes all of $m$'s neighbour, another star of size $n/2-2$, and a node $a_0$ with no edges. Adding an edge towards $a_0$ is guaranteeing utility $0$ for the manipulator, as him being  being with $a_0$ and the smaller star is the only CS yielding egalitarian SW of $1$. Hence it is both LB and UB-unsafe.
If $n/2-1 < |N(m)| < n-1$, it is possible the graph is a clique guaranteeing $m$ a full coalition of neighbours. By adding any edge this might make some of the neighbours fake, therefore UB-unsafe. Lastly, look at a graph where $n/2 -1$ neighbours of $m$ form a circle $C_m$. Also, another $n/2-1$ nodes form a circle $C_0$, and call the last node $a_0$. $a_0$ is connected to two nodes $C_0$ and to one in $C_m$. Also, there is a node $a_1$ in $C_m$ that is connected to two nodes in $C_0$. Adding an edge to $a_0$ is LB-unsafe, as now a CS where $m$ is with $a_0$ instead of $a_1$ is a solution.

\subsubsection*{At-Leas-1}
If $|N(m)|<\lfloor n/2 \rfloor$, removing can still lead to an infeasible instance. If $|N(m)|<\lfloor n/2 \rfloor$ then removing edges while still having more than $\lfloor n/2 \rfloor$ does not change the outcome, as any previous solution is still satisfying At-Least-1. This is because the manipulator is still guaranteed at least $1$ out neighbour in every CS, and the other agents' utility has not changed. Removing edge until having less than $\lfloor n/2 \rfloor$ can again result in infeasible instance.

\begin{figure*}[t]
    \centering
    \begin{subfigure}{0.6\textwidth}
    \centering  
        \begin{subfigure}{0.15\textwidth}
        \renewcommand\thesubfigure{\alph{subfigure}1}
            \centering
        \includegraphics[page=5,width=\textwidth]{Graphs/graphs.pdf}
        \caption{\\U/S/ES}
        \label{fig:util_size_add}
        \end{subfigure}
        \hfill
        \begin{subfigure}{0.15\textwidth}
            \addtocounter{subfigure}{-1}
            \renewcommand\thesubfigure{\alph{subfigure}2}
            \centering
        \includegraphics[page=23,width=\textwidth]{Graphs/graphs.pdf}
        \caption{\\E/LB/ES}
        \label{fig:Egal_size_undirected_add_LB}
        \end{subfigure}
        \hfill
        \begin{subfigure}{0.15\textwidth}
            \addtocounter{subfigure}{-1}
            \renewcommand\thesubfigure{\alph{subfigure}3}
            \centering
        \includegraphics[page=25,width=\textwidth]{Graphs/graphs.pdf}
        \caption{\\E/UB/ES}
        \label{fig:Egal_size_undirected_add_UB}
        \end{subfigure}
        \hfill
        \begin{subfigure}{0.15\textwidth}
                    \addtocounter{subfigure}{-1}
            \renewcommand\thesubfigure{\alph{subfigure}4}
            \centering
        \includegraphics[page=33,width=\textwidth]{Graphs/graphs.pdf}
        \caption{\\1/S/ES}
        \label{fig:Least1_size_undirected_add}
        \end{subfigure}
    \addtocounter{subfigure}{-1}
    \caption{$m^+$}
    \label{fig:add_es_subfig} 
    \end{subfigure}
    \hfill
    \begin{subfigure}{0.3\textwidth}
    \centering  
        \begin{subfigure}{0.3\textwidth}
            \renewcommand\thesubfigure{\alph{subfigure}1}
            \centering
        \includegraphics[page=45,width=\textwidth]{Graphs/graphs.pdf}
        \caption{\\U/S/ES}
        \label{fig:Util_size_remove}
        \end{subfigure}    
        \hfill
        \begin{subfigure}{0.3\textwidth}
        \addtocounter{subfigure}{-1}
        \renewcommand\thesubfigure{\alph{subfigure}2}
            \centering
        \includegraphics[page=21,width=\textwidth]{Graphs/graphs.pdf}
        \caption{\\E/S/ES}
        \label{fig:Egal_size_remove}
        \end{subfigure}
        \hfill
        \begin{subfigure}{0.3\textwidth}
                    \addtocounter{subfigure}{-1}
        \renewcommand\thesubfigure{\alph{subfigure}3}
        \centering
        \includegraphics[page=27,width=\textwidth]{Graphs/graphs.pdf}
        \caption{\\1/LB/ES}
        \label{fig:least1_size_remove}
        \end{subfigure}    
        \addtocounter{subfigure}{-1}
    \caption{$m^-$}
    \label{fig:remove_es_subfig} 
    \end{subfigure}
%    \hfill
%    \begin{subfigure}{0.1\textwidth}

%        \renewcommand\thesubfigure{\alph{subfigure}1}
%    \centering
%        \begin{subfigure}{0.4\textwidth}
%        \centering
%        \includegraphics[page=9,width=\textwidth]{Graphs/Distance 1 manipulation.pdf}
%        \caption{\\E/UB}
%        \label{fig:distance1_egal_ub_any}
%    \end{subfigure} \hfill
%        \begin{subfigure}{0.4\textwidth}
%        \addtocounter{subfigure}{-1}
%        \renewcommand\thesubfigure{\alph{subfigure}2}
%        \centering
%        \includegraphics[page=7,width=\textwidth]{Graphs/Distance 1 manipulation.pdf}
%        \caption{\\1/UB}
%        \label{fig:distance1_least1}
%    \end{subfigure}
%    \addtocounter{subfigure}{-1}
%    \caption{Distance 1}
    %%graphs showing manipulation of different objectives by adding edges. Dotted edges are added by the manipulator.
%    \label{fig:distance1_manipulation}
%    \end{subfigure}
    
    
    \caption{Figures providing examples of susceptibility to manipulation with the equal size constraint. Key: U=Max-Util, E=Max-Egal, 1=At-least-1, S=strict-improvement, UB=UB-improvement, LB=LB-improvement}
    %graphs showing manipulation of different objectives by adding edges. Dotted edges are added by the manipulator.
    \label{fig:es_graphs}
\end{figure*}

\begin{figure}
    \centering
    \begin{subfigure}{0.07\textwidth}
        \centering
        \includegraphics[page=1,width=\textwidth]{Graphs/Distance 1 special cases.pdf}
        \caption{R,UB}
        \label{fig:thrm5_1}
    \end{subfigure}
    \hfill
    \begin{subfigure}{0.07\textwidth}
        \centering
        \includegraphics[page=2,width=\textwidth]{Graphs/Distance 1 special cases.pdf}
        \caption{R,UB}
        \label{fig:thrm5_2}
    \end{subfigure}
    \hfill
    \begin{subfigure}{0.07\textwidth}
        \centering
        \includegraphics[page=3,width=\textwidth]{Graphs/Distance 1 special cases.pdf}
        \caption{R,UB}
        \label{fig:thrm5_3}
    \end{subfigure}
    \hfill
    \begin{subfigure}{0.07\textwidth}
        \centering
        \includegraphics[page=4,width=\textwidth]{Graphs/Distance 1 special cases.pdf}
        \caption{R,UB}
        \label{fig:thrm5_4}
    \end{subfigure}
    \hfill
    \begin{subfigure}{0.07\textwidth}
        \centering
        \includegraphics[page=5,width=\textwidth]{Graphs/Distance 1 special cases.pdf}
        \caption{R,UB}
        \label{fig:thrm5_5}
    \end{subfigure}
    \\
    \begin{subfigure}{0.07\textwidth}
        \centering
        \includegraphics[page=6,width=\textwidth]{Graphs/Distance 1 special cases.pdf}
        \caption{A,UB}
        \label{fig:unsafeutil1}
    \end{subfigure}
    \hfill
    \begin{subfigure}{0.07\textwidth}
        \centering
        \includegraphics[page=7,width=\textwidth]{Graphs/Distance 1 special cases.pdf}
        \caption{A,UB}
        \label{fig:unsafeutil2}
    \end{subfigure}
    \hfill
    \begin{subfigure}{0.07\textwidth}
        \centering
        \includegraphics[page=8,width=\textwidth]{Graphs/Distance 1 special cases.pdf}
        \caption{A,LB}
        \label{fig:unsafeutil3}
    \end{subfigure}
    \hfill
    \begin{subfigure}{0.07\textwidth}
        \centering
        \includegraphics[page=9,width=\textwidth]{Graphs/Distance 1 special cases.pdf}
        \caption{A,LB}
        \label{fig:unsafeutil4}
    \end{subfigure}
    \\
    \begin{subfigure}{0.07\textwidth}
        \centering
        \includegraphics[page=10,width=\textwidth]{Graphs/Distance 1 special cases.pdf}
        \caption{A,UB}
        \label{fig:unsafeutil5}
    \end{subfigure}
    \hfill
    \begin{subfigure}{0.07\textwidth}
        \centering
        \includegraphics[page=11,width=\textwidth]{Graphs/Distance 1 special cases.pdf}
        \caption{A,UB}
        \label{fig:unsafeutil6}
    \end{subfigure}
    \hfill
    \begin{subfigure}{0.07\textwidth}
        \centering
        \includegraphics[page=12,width=\textwidth]{Graphs/Distance 1 special cases.pdf}
        \caption{A,LB}
        \label{fig:unsafeutil7}
    \end{subfigure}
    \hfill
    \begin{subfigure}{0.07\textwidth}
        \centering
        \includegraphics[page=13,width=\textwidth]{Graphs/Distance 1 special cases.pdf}
        \caption{A,LB}
        \label{fig:unsafeutil8}
    \end{subfigure}
    \caption{Unsafe manipulations, Max-Util, with the equal size constraint. Legend: A - add, R - remove, LB/UB - LB/UB-unsafe. In the figures $n$ stands for the total number of agents and $x$ for any number between $0$ and the clique's size}
    %graphs showing manipulation of different objectives by adding edges. Dotted edges are added by the manipulator.
    \label{fig:unsafe_util_equalsize}
\end{figure}


\begin{figure}
    \centering
\begin{subfigure}{0.07\textwidth}
        \centering
        \includegraphics[page=1,width=\textwidth]{Graphs/Distance 1 util unsafe.pdf}
        \caption{R/LB}
        \label{fig:distance1_util_unsafe_remove_LB_appendix}
    \end{subfigure}
    \hfill
    \begin{subfigure}{0.07\textwidth}
        \centering
        \includegraphics[page=2,width=\textwidth]{Graphs/Distance 1 util unsafe.pdf}
        \caption{R/UB}
        \label{fig:distance1_util_unsafe_remove_UB_appendix}
    \end{subfigure}
    \hfill
    \begin{subfigure}{0.07\textwidth}
        \centering
        \includegraphics[page=3,width=\textwidth]{Graphs/Distance 1 util unsafe.pdf}
        \caption{A/LB}
        \label{fig:distance1_util_unsafe_add_LB}
    \end{subfigure}
    \hfill
    \begin{subfigure}{0.07\textwidth}
        \centering
        \includegraphics[page=4,width=\textwidth]{Graphs/Distance 1 util unsafe.pdf}
        \caption{A/UB}
        \label{fig:distance1_util_unsafe_add_UB}
    \end{subfigure}

    \caption{Unsafe manipulations, Max-Util. Legend: A - add, R - remove, LB/UB - LB/UB-unsafe}
    %graphs showing manipulation of different objectives by adding edges. Dotted edges are added by the manipulator.
    \label{fig:distance1_unsafe_util_appendix}
\end{figure}

\begin{figure}
    \centering
    \begin{subfigure}{0.07\textwidth}
        \centering
        \includegraphics[page=1,width=\textwidth]{Graphs/egal unsafe distance1.pdf}
        \caption{R/UB}
        \label{fig:egalremove1}
    \end{subfigure}
    \hfill
    \begin{subfigure}{0.07\textwidth}
        \centering
        \includegraphics[page=2,width=\textwidth]{Graphs/egal unsafe distance1.pdf}
        \caption{R/UB}
        \label{fig:egalremove2}
    \end{subfigure}
    \hfill
    \begin{subfigure}{0.07\textwidth}
        \centering
        \includegraphics[page=3,width=\textwidth]{Graphs/egal unsafe distance1.pdf}
        \caption{R/UB}
        \label{fig:egalremove3}
    \end{subfigure}
    \hfill
    \begin{subfigure}{0.07\textwidth}
        \centering
        \includegraphics[page=4,width=\textwidth]{Graphs/egal unsafe distance1.pdf}
        \caption{A/UB}
        \label{fig:egalremove4}
    \end{subfigure}

    \caption{Unsafe manipulations, Max-Egal. Legend: A - add, R - remove, LB/UB - LB/UB-unsafe}
    %graphs showing manipulation of different objectives by adding edges. Dotted edges are added by the manipulator.
    \label{fig:distance1_unsafe_egal}
\end{figure}

\begin{figure}
    \centering
    \begin{subfigure}{0.07\textwidth}
        \centering
        \includegraphics[page=2,width=\textwidth]{Graphs/Distance 1 manipulation.pdf}
        \caption{\\U,E/LB}
        \label{fig:distance1_util/egal_lb}
    \end{subfigure} \hspace{20pt}
    \begin{subfigure}{0.07\textwidth}
        \centering
        \includegraphics[page=4,width=\textwidth]{Graphs/Distance 1 manipulation.pdf}
        \caption{\\U/UB}
        \label{fig:distance1_util_ub}
    \end{subfigure} \hspace{20pt}
    \begin{subfigure}{0.07\textwidth}
        \centering
        \includegraphics[page=7,width=\textwidth]{Graphs/Distance 1 manipulation.pdf}
        \caption{\\1/UB}
        \label{fig:distance1_least1}
    \end{subfigure}
    \caption{Manipulations by removing edges, for distance 1. Legend: Objective/Manipulation~Type. Key: U=Max-Util, E=Max-Egal, 1=At-Least-1, UB=UB-improvement, LB=LB-improvement.}
    %graphs showing manipulation of different objectives by adding edges. Dotted edges are added by the manipulator.
    \label{fig:distance1_manipulation}
\end{figure}


\clearpage

\subsection*{Max-Util Distance 2 Conjecture}
\begin{conjecture}
\label{conj:util}
Max-Util is 2-safe weak-proof against manipulator $m^-$ over undirected networks.
\end{conjecture}

We lay out some of the advances we have made trying to prove the conjecture:

Let $G_2=(A,E)$ be a partial network and $m^-$ a manipulator. Assume by negation a 2-safe weak-improvement $r^m$ exists over $G_2$. Hence there exists a possible game $\overline{G}$ of $G_2$ for which $r^m$ is a weak-improvement. We want to show that either the manipulation does not really improve the upper bound, or there exists another supplement $\overline{G'}$ of $G$ for which $r^m$ lowers the manipulator's upper bound.

Denote $\overline{G}^m$ as the network $\overline{G}$ after the manipulation $r^m$. Let $P^m=\{B_1,B_2\}$ be (one of) the CSs with higher utility for $m$ than all CSs in $\overline{G}$, and denote by $B_1$ the coalition in which $m$ is. Formally:
\begin{equation}
\label{ineq:util_ub_safe}
u(m,P_m) > \underset{P\in O_{obj}(\overline{G})}{\min}(u(m,P)).    
\end{equation}
Denote by $c_0$ and $c_m$ the minimum cut size of $\overline{G}$ and $\overline{G}^m$ respectively.
We will define $c(P,G)$ as the cut size of coalition structure $P$ in network $G$.


\begin{lemma}
$B_2$ contains at least $2$ node $a_1,a_2$ such that $a_1,a_2\in N$ and $a_1,a_2\notin N^m$.
\end{lemma}
\begin{proof}
Since $r^m$ is a LB-improvement there is a solution for $\overline{G}$ which is not a solution in $\overline{G}^m$. Since we only removed edges, the only reason it would happen is that $c_0 > c_m$.
Since $P^m$ is not a solution in $\overline{G}$ (as it is better than all solutions in it) we get that $c(P^m,\overline{G}) > c_0$. Combining both outcomes we get that $c(P^m,\overline{G}) \geq c_m+2$. Because we know that $c(P^m,\overline{G}^m) = c_m$ and the only changes between $\overline{G}$ and $\overline{G}^m$ are edges that the manipulator removed, it has to be the case that he removed at least $2$ edges going out to $B_2$.
\end{proof}
We get that the maximum utility $m$ can have after the manipulation is $|N(m)|-2$. Because of that, the minimum degree $\delta(\overline{G}) = min(\{deg(v) | v\in V\ and v != m \})$ of all the nodes but $m$ is at least $c_0+1$. Otherwise the CS where a node with at most $c_0$ neighbours is alone is a minimal cut and yields $m$ a utility of $|N(m)|-1$ or $|N(m)|$ in contradiction to inequality~\ref{ineq:util_ub_safe}.

We distinguish between multiple cases. 
Case one: $B_2\subseteq N$, i.e. contains only true neighbours of $m$.
Since $P^m$ has higher utility for the manipulator than all previous solutions, in all previous solutions there are at least $|B_2|+1$ neighbours of her not in her coalition. Hence $c_0 \geq |B_2|+1$. By the previous statement we get that $\delta(\overline{G}) \geq |B_2| + 2$. Since the manipulator might have removed one edge in $\overline{G}^m$ , we get that $\delta(\overline{G}^m) \geq |B_2| + 1$. Because of that, in $P^m$ every node in $B_2$ has at least $2$ neighbours outside $B_2$, which we concludes that $c_m \geq 2|B_2| \rightarrow c_0 > 2|B_2| \rightarrow \delta(\overline{G}) > 2|B_2| + 1 \rightarrow \delta(\overline{G}^m) \geq 2|B_2|+1$. Hence all nodes in $B_2$ has at least $|B_2|+2$ edges going out from $B_2$, so we get that $c_m > |B_2|*(|B_2|+2)$. By repeating this process we get that the cut is unbounded, in contradiction since the network is finite.

Case two: $|B_2\cap N| = |B_2| -1$, i.e. $B_2$ contains exactly one non-neighbour of $m$, denote her by $a$.
By the same logic from above we get that $c_0 \geq |B_2|$ (as in $P^m$ the manipulator only has $|B_2| -1$ neighbours. So $\delta(\overline{G}) \geq |B_2| + 1 \rightarrow \delta(\overline{G}^m) \geq |B_2|$. Since $a$ is not a neighbour of $m$, her degree in $\overline{G}^m$ has not changed and it is still at least $|B_2| +1$. In conclusion in $B_2$ we have $|B_2|-1$ nodes of degree at least $|B_2|$ and one with at least $|B_2| +1$. Hence $c_m \geq |B_2| +1 \rightarrow c_0 \geq |B_2| + 2$, and yet again by repeating the process we get that the cut is unbounded.

Case three: $B_2$ contains at least 2 non-neighbours of $m$.

This is the case we could not prove. Note that in any solution in Max-Util any agent is guaranteed to be in the coalition where she has more neighbours. Therefore if $B_2$ contains at least $2$ neighbours of $m$, then $B_1$ contains at least $3$. However, if the manipulator had only $5$ neighbours originally, he would have had at least $3$ neighbours before the manipulation. So he has at least $4$ neighbours in $B_1$.


To sum it up, if a weak-improvement did exist, it has to be that $B_1$ contains at least $4$ neighbours of $m$, $B_2$ contains at least $2$ neighbours and $2$ non-neighbours of $m$, and the minimum degree of the graph is at least $4$.



\end{document}}

\end{document}
