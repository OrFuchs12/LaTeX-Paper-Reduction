
an extension of the Explicit Estimation 



short paths to a goal 
An example of such a 
For example, While $u_C(n)$ serves as a reasonable proxy to the potential of a node, in some cases, 
it is not clear whether searching 

, especially  is that it does not consider the cost 

Computing the potential function in general is challenging. However, 
We have developed a family of proxy functions that under some assumptions, can be used instead of the potential function and yield exactly the same expansion order. 
The mentioned above assumptions relate the values outputted by a given heuristic function and the actual cost of the lowest-cost path to a goal. The difference between the two is referred to as the \emph{heuristic error}. 
Given such a proxy function, one can implement a best-first search that expands exactly the nodes as \PTS, without access to the exact potential function. %exactly computing the potential of each node. 


One such assumption is that the heuristic error grows linearly with the heuristic value. That is, for a node $n$
we have that 
\begin{equation}
 h^*(n)=\alpha\cdot h(n)   
\end{equation}
where $\alpha$ is an i.i.d random variable. 
\roni{Maybe put some figures}
The following function
For such a relation between $h(n)$ and $h^*(n)$, 



We refer to such a probabilistic model of the In such a case, 


A particular example of how the heuristic error relates For example, consider a heuristic function whose heuristic error grows linearly 


close inspection of several standard 


is likelihood 
Computing, or even formally defining this potential function, is non-trivial.  








\PTS uses this potential function 



In some cases, the 
A \emph{bounded-cost search problem}

the bounded-cost search problem, where the task is to find, as quickly as possible, a solution with a cost that is lower than a given cost bound. We then explain why existing search algorithms, such as optimal search algorithms and bounded-suboptimal search algorithms, are not suited to solve bounded-cost search problems. Finally, we introduce a new algorithm called Potential Search (PTS), specifically designed for solving bounded-cost search problems. We define a bounded-cost search problem as follows.
Definition 1 Bounded-cost search problem

Given an initial state s, a goal test function, and a constant C, a bounded-cost search problem is the problem of finding a path from s to a goal state with cost less than C.






Interestingly, 







recent years shown 


is 
All nodes in \open for which 
\roni{A word on how they work}


The theoretical understanding of bounded-suboptimal search algorithm, to-date, is quite limited. For example, we expect an effective bounded-suboptimal search algorithm to trade-off time for solution quality, i.e., that increasing $B$ will result in faster runtime. However, this is not necessarily the case. 

1. Reopening
2. Focal, speedy, greedy
3. Nathan


There is a range 
An alternative 

3. Dynamic potential






Search algorithms that guarantee

way to define the \emph{suboptimality} of a given solution~\cite{rick}. 
Suboptimality, 
The gap between the strict quality guarantee provided by optimal search algorithm and the absence of any solution quality guarantee provided by the mentioned above search algorithm

% It is possible to distinguish between two types of combinatorial search problems. In one type, which I will refer to as \emph{state-oriented problems}, the problem is to find a goal state. 
% In such problems the set of goals is not given explicitly, and the challenge is to find the assignment of state variable $n$ values that consists of a g

\section{Natural Solvers}
To solve state finding problems, one needs to search in the space of possible assignments of values to state variables that may satisfy the requirements of a goal state. 




. To solve path finding problems, one needs to search in the space of possible paths. 


\subsection{From State Finding to Path Finding}


\subsection{From Path Finding to State Finding}



\section{Searching for Faults, Diagnoses, and Bugs}

\section{Search Challenges in Multi-Agent Path Finding}

\section{Search Problems in Plan and Goal Recognition Design}

\section{Model-Free Search}


objective is to 
A possible way to classify distinguish between 



A \emph{state-transition operator}, or simply an operator, is a function that accepts a state and outputs a state. 
Every state $s$ is associated with a set of state-transition operators, denoted $O(s)$. 
We say that $o$ is \emph{applicable} in $s$ if $o\in O(s)$. 
A sequence of operators $o_1, o_2, \ldots o_m$ is applicable in a state $s$
if $o_m$ is applicable in $s$, 
$o_{m-1}$ is applicable in $o_m(s)$,
$o_{m-1}$ is applicable in $o_{m-1}(o_m(s))$, 
and so. 

In a quite general class of combinatorial 
In a general class of combinatorial 




A state $s'$ is \emph{reachable} from a state $s$ if there is a sequence of state transition operators 

where the set of states. 

is zero or more 


\subsection{Heuristic Search}

\subsection{Compilation}

\section{Single-Agent Planning}
\section{Multi-Agent Planning}
\section{Model-Based Diagnosis}