%File: formatting-instructions-latex-2024.tex
%release 2024.0
\documentclass[letterpaper]{article} % DO NOT CHANGE THIS
\usepackage{aaai24}  % DO NOT CHANGE THIS
\usepackage{times}  % DO NOT CHANGE THIS
\usepackage{helvet}  % DO NOT CHANGE THIS
\usepackage{courier}  % DO NOT CHANGE THIS
\usepackage[hyphens]{url}  % DO NOT CHANGE THIS
\usepackage{graphicx} % DO NOT CHANGE THIS
\urlstyle{rm} % DO NOT CHANGE THIS
\def\UrlFont{\rm}  % DO NOT CHANGE THIS
\usepackage{natbib}  % DO NOT CHANGE THIS AND DO NOT ADD ANY OPTIONS TO IT
\usepackage{caption} % DO NOT CHANGE THIS AND DO NOT ADD ANY OPTIONS TO IT
\frenchspacing  % DO NOT CHANGE THIS
\setlength{\pdfpagewidth}{8.5in}  % DO NOT CHANGE THIS
\setlength{\pdfpageheight}{11in}  % DO NOT CHANGE THIS
%
% These are recommended to typeset algorithms but not required. See the subsubsection on algorithms. Remove them if you don't have algorithms in your paper.
\usepackage{algorithm}
\usepackage{algorithmic}
\usepackage{booktabs}
\usepackage{adjustbox}
\usepackage{xcolor}
\usepackage{bm}
\usepackage{multirow}
\usepackage{threeparttable}
\usepackage{eso-pic}
\usepackage{array}
\usepackage{bbding}
\usepackage{amsmath}
%
% These are are recommended to typeset listings but not required. See the subsubsection on listing. Remove this block if you don't have listings in your paper.
\usepackage{newfloat}
\usepackage{listings}
\DeclareCaptionStyle{ruled}{labelfont=normalfont,labelsep=colon,strut=off} % DO NOT CHANGE THIS
\lstset{%
	basicstyle={\footnotesize\ttfamily},% footnotesize acceptable for monospace
	numbers=left,numberstyle=\footnotesize,xleftmargin=2em,% show line numbers, remove this entire line if you don't want the numbers.
	aboveskip=0pt,belowskip=0pt,%
	showstringspaces=false,tabsize=2,breaklines=true}
\floatstyle{ruled}
\newfloat{listing}{tb}{lst}{}
\floatname{listing}{Listing}
%
% Keep the \pdfinfo as shown here. There's no need
% for you to add the /Title and /Author tags.
\pdfinfo{
/TemplateVersion (2024.1)
}

% DISALLOWED PACKAGES
% \usepackage{authblk} -- This package is specifically forbidden
% \usepackage{balance} -- This package is specifically forbidden
% \usepackage{color (if used in text)
% \usepackage{CJK} -- This package is specifically forbidden
% \usepackage{float} -- This package is specifically forbidden
% \usepackage{flushend} -- This package is specifically forbidden
% \usepackage{fontenc} -- This package is specifically forbidden
% \usepackage{fullpage} -- This package is specifically forbidden
% \usepackage{geometry} -- This package is specifically forbidden
% \usepackage{grffile} -- This package is specifically forbidden
% \usepackage{hyperref} -- This package is specifically forbidden
% \usepackage{navigator} -- This package is specifically forbidden
% (or any other package that embeds links such as navigator or hyperref)
% \indentfirst} -- This package is specifically forbidden
% \layout} -- This package is specifically forbidden
% \multicol} -- This package is specifically forbidden
% \nameref} -- This package is specifically forbidden
% \usepackage{savetrees} -- This package is specifically forbidden
% \usepackage{setspace} -- This package is specifically forbidden
% \usepackage{stfloats} -- This package is specifically forbidden
% \usepackage{tabu} -- This package is specifically forbidden
% \usepackage{titlesec} -- This package is specifically forbidden
% \usepackage{tocbibind} -- This package is specifically forbidden
% \usepackage{ulem} -- This package is specifically forbidden
% \usepackage{wrapfig} -- This package is specifically forbidden
% DISALLOWED COMMANDS
\nocopyright
% \nocopyright -- Your paper will not be published if you use this command
% \addtolength -- This command may not be used
% \balance -- This command may not be used
% \baselinestretch -- Your paper will not be published if you use this command
% \clearpage -- No page breaks of any kind may be used for the final version of your paper
% \columnsep -- This command may not be used
% \newpage -- No page breaks of any kind may be used for the final version of your paper
% \pagebreak -- No page breaks of any kind may be used for the final version of your paperr
% \pagestyle -- This command may not be used
% \tiny -- This is not an acceptable font size.
% \vspace{- -- No negative value may be used in proximity of a caption, figure, table, section, subsection, subsubsection, or reference
% \vskip{- -- No negative value may be used to alter spacing above or below a caption, figure, table, section, subsection, subsubsection, or reference

\setcounter{secnumdepth}{0} %May be changed to 1 or 2 if section numbers are desired.

% The file aaai24.sty is the style file for AAAI Press
% proceedings, working notes, and technical reports.
%

% Title

% Your title must be in mixed case, not sentence case.
% That means all verbs (including short verbs like be, is, using,and go),
% nouns, adverbs, adjectives should be capitalized, including both words in hyphenated terms, while
% articles, conjunctions, and prepositions are lower case unless they
% directly follow a colon or long dash
% \iffalse
\title{CityPulse: Fine-Grained Assessment of Urban Change with\\ Street View Time Series}
\author{
    %Authors
    % All authors must be in the same font size and format.
    % Written by AAAI Press Staff\textsuperscript{\rm 1}\thanks{With help from the AAAI Publications Committee.}\\
    % AAAI Style Contributions by Pater Patel Schneider,
    % Sunil Issar,\\
    Tianyuan Huang\textsuperscript{\rm 1}\equalcontrib,
    Zejia Wu\textsuperscript{\rm 2}\equalcontrib,
    Jiajun Wu\textsuperscript{\rm 1},
    Jackelyn Hwang\textsuperscript{\rm 1},
    Ram Rajagopal\textsuperscript{\rm 1}
}
\affiliations {
    % Affiliations
    \textsuperscript{\rm 1}Stanford University
    \textsuperscript{\rm 2}University of California San Diego\\
    \{tianyuah, jihwang, ramr\}@stanford.edu, 
    zew024@ucsd.edu, 
    jiajunwu@cs.stanford.edu
}
% \affiliations {
%     % Affiliations
%     \textsuperscript{\rm 1}Stanford University\\
%     % zhecheng@stanford.edu,
%     % rajanie@stanford.edu,
%     % tianyuah@stanford.edu, 
%     % jiajunwu@cs.stanford.edu,
%     % ramr@stanford.edu
%     \{zhecheng, rajanie, tianyuah, ramr\}@stanford.edu, jiajunwu@cs.stanford.edu
% }
% \affiliations{
%     %Afiliations
%     \textsuperscript{\rm 1}Association for the Advancement of Artificial Intelligence\\
    % If you have multiple authors and multiple affiliations
    % use superscripts in text and roman font to identify them.
    % For example,

    % Sunil Issar\textsuperscript{\rm 2}, 
    % J. Scott Penberthy\textsuperscript{\rm 3}, 
    % George Ferguson\textsuperscript{\rm 4},
    % Hans Guesgen\textsuperscript{\rm 5}
    % Note that the comma should be placed after the superscript

    % 1900 Embarcadero Road, Suite 101\\
    % Palo Alto, California 94303-3310 USA\\
    % email address must be in roman text type, not monospace or sans serif
    % proceedings-questions@aaai.org
%
% See more examples next
% }

%Example, Single Author, ->> remove \iffalse,\fi and place them surrounding AAAI title to use it
\iffalse
\title{My Publication Title --- Single Author}
\author {
    Author Name
}
\affiliations{
    Affiliation\\
    Affiliation Line 2\\
    name@example.com
}
\fi

\iffalse
%Example, Multiple Authors, ->> remove \iffalse,\fi and place them surrounding AAAI title to use it
\title{CityPulse: Fine-Grained Assessment of Urban Change with\\ Street View Time Series}
\author {
    % Authors
    First Author Name\textsuperscript{\rm 1,\rm 2},
    Second Author Name\textsuperscript{\rm 2},
    Third Author Name\textsuperscript{\rm 1}
}
\affiliations {
    % Affiliations
    \textsuperscript{\rm 1}Affiliation 1\\
    \textsuperscript{\rm 2}Affiliation 2\\
    firstAuthor@affiliation1.com, secondAuthor@affilation2.com, thirdAuthor@affiliation1.com
}
\fi


% REMOVE THIS: bibentry
% This is only needed to show inline citations in the guidelines document. You should not need it and can safely delete it.
\usepackage{bibentry}
% END REMOVE bibentry
\newtheorem{definition}{Definition}
\begin{document}

\maketitle

\begin{abstract}
Urban transformations have profound societal impact on both individuals and communities at large. Accurately assessing these shifts is essential for understanding their underlying causes and ensuring sustainable urban planning. Traditional measurements often encounter constraints in spatial and temporal granularity, failing to capture real-time physical changes. While street view imagery, capturing the heartbeat of urban spaces from a pedestrian point of view, can add as a high-definition, up-to-date, and on-the-ground visual proxy of urban change. We curate the largest street view time series dataset to date, and propose an end-to-end change detection model to effectively capture physical alterations in the built environment at scale. We demonstrate the effectiveness of our proposed method by benchmark comparisons with previous literature and implementing it at the city-wide level. Our approach has the potential to supplement existing dataset and serve as a fine-grained and accurate assessment of urban change.

\end{abstract}

\section{Introduction}
% \textbf{Significance of the problem: The social impact problem considered by this paper is significant and has not been adequately addressed by the AI community}\\
% Urbanization has led to profound changes in the physical, social, and economic landscapes of cities worldwide. 
Our cities are evolving, and understanding how cities change at a granular level has far-reaching societal impact --- from facilitating better urban planning and infrastructure assessment to enabling more sustainable social and environmental interventions \cite{Daniel2015Goal1M,Seto2017SustainabilityIA}. 
Current measurements of urban change rely on datasets ranging from survey data such as American Community Survey (ACS), to government open data like construction permits, to remote sensing data such as satellite and aerial imagery. However, survey data often fall short of spatial and temporal granularity \cite{hwang14}, and top-down perspectives from the remote sensing data may not adequately represent the street-level changes that directly impact the daily lives of urban residents. And some construction permits data are not universally accessible.
Street view imagery, on the other hand, offers a high-resolution and frequently updated visual representation of urban environments from a ground-level perspective \cite{Huang2022DetectingNG}. By curating and analyzing the time series data of street view imagery, we can establish a more precise and direct proxy for how cities evolve over time.
\documentclass[letterpaper]{article} % DO NOT CHANGE THIS
\usepackage{aaai20}  % DO NOT CHANGE THIS
\usepackage{times}  % DO NOT CHANGE THIS
\usepackage{helvet} % DO NOT CHANGE THIS
\usepackage{courier}  % DO NOT CHANGE THIS
\usepackage[hyphens]{url}  % DO NOT CHANGE THIS
\usepackage{graphicx} % DO NOT CHANGE THIS
\urlstyle{rm} % DO NOT CHANGE THIS
\def\UrlFont{\rm}  % DO NOT CHANGE THIS
\usepackage{graphicx}  % DO NOT CHANGE THIS
\frenchspacing  % DO NOT CHANGE THIS
\setlength{\pdfpagewidth}{8.5in}  % DO NOT CHANGE THIS
\setlength{\pdfpageheight}{11in}  % DO NOT CHANGE THIS

\usepackage[utf8]{inputenc}
\usepackage{bbold}
\usepackage{amssymb}
\usepackage{amsmath}
\usepackage{balance}
\usepackage{tikz}
\usepackage{subfig}
\usepackage{bbm}
\usepackage{enumitem}
\usepackage{appendix}

\DeclareMathOperator*{\argmax}{arg\,max}
\providecommand{\TODO}[1]{\textcolor{red}{\textbf{#1}}}



%\title{Automating Product Placement in Retail via Stochastic Demand Simulation}

\title{A Probabilistic Simulator of Spatial Demand for Product Allocation}

%Your title must be in mixed case, not sentence case. 
% That means all verbs (including short verbs like be, is, using,and go), 
% nouns, adverbs, adjectives should be capitalized, including both words in hyphenated terms, while
% articles, conjunctions, and prepositions are lower case unless they
% directly follow a colon or long dash
\author{Porter Jenkins \textsuperscript{\rm 1}, Hua Wei \textsuperscript{\rm 1}, J. Stockton Jenkins \textsuperscript{\rm 2}, Zhenhui Li \textsuperscript{\rm 1} \\ 
\textsuperscript{\rm 1} Penn State University \\
\textsuperscript{\rm 2} Brigham Young University \\%If you have multiple authors and multiple affiliations
% use superscripts in text and roman font to identify them. For example, Sunil Issar,\textsuperscript{\rm 2} J. Scott Penberthy\textsuperscript{\rm 3} George Ferguson,\textsuperscript{\rm 4} Hans Guesgen\textsuperscript{\rm 5}. Note that the comma should be placed BEFORE the superscript for optimum readability
}


\begin{document}
\maketitle

\begin{abstract}
Connecting consumers with relevant products is a very important problem in both online and offline commerce. In physical retail, product placement is an effective way to connect consumers with products. However, selecting product locations within a store can be a tedious process. Moreover, learning important spatial patterns in offline retail is challenging due to the scarcity of data and the high cost of exploration and experimentation in the physical world. To address these challenges, we propose a stochastic model of spatial demand in physical retail. We show that the proposed model is more predictive of demand than existing baselines. We also perform a preliminary study into different automation techniques and show that an optimal product allocation policy can be learned through Deep Q-Learning. 

\end{abstract}



\section{Introduction}
%%%%%%%%%%%%%%%%%%%%%%%%%%%%%%
% 1.定义image captioning任务 
%%%%%%%%%%%%%%%%%%%%%%%%%%%%%%
Image captioning is a fundamental task in vision-language understanding that involves generating natural language descriptions for a given image. It plays a critical role in facilitating more complex vision-language tasks, such as visual question answering \cite{Agrawal2015VQAVQ,gqa,okvqa} and visual dialog \cite{Das2016VisualD,Niu2018RecursiveVA,llava}.
%%%%%%%%%%%%%%%%%%%%%%%%%%%%%%
% text-only training 的介绍
%%%%%%%%%%%%%%%%%%%%%%%%%%%%%%
The mainstream image captioning methods \cite{conimgcap4,conimgcap1,conimgcap3,conimgcap2} require expensive human annotation of image-text pairs for training neural network models in an end-to-end manner. Recent developments in Contrastive Image Language Pre-training (CLIP) \cite{clip} have enabled researchers to explore a new paradigm, zero-shot image captioning, through text-only training. In particular, CLIP learns a multi-modal embedding space where semantically related images and text are encoded into features with close proximity. As such, if a model learns to map the CLIP text features to their corresponding texts, it is feasible to generate image captions from the CLIP image features without needing supervision from caption annotations.

%%%%%%%%%%%%%%%%%%%%%%%%%%%%%%
% text-only training 的优势
%%%%%%%%%%%%%%%%%%%%%%%%%%%%%%

One main advantage of this zero-shot captioning paradigm is that it enables a Large Language Model (LLM) \cite{gpt3, Zhang2022OPTOP} with image captioning capabilities using only text data and affordable computational resources. Despite the impressive performance achieved by recent powerful multimodal models \cite{miniGPT4,llava}, they typically require large-scale, high-quality human-annotated data and expensive computational resources for fine-tuning an LLM. Zero-shot captioning methods can significantly reduce such costs, which is particularly important in situations of data scarcity and limited resources. Moreover, recent work \cite{Guo2022FromIT, Changpinyo2022AllYM,Tiong2022PlugandPlayVZ} demonstrates that other vision-language tasks, such as VQA, can be addressed by LLMs and image captions. Consequently, the paradigm of zero-shot captioning has the potential to pave the way to solving complex vision-language tasks with LLMs through efficient text-only training. 


%%%%%%%%%%%%%%%%%%%%%%%%%%%%%%
% zero-shot image captioning via text-only training 的challenge
%%%%%%%%%%%%%%%%%%%%%%%%%%%%%%
A critical challenge in zero-shot image captioning through text-only training is to mitigate a widely observed phenomenon known as the \textit{modality gap}. While the features of paired texts and images are close in the CLIP embedding space, there remains a gap between them \cite{MindGap}. This gap often results in inaccurate mappings from the image embeddings to the text ones. Consequently, without fine-tuning with paired data, it significantly impairs the performance of zero-shot image captioning.
%%%%%%%%%%%%%%%%%%%%%%%%%%%%%%
% current works intro
%%%%%%%%%%%%%%%%%%%%%%%%%%%%%%
Several works have attempted to address the modality gap in zero-shot image captioning, relying mainly on two strategies: (1) The first strategy leverages a memory bank from training text data to project visual embeddings into the text embedding space \cite{DeCap}. However, this projection prevents it from representing any semantic content outside the distribution of the memory bank features and introduces extra inference costs; (2) The second approach injects noise during training to encourage the visual embeddings to be included inside the semantic neighborhood of the corresponding text embeddings \cite{CapDec}. Nonetheless, the noise injection tends to diffuse the distribution of visual inputs at the cost of weakening the semantic correlation between paired images and text embeddings. 

%However, in the first strategy, the projection of visual embeddings prevents them from  For the second strategy, noise injection during training diffuses the input distribution at the cost of degrading the semantic correlation between paired images and text embeddings.

%Previous attempts \cite{CapDec,DeCap} to reduce the modality gap in zero-shot image captioning can be summarized into two aspects: (1) Decap\cite{DeCap} leverages a memory bank from training text data to project visual embeddings into text embedding space. However, the projection of visual embeddings prevents it from representing any semantic content outside the distribution of the memory bank and introduce extra inference cost. (2) CapDec\cite{CapDec}proposes to inject noise during training to encourage the visual embedding to be included inside the text embedding space. 
% current work weakness
%Nevertheless, noise injection during training diffuses the input distribution at the cost of degrading the semantic correlation between paired images and text embeddings.


%%%%%%%%%%%%%%%%%%%%%%%%%%%%%%
% 我们工作的流程
% 分析得到两个结论:1.subregion带来更好的匹配2.image text gap符合高斯分布
%%%%%%%%%%%%%%%%%%%%%%%%%%%%%%
To tackle these challenges, we first conduct a thorough analysis of the CLIP feature space, leading to two key observations. First, most text descriptions are unable to fully capture the content of their paired images. However, we empirically find that the visual embedding of certain local regions of an image, named image subregions, have closer proximity to the text embedding of the paired caption. Integrating such image subregions with the global image representation generates a tighter alignment between image and text. Additionally, we analyze the distribution of the gap between the CLIP features of image or subregion-text pairs and find that it closely resembles a zero-mean Gaussian distribution.
%initiate our investigation by conducting a thorough analysis of the CLIP latent space. Building upon the insights from the work \cite{MindGap}, we identify a key factor contributing to the existence of a modality gap. Due to the inherent disparities between textual and visual modalities, text is incapable of comprehensively describing the information within an image. However, we empirically demonstrate that the CLIP embedding of some part of image, named image subregions, exhibit closer proximity to the CLIP embedding of the paired caption. The integration between image subregion information and global image feature leads to more compact image text alignment. Besides, we collect the statistics of the gap between CLIP image and text feature. The results demonstrate the gap is close to gaussian distribution. 

%%%%%%%%%%%%%%%%%%%%%%%%%%%%%%
% 我们的方法简略介绍
%%%%%%%%%%%%%%%%%%%%%%%%%%%%%%

Based on our findings, we propose a novel zero-shot image captioning framework, named \textit{\textbf{M}ining Fine-Grained Image-Text \textbf{A}lignment in \textbf{C}LIP for \textbf{Cap}tioning} (MacCap), to address the aforementioned challenges. In this framework, we introduce a region-aware cross-modal representation based on CLIP and an effective unimodal training strategy for an LLM-based caption generator. Our cross-modal representation maps an input image into the language space of LLMs and consists of two main components. First, we design a \textit{sub-region feature aggregation} module to fuse both global and subregion-level CLIP image features, resulting in a smaller gap between the corresponding CLIP text embedding. Next, we introduce a learnable adaptor-decoder to transform the CLIP representation into the LLM's language space.
To train our model with text-only data, we develop a robust procedure to learn a projection from the CLIP embedding space to a language representation, enabling the LLM to reconstruct captions. Specifically, our learning procedure first injects noise into our region-aware CLIP-text representation, mimicking the modality gap between image and text features. This is followed by a multiple sampling and filtering step that leverages the CLIP knowledge to improve the quality of the captioning.
%tackles the problem from three key perspectives. Firstly, we focus on learning a robust projection from CLIP embedding space to language model space by text reconstruction training, which enable model to generate text based on both CLIP image and text feature. The region noise injection in training alleviate the \textit{modality gap} between image and text feature, which makes the projection works for both image and text features. Secondly, we design \textit{sub-region feature aggregation} to obtain a more compact CLIP image feature, which is based on the observation that CLIP subregion feature exhibit closer disntance with corresponding text feature. Third, we propose multiple sampling and filtering to mitigate the drawbacks of noise injection, which leverage CLIP knowledge to further boost caption performance. Finally, we design a pipeline for zero-shot VQA to demonstrate the extensibility of ouir methods to more intricate vision-language tasks.
In addition to the image captioning task, we further extend our framework to build a zero-shot VQA pipeline, demonstrating the generality of our cross-modal representation for more complex vision-language tasks.

%%%%%%%%%%%%%%%%%%%%%%%%%%%%%%
% 我们的方法简略介绍
%%%%%%%%%%%%%%%%%%%%%%%%%%%%%%

We evaluate our framework on several widely-adopted image captioning benchmarks, such as MSCOCO \cite{mscoco} and Flickr30k \cite{Flickr30k}, as well as a standard VQA benchmark, VQAV2 \cite{vqav2}. Our extensive experiments cover multiple vision-language tasks, including zero-shot in-domain image captioning, zero-shot cross-domain image, and zero-shot VQA. The results not only demonstrate the superiority of our methods but also validate our findings on the CLIP embedding space.

% demonstrate through experiments that our proposed methods outperform previous approaches on popular captioning benchmarks, such as MSCOCO, Flickr30k, which further verify our understanding of \textit{concept region}



% Specifically, we evaluate the distribution of the image and text embedding space under hyperspherical coordinates and observe a geometric phenomenon \textit{concept region} 
% where semantically correlated image and text embedding tend to clustering despite the \textit{modality gap}.
% 我们基于concept region的观察提出的方法:concept region和modality gap的cause里面有mismatch pair data导致的semantic ambiguity,总体思路是在train的时候模拟在concept region。在training的时候,我们给text embedding加上region noise,具体而言就是以原本text embedding为中心,一定范围内的多个随机sample的related text embedding,这样的获得的text embedding全都是在输入text对应的concept region内部。在zs captioning的inference时,部分image sub-region inforamtion 会比global image 对text匹配度更高,因此我们基于部分image sub-region inforamtion
% Motivated by the semantic ambiguity of mismatched data observed in \textit{concept region}, we propose two 
% an image sub-region information aggregation strategy for .In detail

% result summary

\section{Problem Definition}\label{prob-def}
In the following section, we provide a formal definition of the optimal allocation problem. Additionally, we define the necessary components of our reinforcement learning agent: the state space, action space, reward function, and state transition function.
\subsection{Optimal Allocation Problem}

In a physical retail environment $\mathcal{R}$ with a set of $n$ spatial regions, we represent the environment with a spatial graph $\mathcal{R} = (\mathcal{V}, \mathcal{E})$, where each region $r_i\in \mathcal{V}$ is a vertex in the graph, the spatial neighboring relation between two regions $r_i$ and $r_j$ are represented as $e_{ij}\in \mathcal{V}$. From $\mathcal{G}$, we can construct the adjacency matrix, $\textbf{A}$.

Additionally, we observe a set of $k$ products, $\mathcal{M} = \{m_j : 0 < j <=k\}$ that are sold. For each product, $m_j$, we know the retail price, $p_j$. 

The decision process faced by the retailer is to allocate each product in $\mathcal{M}$ across regions in $\mathcal{R}$. We define the allocation policy as a function $f$:

\begin{equation}
    f: \mathcal{R} \times \mathcal{M} \rightarrow \mathcal{Z}
\end{equation}
\begin{equation}
    \mathcal{Z} = \{\langle r_i, p_j \rangle , ... \langle r_w, p_q \rangle \}
\end{equation}

Where $\mathcal{Z}$ is the set of selected product region, such that $w <= n$, $q <= k$ and $\mathcal{Z} \subseteq \mathcal{R} \times \mathcal{M}$. This function is typically dynamic over time, which we denote as $f^{t}$. To simplify computation, we treat $\mathcal{Z}^{t}$ as an $(n \times k)$ grid and refer to it as the board configuration at time, $t$. An optimal retail strategy is to find the allocation policy that maximizes revenue:

\begin{equation}
    f^{\ast} = \sum_{t}^{T} \argmax_{f^{t}} \sum_{i, j \in f^{t}(\mathcal{R}, \mathcal{M})} p_j q_i
\end{equation}

where $p_j$ is the price for product $m_j$, and $q_i$ is the quantity sold in region $r_i$ and $T$ is the future time horizon of analysis. The main idea of the current work is to discover the long-term, optimal allocation policy, $f^{\ast}$ from data.

\subsection{Optimal Allocation as a Markov Decision Process}
We believe that the optimal allocation problem is well suited for reinforcement learning because the RL agent is designed for sequential decision making that maximizes expected discounted reward over time. We frame the inputs as a Markov Decision Process (MDP). An MDP is defined by the tuple $\langle \mathcal{S}, \mathcal{A}, P, r, \delta  \rangle$, where $\mathcal{S}$ is the state space, $\mathcal{A}$ is the set of possible actions, $P$ is the (typically unkown) state transition function, $r$ is the reward function and $\delta \in [0,1]$ is the discount factor. 

\begin{itemize}
    \item \textbf{State} At each time, $t$, we observe the state of the retail environment, $\mathcal{E}$. We define the state, $s_t \in \mathcal{S}$, as the tuple of state features, $s_t = \langle \mathcal{Z}^{{t}}, d^{t}, \textbf{g}^{(t-1)}  \rangle$, where $\mathcal{Z}^{{t}}$ is the current board configuration, $d^t$ is the current day of the week (e.g., Sunday $\rightarrow$ 0), and $\textbf{g}^{(t-1)}$ is a vector denoting the revenue at the previous time, $(p_j q_i)^{(t-1)} \forall z \in \mathcal{Z}^t$

    \item \textbf{Action} We define the action space  $\mathcal{A} = \mathcal{R} \times \mathcal{M} \times \{-1, 1\} \cup \{0\}$, indicating ``to place'', ``take way'' or ``do nothing'' for each product, $m_j$ in each region, $r_i$.
    \item \textbf{Reward} The reward function in this case is the total product revenue at time $t$, constrained by the monetary cost, $c$, of placing a set of products in each region:
    \begin{equation}
        r(t) = \sum_{i=1}^n \sum_{j=1}^k p_j q_{ij}^{t} - c \sum_{i=1}^n \mathbbm{1}_{\mathcal{Z}}(r_i)
    \end{equation}
    
    \item \textbf{State transition function}: The state transition, $P$ is defined as $p(s^{t+1} | s^t, a^t): \mathcal{S} \times \mathcal{A} \times \mathcal{S} \rightarrow [0,1]$, which gives the probability of moving to state, $s^{(t+1)}$ given the current state and action. In the optimal allocation problem the exact transition function, $P$ is unknown since the current state, $s^t$ depends on the results of the previous time, $\textbf{g}^{(t-1)}$. We model this transition as a stochastic process.
\end{itemize}
\section{Commonsense for Zero-Shot NLVL}
\label{sec:proposedSection}

\subsection{Problem Formulation}
We denote an input video as $V$, and its grounding annotations as \(\left( Q,V_{\text{span}}\right) \), where $Q$ is the query representation and \(V_{\text{span}}\!=\!\left( t_{s},t_{e}\right)\) is the corresponding video moment span annotation, with \(t_{s}\) and \(t_{e}\) representing the start and end timestamps, respectively. Learning to localize a video moment conditioned on a query entails maximizing the expected log-likelihood of the model parameterized by \(\theta\). In its typical setting, this can be formulated as follows:
\begin{equation}
\label{eq:groundingOriginal}
    \theta ^{\ast }=\arg \max _{\theta } \mathbb{E}\left[ \log p_{\theta }\left(  V_{\text{span}} | V,Q\right) \right]. 
\end{equation}
In the zero-shot setting, the goal is to learn this task without parallel video-query annotations. Hence, the query and video moment annotations are derived from $V$, using a dynamic video moment proposal method followed by a pseudo-query generation mechanism. Formally,  \(V_{\text{span}}\,\!{=}\!\,f_{\text{span}}(V)\) and \(Q\,\!{=}\!\,f_{pq}(V_{\text{span}})\), where $f_{\text{span}}$ and $f_{\text{pq}}$ are video moment proposal and pseudo-query generation mechanisms, respectively. Given that $f_{\text{span}}$ and $f_{\text{pq}}$ are responsible for generating the annotations, the performance of the localization model heavily depends on the quality of these modules. Existing methods face challenges in aligning \(Q\) to \(V_{\text{span}}\) due to noise introduced by ungrounded pseudo-query generation mechanisms. 
To address this, we simplify \(f_{\text{pq}}\) while augmenting cross-modal understanding by leveraging external information in the form of a commonsense graph \(G_{C}(C, E)\) with \(n_c\) nodes, where \(C\!=\!\left\{c_{1}, c_{2}, \dots, c_{n_{C}}\right\}\) are the concept node vector representations and \(E\) is the set of weighted directed edges, respectively. Accordingly, learning can be formulated as
\begin{equation}
\label{eq:groundingOurs}
    \theta ^{\ast }=\arg \max _{\theta } \mathbb{E}\left[ \log p_{\theta }\left(  V_{\text{span}}| V,Q,G_{C}\right) \right].
\end{equation}

\noindent Figure \ref{fig:approach} shows both training and inference flows.
\subsection{Pseudo-supervised Setup}
\modelname first processes a raw video with a video moment proposal $f_{\text{span}}$ module that extracts important video segments capturing key events, and a pseudo-query generation $f_{\text{pq}}$ that generates text query annotations corresponding to the extracted video segments.

\paragraph{Dynamic Video Moment Proposal ($f_{\text{span}}$).}
We adopt the dynamic video moment proposal approach proposed by \citet{nam_zero-shot_2021}. Specifically, $f_{\text{span}}$ primarily comprises a k-means clustering mechanism that groups semantically similar and temporally proximal video frame features together to extract atomic moments. To obtain frame features, we consider the columns of a frame-wise similarity matrix derived from the CNN features of individual frames. We enforce temporal proximity by concatenating the frame index to the features. Composite video moments are then formed by combining neighboring atomic moments, and a subset of all possible combinations is sampled uniformly at random. The resulting set of video moments corresponds to $V_{\text{span}}$.

\paragraph{Pseudo-query Generation ($f_{\text{pq}}$).} The pseudo-query is constructed as a collection of objects present in the video. To generate the pseudo-query, we employ an off-the-shelf object detector, enabling the extraction of pertinent objects in \(V_{\text{span}}\). We adopt a top-$k$ strategy to sample the $k$ most probable object predictions associated with the query \query.

\paragraph{Video Encoder.}
We uniformly sample $T$ frames from $V$ and extract their CNN (\eg, I3D~\cite{qian_locate_2022}) features. These features are contextually encoded using a video encoder ${\phi}_{v}$ to yield frame features ${\phi}_{v}(V)\!=\!\left\{ v_{1},v_{2},\ldots,v_{T}\right\}$ where $v_{i}\in\mathbb{R}^{d}$, and $d$ is the common video/query encoding dimension. We implement ${\phi}_{v}$ as a GRU-based encoder.

\paragraph{Query Encoder.}
Our pseudo-query $Q$, composed of up to $k$ tokens, is encoded using a query encoder ${\phi}_{q}$ that generates query embeddings ${\phi}_{q}(Q)\!=\!\left\{ q_{1},q_{2},\ldots,q_{k}\right\}$, for the top-$k$ detected objects extracted from the pseudo-query generation. Here, $q_{i}\in \mathbb{R}^{d}$ and $d$ is the common video/query encoding dimension. We implement ${\phi}_{q}$ as a bi-directional GRU-based encoder preceded by a trainable embedding layer. 

\subsection{Commonsense Enhancement Module}
\label{sec:cem}
To enrich the encoded video and query features with information grounded in commonsensical knowledge, we introduce a Commonsense Enhancement Module (CEM), pictorially described in Figure~\ref{fig:cem}. This enhancement helps inject necessary information into video and query representations, which can not just help bridge the gap between the available visual and textual cues but also provide rich information to the downstream span localization module. 

\begin{figure}[t!]
    \centering
    \includegraphics[width=0.8\linewidth]{figures/figure_files/Cem.pdf}
    \caption{\modelname Commonsense Enhancement Module (CEM). CEM comprises a concept encoder and an enhancement mechanism that uses the previously encoded concept vectors to update a given input vector (video/query vectors). The concept encoder employs a Graph Convolution Network for encoding the nodes (concepts) of \(G_C\). 
    }
  \label{fig:cem}
\end{figure}

CEM includes a set \(C\!=\!\left\{c_{1}, c_{2}, \dots, c_{n_{C}}\right\}\) of \(n_{C}\) concept vectors, where \(c_{i} \in \mathbb{R}^{d}\) and \(d\) is the concept feature dimension (same dimension as $\forall v_i \in V$ and $\forall q_i \in Q$). In general, given source feature vectors $S\!=\!\left\{ s_{1},s_{2},\ldots,s_{n}\right\}$ with individual feature vectors $s_{i \in [1,n]} \in \mathbb{R}^{d}$, the enhanced feature vectors $S_{C}$ are obtained using a commonsense enhancement mechanism $\phi_{C}$.
We implement this commonsense enhancement step $\phi_{C}$ as a cross-attention mechanism that enriches source input features, attending over $S$ guided by the commonsense concept vectors $C$, \ie, 
\begin{equation}
\label{eq:cenhance}
\scalemath{1}{
    }
    S_{C} = S + \phi_{C}(S) = S + \sigma \left( \frac{SW_{Q}(CW_{K})^{T}}{\sqrt{d}} \right) C W_{V},
\end{equation}
where $\sigma$ is a softmax activation, \(W_{Q}\), \(W_{K}\), \(W_{V}\) are trainable matrices and \(d\) is the common dimension of the vectors \(S\) and \(C\). In our setting, the source feature vectors $S$ are either video $V$ or pseudo-query $Q$ features. We build separate enhancement mechanisms for $V$ and $Q$, \ie, the projection matrices \(W_{Q}\), \(W_{K}\), \(W_{V}\) are not shared between $Q$ and $V$. We elaborate more on the rationale in the appendix.
The enriched video and pseudo-query features are denoted as \(V_{C}\!=\!\phi_{C_{\text{vid}}}(V)\) and \(Q_{C}\!=\!\phi_{C_{\text{pq}}}(Q)\), respectively.

\paragraph{Concept Encoder.}
The concept vectors \(C\) mentioned above are feature representations that internally form the nodes of the commonsense graph, \(G_C\). Accordingly, graph \(G_{C}\) is represented as a matrix, where \(G_{C(i,j)}\) represents the total number of directed relational edges between \(c_{i},c{j} \in C\) that start at \(c_i\) and end at \(c_j\). To encode the commonsense information, we employ Graph Convolutional Networks (GCN) \cite{hammond_wavelets_2011}. The concept encoder is composed of $L$ graph convolution layers, each of which performs a convolution step
\begin{equation}
\scalemath{1}{
    C^{\left(l+1\right)}=\sigma \left( AC^{\left(l\right) }W^{\left( l\right) }\right),
    }
\end{equation}
where $C^{\left(l\right)}$ are node (concept) features and $W^{\left( l\right)}$ trainable weight matrix of layer $l \in [1, L]$, $\sigma$ is a nonlinear activation function, and $A$ is the adjacency matrix obtained by normalizing graph $G_C$ with the degree matrix $D$. Since $G_C$ is a directed graph, normalization can be formulated as $A\!=\!D^{-1}G_{C}$.

\paragraph{Commonsense Information.}
We use ConceptNet \cite{speer_conceptnet_2017}, a popular knowledge graph that provides information spanning various types of relationships such as physical, spatial, behavioral, \etc To ensure that the ConceptNet information utilized is relevant to themes found in the video data, we consider the set of objects available in pseudo-queries and include the top-$k$ most frequently occurring objects to be the seed concept set \(C\). We extract the  ConceptNet subgraph that includes all edges incident between the concepts in \(C\). 
We filter the edge types based on a pre-determined relation set \(R\), which is compiled to involve relations that are relevant to the nature of the video localization task, \eg, spatial (\textit{AtLocation}, \etc) and temporal (\textit{HasSubevent}, \etc) relations are useful for video understanding, while \textit{RelatedTo} and \textit{Synonym} are fairly generic relations that add little information to the localization task. Table \ref{tab:relations} shows the relations included in \(G_C\).

\paragraph{Cross-Modal Interaction Module.} The commonsense enriched video and query features, \(V_{C}\) and \(Q_{C}\), are fused with a multi-modal cross-attention mechanism. We employ a two-step fusion process. First, Query-guided Video Attention (QVA) is applied to attend over video $V_C$, and Video-guided Query Attention (VQA) attends over query $Q_C$ guided by video $V_C$, resulting in updated features $V_C'$ and $Q_C'$, respectively. Both QVA and VQA utilize Attention Dynamic Filters~\cite{rodriguez_proposal-free_2020} that adaptively modify video features, dynamically adjusting them in response to the query, and vice versa. Next, the attended features are fused using a cross-attention mechanism over $V_C'$ guided by $Q_C'$, resulting in localized video features $V_{C_{\text{loc}}}$.

\paragraph{Temporal Regression Module.}
The final step involves a regression layer that approximates $\hat{V}_{\text{span}}$. We employ attention-guided temporal regression to estimate the span of the target video moment. To find important temporal segments relevant to the query, the fused features $V_{C_{\text{loc}}}$ are temporally attended based on the query features to obtain $V_{\text{ta}}$. Then, the span boundaries are localized using a regressor implemented as a Multi-Layer Perceptron (MLP).

\begin{align}
{o}_i = \sigma\left({W}_{1} V_{C_{\text{loc}_i}} + {b}_{{1}}\right) \\
V_{\text{ta}} = \sum_{i=1}^{T} o_i V_{C_{\text{loc}_{i}}} \\
[\hat{t}_s, \hat{t}_e] = {W}_2 {V}_{\text{ta}} + {b}_{2}.
\end{align}
Here, ${W}_{1}$ and ${b}_1$ are the weight matrix and bias vector of the temporal attention MLP, $\sigma$ represents the sigmoid activation function, $V_{C_{\text{loc}_i}}$ stands for the encoded localized video features, ${V}_{\text{ta}}$ represents the temporally attended video features, ${W}_2$ and ${b}_2$ denote the weight matrix and bias vector of the regression MLP, and $[\hat{t}_s, \hat{t}_e]$ correspond to the start and end timestamps of the predicted video span $\hat{V}_{\text{span}}$.

\begin{table}[t!]
\centering
\resizebox{\linewidth}{!}{
\begin{tabular}{ll}
\toprule
\textbf{Category} & \textbf{Relations}                                                                                         \\ \toprule
Spatial           & AtLocation, LocatedNear                                                                                    \\ \midrule
Temporal          & \begin{tabular}[c]{@{}l@{}}HasSubevent, HasFirstSubevent, HasLastSubevent, HasPrerequisite\end{tabular} \\ \midrule
Functional        & UsedFor                                                                                                    \\ \midrule
Causal            & Causes                                                                                                     \\ \midrule
Motivation        & MotivatedByGoal,  ObstructedBy                                                                             \\ \midrule
Other             & CreatedBy, MadeOf                                                                                          \\ \midrule
Physical          & \begin{tabular}[c]{@{}l@{}}HasA, HasProperty, Antonym, SimilarTo\end{tabular}                      
\\ \bottomrule
\end{tabular}
}

\caption{Relations in the Commonsense Enhancement Module (CEM) grouped by categories.}
\label{tab:relations}

\end{table}
\subsection{Training and Inference}
The training objective is 
$\mathcal{L}_{loc} = \mathcal{L}_{treg}+\lambda \mathcal{L}_{ta},$ where \(\lambda\) is a balancing hyperparameter, \(\mathcal{L}_{ta}\) is a temporal attention guided loss and \(\mathcal{L}_{treg}\) is the regression loss.  The temporal attention-guided loss is defined as
\begin{equation}
\label{tatt}
\mathcal{L}_{ta} = \frac{\sum^{T}_{i=1}g_{i}\log \left( a_{i}\right)}{\sum^{T}_{i=1}g_{i}},
\end{equation}
where \(a_{i}\) is the attention weight for video frame \(v_{i}\) and \(g_{i}\) is the attention mask for \(v_{i}\), that is assigned to \(1\) if \(v_{i}\) is inside the target video segment, and \(0\) otherwise. 
This objective encourages the model to produce higher attention weights for video segments that are relevant to the query. 
On the other hand, \(\mathcal{L}_{treg}\) dictates the video span boundary regression and is the sum of smooth $\ell_1$ distances between start and end timestamps of the ground truth and predicted spans, \ie,
\begin{equation}
\label{treg}
\mathcal{L}_{treg} = \text{smooth}{\ell_1}(t_{s}, \hat{t}_{s}) + \text{smooth}{\ell_1}(t_{e}, \hat{t}_{e}).
\end{equation}
Here, $t_{s}$ and ${t}_{e}$ represent the ground truth start and end timestamps and $\hat{t}_{s}$ and $\hat{t}_{e}$ the predicted start and end timestamps, respectively.
The integration of a smoothing mechanism enhances training stability and improves the model's ability to handle outliers. Finally, during inference, we employ an off-the-shelf part-of-speech tagger to extract nouns from the text input query and feed them as query input to the trained \modelname video localizer.
\section{Assessment}
\label{sec:assessment}
\subsection{Experimental Setup}
We implement our PCDNet in PyTorch \cite{paszke2019pytorch} and train it for 300 epochs with the batch size of 32 on two NVIDIA GeForce RTX 3090 GPUs. We use stochastic gradient descent (SGD) \cite{amari1993backpropagation} with a momentum of 0.937 and a weight decay of $5 \times 10 ^{-4}$ during training. The initial learning rate is set to 0.01 and decayed to 0.001 using a cosine annealing schedule. We initialize PCDNet randomly and load the weights of CSPDarknet53 \cite{wang2020cspnet} pre-trained on ImageNet \cite{imagenet_cvpr09} for the encoder part. To increase the diversity and complexity of the training samples, we apply data augmentations including random cropping, random flipping, and mosaic \cite{redmon2018yolov3}. We use the evaluation metrics of Microsoft COCO \cite{lin2014microsoft} for validation.

\begin{table}[ht]
\caption{Quantitative comparison against state-of-the-art polarization-based detectors ($\star$), single-stage detectors ($\dag$), two-stage detectors ($\ddag$), anchor-based detectors ($\triangle$), anchor-free detectors ($\circ$), and self-supervised method ($\S$).}
\small
\centering
\renewcommand\arraystretch{0.9}
\setlength{\tabcolsep}{2.6pt}
\begin{tabular}{lccccc}
\hline\hline
Methods	&	Pub'Year	&	Backbone	&	AP	&	AP50	&	AP75	\\
\hline
Faster R-CNN$^{\ddag\triangle}$ 	&	NeurIPS'15	&	Res50	&	44.8	&	75.4	&	45.4	\\
SSD$^{\dag\circ}$ 	&	ECCV'16	&	VGG16	&	25.5	&	52.6	&	22.6	\\
Cascade R-CNN$^{\ddag\triangle}$ 	&	CVPR'18	&	Res50	&	45.8	&	73.2	&	47.8	\\
CornerNet$^{\dag\circ}$ 	&	ECCV'18	&	Res50	&	19.8	&	47.4	&	29.6	\\
P-SSD I$^{\star\dag\circ}$ 	&	ITSC'19	&	VGG16	&	25.9 	&	53.1	&	22.7	\\
P-SSD S$^{\star\dag\circ}$ 	&	ITSC'19	&	VGG16	&	23.0 	&	48.9	&	20.1	\\
FCOS$^{\dag\circ}$ 	&	ICCV'19	&	Res50	&	23.1	&	50.9	&	18.4	\\
DH R-CNN$^{\ddag\triangle}$ 	&	CVPR'20	&	Res50	&	32.7	&	65.3	&	28.2	\\
Dynamic R-CNN$^{\ddag\triangle}$ 	&	ECCV'20	&	Res50	&	46.2	&	74.2	&	48.0	\\
EfficientDet$^{\ddag\triangle}$ 	&	CVPR'20	&	D3	&	45.3	&	73.0	&	46.3	\\
VarifocalNet$^{\dag\circ}$  & CVPR'21 & Res50 & 44.2 &	73.5 &	44.4	\\
D-DETR$^{\dag\circ}$ 	&	ICLR'21	&	Res50	&	43.8	&	74.9	&	44.3	\\
DDOD$^{\dag\circ}$ 	&	MM'21	&	Res50	&	43.5	&	73.0	&	43.3	\\
TOOD$^{\dag\triangle}$ 	&	ICCV'21	&	Res50	&	44.3	&	74.3	&	44.6	\\
YOLOX$^{\dag\circ}$ 	&	arXiv'21	&	YOLOX-l	&	54.3	&	82.5	&	56.7	\\
YOLOv7$^{\dag\triangle}$	&	arXiv'22	&	Dark53	&	57.6	&	84.3	&	60.3	\\
RTMDet$^{\dag\circ}$ 	&	arXiv'22	&	RTMDet-l	&	53.9	&	81.4	&	56.7	\\
DINO$^{\dag\circ\S}$ 	&	ICLR'22	&	Res50	&	52.7	&	81.8	&	54.8	\\
YOLOv8$^{\dag\circ}$ 	&	-'23	&	YOLOv8-l	&	56.8	&	83.6	&	59.0	\\
\hline
\textbf{PCDNet$^\star$}	&	\textbf{Ours}	&	Dark53	&	\textbf{58.5}	&	\textbf{85.2}	&	\textbf{61.5}	\\
\hline\hline
\end{tabular}
\label{tab:comparison}
\end{table}

\begin{figure*}[htp]
    \centering
    \begin{center}
        % \includegraphics[width=\linewidth]{figure/comparison.pdf}
        \includegraphics[width=\linewidth,height=10.5cm]{figure/comparison.pdf}
    \end{center}
    \caption{Qualitative comparison of PCDNet against state-of-the-art detectors retrained on RGB-P Car dataset.} 
    \label{fig:comparison}
\end{figure*}

\subsection{Qualitative and Quantitative Evaluation}
We extensively compare our PCDNet with 19 state-of-the-art methods by retraining and testing all methods on the RGB-P Car dataset using their original settings. The compared methods include two-stage detectors such as EfficientDet \cite{tan2020efficientdet} and the R-CNN family \cite{Ren_2017, Cai_2019, zhang2020dynamic}, and one-stage detectors such as SSD \cite{liu2016ssd}, and YOLO family \cite{ge2021yolox, wang2022yolov7, ultralytics2023yolov8}. These methods also comprise anchor-based methods such as the R-CNN family and YOLOv7 \cite{wang2022yolov7}, and anchor-free methods such as CornerNet \cite{law2018cornernet}, VarifocalNet \cite{zhang2021varifocalnet}, and YOLOv8 \cite{ultralytics2023yolov8}. Some detectors use traditional convolutional networks such as FCOS \cite{tian2019fcos} and RTMDet \cite{lyu2022rtmdet} while others use transformer structures, such as DeformableDETR \cite{zhu2020deformable} and DINO \cite{zhang2022dino} that employs self-supervised learning. We also include the P-SSD \cite{blin2019road} that utilizes polarization information. The quantitative evaluation results are reported in Tab. \ref{tab:comparison}. We can see that our method outperforms all competing state-of-the-art methods. 

Fig. \ref{fig:comparison} further qualitatively demonstrates the benefits of our method: a) in poorly lit indoor parking lots, distinguishing black cars behind pillars is extremely challenging (the first two rows). The compared methods tend to conflate the shadow and the black car (\textit{i.e.}, merging cars on either side of the pillar into a single entity or treating partial views of the car as one object) while our PCDNet can handle such ambiguities; b) in the third example, all methods except our PCDNet fail to detect a partially visible car obstructed by another car or misplace it with the previous car; c) in the fourth example, RGB-based methods wrongly identify distant pedestrians as cars, but our PCDNet method can effectively eliminate such interference with the help of polarization cues; d) the fifth and sixth examples depict black cars in an outdoor parking lot at night which are very hard to be distinguished in the RGB image. Despite the enhancement through ZeroDCE \cite{guo2020zero}, the sixth example remains unclear. By contrast, polarization imaging is robust to low light conditions, enabling our robust car detector PCDNet; and e) the last row shows a virtual car reflected in a mirror located at the upper-left corner of the image. The mirrored virtual car and the rest of the mirror regions exhibit similar and smooth AoLP, providing useful cues for PCDNet to recognize this region as background. 


\subsection{Ablation Study}
\textbf{Impact of Spectral Intensity and Polarization Cues.} We conduct a series of ablation experiments to demonstrate the effects of spectral intensity and polarization cues on car detection (Tab. \ref{tab:abl_input}).
The results show that: a) combining different forms of polarization cues with RGB as the input of PCDNet can improve the car detection accuracy (\textit{C}, \textit{D}, \textit{F}, \textit{G}, \textit{K} and \textit{L} are higher than \textit{B}); b) DoLP cues have a greater impact than AoLP cues (\textit{D}, \textit{J} and \textit{L} are better than \textit{C}, \textit{I} and \textit{K}, respectively); c) stacking AoLP and DoLP on RGB in the channel dimension does not boost performance (\textit{E} is slightly lower than \textit{B}), possibly because the characteristic gap between different modalities hinders effective features extraction; d) spectral intensity and polarization are more beneficial than monochromatic intensity and polarization for car detection (comparing paired \textit{B} and \textit{H}, \textit{C} and \textit{K}, \textit{D} and \textit{L}, \textit{I} and \textit{K}, \textit{J} and \textit{L}); e) enhancing RGB image via ZeroDCE \cite{guo2020zero} is less effective than introducing polarization (\textit{M} performs worse than \textit{C}-\textit{G}, \textit{K} and \textit{L}).
Fig. \ref{fig:abl_input} provides visual support for these observations.

\begin{table}[t]
\small
\centering
\caption{Quantitative comparisons of ablation with different inputs. ``stacked I'' denotes the stacked intensity measurements with a linear polarization angle of 0$^{\circ}$, 45$^{\circ}$ and 135$^{\circ}$ and ``stacked S'' refers to the stacked Stokes elements S0, S1 and S2 \cite{blin2019road}.}
\begin{tabular}{clccc}
\hline\hline
	&	PCDNet Input	&	AP	&	AP50	&	AP75	\\
 \hline
\textit{A}	&	RGB, AoLP and DoLP (original)	&	58.5 	&	85.2 	&	61.5 	\\
\hline
\textit{B}	&	RGB only	&	57.6 	&	84.3 	&	60.2 	\\
\textit{C}	&	RGB and AoLP	&	58.0 	&	84.6 	&	60.7 	\\
\textit{D}	&	RGB and DoLP	&	58.3 	&	85.4 	&	61.1 	\\
\textit{E}	&	stacked RGB, AoLP and DoLP	&	57.5 	&	84.3 	&	59.9 	\\
\textit{F}	&	RGB and stacked I	&	58.0 	&	84.1 	&	61.0 	\\
\textit{G}	&	RGB and stacked S	&	57.8 	&	84.8 	&	60.4 	\\
\textit{H}	&	Gray only	&	57.4 	&	84.3 	&	60.0 	\\
\textit{I}  &   Gray and mono AoLP & 57.5 & 84.5 & 60.5 \\
\textit{J}  &   Gray and mono DoLP & 57.6 & 84.9 & 60.1 \\
\textit{K}	&	RGB and mono AoLP	&	57.9 	&	84.6 	&	60.5 	\\
\textit{L}	&	RGB and mono DoLP	&	58.2 	&	84.9 	&	60.6 	\\
\textit{M}  &   Enhanced RGB & 57.4 & 84.0 & 60.0 \\
\hline\hline
\end{tabular}
\label{tab:abl_input}
\end{table}

\begin{figure}[t]
    \centering
    \includegraphics[width=1\linewidth]{figure/abl_input.pdf}
    \caption{Qualitative comparison of ablation with different inputs. The model with RGB intensity only is susceptible to interference from ghost car caused by water on the road.}
    \label{fig:abl_input}
\end{figure}

\textbf{Influence of PCDNet Components.}
First, we investigate the performance of different strategies for fusing AoLP and DoLP inputs. From Tab. \ref{tab:abl_module}(\textit{A}-\textit{D}), we observe that our PI module is more effective than the simple fusion methods including concatenation, addition and element-wise multiplication.
Second, by removing MP module \ref{tab:abl_module}(\textit{E}) from the original PCDNet (A), the detection performance declines. This demonstrates that exploring the polarized material features of cars across all learning samples is useful. We also explore the influence of applying MSP and MCP on different levels of features. The results in Tab. \ref{tab:abl_module}(\textit{A},\textit{F}-\textit{G}) show that applying MSP on shallower features and MCP on deeper features can yield better performance.
Finally, we validate the effectiveness of CDDQ module.
Removing the CDDQ module (\textit{I}) from PCDNet (\textit{A}), which causes the feature extraction processes of the RGB and polarization to be independent from each other, leads to the performance drop. We also demonstrate the benefits of the CWDA and SDMD in the CDDQ module by removing either of them (\textit{J} and \textit{K}). 

\begin{table}[t]
\small
\centering
\caption{Quantitative comparisons of ablation with different modules demonstrate that all component of PCDNet contributes to the overall performance. We used sequences of three letters separated by '-' and enclosed in parentheses to represent different combinations of MSP and MCP.}
\begin{tabular}{clccc}
\hline\hline
	&	Ablation	&	AP	&	AP50	&	AP75	\\
 \hline
\textit{A}	&	PCDNet (original)	&	58.5 	&	85.2 	&	61.5 	\\
\hline
\textit{B}	&	Input RGB and [AoLP DoLP]	&	58.2 	&	85.4 	&	60.9 	\\
\textit{C}	&	Input RGB and AoLP+DoLP	&	58.1 	&	84.8 	&	60.5 	\\
\textit{D}	&	Input RGB and AoLP*DoLP	&	58.1 	&	84.8 	&	60.5 	\\
\hline
\textit{E}	&	A \textit{w/o} MP	&	56.9 	&	84.2 	&	59.2 	\\
\textit{F}	&	A \textit{w/} M(S-S-S)P	&	58.2 	&	85.2 	&	60.8 	\\
\textit{G}	&	A \textit{w/} M(S-C-C)P	&	58.2 	&	85.0 	&	60.9 	\\
\textit{H}	&	A \textit{w/} M(C-C-C)P	&	58.1 	&	85.0 	&	61.1 	\\
\hline
\textit{I}	&	A \textit{w/o} CDDQ	&	58.0 	&	84.7 	&	60.8 	\\
\textit{J}	&	A \textit{w/o} SDMD	&	58.2 	&	85.2 	&	60.8 	\\
\textit{K}	&	A \textit{w/o} CWDA	&	58.3 	&	85.1 	&	61.1 	\\
\hline\hline
\end{tabular}
\label{tab:abl_module}
\end{table}

\subsection{Limitations}

When both the RGB intensity and the polarization measurement yield weak car signals, our method's effectiveness declines. Specifically, in low-light scenarios, when a car approaches on an unlit road, the strong light from its headlights can create a ``hole'' in both the RGB and polarization and obscure the entire car. We illustrate such an example in Fig. \ref{fig:failure} where the extreme HDR and heavy motion blur in the captured image limit its depiction of both RGB and polarization. In these challenging scenarios, prior RGB-based methods and even human vision are powerless.

\begin{figure}[t]
    \centering
    \includegraphics[width=1\linewidth]{figure/failure.pdf}
    \caption{PCDNet has limited ability to handle extreme HDR or heavy motion blur cases.}
    \label{fig:failure}
\end{figure}

\section{Related Work}
\label{sec:related-work}

\paragraph{Datasets.}
The lack of cross-file and cross-project (e.g. dependencies) information is a general issue in current evaluation datasets for code.
In terms of code completion, common choices are Py150 \citep{raychev2016probabilistic} for Python and Github Java Corpus \citep{allamanis2013mining} for Java. Both datasets are constructed at file level, where source files are isolated from their project and dependencies and no consideration of project separation is taken in constructing training and test sets.
\citet{lu2022reacc} constructed a code completion dataset from CodeNet \citep{puri2021project}, which contains coding problems and solutions from online judge websites and also lacks project context. 
\citet{clement2021long} presented a real-world Python method generation task based on CodeSearchNet \citep{husain2019codesearchnet} but the auxiliary information they extract still comes from within a local file. 
\citet{svyatkovskiy2021fast} constructed a completion dataset based on top Python repositories on GitHub and released the URLs for these repositories. 
However, those repositories are not write-protected and can change over time. Besides, setting up the dependency environments at scale for further analysis is not easy. 
Both make their dataset difficult to reproduce.
In the contrast, we release the code and the dependencies for the projects to ensure reproducibility.
Apart from code completion, datasets for other code tasks such as Cloze test \citep[e.g.][]{feng2020codebert}, code refinement \citep[e.g.][]{tufano2019empirical, yasunaga2021break, haque2022fixeval}, and generating code from text descriptions \citep[e.g.][]{chen2021evaluating, hendrycks2021measuring, austin2021program}, are often small and mostly without project-level code context. 
Beyond-local information is beneficial for programmers to solve programming tasks in real-world settings. The lack of such information in the current dataset would restrict the progress into high-level semantic understanding and reasoning in the code domain.


\paragraph{Code language models.}
Encouraged by the success of pretrained language models in natural language processing \citep{devlin2019bert, liu2019roberta, lewis2019bart, raffel2020exploring} and the promise of naturalness in code \citep{hindle2016naturalness, allamanis2018survey}, we have seen rising adaptations of language models for code. For example, CuBERT \citep{kanade2020learning} and CodeBERT \citep{feng2020codebert} are pretrained based on masked language modeling. GPT-C \citep{svyatkovskiy2020intellicode} and CodeGPT \citep{lu2021codexglue} are both pretrained based on unidirectional language modeling. PLBART \citep{ahmad2021unified} and CodeT5 \citep{wang2021codet5} are pretrained encoder-decoder structures which adopts denoising objectives and can support code understanding and code generation. UnixCoder \citep{guo2022unixcoder} combines the above three pretraining objectives for a unified pretrained model. 




\paragraph{Code completion.}
Code completion is an essential feature for modern IDEs and an important topic for code intelligence. 
In recent years, deep neural networks \citep{liu2016neural, li2018code, alon2020structural, liu2020multi, kim2021code}, especially pretrained language models \citep{svyatkovskiy2020intellicode, lu2021codexglue} become the mainstream solution to this task. 
Still, incorporating additional information proved beneficial.
One popular choice is abstract syntax tree, e.g. \citet{kim2021code, peng2021could, guo2022unixcoder}. 
However, \citet{lopez2022ast} suggested that pretrained code language models may have already encoded the syntax.  
Other proposals seek to use data flow graph, control graph, and various graph relations, e.g. \citet{guo2020graphcodebert, hellendoorn2019global}.
However, information is still restricted from a single file.
We instead try to enhance the model with out-of-file information, similar to what is accessible in a development environment.

For project-level analyzer induced information, \citet{svyatkovskiy2021fast} described a way to use a static analyzer to refine completion candidates from neural methods.
\citet{weyssow2020combining} considered leveraging the project-wise contexts via embeddings for better function call completion performance.
Other than code completion, project-level information has been utilized for methods name prediction~\citep{wang2021lightweight} and generating code from text descriptions~\citep{lyu2021embedding}.
However, none of them tested their approaches with pretrained code language models. 
In terms of incorporating additional context through concatenation,
\citet{clement2021long} reported improvements from prioritize certain parts of in-file context.
Recently, \citet{lu2022reacc} proposed to enhance code language models by concatenating similar code fragments retrieved by a neural network. Despite the general similarity, we 1) use a simple lightweight way to retrieve auxiliary information instead of training a heavy retriever; 2) do not restrict ourselves on similar code fragments and show that dissimilar code fragments (function implementation) can be helpful; 3) explore task-specific fine-tuning with retrieved information for better completion.





% \vspace{-1em}
\section{Conclusions}
% \vspace{-1em}
In this paper, we introduced a benchmark task for commonsense reasoning that aims at uncovering unspoken intents that humans can easily uncover in a given statement by making presumptions supported by their common sense. In order to solve this task, we propose
CORGI (COmmon-sense ReasoninG by Instruction),  a neuro-symbolic theorem prover that performs commonsense reasoning by initiating a conversation with a user. CORGI has access to a small knowledge base of commonsense facts and completes it as she interacts with the user. We further conduct a user study that indicates the possibility of using conversational interactions with humans for evoking commonsense knowledge and verifies the effectiveness of our proposed theorem prover.
% We defined common-sense reasoning as the process of finding a chain of reasoning in a logic program given an if/then/because statement. We showed that obtaining the because statement is crucial in extracting a relevant chain of reasoning given an if/then statement. Moreover, we introduced a soft backward chaining algorithm that allows us to combat variations in natural language by learning embeddings for the facts and rules in the knowledge base. This algorithm combines symbolic AI with neural approaches allowing us to bridge a gap between symbolic AI and the recent advances in deep learning.

\bibliographystyle{aaai}
\bibliography{main}

\end{document}

Previous studies on street view change detection have primarily focused on comparing pairwise images from identical locations but at different times, utilizing pixel-level annotations \cite{Sakurada2015ChangeDF,sakurada2020weakly}, which is similar to change detection tasks using satellite imagery \cite{Leenstra2021SelfsupervisedPE,Shi2021ADS}. Recent works have also demonstrated the applicability of street view pairwise change detection by collecting large-scale historical street view datasets and apply them on a range of urban applications, such as mapping out physical improvements and declines in cities, as well as correlating with socio-economic attributes and neighborhood gentrification status \cite{Naik7571, Huang2022DetectingNG}. 

However, unlike satellite imagery, street view imagery can be more susceptible to noisy signals, such as varying camera angles and noisy background elements like shadows and lighting changes. Additionally, existing street view datasets for change detection tasks are often limited in both spatial and temporal scales due to the fact that pixel-level semantic annotations can be costly. Such constraints hamper the model's generalizability and scalability, making it challenging to directly apply them to downstream tasks and thus restricting their potential social impacts. 
% Additionally,  In, these works fall short in providing an open, transferable model and corresponding benchmark datasets. 

To address these challenges, we first introduce a comprehensive multi-city street view time series dataset with image-level semantic labels, and then propose an end-to-end framework to detect urban change with street view data. We demonstrate the effectiveness of our approach with a fine-grained assessment of urban change across Seattle, Washington. Specifically, our major contributions in this study are threefold: 
\begin{itemize}
    \item We collect and curate a Google Street View (GSV) time series dataset, covering more than $1000$ coordinates across $6$ different cities, which is the largest street view change detection dataset available up to date. Each street view time series is labeled with change or no change on the image level, and each series has an average length of $10$ images, covering a time interval of $16$ years (from $2007$ to $2023$). We further validate the benefits of the time series data over pairwise data in our experiment.
    \item We propose an end-to-end change detection pipeline that effectively learns feature representations with semantic contexts from street view time series data, which allows the model to not only extract object shape, color, and structural information of the built environment, but also mitigate noisy effects from lighting changes and angle misalignment, enhancing the overall robustness of the change detection.
    \item Our method enables scalable applications for urban scene change detection, providing a more accurate proxy for assessing neighborhood socio-economic status changes. We demonstrate the efficacy of our approach by evaluating its correlation with social-demographic data and comparing against construction permits through a case study in Seattle.
\end{itemize}
% Figure showing the assessment visualization \todo{Figure}.

\section{Related work}
% \textbf{Engagement with literature: Shows an excellent understanding of other literature on the problem, including that outside computer science.}
\subsection{Urban change assessment}
Measuring physical change in urban environments offers profound insights into urban policies and economics, illuminating housing value trends, shifts in urban areas' roles, and spatial segregation effects \cite{Temkin1996NeighborhoodCA,Hwang2020UnequalDG}. It also has significant values in a various downstream tasks, such as detecting neighborhood gentrification \cite{hwang_gentrification}, monitoring the disaster recovery \cite{Stevenson2010UsingBP}. Prior research primarily utilizes satellite imagery for large-scale urban change detection \cite{Pandey2015UrbanizationAA,Daudt2018UrbanCD}. However, satellite data are hindered by constraints in spatial and temporal resolution, and lack detail on fine-grained and street-level changes. Several researchers use building permits data as a fine-grained proxy for physical urban change \cite{Stevenson2010UsingBP,Strauss2013DoesHD}. While these data also have limitations in availability and spatial coverage and may not accurately represent actual changes due to potential delays, as indicated by our evaluations.

\subsection{Street view imagery}
Street view imagery have been used in a wide range of applications in urban studies, such as quantifying urban greenery \cite{Li2015AssessingSU}, indicating region functions \cite{Gong2019ClassifyingSS}, revealing economic and social-demographic patterns\cite{gebru2017using,wang2020urban2vec,tian2021}, predicting populace' well-beings \cite{Lee2021PredictingLI} and estimating building energy efficiency \cite{MAYER2023120542}. 
It demonstrates substantial value as a medium closely reflecting human perception of the city. 
Moreover, recent studies analyzed historical street view data in the temporal dimension to map physical improvement and decline in the built environment and uncover how cities changed over time \cite{Naik2014StreetscoreP,Naik7571,Huang2022DetectingNG}. However, existing methods rely on comparing pairs of historical street views for each location rather than a comprehensive time series of street view data, which is limited in tracking the complete range of transformations within urban environments.

\subsection{Change detection}
Change detection is commonly presented in the field of remote sensing, as the task to identify changes between the pixel-level features of two temporally separated images. Previous works have trained convolutional network, recurrent network and Siamese network for change detection task on satellite imagery \cite{Gong2016ChangeDI,Daudt2018FullyCS,Lyu2016LearningAT}. Recent works also explored self-supervised pretraining and unsupervised learning in change detection to rely less on labels and generate meaningful representations for other downstream tasks \cite{Cong2022SatMAEPT,Mall2022ChangeED}. Unlike satellite imagery, Street view change detection faces additional challenges such as noisy signals including shades and angle misalignments due to the less fixed and more variable nature of acquiring street-level visual data. Previous works introduced benchmarking datasets for scene change detection and adopted deconvolutional networks or temporal attention networks \cite{Alcantarilla2016StreetviewCD,Sakurada2015ChangeDF,Chen2021DRTANetDR}. However, current research not only lacks comprehensive benchmark datasets with expansive spatial and temporal coverage, but also falls short of the model generalizability, limiting their societal impact. To tackle this, we introduce the largest image-level change detection dataset to date, featuring a complete time series of street views for each sampled location, applied at the city scale.

\begin{figure*}[h!]
    \centering
    \includegraphics[width=0.98\linewidth]{figure/data_geo.png}
    \caption{Geo-spatial distribution of our street view time series dataset across 5 different cities in the US. Locations are selected based on open-access building footprint data, and historical Google Street View imagery from these coordinates is comprehensively downloaded and labeled with urban change points.}
    \label{fig:data_geo}
\end{figure*}

% \begin{figure*}[h!]
%     \centering
%     \includegraphics[width=0.98\linewidth]{figure/data_geo.png}
%     \caption{Geo-spatial distribution of our street view time series dataset. We sample street view time series data across 5 different cities in the US. Each location is determined through open-access building footprint data, and we download all the historical Google Street View that recorded there.}
%     \label{fig:data_geo}
% \end{figure*}
% \subsection{Self-supervised Pretraining}
\section{Methods}
% \textbf{Novelty of approach:Introduces a new model, data gathering technique, algorithm, and/or data analysis technique.}
\subsection{Problem Statement}
\begin{definition}
\textbf{Street view time series.}
Each street view time series, comprises $n$ street view images depicting the consistent street-level scene $s^{(i)}=(s^{(i)}_{1}, s^{(i)}_{2}, ... , s^{(i)}_{n})$. These images are chronologically arranged such that $s^{(i)}_{k}$ corresponds to the timestamp $t^{(i)}_k$.
\end{definition}

\begin{definition}
\textbf{Urban change point.}
In the street view time series $s^{(i)}$, the image $s^{(i)}_{c}$ is identified as an urban change point if the built environment in $s^{(i)}_{c}$ exhibits deviations (e.g., building constructions) relative to preceding images.
\end{definition}

Our objective is to accurately and efficiently detect urban change points in street view time series. To achieve this, we begin by creating a large street view time series dataset, followed by proposing an end-to-end training and evaluation pipeline.

\subsection{Street view time series dataset}
To start, we sample the geospatial coordinate for each street-level scene $s^{(i)}$. Specifically, the coordinate of each building is determined by computing the centroid of its footprint polygon, using the Microsoft building footprint dataset.
After locating scene $s^{(i)}$, we gather all the available historical street view metadata and subsequently download the associated images using their panoid ID. For each scene $s^{(i)}$, we retrieve the nearest-photographed panorama. The image heading is subsequently determined, facing the building from the panorama's coordinate. All our street view images and meta data are sourced from the Google Static Street View API.
\begin{figure}[t]
    \centering
    \subfloat[relationship among scenes]{\resizebox{0.22\textwidth}{!}{
        \includegraphics[]{figure/chord.png}
    }}
    \hspace{4mm}
    \subfloat[car instance $ln$ distribution]{\resizebox{0.22\textwidth}{!}{
        \includegraphics[]{figure/polar.png}
    }}
    \caption{The images in our RGB-P Car dataset vary in terms of (a) scenarios and (b) the number of car instances.}
    \label{fig:dataset}
\end{figure}

\begin{table}[tp]
\caption{Comparison of existing car detection datasets with polarization measurements.}
\small
\centering
\setlength{\tabcolsep}{2.6pt}
\begin{tabular}{c|c|c|c|c}
\hline\hline
Datasets         & Pol. & \begin{tabular}[c]{@{}c@{}}Pixel\\ align\end{tabular} & \begin{tabular}[c]{@{}c@{}}Num.images \\ Train / Test\end{tabular} & \begin{tabular}[c]{@{}c@{}}Num. cars \\ Train / Test \end{tabular} \\ 
\hline
PolarLITIS       & Mono & $\times$                                                     & \begin{tabular}[c]{@{}c@{}}2569 \\ 1640 / 929 \end{tabular}         & \begin{tabular}[c]{@{}c@{}}17428\\ 6061 / 11367 \end{tabular}    \\
\hline
\textbf{RGBP-Car (Ours)} & Tri  & \checkmark                                                     & \begin{tabular}[c]{@{}c@{}}2601 \\ 1611 / 990 \end{tabular}         & \begin{tabular}[c]{@{}c@{}}31234 \\ 19582 / 11652 \end{tabular}   \\ 
\hline\hline
\end{tabular}
\label{tab:datasetcomp}
\end{table}


\section{RGB-P Car Detection Dataset}
\label{sec:dataset}
We construct the first pixel-aligned RGB-polarization car detection dataset called RGBP-Car with trichromatic polarization measurements. We record cars in diverse traffic scenes using FLIR-Blackfly-S, a polarized color camera that simultaneously obtain pixel-aligned polarization measurements in four linear polarization directions (0$^\circ$, 45$^\circ$, 90$^\circ$, and 135$^\circ$) for each color channel (\textit{i.e.}, R, G, and B). RGBP-Car contains 2601 RGB, AoLP, and DoLP image triplets. Each image has manually labeled bounding boxes indicating the position and size of each car. To ensure the diversity and challenge of our dataset, we take the RGB-P images under different weather conditions (clear and rainy), different lighting conditions (daytime and nighttime), different driving environments (indoor, outdoor, road and parking lot), and different car densities. 
Fig. \ref{fig:samples} gives representative examples and Fig. \ref{fig:dataset} analyzes (a) the relationship among different scenes and (b) the density distribution of car instances. Tab. \ref{tab:datasetcomp} further shows the superiority of our RGBP-Car over existing car detection datasets with polarization measurements.



In total, we select $931$ locations and retrieve their corresponding street view time series, which consist of $10,878$ images. We then annotate each time series $s^{(i)}$ to identify the urban change points $s_c^{(i)}$. Among them, $371$ time series have been labeled with a total of $433$ urban change points, while the remaining ones exhibit no substantial urban change. Figure \ref{fig:data_geo} demonstrates the geo-spatial distribution of our sampled street views. 
As is shown in table \ref{table:data_compare}, our dataset not only consists of more image pairs compared with previous scene change detection datasets TSUNAMI \cite{Sakurada2015ChangeDF} and PSCD \cite{sakurada2020weakly}, but also covers a significantly broader spatial and temporal scope. 


\subsubsection{Data partitioning.}
To train our change detection model on street view time series dataset labeled with urban change points, we introduce a partitioning scheme for generating training and evaluation sets as is shown in Figure \ref{fig:data_method}. For every street view time series $s^{(i)}$, we segment the views from $s_1^{(i)}$ to $s_n^{(i)}$ based on their occurrence relative to $s_c^{(i)}$. More precisely, suppose the time series $s^{(i)}$ has $q$ urban change points denoted as $s_{c_1}^{(i)}$, $\ldots$, $s_{c_q}^{(i)}$, we allocate the street view $s_j^{(i)}$ ($1\leq j\leq n$) to segment $seg(s_j^{(i)})$ as follows:
\begin{equation}
    seg(s_j^{(i)})=
    \begin{cases}
    1 & \text{if}\ j<c_1\\
    k & \text{if}\ c_1\leq j < c_q\ \text{and}\ c_{k-1}\leq k<c_k\\
    q+1 & \text{if}\ j \geq c_q
    \end{cases}
\end{equation}
\begin{figure}
    \centering
    \includegraphics[width=1.01\linewidth]{figure/data_method.png}
    \caption{Partitioning of street view time series data. All possible pairwise combinations of street view samples are generated from each time series. Each pair's label is assigned based on its corresponding position with the urban change points.}
    \label{fig:data_method}
\end{figure}

% \begin{figure}
%     \centering
%     \includegraphics[width=1.01\linewidth]{figure/data_method.png}
%     \caption{Street view time series data partitioning. We generate all the combinations of pairwise street view samples by partitioning each street view time series. Each pair's label is determined based on their corresponding position with the change point labels.}
%     \label{fig:data_method}
% \end{figure}
\begin{figure}[h!]
    \centering
    \includegraphics[width=1.05\linewidth]{figure/change_model.png}
    \caption{Overview of the change detection model architecture. Pairs of input images are processed using Siamese-based networks with DINOv2 as the backbone. The CLS tokens serve as the image representation, with a subsequent linear layer projecting them to a prediction score.}
    \label{fig:change_model}
\end{figure}


% \begin{figure}[h!]
%     \centering
%     \includegraphics[width=1.05\linewidth]{figure/change_model.png}
%     \caption{Overview of the change detection model architecture. Each pair of input images are passed through a Siamese-based networks with DINOv2 as backbone modules, we treat the CLS token as the representation of each image and add a linear layer to preject into a prediction score.}
%     \label{fig:change_model}
% \end{figure}
For each street view time series $s^{(i)}$, we then generate a set of pairwise street view pairs by considering all combinations from $s_1^{(i)}$ to $s_n^{(i)}$. Each pair of samples is sorted in chronological order based on their timestamps, resulting in pairs like $(s_1^{(i)}, s_2^{(i)})$. The total number of such combinations for the time series $s^{(i)}$ with $n$ street views is $\binom{n}{2}$. The labeling of these pairwise samples is determined by their associated segments as follows:
\begin{equation}
    \textsc{label}{(s_a^{(i)},s_b^{(i)})}=
    \begin{cases}
        1 & \text{if}\ seg(s_a^{(i)})\neq seg(s_b^{(i)})\\
        0 & \text{if}\ seg(s_a^{(i)}) = seg(s_b^{(i)})
    \end{cases}
\end{equation}
\subsection{Change detection model}
To classify each pairwise pair $(s_a^{(i)},s_b^{(i)})$, we adopt a Siamese network to include a twin DINOv2 \cite{Oquab2023DINOv2LR} backboned module to realize a non-linear embedding from the input domain and a final linear layer transforming the concatenation of both images' hidden vectors and their distance, represented by their element-wise difference, into a scalar predictor as follows:
\begin{equation} \label{eq: sia_concat}
\mathbf{h}^{(i)}_{L} = \left[(\mathbf{h}^{(i)}_{l, L})^\top, (\mathbf{h}^{(i)}_{e, L})^\top, (\mathbf{h}^{(i)}_{l, L}-\mathbf{h}^{(i)}_{e, L})^{\top}\right]^{\top}
\end{equation}
Figure \ref{fig:change_model} visualizes the model architecture. We adopt a cross-entropy loss function to train such a urban change classifier, and let $\textsc{label}{(s_a^{(i)},s_b^{(i)})}$ be the label for the street view pair $(s_a^{(i)},s_b^{(i)})$.

% in the following form \todo{finalize loss function}:
% \begin{multline} \label{eq: bce}
%     \mathcal{L_S}(t^{(i)}) = \mathbf{y}(t^{(i)})\log\mathbf{p}(t^{(i)})+\\ (1-\mathbf{y}(t^{(i)}))\log(1-\mathbf{p}(t^{(i)})).
% \end{multline}

\section{Experiments}
% \textbf{Justification of approach: Thoroughly and convincingly justifies the approach taken, explaining strengths and weaknesses as compared to other alternatives.}\\
% \textbf{Quality of evaluation: Evaluation was exemplary: data described the real world and was analyzed thoroughly.}
To evaluate the efficacy of our proposed method, we conduct experiments from three perspectives: 1) Backbone models --- we benchmark the performance of selected visual foundational models in the context of street view change detection tasks. 2) Street view time series data --- we employ experiments to substantiate the advantage of time series data as a natural form of data augmentation \cite{seco}, compared with results achieved through artificial data augmentation. 3) Self-supervised pre-training --- we explore 2 pre-train methodologies using a larger-scale unlabeled street view dataset in order to evaluate the performance of a domain-specific pre-trained models for our change detection task.
% Firstly, we evaluate the performance of diverse generic visual foundational models in the context of street view image change detection tasks. 
% Subsequently, we employ experiments to substantiate the advantage of time series imagery, a natural form of data augmentation within our collected dataset, which not only showcases superiority over pairwise datasets but also significantly outperforms results achieved through artificial data augmentation. Lastly, we introduce two pre-train methodologies, which are applied to a larger-scale unlabeled dataset of street-level images. This enables a comparative analysis between domain-specific pre-trained models and generic foundational models concerning their respective performances in the street view image change detection task.
\subsection{Training details}
% Our labeled dataset comprises a total of $371$ street view time series, comprised of $4,465$ images, sampled from $371$ distinct locations across $5$ different cities. By employing the data partitioning methodology described in Figure \ref{fig:data_method}, the dataset can be transformed into a collection of $25,423$ image pairs with binary labels. 
Street view images differ from satellite imagery and high-quality object images in that they often have a lower signal-to-noise ratio, primarily due to varying camera positions and environment conditions as shown earlier. As a result, evaluating on a small-scale test set could suffer from a significant variance. To ensure a robust assessment, we randomly select $50\%$ of the street view time series in our dataset as the test set. It includes $25$ locations in Seattle, $13$ locations in San Francisco, $21$ locations in Oakland, $97$ locations in Los Angeles, and $29$ locations in Boston, constituting a total of $12,221$ image pairs. The remaining data are allocated with $90\%$ as the training set for model fine-tuning and $10\%$ as the validation set.
% which to $185$ temporal sequences, as the test set. It includes $25$ locations in Seattle, $13$ locations in San Francisco, $21$ locations in Oakland, $97$ locations in Los Angeles, and $29$ locations in Boston, constituting a total of $12,221$ image pairs. The remaining data are allocated with $90\%$ as the training set for model fine-tuning and $10\%$ as the validation set.
During fine-tuning, we employ the Adam optimizer to train models with a learning rate set at $1 \times 10^{-5}$ and a batch size of $16$. The global norm of gradients is clipped to be $\leq 0.5$, and we use random weight averaging for optimization. Our training and evaluation are conducted on $4$ Nvidia Tesla T4 GPUs. For all the backbone models, we experimented with two common approaches: global fine-tuning and linear probing, i.e. training with the backbone network frozen.

\subsubsection{Backbone models.}
We initiate our evaluation by assessing the performance of 4 pre-trained generic visual models—ResNet101 \cite{he2016deep}, DINO \cite{dino}, CLIP \cite{radford2021learning}, and DINOv2 \cite{Oquab2023DINOv2LR}—as backbone networks. For ViT-based models such as DINO and DINOv2, we experiment using the CLS Token as the backbone output.
% For ViT-based models such as DINO and DINOv2, we experiment using either the CLS Token or the patch features as the backbone output, retaining the best results on the test set. When utilizing patch features as the output, we incorporate both a linear and a non-linear convolutional module ($\textit{Conv2d} \rightarrow \textit{BatchNorm} \rightarrow \textit{ReLU} \rightarrow \textit{Conv2d}$). Their outputs are then summed and dimensionally reduced to $100$ using a $1 \times 1$ convolutional kernel size.
% Specifically when using patch features as the output, we introduce a linear and a non-linear convolutional module, followed by the summation of their outputs, to reduce the dimensionality of the patch features to $100$, with a $1 \times 1$ convolutional kernel size. % The non-linear convolutional module comprises the sequence: $Conv2d \rightarrow BatchNorm \rightarrow ReLU \rightarrow Conv2d$.
% For the prediction layers after the backbone model, we explored both linear prediction layers and non-linear prediction heads composed of $\textit{Linear }\rightarrow \textit{BatchNorm} \rightarrow \textit{ReLU} \rightarrow \textit{Linear}$ sequences. 
% Table \ref{table:backbones} presents a comprehensive display of the best performances achieved by each respective backbone network on our test set.
\begin{table*}[t!]
  \centering
  \resizebox{0.8\linewidth}{!}{
  \begin{tabular}{l|cccc|cccc}
    \toprule
    Models & \multicolumn{4}{c|}{Linear Probing} & \multicolumn{4}{c}{Fine-Tuning} \\
    \cmidrule(r){2-5} \cmidrule(l){6-9}
    & Accuracy & Precision & Recall & F1-Score & Accuracy & Precision & Recall & F1-Score \\
    \midrule
    ResNet101 & 69.18 & 62.43 & \textbf{92.17} & 74.44 & 84.52 & 79.92 & \textbf{91.07} & 85.13 \\
    % DINO (ViT-B/16) & 86.86 & \textbf{92.96} & 78.99 & 85.41 & 87.76 & 90.68 & 83.43 & 86.90 \\
    DINO (ViT-B/16) & 82.33 & 82.24 & 81.25 & 81.74 & 86.24 & \textbf{93.30} & 77.28 & 84.54 \\
    CLIP & 86.01 & 86.95 & 83.85 & 85.37 & 87.06 & 91.52 & 80.91 & 85.89 \\
    DINOv2 (ViT-B/14) & \textbf{88.20} & \textbf{92.08} & 82.88 & \textbf{87.24} & \textbf{88.85} & 92.77 & 83.62 & \textbf{87.96} \\
    \bottomrule
  \end{tabular}}
  \caption{Performance of different backbone models using linear probing and fine-tuning.}
  \label{table:backbones}
\end{table*}


% \begin{table*}[t!]
%   \centering
%   \resizebox{0.56\linewidth}{!}{
%   \begin{tabular}{lclccc}
%     \toprule
%     \textbf{Models} & \textbf{Accuracy} & \textbf{Precision} & \textbf{Recall} & \textbf{F1-Score} \\
%     \midrule
%     ResNet101 & 69.18  & 62.43  & 92.17  & 74.44 \\
%     DINO (ViT-B/16) & 86.86  & 92.96  & 78.99 & 85.41 \\
%     CLIP & 86.01  & 86.95 & 83.85 & 85.37 \\
%     DINOv2 (ViT-B/14) & 88.20 & 92.08 & 82.88 & 87.24 \\
%     \bottomrule
%   \end{tabular}}
%   \caption{Performance comparison of different backbone models using linear probing.}
%   \label{table:backbones}
% \end{table*}

% \begin{table*}[t!]
%   \centering
%   \resizebox{0.56\linewidth}{!}{
%   \begin{tabular}{lclccc}
%     \toprule
%     \textbf{Models} & \textbf{Accuracy} & \textbf{Precision} & \textbf{Recall} & \textbf{F1-Score} \\
%     \midrule
%     ResNet101 & 84.52 & 79.92 & 91.07 & 85.13 \\
%     DINO (ViT-B/16)  & 87.76 & 90.68 & 83.43 & 86.90 \\
%     CLIP & 87.06 & 91.52 & 80.91 & 85.89 \\
%     DINOv2 (ViT-B/14) & \textbf{88.85}  & 92.77 & 83.62 & \textbf{87.96} \\
%     \bottomrule
%   \end{tabular}}
%   \caption{Performance comparison of different backbone models in fine-tunning}
%   \label{table:backbones}
% \end{table*}

% \begin{table*}[t!]
%   \centering
%   \resizebox{0.56\linewidth}{!}{
%   \begin{tabular}{lclccc}
%     \toprule
%     \textbf{Models} & \textbf{Unfrozen} & \textbf{Accuracy} & \textbf{Precision} & \textbf{Recall} & \textbf{F1-Score} \\
%     \midrule
%     ResNet101 & No & 69.18  & 62.43  & 92.17  & 74.44 \\
%     ResNet101 & Yes & 84.52 & 79.92 & 91.07 & 85.13 \\
%     DINO (ViT-B/16) & No & 86.86  & 92.96  & 78.99 & 85.41 \\
%     DINO (ViT-B/16) & Yes & 87.76 & 90.68 & 83.43 & 86.90 \\
%     CLIP & No & 86.01  & 86.95 & 83.85 & 85.37 \\
%     CLIP & Yes & 87.06 & 91.52 & 80.91 & 85.89 \\
%     DINOv2 (ViT-B/14) & No & 88.20 & 92.08 & 82.88 & 87.24 \\
%     DINOv2 (ViT-B/14) & Yes & \textbf{88.85}  & 92.77 & 83.62 & \textbf{87.96} \\
%     \bottomrule
%   \end{tabular}}
%   \caption{Performance comparison of different backbone models based on their unfrozen status.}
%   \label{table:backbones}
% \end{table*}
\begin{table*}[h!]
  \centering
  \resizebox{0.93\linewidth}{!}{
    \begin{tabular}{l|c|c|cccc}
    \toprule
    Data & Data Augmentation & \# pairs & Accuracy & Precision & Recall & F1-Score \\
    \midrule
    Pairwise data & None & 336 & 85.99 & 85.28 & \textbf{86.07} & 85.67 \\
    Pairwise data & HorizontalFlip + ColorJitter + GrayScale + GaussianBlur& 11922 & 85.63 & 88.60 & 80.89 & 84.57 \\
    Time series data & None& 11922 & \textbf{88.85} & \textbf{92.77} & 83.63 & \textbf{87.96} \\
    \bottomrule
    \end{tabular}}
  \caption{Street view time series vs. pairwise data. }
  \label{table:timeseries}
\end{table*}


% \begin{table*}[h!]
%   \centering
%   \resizebox{0.95\linewidth}{!}{
%     \begin{tabular}{cccccccccc}
%     \toprule
%     \# Samples & \# Training Pairs & HorizontalFlip & ColorJitter & GrayScale & GaussianBlur & Accuracy & Precision & Recall & F1-Score \\
%     \midrule
%     2  & 336 &             &             &             &            & 85.99 & 85.28 & 86.07 & 85.67 \\
%     2  & 11922 & \Checkmark  & \Checkmark  & \Checkmark  & \Checkmark & 85.63 & 88.60 & 80.89 & 84.57 \\
%     % 2  & 11922 &             & \Checkmark  & \Checkmark  & \Checkmark & 85.20 & 87.29 & 81.47 & 84.28 \\
%     % 2  & 11922 &             & \Checkmark  & \Checkmark  &            & 85.75 & 90.28 & 79.26 & 84.41 \\
%     % 4  & 12084 & \Checkmark  & \Checkmark  & \Checkmark  & \Checkmark & 86.70 & 93.14 & 78.46 & 85.17 \\
%     % 10 & 12620 & \Checkmark  & \Checkmark  & \Checkmark  & \Checkmark & 87.37 & 91.92 & 81.20 & 86.23 \\
%     all & 11922 &         &         &         &         & \textbf{88.85} & 92.77 & 83.63 & \textbf{87.96} \\
%     \bottomrule
%     \end{tabular}}
%   \caption{Natural time series augmentation vs. artificial random augmentation.}
%   \label{table:timeseries}
% \end{table*}




% \begin{table*}[h!]
%   \centering
%   \resizebox{0.85\linewidth}{!}{
%     \begin{tabular}{ccccccccccc}
%     \toprule
%     \textbf{Samples in one Seq} & \textbf{Backbone} & \textbf{\# Training Pairs} & \textbf{HorizontalFlip} & \textbf{ColorJitter} & \textbf{GrayScale} & \textbf{GaussianBlur} & \textbf{Accuracy} & \textbf{Precision} & \textbf{Recall} & \textbf{F1-Score} \\
%     \midrule
%     2  & DINOv2 & 336 &             &             &             &            & 85.99 & 85.28 & 86.07 & 85.67 \\
%     2  & DINOv2 & 11922 & \Checkmark  & \Checkmark  & \Checkmark  & \Checkmark & 85.63 & 88.60 & 80.89 & 84.57 \\
%     2  & DINOv2 & 11922 &             & \Checkmark  & \Checkmark  & \Checkmark & 85.20 & 87.29 & 81.47 & 84.28 \\
%     2  & DINOv2 & 11922 &             & \Checkmark  & \Checkmark  &            & 85.75 & 90.28 & 79.26 & 84.41 \\
%     4  & DINOv2 & 12084 & \Checkmark  & \Checkmark  & \Checkmark  & \Checkmark & 86.70 & 93.14 & 78.46 & 85.17 \\
%     10 & DINOv2 & 12620 & \Checkmark  & \Checkmark  & \Checkmark  & \Checkmark & 87.37 & 91.92 & 81.20 & 86.23 \\
%     len(Seq) & DINOv2 & 11922 &         &         &         &         & \textbf{88.85} & 92.77 & 83.63 & \textbf{87.96} \\
%     \bottomrule
%     \end{tabular}}
%   \caption{Natural time series augmentation vs. artificial random augmentation.}
%   \label{table:timeseries}
% \end{table*}
\subsubsection{Time series data.}
% with DINOv2 as the backbone network, we utilize the architecture in Figure \ref{fig:change_model} to fine-tune the model. 
To validate the advantages of our proposed street view time series dataset, we constructed a pair-based dataset similar to TSUNAMI \cite{Sakurada2015ChangeDF} and PSCD \cite{sakurada2020weakly} dataset. Specifically, for each street view time series, we randomly sample $2$ images as a pair. We conduct model fine-tuning on this pair-based dataset and evaluated its performance on the test set described earlier.
% In sequences where changes occurred (positive sequences), we ensured the formation of one positive sample pair and one negative sample pair. This resulted in a pairwise dataset comprising $336$ sets of image pairs. 
To align the pair-based dataset with the size of our time series dataset, we randomly apply a combination of standard image augmentation techniques, including horizontal flip, color jitter, grayscale, and Gaussian blur. It seeks to validate our hypothesis that time series images, serving as natural augmentation, are more effective than artificial augmentations to supervise change detection model amidst noisy signals, thus bolstering its robustness.
% Assuming that we randomly sample $n$ pairs of images from time series $s^{(i)}$ (where $n\ mod\ 2 = 0$), the quantity for augmentation $a_{s^{(i)}}$ in $s^{(i)}$ can be determined using:
% \begin{equation}
% \begin{aligned}
%     a_{s^{(i)}} = \frac{n}{2} &\times [\mathrm{Ceil}(\frac{2 \times \text{\# positive image pairs}}{n})\\
%     &+\mathrm{Ceil}(\frac{2 \times \text{\# negtive image pairs}}{n})-2]\\
% \end{aligned}
% \end{equation}
% if $s^{(i)}$ is a positive sequence, or:
% \begin{equation}
% \begin{aligned}
%     &a_{s^{(i)}} = n \times [\mathrm{Ceil}(\frac{\text{\# image pairs}}{n})-1]
% \end{aligned}
% \end{equation}
% if $s^{(i)}$ is a negative sequence.

% We delved into different combinations of random data augmentation techniques and progressively increased the sampling quantity n from $2$ to $10$. This exploration aims to verify our hypothesis that, in contrast to artificial data augmentation, time series images as a form of natural augmentation can better enable the model to adapt to environmental noise, thereby enhancing its robustness. The results are presented in Table \ref{table:timeseries}.

%\subsection{Baselines}
%\begin{itemize}
%    \item[1] DINO (**freeze Resnet**, **freeze CLIP**, **DINO\_v2**) (DINO+?) (heading+batch norm) - zero-shot
%    \item[2] Time Series (vs random augmentation)
%    \item[3] Pretrain (BYOL, segmentation+BYOL, **SatMAE**) (**$10\%$ samples**) 
%\end{itemize}

\subsubsection{Self-supervised pre-training.}
% \todo{Model structure digram}\\
% With the advancement of Natural Language Processing (NLP), an increasing body of research is dedicated to the training of foundational models in Computer Vision. These models possess the capacity to generate generic visual features and enhance performance across downstream tasks. The majority of these foundational models are established through large-scale self-supervised training, 
Recent studies on self-supervised pre-training highlight its efficacy to extract image features when labels are limited and enhance performance in downstream tasks. Specifically, 2 primary branches of self-supervised pre-training are pursued: intra-image self-supervised training \cite{He2022masked,satmae2022} and discriminative self-supervised learning \cite{grill2020bootstrap}. Correspondingly, we adapt 2 pre-training procedures on street view data and benchmark their performance on the change detection task --- StreetMAE and StreetBYOL. StreetMAE uses masked autoencoders \cite{He2022masked} to reconstruct randomly masked patches in street view imagery. It also incorporates temporal encoding to represent each street view time series as a contextual sequence. StreetBYOL, on the other hand, is a self-distillation approach building upon the online and target networks \cite{grill2020bootstrap}. While retaining pivotal components such as the prediction head and the stop gradient mechanism, we try add an unsupervised segmentation head \cite{hamilton2021unsupervised} to identify building pixels and feed them alongside the original images into the networks in experiment seg+StreetBYOL. We adopt the ViT-B/16 architecture as the backbone network and initialize it with the parameters from DINO pre-trained model. Pre-training is conducted on an unlabeled dataset comprising 150,000 street view images randomly sampled in our studied areas. To mitigate noise interference, we apply a filtering process to remove images where the proportion of building pixels was less than $2\%$. After the pre-training phase, we plug it into the Siamese network and fine-tune the model on our labeled training set. 
% Figure \todo{Figure} shows a diagram of two self-supervised learning methods.

% We also introduce a self-distillation approach building upon the BYOL framework \cite{grill2020bootstrap}, which is in the second line of self-supervised computer vision work, i.e., discriminative self-supervised learning. While retaining pivotal components such as the prediction head and the stop gradient mechanism, we employ an unsupervised semantic segmentation method named STEGO \cite{hamilton2021unsupervised} to segment images, isolating architectural elements. Specifically, we input images containing solely the architectural segments, alongside unaltered complete images, into student and teacher networks, respectively. This scheme serves to guide the model in extracting features closely associated with architectural attributes. Additionally, we also explored an alternative approach using diverse random transformations on the original images, separately as inputs to the teacher and student networks. However, empirical results distinctly demonstrate the superiority of the former approach.

% The first approach  based on MAE, involves the random masking of certain patches within images and training the model to reconstruct these masked regions. We extend the MAE framework to sequence level, considering independent patch masking across different images while incorporating temporal encoding during training. 
% Figure \ref{MAE_visual} shows the masked and reconstructed patches. We anticipated that the sequence-level GSVMAE can effectively capture the inter-image contextual information within sequences, thereby attenuating noise that fluctuate temporally. 
% For instance, intuitively, when a patch within an image is masked while the corresponding patch in preceding or succeeding images remains unmasked, the model can leverage information from other images to facilitate patch reconstruction. Consequently, the model becomes adept at encoding more stable features present within the temporal sequence, such as those closely tied to architectural structures.

% i.e. design of pretext tasks \cite{He2022masked}, or through text-guided training \cite{radford2021learning}. Conversely, there exist studies that substantiate the superiority of domain-specific pre-trained models in contrast to generic foundational models, particularly within specific domains and tasks, showcasing heightened performance and superior generalization capabilities \cite{satmae2022}.


% In our work, we embark upon an exploration centered around the domain-specific pre-trained models' potential enhancements vis-à-vis the generic foundation models in the context of street view image change detection tasks.

% As previously elucidated, street view images exhibit a comparatively lower signal-to-noise ratio when contrasted with high-quality object image datasets. While our specific focus on detecting temporal changes in architectural structures, variations stemming from factors such as capturing perspective, lighting conditions, color variations, and vegetation dynamics engender substantial interference in the detection process. Thus, our aspiration is for the SSL methods to acquire an enhanced proficiency in learning object attributes like shape and structure, which facilitates the encoding of visual features that encompass a more comprehensive representation of object structure and spatial positioning. 
\begin{table}[h!]
  \centering
  \resizebox{1.0\linewidth}{!}{
    \begin{tabular}{lcccc}
    \toprule
    Pre-training & Accuracy & Precision & Recall & F1-Score \\
    \midrule
    DINOv2 (ViT-B/14) & \textbf{88.85}  &  \textbf{92.77} & \textbf{83.62} & \textbf{87.96} \\
    StreetMAE & 78.49 & 81.97 & 71.54 & 76.40 \\
    StreetBYOL & 86.25 & 91.98 & 78.62 & 84.78 \\
    Seg+StreetBYOL & 87.42 & 91.03 & 82.27 & 86.43 \\
    \bottomrule
    \end{tabular}}
  \caption{Performance of different pre-train methods.}
  \label{table:pretrain}
\end{table}




% \begin{table}[h!]
%   \centering
%   \resizebox{1.0\linewidth}{!}{
%     \begin{tabular}{clcccc}
%     \toprule
%     \textbf{Freeze} & \textbf{Pre-training} & \textbf{Accuracy} & \textbf{Precision} & \textbf{Recall} & \textbf{F1-Score} \\
%     \midrule
%     \Checkmark & StreetMAE & 74.80  & 81.84  & 61.98  & 70.54 \\
%     \XSolidBrush & StreetMAE & 78.49 & 81.97 & 71.54 & 76.40 \\
%     \Checkmark & StreetBYOL & 83.48 & 91.02 & 73.28 & 81.19 \\
%     \XSolidBrush & StreetBYOL & 86.25 & 91.98 & 78.62 & 84.78 \\
%     \Checkmark & Seg+StreetBYOL & 84.48 & 91.35 & 75.25 & 82.52 \\
%     \XSolidBrush & Seg+StreetBYOL & 87.42 & 91.03 & 82.27 & 86.43 \\
%     \Checkmark & DINOv2 (ViT-B/14) & 88.20 & 92.08 & 82.88 & 87.24 \\
%     \XSolidBrush & DINOv2 (ViT-B/14) & \textbf{88.85}  & 92.77 & 83.62 & \textbf{87.96} \\
%     \bottomrule
%     \end{tabular}}
%   \caption{Best results of different pre-train methods.}
%   \label{table:pretrain}
% \end{table}
\begin{figure}[ht!]
    \centering
    \includegraphics[width=1.0\linewidth]{figure/res_noise.png}
    \caption{Sampled prediction results. Our proposed change detection model effectively identifies structural changes in buildings, while filtering our random variations such like lighting, shadows, vegetation, and vehicles.}
    \label{fig:res_noisy}
\end{figure}


% \begin{figure}[h!]
%     \centering
%     \includegraphics[width=1.0\linewidth]{figure/res_noise.png}
%     \caption{Sampled prediction results. Our proposed change detection model is able to filter out noisy signals such as changes in lighting, shades, vegetation, and vehicles and focus on change of building structures.}
%     \label{fig:res_noisy}
% \end{figure}
\begin{figure*}[h!]
    \centering
    \includegraphics[width=1.01\linewidth]{figure/seattle.png}
    \caption{Assessing urban change in Seattle. \textbf{Left:} Location of approximately $800$k sampled street view images, each represented by a blue dot. \textbf{Middle:} Results from deploying our change detection model on the sampled images to pinpoint urban changes shown in red bounding boxes. \textbf{Right:} Change points, aggregated at the census tract level, with color denoting the proportion of street view time series that have been identified as change.}
    \label{fig:seattle}
\end{figure*}


% \begin{figure*}[h!]
%     \centering
%     \includegraphics[width=1.01\linewidth]{figure/seattle.png}
%     \caption{Assessing urban change in Seattle. \textbf{Left:} We sampled around $800k$ street view images in total and each blue dot indicates its location. \textbf{Middle:} Applying our proposed change detection model directly on all sampled images and identify urban change points. \textbf{Right:} Aggregating detected change points to the census tract level, the color indicates the percentage of sampled time series detected with urban change points.}
%     \label{fig:seattle}
% \end{figure*}
\section{Results and discussion}
As shown in Table \ref{table:backbones}, DINOv2 has demonstrated the best performance in our evaluation, achieving $88.85\%$ accuracy through fine-tuning. Notably, considering the presence of challenges like shadows and occlusions in the images, human performance in this change detection task is approximately around $90\%$ during our labeling process. This observation suggests that fine-tuning DINOv2 as the backbone network has enabled the model to approach human-level performance. Furthermore, freezing the DINOv2 network and training only the linear layers surpass the fine-tuning outcomes of all other backbone networks, strongly affirming the capacity of DINOv2 to generate potent visual features suitable for change detection. 
% It is worth noting that, when using patch features as the output and employing non-linear prediction heads, DINO achieves the best result as indicated in the table. On the other hand, DINOv2 shows distinct behavior; its optimal performance is attained when using the CLS Token as the output and employing a single linear layer as the prediction head.

We find the performance of the change detection model fine-tuned on the pairwise dataset is significantly lower than its performance attained after fine-tuning on our time series dataset, as illustrated in table \ref{table:timeseries}. 
Moreover, augmenting the dataset using artificial techniques can lead to adverse effects. 
% This phenomenon could be attributed to the introduction of additional noise and interference into the building elements of the street view images. Typically, artificial data augmentation involves transformations applied to the entire image, resulting in alterations in the appearance of buildings within the images, which subsequently disrupts the ability of the model to detect architectural changes.
As a form of natural data augmentation, time series data equips the model with sufficient information to identify and filter out irrelevant variations that occur over time, such as changes in lighting, vegetation, and vehicles as is shown in Figure \ref{fig:res_noisy}, which guides the model to focus on more temporally stable elements such as building structures. These results validate the critical role of our proposed time series data in the context of street view image change detection task.

% By progressively increasing the number of sampled image pairs from the sequences, we observe a gradual improvement in model performance. This underscores the essential significance of time series data. 
The performance of the street-view pre-trained models is presented in Table \ref{table:pretrain}. The results of StreetMAE are significantly lower compared to those of StreetBYOL. This may be because patch reconstruction process is more prone to learning color and texture information, which aligns with the noise we aim to eliminate in change detection, rather than building structure. The addition of the semantic segmentation module leads to a performance enhancement in StreetBYOL. Nevertheless, domain-specific pre-trained models, whether StreetMAE or StreetBYOL, do not surpass the performance of the generic visual model DINOv2. This can be attributed to the smaller training data size used for domain-specific pre-training compared to generic visual models. Specifically, the inherently noisy street view images, with their complex and cluttered scenes, make it difficult for models to grasp fundamental concepts like shape, location, and architecture from limited data.
% This could be attributed, on one hand, to the considerably smaller training data volume used for domain-specific pre-training compared to the vast data utilized by generic visual models. On the other hand, this is also caused by the inherently noisy street view images as training data, often comprising complex and cluttered scenes that make it challenging for models to learn fundamental concepts about shape, location, environment, and architecture from a limited set of data. 
% Given the relatively high annotation cost of street view images and the substantial amount of unannotated data that persists, the task of curating the data and designing larger-scale pre-training methods to enhance the accuracy of change detection remains a noteworthy avenue for exploration.

\section{Case study: Assessing urban change in Seattle}
% \textbf{Scope and promise for social impact: Likelihood of social impact is extremely high: the paper’s ideas are already being used in practice or could be immediately.}
To evaluate the generalizability of our proposed change detection model on a large scale, we prepare a large-scale street view time series data for the city of Seattle, Washington. Figure \ref{fig:seattle} demonstrates the sampling process. We then apply our change detection model to identify urban change points: In total, we detect $11,838$ change points from $795,919$ sampled images in Seattle. 
% Our data shows significant correlation with change in median household income and population size in each census tract.
% \begin{itemize}
%     \item visualization
%     sampled results (corner cases)
%     \item correlation benchmark with permits data
%     \item ACS prediction
% \end{itemize}
\subsubsection{Construction permits data.}
% We obtain permits data fro Seattle 
% As mentioned earlier, construction permits data are often adopted as a fine-grained proxy for urban change in literatures. To compare with our change detection model, We fetch construction permit data from the online permit center of the Seattle city government\footnote{Available at \url{https://data.seattle.gov/Permitting/Building-Permits}}. 
As previously noted, previous works frequently rely on construction permit data as a detailed proxy for urban evolution. To further compare them with the results from our proposed change detection model, we obtain construction permit data from the Seattle city government's online permit center.
Each entry in the permit data provides details such as the date of issuance, permit category, geospatial coordinates, estimated cost, and other requisite information as mandated by the government.
As a data pre-processing step, we keep the ``new'', ``alteration'' and ``addition'' categories to align with the definition of urban change points. Additionally, we curated a subset of permits that had a total estimated cost exceeding $\$100,000$ within a single year. This approach allows us to compare our findings with both the complete permit dataset and the high-value permits, the latter of which are more probable to signify visible physical alterations.

% This way, urban change recorded by our filtered permits is significant enough to be viewed as potential signals of physical change.
\begin{figure*}[h!]
    \centering
    \includegraphics[width=0.93\linewidth]{figure/acs.png}
    \caption{Linear correlation with socio-demographic indicators. \textbf{Top:} Median household income. \textbf{Bottom:} Population size. Each dot represents a Seattle census tract. The change detection results show statistically significant correlations with socio-demographic metrics, in contrast to construction permit data which lacks such correlation.}
    \label{fig:acs}
\end{figure*}

% \begin{figure*}[h!]
%     \centering
%     \includegraphics[width=0.9\linewidth]{figure/acs.png}
%     \caption{Correlation of change detection results with socio-demographic indicators. \textbf{Top:} Median household income. \textbf{Bottom:} Population size. Each dot corresponds to a Seattle census tract. Notably, these results exhibit significant correlations with socio-demographic metrics, unlike the construction permit data.}
%     \label{fig:acs}
% \end{figure*}


% \begin{figure*}[h!]
%     \centering
%     \includegraphics[width=0.96\linewidth]{./figure/acs.png}
%     \caption{Linear correlation with social-demographic data. \textbf{Top:} Median household income. \textbf{Bottom: } Population size. Each dot represents a census tract in Seattle. Change detection results demonstrates statiscally signifcant correlations with social-demographic data, while construction permit data fail to do so.
%     }
%     % between ACS attributes and total number of predicted change points in each census tract vs. correlation between ACS attributes and total number of construction permits in each census tract}
%     \label{fig:acs}
% \end{figure*}
\subsubsection{Correlation with social-demographic data.} 
We prepare social-demographic data at the census tract level from the American Community Survey (ACS) 5-year estimates. Specifically, we select population size and median household incomes as our target variables, and calculate their relative percentage change from 2009 to 2021 for each census tract in Seattle. 
To quantify the linear correlation between proxies for urban change and shifts in socio-demographics, we compare three distinct proxies: the entire set of permits, high-value permits (those exceeding $\$100$k), and the percentage of locations with urban change points identified using our proposed methodology.
As is shown in Figure \ref{fig:acs}, both the entire set of permits and the high-value permits fail to show a linear correlation with the change of median household income and population size in each census tract with $p$ value larger than $0.05$ and $R^2$ close to $0$. 
While the change points results from our proposed method reach an $R^2$ of $0.19$ and $0.15$ for median household income change and population size change respectively, and achieve a $p$ value much less than $0.05$ supporting the statistical significance.
These results not only indicate that the detected urban change points provide a more accurate assessment of real-world urban transformations and socio-economic shifts, but also validates that our proposed change detection model can effectively complement existing proxies as a credible indicator of urban change.

\section{Conclusion and Future Work}
In this work, we propose a framework to assess fine-grained urban change at scale with street view time series. We have curated the largest street-level scene change detection dataset by far, and proposed an end-to-end change detection pipeline to identify urban change points at scale. We validate the proposed model by correlating with social-demographic data and prove its potential as a high-definition, up-to-date, and on-the-ground visual proxy of urban change.

While our data-driven approach provides a novel method to assess urban change, it is still subjective to a few limitations: 1) Street view data focus on changes observable at the street level, excluding alterations that might be non-visible, such as interior renovations. 2) The spatial-temporal distribution of Google Street View data is not consistent. Since its debut in 2007, Google has frequently updated its imagery in countries such as the US, but has been less consistent in updating images in many other regions, especially in developing countries. 
Despite these limitations, we believe our proposed method offer a comprehensive and extensive resource for urban change detection task, helping expand its social impact and advance sustainable development goals. 
% More broadly, identifying fine-grained urban change will benefit urban planing and advancing sustainable development goals. 
In future works, we can explore multi-task models to identify changes in a wider array of objects, 
% Additionally, our method can be integrated with satellite change detection systems, 
enhancing the applicability to a broader range of downstream tasks in cities.

\section{Acknowledgments}
This project was supported by the Google Cloud Grant from the Stanford Institute for Human-Centered Artificial Intelligence. The author would like to thank Zhecheng Wang, Sarthak Kanodia and Timothy Dai for their extensive guidance.
\bibliography{aaai24}

\end{document}
