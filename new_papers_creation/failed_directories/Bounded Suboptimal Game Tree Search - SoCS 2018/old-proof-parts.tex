
\paragraph{Case \#3.1: there is a child $c$ such that $\alpha(c)=\frac{\alpha(n)-\opti(n)}{\pi(c)}+\opti(c)$ and $\beta(c)=\opti(c)$.} 
The induction hypothesis gives
\begin{align}
\opti(c) \leq & \frac{\alpha(n)-\opti(n)}{\pi(c)}+\opti(c)+\epsilon \\
\opti(n) \leq & \alpha(n) + \pi(c)\epsilon
\end{align}
Since $\beta(n)\!\leq\!\!\opti(n)$ and $0\!\leq\!\pi(c)\!\leq\!1$, we get $\beta(n)\!\leq\!\alpha(n) + \epsilon$.

\paragraph{Case \#3.2: there is a child $c$ such that $\alpha(c)=\pess(c)$ 
	and $\beta(c)=\frac{\beta(n)-\pess(n)}{\pi(c)}+\pess(c)$.} 
The induction hypothesis gives
\begin{align}
\frac{\beta(n)-\pess(n)}{\pi(c)}+\pess(c) \leq & \pess(c)+\epsilon \\
\beta(n) \leq & \pess(n) + \pi(c) \epsilon
\end{align}
Since $\pess(n) \leq \alpha(n)$ and $\pi(c) \leq 1$, we get $\beta(n) \leq \alpha(n) + \epsilon$.

\paragraph{Case \#3.3: there is a child $c$ s.t. $\alpha(c)=\frac{\alpha(n)-\opti(n)}{\pi(c)}+\opti(c)$ and $\beta(c)\!\!=\!\frac{\beta(n)-\pess(n)}{\pi(c)} +\!\pess(c)$.\!\!\!\!}
The induction hypothesis gives
\begin{align}
\frac{\beta(n)-\pess(n)}{\pi(c)}+\pess(c) \leq \frac{\alpha(n)-\opti(n)}{\pi(c)}+\opti(c) + \epsilon\label{eq:3.3-a}
\end{align}
Since $c\in C_A(n)$, then we have
\begin{align}
\pi(c)(\opti(c)-\pess(c)) \leq & \!\!\sum_{c'\in C_A(n)\!\!\!\!\!\!\!\!\!} \!\pi(c')(\opti(c')-\pess(c'))\\
\pi(c)(\opti(c)-\pess(c)) \leq & \opti(n)-\pess(n)\\
\pess(n) - \pi(c)\pess(c) \leq & \opti(n)-\pi(c)\opti(c) \label{eq:3.3-b}
\end{align}
Adding $\pi(c) \cdot$ Eq.~\eqref{eq:3.3-a} to Eq.~\eqref{eq:3.3-b} gives $\beta(n) \leq \alpha(n)+\epsilon$.

\paragraph{Case \#3.4: for every child $c$, we have} $\alpha(c)=\pess(c)$ and $\beta(c)=\opti(c)$. 
%By definition of $n$, 
Since $\beta(c)\leq \alpha(c)+\epsilon$, we have that
\begin{align}
\sum_{c\in C_A(n)} \!\!\pi(c) \beta(c) \leq & \sum_{c\in C_A(n)} \pi(c)(\alpha(c) + \epsilon)\\
\sum_{c\in C_A(n)} \!\!\!\pi(c)\opti(c) \leq & \!\!\sum_{c\in C_A(n)} \!\!\!\!\pi(c)\pess(c) + \!\!\!\sum_{c\in C_A(n)} \!\!\!\!\pi(c)\epsilon \label{eq:opti-pess-eps}
\end{align}
Since $n$ is fully expanded, $\sum_{c\in C_A(n)} \pi(c) = 1$ and C3 gives
\begin{align}
\opti(n) \leq & \pess(n) + \epsilon \label{eq:opti-pess-eps}
\end{align}
from Lemma~\ref{lem:alpha-beta-between-l-u} conclude that $\beta(n) \leq \alpha(n) + \epsilon$.
\end{proof}





\begin{enumerate}
\item $\pess(n) \leq \MM(n) \leq \opti(n)$ for every leaf node $n$; \label{item:adm-leaves}
\item $\opti(n)-\pess(n)\leq \epsilon$ for every leaf node $n$; \label{item:leaf-gap}
\item $\beta(n) \leq \alpha(n)+\epsilon$ for every partially expanded node $n$; \label{item:partial-gap}
\end{enumerate}