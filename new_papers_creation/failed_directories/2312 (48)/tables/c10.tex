\begin{table}[t]
\centering
\small
\resizebox{0.99\linewidth}{!}{
\begin{tabular}{lcccc}
\multirow{2}{1.5cm}{\textbf{Methods}} & \multicolumn{2}{c}{\textbf{Far-OOD}} & \multicolumn{2}{c}{\textbf{Near-OOD}}\\
\cmidrule{2-5}
& \textbf{FPR95}  & \textbf{AUROC} & \textbf{FPR95}  & \textbf{AUROC}\\
& $\downarrow$ & $\uparrow$ & $\downarrow$ & $\uparrow$ \\
\toprule
\emph{Using model outputs}\\
MSP~\cite{hendrycks2016baseline} & 52.11 & 91.79 & 64.66 & 85.28 \\
ODIN~\cite{liang2018enhancing}  & 26.47 & 94.48 & 52.32 & 88.90\\
GODIN~\cite{hsu2020generalized}  & 17.42  & 95.84 & 60.69 & 82.37 \\
Energy score~\cite{liu2020energy}  & 28.40 & 94.22 & 50.64 & 88.66 \\
ReAct~\cite{sun2021react} & 33.12 & 94.32 & 53.51 & 88.96\\
GradNorm~\cite{huang2021importance} & 24.79 & 92.58 & 65.44 & 79.31\\
LogitNorm~\cite{wei2022mitigating}  & 19.61 & 95.51 & 55.08 & 88.03\\
DICE~\cite{sun2022dice}  & 20.83 & 95.24 & 58.60 & 87.11 \\
\midrule
\emph{Using feature representations}\\
Mahalanobis~\cite{lee2018simple} & 44.55 & 82.56 & 87.71 & 78.93 \\
KNN~\cite{sun2022knn}  & 18.50 & 96.36 & 58.34 & 87.90 \\
\midrule 
 \name (ours) & \textbf{14.99} & \textbf{97.15}  & \textbf{50.10} &  \textbf{89.80}\\
 & $\pm{0.87}$ & $\pm{0.27}$ & $\pm{1.09}$ & $\pm{0.65}$\\
\bottomrule
\end{tabular}}
\caption{\small Performance comparison on near-OOD and far-OOD detection task. Architecture used is DenseNet-101 and ID data is CIFAR-10. We report the mean and variance across 3 training runs.}
\label{tab:hard_ood}
\end{table}