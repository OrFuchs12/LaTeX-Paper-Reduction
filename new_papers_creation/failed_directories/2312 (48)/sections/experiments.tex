
\section{Experiments}
\label{sec:experiment}

\begin{table}[t]
\centering
\small
\resizebox{0.99\linewidth}{!}{
\begin{tabular}{lcccc}
\multirow{2}{1.5cm}{\textbf{Methods}} & \multicolumn{2}{c}{\textbf{Far-OOD}} & \multicolumn{2}{c}{\textbf{Near-OOD}}\\
\cmidrule{2-5}
& \textbf{FPR95}  & \textbf{AUROC} & \textbf{FPR95}  & \textbf{AUROC}\\
& $\downarrow$ & $\uparrow$ & $\downarrow$ & $\uparrow$ \\
\toprule
\emph{Using model outputs}\\
MSP~\cite{hendrycks2016baseline} & 52.11 & 91.79 & 64.66 & 85.28 \\
ODIN~\cite{liang2018enhancing}  & 26.47 & 94.48 & 52.32 & 88.90\\
GODIN~\cite{hsu2020generalized}  & 17.42  & 95.84 & 60.69 & 82.37 \\
Energy score~\cite{liu2020energy}  & 28.40 & 94.22 & 50.64 & 88.66 \\
ReAct~\cite{sun2021react} & 33.12 & 94.32 & 53.51 & 88.96\\
GradNorm~\cite{huang2021importance} & 24.79 & 92.58 & 65.44 & 79.31\\
LogitNorm~\cite{wei2022mitigating}  & 19.61 & 95.51 & 55.08 & 88.03\\
DICE~\cite{sun2022dice}  & 20.83 & 95.24 & 58.60 & 87.11 \\
\midrule
\emph{Using feature representations}\\
Mahalanobis~\cite{lee2018simple} & 44.55 & 82.56 & 87.71 & 78.93 \\
KNN~\cite{sun2022knn}  & 18.50 & 96.36 & 58.34 & 87.90 \\
\midrule 
 \name (ours) & \textbf{14.99} & \textbf{97.15}  & \textbf{50.10} &  \textbf{89.80}\\
 & $\pm{0.87}$ & $\pm{0.27}$ & $\pm{1.09}$ & $\pm{0.65}$\\
\bottomrule
\end{tabular}}
\caption{\small Performance comparison on near-OOD and far-OOD detection task. Architecture used is DenseNet-101 and ID data is CIFAR-10. We report the mean and variance across 3 training runs.}
\label{tab:hard_ood}
\end{table}
\begin{table}[t]
\small
\centering
\resizebox{0.99\linewidth}{!}{
\begin{tabular}{lccc}
\textbf{Method} & \textbf{FPR95}  & \textbf{AUROC} & \textbf{ID Acc.}\\
& $\downarrow$ & $\uparrow$ & $\uparrow$ \\
\toprule
\emph{Methods using model outputs}\\
MSP~\cite{hendrycks2016baseline} & 77.59 & 76.47 &  75.14\\
ODIN~\cite{liang2018enhancing} & 56.39 & 86.02 & 75.14\\
GODIN~\cite{hsu2020generalized} & 44.08 &  89.05 & 74.37\\
Energy score~\cite{liu2020energy} & 57.07 &  84.83 &  75.14\\
ReAct~\cite{sun2021react} & 75.06 & 79.51 & 66.56\\
GradNorm~\cite{huang2021importance} & 63.05 & 79.80 & 75.14\\
LogitNorm~\cite{wei2022mitigating} & 61.10 & 84.72 & 75.42\\
DICE~\cite{sun2022dice} & 49.72 & 87.23 & 68.65 \\
\midrule 
\emph{Methods using feature representations}\\
Mahalanobis~\cite{lee2018simple} & 56.93 & 80.27 &  75.14\\
KNN~\cite{sun2022knn} & 47.21 & 85.27 & 75.14\\
\midrule 
 \name (ours) & \textbf{31.25} & \textbf{90.76} & \textbf{75.59}\\
& $\pm{1.25}$ & $\pm{0.36}$ & $\pm{0.08}$\\
\bottomrule
\end{tabular}}
\caption{\small Performance comparison on CIFAR-100 dataset. We use DenseNet-101 for all baselines. Best  results are in \textbf{bold}. We report the mean and variance across 3 different training runs.}
\label{tab:cifar-100}
\end{table}
In this section, we extensively evaluate the effectiveness of our proposed method. 
The goal of our experimental sections is to mainly answer the following questions: (1) Can \name alleviate the curse of dimensionality? (2) How does \name compare against the state-of-the-art OOD detection methods?  Due to space constraints, extensive experimental details are in Appendix C. Our code is open-sourced for the research community.


\subsection{Evaluation on Common Benchmarks}
\label{subsec:common_benchmark}

\noindent \textbf{Datasets.} In this section, we make use of commonly studied CIFAR-10 (10 classes) and CIFAR-100 (100 classes)~\cite{krizhevsky2009learning} datasets as ID. Both datasets consist of images of size $32 \times 32$. We use the standard split with $50,000$ images for training and $10,000$ images for testing. We evaluate the methods on common OOD datasets: \texttt{Textures}~\cite{cimpoi2014describing}, \texttt{SVHN}~\cite{svhn}, \texttt{LSUN-Crop}~\cite{yu2015lsun}, \texttt{LSUN-Resize}~\cite{yu2015lsun}, \texttt{iSUN}~\cite{xu2015turkergaze}, and \texttt{Places365}~\cite{zhou2017places}. Images in all these test datasets are of size $32 \times 32$. 


\paragraph{Evaluation metrics.} We compare the performance of various methods using the following metrics: 
(1) {FPR95} measures the false positive rate (FPR) of OOD samples when $95\%$ of ID samples are correctly classified;
(2) {AUROC} is the area under the Receiver Operating Characteristic curve; 
and (3) {ID Acc.} measures the ID classification accuracy.

\vspace{0.2cm}
\noindent \textbf{Comparison with competitive methods.} In Table~\ref{tab:cifar-100}, we provide a comprehensive comparison with competitive OOD detection baselines on  CIFAR-100. {We provide a detailed description of baseline approaches in Appendix C.3.} We observe that our proposed method \name significantly outperforms the latest rivals. For a fair comparison, we divide the baselines into two categories: methods using model outputs and methods using feature representations.
From Table~\ref{tab:cifar-100}, we highlight two salient observations: (1) Considering methods based on feature representations, \name outperforms KNN (non-parametric) and Mahalanobis (parametric) by \textbf{15.96\%} and \textbf{25.68\%} respectively in terms on FPR95. The results validate that learning feature subspace effectively alleviates the ``curse-of-dimensionality" problem that is troubling the existing KNN approach. (2) Further, \name also performs better than output-based methods such as ReAct~\cite{sun2021react}. Specifically, with CIFAR-100 as ID, \name provides a $\mathbf{43.81}\%$ improvement in FPR95 as compared to ReAct~\cite{sun2021react}. Notably, \name provides a \textbf{18.47\%} improvement compared to~\cite{sun2022dice}, a post-hoc sparsification method. While DICE can severely affect the ID test accuracy (68.65\%), \name exhibits stronger classification performance (75.59\%) by baking in the inductive bias of subspaces through training. An extensive discussion is provided in Section~\ref{sec:discussion}. 



\paragraph{Evaluation on near-OOD data.} In Table~\ref{tab:hard_ood}, we compare the performance in detecting near-OOD data, which refers to samples near the ID data. Near-OOD is particularly challenging to detect, and can often be misclassified as ID. We report the performance on CIFAR-10 (ID) vs. CIFAR-100 (OOD), which is the most commonly used dataset pair for this task. We observe that \name consistently outperforms existing algorithms for near-OOD detection tasks, further demonstrating its strengths. Compared to KNN, \name reduces the FPR95 by 8.24\%. For completeness, we also provide far-OOD evaluation results on CIFAR-10, where \name achieves an average FPR95 of 14.99\%. Full result on each test dataset for CIFAR-10 is available in Appendix D.4.



\begin{table}[t]
\small
\centering
\resizebox{0.95\linewidth}{!}{
\begin{tabular}{lccc}
\textbf{Method} & \textbf{Dataset (ID)} & \textbf{FPR95}  & \textbf{AUROC} \\
& & $\downarrow$ & $\uparrow$  \\
\toprule
Mahalanobis~\cite{lee2018simple} & CIFAR-10 & 44.55 & 82.56  \\

\name (w. Mahalanobis) & CIFAR-10 &  \textbf{34.68} &  \textbf{87.87} \\
\midrule
Mahalanobis~\cite{lee2018simple} & CIFAR-100 & 56.93 & 80.27  \\

\name (w. Mahalanobis) & CIFAR-100 &  \textbf{55.05} &  \textbf{80.77} \\
\bottomrule
\end{tabular}}
\caption{\small \name is also compatible with parametric approaches such as Mahalanobis distance~\cite{lee2018simple}. The model is DenseNet. All values are averaged over six OOD test datasets.}
\label{tab:compatibility}
\end{table}


\paragraph{Compatibility with other distance-based approaches.} 

Beyond KNN~\cite{sun2022knn}, the Mahalanobis distances~\cite{lee2018simple} is also one of the most popular distance-based approaches to detect OOD. 
However, all prior solutions measure the distance with a full feature space which can also suffer from the curse of dimensionality. 
In this section, we show that subspace learning can also benefit parametric approaches like Mahalanobis distance~\cite{lee2018simple}. In Table~\ref{tab:compatibility}, we compare the OOD detection performance of using Mahalanobis distance on the vanilla model and the model trained with \name. 
We see that coupling subspace learning (in training) with Mahalanobis distance (in testing) reduces FPR95 by {9.87\%} and {1.88\%} on CIFAR-10 and CIFAR-100 datasets respectively.

\begin{table}[t]
\small
\centering
\resizebox{0.55\linewidth}{!}{
\begin{tabular}{lcc}
\toprule
\multirow{2}{2cm}{\textbf{Training Method}} &  CIFAR-10 & CIFAR-100 \\ 
& \multicolumn{2}{c}{(Train time in hours)} \\
\midrule
Standard & $2.10$ & $2.25$\\ 
 \name & $1.75$ & $1.89$ \\
\bottomrule
\end{tabular}}
\caption{\small \textbf{Computational cost for training}. trained using ResNet-101. 
Model used is DenseNet-101. For the comparison, we used the software configuration as reported in Appendix C.2.}
\label{tab:train_time}
\end{table}
% 
\paragraph{Computational complexity.}  In Table~\ref{tab:train_time}, we compare the training time of \name with the standard training method using cross-entropy loss. We observe that training using \name incurs no additional computation overhead but rather is slightly more efficient compared to standard training procedures. This is because we perform gradient descent only on a subset of weights corresponding to the selected feature subspace. Thus, our method overall leads to faster updates and convergence. {In Appendix D.1, we further show that
\name remains competitive and outperforms the KNN counterpart on other common architecture.}