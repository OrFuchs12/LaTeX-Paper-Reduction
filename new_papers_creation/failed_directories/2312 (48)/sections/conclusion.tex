\section{Conclusion}
\label{sec:conclusion}

Our work highlights the challenge of curse-of-dimensionality in OOD detection, and introduces a new solution called \name for detecting OOD samples. Traditional distance-based methods for OOD detection suffer from the curse-of-dimensionality, which makes it difficult to distinguish between ID and OOD samples in high-dimensional feature spaces. To address this issue, \name learns subspaces that capture the most informative feature dimensions for the task. Our method is supported by theoretical analysis, which shows that reducing the feature dimensions improves the distinguishability between ID and OOD samples. Extensive experiments demonstrate that \name achieves significant improvements in both OOD detection and ID calibration performance. We hope that our approach will inspire future research on this important problem.

