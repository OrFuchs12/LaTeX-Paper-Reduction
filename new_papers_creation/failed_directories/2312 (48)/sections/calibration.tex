\subsection{Model Calibration}
\label{subsec:calibration}

\newcolumntype{?}{!{\vrule width 1pt}}
\begin{table*}[h!]
% \scriptsize
\centering
\resizebox{0.9\textwidth}{!}{%
\begin{tabular}{lcccc?cccc?cccc}
\multirow{2}{*}{\textbf{Dataset}} & \multicolumn{2}{c}{\textbf{NLL}} & \multicolumn{2}{c?}{\textbf{NLL (w. Subspace)}} & 
\multicolumn{2}{c}{\textbf{LS}} & \multicolumn{2}{c?}{\textbf{LS (w. Subspace)}} & \multicolumn{2}{c}{\textbf{FL}} & \multicolumn{2}{c}{\textbf{FL (w. Subspace)}} \\

% \multicolumn{2}{c}{\textbf{LS}} & \multicolumn{2}{c?}{\textbf{LS (w. Subspace)}}&
\cmidrule{2-13}
& SCE & ECE & SCE & ECE & SCE & ECE & SCE & ECE & SCE & ECE & SCE & ECE \\
\midrule
\multirow{1}{*}{CIFAR-10} & 11.5 & 5.6 & 9.6 &  4.4 & 8.8 & 3.9 & 8.2 & 3.8 & 6.9 & 2.3 & \textbf{3.9} & \textbf{0.8} \\
\midrule
\multirow{1}{*}{CIFAR-100} & 3.7 & 15.1 & 2.5 & 5.8  & 2.2 & 7.8 & 2.1 & \textbf{2.7} & 2.3 & 6.8 & \textbf{2.1} & 4.3 \\
% & & $\pmb{\downarrow}$ & $\pmb{\downarrow}$ & $\pmb{\downarrow}$ & $\pmb{\downarrow}$ & $\pmb{\downarrow}$ & $\pmb{\downarrow}$ & $\pmb{\downarrow}$ & $\pmb{\downarrow}$ \\  
% \multirow{1}{*}{CIFAR-10} & 11.5 & 5.6 & \cellcolor{COLOR_ZS} 9.6 & \cellcolor{COLOR_ZS} 4.4 & 8.8 & 3.9 &  \cellcolor{COLOR_ZS} 6.5 & \cellcolor{COLOR_ZS} 1.7 & 6.9 & 2.3 & \cellcolor{COLOR_ZS}\textbf{3.9} & \cellcolor{COLOR_ZS}\textbf{0.8} \\
% & ResNet-34 & 10.2 & 4.9 & \cellcolor{COLOR_ZS}9.6 & \cellcolor{COLOR_ZS}4.5 & 7.8 & 4.0 & \cellcolor{COLOR_ZS}6.3 & \cellcolor{COLOR_ZS}2.7 & 5.4 & 2.1 & \cellcolor{COLOR_ZS}\textbf{3.6} & \cellcolor{COLOR_ZS}\textbf{0.5} \\
% \midrule
% \multirow{1}{*}{CIFAR-100} & 3.7 & 15.1 & \cellcolor{COLOR_ZS}2.5 & \cellcolor{COLOR_ZS}5.8 & 2.2 & 7.8 & \cellcolor{COLOR_ZS}\textbf{2.1} & \cellcolor{COLOR_ZS}\textbf{1.9} & 2.3 & 6.8 & \cellcolor{COLOR_ZS}2.1 & \cellcolor{COLOR_ZS}4.3 \\
 % & ResNet-34 & 3.5 & 16.0 & \cellcolor{COLOR_ZS}3.5 & \cellcolor{COLOR_ZS}15.0 & 2.4 & 4.5 &\cellcolor{COLOR_ZS} \textbf{2.1} &\cellcolor{COLOR_ZS} \textbf{2.4} & 2.2 & 5.7 & \cellcolor{COLOR_ZS}2.1 & \cellcolor{COLOR_ZS}4.7 \\
%  \midrule
%  Tiny-ImageNet & ResNet-34 & 2.2 & 15.7 & \cellcolor{COLOR_ZS}1.9 & \cellcolor{COLOR_ZS}13.5 & 1.4 & 5.96 &\cellcolor{COLOR_ZS} 1.5 & \cellcolor{COLOR_ZS}3.8 & 1.4 & 2.2 & \cellcolor{COLOR_ZS}\textbf{1.3} & \cellcolor{COLOR_ZS}\textbf{1.2} \\ 
 
 
 \bottomrule

\end{tabular}%
}
\caption{\small \textbf{Calibration Results.} Comparison of calibration performance when using subspace learning with commonly used loss functions (NLL/LS/FL). The model is ResNet-50. Best performing results are marked in \textbf{bold}.}
\label{tab:calibration_1}
\end{table*}



Beyond\ood detection tasks, in this section we show that our subspace training algorithm also leads to a significant improvement in model calibration. 

\paragraph{Datasets.} We train on three common benchmark datasets: CIFAR-10/100~\cite{krizhevsky2009learning} and Tiny-ImageNet\cite{deng2009imagenet} (200 classes). For each dataset, we have a separate train and test set. We further split the train set into two mutually exclusive sets: (1) $90\%$ of the train samples are used for training the model, and (2) the remaining $10\%$ of samples are used for validation and hyper-parameter adjustment for post-hoc calibration.

\paragraph{Evaluation Metrics.} To measure model calibration we report: (1) \textbf{Expected Calibration Error (ECE)}~\cite{naeini2015obtaining}: is the weighted average of differences in confidence of predicted class and accuracy of samples predicted with particular confidence. (2) \textbf{Static Calibration Error (SCE)}~\cite{nixon2019measuring}: is a simple class-wise extension of ECE. Refer Appendix for a detailed overview of the metrics used. 


\paragraph{Learning subspace improves calibration.} In Table~\ref{tab:calibration_1}, we observe that our subspace-based training algorithm improves calibration performance. In particular, we consider three losses that are commonly studied for calibration: (1) Cross Entropy Loss (NLL), (2) Label Smoothing (LS)~\cite{muller2019does}, and (3) Focal Loss (FL)~\cite{lin2017focal}. We train the model with each of these losses and measure calibration performance with and without subspace regularization. For clarity, we refer to the technique using our subspace-based learning as ``\textbf{ * + Subspace }'', where * refers to NLL/FL/LS. We see from Table~\ref{tab:calibration_1} that across multiple datasets, using subspace improves the calibration performance of all the loss functions. Specifically, we note that FL+Subspace gives the best performance on multiple datasets. In the Appendix\SL{???}, we further study the effect of combining different post-hoc calibration methods, namely Temperature Scaling and Dirichlet Calibration, over various calibration strategies studied in Table~\ref{tab:calibration_1}.
