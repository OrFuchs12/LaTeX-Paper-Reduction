%File: anonymous-submission-latex-2024.tex
\documentclass[letterpaper]{article} % DO NOT CHANGE THIS
\usepackage{aaai24}  % DO NOT CHANGE THIS
\usepackage{times}  % DO NOT CHANGE THIS
\usepackage{helvet}  % DO NOT CHANGE THIS
\usepackage{courier}  % DO NOT CHANGE THIS
\usepackage[hyphens]{url}  % DO NOT CHANGE THIS
\usepackage{graphicx} % DO NOT CHANGE THIS
\urlstyle{rm} % DO NOT CHANGE THIS
\def\UrlFont{\rm}  % DO NOT CHANGE THIS
\usepackage{natbib}  % DO NOT CHANGE THIS AND DO NOT ADD ANY OPTIONS TO IT
\usepackage{caption} % DO NOT CHANGE THIS AND DO NOT ADD ANY OPTIONS TO IT
\frenchspacing  % DO NOT CHANGE THIS
\setlength{\pdfpagewidth}{8.5in} % DO NOT CHANGE THIS
\setlength{\pdfpageheight}{11in} % DO NOT CHANGE THIS
\usepackage{algorithm}
\usepackage{algorithmic}
\usepackage{multirow}
\usepackage{enumitem, array}
\input{maths_commands}
\usepackage{newfloat}
\usepackage{listings}
\usepackage{tikz}
\usepackage{comment}
\usepackage{amsmath,amssymb} % define this before the line numbering.
% \usepackage{color}
\usepackage{cite}
\usepackage{subcaption}
\usepackage{tabularray}
\usepackage{booktabs}
% \usepackage{unicode-math}
\usepackage[utf8]{inputenc} % allow utf-8 input     % hyperlinks
\usepackage{url}            % simple URL typesetting
\usepackage{amsfonts}       % blackboard math symbols
\usepackage{nicefrac}       % compact symbols for 1/2, etc.
\usepackage{microtype}
\usepackage{xspace}
\def\X{{\mathcal{X}}}
\def\Y{{\mathcal{Y}}}
\def\D{{\mathcal{D}}}
\def\etal{{et al.\xspace}}
\def\dP{{\mathbb{P}}}
\def\cP{{\mathcal{P}}}
\def\cQ{{\mathcal{Q}}}
\def\cM{{\mathcal{M}}}

\def\real{{\mathbb{R}}}

\def\ood{\textsc{ood}\xspace}
\def\iid{\emph{i.i.d}\xspace}
\def\id{\textsc{id}\xspace}
\def\sota{\textsc{sota}\xspace}
\def\name{$\mathcal{S}$NN\xspace}

\def\xt{{\Tilde{x}}}
\def\yt{{\Tilde{y}}}
\def\hbe{{\mathbf{\emph{h}}}}
\def\HC{{\mathcal{H}}}

\def\X{{\mathcal{X}}}
\def\Y{{\mathcal{Y}}}
\def\D{{\mathcal{D}}}

\def\dP{{\mathbb{P}}}
\def\cP{{\mathcal{P}}}
\def\cQ{{\mathcal{Q}}}
\def\cM{{\mathcal{M}}}

\def\real{{\mathbb{R}}}
\def\*#1{\mathbf{#1}}
\newcommand{\theHalgorithm}{\arabic{algorithm}}

\usepackage{url}
\usepackage{mathtools}
\usepackage{amsthm}

\theoremstyle{plain}
\newtheorem{theorem}{Theorem}[section]
\newtheorem{proposition}[theorem]{Proposition}
\newtheorem{lemma}[theorem]{Lemma}
\newtheorem{corollary}[theorem]{Corollary}
\theoremstyle{definition}
\newtheorem{definition}[theorem]{Definition}
\newtheorem{assumption}[theorem]{Assumption}
\theoremstyle{remark}


%%%%%%%%%%%%%%%%%%%%%%% Annotation Libraries begin %%%%%%%%%%%%%%%%%%%%%%%
\usepackage{xspace}
\usepackage{array}
\usepackage{ragged2e}
\newcolumntype{P}[1]{>{\RaggedRight\hspace{0pt}}p{#1}}
\newcolumntype{X}[1]{>{\RaggedRight\hspace*{0pt}}p{#1}}
\usepackage{tcolorbox}
\usepackage{tikz}
\usetikzlibrary{arrows,shapes,positioning,shadows,trees,mindmap}
\usepackage[edges]{forest}
\usetikzlibrary{arrows.meta}
\colorlet{linecol}{black!75}
\usepackage{xkcdcolors}


% for colorful equation
\usetikzlibrary{backgrounds}
\usetikzlibrary{arrows,shapes}
\usetikzlibrary{tikzmark}
\usetikzlibrary{calc}
% Commands for Highlighting text -- non tikz method
\newcommand{\highlight}[2]{\colorbox{#1!17}{$\displaystyle #2$}}
\newcommand{\highlightdark}[2]{\colorbox{#1!47}{$\displaystyle #2$}}


% Commands for Highlighting text -- non tikz method
\renewcommand{\highlight}[2]{\colorbox{#1!17}{#2}}
\renewcommand{\highlightdark}[2]{\colorbox{#1!47}{#2}}
\usepackage[capitalize,noabbrev]{cleveref}
%
% These are are recommended to typeset listings but not required. See the subsubsection on listing. Remove this block if you don't have listings in your paper.
\usepackage{newfloat}
\usepackage{listings}
\DeclareCaptionStyle{ruled}{labelfont=normalfont,labelsep=colon,strut=off} % DO NOT CHANGE THIS
\lstset{%
	basicstyle={\footnotesize\ttfamily},% footnotesize acceptable for monospace
	numbers=left,numberstyle=\footnotesize,xleftmargin=2em,% show line numbers, remove this entire line if you don't want the numbers.
	aboveskip=0pt,belowskip=0pt,%
	showstringspaces=false,tabsize=2,breaklines=true}
\floatstyle{ruled}
\newfloat{listing}{tb}{lst}{}
\floatname{listing}{Listing}
%
% Keep the \pdfinfo as shown here. There's no need
% for you to add the /Title and /Author tags.

% DISALLOWED PACKAGES
% \usepackage{authblk} -- This package is specifically forbidden
% \usepackage{balance} -- This package is specifically forbidden
% \usepackage{color (if used in text)
% \usepackage{CJK} -- This package is specifically forbidden
% \usepackage{float} -- This package is specifically forbidden
% \usepackage{flushend} -- This package is specifically forbidden
% \usepackage{fontenc} -- This package is specifically forbidden
% \usepackage{fullpage} -- This package is specifically forbidden
% \usepackage{geometry} -- This package is specifically forbidden
% \usepackage{grffile} -- This package is specifically forbidden
% \usepackage{hyperref} -- This package is specifically forbidden
% \usepackage{navigator} -- This package is specifically forbidden
% (or any other package that embeds links such as navigator or hyperref)
% \indentfirst} -- This package is specifically forbidden
% \layout} -- This package is specifically forbidden
% \multicol} -- This package is specifically forbidden
% \nameref} -- This package is specifically forbidden
% \usepackage{savetrees} -- This package is specifically forbidden
% \usepackage{setspace} -- This package is specifically forbidden
% \usepackage{stfloats} -- This package is specifically forbidden
% \usepackage{tabu} -- This package is specifically forbidden
% \usepackage{titlesec} -- This package is specifically forbidden
% \usepackage{tocbibind} -- This package is specifically forbidden
% \usepackage{ulem} -- This package is specifically forbidden
% \usepackage{wrapfig} -- This package is specifically forbidden
% DISALLOWED COMMANDS
% \nocopyright -- Your paper will not be published if you use this command
% \addtolength -- This command may not be used
% \balance -- This command may not be used
% \baselinestretch -- Your paper will not be published if you use this command
% \clearpage -- No page breaks of any kind may be used for the final version of your paper
% \columnsep -- This command may not be used
% % \newpage -- No page breaks of any kind may be used for the final version of your paper
% \pagebreak -- No page breaks of any kind may be used for the final version of your paperr
% \pagestyle -- This command may not be used
% \tiny -- This is not an acceptable font size.
% \vspace{- -- No negative value may be used in proximity of a caption, figure, table, section, subsection, subsubsection, or reference
% \vskip{- -- No negative value may be used to alter spacing above or below a caption, figure, table, section, subsection, subsubsection, or reference

\setcounter{secnumdepth}{2} %May be changed to 1 or 2 if section numbers are desired.

% The file aaai24.sty is the style file for AAAI Press
% proceedings, working notes, and technical reports.
%

% Title

% Your title must be in mixed case, not sentence case.
% That means all verbs (including short verbs like be, is, using,and go),
% nouns, adverbs, adjectives should be capitalized, including both words in hyphenated terms, while
% articles, conjunctions, and prepositions are lower case unless they
% directly follow a colon or long dash
\title{How to Overcome Curse-of-Dimensionality for Out-of-Distribution Detection?}

% \title{How to Overcome Curse-of-Dimensionality for OOD Detection?}
\author{
    %Authors
    % All authors must be in the same font size and format.
    Soumya Suvra Ghosal\footnote{Equal contributions},
    Yiyou Sun*,
    Yixuan Li
}
\affiliations{
    %Afiliations
    Department of Computer Sciences, University of Wisconsin -- Madison\\
    \{sghosal, sunyiyou, sharonli\}@cs.wisc.edu

}
% \author{
%     %Authors
%     % All authors must be in the same font size and format.
%     Written by AAAI Press Staff\textsuperscript{\rm 1}\thanks{With help from the AAAI Publications Committee.}\\
%     AAAI Style Contributions by Pater Patel Schneider,
%     Sunil Issar,\\
%     J. Scott Penberthy,
%     George Ferguson,
%     Hans Guesgen,
%     Francisco Cruz\equalcontrib,
%     Marc Pujol-Gonzalez\equalcontrib
% }
% \affiliations{
%     %Afiliations
%     \textsuperscript{\rm 1}Association for the Advancement of Artificial Intelligence\\
%     % If you have multiple authors and multiple affiliations
%     % use superscripts in text and roman font to identify them.
%     % For example,

%     % Sunil Issar\textsuperscript{\rm 2},
%     % J. Scott Penberthy\textsuperscript{\rm 3},
%     % George Ferguson\textsuperscript{\rm 4},
%     % Hans Guesgen\textsuperscript{\rm 5}
%     % Note that the comma should be placed after the superscript

%     1900 Embarcadero Road, Suite 101\\
%     Palo Alto, California 94303-3310 USA\\
%     % email address must be in roman text type, not monospace or sans serif
%     proceedings-questions@aaai.org
% %
% % See more examples next
% }

%Example, Single Author, ->> remove \iffalse,\fi and place them surrounding AAAI title to use it
\iffalse
\title{My Publication Title --- Single Author}
\author {
    Author Name
}
\affiliations{
    Affiliation\\
    Affiliation Line 2\\
    name@example.com
}
\fi

\iffalse
%Example, Multiple Authors, ->> remove \iffalse,\fi and place them surrounding AAAI title to use it
\title{My Publication Title --- Multiple Authors}
\author {
    % Authors
    First Author Name\textsuperscript{\rm 1},
    Second Author Name\textsuperscript{\rm 2},
    Third Author Name\textsuperscript{\rm 1}
}
\affiliations {
    % Affiliations
    \textsuperscript{\rm 1}Affiliation 1\\
    \textsuperscript{\rm 2}Affiliation 2\\
    firstAuthor@affiliation1.com, secondAuthor@affilation2.com, thirdAuthor@affiliation1.com
}
\fi


% REMOVE THIS: bibentry
% This is only needed to show inline citations in the guidelines document. You should not need it and can safely delete it.
\usepackage{bibentry}
% END REMOVE bibentry

\begin{document}

\maketitle

\begin{abstract}
Machine learning models deployed in the wild can be challenged by out-of-distribution (OOD) data from unknown classes. Recent advances in OOD detection rely on distance measures to distinguish samples that are relatively far away from the in-distribution (ID) data. Despite the promise, distance-based methods can suffer from the curse-of-dimensionality problem, which limits the efficacy in high-dimensional feature space.
To combat this problem, we propose a novel framework, Subspace Nearest Neighbor (\name), for OOD detection. In training, our method regularizes the model and its feature representation by leveraging the most relevant subset of dimensions (i.e. subspace).  Subspace learning yields highly distinguishable distance measures between ID and OOD data.
We provide comprehensive experiments and ablations to validate the efficacy of \name.
Compared to the current best distance-based method, \name reduces the average FPR95 by $15.96\%$ on the CIFAR-100 benchmark.
\end{abstract}
\section{Introduction}
Justice et al. \cite{justiceguide} state in their book that ``Children develop their knowledge of the world around them as they interact with their environment directly and indirectly. The direct experiences children have in their homes, schools and communities certainly provide the greatest amount of input to the world knowledge base.''. This knowledge arises from both physical and conversational interactions. In this paper, we test the hypothesis that just like a human child, machines need interaction to acquire world knowledge and develop commonsense reasoning abilities, and we study the effect of conversational interactions on this knowledge acquisition. Most of the literature on commonsense reasoning 
relies %rely [kmm- most-> relies]
on extracting the largest possible snapshot of 
%the [kmm- removed]
world knowledge and either 
query %query [kmm- on-> extracting and querying]
it or 
propose %propose [kmm- most-> proposes][could also parse as 'relies on-> proposing' or 'querying or proposing', may be better to restructure the sentence][fa- it was the later, so i restructured]
automated knowledge base completion methods for it. We argue that it is necessary to equip reasoning engines with an interaction strategy facilitating the extraction of just-in-time information needed for reasoning. 
%, through conversation with a human user [kmm- removed; conversation is covered by 'interaction' earlier in the sentence]
In this paper, we 
take up %take a few steps towards [kmm- rephrase (take steps/take steps repetitive)]
this grand goal, %[kmm- comma added]
and although we do not solve the whole challenge, we take the first steps needed for addressing it. 
Specifically, here we propose a ``soft'' commonsense reasoning engine and solve targeted knowledge base completion problems based on the information provided by the user through a conversational interface.

% We state this as our overarching grand research goal and mention carefully that we are taking a few steps towards this grand goal. Although it does not solve all of it but it is a step towards achieving this goal. This is just a first step however its a part of a very well reasoned and ambitious project. Then we also carefully describe the limitations of the project
% In other words, our overarching goal is having a human construct a reasoning system that does not have commonsense and extract commonsense from the user through conversation.
% \amoscomment{I think that it might be better saying something like: this work takes the first step towards ... I think that the paper could also benefit from adding a few sentences at the beginning.} \facomment{Is this resolved now?}

We believe that this is the right time for this proposal specifically since conversational agents such as Siri, Google home, Alexa and Cortana among others are starting to enter our daily lives. Therefore, it is plausible to assume that 
such agents %we [kmm- rephrase]
have access to conversation with a human for extracting commonsense knowledge. In this paper, we work with the Learning by Instruction Agent (LIA) \citep{azaria2016instructable,labutov2018lia} and develop a commonsense reasoning system for her called CORGI (\textbf{CO}mmonsense \textbf{R}easonin\textbf{G} by \textbf{I}nstruction). In what follows, we present our definition of commonsense reasoning for LIA after briefly introducing her. % It is worth noting, however, that the proposed method is not limited to a specific conversational agent. 
% \kmcomment{Anthropomorphizing LIA (referring to the agent as 'her') is a somewhat political choice -- it's okay to make it, but make it consciously.}

LIA is an intelligent agent that operates on 
a user's smartphone. %the phone [kmm- rephrase (you do not call LIA; there are other agents where you call in so it's important to make the distinction)]
%and can be taught new commands through user instructions. [kmm- removed (covered in the very next sentence)]
End users add new functionalities to LIA through verbal instructions and teach her how to perform new tasks. For example, the user can tell LIA, ``whenever it snows at night, wake me up 30 minutes early''. If LIA does not understand how to perform this task, she will ask the user to instruct her by breaking the task down into a set of steps in a teaching session. In this case, the user can say, ``(first) open the weather app, (second) see if the night weather condition is snow, (third) if true then adjust my alarm to 30 minutes earlier''. After this teaching session, LIA can perform this task. 

One phenomenon we have noticed in collecting these types of ``Whenever $S$ occurs, then do $A$'' instructions is that people often {\em underspecify} the precondition $S$. For example, one instructor might want to wake up early when it snows because they are concerned about getting to work on time.  For this user, the implied precondition is not really ``whenever it snows,'' but instead ``whenever it snows enough to cause traffic slowdowns, and it's a workday.'' The point is %Amos: I think that "the point is" doesn't sound good. How about "Naturally,"?
that people often fail to specify  such detailed conditions, perhaps because they are used to speaking to other people who possess the common sense needed to infer the more specific intent of the speaker.

Our goal for LIA is to use background commonsense knowledge to reason about the user's more specific intent, and to discuss this with the user in order to create the correct preconditions for the recommended action.  Therefore, we assume LIA can obtain statements from the user that fit the logical template ``Whenever $S$ occurs, do $A$ because I want to achieve goal $G$.''\footnote{Note in LIA's conversational setting, if the user gives an instruction of the form ``Whenever $S$ occurs, do $A$.'' and omits the reason, then LIA can simply respond ``Why do you want to do that?'' in order to prompt for the missing reason $G$.}
%LIA then generalizes from this statement to other actions. For example, if the user says, ``if the weather is rainy tomorrow then set an alarm for 1 hour later'', LIA can perform this action without needing to be taught again. However, this generalization has some limitations which 
%stem %stems [kmm- limitations->stem]
%from the lack of reasoning capabilities in LIA. 
For example consider the following two statements: %, [kmm- colon replaces comma]
\begin{itemize}
\item Whenever it snows at night, wake me up 30 minutes early because I don't want to be late to work
\item Whenever it snows at night, wake me up 30 minutes early because I have never seen the snow before 
\end{itemize}
Note that in the first statement, the user will not want to wake up early on a weekend or a holiday (assuming that they do not work then) whereas in the second scenario, the user will want to wake up early regardless of the date in order to see snow for the first time -- but might not want to wake up early once she has seen snow for the first time.

In CORGI, the role of commonsense reasoning is to derive the intended condition to use in place of the stated $S$ given an ``If $S$ then do $A$ because $G$'' statement from the user. Its general approach is to derive an explanation of how action $A$, performed in state $S$ will achieve goal $G$, and then to derive the intended precondition $S$ by collecting the preconditions on $S$ that allow this explanation to hold.  CORGI has access to a limited amount of general background knowledge about the world, represented in a logic programming language. Reasoning reduces to using this background knowledge to perform multi-hop logical inference. If no reasoning path is found, CORGI initiates a conversation with the user to extract relevant background knowledge and adds it to its underlying understanding of the world.  This newly acquired background knowledge will be used in future user interactions with CORGI. In essence, we are performing knowledge base completion through conversation, on a need-driven basis. Note that in earlier work Hixon et al. \cite{hixon2015learning} perform relation extraction using human interaction for question answering. Although the general idea of using human interaction is similar to our proposal, the information extraction method and the problem studied in \cite{hixon2015learning} differs from our setting. To the best of our knowledge, CORGI is the first conversational assistant that targets completing reasoning paths.
% \amoscomment{'their' seems like a typo, not sure what you are saying} --> resolved
% Therefore, our reasoning system is a commonsense reasoning by instruction engine. 

% \amoscomment{I find it hard to understand when 'LIA' refers to the agent from previous work, and when it refers to new capabilities added by this work.} \facomment{is this resolved now, Amos?} %Yes, Thanks!

% In this paper we develop a reasoning system for LIA that is capable of commonsense reasoning in order to generalize correctly given if-then user commands through the because statement.

CORGI's main reasoning component is the multi-hop inference system. Since the knowledge is represented in a logic programming language, the underlying inference algorithm is backward chaining. However, backward chaining in its traditional form is not robust to variations in natural language. This is specifically of importance since CORGI allows open-domain dialog with the user
to reduce the startup cost of the user having to learn a %so that the user is not limited to a [kmm- is this rephrase correct?]
specific grammar or vocabulary. Therefore, there is no parsing algorithm to resolve these variations. For example, in 
%the [kmm- removed]
traditional backward chaining, the statements ``if the forecast is snow tonight'' and ``if the weather is snowy tonight'' are thought of as two different statements whereas we want them both to map to the same representation. In order to address this, we propose a ``soft backward chaining'' algorithm that learns continuous representations or embeddings of the logical statements in the background knowledge. This will allow CORGI to indicate the equivalence of semantically similar statements based on the distance of their learned representations in the vector space. This soft backward chaining allows us to bridge a gap between symbolic AI and neural approaches using the best of both worlds.

% CORGI's soft backward chaining algorithm is end-to-end differentiable and is trained by looking at the proof traces of similar 

% kmm: resolve AA's confusion here with "compatible with deep-learning techniques"

% . This multi-hop reasoning system is end-to-end differentiable and supports soft multi-hop reasoning to account for natural language variations. \amoscomment{I might be missing something, but what does it mean being end-to-end differentiable, are you referring to differentiable functions (those that have a derivative), is this required in order to train the system? Or do you mean that the system obtains knowledge piece by piece. I guess you mean the former, but I did struggle with this.}

% \tmcomment{There are two main themes: 1. claiming that the reasoning can help get the generalization right, 2. how to do the reasoning in a way that is correct}

% \tmcomment{why are we doing reasoning this way and how can we make sure we can do it successfully. we need to compare it with the approximate inference and probabilistic inference methods for performing reasoning}

% \tmcomment{Our contributions are two fold. one is that we are proposing a reasoning strategy through conversation and are proposing to extract the missing information just in time to perform the correct reasoning. No one has the capacity to store the world's largest kb and until now everyone has tries to maintain the largest knowledge bases that there are. However, we are proposing a new way of doing this and it is to extract the correct part of the missing knowledge from the user. This is our grand goal and we have performed a set of small steps towards it... [layout the steps]. Another contribution is the soft unification part. In order to make this work we need to combine symbolic AI with neural approaches to bridge the gap and use the best of both worlds.}

% \tmcomment{reviewer question: How do we know if our method scales? No one has a large enough knowledge base that contains all the information there is in the world. And currently everyone in the field is trying to do this. However, we are proposing a method for extracting the right information just in time needed to perform the reasoning}

% \tmcomment{We do not know the user will give us the right answer even if we ask the right question} \kmcomment{Focus less on ``right'' answer/question here; there are many-to-many possible question/answer pairs that will give a good result. Make a definition of what success means in this context.}

% \tmcomment{Our goal is to have a conversation with the user and the main goal is to have the user give us the missing part of the information and in a funny/not so funny way this is a feature of the system}

% \tmcomment{consider the problem of learning procedures including triggers by conversation. When humans give instructions they are imprecise. In this project we are interested in having the human construct a reasoning system that does not have the commonsense and we want to use conversation to extract the commonsense from the user. We state this as our overarching grand research goal and mention carefully that we are taking a few steps towards this grand goal. Although it does not solve all of it but it is a step towards achieving this goal. This is just a first step however its a part of a very well reasoned and ambitious project. Then we also carefully describe the limitations of the project.}
\section{Preliminaries}
\label{sec:background}
\paragraph{Setup.} In this paper, we consider a supervised multi-class classification problem, where $\mathcal{X}$ denotes the input space and $\mathcal{Y}=\{1,2,...,C\}$ denotes the label space. The training set $\mathbb{D}_\text{in} = \{(\*x_i, y_i)\}_{i=1}^N$ is drawn \emph{i.i.d.} from the joint data distribution $P_{\mathcal{X}\mathcal{Y}}$.
Let $\mathcal{P}_\text{in}$ denote the marginal distribution on $\mathcal{X}$. Let $f: \mathcal{X} \mapsto \mathbb{R}^{|\mathcal{Y}|}$ be a neural network trained on samples drawn from $P_{\mathcal{X}\mathcal{Y}}$ to output a logit vector, which is used to predict the label of an input sample. 

\vspace{0.05cm}
\noindent\textbf{Out-of-distribution detection.} 
Our framework concerns a common real-world scenario in which the algorithm is trained on the ID data but will then be deployed in environments containing \emph{out-of-distribution} samples from {unknown} class $y\notin \mathcal{Y}$ and therefore should not be predicted by $f$. Formally, OOD detection can be formulated as a level-set estimation problem. At test time, one can perform the  following test to determine whether a sample $\*x \in \mathcal{X}$ is from $\mathcal{P}_{\text{in}}$ (ID) or not (OOD):
\begin{equation}
	G_{\lambda}(\*x) =
	\begin{cases}
		\id & \text{if}\ S(\*x) \geq \lambda, \\
		\ood & \text{if}\ S(\*x) < \lambda \\
	\end{cases}
	\label{equ:ood_score}
\end{equation}
where $S(\*x$) denotes a scoring function and $\lambda$ is a chosen threshold such that a high fraction ($95\%$) of \id data is correctly classified. Test samples with higher values of S($\*x$) are classified as \id and vice-versa. 
\section{Commonsense for Zero-Shot NLVL}
\label{sec:proposedSection}

\subsection{Problem Formulation}
We denote an input video as $V$, and its grounding annotations as \(\left( Q,V_{\text{span}}\right) \), where $Q$ is the query representation and \(V_{\text{span}}\!=\!\left( t_{s},t_{e}\right)\) is the corresponding video moment span annotation, with \(t_{s}\) and \(t_{e}\) representing the start and end timestamps, respectively. Learning to localize a video moment conditioned on a query entails maximizing the expected log-likelihood of the model parameterized by \(\theta\). In its typical setting, this can be formulated as follows:
\begin{equation}
\label{eq:groundingOriginal}
    \theta ^{\ast }=\arg \max _{\theta } \mathbb{E}\left[ \log p_{\theta }\left(  V_{\text{span}} | V,Q\right) \right]. 
\end{equation}
In the zero-shot setting, the goal is to learn this task without parallel video-query annotations. Hence, the query and video moment annotations are derived from $V$, using a dynamic video moment proposal method followed by a pseudo-query generation mechanism. Formally,  \(V_{\text{span}}\,\!{=}\!\,f_{\text{span}}(V)\) and \(Q\,\!{=}\!\,f_{pq}(V_{\text{span}})\), where $f_{\text{span}}$ and $f_{\text{pq}}$ are video moment proposal and pseudo-query generation mechanisms, respectively. Given that $f_{\text{span}}$ and $f_{\text{pq}}$ are responsible for generating the annotations, the performance of the localization model heavily depends on the quality of these modules. Existing methods face challenges in aligning \(Q\) to \(V_{\text{span}}\) due to noise introduced by ungrounded pseudo-query generation mechanisms. 
To address this, we simplify \(f_{\text{pq}}\) while augmenting cross-modal understanding by leveraging external information in the form of a commonsense graph \(G_{C}(C, E)\) with \(n_c\) nodes, where \(C\!=\!\left\{c_{1}, c_{2}, \dots, c_{n_{C}}\right\}\) are the concept node vector representations and \(E\) is the set of weighted directed edges, respectively. Accordingly, learning can be formulated as
\begin{equation}
\label{eq:groundingOurs}
    \theta ^{\ast }=\arg \max _{\theta } \mathbb{E}\left[ \log p_{\theta }\left(  V_{\text{span}}| V,Q,G_{C}\right) \right].
\end{equation}

\noindent Figure \ref{fig:approach} shows both training and inference flows.
\subsection{Pseudo-supervised Setup}
\modelname first processes a raw video with a video moment proposal $f_{\text{span}}$ module that extracts important video segments capturing key events, and a pseudo-query generation $f_{\text{pq}}$ that generates text query annotations corresponding to the extracted video segments.

\paragraph{Dynamic Video Moment Proposal ($f_{\text{span}}$).}
We adopt the dynamic video moment proposal approach proposed by \citet{nam_zero-shot_2021}. Specifically, $f_{\text{span}}$ primarily comprises a k-means clustering mechanism that groups semantically similar and temporally proximal video frame features together to extract atomic moments. To obtain frame features, we consider the columns of a frame-wise similarity matrix derived from the CNN features of individual frames. We enforce temporal proximity by concatenating the frame index to the features. Composite video moments are then formed by combining neighboring atomic moments, and a subset of all possible combinations is sampled uniformly at random. The resulting set of video moments corresponds to $V_{\text{span}}$.

\paragraph{Pseudo-query Generation ($f_{\text{pq}}$).} The pseudo-query is constructed as a collection of objects present in the video. To generate the pseudo-query, we employ an off-the-shelf object detector, enabling the extraction of pertinent objects in \(V_{\text{span}}\). We adopt a top-$k$ strategy to sample the $k$ most probable object predictions associated with the query \query.

\paragraph{Video Encoder.}
We uniformly sample $T$ frames from $V$ and extract their CNN (\eg, I3D~\cite{qian_locate_2022}) features. These features are contextually encoded using a video encoder ${\phi}_{v}$ to yield frame features ${\phi}_{v}(V)\!=\!\left\{ v_{1},v_{2},\ldots,v_{T}\right\}$ where $v_{i}\in\mathbb{R}^{d}$, and $d$ is the common video/query encoding dimension. We implement ${\phi}_{v}$ as a GRU-based encoder.

\paragraph{Query Encoder.}
Our pseudo-query $Q$, composed of up to $k$ tokens, is encoded using a query encoder ${\phi}_{q}$ that generates query embeddings ${\phi}_{q}(Q)\!=\!\left\{ q_{1},q_{2},\ldots,q_{k}\right\}$, for the top-$k$ detected objects extracted from the pseudo-query generation. Here, $q_{i}\in \mathbb{R}^{d}$ and $d$ is the common video/query encoding dimension. We implement ${\phi}_{q}$ as a bi-directional GRU-based encoder preceded by a trainable embedding layer. 

\subsection{Commonsense Enhancement Module}
\label{sec:cem}
To enrich the encoded video and query features with information grounded in commonsensical knowledge, we introduce a Commonsense Enhancement Module (CEM), pictorially described in Figure~\ref{fig:cem}. This enhancement helps inject necessary information into video and query representations, which can not just help bridge the gap between the available visual and textual cues but also provide rich information to the downstream span localization module. 

\begin{figure}[t!]
    \centering
    \includegraphics[width=0.8\linewidth]{figures/figure_files/Cem.pdf}
    \caption{\modelname Commonsense Enhancement Module (CEM). CEM comprises a concept encoder and an enhancement mechanism that uses the previously encoded concept vectors to update a given input vector (video/query vectors). The concept encoder employs a Graph Convolution Network for encoding the nodes (concepts) of \(G_C\). 
    }
  \label{fig:cem}
\end{figure}

CEM includes a set \(C\!=\!\left\{c_{1}, c_{2}, \dots, c_{n_{C}}\right\}\) of \(n_{C}\) concept vectors, where \(c_{i} \in \mathbb{R}^{d}\) and \(d\) is the concept feature dimension (same dimension as $\forall v_i \in V$ and $\forall q_i \in Q$). In general, given source feature vectors $S\!=\!\left\{ s_{1},s_{2},\ldots,s_{n}\right\}$ with individual feature vectors $s_{i \in [1,n]} \in \mathbb{R}^{d}$, the enhanced feature vectors $S_{C}$ are obtained using a commonsense enhancement mechanism $\phi_{C}$.
We implement this commonsense enhancement step $\phi_{C}$ as a cross-attention mechanism that enriches source input features, attending over $S$ guided by the commonsense concept vectors $C$, \ie, 
\begin{equation}
\label{eq:cenhance}
\scalemath{1}{
    }
    S_{C} = S + \phi_{C}(S) = S + \sigma \left( \frac{SW_{Q}(CW_{K})^{T}}{\sqrt{d}} \right) C W_{V},
\end{equation}
where $\sigma$ is a softmax activation, \(W_{Q}\), \(W_{K}\), \(W_{V}\) are trainable matrices and \(d\) is the common dimension of the vectors \(S\) and \(C\). In our setting, the source feature vectors $S$ are either video $V$ or pseudo-query $Q$ features. We build separate enhancement mechanisms for $V$ and $Q$, \ie, the projection matrices \(W_{Q}\), \(W_{K}\), \(W_{V}\) are not shared between $Q$ and $V$. We elaborate more on the rationale in the appendix.
The enriched video and pseudo-query features are denoted as \(V_{C}\!=\!\phi_{C_{\text{vid}}}(V)\) and \(Q_{C}\!=\!\phi_{C_{\text{pq}}}(Q)\), respectively.

\paragraph{Concept Encoder.}
The concept vectors \(C\) mentioned above are feature representations that internally form the nodes of the commonsense graph, \(G_C\). Accordingly, graph \(G_{C}\) is represented as a matrix, where \(G_{C(i,j)}\) represents the total number of directed relational edges between \(c_{i},c{j} \in C\) that start at \(c_i\) and end at \(c_j\). To encode the commonsense information, we employ Graph Convolutional Networks (GCN) \cite{hammond_wavelets_2011}. The concept encoder is composed of $L$ graph convolution layers, each of which performs a convolution step
\begin{equation}
\scalemath{1}{
    C^{\left(l+1\right)}=\sigma \left( AC^{\left(l\right) }W^{\left( l\right) }\right),
    }
\end{equation}
where $C^{\left(l\right)}$ are node (concept) features and $W^{\left( l\right)}$ trainable weight matrix of layer $l \in [1, L]$, $\sigma$ is a nonlinear activation function, and $A$ is the adjacency matrix obtained by normalizing graph $G_C$ with the degree matrix $D$. Since $G_C$ is a directed graph, normalization can be formulated as $A\!=\!D^{-1}G_{C}$.

\paragraph{Commonsense Information.}
We use ConceptNet \cite{speer_conceptnet_2017}, a popular knowledge graph that provides information spanning various types of relationships such as physical, spatial, behavioral, \etc To ensure that the ConceptNet information utilized is relevant to themes found in the video data, we consider the set of objects available in pseudo-queries and include the top-$k$ most frequently occurring objects to be the seed concept set \(C\). We extract the  ConceptNet subgraph that includes all edges incident between the concepts in \(C\). 
We filter the edge types based on a pre-determined relation set \(R\), which is compiled to involve relations that are relevant to the nature of the video localization task, \eg, spatial (\textit{AtLocation}, \etc) and temporal (\textit{HasSubevent}, \etc) relations are useful for video understanding, while \textit{RelatedTo} and \textit{Synonym} are fairly generic relations that add little information to the localization task. Table \ref{tab:relations} shows the relations included in \(G_C\).

\paragraph{Cross-Modal Interaction Module.} The commonsense enriched video and query features, \(V_{C}\) and \(Q_{C}\), are fused with a multi-modal cross-attention mechanism. We employ a two-step fusion process. First, Query-guided Video Attention (QVA) is applied to attend over video $V_C$, and Video-guided Query Attention (VQA) attends over query $Q_C$ guided by video $V_C$, resulting in updated features $V_C'$ and $Q_C'$, respectively. Both QVA and VQA utilize Attention Dynamic Filters~\cite{rodriguez_proposal-free_2020} that adaptively modify video features, dynamically adjusting them in response to the query, and vice versa. Next, the attended features are fused using a cross-attention mechanism over $V_C'$ guided by $Q_C'$, resulting in localized video features $V_{C_{\text{loc}}}$.

\paragraph{Temporal Regression Module.}
The final step involves a regression layer that approximates $\hat{V}_{\text{span}}$. We employ attention-guided temporal regression to estimate the span of the target video moment. To find important temporal segments relevant to the query, the fused features $V_{C_{\text{loc}}}$ are temporally attended based on the query features to obtain $V_{\text{ta}}$. Then, the span boundaries are localized using a regressor implemented as a Multi-Layer Perceptron (MLP).

\begin{align}
{o}_i = \sigma\left({W}_{1} V_{C_{\text{loc}_i}} + {b}_{{1}}\right) \\
V_{\text{ta}} = \sum_{i=1}^{T} o_i V_{C_{\text{loc}_{i}}} \\
[\hat{t}_s, \hat{t}_e] = {W}_2 {V}_{\text{ta}} + {b}_{2}.
\end{align}
Here, ${W}_{1}$ and ${b}_1$ are the weight matrix and bias vector of the temporal attention MLP, $\sigma$ represents the sigmoid activation function, $V_{C_{\text{loc}_i}}$ stands for the encoded localized video features, ${V}_{\text{ta}}$ represents the temporally attended video features, ${W}_2$ and ${b}_2$ denote the weight matrix and bias vector of the regression MLP, and $[\hat{t}_s, \hat{t}_e]$ correspond to the start and end timestamps of the predicted video span $\hat{V}_{\text{span}}$.

\begin{table}[t!]
\centering
\resizebox{\linewidth}{!}{
\begin{tabular}{ll}
\toprule
\textbf{Category} & \textbf{Relations}                                                                                         \\ \toprule
Spatial           & AtLocation, LocatedNear                                                                                    \\ \midrule
Temporal          & \begin{tabular}[c]{@{}l@{}}HasSubevent, HasFirstSubevent, HasLastSubevent, HasPrerequisite\end{tabular} \\ \midrule
Functional        & UsedFor                                                                                                    \\ \midrule
Causal            & Causes                                                                                                     \\ \midrule
Motivation        & MotivatedByGoal,  ObstructedBy                                                                             \\ \midrule
Other             & CreatedBy, MadeOf                                                                                          \\ \midrule
Physical          & \begin{tabular}[c]{@{}l@{}}HasA, HasProperty, Antonym, SimilarTo\end{tabular}                      
\\ \bottomrule
\end{tabular}
}

\caption{Relations in the Commonsense Enhancement Module (CEM) grouped by categories.}
\label{tab:relations}

\end{table}
\subsection{Training and Inference}
The training objective is 
$\mathcal{L}_{loc} = \mathcal{L}_{treg}+\lambda \mathcal{L}_{ta},$ where \(\lambda\) is a balancing hyperparameter, \(\mathcal{L}_{ta}\) is a temporal attention guided loss and \(\mathcal{L}_{treg}\) is the regression loss.  The temporal attention-guided loss is defined as
\begin{equation}
\label{tatt}
\mathcal{L}_{ta} = \frac{\sum^{T}_{i=1}g_{i}\log \left( a_{i}\right)}{\sum^{T}_{i=1}g_{i}},
\end{equation}
where \(a_{i}\) is the attention weight for video frame \(v_{i}\) and \(g_{i}\) is the attention mask for \(v_{i}\), that is assigned to \(1\) if \(v_{i}\) is inside the target video segment, and \(0\) otherwise. 
This objective encourages the model to produce higher attention weights for video segments that are relevant to the query. 
On the other hand, \(\mathcal{L}_{treg}\) dictates the video span boundary regression and is the sum of smooth $\ell_1$ distances between start and end timestamps of the ground truth and predicted spans, \ie,
\begin{equation}
\label{treg}
\mathcal{L}_{treg} = \text{smooth}{\ell_1}(t_{s}, \hat{t}_{s}) + \text{smooth}{\ell_1}(t_{e}, \hat{t}_{e}).
\end{equation}
Here, $t_{s}$ and ${t}_{e}$ represent the ground truth start and end timestamps and $\hat{t}_{s}$ and $\hat{t}_{e}$ the predicted start and end timestamps, respectively.
The integration of a smoothing mechanism enhances training stability and improves the model's ability to handle outliers. Finally, during inference, we employ an off-the-shelf part-of-speech tagger to extract nouns from the text input query and feed them as query input to the trained \modelname video localizer.
\section{Assessment}
\label{sec:assessment}
\subsection{Experimental Setup}
We implement our PCDNet in PyTorch \cite{paszke2019pytorch} and train it for 300 epochs with the batch size of 32 on two NVIDIA GeForce RTX 3090 GPUs. We use stochastic gradient descent (SGD) \cite{amari1993backpropagation} with a momentum of 0.937 and a weight decay of $5 \times 10 ^{-4}$ during training. The initial learning rate is set to 0.01 and decayed to 0.001 using a cosine annealing schedule. We initialize PCDNet randomly and load the weights of CSPDarknet53 \cite{wang2020cspnet} pre-trained on ImageNet \cite{imagenet_cvpr09} for the encoder part. To increase the diversity and complexity of the training samples, we apply data augmentations including random cropping, random flipping, and mosaic \cite{redmon2018yolov3}. We use the evaluation metrics of Microsoft COCO \cite{lin2014microsoft} for validation.

\begin{table}[ht]
\caption{Quantitative comparison against state-of-the-art polarization-based detectors ($\star$), single-stage detectors ($\dag$), two-stage detectors ($\ddag$), anchor-based detectors ($\triangle$), anchor-free detectors ($\circ$), and self-supervised method ($\S$).}
\small
\centering
\renewcommand\arraystretch{0.9}
\setlength{\tabcolsep}{2.6pt}
\begin{tabular}{lccccc}
\hline\hline
Methods	&	Pub'Year	&	Backbone	&	AP	&	AP50	&	AP75	\\
\hline
Faster R-CNN$^{\ddag\triangle}$ 	&	NeurIPS'15	&	Res50	&	44.8	&	75.4	&	45.4	\\
SSD$^{\dag\circ}$ 	&	ECCV'16	&	VGG16	&	25.5	&	52.6	&	22.6	\\
Cascade R-CNN$^{\ddag\triangle}$ 	&	CVPR'18	&	Res50	&	45.8	&	73.2	&	47.8	\\
CornerNet$^{\dag\circ}$ 	&	ECCV'18	&	Res50	&	19.8	&	47.4	&	29.6	\\
P-SSD I$^{\star\dag\circ}$ 	&	ITSC'19	&	VGG16	&	25.9 	&	53.1	&	22.7	\\
P-SSD S$^{\star\dag\circ}$ 	&	ITSC'19	&	VGG16	&	23.0 	&	48.9	&	20.1	\\
FCOS$^{\dag\circ}$ 	&	ICCV'19	&	Res50	&	23.1	&	50.9	&	18.4	\\
DH R-CNN$^{\ddag\triangle}$ 	&	CVPR'20	&	Res50	&	32.7	&	65.3	&	28.2	\\
Dynamic R-CNN$^{\ddag\triangle}$ 	&	ECCV'20	&	Res50	&	46.2	&	74.2	&	48.0	\\
EfficientDet$^{\ddag\triangle}$ 	&	CVPR'20	&	D3	&	45.3	&	73.0	&	46.3	\\
VarifocalNet$^{\dag\circ}$  & CVPR'21 & Res50 & 44.2 &	73.5 &	44.4	\\
D-DETR$^{\dag\circ}$ 	&	ICLR'21	&	Res50	&	43.8	&	74.9	&	44.3	\\
DDOD$^{\dag\circ}$ 	&	MM'21	&	Res50	&	43.5	&	73.0	&	43.3	\\
TOOD$^{\dag\triangle}$ 	&	ICCV'21	&	Res50	&	44.3	&	74.3	&	44.6	\\
YOLOX$^{\dag\circ}$ 	&	arXiv'21	&	YOLOX-l	&	54.3	&	82.5	&	56.7	\\
YOLOv7$^{\dag\triangle}$	&	arXiv'22	&	Dark53	&	57.6	&	84.3	&	60.3	\\
RTMDet$^{\dag\circ}$ 	&	arXiv'22	&	RTMDet-l	&	53.9	&	81.4	&	56.7	\\
DINO$^{\dag\circ\S}$ 	&	ICLR'22	&	Res50	&	52.7	&	81.8	&	54.8	\\
YOLOv8$^{\dag\circ}$ 	&	-'23	&	YOLOv8-l	&	56.8	&	83.6	&	59.0	\\
\hline
\textbf{PCDNet$^\star$}	&	\textbf{Ours}	&	Dark53	&	\textbf{58.5}	&	\textbf{85.2}	&	\textbf{61.5}	\\
\hline\hline
\end{tabular}
\label{tab:comparison}
\end{table}

\begin{figure*}[htp]
    \centering
    \begin{center}
        % \includegraphics[width=\linewidth]{figure/comparison.pdf}
        \includegraphics[width=\linewidth,height=10.5cm]{figure/comparison.pdf}
    \end{center}
    \caption{Qualitative comparison of PCDNet against state-of-the-art detectors retrained on RGB-P Car dataset.} 
    \label{fig:comparison}
\end{figure*}

\subsection{Qualitative and Quantitative Evaluation}
We extensively compare our PCDNet with 19 state-of-the-art methods by retraining and testing all methods on the RGB-P Car dataset using their original settings. The compared methods include two-stage detectors such as EfficientDet \cite{tan2020efficientdet} and the R-CNN family \cite{Ren_2017, Cai_2019, zhang2020dynamic}, and one-stage detectors such as SSD \cite{liu2016ssd}, and YOLO family \cite{ge2021yolox, wang2022yolov7, ultralytics2023yolov8}. These methods also comprise anchor-based methods such as the R-CNN family and YOLOv7 \cite{wang2022yolov7}, and anchor-free methods such as CornerNet \cite{law2018cornernet}, VarifocalNet \cite{zhang2021varifocalnet}, and YOLOv8 \cite{ultralytics2023yolov8}. Some detectors use traditional convolutional networks such as FCOS \cite{tian2019fcos} and RTMDet \cite{lyu2022rtmdet} while others use transformer structures, such as DeformableDETR \cite{zhu2020deformable} and DINO \cite{zhang2022dino} that employs self-supervised learning. We also include the P-SSD \cite{blin2019road} that utilizes polarization information. The quantitative evaluation results are reported in Tab. \ref{tab:comparison}. We can see that our method outperforms all competing state-of-the-art methods. 

Fig. \ref{fig:comparison} further qualitatively demonstrates the benefits of our method: a) in poorly lit indoor parking lots, distinguishing black cars behind pillars is extremely challenging (the first two rows). The compared methods tend to conflate the shadow and the black car (\textit{i.e.}, merging cars on either side of the pillar into a single entity or treating partial views of the car as one object) while our PCDNet can handle such ambiguities; b) in the third example, all methods except our PCDNet fail to detect a partially visible car obstructed by another car or misplace it with the previous car; c) in the fourth example, RGB-based methods wrongly identify distant pedestrians as cars, but our PCDNet method can effectively eliminate such interference with the help of polarization cues; d) the fifth and sixth examples depict black cars in an outdoor parking lot at night which are very hard to be distinguished in the RGB image. Despite the enhancement through ZeroDCE \cite{guo2020zero}, the sixth example remains unclear. By contrast, polarization imaging is robust to low light conditions, enabling our robust car detector PCDNet; and e) the last row shows a virtual car reflected in a mirror located at the upper-left corner of the image. The mirrored virtual car and the rest of the mirror regions exhibit similar and smooth AoLP, providing useful cues for PCDNet to recognize this region as background. 


\subsection{Ablation Study}
\textbf{Impact of Spectral Intensity and Polarization Cues.} We conduct a series of ablation experiments to demonstrate the effects of spectral intensity and polarization cues on car detection (Tab. \ref{tab:abl_input}).
The results show that: a) combining different forms of polarization cues with RGB as the input of PCDNet can improve the car detection accuracy (\textit{C}, \textit{D}, \textit{F}, \textit{G}, \textit{K} and \textit{L} are higher than \textit{B}); b) DoLP cues have a greater impact than AoLP cues (\textit{D}, \textit{J} and \textit{L} are better than \textit{C}, \textit{I} and \textit{K}, respectively); c) stacking AoLP and DoLP on RGB in the channel dimension does not boost performance (\textit{E} is slightly lower than \textit{B}), possibly because the characteristic gap between different modalities hinders effective features extraction; d) spectral intensity and polarization are more beneficial than monochromatic intensity and polarization for car detection (comparing paired \textit{B} and \textit{H}, \textit{C} and \textit{K}, \textit{D} and \textit{L}, \textit{I} and \textit{K}, \textit{J} and \textit{L}); e) enhancing RGB image via ZeroDCE \cite{guo2020zero} is less effective than introducing polarization (\textit{M} performs worse than \textit{C}-\textit{G}, \textit{K} and \textit{L}).
Fig. \ref{fig:abl_input} provides visual support for these observations.

\begin{table}[t]
\small
\centering
\caption{Quantitative comparisons of ablation with different inputs. ``stacked I'' denotes the stacked intensity measurements with a linear polarization angle of 0$^{\circ}$, 45$^{\circ}$ and 135$^{\circ}$ and ``stacked S'' refers to the stacked Stokes elements S0, S1 and S2 \cite{blin2019road}.}
\begin{tabular}{clccc}
\hline\hline
	&	PCDNet Input	&	AP	&	AP50	&	AP75	\\
 \hline
\textit{A}	&	RGB, AoLP and DoLP (original)	&	58.5 	&	85.2 	&	61.5 	\\
\hline
\textit{B}	&	RGB only	&	57.6 	&	84.3 	&	60.2 	\\
\textit{C}	&	RGB and AoLP	&	58.0 	&	84.6 	&	60.7 	\\
\textit{D}	&	RGB and DoLP	&	58.3 	&	85.4 	&	61.1 	\\
\textit{E}	&	stacked RGB, AoLP and DoLP	&	57.5 	&	84.3 	&	59.9 	\\
\textit{F}	&	RGB and stacked I	&	58.0 	&	84.1 	&	61.0 	\\
\textit{G}	&	RGB and stacked S	&	57.8 	&	84.8 	&	60.4 	\\
\textit{H}	&	Gray only	&	57.4 	&	84.3 	&	60.0 	\\
\textit{I}  &   Gray and mono AoLP & 57.5 & 84.5 & 60.5 \\
\textit{J}  &   Gray and mono DoLP & 57.6 & 84.9 & 60.1 \\
\textit{K}	&	RGB and mono AoLP	&	57.9 	&	84.6 	&	60.5 	\\
\textit{L}	&	RGB and mono DoLP	&	58.2 	&	84.9 	&	60.6 	\\
\textit{M}  &   Enhanced RGB & 57.4 & 84.0 & 60.0 \\
\hline\hline
\end{tabular}
\label{tab:abl_input}
\end{table}

\begin{figure}[t]
    \centering
    \includegraphics[width=1\linewidth]{figure/abl_input.pdf}
    \caption{Qualitative comparison of ablation with different inputs. The model with RGB intensity only is susceptible to interference from ghost car caused by water on the road.}
    \label{fig:abl_input}
\end{figure}

\textbf{Influence of PCDNet Components.}
First, we investigate the performance of different strategies for fusing AoLP and DoLP inputs. From Tab. \ref{tab:abl_module}(\textit{A}-\textit{D}), we observe that our PI module is more effective than the simple fusion methods including concatenation, addition and element-wise multiplication.
Second, by removing MP module \ref{tab:abl_module}(\textit{E}) from the original PCDNet (A), the detection performance declines. This demonstrates that exploring the polarized material features of cars across all learning samples is useful. We also explore the influence of applying MSP and MCP on different levels of features. The results in Tab. \ref{tab:abl_module}(\textit{A},\textit{F}-\textit{G}) show that applying MSP on shallower features and MCP on deeper features can yield better performance.
Finally, we validate the effectiveness of CDDQ module.
Removing the CDDQ module (\textit{I}) from PCDNet (\textit{A}), which causes the feature extraction processes of the RGB and polarization to be independent from each other, leads to the performance drop. We also demonstrate the benefits of the CWDA and SDMD in the CDDQ module by removing either of them (\textit{J} and \textit{K}). 

\begin{table}[t]
\small
\centering
\caption{Quantitative comparisons of ablation with different modules demonstrate that all component of PCDNet contributes to the overall performance. We used sequences of three letters separated by '-' and enclosed in parentheses to represent different combinations of MSP and MCP.}
\begin{tabular}{clccc}
\hline\hline
	&	Ablation	&	AP	&	AP50	&	AP75	\\
 \hline
\textit{A}	&	PCDNet (original)	&	58.5 	&	85.2 	&	61.5 	\\
\hline
\textit{B}	&	Input RGB and [AoLP DoLP]	&	58.2 	&	85.4 	&	60.9 	\\
\textit{C}	&	Input RGB and AoLP+DoLP	&	58.1 	&	84.8 	&	60.5 	\\
\textit{D}	&	Input RGB and AoLP*DoLP	&	58.1 	&	84.8 	&	60.5 	\\
\hline
\textit{E}	&	A \textit{w/o} MP	&	56.9 	&	84.2 	&	59.2 	\\
\textit{F}	&	A \textit{w/} M(S-S-S)P	&	58.2 	&	85.2 	&	60.8 	\\
\textit{G}	&	A \textit{w/} M(S-C-C)P	&	58.2 	&	85.0 	&	60.9 	\\
\textit{H}	&	A \textit{w/} M(C-C-C)P	&	58.1 	&	85.0 	&	61.1 	\\
\hline
\textit{I}	&	A \textit{w/o} CDDQ	&	58.0 	&	84.7 	&	60.8 	\\
\textit{J}	&	A \textit{w/o} SDMD	&	58.2 	&	85.2 	&	60.8 	\\
\textit{K}	&	A \textit{w/o} CWDA	&	58.3 	&	85.1 	&	61.1 	\\
\hline\hline
\end{tabular}
\label{tab:abl_module}
\end{table}

\subsection{Limitations}

When both the RGB intensity and the polarization measurement yield weak car signals, our method's effectiveness declines. Specifically, in low-light scenarios, when a car approaches on an unlit road, the strong light from its headlights can create a ``hole'' in both the RGB and polarization and obscure the entire car. We illustrate such an example in Fig. \ref{fig:failure} where the extreme HDR and heavy motion blur in the captured image limit its depiction of both RGB and polarization. In these challenging scenarios, prior RGB-based methods and even human vision are powerless.

\begin{figure}[t]
    \centering
    \includegraphics[width=1\linewidth]{figure/failure.pdf}
    \caption{PCDNet has limited ability to handle extreme HDR or heavy motion blur cases.}
    \label{fig:failure}
\end{figure}


\section{Further Understanding of \name}
\label{sec:ablations}

Through comprehensive evaluations in Section~\ref{sec:experiment}, we have established the effectiveness of \name for OOD detection. 
\noindent In this section, we provide an in-depth analysis of several questions: (1) How does the subspace dimension impact the performance? (2) How does OOD detection performance change if we change the subspace selection strategy? 
(3) How does the $k$ in $k$-NN distance impact the OOD detection performance? We show comprehensive ablation studies to answer these questions. 
 For consistency, all ablations are based on CIFAR-100 as ID dataset and DenseNet~\cite{huang2018densely} architecture unless specified otherwise.
 

\begin{figure*}[t!]
\centering
\begin{minipage}{.49\textwidth}
  \centering
 \vspace{0.7cm}
  \includegraphics[width = 1\columnwidth]{images/Frame_6.pdf}
  \vspace{0cm}
    \caption{\small \textbf{Left}: Effect of varying the relevance ratio ($r$) on OOD detection performance when $k=20$. \textbf{Right}: Effect of varying the number of neighbors ($k$) when $r=0.25$. }

    \label{fig:ablations}
\end{minipage}
\hspace{0.1cm}
\begin{minipage}{.48\textwidth}
  \centering
\centering
    \includegraphics[width = 0.78\columnwidth]{images/subspace_new.pdf}
    \caption{
    \small Visualization of learned final-layer weight matrix for $r \in \{1, 0.75, 0.25, 0.05\}$. For each $r$, we visualize a $342$-dimensional weight vector corresponding to each class in CIFAR-100. }
    \label{fig:subspace}
\end{minipage}%


\end{figure*}
\paragraph{Ablation on subspace dimension.} In this ablation, we aim to empirically verify our theoretical analysis in Section~\ref{sec:theory}, and understand the effect of subspace dimension. 
We start by defining the \textbf{relevance ratio} $r = \frac{s}{m} \in (0,1]$, which captures the sparsity of feature space used in training \name. Recall that $m$ is the original dimension of features and $s$ is the dimension we kept in \name. 

In simple words, $r$ represents the ratio between the dimension of the subspace and the full feature dimension.
Specifically, we train and compare multiple models by varying $r = \{ 0.05, 0.15, 0.25, 0.35, 0.55, 0.75\}$. Figure~\ref{fig:ablations} (left) summarizes the effect of $r$ on OOD detection. We observe that: (1) Irrespective of the relevance ratio used, \name is consistently better than KNN~\cite{sun2022knn} when $r < 1$. This validates the efficacy of learning feature subspaces for OOD detection, without excessive hyperparameter tuning. (2) Setting a mild ratio (\emph{e.g.} $r=0.25$) provides the optimal OOD detection performance, which is consistent with one chosen using our validation strategy (see Appendix C.4). (3) In the extreme case, when $r$ is too small (\emph{e.g.} $r={0.05}$), we observe a deterioration in the OOD detection performance. Overall our empirical observations align well with our theoretical insight provided in Section~\ref{sec:theory}. 

\paragraph{Visualization of the learned weight matrix.}

To further verify our method, we visualize in Figure~\ref{fig:subspace} the 
learned final-layer weight matrix under different relevance ratios ($r = s/m$). For each $r$, we visualize the $342$ dimensional weight vector corresponding to each class in CIFAR-100, \emph{i.e.}, a $342 \times 100$ matrix. When $r=1$ (\emph{i.e.} without any subspace constraint), the model utilizes the full feature space. Further, the visualization confirms that decreasing the relevance ratio effectively reduces feature subspace dimensionality. 

\begin{table}[h]
\centering
\small
\resizebox{0.85\linewidth}{!}{
\begin{tabular}{lccc}
\textbf{Method} & \textbf{FPR95}  & \textbf{AUROC} \\
& $\downarrow $& $\uparrow$ \\
\toprule
Random subspace~\cite{ho1998nearest} & 42.21 & 84.97 \\
Subspace with least relevance & 63.96 & 80.82\\
\name (ours) & \textbf{31.25} & \textbf{90.76} \\
\bottomrule
\end{tabular}}
\caption{\small Ablation on subspace selection methods. Best performing results are marked in \textbf{bold}. Model is DenseNet. All values are averaged over multiple OOD test datasets.}
\label{tab:subspace-selection}
\end{table}


\paragraph{Ablation on subspace selection.} A core component of our algorithm involves selecting the  \emph{most relevant} dimension for a class prediction. In particular, the subspace is chosen based on the dimensions that contributed most to the model's output. In this ablation, we contrast our subspace selection mechanism with random subspace~\cite{ho1998nearest}, a classical alternative. The random subspace relies on a stochastic process that randomly selects $s$ components in the feature vector. Simply put, this approach randomly sets $s$ out of $m$ elements in each $R_c$ vector to be 1 and 0 elsewhere. We report ablation results  in Table~\ref{tab:subspace-selection}. For a fair comparison, we use relevance ratio $r = 0.25$ and the number of neighbors $k = 20$ for all methods.  Empirical results highlight that randomly chosen subsets of dimensions are sub-optimal for the OOD detection task. Lastly, we also contrast with selecting the \emph{least relevant} feature dimensions. We replace Equation~\ref{eq:relevance} with: 
\begin{align}
     f_c(\*x; \theta, R_c(\*x)) & =  \min_{R_c(\*x)\in \{0,1\}^m, \|R_c(\*x)\|_1 = s} \langle \*w_c, h(\*x) \odot R_c(\*x) \rangle,
     \label{eq:least_relevance}
\end{align}
which essentially changes from \textsc{max} to \textsc{min}.  As expected, using the least relevant feature dimension results in significantly worse OOD detection performance.


\paragraph{Ablation on number of nearest neighbors $k$.} 
In Figure~\ref{fig:ablations} (right), we visualize the effect of varying the number of nearest-neighbors ($k$) on OOD detection performance. Here the model is trained with $r=0.25$ on CIFAR-100. Specifically, we vary $k \in \{5,10,20,50,100,200,500,1000\}$. We observe that the  OOD detection performance is relatively stable under a mild $k$. In Appendix D.2, we further visualize the interaction between the two hyper-parameters $r$ and $k$ through OOD detection performance. 

\vspace{0.1cm}
\noindent\textbf{\name improves calibration performance.} {
In addition to superior OOD detection performance, we aim to further  investigate the calibration performance of ID data itself. As a quick recap,  the calibration performance measures the alignment between the model's confidence and its actual predictive accuracy. We hypothesize that learning the feature subspace helps alleviate the problem of over-confident predictions on ID data, thereby improving model calibration. We verify in  Appendix D.3 that training with \name indeed significantly improves the model calibration. }

\vspace{0.1cm}
\noindent\textbf{\name scales to large datasets.} 
In this section, we evaluate \name on a more realistic high-resolution dataset ImageNet ~\cite{deng2009imagenet}. Compared to CIFAR-100, inputs scale up in size in ImageNet-100 (we follow standard data augmentation pipelines and resize the input to 224 by 224). For OOD test datasets, we use the same ones in~\cite{huang2021mos}, including subsets of \texttt{Places365}, \texttt{Textures}, \texttt{iNaturalist} and \texttt{SUN}. 
We train the ResNet-101 model for 100 epochs using a batch size of 256, starting from randomly initialized weights. We use SGD with a momentum of $0.9$, and a weight decay of 1e-4. We set the initial learning rate as $0.1$ and use a cosine-decay schedule. We set $r = 0.35$ and $k = 200$ based on our validation strategy described in Appendix C.4. We contrast two models trained with and without subspace learning. The results in terms of FPR95 and AUROC are shown in Figure~\ref{fig:imagenet}. The results suggest that \name remains effective on all the OOD test sets and consistently outperforms KNN. This further verifies the benefits of explicitly promoting feature subspace to combat curse-of-dimensionality. 

\begin{figure}[t]
    \centering
    \includegraphics[width = \linewidth]{images/imagenet_plot.pdf}
    \caption{\small OOD detection performance comparison on ImageNet dataset (ID). \name consistently outperforms KNN across all OOD test datasets on the same architecture.  }
   
    \label{fig:imagenet}
\end{figure}



\vspace{-0.5em}
\subsection{Discussion}
In this section, we analyze the results from the study and provide examples of the 4 scenarios in Section \ref{sec:corgi} that we encountered. As hypothesized, scenario $\mathfrak{A}$ 
% (users not knowing the answer) 
hardly occurred as the commands are about day-to-day activities that all users are familiar with.
% Since our domain is concerned with commonsense knowledge about day-to-day activities, scenario $\mathfrak{A}$ (users not knowing the answer) hardly occurred. \amoscomment{This exact sentence appeared earlier (in the method section). Either remove or at least add "as mentioned / stated" or maybe "as hypothesized" (assuming the next sentence strengthens this)?}
% We did not observe any meaningful correlation between the student's background and their success rate, which is expected since the reasoning tasks are in the context of day-to-day activities. \amoscomment{What do you mean by "student's background", did you measure Pearson's correlation? What exact values did you get? I would recommend just removing this sentence due to lack of space.}
% Let us now discuss why CORGI is not able to achieve
% our observations of the study's dialog transcript and explain why there is a 
% ideal performance in the oracle scenario. This is mainly caused by the inability to extract the relevant knowledge from the human subjects, and partly caused by the limitations of the parser. 
% Scenario $\mathfrak{B}$ refers to cases where users misunderstood the question.
We did encounter scenario $\mathfrak{B}$, however. The study's dialogs show that some users provided means of \emph{sensing} the \textGoal rather than the \emph{cause} of the \textGoal.
% provided the meaning of certain queries rather than commonsense knowledge.
For example, for the reasoning task \emph{``If there are thunderstorms in the forecast within a few hours then remind me to close the windows because I want to keep my home dry''}, in response to the system's prompt \emph{``How do I know if `I keep my home dry'?''} a user responded \emph{``if the floor is not wet''} as opposed to an answer such as \emph{``if the windows are closed''}. Moreover, some users did not pay attention to the context of the reasoning task. For example, another user responded to the above prompt (same reasoning task) %Amos: I recommend removing "same reasoning task"
with \emph{``if the temperature is above 80''}! %Amos: This is an excellent paragraph!
% which is not even correct common sense (except perhaps in certain climates). 
Overall, we noticed that CORGI's ability to successfully reason about an if-then-because statement was heavily dependent on whether the user knew how to give the system what it needed, and not necessarily what it asked for; see Table \ref{tab:dialog} for an example. As it can be seen in Table \ref{tab:user_study}, expert users are able to more effectively provide answers that complete CORGI's reasoning chain, likely because they know that regardless of what CORGI asks, the object of the dialog is to connect the because \textGoal back to the knowledge base in some series of if-then rules (\textGoal/sub-\textGoal path in Sec.\ref{sec:corgi}). Therefore, one interesting future direction is to develop a dynamic context-dependent Natural Language Generation method for asking more effective questions.
% In contrast, CORGI's current method of asking, ``How do I know if \textGoal?'' often mis-states the immediate information need of the system. If users are not prompted with a good question, they cannot be expected to provide useful answers to the system. Therefore, one interesting future direction is to develop a dynamic context-dependent Natural Language Generation method for asking more effective questions. 
%This can be addressed by asking more informative questions that help users to consider the context in their answers. In the future, we will develop dynamic context-dependent Natural Language Generation pipelines to address this. % For example, instead of asking ``how do I know if $\textGoal$'', CORGI could ask, ``what does the $\textAction$ cause that allows you to achieve the $\textGoal$ ?''.

We would like to emphasize that although it seems to us, humans, that the previous example requires very simple background knowledge that likely exists in SOTA large commonsense knowledge graphs such as ConcepNet\footnote{\url{http://conceptnet.io/}}, ATOMIC\footnote{\url{https://mosaickg.apps.allenai.org/kg_atomic}} or COMET \cite{bosselut2019comet}, this is not the case (verifiable by querying them online). 
% Even the generative COMET model \cite{bosselut2019comet} is not able to generate commonsense knowledge relevant to this reasoning task. 
For example, for queries such as \emph{``the windows are closed''}, COMET-ConceptNet generative model\footnote{\url{https://mosaickg.apps.allenai.org/comet_conceptnet}} returns knowledge about blocking the sun, and COMET-ATOMIC generative model\footnote{\url{https://mosaickg.apps.allenai.org/comet_atomic}} returns knowledge about keeping the house warm or avoiding to get hot; which while being correct, is not applicable in this context. For \emph{``my home is dry''}, both COMET-ConceptNet and COMET-ATOMIC generative models return knowledge about house cleaning or house comfort. On the other hand, the fact that 40\% of the novice users in our study were able to help CORGI reason about this example with responses such as \emph{``If I close the windows''} to CORGI's prompt, is an interesting result. This tells us that conversational interactions with humans could pave the way for commonsense reasoning and enable computers to extract just-in-time commonsense knowledge, which would likely either not exist in large knowledge bases or be irrelevant in the context of the particular reasoning task.  
% Overall, the success rate of reasoning for this statement was $40\%$ i.e. 18 out of the 45 user interactions for this statement succeeded. %\facomment{make it more precise}.
% \tmcomment{maybe delete next two sentences to save space?} We would like to add that a comprehensive analysis of the performance of these large knowledge bases requires developing models targeted at solving our proposed reasoning task and is out of the scope of our current study. We hope instead, that this discussion reveals gaps in the commonsense reasoning literature to be addressed by the reasoning community's future endeavours.
Lastly, we re-iterate that as conversational agents (such as Siri and Alexa) enter people's lives, leveraging conversational interactions for learning has become a more realistic opportunity than ever before.
% \begin{figure}[t]
    \centering
    \resizebox{0.7\textwidth}{!}{
    \begin{tabular}{p{\columnwidth}}%{p{0.3\textwidth}}
    \toprule
    Successful task\\
    \midrule
    If it's going to rain in the afternoon then remind me to bring an umbrella because I want to remain dry.\\
    \hspace{2em}\textit{How do I know if ``I remain dry''?}\\
    If I have my umbrella.\\
    \hspace{2em}\textit{How do I know if ``I have my umbrella''?}\\
    If you remind me to bring an umbrella.\\
    \hspace{2em}\textit{Okay, I will perform ``remind me to bring an umbrella'' in order to achieve ``I remain dry''.}\\
    \\
    \toprule
    Failed task\\
    \midrule
    If it's going to rain in the afternoon then remind me to bring an umbrella because I want to remain dry.\\
    \hspace{2em}\textit{How do I know if ``I remain dry''?}\\
    If I have my umbrella.\\
    \hspace{2em}\textit{How do I know if ``I have my umbrella''?}\\
    If it's in my office.\\
    \hspace{2em}\textit{How do I know if ``it's in my office''?}\\
    ...
    \\
    \bottomrule
    \end{tabular}
    }
    \caption{Sample dialogs from our user study. CORGI's responses are noted in italics.}
    \label{fig:dialog}
\end{figure}

% \amoscomment{In the previous two scenarios you stated what they were. Either add it here as well, or if there isn't enough space, just remove and let the reader go back to section 3.1.} \facomment{sure, removed the previous ones, thanks}
In order to address scenario $\mathfrak{C}.1$, the conversational prompts of CORGI 
% Another point of failure for CORGI is caused by the fact that conversational prompts of the computer 
ask for specific small pieces of knowledge that can be easily parsed into a predicate and a set of arguments. However, some users in our study tried to provide additional details, which challenged CORGI's natural language understanding. 
% that some users gave all the reasoning steps in answer to the first question, which confused the parser. 
For example, for the reasoning task \emph{``If I receive an email about water shut off then remind me about it a day before because I want to make sure I have access to water when I need it.''}, in response to the system's prompt \emph{``How do I know if `I have access to water when I need it.'?''} one user responded \emph{``If I am reminded about a water shut off I can fill bottles''}. This is a successful knowledge transfer. However, the parser expected this to be broken down into two steps. If this user responded to the prompt with \emph{``If I fill bottles''} first, CORGI would have asked \emph{``How do I know if `I fill bottles'?''} and if the user then responded \emph{``if I am reminded about a water shut off''} CORGI would have succeeded. The success from such conversational interactions are not reflected in the overall performance mainly due to the limitations of natural language understanding.%, which we are planning to address in the future.

% The effectiveness of interactive reasoning and well as the learned rule embeddings is shown in Table \ref{tab:user_study}. 
Table \ref{tab:user_study} evaluates the effectiveness of conversational interactions for proving compared to the no-feedback model. The 0\% success rate there reflects the incompleteness of \KB. The improvement in task success rate between the no-feedback case and the other rows indicates that when it is possible for users to contribute useful common-sense %Amos: commonsense is usually written as one word in this paper.
knowledge to the system, performance improves. The users contributed a total number of 96 rules to our knowledge base, 31 of which were unique rules. 
Scenario $\mathfrak{C}.2$ occurs when there is variation in the user's natural language statement and is addressed with our neuro-symbolic theorem prover. Rows 2-3 in Table \ref{tab:user_study} evaluate our theorem prover (\emph{soft unification}). 
% and replaces the Neuro-Symbolic theorem prover in Fig \ref{fig:model} with different variations.
% in parallel with soft proving.
% CORGI's performance is a combination of both successful knowledge acquisition and successful soft logical proving. But  CORGI has a serial design, such that it can only succeed when both components succeed.
% The no-feedback scenario, in which we do not engage in a conversation with the user, compared with all the other rows reflect the incompleteness of the knowledge base and the need for missing knowledge acquisition. We address it here by conversing with the users. The other 3 rows evaluate the effectiveness of the learned rule embeddings.
% Hard unification 
% (row 2) uses vanilla Prolog for proving and depends on hard unification (i.e. exact matches) for predicates. Therefore, it 
% is not capable of supporting the variations of natural language.
% HARD HERE: Soft unification improves the performance of hard unification significantly, since the latter is not capable of supporting the variations of natural language. Oracle unification (row 4) results show that 
% is the performance of our inference algorithm (Alg.\ref{alg:inference}) when oracle rule and variable embeddings are available. 
Having access to the optimal rule for unification 
% at each step of the proof 
does still better, but the task success rate is not 100\%, mainly due to the limitations of natural language understanding explained earlier.  %However, this usually did not occur. %Amos: I removed this, I think that it is confusing. You can write "However, even when using oracle unification the majority of rules cannot be proven".

% %Some of the performance loss in Table~\ref{tab:user_study}, is due to an imperfect parser. Better parsing would let us extract predicate arguments in a more robust manner. % \kmcomment{identify types? extract phrases? something specific}.
% There are several other opportunities for exploring improvements to the system. 
% Extending CORGI to handle conjunction statements would let us cover more of the if-then-because scenarios from Table \ref{tab:statement_stats}. Furthermore, the rule embeddings are not currently updated as the user interacts with the model. While a successful dialog with the system does add new facts to the knowledge base as shown in Figure \ref{fig:model} (\emph{knowledge update loop}), a neural model that could adaptively update and add embeddings for new facts and rules on the fly would support handling \emph{future} statements that are semantically similar to the added rule. 

% \kmcomment{run the experiment for no-feedback so we can include denominators} \kmcomment{run an experiment without soft-unification} \amoscomment{Table 4 is in the Appendix, don't reference it here.}
%
%
%\begin{table}[t]
%    \centering
%    \begin{tabular}{ccc}
%    Statement & Novice & Expert \\\hline
%    CORGI total & 16/88 & 8/24  \\\hline
%%If I have an early morning meeting then wake me up early because I want to be ontime. 
%1 & 1 & 1 \\
%%If there are thunderstorms in the forecast within a few hours then remind me to close the windows because I want to keep my home dry. 
%2 & 3& 2  \\
%%If I schedule an appointment that overlaps with another appointment then notify me immediately because I want to let my colleagues know of the conflict. 
%3* & 0& 0 \\
%%If I search for a gas station in the navigation app and there is a cheaper gas station that is not too much further away then ask me immediately whether I want to switch the destination to the new gas station because I want to save money. 
%4 & 0& 0 \\
%%If I search for a restaurant in the navigation app and there is a cheaper restaurant with a similar rating then notify me immediately whether I want to switch the destination to the new restaurant because I want to save money. 
%5 & 0& 0 \\
%% If I receive an email about a critical software update then notify me immediately because I want to keep my computers safe from malware. 
%6 & 0& 0 \\
%%If it's going to rain in the afternoon then remind me to bring an umbrella because I want to remain dry.
%7* & 0& 0\\
%%If I haven't been to the gym for more than 3 days then remind me to go to the gym because I want to stay fit. 
%8 & 6 & 3  \\
%%If I receive an email about water shut off then remind me about it a day before because I want to make sure I have access to water when I need it. 
%9 & 0 & 0 \\
%%If my calender is clear today, then remind me to go to gym in the afternoon, because I want to keep myself healthy. 
%10 & 6 &2 \\
%    \end{tabular}
%    \caption{Percentage of users who were able to prove each statement in the target study set. *: These statements saw fewer proof attempts due to a bug in the initial round of tests.}
%    \label{tab:user_study}
%\end{table}

\textbf{Language conditioned manipulation.} 
Significant work has been performed in  learning concepts and tasks for robots in interactive settings~\cite{gopalan2018sequence,gopalan2020simultaneously,tellex2020robots} even with the use of dialog~\cite{chai2018language,matuszek2012joint}.
Our work differs from previous works as it learns visual concepts for manipulation one-shot, and improves generalization by updating other known concepts.
Moreover, our approach can learn a concept hierarchy starting from zero known concepts, displaying the adaptability of our model under a continual learning setup.
%Our work differs from previous works as it is attempting to learn visual concepts for manipulation one-shot, while updating other known concepts to improve generalization.
%Moreover, our approach is completely differentiable and can start with zero known concepts, which is important for a continual learning setup. 
Previous work has focused on language conditioned manipulation~\citep{shridhar2021cliport, liu2021structformer, brohan2023rt1, brohan2023rt2}. \citealt{shridhar2021cliport} computes a pick and place location conditioned on linguistic and visual inputs. 
\citealt{liu2021structformer} focuses on semantic arrangement on unseen objects. 
Other works train on large scale linguistic and visual data and can perform real-life robotic task based on language instructions~\citep{ahn2022i, brohan2023rt1, brohan2023rt2}. Our work focuses on interactive teaching of tasks and concepts instead of focusing on the emergent behaviors from large models. 
% \citealt{ahn2022i} uses a pre-trained to propose plans to complete task and executes the feasible grounded plan. 
% None of these approaches discussed above focus on one shot teaching of concepts and task types to solve novel tasks in a zero-shot setting.
\citealt{daruna2019robocse} learns a representation of a knowledge graph by predicting directed relations between objects allowing a robot to predict object locations. 
To the best of the author's knowledge, our work  is the first that learns concepts and tasks one-shot to generalize to novel task scenarios on a robot, making our contributions significant compared to other related works.   


\noindent\textbf{Visual reasoning and visual concept learning.} Our work is related to visual concept learning \citep{mei2022falcon, mao2018the, yi2019neuralsymbolic, han2020visual, li2020competenceaware} and visual reasoning \citep{Mascharka_2018, DBLP:journals/corr/abs-1807-08556, DBLP:journals/corr/JohnsonHMHLZG17, DBLP:journals/corr/abs-1803-03067}. To perform the visual reasoning task, traditional methods \citep{Mascharka_2018, DBLP:journals/corr/abs-1807-08556, DBLP:journals/corr/JohnsonHMHLZG17, DBLP:journals/corr/abs-1803-03067} decompose the visual reasoning task into visual feature extraction and reasoning by parsing the queries into executable neuro-symbolic programs.  On top of that, many concept learning frameworks \citep{mei2022falcon, mao2018the, yi2019neuralsymbolic, han2020visual, li2020competenceaware} learn the representation of concepts by aligning concepts onto objects in the visual scene. 
% \ngnote{\citealt{yi2019neuralsymbolic} parses the visual scene into a structural scene representation, which makes the results of the neural network more interpretable.  \citealt{mao2018the} presents a concept learner that jointly learns a visual feature extractor, visual concept representation, and semantic parsing. \citealt{han2020visual} shows that introducing the relationships between concepts and higher-level concepts is helpful in learning the concept’s representation. \citealt{mei2022falcon} even shows that it is possible to train a module that learns a new concept with a very limited number of examples and its conceptual relationship to known concepts.} 
As far as we know, \citealt{mei2022falcon}'s FALCON is the most similar work to our work in this line of research. However, when introducing a new concept, our work continually updates the representation of all related concepts, whereas \citealt{mei2022falcon} does not, which makes it ill-suited for continual learning settings. Our work is also related to the area of few-shot learning~\citep{snell2017prototypical, tian2020rethinking, vinyals2017matching}, which learns to recognize new objects or classes from only a few examples but does not represent a concept hierarchy which is useful in robotics settings.



% \textbf{Few-shot learning.} Our work is also related to the area of few-shot learning, which learns to recognize new objects or classes from only a few examples. \citealt{vinyals2017matching} and \citealt{snell2017prototypical} use the visual features of a small number of annotated images as representation for the new class. 
% While the existing frameworks use Euclidean distance or cosine similarity to compute similarity between classes, \citealt{sung2018learning} trains a learnable module to predict the similarity between examples. Provided the relationship between the new class and known class, \citealt{wang2018zeroshot} and Kampffmeyer et al. \cite{kampffmeyer2019rethinking} learn the representation of a new class with no visual example. \citealt{tian2020rethinking} trains a transferable embedding model that can generalize to new classes. 

% \textbf{Continual learning and knowledge graphs.} Another area of work related to ours is continual learning and knowledge graphs. \citealt{daruna2019robocse} learn a representation of a knowledge graph by predicting whether a directed edge between two vertices is within the graph or not. On top of the existing framework, \citealt{daruna2021continual} shows that it is possible to continually update the knowledge graph whenever a new edge or a new vertex appears while avoiding catastrophic forgetting. 

\noindent\textbf{Scene graph.} Scene graphs are  structural representations of all objects and their relationships within an image. The scene graph representation~\cite{Chang_2023} of images is widely used in the visual domains for various tasks, such as image retrieval\cite{DBLP:journals/corr/JohnsonHMHLZG17}, image generation\citep{johnson2018image}, and question answering\citep{teney2017graphstructured}.
This form of representation has also been used in the robotics domains 
% combines geometric scene graph and symbolic scene graph as a representation of the scene 
for long-horizon manipulation~\citep{zhu2021hierarchical}.

% \vspace{-1em}
\section{Conclusions}
% \vspace{-1em}
In this paper, we introduced a benchmark task for commonsense reasoning that aims at uncovering unspoken intents that humans can easily uncover in a given statement by making presumptions supported by their common sense. In order to solve this task, we propose
CORGI (COmmon-sense ReasoninG by Instruction),  a neuro-symbolic theorem prover that performs commonsense reasoning by initiating a conversation with a user. CORGI has access to a small knowledge base of commonsense facts and completes it as she interacts with the user. We further conduct a user study that indicates the possibility of using conversational interactions with humans for evoking commonsense knowledge and verifies the effectiveness of our proposed theorem prover.
% We defined common-sense reasoning as the process of finding a chain of reasoning in a logic program given an if/then/because statement. We showed that obtaining the because statement is crucial in extracting a relevant chain of reasoning given an if/then statement. Moreover, we introduced a soft backward chaining algorithm that allows us to combat variations in natural language by learning embeddings for the facts and rules in the knowledge base. This algorithm combines symbolic AI with neural approaches allowing us to bridge a gap between symbolic AI and the recent advances in deep learning.

\section*{Acknowledgement}
Research is supported by the AFOSR Young Investigator Program under award number FA9550-23-1-0184, National Science Foundation (NSF) Award No. IIS-2237037 \& IIS-2331669, Office of Naval Research under grant number N00014-23-1-2643, and faculty research awards/gifts from Google and Meta.  Any opinions, findings, conclusions, or recommendations
 expressed in this material are those of the authors and do not necessarily reflect the views, policies, or endorsements either expressed or implied, of the sponsors.

Odit deleniti ipsam possimus est quas inventore recusandae sint ad, labore suscipit est debitis facilis earum ipsam, rem ratione itaque iure ipsa est animi illo eaque quod, a aliquam sunt aut laboriosam necessitatibus qui culpa?Debitis asperiores suscipit ducimus, deserunt voluptatum temporibus tempora magni facere praesentium id totam aspernatur illo sapiente, illo hic similique fugiat temporibus quisquam dicta iste perspiciatis nobis alias, ducimus qui provident est temporibus porro a voluptate at dicta, facilis laboriosam laborum quibusdam quasi provident neque dolore cupiditate voluptatum a.Perferendis quo doloremque amet itaque veniam saepe incidunt atque illo beatae, esse vero aperiam velit fuga ad?Amet consequuntur voluptas vero commodi, excepturi hic reiciendis earum sunt.Odit dolore nisi porro quae placeat voluptas labore excepturi autem reprehenderit sed, laudantium ullam mollitia eligendi adipisci obcaecati sit consequatur blanditiis, tempora totam doloremque commodi earum quidem reprehenderit distinctio, ad sint deserunt, expedita fuga asperiores optio quaerat?Recusandae ipsum culpa cupiditate autem ab totam officia sit consequatur similique sint, maiores corporis nihil.Saepe libero beatae voluptatibus ab a magni officia rem, aliquid doloremque odio, doloribus culpa atque cumque quo, non voluptas minima nesciunt autem laborum similique adipisci eveniet porro deleniti quia, neque eligendi architecto.Delectus magni corrupti sit non, temporibus quidem accusamus nesciunt itaque blanditiis impedit molestiae alias maiores dolore commodi, quam sapiente aliquid nisi obcaecati sint.Architecto quis in deleniti molestias ad debitis eveniet natus, voluptas saepe aliquid unde doloremque, ut fuga hic esse debitis, accusantium excepturi omnis nisi illo quasi.Magni deleniti distinctio nam porro mollitia quae unde quis commodi iusto quia, minima ipsam doloremque, perferendis nesciunt delectus sequi quis non eligendi, expedita earum itaque ratione aut tenetur ad animi, suscipit eius unde illum accusamus ratione laboriosam ipsa cum?Aliquam fugit cumque rerum est, minus quo quod dignissimos nisi velit provident autem iste.Delectus ullam officia expedita reiciendis ipsam modi maiores quidem natus, illo explicabo quibusdam nesciunt voluptatum id earum at, distinctio deleniti corrupti architecto dolores quis amet itaque soluta, culpa omnis ipsa recusandae debitis.Hic aperiam quasi ipsa recusandae voluptatem ab distinctio tempora ad, consequatur odio ut voluptas hic, enim libero sint dolores culpa?Omnis doloremque animi provident unde aut maiores tempore recusandae, enim quo reprehenderit illo nihil eaque voluptate accusantium eligendi, veniam iusto dolore officia aliquid laborum dolorum impedit consectetur?Debitis officia labore, a consequatur saepe voluptatibus temporibus qui earum numquam officiis eaque assumenda impedit.Repudiandae similique necessitatibus adipisci, in blanditiis obcaecati ex veniam explicabo?Officia blanditiis quas culpa sint, minima iure iste, ab cum exercitationem.Sed beatae sint consectetur perspiciatis dolor temporibus, quod consequatur facere, nostrum culpa deserunt facilis autem voluptates nesciunt quidem dignissimos aperiam rerum voluptatum, quidem aut eaque a culpa obcaecati saepe quam laborum veritatis accusantium exercitationem?Officia debitis quo, ipsum quod illo repellat laborum id illum pariatur, omnis in tenetur maiores at.\clearpage
\bibliography{egbib}
\clearpage

\appendix
\section{Societal Impact}
\label{sec:impact}
In this paper, we show that in high-dimensional spaces, the efficacy of distance-based out-of-distribution (OOD) detection methods can be limited by curse-of-dimensionality. To combat this problem, we propose a novel framework of subspace learning for OOD detection. OOD detection is a critically important component for a vast range of systems which include
business applications (e.g., content understanding), transportation (e.g., autonomous vehicles), and health care (e.g., unseen disease identification). Our study has positive societal impacts. We hope that it will further enhance the understanding regarding the crucial issue of how curse-of-dimensionality affects distance-based OOD detection methods. Our study does not involve any human subjects or violation of legal compliance. We do not anticipate the potentially harmful consequences of our work. Through our study and releasing our code, we hope to raise stronger research and societal attention to the problem of OOD detection.

\section{Proof of Main Theorem}
\label{app:proof}

\begin{theorem} (Recap of Th.~\ref{th:main}) We let $\mathbb{E}[p_{in}(\*z)|{\*z \in \mathcal{Z}_{in}}] - \mathbb{E}[p_{in}(\*z)|{\*z \in \mathcal{Z}_{out}}] = \Delta(m)$ as a function of the feature's dimensionality $m$. We have the following bound:
    \begin{align}
            \hat{\Delta}(m) & \geq \Delta(m) - O((\frac{k}{N})^{\frac{1}{m}} + k^{-\frac{1}{2}})
    \end{align}
\label{th:sup_main}
\end{theorem}

\begin{proof}

\begin{align*}
     &\mathbb{E}[p_{in}(\*z)|{\*z \in \mathcal{Z}_{in}} ]  - 
    \mathbb{E}[p_{in}(\*z)|{\*z \in \mathcal{Z}_{out}}]
    \\ 
    &~~=  \mathbb{E}[p_{in}(\*z) - \hat{p}_{in}(\*z)|{\*z \in \mathcal{Z}_{in}}] + 
    \mathbb{E}[\hat{p}_{in}(\*z)|{\*z \in \mathcal{Z}_{in}}] \\
    &~- \mathbb{E}[\hat{p}_{in}(\*z)|{\*z \in \mathcal{Z}_{out}}] +
    \mathbb{E}[\hat{p}_{in}(\*z) - p_{in}(\*z)|{\*z \in \mathcal{Z}_{out}}] 
    \\ &~~\leq  \mathbb{E}[|p_{in}(\*z) - \hat{p}_{in}(\*z)| | {\*z \in \mathcal{Z}_{in}}] + 
    \mathbb{E}[\hat{p}_{in}(\*z)|{\*z \in \mathcal{Z}_{in}}]\\
    &~- \mathbb{E}[\hat{p}_{in}(\*z)|{\*z \in \mathcal{Z}_{out}}] +
    \mathbb{E}[|\hat{p}_{in}(\*z) - p_{in}(\*z)| | {\*z \in \mathcal{Z}_{out}}] 
    \\
    &~~= \frac{\int_{\mathcal{Z}_{in}} |p_{in}(\*z) - \hat{p}_{in}(\*z)| p_{in}(\*z) d\*z }{\int_{\mathcal{Z}_{in}}  p_{in}(\*z) d\*z } \\
    &~+ \frac{\int_{\mathcal{Z}_{out}} |p_{in}(\*z) - \hat{p}_{in}(\*z)| p_{in}(\*z) d\*z }{\int_{\mathcal{Z}_{out}}  p_{in}(\*z) d\*z }  \\
    &~+  \mathbb{E}[\hat{p}_{in}(\*z)|{\*z \in \mathcal{Z}_{in}}] - \mathbb{E}[\hat{p}_{in}(\*z)|{\*z \in \mathcal{Z}_{out}}]\\
    \\ &\leq \frac{\int_{\mathcal{Z}_{in}} |p_{in}(\*z) - \hat{p}_{in}(\*z)| p_{in}(\*z) d\*z  + \int_{\mathcal{Z}_{out}} |p_{in}(\*z) - \hat{p}_{in}(\*z)| p_{in}(\*z) d\*z }{\min(\int_{\mathcal{Z}_{in}}  p_{in}(\*z) d\*z, \int_{\mathcal{Z}_{out}}  p_{in}(\*z) d\*z) }\\
    &~+ \mathbb{E}[\hat{p}_{in}(\*z)|{\*z \in \mathcal{Z}_{in}}] - \mathbb{E}[\hat{p}_{in}(\*z)|{\*z \in \mathcal{Z}_{out}}] \\
    \\ &~~= \frac{\int_{\mathcal{Z}} |p_{in}(\*z) - \hat{p}_{in}(\*z)| p_{in}(\*z) d\*z  }{\min(\int_{\mathcal{Z}_{in}}  p_{in}(\*z) d\*z, \int_{\mathcal{Z}_{out}}  p_{in}(\*z) d\*z) } \\
    &~+ \mathbb{E}[\hat{p}_{in}(\*z)|{\*z \in \mathcal{Z}_{in}}] - \mathbb{E}[\hat{p}_{in}(\*z)|{\*z \in \mathcal{Z}_{out}}]
    \\ &~~= \frac{\mathbb{E}[|p_{in}(\*z) - \hat{p}_{in}(\*z)|] }{\min(\int_{\mathcal{Z}_{in}}  p_{in}(\*z) d\*z, \int_{\mathcal{Z}_{out}}  p_{in}(\*z) d\*z) } \\
    &~+ \mathbb{E}[\hat{p}_{in}(\*z)|{\*z \in \mathcal{Z}_{in}}] - \mathbb{E}[\hat{p}_{in}(\*z)|{\*z \in \mathcal{Z}_{out}}].
\end{align*}


\begin{lemma}
According to Theorem 1 in ~\cite{zhao2022analysis}, the estimation error of $k$-NN distances can be bounded by:
$$
\mathbb{E}[|p_{in}(\*z) - \hat{p}_{in}(\*z)|]  \leq \mathcal{O}(\left(\frac{k}{N}\right)^{\frac{1}{m}} + k^{-\frac{1}{2}})
$$
\label{lemma:est_err}
\end{lemma}
By Lemma~\ref{lemma:est_err}, we have the final results: 
\begin{align*}
     \mathbb{E}[\hat{p}_{in}(\*z)|{\*z \in \mathcal{Z}_{in}}] - & \mathbb{E}[\hat{p}_{in}(\*z)|{\*z \in \mathcal{Z}_{out}}] \geq \mathbb{E}[p_{in}(\*z)|{\*z \in \mathcal{Z}_{in}} ] \\
     &- \mathbb{E}[p_{in}(\*z)|{\*z \in \mathcal{Z}_{out}}] - O((\frac{k}{N})^{\frac{1}{m}} + k^{-\frac{1}{2}})
\end{align*}

\end{proof}



\section{Supplementary Experiment Details}
\label{app:experimental_details}
 \subsection{Training details}
\label{app:train_details}
For main experimentation, we train DenseNet-101~\cite{huang2018densely} for 100 epochs using SGD with a momentum of 0.9, a weight decay of 0.0005, and a batch size of 64. We set the
initial learning rate as 0.1 and reduce it by a factor of 10 at 50, 75, and 90 epochs. For ResNet-50~\cite{he2016deep}, we use SGD with a momentum of 0.9, weight decay of 0.0001, batch size of 128, and train the model for 100 epochs. The learning rate is adjusted using the same schedule as used for training the DenseNet model.
The relevance ratio $r \in \{0.05, 0.15, 0.25, 0.35, 0.55, 0.75\}$ and number of neighbors $k \in \{5,10,20,50,100,200,500,1000\}$ are cross-validated as described in Appendix~\ref{app:validation}. For all experiments on CIFAR-10/100 benchmark using DenseNet~\cite{huang2018densely}, we use $r=0.25$ and $k=20$ based on our validation strategy. For experimentation using ResNet-50~\cite{he2016deep}, we set $r=0.05$ and $k=20$. We report ablation results for the effect of $r$ and $k$ in Section~\ref{sec:ablations}.

\subsection{Software and Hardware}
\label{app:hardware}
We run all experiments with Python 3.7.4 and PyTorch 1.9.0. For all experimentation, we use Nvidia RTX 2080-Ti and A6000 GPUs.


\subsection{Description of OOD baselines}
\label{app:ood_description}
In this section, we include a brief description of all the OOD baseline methods.
\subsubsection{Methods using model outputs}


\paragraph{Maximum Softmax Probability (MSP)~\cite{hendrycks2016baseline}} uses the maximum softmax probability (or the confidence score) to detect OOD examples.

\paragraph{ODIN~\cite{liang2018enhancing}} ODIN utilizes the confidence score after temperature scaling and input perturbations for OOD detection. We set temperature parameter $T=1000$ for all experiments on the CIFAR-10/100 benchmark. Perturbation Magnitude $\eta$ is chosen by validating on 1000 images randomly sampled from the ID test set. We set the perturbation magnitude $\eta = 0.0016$ for CIFAR-10 and $\eta = 0.0012$ for CIFAR-100.

\paragraph{Energy~\cite{liu2020energy}} Liu~\etal~proposed using energy score for OOD detection. The energy function maps the logits to a scalar output, which is relatively
lower for ID data. This score is hyperparameter free and does not require any tuning.


\paragraph{Generalized-ODIN \cite{hsu2020generalized}} Hsu \etal~propose a decomposed confidence model for the purpose of OOD detection, where the logits of a classifier are defined using a dividend/divisor structure. The authors propose three variants of OOD detectors, namely, DeConf-I, DeConf-E, and DeConf-C --- which uses Inner-Product, Negative Euclidean Distance, and Cosine Similarity respectively. In this study, we use the DeConf-C variant, since it is shown to be the most robust of all the variants. Finally, input samples are perturbed to improve OOD performance. Similar to ODIN~\cite{liang2018enhancing}, the perturbation magnitude $\epsilon$ is chosen by validating on 1000 images randomly sampled from the ID test set. We set perturbation magnitude $\epsilon = 0.02$ for both CIFAR-10/100 benchmarks.

\paragraph{ReAct~\cite{sun2021react}} ReAct is a post-hoc OOD detection approach based on activation truncation. The paper states that the optimal OOD performance is obtained with the ReAct+Energy setting. Hence, in this study, we use the energy score for OOD detection using ReAct. Following the original paper, we calculate the clipping threshold based on the $90$-th percentile of activations estimated on the ID data.

\paragraph{GradNorm~\cite{huang2021importance}} GradNorm employs the magnitude of gradient vectors for detecting OOD samples. The gradient is derived from the KL-divergence between the softmax output and uniform probability distribution. For GradNorm, following the original implementation, we set the temperature $T = 1$.

\paragraph{LogitNorm~\cite{wei2022mitigating}} LogitNorm proposes a simple fix to the common cross-entropy loss by enforcing a constant vector norm on the logits during training. A temperature parameter $\tau$ is used to modulate the magnitude of the logits. In this study, we set $\tau = 0.04$ for both CIFAR-10/100 datasets.

\paragraph{DICE~\cite{sun2022dice}} DICE ranks weights based on a measure of contribution, and selectively uses the most salient weights to derive the output for OOD detection. By pruning away irrelevant weights, DICE reduces the output variance for OOD
data, resulting in better separability between ID and OOD. Following the original implementation, we set the sparsity parameter $p = 0.9$ for both CIFAR-10/100 benchmarks.

\subsubsection{Methods using feature representations}

\paragraph{Mahalanobis \cite{lee2018simple}} This method models the feature space as a mixture of multivariate Gaussian distributions, and calculates Mahalanobis distance~\cite{mahalanobis1936generalized} for OOD detection. The basic idea is that the testing OOD samples should be relatively far away from the centroids or prototypes of ID classes. The minimum Mahalanobis distance to all class centroids is used for OOD detection. Previous works~\cite{sun2022knn, 2021ssd} have shown that for the Mahalanobis score, stronger performance is obtained using normalized penultimate feature vectors. Hence, we use normalized penultimate feature vectors for the Mahalanobis baseline.


\paragraph{KNN~\cite{sun2022knn}} Recently Sun \etal~proposed using non-parametric nearest-neighbor distance
for OOD detection. Unlike Mahalanobis~\cite{lee2018simple}, the non-parametric approach does not impose any distributional assumption about the underlying feature space, hence providing stronger
flexibility and generality. Following original implementation, we set the number of neighbors $k=50$ for CIFAR-10 and $k=200$ for CIFAR-100.


\subsection{Validation Strategy}
\label{app:validation}
For finding the optimal value of relevance ratio $r\in\{0.05,0.15,0.25,0.35,0.55,0.75\}$ and nearest-neighbors $k\in\{5,10,20,50,100,200,500,1000\}$, we use a validation set of Gaussian noise images. For generating these images, each pixel is sampled from $\mathcal{N} (0, 1)$. We do a grid search over all possible values of $r \times k$ and the configuration providing the best AUROC is chosen as optimal. Using DenseNet-101~\cite{huang2018densely}, we find that $r = 0.25$ and $k=20$ provides the optimal performance on both CIFAR-10/100 dataset. For ResNet-50~\cite{he2016deep}, $r=0.05$ and $k=20$ provides optimal performance on CIFAR-10/100 datset. For ImageNet-100, $r = 0.35$ and $k=200$ is optimal.



\subsection{Algorithm Pseudo Code}
\label{app:pseudo}
In this section, we provide the PyTorch code for implementing SNN. Specifically, we replace the final linear layer in a neural network with the \verb|SNN| layer to learn class-relevant subspaces.


{\small
\begin{lstlisting}[language=Python]
class SNN(nn.Linear):

    def __init__(self, in_features, out_features, bias=True, r=0.25):
        super(SNN, self).__init__(in_features, out_features, bias)
        self.r = r
        self.s = int(self.r * in_features) #subspace dimension

    def forward(self, input):
        vote = input[:, None, :] * self.weight
        if self.bias is not None:
            out = vote.topk(self.s, 2)[0].sum(2) + self.bias
        else:
            out = vote.topk(self.s, 2)[0].sum(2)
        return out

\end{lstlisting}}
\section{Supplementary Experimental Studies}
\subsection{Performance on Different Architectures}
\label{app:diff_arch}


 In Table~\ref{tab:cifar-100} and \ref{tab:hard_ood} (main paper), we have established the superiority of our proposed algorithm on DenseNet~\cite{huang2018densely}. Going beyond, in Table~\ref{tab:arch}, we show that \name remains competitive and outperforms the KNN counterpart for other common architectures such as ResNet~\cite{he2016deep}. From Table~\ref{tab:arch}, we observe that: (1) On ResNet-50, \name reduces FPR95 by \textbf{7.04}\% compared to the KNN baseline. This highlights precisely the benefits of using feature subspace for deriving the nearest neighbor distance. In contrast, \cite{sun2022knn} employed the original feature space, where irrelevant feature dimensions can impede the separability between ID and OOD data.
(2) Learning subspace during training time can preserve the ID test accuracy for both architectures.

% \newcolumntype{?}{!{\vrule width 1pt}}

\begin{table}[t]
\centering
\small 

\resizebox{0.95\linewidth}{!}{%
\begin{tabular}{lcccc}
\textbf{Method} & \textbf{Architecture} & \textbf{FPR95}  & \textbf{AUROC} & \textbf{ID Acc.}\\
& & $\downarrow$ & $\uparrow$ & $\uparrow$ \\
\toprule
KNN~\cite{sun2022knn} & DenseNet-101 & 47.21 & 85.27 & 75.14\\
\name (ours) & DenseNet-101 &  \textbf{31.25} &  \textbf{90.76} & \textbf{75.59}\\
\midrule
KNN~\cite{sun2022knn} & ResNet-50 & 53.05 & 83.61 & 74.07\\
\name (ours) & ResNet-50 &  \textbf{46.01} & \textbf{86.54} & \textbf{74.36}\\
\bottomrule
\end{tabular}}
\caption{\small Performance comparison on CIFAR-100 dataset for various network architectures. All values are averaged over multiple OOD test datasets. The best results are in \textbf{bold}.}
\label{tab:arch}

\end{table}

\subsection{Understanding relationship between $r$ and $k$}
\label{app:rel}

In Figure~\ref{fig:ablations} (main paper), we show how varying the relevance ratio ($r$) and nearest-neighbor ($k$) independently modulate the OOD detection performance. In Figure~\ref{fig:fpr} and Figure~\ref{fig:auroc}, we visualize the relationship between the hyper-parameters $r$ and $k$ through OOD detection performance. The model is DenseNet-101 and ID is CIFAR-100. We observe: (1) for all values of $r$, the OOD performance is relatively stable for a mild value of $k$. (2) $r=0.25$ provides the optimal OOD performance which is the same as obtained by our validation strategy (Appendix~\ref{app:validation}).


\begin{table}[t]
\small
\centering
\resizebox{0.99\linewidth}{!}{
\begin{tabular}{lccc}
\textbf{Method} & \textbf{FPR95}  & \textbf{AUROC} & \textbf{ID Acc.}\\
& $\downarrow$ & $\uparrow$ & $\uparrow$ \\
\toprule
Wong et al.~\cite{wong2021leveraging} & 68.06 & 79.63 & 65.89\\
Unit Dropout~\cite{srivastava2014dropout} & 62.98 & 81.40 & 72.37\\
Adaptive Dropout~\cite{ba2013adaptive} & 51.39 & 81.57 & 75.39\\
Targeted Dropout~\cite{gomez2019learning} & 69.15 & 79.80 & 73.26\\
 \name (ours) &  \textbf{31.25} & \textbf{90.76} & \textbf{75.59} \\
\bottomrule
\end{tabular}}
\caption{\small Ablation on training-time regularization methods. For OOD detection using Dropout algorithms, we calculate KNN score~\cite{sun2022knn} using feature vector $h(\*x)$. }
\label{tab:sparsification-method}
\end{table}
\begin{figure*}[h!]
\begin{subfigure}{0.5\textwidth}

  \includegraphics[width=0.90\textwidth]{images/ablations_NEW_FPR.png}
  \caption{}
  \label{fig:fpr}
\end{subfigure}%
\begin{subfigure}{0.5\textwidth}
  \includegraphics[width=0.90\textwidth]{images/ablations_NEW_AUROC.png}
  \caption{}
  \label{fig:auroc}
\end{subfigure}
\caption{\small Visualization of the relationship between the hyper-parameters $r$ and $k$ through OOD detection performance. The model is DenseNet-101 and ID is CIFAR-100. The OOD performance is averaged over six test datasets as mentioned in Section~\ref{sec:experiment}.}
\label{fig:relationship}
\end{figure*}

\subsection{Evaluation on Calibration}
\label{app:calibration}

\newcolumntype{?}{!{\vrule width 1pt}}
\begin{table*}[h!]
% \scriptsize
\centering
\resizebox{0.9\textwidth}{!}{%
\begin{tabular}{lcccc?cccc?cccc}
\multirow{2}{*}{\textbf{Dataset}} & \multicolumn{2}{c}{\textbf{NLL}} & \multicolumn{2}{c?}{\textbf{NLL (w. Subspace)}} & 
\multicolumn{2}{c}{\textbf{LS}} & \multicolumn{2}{c?}{\textbf{LS (w. Subspace)}} & \multicolumn{2}{c}{\textbf{FL}} & \multicolumn{2}{c}{\textbf{FL (w. Subspace)}} \\

% \multicolumn{2}{c}{\textbf{LS}} & \multicolumn{2}{c?}{\textbf{LS (w. Subspace)}}&
\cmidrule{2-13}
& SCE & ECE & SCE & ECE & SCE & ECE & SCE & ECE & SCE & ECE & SCE & ECE \\
\midrule
\multirow{1}{*}{CIFAR-10} & 11.5 & 5.6 & 9.6 &  4.4 & 8.8 & 3.9 & 8.2 & 3.8 & 6.9 & 2.3 & \textbf{3.9} & \textbf{0.8} \\
\midrule
\multirow{1}{*}{CIFAR-100} & 3.7 & 15.1 & 2.5 & 5.8  & 2.2 & 7.8 & 2.1 & \textbf{2.7} & 2.3 & 6.8 & \textbf{2.1} & 4.3 \\
% & & $\pmb{\downarrow}$ & $\pmb{\downarrow}$ & $\pmb{\downarrow}$ & $\pmb{\downarrow}$ & $\pmb{\downarrow}$ & $\pmb{\downarrow}$ & $\pmb{\downarrow}$ & $\pmb{\downarrow}$ \\  
% \multirow{1}{*}{CIFAR-10} & 11.5 & 5.6 & \cellcolor{COLOR_ZS} 9.6 & \cellcolor{COLOR_ZS} 4.4 & 8.8 & 3.9 &  \cellcolor{COLOR_ZS} 6.5 & \cellcolor{COLOR_ZS} 1.7 & 6.9 & 2.3 & \cellcolor{COLOR_ZS}\textbf{3.9} & \cellcolor{COLOR_ZS}\textbf{0.8} \\
% & ResNet-34 & 10.2 & 4.9 & \cellcolor{COLOR_ZS}9.6 & \cellcolor{COLOR_ZS}4.5 & 7.8 & 4.0 & \cellcolor{COLOR_ZS}6.3 & \cellcolor{COLOR_ZS}2.7 & 5.4 & 2.1 & \cellcolor{COLOR_ZS}\textbf{3.6} & \cellcolor{COLOR_ZS}\textbf{0.5} \\
% \midrule
% \multirow{1}{*}{CIFAR-100} & 3.7 & 15.1 & \cellcolor{COLOR_ZS}2.5 & \cellcolor{COLOR_ZS}5.8 & 2.2 & 7.8 & \cellcolor{COLOR_ZS}\textbf{2.1} & \cellcolor{COLOR_ZS}\textbf{1.9} & 2.3 & 6.8 & \cellcolor{COLOR_ZS}2.1 & \cellcolor{COLOR_ZS}4.3 \\
 % & ResNet-34 & 3.5 & 16.0 & \cellcolor{COLOR_ZS}3.5 & \cellcolor{COLOR_ZS}15.0 & 2.4 & 4.5 &\cellcolor{COLOR_ZS} \textbf{2.1} &\cellcolor{COLOR_ZS} \textbf{2.4} & 2.2 & 5.7 & \cellcolor{COLOR_ZS}2.1 & \cellcolor{COLOR_ZS}4.7 \\
%  \midrule
%  Tiny-ImageNet & ResNet-34 & 2.2 & 15.7 & \cellcolor{COLOR_ZS}1.9 & \cellcolor{COLOR_ZS}13.5 & 1.4 & 5.96 &\cellcolor{COLOR_ZS} 1.5 & \cellcolor{COLOR_ZS}3.8 & 1.4 & 2.2 & \cellcolor{COLOR_ZS}\textbf{1.3} & \cellcolor{COLOR_ZS}\textbf{1.2} \\ 
 
 
 \bottomrule

\end{tabular}%
}
\caption{\small \textbf{Calibration Results.} Comparison of calibration performance when using subspace learning with commonly used loss functions (NLL/LS/FL). The model is ResNet-50. Best performing results are marked in \textbf{bold}.}
\label{tab:calibration_1}
\end{table*}



\noindent Now we move beyond OOD detection tasks and systematically investigate the calibration performance on ID data itself. With our subspace learning, the model learns the feature subspace for each class. We hypothesize that learning the feature subspace helps alleviate the problem of over-confident predictions, thereby improving model calibration. Following the literature, we evaluate calibration performance based on two common metrics: {Expected Calibration Error (ECE)}~\cite{naeini2015obtaining} and  {Static Calibration Error (SCE)}~\cite{nixon2019measuring}. 

\paragraph{Description of calibration baselines.}

Before comparing the calibration performance, we first provide a brief description of loss functions for model calibration, along with hyperparameters in training: (1)~\textbf{Label Smoothing~\cite{muller2019does}.} In Label smoothing (LS), instead of using a one-hot encoded target $y$, a soft target vector $\*q$ is defined for each sample. Specifically, $q_i = \frac{\alpha}{C-1}$ if $i \neq y$, else $q_i = 1-\alpha, ~~~\forall i \in \{1,2,...,C\}$. Here, $\alpha$ is a hyperparameter. In this study, we set $\alpha = 0.05$. (2)~\textbf{Focal Loss~\cite{lin2017focal, mukhoti2020calibrating}.} Given input $\*x$, let $\hat{p}_c = \mathbb{P}(\hat{y}=c|\*x)$ be the output softmax probability of $\*x$ belonging to class $c$. The focal loss is defined as -$(1 - \hat{p}_y)^{\gamma}\text{log}(\hat{p}_y)$, where $y$ is the ground truth label and $\gamma$ is a user-defined hyperparameter. Following the original implementation, we set $\gamma=3$ for all experiments. 


\paragraph{Learning subspace improves calibration.} 
In Table~\ref{tab:calibration_1}, we observe that our subspace-regularized training algorithm improves calibration performance. In particular, we consider three losses that are commonly studied for calibration: (1) Cross Entropy Loss (NLL), (2) Label Smoothing~\cite{muller2019does}, and (3) Focal Loss (FL)~\cite{lin2017focal}.  We train the model with each of these losses and compare calibration performance with and without subspace regularization. For each dataset, we split the train set into two mutually exclusive sets: (1) $90\%$ of the train samples are used for training the model, and (2) the remaining $10\%$ of samples are used for validation. We observe from Table~\ref{tab:calibration_1} that \name improves the calibration performance for all loss functions.


\subsection{Detailed Results on All OOD Datasets}
\label{app:results}

In Table~\ref{tab:ablation_complete_c10} and Table~\ref{tab:ablation_complete_c100}, we report detailed results on six OOD test datasets when ID is CIFAR-10/100 respectively. The architecture used for all methods (including baselines) is DenseNet-101~\cite{huang2018densely}.





\subsection{Additional Discussion }
\label{app:add_discuss}
\paragraph{Relations to Wong \etal~\cite{wong2021leveraging}.} Wong \etal~\cite{wong2021leveraging} proposed an elastic net formulation to enforce sparsity for model interpretability. Hence, their motivation is fundamentally different from the problem we are trying to solve. Specifically, we learn a feature subspace for better ID-OOD separability, whereas Wong \etal improve the debuggability of neural nets. Given penultimate feature representations $h(\*x)$, \cite{wong2021leveraging} learns a sparse linear model $h(\*x)^{\top}\*w + w_0$ using the following optimization:
\begin{equation*}
\small  \min_{\*w}  \cfrac{1}{2N}||h(\*x)^{\top}\*w + w_0 -  y||^2_{2} - \lambda \left ( \cfrac{(1-\alpha)}{2}||\*w||_2^2 + \alpha||\*w||_1 \right),
\end{equation*}

where $\lambda$ and $\alpha$ are hyperparameters. In Table~\ref{tab:sparsification-method}, we compare the OOD performance between \name and Wong \etal. We make two concrete observations: (1) \name clearly outperforms \cite{wong2021leveraging} in terms of OOD detection performance. This result validates the effectiveness of our proposed subspace learning algorithm. (2) Model trained using ~\cite{wong2021leveraging} leads to suboptimal ID accuracy (65.89\%). In contrast, \name maintains the ID accuracy (75.59\%) along with improved OOD performance.




\begin{table*}[h!]

\centering

\resizebox{\textwidth}{!}{%
\begin{tabular}{l*{16}c}
 \multirow{2}{1.5cm}{\textbf{Methods}} & \multicolumn{12}{c}{\textbf{OOD Datasets}} & \multicolumn{2}{c}{\multirow{2}{*}{\centering\textbf{Average}}} & \multirow{2}{*}{\centering\textbf{ID Acc.}}\\

\cmidrule{2-13}

& \multicolumn{2}{c}{\textbf{SVHN}} & \multicolumn{2}{c}{\textbf{LSUN-c}} &
\multicolumn{2}{c}{\textbf{LSUN-r}} &
\multicolumn{2}{c}{\textbf{iSUN}} & \multicolumn{2}{c}{\textbf{Textures}} & \multicolumn{2}{c}{\textbf{Places365}} & &  \\

& FPR95 & AUROC & FPR95 & AUROC & FPR95 & AUROC &  FPR95 & AUROC & FPR95 & AUROC & FPR95 & AUROC & FPR95 & AUROC & \\
 & $\pmb{\downarrow}$ & $\pmb{\uparrow}$  & $\pmb{\downarrow}$ & $\pmb{\uparrow}$ & $\pmb{\downarrow}$ &  $\pmb{\uparrow}$ & $\pmb{\downarrow}$ & $\pmb{\uparrow}$ & $\pmb{\downarrow}$ & $\pmb{\uparrow}$ & $\pmb{\downarrow}$ & $\pmb{\uparrow}$ &
 $\pmb{\downarrow}$ & $\pmb{\uparrow}$ & $\pmb{\uparrow}$  \\
\midrule
\emph{Methods using Model Outputs}\\

MSP~\cite{hendrycks2016baseline} & 43.49 & 94.01 & 44.42 & 94.13 & 47.39 & 93.48 & 47.80 & 93.48 & 66.03 & 87.26 & 63.52 & 88.37 & 52.11 & 91.79 & 94.03\\
ODIN~\cite{liang2018enhancing} & 34.15 & 94.73 & 8.39 & 98.42 & 8.93 & 98.20 & 9.33 & 98.17 & 56.37 & 85.83 & 41.76 & 91.50 & 26.47 & 94.48 & 94.03 \\ 
GODIN~\cite{hsu2020generalized} & 3.78 & 99.17 & 9.47 & 97.83 & 5.40 & 98.67 & 6.73 & 98.54 & 23.90 & 94.32 & 55.24 & 86.52 & 17.42 & 95.84 & 94.22  \\
Energy Score~\cite{liu2020energy} & 33.07 & 95.01 & 8.10 & 98.43 & 14.04 & 97.47 & 14.58 & 97.42 & 59.61 & 85.42 & 41.98 & 91.48 & 28.40 & 94.22 & 94.03 \\

ReAct~\cite{sun2021react} & 43.67 & 94.27 & 22.37 & 96.22 & 16.68 & 97.09 & 19.81 & 96.74 & 53.44 & 89.63 & 43.23 & 91.88 & 33.12 & 94.32 & 93.27\\

GradNorm~\cite{huang2021importance} & 25.07 & 93.91 & 0.41 & 99.85 & 9.51 & 98.08 & 10.41 & 97.97 & 44.72 & 83.23 & 58.65 & 82.45 & 24.79 & 92.58 & 94.03 \\
LogitNorm~\cite{wei2022mitigating} & 14.31 & 97.63 & 2.61 & 99.37 & 17.16 & 97.18 & 17.14 & 97.16 & 39.66 & 91.17 & 47.30 & 90.40 & 19.61 & 95.51 & 93.94\\ 
DICE~\cite{sun2022dice} & 27.84 & 94.98 & 0.38 & 99.90 & 4.43 & 99.03 & 5.14 & 98.97 & 45.85 & 86.97 & 45.41 & 90.03 & 20.83 & 95.24 & 94.38 \\
\midrule
\emph{Methods using feature representations}\\
Mahalanobis~\cite{lee2018simple} & 17.85 & 94.66 & 68.49 & 76.21 & 30.06 & 92.16 & 29.86 & 91.15 & 30.73 & 88.83 & 90.34 & 52.37  & 44.55 & 82.56 & 94.03 \\ 

KNN~\cite{sun2022knn} & 3.87 & 99.31 & 10.81 & 98.13 & 12.58 & 97.75 & 12.24 & 97.87 & 21.61 & 96.07 & 49.36 & 89.54 & 18.50 & 96.36 & 94.03\\
\midrule
 \name (Ours) & 2.67 & 99.52 & 5.22 & 99.14 & 9.70 & 98.35 & 8.94 & 98.44 & 19.84 & 96.51 & 43.62 & 90.98 & \textbf{15.00} & \textbf{97.16} & 94.15 \\
\bottomrule

\end{tabular}}
\caption{table}{\small Detailed results on six OOD benchmark datasets: \texttt{Textures}~\cite{cimpoi2014describing}, \texttt{SVHN}~\cite{svhn}, \texttt{LSUN-Crop}~\cite{yu2015lsun}, \texttt{LSUN-Resize}~\cite{yu2015lsun}, \texttt{iSUN}~\cite{xu2015turkergaze}, and \texttt{Places365}~\cite{zhou2017places}. Model is DenseNet and ID is CIFAR-10.}
\label{tab:ablation_complete_c10}
\end{table*}
\begin{table*}[h!]
\centering
\resizebox{\textwidth}{!}{%
\begin{tabular}{l*{16}c}
 \multirow{2}{1.5cm}{\textbf{Methods}} & \multicolumn{12}{c}{\textbf{OOD Datasets}} & \multicolumn{2}{c}{\multirow{2}{*}{\centering\textbf{Average}}} & \multirow{2}{*}{\centering\textbf{ID Acc.}}\\

\cmidrule{2-13}

& \multicolumn{2}{c}{\textbf{SVHN}} & \multicolumn{2}{c}{\textbf{LSUN-c}} &
\multicolumn{2}{c}{\textbf{LSUN-r}} &
\multicolumn{2}{c}{\textbf{iSUN}} & \multicolumn{2}{c}{\textbf{Textures}} & \multicolumn{2}{c}{\textbf{Places365}} & &  \\

& FPR95 & AUROC & FPR95 & AUROC & FPR95 & AUROC &  FPR95 & AUROC & FPR95 & AUROC & FPR95 & AUROC & FPR95 & AUROC & \\
 & $\pmb{\downarrow}$ & $\pmb{\uparrow}$  & $\pmb{\downarrow}$ & $\pmb{\uparrow}$ & $\pmb{\downarrow}$ &  $\pmb{\uparrow}$ & $\pmb{\downarrow}$ & $\pmb{\uparrow}$ & $\pmb{\downarrow}$ & $\pmb{\uparrow}$ & $\pmb{\downarrow}$ & $\pmb{\uparrow}$ &
 $\pmb{\downarrow}$ & $\pmb{\uparrow}$ & $\pmb{\uparrow}$  \\
\midrule
\emph{Methods using Model Outputs}\\

MSP~\cite{hendrycks2016baseline} & 83.67 & 75.46 & 61.00 & 86.00 & 74.73 & 76.13 & 76.10 & 75.48 & 86.17 & 71.65 & 83.31 & 73.97 & 77.59 & 76.47 & 75.14 \\
ODIN~\cite{liang2018enhancing} & 91.51 & 76.16 & 15.16 & 97.41 & 31.92 & 93.93 & 36.75 & 92.89 & 83.92 & 72.70 & 79.12 & 77.13 & 56.39 & 86.02 & 75.14 \\
GODIN~\cite{hsu2020generalized} & 15.25 & 97.15 & 30.65 & 93.66 & 42.75 & 93.02 & 38.50 & 93.53 & 47.98 & 89.62 & 89.37 & 70.23 & 44.08 & 89.05 & 74.22 \\
Energy Score~\cite{liu2020energy} & 87.94 & 78.07 & 13.81 & 97.61 & 35.82 & 92.98 & 40.75 & 91.78 & 84.38 & 71.81 & 79.91 & 76.71 & 57.07 & 84.83 & 75.14\\
ReAct~\cite{sun2021react} & 93.65 & 74.20 & 52.07 & 87.63 & 63.14 & 88.13 & 69.96 & 85.56 & 87.07 & 72.56 & 87.90 & 67.66 & 75.06 & 79.51 & 66.56 \\ 
GradNorm~\cite{huang2021importance} & 60.62 & 87.76 & 0.65 & 99.78 & 82.20 & 75.48 & 78.68 & 78.14 & 65.73 & 71.99 & 90.41 & 65.65 & 63.05 & 79.80 & 75.14 \\
LogitNorm~\cite{wei2022mitigating} & 57.65 & 89.32 & 12.37 & 97.76 & 70.53 & 84.94 & 71.27 & 84.54 & 74.91 & 75.20 & 78.00  & 78.42 & 61.10 & 84.72 & 75.42 \\
DICE~\cite{sun2022dice} & 59.80 & 88.29 & 0.91 & 99.74 & 51.62 & 89.32  & 49.48 & 89.51 & 61.42 & 77.12 & 80.27 & 77.40 & 49.72 & 87.23 & 68.65 \\
\midrule
\emph{Methods using feature representations}\\
Mahalanobis~\cite{lee2018simple} & 70.19 & 80.49 & 93.98 & 66.81 & 24.83 & 94.97 & 26.20 & 94.19 & 31.76 & 90.01 & 94.60 & 55.17 & 56.93 & 80.27 & 75.14\\
KNN~\cite{sun2022knn} & 23.54 & 95.34 & 66.59 & 77.98 & 37.83 & 92.88 & 32.83 & 93.63 & 28.58 & 92.36 & 93.92 & 59.42 & 47.21 & 85.27  & 75.14 \\
\midrule
 \name (Ours) & 11.56 & 97.68 & 24.43 & 94.44 & 19.19 & 96.17 & 21.21 & 95.46 & 22.93 & 94.75 & 88.17 & 66.58 & \textbf{31.25} & \textbf{90.85} & 75.59 \\
\bottomrule

\end{tabular}}
\caption{table}{\small Detailed results on six OOD benchmark datasets: \texttt{Textures}~\cite{cimpoi2014describing}, \texttt{SVHN}~\cite{svhn}, \texttt{LSUN-Crop}~\cite{yu2015lsun}, \texttt{LSUN-Resize}~\cite{yu2015lsun}, \texttt{iSUN}~\cite{xu2015turkergaze}, and \texttt{Places365}~\cite{zhou2017places}. Model is DenseNet and ID is CIFAR-100.}
\label{tab:ablation_complete_c100}
\end{table*}
\end{document}