%File: formatting-instructions-latex-2024.tex
%release 2024.0
\documentclass[letterpaper]{article} % DO NOT CHANGE THIS
\usepackage{aaai24}  % DO NOT CHANGE THIS
\usepackage{times}  % DO NOT CHANGE THIS
\usepackage{helvet}  % DO NOT CHANGE THIS
\usepackage{courier}  % DO NOT CHANGE THIS
\usepackage[hyphens]{url}  % DO NOT CHANGE THIS
\usepackage{graphicx} % DO NOT CHANGE THIS
\urlstyle{rm} % DO NOT CHANGE THIS
\def\UrlFont{\rm}  % DO NOT CHANGE THIS
\usepackage{natbib}  % DO NOT CHANGE THIS AND DO NOT ADD ANY OPTIONS TO IT
\usepackage{caption} % DO NOT CHANGE THIS AND DO NOT ADD ANY OPTIONS TO IT
\frenchspacing  % DO NOT CHANGE THIS
\setlength{\pdfpagewidth}{8.5in}  % DO NOT CHANGE THIS
\setlength{\pdfpageheight}{11in}  % DO NOT CHANGE THIS
%
% These are recommended to typeset algorithms but not required. See the subsubsection on algorithms. Remove them if you don't have algorithms in your paper.
\usepackage{algorithm}
\usepackage{algorithmic}

% additional packages
\usepackage{makecell}
\usepackage{booktabs}
\usepackage[table,xcdraw]{xcolor}
\usepackage{multirow}
\usepackage{amssymb}
\usepackage{amsmath}

\def\eg{e.g.} \def\Eg{E.g.}
\def\ie{i.e.} \def\Ie{I.e.}
\def\cf{c.f.} \def\Cf{C.f.}

%
% These are are recommended to typeset listings but not required. See the subsubsection on listing. Remove this block if you don't have listings in your paper.
\usepackage{newfloat}
\usepackage{listings}
\DeclareCaptionStyle{ruled}{labelfont=normalfont,labelsep=colon,strut=off} % DO NOT CHANGE THIS
\lstset{%
	basicstyle={\footnotesize\ttfamily},% footnotesize acceptable for monospace
	numbers=left,numberstyle=\footnotesize,xleftmargin=2em,% show line numbers, remove this entire line if you don't want the numbers.
	aboveskip=0pt,belowskip=0pt,%
	showstringspaces=false,tabsize=2,breaklines=true}
\floatstyle{ruled}
\newfloat{listing}{tb}{lst}{}
\floatname{listing}{Listing}
%
% Keep the \pdfinfo as shown here. There's no need
% for you to add the /Title and /Author tags.

% DISALLOWED PACKAGES
% \usepackage{authblk} -- This package is specifically forbidden
% \usepackage{balance} -- This package is specifically forbidden
% \usepackage{color (if used in text)
% \usepackage{CJK} -- This package is specifically forbidden
% \usepackage{float} -- This package is specifically forbidden
% \usepackage{flushend} -- This package is specifically forbidden
% \usepackage{fontenc} -- This package is specifically forbidden
% \usepackage{fullpage} -- This package is specifically forbidden
% \usepackage{geometry} -- This package is specifically forbidden
% \usepackage{grffile} -- This package is specifically forbidden
% \usepackage{hyperref} -- This package is specifically forbidden
% \usepackage{navigator} -- This package is specifically forbidden
% (or any other package that embeds links such as navigator or hyperref)
% \indentfirst} -- This package is specifically forbidden
% \layout} -- This package is specifically forbidden
% \multicol} -- This package is specifically forbidden
% \nameref} -- This package is specifically forbidden
% \usepackage{savetrees} -- This package is specifically forbidden
% \usepackage{setspace} -- This package is specifically forbidden
% \usepackage{stfloats} -- This package is specifically forbidden
% \usepackage{tabu} -- This package is specifically forbidden
% \usepackage{titlesec} -- This package is specifically forbidden
% \usepackage{tocbibind} -- This package is specifically forbidden
% \usepackage{ulem} -- This package is specifically forbidden
% \usepackage{wrapfig} -- This package is specifically forbidden
% DISALLOWED COMMANDS
% \nocopyright -- Your paper will not be published if you use this command
% \addtolength -- This command may not be used
% \balance -- This command may not be used
% \baselinestretch -- Your paper will not be published if you use this command
% \clearpage -- No page breaks of any kind may be used for the final version of your paper
% \columnsep -- This command may not be used
% % \newpage -- No page breaks of any kind may be used for the final version of your paper
% \pagebreak -- No page breaks of any kind may be used for the final version of your paperr
% \pagestyle -- This command may not be used
% \tiny -- This is not an acceptable font size.
% \vspace{- -- No negative value may be used in proximity of a caption, figure, table, section, subsection, subsubsection, or reference
% \vskip{- -- No negative value may be used to alter spacing above or below a caption, figure, table, section, subsection, subsubsection, or reference

\setcounter{secnumdepth}{0} %May be changed to 1 or 2 if section numbers are desired.

% The file aaai24.sty is the style file for AAAI Press
% proceedings, working notes, and technical reports.
%

% Title

% Your title must be in mixed case, not sentence case.
% That means all verbs (including short verbs like be, is, using,and go),
% nouns, adverbs, adjectives should be capitalized, including both words in hyphenated terms, while
% articles, conjunctions, and prepositions are lower case unless they
% directly follow a colon or long dash
\title{Weakly Supervised Semantic Segmentation for Driving Scenes}
\author{
    %Authors
    % All authors must be in the same font size and format.
    Dongseob Kim\equalcontrib\textsuperscript{\rm 1},
    Seungho Lee\equalcontrib\textsuperscript{\rm 1},
    Junsuk Choe\textsuperscript{\rm 2},
    Hyunjung Shim\thanks{Hyunjung Shim is a corresponding author.}\textsuperscript{\rm 3}
}
\affiliations{
    %Afiliations
    \textsuperscript{\rm 1} Yonsei University, South Korea\\
    \textsuperscript{\rm 2} Sogang University, South Korea\\
    \textsuperscript{\rm 3} Korea Advanced Institute of Science \& Technology, South Korea\\
    \{kou.k, seungholee\}@yonsei.ac.kr, jschoe@sogang.ac.kr, kateshim@kaist.ac.kr
}

\begin{document}

\maketitle

\begin{abstract}
State-of-the-art techniques in weakly-supervised semantic segmentation (WSSS) using image-level labels exhibit severe performance degradation on driving scene datasets such as Cityscapes. To address this challenge, we develop a new WSSS framework tailored to driving scene datasets. Based on extensive analysis of dataset characteristics, we employ Contrastive Language-Image Pre-training (CLIP) as our baseline to obtain pseudo-masks. However, CLIP introduces two key challenges: (1) pseudo-masks from CLIP lack in representing small object classes, and (2) these masks contain notable noise. We propose solutions for each issue as follows. (1) We devise Global-Local View Training that seamlessly incorporates small-scale patches during model training, thereby enhancing the model's capability to handle small-sized yet critical objects in driving scenes (\eg, \textit{traffic light}). (2) We introduce Consistency-Aware Region Balancing (CARB), a novel technique that discerns reliable and noisy regions through evaluating the consistency between CLIP masks and segmentation predictions. It prioritizes reliable pixels over noisy pixels via adaptive loss weighting. Notably, the proposed method achieves 51.8\% mIoU on the Cityscapes test dataset, showcasing its potential as a strong WSSS baseline on driving scene datasets. Experimental results on CamVid and WildDash2 demonstrate the effectiveness of our method across diverse datasets, even with small-scale datasets or visually challenging conditions. The code is available at https://github.com/k0u-id/CARB.
\end{abstract}

\section{Introduction}

Recent advancements in weakly supervised semantic segmentation (WSSS) using image-level labels have demonstrated impressive results, achieving performance levels of over 90\% compared to full supervised models on the PASCAL VOC dataset~\cite{lee2022threshold,yoon2022adversarial}. Given this success, it is crucial to transfer the WSSS framework to driving scenes, which are a significant scenario in semantic segmentation. Obtaining pixel-level labels in driving scenes is prohibitively expensive, making label-efficient training methods imperative in this context. For instance, Cityscapes required 1.5 hours per image~\cite{cordts2016cityscapes}, while PASCAL VOC required 239.7 seconds per image~\cite{bearman2016s}.

However, when applied to driving scene datasets like Cityscapes, WSSS models exhibit significant performance degradation. \citeauthor{akiva2023single} attributed this issue to the specific characteristics of the dataset, such as small object size, a high number of objects in each image, and limited diversity in object appearance~\cite{akiva2023single}. However, they only reported this tendency implicitly. In our study, we explicitly compare the driving scene datasets to the existing benchmark datasets (\ie, PASCAL VOC and MS COCO). As a result, we find that the driving scenes datasets lack negative samples and exhibit a remarkably high level of co-occurrence among classes. This poses a challenge in identifying individual objects through image classification, which hinders the effectiveness of common WSSS baselines, such as class activation mapping (CAM).

Recently, Contrastive Language Image Pre-training (CLIP), a model trained on a massive set of 400 million image-text pairs, has remarkably performed in open vocabulary classification. Using the open vocabulary classification ability of CLIP, we can avoid the characteristic of the dataset degrading the classifier's performance. As a result, as opposed to CAM, the seed mask generated by CLIP better distinguishes the object regions on the driving dataset like Cityscapes. Despite the potential, it often fails to identify small objects and produces noisy masks (Fig.~\ref{fig:mask_resizencrop}~(a)).

In this paper, we present a novel WSSS framework for driving scene datasets, to address the above two challenges inherent in CLIP. Considering CLIP as a baseline mask generator, we propose (1) global-local view training to handle small-sized objects and (2) \textit{Consistency-Aware Region Balancing (CARB)} to mitigate the negative effects of noisy pseudo-masks. Firstly, we found a unique property of CLIP: it offers considerably different pseudo-masks across input scales. Based on this observation, we use both a local view (\ie, a small-sized patch) and a global view (\ie, an original-sized image) during model training for accurately detecting small but critical objects in driving scenes (\eg, \textit{traffic light}).

\begin{figure*}[t]
\begin{center}
\includegraphics[width=17cm]{figures/fig_dataset.pdf}
\end{center}
\caption{Dataset statistics for Cityscapes, CamVid, MS COCO, and PASCAL VOC. (a) Counting the number of images given by the number of classes in a single image. (b) Histogram of co-occurrence ratio between classes. (c) The number of positive and negative images for each class.}
\label{fig:dataset}
\end{figure*}

Next, we propose CARB, which suppresses the erroneous region of pseudo-masks to train the segmentation model. Specifically, we divide the noisy pseudo-mask into consistent and inconsistent regions according to prediction consistency between the segmentation model and CLIP. The inconsistent region contains more false predictions than the consistent region, resulting in a higher loss. This discrepancy in the magnitude of loss values leads to a negative impact on the overall training process. To mitigate this, we propose a strategy to balance the losses from both regions, thereby suppressing the high loss of the inconsistent region.

In summary, we examine the distinct characteristics of driving scenes over the commonly evaluated datasets and highlight the issue of ineffective CAM-based approaches in these scenes. We introduce a new WSSS framework utilizing pseudo-masks generated from CLIP, suggesting global-local view training to handle small-sized objects and CARB to mitigate the negative effects of noisy pseudo-masks.

We demonstrate that the proposed method achieved 51.8\% mIoU on the Cityscapes dataset, showcasing the potential as a strong WSSS baseline for driving scenes. The effectiveness of the proposed method was confirmed on CamVid representing a small-scale dataset, and on WildDash2 containing more visually challenging scenes (\eg, diverse weather and lighting conditions). Owing to its advantage in performance and simple training, our method can serve as a valuable baseline for future research to address the challenges of WSSS in the driving scenes.

\section{Related Work}
\subsubsection{Earlier Works in WSSS.} Most WSSS techniques using image-level labels utilize CAM~\cite{zhou2016learning}. Due to its sparse coverage, recent studies have focused on expanding discriminative regions~\cite{jiang2019integral, wei2018revisiting, choe2019attention}. In terms of using global-local view, L2G~\cite{jiang2022l2g} strengthened classifier learning by using local attention. This method was also used to widen the discriminative region. Recently, some approaches have attempted to solve co-occurrence problem by incorporating additional information~\cite{lee2021railroad, lee2022weakly, Xie_2022_CVPR}. However, most of existing methods were only evaluated on PASCAL VOC~\cite{everingham2015pascal} or MS COCO~\cite{lin2014microsoft}. \citeauthor{akiva2023single}.~\citeyear{akiva2023single} conducted evaluations on more complex datasets like Cityscapes~\cite{cordts2016cityscapes} and ADE20k~\cite{zhou2019semantic}, but only revealed the performance limitations of existing WSSS studies. \citeauthor{wang2020deep}.~\citeyear{wang2020deep} introduced a clustering-based approach in driving scene datasets, while they only achieved a marginal improvement. Unlike most WSSS studies, we analyze distinct characteristics of the driving scene datasets compared to existing benchmark datasets and suggest a new direction for WSSS in driving scene scenarios.

\subsubsection{CLIP-based Segmentation.} CLIP~\cite{radford2021learning} is a framework trained on a large amount of image-text pairs. Several attempts have been made to utilize the characteristics of the multimodal embedding space in the field of segmentation~\cite{ding2022decoupling,Wang_2022_CVPR}. In WSSS, CLIMS~\cite{Xie_2022_CVPR} employed the embedding spaces by optimizing the mask based on the similarity between masked image and text embedding. CLIP-ES~\cite{Lin_2023_CVPR} generates seed masks in Grad-CAM manner~\cite{selvaraju2017grad}. Then, it refines masks with class-wise attention-based affinity of the CLIP image encoder.

Existing studies~\cite{li2022languagedriven, xu2021} have also shown a significant improvement in zero-shot and few-shot segmentation by leveraging CLIP's zero-shot ability. Recently, MaskCLIP~\cite{zhou2022extract} has been proposed to create dense masks from CLIP with category information at the dataset level rather than the image level. We employ MaskCLIP to extract dense labels from images and further propose a training strategy for handling the noise present in its pseudo-masks.

\begin{figure*}[t!]
\centering
\includegraphics[width=17cm]{figures/fig_framework.pdf}
\caption{Overall framework of proposed method. (Global-local View Training) CLIP gives different pseudo masks for cropping and resizing. (CARB) The pseudo-mask is divided into the consistent / inconsistent regions and the high loss of inconsistent regions is suppressed via adaptive region balancing.}
\label{fig:framework}
\end{figure*}

\subsubsection{Uncertainty Estimation.} Estimating the uncertainty~\cite{DBLP:conf/nips/KendallG17} has been discussed in deep learning since deep neural networks learn to approximate probabilistic models. Focusing on semantic segmentation, the \citeauthor{feng2022dmt} utilizes an ensemble of models that are initialized differently to separate the uncertainty region. In a similar vein, several methods~\cite{oh2021background,DBLP:conf/aaai/ZhangXWSH20} utilized a combination of confidence thresholding and consistency between the CRF-refined mask and the original mask to define a reliable region. ST++~\cite{yang2022st++} identifies reliable images by utilizing the results of previous checkpoints. Recently, several methods suggest pixel-level entropy to measure the pixel-level uncertainty~\cite{NEURIPS2020_f73b76ce,Huynh:CVPR22,li2022uncertainty, wang2022semi}.

\section{Statistics of Datasets}
In this section, to identify the cause of the poor performance of existing WSSS methods on driving scenes, we compare the characteristics of two types of datasets: standard benchmark datasets (\eg, PASCAL VOC and MS COCO) and driving scene datasets (\eg, Cityscapes and CamVid). Specifically, we investigate the histograms of 1) the number of classes per image, 2) the co-occurrence ratio between classes, and 3) the number of positive/negative samples per class (Fig.~\ref{fig:dataset}).

The distinct difference between the two types of datasets is the number of classes in a single image. Although existing benchmark datasets have only one or two classes in most images, driving scene datasets typically contain eight or more classes in a single image, as in Fig.~\ref{fig:dataset}~(a). Next, we calculate the frequency ratio of co-occurrence between every pair of classes and plot a histogram of those ratios in Fig.~\ref{fig:dataset}~(b). Even worse, these classes in driving scenes often appear together, causing contextual bias. It is clearly different from PASCAL VOC and MS COCO.

Another critical point is the scarcity of negative samples in driving scene datasets. Negative samples are important learning signals for training the image classifier. As shown in Fig.~\ref{fig:dataset}~(c), existing datasets have a sufficient number of negative samples, but some classes in driving scene datasets have extremely few or zero negative samples. Most seriously, \textit{road} and \textit{car} in CamVid always appear in all training images and cannot be distinguished using only image-level labels. Understanding these characteristics is essential for developing productive approaches for WSSS using image-level labels in driving scene applications.

\begin{figure}[t]
\begin{center}
\includegraphics[width=8cm]{figures/fig_mask_resizencrop.pdf}
\end{center}
\caption{Pseudo-masks after resizing and cropping. (a) The original CLIP mask. (b) CLIP mask with resize ratio 2. (c) The concatenation of quarter-size cropped CLIP masks (d) The mask applying both operations. For visual clarity, we modify color palette of motorcycle to \textit{cyan} in this figure.}
\label{fig:mask_resizencrop}
\end{figure}

\begin{figure}[t]
\begin{center}
\includegraphics[width=8cm]{figures/fig_mask_character.pdf}
\end{center}
\caption{The characteristics of two different masks. (a) The mask from CLIP contains small and blob-like noisy regions. (b) The output mask from the segmentation network is more systematic. We identify reliable regions (c) based on prediction consistency between (a) and (b).}
\label{fig:mask_character}
\end{figure}

\begin{figure}[t]
\begin{center}
\includegraphics[width=\linewidth]{figures/fig_loss1.pdf}
\end{center}
\caption{Changes in (a) loss and (b) area of consistent/inconsistent regions during training. Adaptive region balancing is applied from 16K iteration, affecting the training dynamics.}
\label{fig:fig_loss1}
\end{figure}

\section{Method}

\subsection{Global-local View Training}
Owing to the specific nature of driving scenes, certain classes such as \textit{roads} are consistently large, and others like \textit{traffic light} remain small in size. Since the driving scenes capture road environments with a wide range of depth in each image, object sizes vary significantly with distance even within the same class, such as \textit{car}. In Fig.~\ref{fig:mask_resizencrop}~(a), we observe that the pseudo-masks generated by the CLIP model exhibit notably high quality for relatively large objects but poor quality for small objects. We conjecture that this performance degradation may occur from the training mechanism of CLIP (\ie, it mainly concentrate on salient objects corresponding to text prompt rather than small objects).

Building upon this observation, we manipulate the relative object size within the input by adjusting the \emph{image scale} and \emph{field-of-view} (FOV). We then analyze the resultant changes in the pseudo-mask obtained from CLIP. In Fig.~\ref{fig:mask_resizencrop}~(b), when the input is scaled to twice its original size, the pseudo-mask exhibit more accurate and fine-grained results along the object boundaries. Additionally, reducing the FOV by half (\ie, the network only observes one-quarter of the input at a time) leads to noticeable changes in the pseudo-mask, particularly pronounced for smaller objects such as \textit{motorcycle} (\cf, Fig~\ref{fig:mask_resizencrop}~(c)). This simple case study unveils distinctive characteristics associated with each adjustment.

Summarizing, we observed distinct responses of CLIP to cropping (changing FOV) and resizing (changing scales). (1) Resizing improves the localization at fine-grained areas such as edges. (2) Cropping enhances the classification of small objects. Capitalizing on the distinctive effects of cropping and resizing, we synergistically incorporate both augmentations into our approach. By jointly leveraging these functions, we enhance pseudo-mask performance, particularly for small objects.

Inspired by this observation, we have developed a new method called \textit{Local View Sampling}. This technique leverages the conventional augmented input known as the global view, commonly used for training segmentation networks. We extract a patch of a specific size (typically small) from an arbitrary position inside the global view. Then, the patch is resized randomly before passing it through CLIP. We obtain the local pseudo-mask by calculating the similarity of local features from the image encoder and text embedding as follows:

\begin{equation}
\label{eq_local_mask}
{\small
\mathbf{M}^\mathbf{l} = \arg \max(\frac{\boldsymbol{F}^\mathbf{l} \cdot \boldsymbol{t}}{\left \| \boldsymbol{F}^\mathbf{l} \right \| \cdot \left \| \boldsymbol{t} \right \|}),
}
\end{equation} %
%
where $\boldsymbol{F}^\mathbf{l}$ is the feature from CLIP with a local view image, $\boldsymbol{t}$ is text embedding of CLIP. The local view only contains semantic information in a small, confined, area, so it can fully exploit locality from CLIP. This empowers the pseudo-mask of the local view to better focus on small objects. We leverage the masks of both views to train the segmentation model. The loss for the global-local view training is computed by cross-entropy loss for each region as follows:
\begin{equation}
\label{eq_loss_loc}
{\small
\mathcal{L}_\mathbf{l} = - \frac{1}{|\mathbf{M}^\mathbf{l}|} \underset{i,j \in local}{\sum} \text{ } y^\mathbf{l} \log {f_{i,j}},
}
\end{equation}

\begin{equation}
\label{eq_loss_glo}
{\small
\mathcal{L}_\mathbf{g} = - \frac{1}{|\mathbf{M}^\mathbf{g}|} \underset{i,j}{\sum} \text{ } y^\mathbf{g} \log {f_{i,j}},
}
\end{equation}%
%
where $y^\mathbf{l}$ and $y^\mathbf{g}$ are one-hot labels of  $\mathbf{M}^\mathbf{l}$ and $\mathbf{M}^\mathbf{g}$, respectively. $\mathbf{M}^\mathbf{g}$ and $\mathbf{M}^\mathbf{l}$ are the global pseudo-mask and the local pseudo-mask, respectively. $f \in \mathbb{R}^{C \times H \times W}$ is the probability of segmentation network. The total loss for the global-local view training is $\mathcal{L} = \mathcal{L}_\mathbf{g} + \mathcal{L}_\mathbf{l}$.

\subsection{Consistency-aware Region Balancing} \label{sec:separation}
We identify noisy regions in the pseudo-mask created by CLIP. Fig.~\ref{fig:mask_character} showcases an example of a pseudo-mask containing small and blob-like noisy regions, randomly scattered on the image.

Conversely, training a segmentation network with pseudo-masks removes the randomly scattered noise of CLIP's pseudo-mask in the output, resulting in systematic predictions. (\eg, \textit{road} in Fig.~\ref{fig:mask_character} (b)) However, the segmentation prediction has misclassified pixels that were originally correct in the pseudo-mask. In particular, we observe that the trained segmentation model produces an indistinct boundary of the object even worse than the pseudo-mask generated by CLIP (\eg, \textit{sidewalk} in Fig.~\ref{fig:mask_character} (b)).

Owing to the unique properties of the trained segmentation model and CLIP, we leverage both models to benefit from their respective advantages. However, since the prediction of the segmentation model is already incorporated in the model, directly computing the loss using the segmentation prediction does not provide new evidence for training. To address this, we indirectly employ the segmentation prediction to distinguish the pixels of the pseudo-mask from CLIP. Specifically, utilizing prediction consistency, we regard the pixel as reliable if they are consistent and noisy if the predictions from the two models are inconsistent:

\begin{equation}
\label{eq_region_con}
{\small
\mathbf{M}^\mathbf{c} = \{P_{i,j} | P_{i,j} = S_{i,j}\},
}
\end{equation}
\begin{equation}
\label{eq_region_inc}
{\small
\mathbf{M}^\mathbf{i} = \{P_{i,j} | P_{i,j} \neq S_{i,j}\},
}
\end{equation}%
%
where $\mathbf{M}^\mathbf{c}$ and $\mathbf{M}^\mathbf{i}$ correspond to consistent and inconsistent regions, respectively. $P \in C^{H \times W}$ and $S \in C^{H \times W}$ are the pseudo-mask from CLIP and the prediction of the segmentation network, where $C$ is a set of classes. Furthermore, we apply label filtering when generating $S$ to prevent misprediction with non-existent classes in the image.

The consistent and inconsistent regions are recalculated in each iteration to update the segmentation model. As training progresses, we notice that the size of the consistent region changes, resulting in performance improvement of the segmentation model (Fig.~\ref{fig:fig_loss1}~(b)).

To understand the effects of consistent and inconsistent regions, we separately calculate the cross-entropy loss of each side:
\begin{equation}
\label{eq_loss_con}
{\small
\mathcal{L}_\mathbf{c} = - \frac{1}{|\mathbf{M}^\mathbf{c}|} \underset{i,j}{\sum} \text{ } y^\mathbf{c} \log {f_{i,j}},
}
\end{equation}

\begin{equation}
\label{eq_loss_inc}
{\small
\mathcal{L}_\mathbf{i} = - \frac{1}{|\mathbf{M}^\mathbf{i}|} \underset{i,j}{\sum} \text{ } y^\mathbf{i} \log {f_{i,j}},
}
\end{equation}%
%
where $y^\mathbf{c}$ and $y^\mathbf{i}$ are one-hot labels of  $\mathbf{M}^\mathbf{c}$ and $\mathbf{M}^\mathbf{i}$, respectively. $f \in \mathbb{R}^{C \times H \times W}$ is the probability of segmentation network. We observe that inconsistent regions have much higher loss values than consistent regions in Fig.~\ref{fig:fig_loss1} (a). If we treat the training loss equally across all regions, the network is overly influenced by the high loss produced from the inconsistent regions. Therefore, we suggest assigning different weights to the losses of consistent and inconsistent regions while taking into account the noise level of the data. It helps prevent the conventional cross-entropy loss from being vulnerable to noise in the training data.

To this end, we devise an adaptive region balancing method that dynamically adjusts the loss of the inconsistent region by monitoring the loss profiles in both the consistent and inconsistent regions during training. Specifically, we introduce two fixed-size queues which track the losses of the two regions, denoted as $\mathcal{L}_\mathbf{c}$ and $\mathcal{L}_\mathbf{i}$, respectively. We then compute the average loss from each queue. We use the ratio of two average losses as the weight for the cross-entropy loss of the inconsistent region, denoted as $w$, which is multiplied by the loss of the inconsistent region. The CARB training loss is $\mathcal{L} = \mathcal{L}_\mathbf{c} + w \cdot \mathcal{L}_\mathbf{i}$. This balancing ensures that the training is less influenced by the inconsistent region.

While one might consider that the loss from inconsistent regions can be simply neglected, our observations reveal that the inconsistent regions still possess useful learning signals. Notably, we observe that labels of highly correlated object classes (\ie, the classes sharing visual properties like \textit{bus} and \textit{car}) exist within the inconsistent region. Overlooking those pixels impedes the label imbalance problem. The Cityscapes dataset, for instance, includes classes like \textit{rider} (a subset of \textit{person}), \textit{bus}, and \textit{truck} (subsets of \textit{car}) that are susceptible to such confusion. Considering these challenges, we present a region-balancing method designed to harness meaningful information even from inconsistent regions.

\subsubsection{Overall Training.}
The proposed method consists of two stages. In the first stage, we warm up the baseline segmentation model with global and local views generated from CLIP masks. This step ensures that the segmentation network sufficiently learns the regular patterns of the target dataset. In the second stage, we refine the segmentation network utilizing CARB. We apply CARB for both global and local views.

\section{Experiments}
\subsection{Experimental Setup}
\subsubsection{Dataset \& Evaluation Metric.} For performance evaluation, we utilized the well-known Cityscapes~\cite{cordts2016cityscapes}, CamVid~\cite{brostow2009semantic}, and WildDash2~\cite{Zendel_2022_CVPR}, which are autonomous driving datasets. The Cityscapes dataset consists of 2,975 training, 500 validation, and 1,525 test images with fine annotation. It contains a total of 30 classes, and 19 classes are evaluated for public assessment while the rest are void. The CamVid dataset consists of 367 training, 101 validation, and 233 test images, containing a total of 32 classes. In our experiments, only 11 classes are evaluated by following the convention of previous research~\cite{wang2020deep}. The WildDash2 dataset consist of 3,618 training, 638 validation, and 812 test images, containing a total of 25 classes. In all experiments, we solely utilized image-level labels for training. The image-level labels are acquired from pixel-level labels of each dataset. Mean Intersection over Union (mIoU) was used as the evaluation criterion, a popular and standard metric for semantic segmentation.

\subsubsection{Implementation Detail.}
We employed ViT-B/16~\cite{dosovitskiyimage} as the image encoder for the CLIP, and ResNet50~\cite{he2016deep}-based DeepLab-ASPP~\cite{chen2017deeplab} as the segmentation network. The last convolutional layer of ASPP was replaced with text embedding of CLIP. The segmentation network was initialized with an ImageNet pre-trained model provided by MMSeg~\cite{mmseg2020}. Considering class definition and object words connoting actual objects, we replaced the \textit{vegetation} and \textit{terrain} class names with \textit{tree} and \textit{grass}, respectively. Furthermore, we changed the \textit{person} class to \textit{pedestrian}, since it is a superset of the \textit{rider}. For generating pseudo-masks from CLIP, we adopt MaskCLIP~\cite{zhou2022extract}.

\begin{table}[]
\normalsize
\centering
{\small
\begin{tabular}{@{}lc@{}}
\toprule
Method          & \multicolumn{1}{c}{mIoU}  \\ \midrule
Base                                & \multicolumn{1}{c}{40.1} \\
\multicolumn{1}{l}{+ CARB}           & \multicolumn{1}{c}{45.7} \\
\multicolumn{1}{l}{+ Local}          & \multicolumn{1}{c}{45.1} \\
\multicolumn{1}{l}{+ Local + CARB}   & \multicolumn{1}{c}{50.6} \\
\multicolumn{1}{l}{+ Dual}           & \multicolumn{1}{c}{45.8} \\
\multicolumn{1}{l}{+ Dual + CARB}    & \multicolumn{1}{c}{\textbf{52.1}} \\

\bottomrule
\end{tabular}
}
\caption{Ablation study of the proposed modules. The accuracy (mIoU) is evaluated on the Cityscapes validation set. The best score is in \textbf{bold} throughout all experiments.}
\label{tab:split}
\end{table}

\subsection{Ablation Study}
\label{sec:ablation}
\subsubsection{Effects of Each Module.} We evaluate the effectiveness of each component of our method in Tab.~\ref{tab:split}. When we train the segmentation model with additional local view sampling ($Local$), it shows a remarkable improvement of 5.0\%p. This implies that additional information from local patches through cropping and resizing provides rich learning signals. CARB alone contributed an impressive improvement of 5.6\%p, indicating that adaptively re-weighting the loss according to its reliability plays a critical role in learning with noisy pseudo-masks. By combining local view sampling and CARB ($Local + CARB$), a substantial improvement of 10.5\%p was achieved. Instead of $Local$, we added a slight modification named $Dual$, by re-creating the mask of global views depending on the augmentation using CLIP for each iteration. This modification yielded a 0.7\%p gain over $Local$. Interestingly, our $Dual+CARB$ method shows a 1.5\%p improvement compared to $Local + CARB$, indicating synergy between various-sized mask creations and our noise-handling strategy.

\begin{figure}[t]
\begin{center}
\includegraphics[width=\linewidth]{figures/fig_ablation.pdf}
\end{center}
\caption{Segmentation results (mIoU) on Cityscapes validation set depending on (a) the crop size (b) the weight. We set the length of one side to 512 and varied the length of the other side between 128 and 512. Yellow star indicates the experiment using $256 \times 256$ patch.}
\label{fig:ablation}
\end{figure}

\begin{table}[]
\normalsize
\centering
{\small
\begin{tabular}{@{}lrr@{}}
\toprule
\multicolumn{1}{c}{Method}                              & \multicolumn{1}{c}{val} & \multicolumn{1}{c}{test} \\ \midrule
\multicolumn{1}{l}{DeepLab-ASPP (Full supervision)} & 78.3 & \multicolumn{1}{r}{75.8} \\ \midrule
\multicolumn{1}{l}{AffinityNet} &  8.2 & \multicolumn{1}{r}{-} \\
\multicolumn{1}{l}{SEAM} & 17.3 & \multicolumn{1}{r}{-} \\
\multicolumn{1}{l}{1-Stage} & 11.8 & \multicolumn{1}{r}{-} \\
\multicolumn{1}{l}{\citeauthor{wang2020deep}} & 24.2 & \multicolumn{1}{r}{24.9}          \\
\multicolumn{1}{l}{CAM} & 33.0 & \multicolumn{1}{r}{32.2} \\
% \multicolumn{1}{l}{EPS} & 29.4 & \multicolumn{1}{r}{28.6} \\
\multicolumn{1}{l}{AMN} & 17.5 & \multicolumn{1}{r}{17.8} \\
\multicolumn{1}{l}{CLIMS } & 18.1 & \multicolumn{1}{r}{18.0} \\
\multicolumn{1}{l}{CLIP-ES } & 35.4 & \multicolumn{1}{r}{35.0} \\ \midrule
\multicolumn{1}{l}{Ours}                                & \textbf{52.1}     & \multicolumn{1}{r}{\textbf{51.8}} \\ \bottomrule

\end{tabular}
}
\caption{Segmentation results (mIoU) on Cityscapes.}
\label{tab:seg_cityscapes}
\end{table}

\subsubsection{Effects of the Crop Size and Resize Ratio.} In our empirical investigation, it was consistently observed that vertically long rectangular patches exhibited superior performance in terms of crop sizes compared to patches of other sizes. This finding is supported by Fig.~\ref{fig:ablation}. We conjecture this tendency in the driving scene dataset comes from vertically long structures such as \textit{pole} and \textit{traffic light}. Also, the 512$\times$512 patches for local views are more effective than the 256$\times$256. These experiments suggest that, while the local view represents a smaller portion of the overall scene, excessively small sizes may not benefit from the attention layers equipped in CLIP.

We evaluated our local view sampling under variable resize ratios. Under the fixed ratio from 0.5 to 2.0, we observe the best performance of 52.1\% at the ratio of 1.0 while a significant drop with other ratios. However, when focusing on classwise performances under various resize ratios, we confirmed that a large resize rate benefits the performance of small classes, such as \textit{traffic light} and \textit{rider}. Meanwhile, the performance of the large classes, such as \textit{sidewalk} and \textit{wall}, is decreased. Due to the performance trade-off across different classes, we set a random value between 1.0 and 2.0 as the resize ratio. Our choice leads to the overall best performance in both small and large classes.

\subsubsection{Effects of Adaptive Region Balancing.} We compare our adaptive region balancing strategy with a fixed weighting strategy, where the loss weight for the inconsistent region ($w$) is set to a specific value (\cf, Fig.~\ref{fig:ablation} (b)). When changing the fixed weight $w$ gradually from 0 to 0.5, we observe the best score at the weight of 0.1 and a significant drop with other values. Although the highest performance of the fixed weight strategy is similar to that of our method (51.06\% for fixed strategy and 52.1\% for our method), it requires a hyperparameter search for the optimal weight using the validation dataset. In contrast, our method does not require such a search, making it more suitable for WSSS scenario.

\subsection{Quantitative Comparisons}
\subsubsection{Remarks on Comparisons.} Existing WSSS methods focus on handling object-centric datasets such as PASCAL VOC 2012. Therefore, their methods are primarily designed to distinguish relatively simple object shapes with similar scales, which is still a valuable research direction. Given this dataset mismatch, direct comparisons between our method and existing WSSS approaches might not be entirely fair, as they cater to distinct dataset characteristics. Nevertheless, by adapting established WSSS methods to driving scenes, we intend to show that the existing framework is ineffective for our application scenario.

\begin{figure*}[t!]
\centering
\includegraphics[width=16cm]{figures/fig_qualitative_city.pdf}
\caption{Qualitative results on Cityscapes validation set. (a) Input image, (b) Ground-truth, (c) CLIP-ES, and (d) Our method.}
\label{fig:qualitative_city}
\end{figure*}

\subsubsection{Existing WSSS Methods.} Existing methods can be partitioned into CAM-based methods (\ie, an image classifier for generating the pseudo-masks) and CLIP-based methods (\ie, CLIP for generating the pseudo-masks). Among them, we choose several representative methods such as (1) AffinityNet~\cite{ahn2018learning}, (2) SEAM~\cite{wang2020self}, (3) 1-Stage~\cite{araslanov2020single}, (4) SEC~\cite{kolesnikov2016seed}, (5) Wang et al.~\cite{wang2020deep}, (6) CAM~\cite{zhou2016learning}, and  (7) AMN~\cite{lee2022threshold}. To compare with CLIP-based WSSS, we reproduce the driving scene results of (8) CLIMS~\cite{Xie_2022_CVPR} and (9) CLIP-ES~\cite{Lin_2023_CVPR}, both of which use the same level of information, CLIP, as our method.

\begin{table}[]
\normalsize
\centering
{\small
\begin{tabular}{@{}lrr@{}}
\toprule
\multicolumn{1}{c}{Method}          & \multicolumn{1}{c}{val}   & \multicolumn{1}{c}{test} \\ \midrule
\multicolumn{1}{l}{DeepLab-ASPP (Full supervision)}     & 81.6                      & \multicolumn{1}{r}{74.9}              \\ \midrule
\multicolumn{1}{l}{SEC} & - &  \multicolumn{1}{r}{2.5}              \\
\multicolumn{1}{l}{AffinityNet} & 11.0 & \multicolumn{1}{r}{15.5} \\
\multicolumn{1}{l}{\citeauthor{wang2020deep}} & 23.5 & \multicolumn{1}{r}{30.4} \\
\multicolumn{1}{l}{CAM} & 9.6 & \multicolumn{1}{r}{6.6} \\
\multicolumn{1}{l}{AMN} & 10.7 & \multicolumn{1}{r}{7.6} \\
\multicolumn{1}{l}{CLIMS} & 2.7 & \multicolumn{1}{r}{4.3} \\
\multicolumn{1}{l}{CLIP-ES} & 41.7 & \multicolumn{1}{r}{39.6} \\ \midrule
\multicolumn{1}{l}{Ours} & \textbf{55.7} & \multicolumn{1}{r}{\textbf{50.5}} \\ \bottomrule
\end{tabular}
}

\caption{Segmentation results (mIoU) on CamVid.}
\label{tab:seg_camvid}
\end{table}

\subsubsection{Cityscapes.} Tab.~\ref{tab:seg_cityscapes} presents the performance of our proposed CARB compared to other methods in driving scenes. Specifically, our approach achieves 51.8\% on the Cityscapes test set, which outperforms the \citeauthor{wang2020deep} by 26.9\%p and
previous CLIP-based WSSS technique by 16.8\%p. Additionally, we observed that our method consistently performs better than CLIP-ES in every class. Fig.~\ref{fig:qualitative_city} showcases qualitative examples of segmentation results on the Cityscapes. Notably, CARB successfully eliminates misclassified \textit{sidewalk} regions on the \textit{sky} class (see the first and second rows). These results visually confirm that our method correctly captures each class and successfully reduces the prediction errors.

\subsubsection{CamVid.} The CamVid dataset has a much smaller number of training images compared to Cityscapes, with only 367 images. Additionally, it is not possible to differentiate between the \textit{car} and \textit{road} classes using only image-level labels, as they appear in all images. However, our method can distinguish them by utilizing the pre-trained image-text information from the CLIP model. Tab.~\ref{tab:seg_camvid} shows that the performance of CAM-based methods (\eg, SEC, AffinityNet, \citeauthor{wang2020deep}, and CLIMS) is considerably low while our method achieves significantly higher performance. This demonstrates that our proposed approach can address the problem even when the scale of the dataset is small and has severe contextual bias. %(\ie, classes are indistinguishable only with image-level labels).

\subsubsection{WildDash2.} Since the WildDash2 dataset possesses extremely high diversity, it is generally challenging even for the fully supervised model. A classifier-based WSSS method such as CLIMS performs poorly, 1\% in mIoU, which is worse than a random guess. This poor performance is caused by difficulties in training the classifier due to class imbalance and complex class distribution. Since CLIP-ES and our method are built upon CLIP for generating the pseudo masks, both methods provide relatively reasonable performances. Our method achieves considerably high performance compared to CLIP-ES, with the performance gain primarily observed in small classes such as \textit{billboard}, \textit{rider}, \textit{bicycle}, and \textit{road marking}.

\begin{table}[]
\centering
{\small
\begin{tabular}{@{}lr@{}}
\toprule
\multicolumn{1}{c}{Method}                              & \multicolumn{1}{c}{Result} \\ \midrule
\multicolumn{1}{l}{DeepLab-ASPP (Full supervision)}     & \multicolumn{1}{r}{54.0}          \\ \midrule
\multicolumn{1}{l}{CAM} & \multicolumn{1}{r}{15.2} \\
\multicolumn{1}{l}{AMN} & \multicolumn{1}{r}{18.8} \\
\multicolumn{1}{l}{CLIMS} & \multicolumn{1}{r}{1.0} \\
\multicolumn{1}{l}{CLIP-ES} & \multicolumn{1}{r}{24.7}          \\ \midrule
\multicolumn{1}{l}{Ours}                                & \multicolumn{1}{r}{\textbf{32.2}} \\ \bottomrule
\end{tabular}
}
\caption{Segmentation results (mIoU) on WildDash2 \textit{val} set.}
\label{tab:seg_coarse}
\end{table}

\section{Conclusion}
This paper addressed the limitations of conventional CAM-based, weakly-supervised semantic segmentation (WSSS) methods when handling the driving scene datasets. To break the performance bottleneck of the CAM-based methods, we utilized CLIP as the pseudo-mask generator. Then, we proposed global-local view training, which exploits the characteristics of CLIP generating diverse masks depending on the relative object sizes. We also propose a novel training strategy, namely consistency-aware region balancing (CARB). It distinguishes between reliable and noisy regions utilizing prediction consistency and then suppresses the latter regions during training. By incorporating  these two components, our method successfully (1) learns to segment small objects and (2) heavily relies on reliable regions while effectively handling challenging objects from noisy regions. Extensive experiments demonstrate that each component of our method contributes to achieving new state-of-the-art performances on the Cityscapes, CamVid, and WildDash2 datasets in WSSS. Our study introduces a new approach addressing the challenges posed by driving datasets and suggests a promising direction for future research in WSSS.

\section*{Acknowledgments}
This research was supported by the Basic Science Research Program through the National Research Foundation of Korea (NRF) funded by the MSIP (NRF-2022R1A2C3011154, RS-2023-00219019, RS-2023-00240135) and MOE (NRF-2022R1A6A3A13073319), the IITP grant funded by the Korea government(MSIT) (No. 2019-0-00075, Artificial Intelligence Graduate School Program(KAIST)), KEIT grant funded by the Korea government(MOTIE) (No. 2022-0-00680, 2022-0-01045, 2021-0-02068, Artificial Intelligence Innovation Hub) and Samsung Electronics Co., Ltd (IO230508-06190-01).

\documentclass[letterpaper]{article} % DO NOT CHANGE THIS
\usepackage{aaai20}  % DO NOT CHANGE THIS
\usepackage{times}  % DO NOT CHANGE THIS
\usepackage{helvet} % DO NOT CHANGE THIS
\usepackage{courier}  % DO NOT CHANGE THIS
\usepackage[hyphens]{url}  % DO NOT CHANGE THIS
\usepackage{graphicx} % DO NOT CHANGE THIS
\urlstyle{rm} % DO NOT CHANGE THIS
\def\UrlFont{\rm}  % DO NOT CHANGE THIS
\usepackage{graphicx}  % DO NOT CHANGE THIS
\frenchspacing  % DO NOT CHANGE THIS
\setlength{\pdfpagewidth}{8.5in}  % DO NOT CHANGE THIS
\setlength{\pdfpageheight}{11in}  % DO NOT CHANGE THIS

\usepackage[utf8]{inputenc}
\usepackage{bbold}
\usepackage{amssymb}
\usepackage{amsmath}
\usepackage{balance}
\usepackage{tikz}
\usepackage{subfig}
\usepackage{bbm}
\usepackage{enumitem}
\usepackage{appendix}

\DeclareMathOperator*{\argmax}{arg\,max}
\providecommand{\TODO}[1]{\textcolor{red}{\textbf{#1}}}



%\title{Automating Product Placement in Retail via Stochastic Demand Simulation}

\title{A Probabilistic Simulator of Spatial Demand for Product Allocation}

%Your title must be in mixed case, not sentence case. 
% That means all verbs (including short verbs like be, is, using,and go), 
% nouns, adverbs, adjectives should be capitalized, including both words in hyphenated terms, while
% articles, conjunctions, and prepositions are lower case unless they
% directly follow a colon or long dash
\author{Porter Jenkins \textsuperscript{\rm 1}, Hua Wei \textsuperscript{\rm 1}, J. Stockton Jenkins \textsuperscript{\rm 2}, Zhenhui Li \textsuperscript{\rm 1} \\ 
\textsuperscript{\rm 1} Penn State University \\
\textsuperscript{\rm 2} Brigham Young University \\%If you have multiple authors and multiple affiliations
% use superscripts in text and roman font to identify them. For example, Sunil Issar,\textsuperscript{\rm 2} J. Scott Penberthy\textsuperscript{\rm 3} George Ferguson,\textsuperscript{\rm 4} Hans Guesgen\textsuperscript{\rm 5}. Note that the comma should be placed BEFORE the superscript for optimum readability
}


\begin{document}
\maketitle

\begin{abstract}
Connecting consumers with relevant products is a very important problem in both online and offline commerce. In physical retail, product placement is an effective way to connect consumers with products. However, selecting product locations within a store can be a tedious process. Moreover, learning important spatial patterns in offline retail is challenging due to the scarcity of data and the high cost of exploration and experimentation in the physical world. To address these challenges, we propose a stochastic model of spatial demand in physical retail. We show that the proposed model is more predictive of demand than existing baselines. We also perform a preliminary study into different automation techniques and show that an optimal product allocation policy can be learned through Deep Q-Learning. 

\end{abstract}



\section{Introduction}
%%%%%%%%%%%%%%%%%%%%%%%%%%%%%%
% 1.定义image captioning任务 
%%%%%%%%%%%%%%%%%%%%%%%%%%%%%%
Image captioning is a fundamental task in vision-language understanding that involves generating natural language descriptions for a given image. It plays a critical role in facilitating more complex vision-language tasks, such as visual question answering \cite{Agrawal2015VQAVQ,gqa,okvqa} and visual dialog \cite{Das2016VisualD,Niu2018RecursiveVA,llava}.
%%%%%%%%%%%%%%%%%%%%%%%%%%%%%%
% text-only training 的介绍
%%%%%%%%%%%%%%%%%%%%%%%%%%%%%%
The mainstream image captioning methods \cite{conimgcap4,conimgcap1,conimgcap3,conimgcap2} require expensive human annotation of image-text pairs for training neural network models in an end-to-end manner. Recent developments in Contrastive Image Language Pre-training (CLIP) \cite{clip} have enabled researchers to explore a new paradigm, zero-shot image captioning, through text-only training. In particular, CLIP learns a multi-modal embedding space where semantically related images and text are encoded into features with close proximity. As such, if a model learns to map the CLIP text features to their corresponding texts, it is feasible to generate image captions from the CLIP image features without needing supervision from caption annotations.

%%%%%%%%%%%%%%%%%%%%%%%%%%%%%%
% text-only training 的优势
%%%%%%%%%%%%%%%%%%%%%%%%%%%%%%

One main advantage of this zero-shot captioning paradigm is that it enables a Large Language Model (LLM) \cite{gpt3, Zhang2022OPTOP} with image captioning capabilities using only text data and affordable computational resources. Despite the impressive performance achieved by recent powerful multimodal models \cite{miniGPT4,llava}, they typically require large-scale, high-quality human-annotated data and expensive computational resources for fine-tuning an LLM. Zero-shot captioning methods can significantly reduce such costs, which is particularly important in situations of data scarcity and limited resources. Moreover, recent work \cite{Guo2022FromIT, Changpinyo2022AllYM,Tiong2022PlugandPlayVZ} demonstrates that other vision-language tasks, such as VQA, can be addressed by LLMs and image captions. Consequently, the paradigm of zero-shot captioning has the potential to pave the way to solving complex vision-language tasks with LLMs through efficient text-only training. 


%%%%%%%%%%%%%%%%%%%%%%%%%%%%%%
% zero-shot image captioning via text-only training 的challenge
%%%%%%%%%%%%%%%%%%%%%%%%%%%%%%
A critical challenge in zero-shot image captioning through text-only training is to mitigate a widely observed phenomenon known as the \textit{modality gap}. While the features of paired texts and images are close in the CLIP embedding space, there remains a gap between them \cite{MindGap}. This gap often results in inaccurate mappings from the image embeddings to the text ones. Consequently, without fine-tuning with paired data, it significantly impairs the performance of zero-shot image captioning.
%%%%%%%%%%%%%%%%%%%%%%%%%%%%%%
% current works intro
%%%%%%%%%%%%%%%%%%%%%%%%%%%%%%
Several works have attempted to address the modality gap in zero-shot image captioning, relying mainly on two strategies: (1) The first strategy leverages a memory bank from training text data to project visual embeddings into the text embedding space \cite{DeCap}. However, this projection prevents it from representing any semantic content outside the distribution of the memory bank features and introduces extra inference costs; (2) The second approach injects noise during training to encourage the visual embeddings to be included inside the semantic neighborhood of the corresponding text embeddings \cite{CapDec}. Nonetheless, the noise injection tends to diffuse the distribution of visual inputs at the cost of weakening the semantic correlation between paired images and text embeddings. 

%However, in the first strategy, the projection of visual embeddings prevents them from  For the second strategy, noise injection during training diffuses the input distribution at the cost of degrading the semantic correlation between paired images and text embeddings.

%Previous attempts \cite{CapDec,DeCap} to reduce the modality gap in zero-shot image captioning can be summarized into two aspects: (1) Decap\cite{DeCap} leverages a memory bank from training text data to project visual embeddings into text embedding space. However, the projection of visual embeddings prevents it from representing any semantic content outside the distribution of the memory bank and introduce extra inference cost. (2) CapDec\cite{CapDec}proposes to inject noise during training to encourage the visual embedding to be included inside the text embedding space. 
% current work weakness
%Nevertheless, noise injection during training diffuses the input distribution at the cost of degrading the semantic correlation between paired images and text embeddings.


%%%%%%%%%%%%%%%%%%%%%%%%%%%%%%
% 我们工作的流程
% 分析得到两个结论:1.subregion带来更好的匹配2.image text gap符合高斯分布
%%%%%%%%%%%%%%%%%%%%%%%%%%%%%%
To tackle these challenges, we first conduct a thorough analysis of the CLIP feature space, leading to two key observations. First, most text descriptions are unable to fully capture the content of their paired images. However, we empirically find that the visual embedding of certain local regions of an image, named image subregions, have closer proximity to the text embedding of the paired caption. Integrating such image subregions with the global image representation generates a tighter alignment between image and text. Additionally, we analyze the distribution of the gap between the CLIP features of image or subregion-text pairs and find that it closely resembles a zero-mean Gaussian distribution.
%initiate our investigation by conducting a thorough analysis of the CLIP latent space. Building upon the insights from the work \cite{MindGap}, we identify a key factor contributing to the existence of a modality gap. Due to the inherent disparities between textual and visual modalities, text is incapable of comprehensively describing the information within an image. However, we empirically demonstrate that the CLIP embedding of some part of image, named image subregions, exhibit closer proximity to the CLIP embedding of the paired caption. The integration between image subregion information and global image feature leads to more compact image text alignment. Besides, we collect the statistics of the gap between CLIP image and text feature. The results demonstrate the gap is close to gaussian distribution. 

%%%%%%%%%%%%%%%%%%%%%%%%%%%%%%
% 我们的方法简略介绍
%%%%%%%%%%%%%%%%%%%%%%%%%%%%%%

Based on our findings, we propose a novel zero-shot image captioning framework, named \textit{\textbf{M}ining Fine-Grained Image-Text \textbf{A}lignment in \textbf{C}LIP for \textbf{Cap}tioning} (MacCap), to address the aforementioned challenges. In this framework, we introduce a region-aware cross-modal representation based on CLIP and an effective unimodal training strategy for an LLM-based caption generator. Our cross-modal representation maps an input image into the language space of LLMs and consists of two main components. First, we design a \textit{sub-region feature aggregation} module to fuse both global and subregion-level CLIP image features, resulting in a smaller gap between the corresponding CLIP text embedding. Next, we introduce a learnable adaptor-decoder to transform the CLIP representation into the LLM's language space.
To train our model with text-only data, we develop a robust procedure to learn a projection from the CLIP embedding space to a language representation, enabling the LLM to reconstruct captions. Specifically, our learning procedure first injects noise into our region-aware CLIP-text representation, mimicking the modality gap between image and text features. This is followed by a multiple sampling and filtering step that leverages the CLIP knowledge to improve the quality of the captioning.
%tackles the problem from three key perspectives. Firstly, we focus on learning a robust projection from CLIP embedding space to language model space by text reconstruction training, which enable model to generate text based on both CLIP image and text feature. The region noise injection in training alleviate the \textit{modality gap} between image and text feature, which makes the projection works for both image and text features. Secondly, we design \textit{sub-region feature aggregation} to obtain a more compact CLIP image feature, which is based on the observation that CLIP subregion feature exhibit closer disntance with corresponding text feature. Third, we propose multiple sampling and filtering to mitigate the drawbacks of noise injection, which leverage CLIP knowledge to further boost caption performance. Finally, we design a pipeline for zero-shot VQA to demonstrate the extensibility of ouir methods to more intricate vision-language tasks.
In addition to the image captioning task, we further extend our framework to build a zero-shot VQA pipeline, demonstrating the generality of our cross-modal representation for more complex vision-language tasks.

%%%%%%%%%%%%%%%%%%%%%%%%%%%%%%
% 我们的方法简略介绍
%%%%%%%%%%%%%%%%%%%%%%%%%%%%%%

We evaluate our framework on several widely-adopted image captioning benchmarks, such as MSCOCO \cite{mscoco} and Flickr30k \cite{Flickr30k}, as well as a standard VQA benchmark, VQAV2 \cite{vqav2}. Our extensive experiments cover multiple vision-language tasks, including zero-shot in-domain image captioning, zero-shot cross-domain image, and zero-shot VQA. The results not only demonstrate the superiority of our methods but also validate our findings on the CLIP embedding space.

% demonstrate through experiments that our proposed methods outperform previous approaches on popular captioning benchmarks, such as MSCOCO, Flickr30k, which further verify our understanding of \textit{concept region}



% Specifically, we evaluate the distribution of the image and text embedding space under hyperspherical coordinates and observe a geometric phenomenon \textit{concept region} 
% where semantically correlated image and text embedding tend to clustering despite the \textit{modality gap}.
% 我们基于concept region的观察提出的方法:concept region和modality gap的cause里面有mismatch pair data导致的semantic ambiguity,总体思路是在train的时候模拟在concept region。在training的时候,我们给text embedding加上region noise,具体而言就是以原本text embedding为中心,一定范围内的多个随机sample的related text embedding,这样的获得的text embedding全都是在输入text对应的concept region内部。在zs captioning的inference时,部分image sub-region inforamtion 会比global image 对text匹配度更高,因此我们基于部分image sub-region inforamtion
% Motivated by the semantic ambiguity of mismatched data observed in \textit{concept region}, we propose two 
% an image sub-region information aggregation strategy for .In detail

% result summary

\section{Problem Definition}\label{prob-def}
In the following section, we provide a formal definition of the optimal allocation problem. Additionally, we define the necessary components of our reinforcement learning agent: the state space, action space, reward function, and state transition function.
\subsection{Optimal Allocation Problem}

In a physical retail environment $\mathcal{R}$ with a set of $n$ spatial regions, we represent the environment with a spatial graph $\mathcal{R} = (\mathcal{V}, \mathcal{E})$, where each region $r_i\in \mathcal{V}$ is a vertex in the graph, the spatial neighboring relation between two regions $r_i$ and $r_j$ are represented as $e_{ij}\in \mathcal{V}$. From $\mathcal{G}$, we can construct the adjacency matrix, $\textbf{A}$.

Additionally, we observe a set of $k$ products, $\mathcal{M} = \{m_j : 0 < j <=k\}$ that are sold. For each product, $m_j$, we know the retail price, $p_j$. 

The decision process faced by the retailer is to allocate each product in $\mathcal{M}$ across regions in $\mathcal{R}$. We define the allocation policy as a function $f$:

\begin{equation}
    f: \mathcal{R} \times \mathcal{M} \rightarrow \mathcal{Z}
\end{equation}
\begin{equation}
    \mathcal{Z} = \{\langle r_i, p_j \rangle , ... \langle r_w, p_q \rangle \}
\end{equation}

Where $\mathcal{Z}$ is the set of selected product region, such that $w <= n$, $q <= k$ and $\mathcal{Z} \subseteq \mathcal{R} \times \mathcal{M}$. This function is typically dynamic over time, which we denote as $f^{t}$. To simplify computation, we treat $\mathcal{Z}^{t}$ as an $(n \times k)$ grid and refer to it as the board configuration at time, $t$. An optimal retail strategy is to find the allocation policy that maximizes revenue:

\begin{equation}
    f^{\ast} = \sum_{t}^{T} \argmax_{f^{t}} \sum_{i, j \in f^{t}(\mathcal{R}, \mathcal{M})} p_j q_i
\end{equation}

where $p_j$ is the price for product $m_j$, and $q_i$ is the quantity sold in region $r_i$ and $T$ is the future time horizon of analysis. The main idea of the current work is to discover the long-term, optimal allocation policy, $f^{\ast}$ from data.

\subsection{Optimal Allocation as a Markov Decision Process}
We believe that the optimal allocation problem is well suited for reinforcement learning because the RL agent is designed for sequential decision making that maximizes expected discounted reward over time. We frame the inputs as a Markov Decision Process (MDP). An MDP is defined by the tuple $\langle \mathcal{S}, \mathcal{A}, P, r, \delta  \rangle$, where $\mathcal{S}$ is the state space, $\mathcal{A}$ is the set of possible actions, $P$ is the (typically unkown) state transition function, $r$ is the reward function and $\delta \in [0,1]$ is the discount factor. 

\begin{itemize}
    \item \textbf{State} At each time, $t$, we observe the state of the retail environment, $\mathcal{E}$. We define the state, $s_t \in \mathcal{S}$, as the tuple of state features, $s_t = \langle \mathcal{Z}^{{t}}, d^{t}, \textbf{g}^{(t-1)}  \rangle$, where $\mathcal{Z}^{{t}}$ is the current board configuration, $d^t$ is the current day of the week (e.g., Sunday $\rightarrow$ 0), and $\textbf{g}^{(t-1)}$ is a vector denoting the revenue at the previous time, $(p_j q_i)^{(t-1)} \forall z \in \mathcal{Z}^t$

    \item \textbf{Action} We define the action space  $\mathcal{A} = \mathcal{R} \times \mathcal{M} \times \{-1, 1\} \cup \{0\}$, indicating ``to place'', ``take way'' or ``do nothing'' for each product, $m_j$ in each region, $r_i$.
    \item \textbf{Reward} The reward function in this case is the total product revenue at time $t$, constrained by the monetary cost, $c$, of placing a set of products in each region:
    \begin{equation}
        r(t) = \sum_{i=1}^n \sum_{j=1}^k p_j q_{ij}^{t} - c \sum_{i=1}^n \mathbbm{1}_{\mathcal{Z}}(r_i)
    \end{equation}
    
    \item \textbf{State transition function}: The state transition, $P$ is defined as $p(s^{t+1} | s^t, a^t): \mathcal{S} \times \mathcal{A} \times \mathcal{S} \rightarrow [0,1]$, which gives the probability of moving to state, $s^{(t+1)}$ given the current state and action. In the optimal allocation problem the exact transition function, $P$ is unknown since the current state, $s^t$ depends on the results of the previous time, $\textbf{g}^{(t-1)}$. We model this transition as a stochastic process.
\end{itemize}
\section{Commonsense for Zero-Shot NLVL}
\label{sec:proposedSection}

\subsection{Problem Formulation}
We denote an input video as $V$, and its grounding annotations as \(\left( Q,V_{\text{span}}\right) \), where $Q$ is the query representation and \(V_{\text{span}}\!=\!\left( t_{s},t_{e}\right)\) is the corresponding video moment span annotation, with \(t_{s}\) and \(t_{e}\) representing the start and end timestamps, respectively. Learning to localize a video moment conditioned on a query entails maximizing the expected log-likelihood of the model parameterized by \(\theta\). In its typical setting, this can be formulated as follows:
\begin{equation}
\label{eq:groundingOriginal}
    \theta ^{\ast }=\arg \max _{\theta } \mathbb{E}\left[ \log p_{\theta }\left(  V_{\text{span}} | V,Q\right) \right]. 
\end{equation}
In the zero-shot setting, the goal is to learn this task without parallel video-query annotations. Hence, the query and video moment annotations are derived from $V$, using a dynamic video moment proposal method followed by a pseudo-query generation mechanism. Formally,  \(V_{\text{span}}\,\!{=}\!\,f_{\text{span}}(V)\) and \(Q\,\!{=}\!\,f_{pq}(V_{\text{span}})\), where $f_{\text{span}}$ and $f_{\text{pq}}$ are video moment proposal and pseudo-query generation mechanisms, respectively. Given that $f_{\text{span}}$ and $f_{\text{pq}}$ are responsible for generating the annotations, the performance of the localization model heavily depends on the quality of these modules. Existing methods face challenges in aligning \(Q\) to \(V_{\text{span}}\) due to noise introduced by ungrounded pseudo-query generation mechanisms. 
To address this, we simplify \(f_{\text{pq}}\) while augmenting cross-modal understanding by leveraging external information in the form of a commonsense graph \(G_{C}(C, E)\) with \(n_c\) nodes, where \(C\!=\!\left\{c_{1}, c_{2}, \dots, c_{n_{C}}\right\}\) are the concept node vector representations and \(E\) is the set of weighted directed edges, respectively. Accordingly, learning can be formulated as
\begin{equation}
\label{eq:groundingOurs}
    \theta ^{\ast }=\arg \max _{\theta } \mathbb{E}\left[ \log p_{\theta }\left(  V_{\text{span}}| V,Q,G_{C}\right) \right].
\end{equation}

\noindent Figure \ref{fig:approach} shows both training and inference flows.
\subsection{Pseudo-supervised Setup}
\modelname first processes a raw video with a video moment proposal $f_{\text{span}}$ module that extracts important video segments capturing key events, and a pseudo-query generation $f_{\text{pq}}$ that generates text query annotations corresponding to the extracted video segments.

\paragraph{Dynamic Video Moment Proposal ($f_{\text{span}}$).}
We adopt the dynamic video moment proposal approach proposed by \citet{nam_zero-shot_2021}. Specifically, $f_{\text{span}}$ primarily comprises a k-means clustering mechanism that groups semantically similar and temporally proximal video frame features together to extract atomic moments. To obtain frame features, we consider the columns of a frame-wise similarity matrix derived from the CNN features of individual frames. We enforce temporal proximity by concatenating the frame index to the features. Composite video moments are then formed by combining neighboring atomic moments, and a subset of all possible combinations is sampled uniformly at random. The resulting set of video moments corresponds to $V_{\text{span}}$.

\paragraph{Pseudo-query Generation ($f_{\text{pq}}$).} The pseudo-query is constructed as a collection of objects present in the video. To generate the pseudo-query, we employ an off-the-shelf object detector, enabling the extraction of pertinent objects in \(V_{\text{span}}\). We adopt a top-$k$ strategy to sample the $k$ most probable object predictions associated with the query \query.

\paragraph{Video Encoder.}
We uniformly sample $T$ frames from $V$ and extract their CNN (\eg, I3D~\cite{qian_locate_2022}) features. These features are contextually encoded using a video encoder ${\phi}_{v}$ to yield frame features ${\phi}_{v}(V)\!=\!\left\{ v_{1},v_{2},\ldots,v_{T}\right\}$ where $v_{i}\in\mathbb{R}^{d}$, and $d$ is the common video/query encoding dimension. We implement ${\phi}_{v}$ as a GRU-based encoder.

\paragraph{Query Encoder.}
Our pseudo-query $Q$, composed of up to $k$ tokens, is encoded using a query encoder ${\phi}_{q}$ that generates query embeddings ${\phi}_{q}(Q)\!=\!\left\{ q_{1},q_{2},\ldots,q_{k}\right\}$, for the top-$k$ detected objects extracted from the pseudo-query generation. Here, $q_{i}\in \mathbb{R}^{d}$ and $d$ is the common video/query encoding dimension. We implement ${\phi}_{q}$ as a bi-directional GRU-based encoder preceded by a trainable embedding layer. 

\subsection{Commonsense Enhancement Module}
\label{sec:cem}
To enrich the encoded video and query features with information grounded in commonsensical knowledge, we introduce a Commonsense Enhancement Module (CEM), pictorially described in Figure~\ref{fig:cem}. This enhancement helps inject necessary information into video and query representations, which can not just help bridge the gap between the available visual and textual cues but also provide rich information to the downstream span localization module. 

\begin{figure}[t!]
    \centering
    \includegraphics[width=0.8\linewidth]{figures/figure_files/Cem.pdf}
    \caption{\modelname Commonsense Enhancement Module (CEM). CEM comprises a concept encoder and an enhancement mechanism that uses the previously encoded concept vectors to update a given input vector (video/query vectors). The concept encoder employs a Graph Convolution Network for encoding the nodes (concepts) of \(G_C\). 
    }
  \label{fig:cem}
\end{figure}

CEM includes a set \(C\!=\!\left\{c_{1}, c_{2}, \dots, c_{n_{C}}\right\}\) of \(n_{C}\) concept vectors, where \(c_{i} \in \mathbb{R}^{d}\) and \(d\) is the concept feature dimension (same dimension as $\forall v_i \in V$ and $\forall q_i \in Q$). In general, given source feature vectors $S\!=\!\left\{ s_{1},s_{2},\ldots,s_{n}\right\}$ with individual feature vectors $s_{i \in [1,n]} \in \mathbb{R}^{d}$, the enhanced feature vectors $S_{C}$ are obtained using a commonsense enhancement mechanism $\phi_{C}$.
We implement this commonsense enhancement step $\phi_{C}$ as a cross-attention mechanism that enriches source input features, attending over $S$ guided by the commonsense concept vectors $C$, \ie, 
\begin{equation}
\label{eq:cenhance}
\scalemath{1}{
    }
    S_{C} = S + \phi_{C}(S) = S + \sigma \left( \frac{SW_{Q}(CW_{K})^{T}}{\sqrt{d}} \right) C W_{V},
\end{equation}
where $\sigma$ is a softmax activation, \(W_{Q}\), \(W_{K}\), \(W_{V}\) are trainable matrices and \(d\) is the common dimension of the vectors \(S\) and \(C\). In our setting, the source feature vectors $S$ are either video $V$ or pseudo-query $Q$ features. We build separate enhancement mechanisms for $V$ and $Q$, \ie, the projection matrices \(W_{Q}\), \(W_{K}\), \(W_{V}\) are not shared between $Q$ and $V$. We elaborate more on the rationale in the appendix.
The enriched video and pseudo-query features are denoted as \(V_{C}\!=\!\phi_{C_{\text{vid}}}(V)\) and \(Q_{C}\!=\!\phi_{C_{\text{pq}}}(Q)\), respectively.

\paragraph{Concept Encoder.}
The concept vectors \(C\) mentioned above are feature representations that internally form the nodes of the commonsense graph, \(G_C\). Accordingly, graph \(G_{C}\) is represented as a matrix, where \(G_{C(i,j)}\) represents the total number of directed relational edges between \(c_{i},c{j} \in C\) that start at \(c_i\) and end at \(c_j\). To encode the commonsense information, we employ Graph Convolutional Networks (GCN) \cite{hammond_wavelets_2011}. The concept encoder is composed of $L$ graph convolution layers, each of which performs a convolution step
\begin{equation}
\scalemath{1}{
    C^{\left(l+1\right)}=\sigma \left( AC^{\left(l\right) }W^{\left( l\right) }\right),
    }
\end{equation}
where $C^{\left(l\right)}$ are node (concept) features and $W^{\left( l\right)}$ trainable weight matrix of layer $l \in [1, L]$, $\sigma$ is a nonlinear activation function, and $A$ is the adjacency matrix obtained by normalizing graph $G_C$ with the degree matrix $D$. Since $G_C$ is a directed graph, normalization can be formulated as $A\!=\!D^{-1}G_{C}$.

\paragraph{Commonsense Information.}
We use ConceptNet \cite{speer_conceptnet_2017}, a popular knowledge graph that provides information spanning various types of relationships such as physical, spatial, behavioral, \etc To ensure that the ConceptNet information utilized is relevant to themes found in the video data, we consider the set of objects available in pseudo-queries and include the top-$k$ most frequently occurring objects to be the seed concept set \(C\). We extract the  ConceptNet subgraph that includes all edges incident between the concepts in \(C\). 
We filter the edge types based on a pre-determined relation set \(R\), which is compiled to involve relations that are relevant to the nature of the video localization task, \eg, spatial (\textit{AtLocation}, \etc) and temporal (\textit{HasSubevent}, \etc) relations are useful for video understanding, while \textit{RelatedTo} and \textit{Synonym} are fairly generic relations that add little information to the localization task. Table \ref{tab:relations} shows the relations included in \(G_C\).

\paragraph{Cross-Modal Interaction Module.} The commonsense enriched video and query features, \(V_{C}\) and \(Q_{C}\), are fused with a multi-modal cross-attention mechanism. We employ a two-step fusion process. First, Query-guided Video Attention (QVA) is applied to attend over video $V_C$, and Video-guided Query Attention (VQA) attends over query $Q_C$ guided by video $V_C$, resulting in updated features $V_C'$ and $Q_C'$, respectively. Both QVA and VQA utilize Attention Dynamic Filters~\cite{rodriguez_proposal-free_2020} that adaptively modify video features, dynamically adjusting them in response to the query, and vice versa. Next, the attended features are fused using a cross-attention mechanism over $V_C'$ guided by $Q_C'$, resulting in localized video features $V_{C_{\text{loc}}}$.

\paragraph{Temporal Regression Module.}
The final step involves a regression layer that approximates $\hat{V}_{\text{span}}$. We employ attention-guided temporal regression to estimate the span of the target video moment. To find important temporal segments relevant to the query, the fused features $V_{C_{\text{loc}}}$ are temporally attended based on the query features to obtain $V_{\text{ta}}$. Then, the span boundaries are localized using a regressor implemented as a Multi-Layer Perceptron (MLP).

\begin{align}
{o}_i = \sigma\left({W}_{1} V_{C_{\text{loc}_i}} + {b}_{{1}}\right) \\
V_{\text{ta}} = \sum_{i=1}^{T} o_i V_{C_{\text{loc}_{i}}} \\
[\hat{t}_s, \hat{t}_e] = {W}_2 {V}_{\text{ta}} + {b}_{2}.
\end{align}
Here, ${W}_{1}$ and ${b}_1$ are the weight matrix and bias vector of the temporal attention MLP, $\sigma$ represents the sigmoid activation function, $V_{C_{\text{loc}_i}}$ stands for the encoded localized video features, ${V}_{\text{ta}}$ represents the temporally attended video features, ${W}_2$ and ${b}_2$ denote the weight matrix and bias vector of the regression MLP, and $[\hat{t}_s, \hat{t}_e]$ correspond to the start and end timestamps of the predicted video span $\hat{V}_{\text{span}}$.

\begin{table}[t!]
\centering
\resizebox{\linewidth}{!}{
\begin{tabular}{ll}
\toprule
\textbf{Category} & \textbf{Relations}                                                                                         \\ \toprule
Spatial           & AtLocation, LocatedNear                                                                                    \\ \midrule
Temporal          & \begin{tabular}[c]{@{}l@{}}HasSubevent, HasFirstSubevent, HasLastSubevent, HasPrerequisite\end{tabular} \\ \midrule
Functional        & UsedFor                                                                                                    \\ \midrule
Causal            & Causes                                                                                                     \\ \midrule
Motivation        & MotivatedByGoal,  ObstructedBy                                                                             \\ \midrule
Other             & CreatedBy, MadeOf                                                                                          \\ \midrule
Physical          & \begin{tabular}[c]{@{}l@{}}HasA, HasProperty, Antonym, SimilarTo\end{tabular}                      
\\ \bottomrule
\end{tabular}
}

\caption{Relations in the Commonsense Enhancement Module (CEM) grouped by categories.}
\label{tab:relations}

\end{table}
\subsection{Training and Inference}
The training objective is 
$\mathcal{L}_{loc} = \mathcal{L}_{treg}+\lambda \mathcal{L}_{ta},$ where \(\lambda\) is a balancing hyperparameter, \(\mathcal{L}_{ta}\) is a temporal attention guided loss and \(\mathcal{L}_{treg}\) is the regression loss.  The temporal attention-guided loss is defined as
\begin{equation}
\label{tatt}
\mathcal{L}_{ta} = \frac{\sum^{T}_{i=1}g_{i}\log \left( a_{i}\right)}{\sum^{T}_{i=1}g_{i}},
\end{equation}
where \(a_{i}\) is the attention weight for video frame \(v_{i}\) and \(g_{i}\) is the attention mask for \(v_{i}\), that is assigned to \(1\) if \(v_{i}\) is inside the target video segment, and \(0\) otherwise. 
This objective encourages the model to produce higher attention weights for video segments that are relevant to the query. 
On the other hand, \(\mathcal{L}_{treg}\) dictates the video span boundary regression and is the sum of smooth $\ell_1$ distances between start and end timestamps of the ground truth and predicted spans, \ie,
\begin{equation}
\label{treg}
\mathcal{L}_{treg} = \text{smooth}{\ell_1}(t_{s}, \hat{t}_{s}) + \text{smooth}{\ell_1}(t_{e}, \hat{t}_{e}).
\end{equation}
Here, $t_{s}$ and ${t}_{e}$ represent the ground truth start and end timestamps and $\hat{t}_{s}$ and $\hat{t}_{e}$ the predicted start and end timestamps, respectively.
The integration of a smoothing mechanism enhances training stability and improves the model's ability to handle outliers. Finally, during inference, we employ an off-the-shelf part-of-speech tagger to extract nouns from the text input query and feed them as query input to the trained \modelname video localizer.
\section{Assessment}
\label{sec:assessment}
\subsection{Experimental Setup}
We implement our PCDNet in PyTorch \cite{paszke2019pytorch} and train it for 300 epochs with the batch size of 32 on two NVIDIA GeForce RTX 3090 GPUs. We use stochastic gradient descent (SGD) \cite{amari1993backpropagation} with a momentum of 0.937 and a weight decay of $5 \times 10 ^{-4}$ during training. The initial learning rate is set to 0.01 and decayed to 0.001 using a cosine annealing schedule. We initialize PCDNet randomly and load the weights of CSPDarknet53 \cite{wang2020cspnet} pre-trained on ImageNet \cite{imagenet_cvpr09} for the encoder part. To increase the diversity and complexity of the training samples, we apply data augmentations including random cropping, random flipping, and mosaic \cite{redmon2018yolov3}. We use the evaluation metrics of Microsoft COCO \cite{lin2014microsoft} for validation.

\begin{table}[ht]
\caption{Quantitative comparison against state-of-the-art polarization-based detectors ($\star$), single-stage detectors ($\dag$), two-stage detectors ($\ddag$), anchor-based detectors ($\triangle$), anchor-free detectors ($\circ$), and self-supervised method ($\S$).}
\small
\centering
\renewcommand\arraystretch{0.9}
\setlength{\tabcolsep}{2.6pt}
\begin{tabular}{lccccc}
\hline\hline
Methods	&	Pub'Year	&	Backbone	&	AP	&	AP50	&	AP75	\\
\hline
Faster R-CNN$^{\ddag\triangle}$ 	&	NeurIPS'15	&	Res50	&	44.8	&	75.4	&	45.4	\\
SSD$^{\dag\circ}$ 	&	ECCV'16	&	VGG16	&	25.5	&	52.6	&	22.6	\\
Cascade R-CNN$^{\ddag\triangle}$ 	&	CVPR'18	&	Res50	&	45.8	&	73.2	&	47.8	\\
CornerNet$^{\dag\circ}$ 	&	ECCV'18	&	Res50	&	19.8	&	47.4	&	29.6	\\
P-SSD I$^{\star\dag\circ}$ 	&	ITSC'19	&	VGG16	&	25.9 	&	53.1	&	22.7	\\
P-SSD S$^{\star\dag\circ}$ 	&	ITSC'19	&	VGG16	&	23.0 	&	48.9	&	20.1	\\
FCOS$^{\dag\circ}$ 	&	ICCV'19	&	Res50	&	23.1	&	50.9	&	18.4	\\
DH R-CNN$^{\ddag\triangle}$ 	&	CVPR'20	&	Res50	&	32.7	&	65.3	&	28.2	\\
Dynamic R-CNN$^{\ddag\triangle}$ 	&	ECCV'20	&	Res50	&	46.2	&	74.2	&	48.0	\\
EfficientDet$^{\ddag\triangle}$ 	&	CVPR'20	&	D3	&	45.3	&	73.0	&	46.3	\\
VarifocalNet$^{\dag\circ}$  & CVPR'21 & Res50 & 44.2 &	73.5 &	44.4	\\
D-DETR$^{\dag\circ}$ 	&	ICLR'21	&	Res50	&	43.8	&	74.9	&	44.3	\\
DDOD$^{\dag\circ}$ 	&	MM'21	&	Res50	&	43.5	&	73.0	&	43.3	\\
TOOD$^{\dag\triangle}$ 	&	ICCV'21	&	Res50	&	44.3	&	74.3	&	44.6	\\
YOLOX$^{\dag\circ}$ 	&	arXiv'21	&	YOLOX-l	&	54.3	&	82.5	&	56.7	\\
YOLOv7$^{\dag\triangle}$	&	arXiv'22	&	Dark53	&	57.6	&	84.3	&	60.3	\\
RTMDet$^{\dag\circ}$ 	&	arXiv'22	&	RTMDet-l	&	53.9	&	81.4	&	56.7	\\
DINO$^{\dag\circ\S}$ 	&	ICLR'22	&	Res50	&	52.7	&	81.8	&	54.8	\\
YOLOv8$^{\dag\circ}$ 	&	-'23	&	YOLOv8-l	&	56.8	&	83.6	&	59.0	\\
\hline
\textbf{PCDNet$^\star$}	&	\textbf{Ours}	&	Dark53	&	\textbf{58.5}	&	\textbf{85.2}	&	\textbf{61.5}	\\
\hline\hline
\end{tabular}
\label{tab:comparison}
\end{table}

\begin{figure*}[htp]
    \centering
    \begin{center}
        % \includegraphics[width=\linewidth]{figure/comparison.pdf}
        \includegraphics[width=\linewidth,height=10.5cm]{figure/comparison.pdf}
    \end{center}
    \caption{Qualitative comparison of PCDNet against state-of-the-art detectors retrained on RGB-P Car dataset.} 
    \label{fig:comparison}
\end{figure*}

\subsection{Qualitative and Quantitative Evaluation}
We extensively compare our PCDNet with 19 state-of-the-art methods by retraining and testing all methods on the RGB-P Car dataset using their original settings. The compared methods include two-stage detectors such as EfficientDet \cite{tan2020efficientdet} and the R-CNN family \cite{Ren_2017, Cai_2019, zhang2020dynamic}, and one-stage detectors such as SSD \cite{liu2016ssd}, and YOLO family \cite{ge2021yolox, wang2022yolov7, ultralytics2023yolov8}. These methods also comprise anchor-based methods such as the R-CNN family and YOLOv7 \cite{wang2022yolov7}, and anchor-free methods such as CornerNet \cite{law2018cornernet}, VarifocalNet \cite{zhang2021varifocalnet}, and YOLOv8 \cite{ultralytics2023yolov8}. Some detectors use traditional convolutional networks such as FCOS \cite{tian2019fcos} and RTMDet \cite{lyu2022rtmdet} while others use transformer structures, such as DeformableDETR \cite{zhu2020deformable} and DINO \cite{zhang2022dino} that employs self-supervised learning. We also include the P-SSD \cite{blin2019road} that utilizes polarization information. The quantitative evaluation results are reported in Tab. \ref{tab:comparison}. We can see that our method outperforms all competing state-of-the-art methods. 

Fig. \ref{fig:comparison} further qualitatively demonstrates the benefits of our method: a) in poorly lit indoor parking lots, distinguishing black cars behind pillars is extremely challenging (the first two rows). The compared methods tend to conflate the shadow and the black car (\textit{i.e.}, merging cars on either side of the pillar into a single entity or treating partial views of the car as one object) while our PCDNet can handle such ambiguities; b) in the third example, all methods except our PCDNet fail to detect a partially visible car obstructed by another car or misplace it with the previous car; c) in the fourth example, RGB-based methods wrongly identify distant pedestrians as cars, but our PCDNet method can effectively eliminate such interference with the help of polarization cues; d) the fifth and sixth examples depict black cars in an outdoor parking lot at night which are very hard to be distinguished in the RGB image. Despite the enhancement through ZeroDCE \cite{guo2020zero}, the sixth example remains unclear. By contrast, polarization imaging is robust to low light conditions, enabling our robust car detector PCDNet; and e) the last row shows a virtual car reflected in a mirror located at the upper-left corner of the image. The mirrored virtual car and the rest of the mirror regions exhibit similar and smooth AoLP, providing useful cues for PCDNet to recognize this region as background. 


\subsection{Ablation Study}
\textbf{Impact of Spectral Intensity and Polarization Cues.} We conduct a series of ablation experiments to demonstrate the effects of spectral intensity and polarization cues on car detection (Tab. \ref{tab:abl_input}).
The results show that: a) combining different forms of polarization cues with RGB as the input of PCDNet can improve the car detection accuracy (\textit{C}, \textit{D}, \textit{F}, \textit{G}, \textit{K} and \textit{L} are higher than \textit{B}); b) DoLP cues have a greater impact than AoLP cues (\textit{D}, \textit{J} and \textit{L} are better than \textit{C}, \textit{I} and \textit{K}, respectively); c) stacking AoLP and DoLP on RGB in the channel dimension does not boost performance (\textit{E} is slightly lower than \textit{B}), possibly because the characteristic gap between different modalities hinders effective features extraction; d) spectral intensity and polarization are more beneficial than monochromatic intensity and polarization for car detection (comparing paired \textit{B} and \textit{H}, \textit{C} and \textit{K}, \textit{D} and \textit{L}, \textit{I} and \textit{K}, \textit{J} and \textit{L}); e) enhancing RGB image via ZeroDCE \cite{guo2020zero} is less effective than introducing polarization (\textit{M} performs worse than \textit{C}-\textit{G}, \textit{K} and \textit{L}).
Fig. \ref{fig:abl_input} provides visual support for these observations.

\begin{table}[t]
\small
\centering
\caption{Quantitative comparisons of ablation with different inputs. ``stacked I'' denotes the stacked intensity measurements with a linear polarization angle of 0$^{\circ}$, 45$^{\circ}$ and 135$^{\circ}$ and ``stacked S'' refers to the stacked Stokes elements S0, S1 and S2 \cite{blin2019road}.}
\begin{tabular}{clccc}
\hline\hline
	&	PCDNet Input	&	AP	&	AP50	&	AP75	\\
 \hline
\textit{A}	&	RGB, AoLP and DoLP (original)	&	58.5 	&	85.2 	&	61.5 	\\
\hline
\textit{B}	&	RGB only	&	57.6 	&	84.3 	&	60.2 	\\
\textit{C}	&	RGB and AoLP	&	58.0 	&	84.6 	&	60.7 	\\
\textit{D}	&	RGB and DoLP	&	58.3 	&	85.4 	&	61.1 	\\
\textit{E}	&	stacked RGB, AoLP and DoLP	&	57.5 	&	84.3 	&	59.9 	\\
\textit{F}	&	RGB and stacked I	&	58.0 	&	84.1 	&	61.0 	\\
\textit{G}	&	RGB and stacked S	&	57.8 	&	84.8 	&	60.4 	\\
\textit{H}	&	Gray only	&	57.4 	&	84.3 	&	60.0 	\\
\textit{I}  &   Gray and mono AoLP & 57.5 & 84.5 & 60.5 \\
\textit{J}  &   Gray and mono DoLP & 57.6 & 84.9 & 60.1 \\
\textit{K}	&	RGB and mono AoLP	&	57.9 	&	84.6 	&	60.5 	\\
\textit{L}	&	RGB and mono DoLP	&	58.2 	&	84.9 	&	60.6 	\\
\textit{M}  &   Enhanced RGB & 57.4 & 84.0 & 60.0 \\
\hline\hline
\end{tabular}
\label{tab:abl_input}
\end{table}

\begin{figure}[t]
    \centering
    \includegraphics[width=1\linewidth]{figure/abl_input.pdf}
    \caption{Qualitative comparison of ablation with different inputs. The model with RGB intensity only is susceptible to interference from ghost car caused by water on the road.}
    \label{fig:abl_input}
\end{figure}

\textbf{Influence of PCDNet Components.}
First, we investigate the performance of different strategies for fusing AoLP and DoLP inputs. From Tab. \ref{tab:abl_module}(\textit{A}-\textit{D}), we observe that our PI module is more effective than the simple fusion methods including concatenation, addition and element-wise multiplication.
Second, by removing MP module \ref{tab:abl_module}(\textit{E}) from the original PCDNet (A), the detection performance declines. This demonstrates that exploring the polarized material features of cars across all learning samples is useful. We also explore the influence of applying MSP and MCP on different levels of features. The results in Tab. \ref{tab:abl_module}(\textit{A},\textit{F}-\textit{G}) show that applying MSP on shallower features and MCP on deeper features can yield better performance.
Finally, we validate the effectiveness of CDDQ module.
Removing the CDDQ module (\textit{I}) from PCDNet (\textit{A}), which causes the feature extraction processes of the RGB and polarization to be independent from each other, leads to the performance drop. We also demonstrate the benefits of the CWDA and SDMD in the CDDQ module by removing either of them (\textit{J} and \textit{K}). 

\begin{table}[t]
\small
\centering
\caption{Quantitative comparisons of ablation with different modules demonstrate that all component of PCDNet contributes to the overall performance. We used sequences of three letters separated by '-' and enclosed in parentheses to represent different combinations of MSP and MCP.}
\begin{tabular}{clccc}
\hline\hline
	&	Ablation	&	AP	&	AP50	&	AP75	\\
 \hline
\textit{A}	&	PCDNet (original)	&	58.5 	&	85.2 	&	61.5 	\\
\hline
\textit{B}	&	Input RGB and [AoLP DoLP]	&	58.2 	&	85.4 	&	60.9 	\\
\textit{C}	&	Input RGB and AoLP+DoLP	&	58.1 	&	84.8 	&	60.5 	\\
\textit{D}	&	Input RGB and AoLP*DoLP	&	58.1 	&	84.8 	&	60.5 	\\
\hline
\textit{E}	&	A \textit{w/o} MP	&	56.9 	&	84.2 	&	59.2 	\\
\textit{F}	&	A \textit{w/} M(S-S-S)P	&	58.2 	&	85.2 	&	60.8 	\\
\textit{G}	&	A \textit{w/} M(S-C-C)P	&	58.2 	&	85.0 	&	60.9 	\\
\textit{H}	&	A \textit{w/} M(C-C-C)P	&	58.1 	&	85.0 	&	61.1 	\\
\hline
\textit{I}	&	A \textit{w/o} CDDQ	&	58.0 	&	84.7 	&	60.8 	\\
\textit{J}	&	A \textit{w/o} SDMD	&	58.2 	&	85.2 	&	60.8 	\\
\textit{K}	&	A \textit{w/o} CWDA	&	58.3 	&	85.1 	&	61.1 	\\
\hline\hline
\end{tabular}
\label{tab:abl_module}
\end{table}

\subsection{Limitations}

When both the RGB intensity and the polarization measurement yield weak car signals, our method's effectiveness declines. Specifically, in low-light scenarios, when a car approaches on an unlit road, the strong light from its headlights can create a ``hole'' in both the RGB and polarization and obscure the entire car. We illustrate such an example in Fig. \ref{fig:failure} where the extreme HDR and heavy motion blur in the captured image limit its depiction of both RGB and polarization. In these challenging scenarios, prior RGB-based methods and even human vision are powerless.

\begin{figure}[t]
    \centering
    \includegraphics[width=1\linewidth]{figure/failure.pdf}
    \caption{PCDNet has limited ability to handle extreme HDR or heavy motion blur cases.}
    \label{fig:failure}
\end{figure}

\section{Related Work}
\label{sec:related-work}

\paragraph{Datasets.}
The lack of cross-file and cross-project (e.g. dependencies) information is a general issue in current evaluation datasets for code.
In terms of code completion, common choices are Py150 \citep{raychev2016probabilistic} for Python and Github Java Corpus \citep{allamanis2013mining} for Java. Both datasets are constructed at file level, where source files are isolated from their project and dependencies and no consideration of project separation is taken in constructing training and test sets.
\citet{lu2022reacc} constructed a code completion dataset from CodeNet \citep{puri2021project}, which contains coding problems and solutions from online judge websites and also lacks project context. 
\citet{clement2021long} presented a real-world Python method generation task based on CodeSearchNet \citep{husain2019codesearchnet} but the auxiliary information they extract still comes from within a local file. 
\citet{svyatkovskiy2021fast} constructed a completion dataset based on top Python repositories on GitHub and released the URLs for these repositories. 
However, those repositories are not write-protected and can change over time. Besides, setting up the dependency environments at scale for further analysis is not easy. 
Both make their dataset difficult to reproduce.
In the contrast, we release the code and the dependencies for the projects to ensure reproducibility.
Apart from code completion, datasets for other code tasks such as Cloze test \citep[e.g.][]{feng2020codebert}, code refinement \citep[e.g.][]{tufano2019empirical, yasunaga2021break, haque2022fixeval}, and generating code from text descriptions \citep[e.g.][]{chen2021evaluating, hendrycks2021measuring, austin2021program}, are often small and mostly without project-level code context. 
Beyond-local information is beneficial for programmers to solve programming tasks in real-world settings. The lack of such information in the current dataset would restrict the progress into high-level semantic understanding and reasoning in the code domain.


\paragraph{Code language models.}
Encouraged by the success of pretrained language models in natural language processing \citep{devlin2019bert, liu2019roberta, lewis2019bart, raffel2020exploring} and the promise of naturalness in code \citep{hindle2016naturalness, allamanis2018survey}, we have seen rising adaptations of language models for code. For example, CuBERT \citep{kanade2020learning} and CodeBERT \citep{feng2020codebert} are pretrained based on masked language modeling. GPT-C \citep{svyatkovskiy2020intellicode} and CodeGPT \citep{lu2021codexglue} are both pretrained based on unidirectional language modeling. PLBART \citep{ahmad2021unified} and CodeT5 \citep{wang2021codet5} are pretrained encoder-decoder structures which adopts denoising objectives and can support code understanding and code generation. UnixCoder \citep{guo2022unixcoder} combines the above three pretraining objectives for a unified pretrained model. 




\paragraph{Code completion.}
Code completion is an essential feature for modern IDEs and an important topic for code intelligence. 
In recent years, deep neural networks \citep{liu2016neural, li2018code, alon2020structural, liu2020multi, kim2021code}, especially pretrained language models \citep{svyatkovskiy2020intellicode, lu2021codexglue} become the mainstream solution to this task. 
Still, incorporating additional information proved beneficial.
One popular choice is abstract syntax tree, e.g. \citet{kim2021code, peng2021could, guo2022unixcoder}. 
However, \citet{lopez2022ast} suggested that pretrained code language models may have already encoded the syntax.  
Other proposals seek to use data flow graph, control graph, and various graph relations, e.g. \citet{guo2020graphcodebert, hellendoorn2019global}.
However, information is still restricted from a single file.
We instead try to enhance the model with out-of-file information, similar to what is accessible in a development environment.

For project-level analyzer induced information, \citet{svyatkovskiy2021fast} described a way to use a static analyzer to refine completion candidates from neural methods.
\citet{weyssow2020combining} considered leveraging the project-wise contexts via embeddings for better function call completion performance.
Other than code completion, project-level information has been utilized for methods name prediction~\citep{wang2021lightweight} and generating code from text descriptions~\citep{lyu2021embedding}.
However, none of them tested their approaches with pretrained code language models. 
In terms of incorporating additional context through concatenation,
\citet{clement2021long} reported improvements from prioritize certain parts of in-file context.
Recently, \citet{lu2022reacc} proposed to enhance code language models by concatenating similar code fragments retrieved by a neural network. Despite the general similarity, we 1) use a simple lightweight way to retrieve auxiliary information instead of training a heavy retriever; 2) do not restrict ourselves on similar code fragments and show that dissimilar code fragments (function implementation) can be helpful; 3) explore task-specific fine-tuning with retrieved information for better completion.





% \vspace{-1em}
\section{Conclusions}
% \vspace{-1em}
In this paper, we introduced a benchmark task for commonsense reasoning that aims at uncovering unspoken intents that humans can easily uncover in a given statement by making presumptions supported by their common sense. In order to solve this task, we propose
CORGI (COmmon-sense ReasoninG by Instruction),  a neuro-symbolic theorem prover that performs commonsense reasoning by initiating a conversation with a user. CORGI has access to a small knowledge base of commonsense facts and completes it as she interacts with the user. We further conduct a user study that indicates the possibility of using conversational interactions with humans for evoking commonsense knowledge and verifies the effectiveness of our proposed theorem prover.
% We defined common-sense reasoning as the process of finding a chain of reasoning in a logic program given an if/then/because statement. We showed that obtaining the because statement is crucial in extracting a relevant chain of reasoning given an if/then statement. Moreover, we introduced a soft backward chaining algorithm that allows us to combat variations in natural language by learning embeddings for the facts and rules in the knowledge base. This algorithm combines symbolic AI with neural approaches allowing us to bridge a gap between symbolic AI and the recent advances in deep learning.

\bibliographystyle{aaai}
\bibliography{main}

\end{document}


\end{document}