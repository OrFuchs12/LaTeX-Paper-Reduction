%File: formatting-instructions-latex-2024.tex
%release 2024.0
\documentclass[letterpaper]{article} % DO NOT CHANGE THIS

\usepackage{aaai24}  % DO NOT CHANGE THIS
\usepackage{times}  % DO NOT CHANGE THIS
\usepackage{helvet}  % DO NOT CHANGE THIS
\usepackage{courier}  % DO NOT CHANGE THIS
\usepackage[hyphens]{url}  % DO NOT CHANGE THIS
\usepackage{graphicx} % DO NOT CHANGE THIS
\urlstyle{rm} % DO NOT CHANGE THIS
\def\UrlFont{\rm}  % DO NOT CHANGE THIS
\usepackage{natbib}  % DO NOT CHANGE THIS AND DO NOT ADD ANY OPTIONS TO IT
\usepackage{caption} % DO NOT CHANGE THIS AND DO NOT ADD ANY OPTIONS TO IT
\frenchspacing  % DO NOT CHANGE THIS
\setlength{\pdfpagewidth}{8.5in}  % DO NOT CHANGE THIS
\setlength{\pdfpageheight}{11in}  % DO NOT CHANGE THIS
%
% These are recommended to typeset algorithms but not required. See the subsubsection on algorithms. Remove them if you don't have algorithms in your paper.
\usepackage{algorithm}
\usepackage{algorithmic}
\usepackage{multirow}
\usepackage{booktabs}

%
% These are are recommended to typeset listings but not required. See the subsubsection on listing. Remove this block if you don't have listings in your paper.
\usepackage{newfloat}
\usepackage{listings}
\DeclareCaptionStyle{ruled}{labelfont=normalfont,labelsep=colon,strut=off} % DO NOT CHANGE THIS
\lstset{%
	basicstyle={\footnotesize\ttfamily},% footnotesize acceptable for monospace
	numbers=left,numberstyle=\footnotesize,xleftmargin=2em,% show line numbers, remove this entire line if you don't want the numbers.
	aboveskip=0pt,belowskip=0pt,%
	showstringspaces=false,tabsize=2,breaklines=true}
\floatstyle{ruled}
\newfloat{listing}{tb}{lst}{}
\floatname{listing}{Listing}
%
% Keep the \pdfinfo as shown here. There's no need
% for you to add the /Title and /Author tags.

% DISALLOWED PACKAGES
% \usepackage{authblk} -- This package is specifically forbidden
% \usepackage{balance} -- This package is specifically forbidden
% \usepackage{color (if used in text)
% \usepackage{CJK} -- This package is specifically forbidden
% \usepackage{float} -- This package is specifically forbidden
% \usepackage{flushend} -- This package is specifically forbidden
% \usepackage{fontenc} -- This package is specifically forbidden
% \usepackage{fullpage} -- This package is specifically forbidden
% \usepackage{geometry} -- This package is specifically forbidden
% \usepackage{grffile} -- This package is specifically forbidden
% \usepackage{hyperref} -- This package is specifically forbidden
% \usepackage{navigator} -- This package is specifically forbidden
% (or any other package that embeds links such as navigator or hyperref)
% \indentfirst} -- This package is specifically forbidden
% \layout} -- This package is specifically forbidden
% \multicol} -- This package is specifically forbidden
% \nameref} -- This package is specifically forbidden
% \usepackage{savetrees} -- This package is specifically forbidden
% \usepackage{setspace} -- This package is specifically forbidden
% \usepackage{stfloats} -- This package is specifically forbidden
% \usepackage{tabu} -- This package is specifically forbidden
% \usepackage{titlesec} -- This package is specifically forbidden
% \usepackage{tocbibind} -- This package is specifically forbidden
% \usepackage{ulem} -- This package is specifically forbidden
% \usepackage{wrapfig} -- This package is specifically forbidden
% DISALLOWED COMMANDS
% \nocopyright -- Your paper will not be published if you use this command
% \addtolength -- This command may not be used
% \balance -- This command may not be used
% \baselinestretch -- Your paper will not be published if you use this command
% \clearpage -- No page breaks of any kind may be used for the final version of your paper
% \columnsep -- This command may not be used
% % \newpage -- No page breaks of any kind may be used for the final version of your paper
% \pagebreak -- No page breaks of any kind may be used for the final version of your paperr
% \pagestyle -- This command may not be used
% \tiny -- This is not an acceptable font size.
% %\vspace{- -- No negative value may be used in proximity of a caption, figure, table, section, subsection, subsubsection, or reference
% \vskip{- -- No negative value may be used to alter spacing above or below a caption, figure, table, section, subsection, subsubsection, or reference

\setcounter{secnumdepth}{0} %May be changed to 1 or 2 if section numbers are desired.

% The file aaai24.sty is the style file for AAAI Press
% proceedings, working notes, and technical reports.
%

\usepackage{algorithm}
\usepackage{algorithmic}
\usepackage{pifont}% http://ctan.org/pkg/pifont
\usepackage{xcolor,colortbl}
\usepackage{amsmath, amsfonts, amssymb}
\usepackage{bm}

\usepackage{xspace}
\makeatletter
\DeclareRobustCommand\onedot{\futurelet\@let@token\@onedot}
\def\@onedot{\ifx\@let@token.\else.\null\fi\xspace}
\DeclareMathOperator*{\fe}{\mathbb{E}}
\def\eg{\emph{e.g}\onedot} \def\Eg{\emph{E.g}\onedot}
\def\ie{\emph{i.e}\onedot} \def\Ie{\emph{I.e}\onedot}
\def\cf{\emph{c.f}\onedot} \def\Cf{\emph{C.f}\onedot}
\def\etc{\emph{etc}\onedot} \def\vs{\emph{vs}\onedot}
\def\wrt{w.r.t\onedot} \def\dof{d.o.f\onedot}
\def\etal{\emph{et al}\onedot}
\makeatother



\newcommand{\Loss}{\mathcal{L}}
\newcommand{\domw}{\mathcal{W}}
\newcommand{\domx}{\mathcal{X}}
\newcommand{\domy}{\mathcal{Y}}
\newcommand{\cmark}{\ding{51}}%
\newcommand{\xmark}{\ding{55}}%
\newcommand{\cmarkcolor}{\textcolor{teal}{\ding{51}}}%
\newcommand{\xmarkcolor}{\textcolor{red}{\ding{55}}}%

\definecolor{gray0}{gray}{0.95}
\definecolor{gray1}{gray}{0.85}
\definecolor{gray2}{gray}{0.75}
\newcommand\blankfootnote[1]{%
  \let\thefootnote\relax\footnotetext{#1}%
  \let\thefootnote\svthefootnote%
}
% \usepackage{hyperref}
% Title

% Your title must be in mixed case, not sentence case.
% That means all verbs (including short verbs like be, is, using,and go),
% nouns, adverbs, adjectives should be capitalized, including both words in hyphenated terms, while
% articles, conjunctions, and prepositions are lower case unless they
% directly follow a colon or long dash
\title{Federated Learning via Input-Output Collaborative Distillation}
\author{
    %Authors
    % All authors must be in the same font size and format.
    Xuan Gong\textsuperscript{\rm 1,\rm 3 \equalcontrib},
    Shanglin Li\textsuperscript{\rm 2 \equalcontrib},
    Yuxiang Bao\textsuperscript{\rm 2 \equalcontrib},
    Barry Yao\textsuperscript{\rm 1,\rm 4},
    Yawen Huang\textsuperscript{\rm 5},
    Ziyan Wu\textsuperscript{\rm 6},
    Baochang Zhang\textsuperscript{\rm 2,\rm 7,\rm 8,\rm 9 \dag},
    Yefeng Zheng\textsuperscript{\rm 5},
    David Doermann\textsuperscript{\rm 1 \dag}
}
\affiliations{
    %Afiliations
    \textsuperscript{\rm 1} University at Buffalo, Buffalo, NY, USA
    \textsuperscript{\rm 2} Institute of Artificial Intelligence, Beihang University, Beijing, China \\
    \textsuperscript{\rm 3} Harvard Medical School, Boston, MA, USA
    \textsuperscript{\rm 4} Virginia Tech, Blacksburg, VA, USA  \\
    \textsuperscript{\rm 5} Jarvis Research Center, Tencent YouTu Lab, Shenzhen, China
    \textsuperscript{\rm 6} United Imaging Intelligence, Burlington, MA, USA \\
    \textsuperscript{\rm 7} Hangzhou Research Institute, Beihang University, Hangzhou, China
    \textsuperscript{\rm 8} Zhongguancun Laboratory, Beijing, China \\
    \textsuperscript{\rm 9} Nanchang Institute of Technology, Nanchang, China

    xuangong@buffalo.edu, shanglin@buaa.edu.cn, yxbao@buaa.edu.cn, barryyao@vt.edu, yawenhuang@tencent.com, ziyan.wu@uii-ai.com, bczhang@buaa.edu.cn, yefengzheng@tencent.com, doermann@buffalo.edu
    % If you have multiple authors and multiple affiliations
    % use superscripts in text and roman font to identify them.
    % For example,

    % Sunil Issar\textsuperscript{\rm 2},
    % J. Scott Penberthy\textsuperscript{\rm 3},
    % George Ferguson\textsuperscript{\rm 4},
    % Hans Guesgen\textsuperscript{\rm 5}
    % Note that the comma should be placed after the superscript

    % 1900 Embarcadero Road, Suite 101\\
    % Palo Alto, California 94303-3310 USA\\
    % email address must be in roman text type, not monospace or sans serif
    % proceedings-questions@aaai.org
%
% See more examples next
}

%Example, Single Author, ->> remove \iffalse,\fi and place them surrounding AAAI title to use it
\iffalse
\title{My Publication Title --- Single Author}
\author {
    Author Name
}
\affiliations{
    Affiliation\\
    Affiliation Line 2\\
    name@example.com
}
\fi

\iffalse
%Example, Multiple Authors, ->> remove \iffalse,\fi and place them surrounding AAAI title to use it
\title{My Publication Title --- Multiple Authors}
\author {
    % Authors
    First Author Name\textsuperscript{\rm 1,\rm 2},
    Second Author Name\textsuperscript{\rm 2},
    Third Author Name\textsuperscript{\rm 1}
}
\affiliations {
    % Affiliations
    \textsuperscript{\rm 1}Affiliation 1\\
    \textsuperscript{\rm 2}Affiliation 2\\
    firstAuthor@affiliation1.com, secondAuthor@affilation2.com, thirdAuthor@affiliation1.com
}
\fi


% REMOVE THIS: bibentry
% This is only needed to show inline citations in the guidelines document. You should not need it and can safely delete it.
% \usepackage{bibentry}
% END REMOVE bibentry

\begin{document}
\maketitle

\begin{abstract}
   Federated learning (FL) is a machine learning paradigm in which distributed local nodes collaboratively train a central model without sharing individually held private data. Existing FL methods either iteratively share local model parameters or deploy co-distillation. However, the former is highly susceptible to private data leakage, and the latter design relies on the prerequisites of task-relevant real data. Instead, we propose a data-free FL framework based on local-to-central collaborative distillation with direct input and output space exploitation. Our design eliminates any requirement of recursive local parameter exchange or auxiliary task-relevant data to transfer knowledge, thereby giving direct privacy control to local users. In particular, to cope with the inherent data heterogeneity across locals, our technique learns to distill input on which each local model produces consensual yet unique results to represent each expertise. Our proposed FL framework achieves notable privacy-utility trade-offs with extensive experiments on image classification and segmentation tasks under various real-world heterogeneous federated learning settings on both natural and medical images. Code is available at  \url{https://github.com/lsl001006/FedIOD}.
\end{abstract}
\renewcommand{\thefootnote}{\fnsymbol{footnote}}

\begin{figure}[h]
\centering
\includegraphics[width=\linewidth]{fig/fig1.pdf}
%%%removedVspace
\caption{ (a) Parameter-based methods recursively exchange model parameters between each local and server-side \cite{mcmahan2017communication,li2018federated, karimireddy2019scaffold}, which is highly vulnerable to a security attack \cite{zhu2019deep}.
(b) Distillation-based methods utilize auxiliary task-dependent real data to conduct co-distillation between each local and the central server \cite{li2019fedmd, gong2022preserving}.
(c) Our FL method conducts one-way distillation from locals to the server with generated data, eliminating the prerequisite of additional data required by typical distillation, and the security weaknesses of white-box attacks caused by recursive parameter exchange.
}
\label{fig1}
%%%removedVspace
\end{figure}

\section{Introduction}
\label{sec:intro}
The recent success of deep learning in various applications can be attributed to data-driven algorithms typically trained in a centralized fashion, \ie, computational units and data samples residing on the same server. Real-world scenarios, however, tend to disperse this wealth of data throughout separate locations and governed by diverse entities. Due to privacy regulations and communication limitations, collecting all data in one location for centralized training is often impractical, especially true for mobile vision and medical applications.

Accordingly, federated learning (FL) does not necessarily need all data samples to be centralized; instead, it relies on model fusion/distillation techniques to train one centralized model in a decentralized fashion. Privacy is a critical consideration, and it is vital to prevent private data leakage. Another challenge is data heterogeneity among locals, as distributed data centers tend to collect data in different settings.

Most federated learning methods are based on the recursive exchange of local model parameters during the training process~\cite{mcmahan2017communication, li2018federated, karimireddy2019scaffold}. Each local node uploads its model parameters after a particular time of local training. The central server aggregates the parameters of the local model with different schemes~\cite{wang2020federated, li2019fair, hsu2020federated} and then distributes the aggregated parameters. Each local node receives the latest parameters to update its local model accordingly and continues with the next round of local training.
However, naively employing such iterative parameter exchange suffers from known weaknesses: (1) All participating models must have exactly homogeneous architectures. (2) Iteratively sharing the model parameters opens all internal states of the model to white-box inference attacks, resulting in significant privacy leakage~\cite{chang2019cronus}. Recent works \cite{zhu2019deep, geiping2020inverting} obtain private training data from publicly shared model gradients.





Distillation-based methods are proposed to train the central model with aggregated locally-computed logits~\cite{li2019fedmd, lin2020ensemble, gong2022preserving}, eliminating the requirement of identical network architectures. %Some recently proposed methods \cite{lin2020ensemble, gong2022preserving} provide some relaxations on transfer data, but they still require careful selection of the transfer data according to the prior knowledge of the local task and private data.
However, to transfer knowledge, additional public data are commonly assumed to be accessible and sampled from the same underlying distribution as the privately held local data.
This assumption can be strong in practice and unavoidably exposes private data to stealthy attacks. %For instance, the selected transfer data needs to target the same subject with the same modality as the private data.
Although \cite{zhu2021data, zhang2022fedzkt, zhang2022fine} takes a step further to eliminate the requirement of real data for distillation, iterative model parameter exchange is still essential in these frameworks where knowledge transfer is only an auxiliary module for fine-tuning. As noted above, such parameter exchange is limited by identical model architecture and, more importantly, highly susceptible to privacy leakage.
These methods require such recursive parameter exchange primarily because they mainly focus on the output distillation, leaving the input space under-explored.

\begin{figure*}
\centering
%%%removedVspace
\includegraphics[width=\linewidth]{fig/fig2_new.pdf}
%%%removedVspace
\caption{\textbf{The overall pipeline of the proposed FedIOD.} We conduct distillation in input and output spaces to transfer knowledge from the locally trained task model $T_k$ and the auxiliary discriminator $D_k$ to the central task model $S$. \textbf{Input distillation} optimizes central generator $G$ to generate transferred input on which local models (1) achieve consensus on its semantic clarity,
(2) and simultaneously produce diverse predictions. The latter is to exploit each local's unique expertise under the heterogeneous FL setting.
%The former is for the viability of input on which each local model produces confident output to teach $S$. Moreover, the latter is to exploit each local's unique expertise due to the data heterogeneity of FL.
\textbf{Output distillation} leverages per-input importance for output ensemble knowledge transfer.
}
%%%removedVspace
\label{fig2}
\end{figure*}

In this paper, we propose a new federated learning framework  (FedIOD) that conducts a collaborative knowledge distillation in both the input and output space (as Figure \ref{fig1}).
It is purely based on data-free distillation without any prerequisite of auxiliary real data or locally trained model parameters. Besides, we adopt differential privacy protection on the gradients used to train the generator \cite{torkzadehmahani2019dp, chen2020gs}.
This, by design, gives explicit privacy control to each local node. %and eliminates the security vulnerabilities identified in prior works.
Unlike the previous data-free federated distillation counterparts~\cite{zhu2021data, zhang2022fedzkt, zhang2022fine}, which employ both bidirectional distillation and iterative model parameter exchange, our framework makes another difference by conducting one-way distillation from thoroughly trained local models to the central model. These fully trained teacher models immediately enable us to explore the input space and learn the most efficient samples for knowledge distillation. Our critical insight is that each local's unique expertise under the heterogeneous FL setting can be further exploited. Therefore, we implement the input distillation according to the corresponding local products (\cf, Figure \ref{fig2}). This involves learning the transferred input to enable local nodes to reach a consensus on its semantic clarity while simultaneously generating diverse predictions with each task model. The former ensures the fundamental viability of the input data for transferring knowledge. At the same time, the latter allows the input data to leverage the unique aspects of each local node under heterogeneous federated learning scenarios.
Such feedback from local nodes enables us to deploy per-input importance weight for output ensemble distillation.
We demonstrate the effectiveness of our proposed method on natural and medical images through comprehensive experiments on image classification and segmentation tasks under various real-world federated learning scenarios, including the most challenging cross-domain cross-site settings. %with superior or comparable performances and at the same time without any sacrifice of privacy.
Our key contributions can be summarized as follows.
 \begin{itemize}
%     \item
% %%%removedVspace
%  We propose a federated learning framework that explicitly protects the privacy of local proprietary data by only conducting one-way distillation from local nodes to servers in both the input and output space.
     \item
We propose a federated learning framework with collaborative distillation in both the input and output space. It eliminates any requirement on model parameter exchange, model structure identity, prior knowledge of the local task, or auxiliary real data.
     \item
 To cope with the inherent heterogeneity of decentralized clients in federated learning, we introduce an ensemble distillation scheme that learns transferred input with explicit exploitation of each local's consensual and unique expertise.
     \item
We conduct extensive experiments with natural and medical images on classification and segmentation tasks, demonstrating state-of-the-art privacy-utility trade-offs compared to the prior art.
\end{itemize}

\section{Related Work}
\label{sec:related}
\subsection{Knowledge Distillation}

Hinton \etal \cite{hinton2015distilling} first proposed the concept of knowledge distillation \ie, using a cumbersome network as a teacher to generate soft labels to supervise the training of a compact student network. %Later there has been much progress in model distillation \cite{shazeer2017outrageously} and dataset distillation \cite{wang2018dataset}.
Although most of the following works transfer knowledge with one teacher, some techniques focus on multiple teachers and propose a variety of aggregation schemes, \eg, gate learning in the supervised setting \cite{asif2019ensemble, xiang2020learning}, and relative sample similarity for unsupervised scenarios \cite{wu2019distilled}. Recent progress in data-free knowledge transfer \cite{fang2019data, chen2019data} focuses on an adversarial training scheme to generate hard-to-learn and hard-to-mimic samples. Similarly, DeepInversion \cite{yin2020dreaming} utilizes backpropagated gradients to generate transfer samples that cause disagreements between the teacher and the student. \cite{nayak2019zero} crafts a transfer set by modeling and fitting data distributions in output similarities. %\cite{fang2021mosaicking} focuses on


\subsection{Distillation-based Federated Learning}
% \noindent
% \textbf{Parameter-based FL:} At each round of communication, each local model trains on its privately held data and then uploads its parameters/gradients. The central server aggregates them by averaging \cite{mcmahan2017communication} and shares this with the local nodes, which update their corresponding local model and carry out the next training round. Following FedAvg \cite{mcmahan2017communication}, FedAvgM \cite{hsu2019measuring} improves the aggregation scheme with additional momentum. FedProx \cite{li2018federated} proposes new local training strategies using the proximal term. Similarly, Scaffold \cite{karimireddy2019scaffold} employs control variations for better local training. However, such parameter/gradient exchange is highly susceptible to privacy leakage and stealth attack, as demonstrated elsewhere \cite{zhu2019deep, geiping2020inverting}.


% \noindent
% \textbf{Distillation-based FL:}
Beyond the parameter based FL \cite{mcmahan2017communication, hsu2019measuring,li2018federated}, early FL works like \cite{jeong2018communication} employ parameter and model output exchanges. Although the following works \cite{li2019fedmd, chang2019cronus, li2021practical} are purely based on the output of the local model for knowledge transfer, the selection of transfer data is highly dependent on prior knowledge of private data (\ie, they are under similar data distributions). Some recently proposed methods \cite{lin2020ensemble, gong2022preserving} provide some relaxation on transfer data. However, it is still necessary to carefully select the transfer data according to prior knowledge of the local task and private data.
While \cite{zhu2021data, zhang2022fedzkt, zhang2022fine} transfer knowledge without any requirement of real data, all of them need high communication bandwidth due to the iterative exchange of models over hundreds of rounds, leading to high susceptibility to stealth attacks and, hence, privacy concerns.
%In contrast, our method preserves privacy by only exchanging output of domain robust public data and further exploit in-depth feature level information for high efficient communication.\\ \ie, exchange outputs of locally trained model
%, leading to growing concerns on the privacy of local data.



\section{Approach}
\subsection{Problem Statement}
Without loss of generality, we describe our method for the classification task in detail.
Suppose that there are $K$ local nodes in a federated learning scenario, each privately holding a labeled dataset $ \{\domx'_k, \domy'_k \}$, consisting of the input image space $\domx' \in \mathbb{R}^{H \times W \times 3}$, and the label space $\domy' \in \{1, \dots, C \}$, where $C$ is the total number of classes.
%$\mathcal{D}_k=\{(x_k^i, y_k^i)|i=1,\ldots,|\mathcal{D}_k|\}$, where $y_k \in \{1,\ldots,C\}$ and

The proposed FedIOD includes two stages. First, with each private data $\{\domx'_k, \domy'_k \}$ we train the local model $T_k$ to complete. Note that the proposed FedIOD is agnostic to any neural network architecture. Hence, each local node can have its specialized architecture suited to the particular distribution of its local data. In the second stage, each locally trained model, $T_k$, will be frozen and only used as a teacher model in a one-way distillation paradigm. In contrast to \cite{gong2022federated, li2021practical} using carefully deliberated real data to transfer knowledge, we exploit ensemble knowledge in the input space $\mathcal{X}$ with a generator $G$ mapping from random noise $\domw$ to the input space $\mathcal{X}$. Taking such generated samples $x \sim \mathcal{X}$ as input, local models $T_k$ and the central task model $S$ on the server constitute a student-teacher knowledge transfer problem, with the teacher here being a group of local teachers.
Let $\hat{\bm{z}}=S(x) $  and $\bm{z}_k=T_k(x)$ be the output logits of the central model and the $k$-th respectively ($\hat{\bm{z}}, \bm{z}_k \in \mathbb{R}^C$), the corresponding probability can be acquired with the following activation function:
%%%removedVspace
\begin{equation}
%%%removedVspace
\label{eq:softmax}
   p_\tau(\bm{z})= \left[\frac{e^{z^1/\tau}}{\sum_c{e^{ z^c/\tau}}}, \dots, \frac{e^{z^C/\tau}}{\sum_c{e^{ z^c/\tau}}}\right],
\end{equation}
where $\tau$ is a temperature parameter set to 1 by default. We abbreviate $p_\tau(\bm{z}_k)$ and $p_\tau(\hat{\bm{z}})$ as $\bm{q}_k = T_k (x; \tau) $ and $\hat{\bm{q}} = S(x; \tau)$, respectively.
%In the following sections, we will elaborate the details of the second stage, \ie, how we conduct input-output ensemble distillation.
%As we discuss later, only local predictions on $x$ are transferred outside each local node to supervise $G$ and $S$ for distilling input and output space, thereby giving direct privacy control to each local node.

\subsection{Input Ensemble Distillation}
 To efficiently exploit the knowledge from local expertise, exploring the input space for the best fit of the global distribution is vital. We expect the optimal input to achieve (1) realism as a consensus achieved by all local nodes and (2) diversity to represent each local's unique knowledge under the heterogeneous federated learning scenarios.

\textbf{Consensual realism learning.}
Given the locally trained model $T_k$ as teachers and the central model $S$ as a student, we learn a generative model $G$ from randomly sampled noise $w$ to pseudo-data $x$, which will be the input for knowledge transfer. To guarantee the realism and practicality of $x$, we employ an additional discriminator $D_k$ residing at each local node to boost the generative model $G$ training.  $G$ is trained to approximate the global data distribution by fooling each local $D_k$. Following the typical training paradigm of GAN \cite{goodfellow2020generative, radford2015unsupervised}, we train $G$ and $D_k$ in a classical adversarial manner:
\begin{equation}
\label{eq:ganloss}
\begin{aligned}
    &\max_G \min_{D_k}L_\text{gan}^k(G, D_k) \\
    = &\max_G \min_{D_k} \fe_{x'_k \in \domx'_k} [\pi_k D_k(x_k')]  +  \fe_{w \in \domw} [1-D_k(G(w))],
\end{aligned}
\end{equation}
where $\pi_k =  \frac{|\domx'_k|}{\sum_{k'=1}^K |\domx'_{k'}|}$ is individual local weight and $|\domx'_k|$ indicates data size. In addition to this high-level realism, we expect $x$ to be realistic semantically, \ie, with semantic clarity according to the output of each locally trained model. Here, we assume that the input that confuses local models to produce ambiguous results will be less efficient in transferring knowledge. Hence, we expect each local model to produce confident predictions that the input $x$ tends to belong to one particular category. To force such semantic clarity, we maximize the confidence that $x$ belongs to one class. For each local node $k$, taking $\bm{q}_k$ as its corresponding probability, we minimize the Shannon entropy $H({\bm{q}})= - \sum_c \bm{q}^c \text{log} \bm{q}^c $, which can be reformulated as:
% %%removedVspace
\begin{equation}
%%removedVspace
\label{eq:confloss}
\begin{aligned}
    \min_{G} L_\text{conf}(G) &= \min_{G} \fe_{x \in \domx} [\sum_k{ \pi_k H (T_k(x; \tau))}] \\
    & =\min_{G}\fe_{w \in \domw} [\sum_k{\pi_k H (T_k(G(w); \tau))}]. \\
\end{aligned}
\end{equation}
\textbf{Per-local unique representation.}
The supervisions above ensure the realism of $x$, which are agreed upon by all local nodes. However, it can hardly transfer heterogeneous knowledge across local nodes. Our insight is that each local's expertise must be inconsistent, given the data heterogeneity in a federated learning scenario. Hence, the input must be diverse to generalize and transfer each local's unique knowledge.
To this point, we aim to generate $x$, which will tolerate local diversity, \wrt, input data on which local models produce divergent results. Specifically, we use Jensen-Shannon divergence to measure the dissimilarity of local probability outputs:
%%removedVspace
\begin{equation}
%%removedVspace
\label{eq:jsd}
    \begin{aligned}
        \text{JSD}(\bm{q}_1, \dots, \bm{q}_K) = H(\bar{\bm{q}}) - \sum_{k=1}^K \pi_k H(\bm{q}_k),
    \end{aligned}
% %%removedVspace
\end{equation}
where $\bar{\bm{q}} = \sum_{k=1}^{K} \pi_k \bm{q}_k$ is the weighted ensemble of all locals.
We maximize such dissimilarity to encourage the level of local diversity, \wrt, unique local knowledge which has been exploited:
\begin{equation}
%%removedVspace
\label{eq:uniqueloss}
\begin{aligned}
    &\min_{G} L_\text{unique} (G) \\
    = & \min_{G} \fe_{w \in \domw} [-\text{JSD}(T_1(G(w); \tau), \dots, T_K(G(w); \tau)) ].\\
\end{aligned}
%%removedVspace
\end{equation}

\subsection{Output Ensemble Distillation}
Model distillation techniques typically optimize the student model by minimizing the KL divergence between the student model output $\hat{\bm{q}}$ and the teacher model output $\bar{\bm{q}}$ to bridge their performance gap:
\begin{equation}
%%removedVspace
\label{eq:kl}
\begin{aligned}
    \text{KL}(\bar{\bm{q}}|| \hat{\bm{q}}) = H(\bar{\bm{q}},\hat{\bm{q}}) - H(\hat{\bm{q}}),  \\
\end{aligned}
% %%removedVspace
\end{equation}
where $H(\bar{\bm{q}}, \hat{\bm{q}})=-\sum_c \bar{\bm{q}}^c \log \hat{\bm{q}}^c$. Hinton \etal \cite{hinton2015distilling} has shown that minimizing Eq.~\ref{eq:kl} with a high $\tau$ (Eq.~\ref{eq:softmax}) is equivalent to minimizing the $\ell_2$ error between the logits of teacher and student, thereby relating cross-entropy minimization to fitting logits.  For multiple teachers, the conventional ensemble takes an average of all teachers' output probability as $\bar{\bm{q}}$.

However, under the FL scenario, it is not optimal to deploy such a local ensemble under the heterogeneous data distribution. This is mainly due to its inability to cope with the general setting when locally held data are not independent and identically distributed, \eg, they do not share precisely the same set of target classes.
Let $P_{\domx'_k, \domy'_k}$ be the data distribution of the image and label over the $k$-th local data, and $P_{\domx', \domy'}$ be the global data distribution. Thus, we approximate the importance ratio of local prediction based on its training data distribution:
\begin{equation}
%%removedVspace
\label{eq:bias}
\begin{aligned}
    & \frac{P_{\domx'_k, \domy'_k}(y|x)}{P_{\domx', \domy'}(y|x)} = \frac{P_{\domy'_k}(y) P_{\domx'_k, \domy'_k}(x|y) P_{\domx'}(x)}{P_{\domy'}(y) P_{\domx', \domy'}(x|y) P_{\domx'_k}(x)} \\
    & \thickapprox  \frac{P_{\domy'_k}(y)}{P_{\domy'}(y)} \cdot \frac{P_{\domx'}(x)}{P_{\domx'_k}(x)} \thickapprox  \frac{P_{\domy'_k}(y)}{P_{\domy'}(y)} \cdot \frac{P_{\domx}(x)}{P_{\domx'_k}(x)} ,
\end{aligned}
\end{equation}
where we assume $P_{\domx'_k, \domy'_k}(x|y) \thickapprox  P_{\domx', \domy'}(x|y)$ as the local heterogeneity of this term is minor and ignorable compared to the heterogeneity in the image distribution $P_{\domx'}(x)$ and the label distribution $P_{\domy'_k}(y)$. And the global image distribution $\domx'$ is approximated with the generated input domain $\domx \thickapprox \domx'$.

To consider this aspect, we introduce the weight of importance per class per input $\pi_k^c$ for each local node $k$ to reflect the data distribution with which its model was initially trained. Taking $x$ as input, we have the following.
% \begin{equation}
% \begin{aligned}
% \pi_k^c = \frac{N_{k}^c}{\sum_{k} {N_{k}^c}} \cdot \frac{|\mathcal{D}_k|}{\sum_{k} {|\mathcal{D}_k|}}.
% \end{aligned}
% \end{equation}
% \xnote{Do not use \mathcal{D}}
% where the first and second term corresponds to $\frac{P_{\domy'_k}(y)}{P_{\domy'}(y)}$ and $\frac{P_{\domx'}(x)}{P_{\domx'_k}(x)}$ respectively, and $N_{k}^c = \sum_{i=1}^{|\mathcal{D}_k|} (y_k^i=c)$ denotes the number of samples of class $c$ used in training the model at local node $k$.
%%removedVspace

\begin{equation}
\label{eq:aggweight}
\begin{aligned}
\hat{\pi}_k^c(x) = \frac{\fe_{y'_k \in \domy'_k}|y'_k = c|}{{\fe_{ k \in \{1,\cdots,K\}, y'_k \in \domy'_k} |y'_k = c|}} \cdot \frac{D_k(x)}{\fe_{x'_k \in \domx'_k} D_k(x'_k)},
\end{aligned}
\end{equation}
where the first term corresponds to $\frac{P_{\domy'_k}(y)}{P_{\domy'}(y)}$ and can be acquired by statistics of local labels, \ie, the number of samples from class $c$ used to train the model at the local node $k$. The second term corresponds to $\frac{P_{\domx}(x)}{P_{\domx'_k}(x)}$ which can be approximated by the local discriminator's output on pseudo image $x$ and locally held image $x'_k$. We then normalize the importance weight between locals for each $c$: $\pi_k^c(x) = \hat{\pi}_k^c(x) / \sum_{k'=1}^{K} {\hat{\pi}_{k'}^c (x)}$.

\begin{algorithm}[t]
    \caption{FedIOD}
    \label{alg}
 \begin{algorithmic}
    \STATE {\bfseries Input:} Total number of local nodes $K$, locally held data $\{\domx'_k, \domy'_k\}$, local models $\{T_k\}$, central task model $S$, central generator $G$, auxiliary local discriminator $\{D_k\}$.
    \FOR{each local node $k=1,\cdots, K$}
    \STATE Train $T_k$  with $(\domx'_k, \domy'_k)$ to complete
    \ENDFOR
    \FOR{each distillation step}
    \STATE {\color{gray} {$\Box$ Input distillation }}
    \STATE $w$ $\leftarrow$ randomly sampled from $\domw$
    \STATE $x \leftarrow G(w)$
    \FOR {$k =1 ,..., K$ }
    \STATE $\bm{z}_k$, $\bm{q}_k$ $\leftarrow T_k(x)$
    \STATE $x'_k$ $\leftarrow$ randomly sampled from $\domx'_k$
    \STATE $L_\text{gan}^k(G,D_k) \leftarrow D_k(x'_k), D_k(x)$ \hspace*{3em} $\triangleright$  Eq.~\ref{eq:ganloss}
    \STATE Update $D_k$ by descending its stochastic gradient
    % \begin{footnotesize}
    $\nabla_{D_k} L_\text{gan}$
    % \end{footnotesize}
    \ENDFOR
    \STATE $L_\text{conf}(G), L_\text{unique}(G) \leftarrow$ $\{\bm{q}_k\}$  \hspace*{5em}   $\triangleright$  Eq.~\ref{eq:confloss},\ref{eq:uniqueloss}
    \STATE {\color{gray} {$\Box$ Output distillation }}
    \STATE  $\hat{\bm{z}}$, $\hat{\bm{q}}$ $\leftarrow S(x)$
    \STATE $L_\text{mimic}(G, S) \leftarrow$  $\hat{\bm{z}}$, $\{\bm{z}_k\}$   \hspace*{6.7em}   $\triangleright$  Eq.~\ref{eq:mimicloss}
    \STATE {\color{gray} {$\Box$ Update }}
    \STATE Update $G$ by descending its stochastic gradient
    $\nabla_{G} [ L_\text{conf} +  L_\text{unique} - L_\text{mimic} - \sum_{k=1}^K L_\text{gan}^k ]$
    \STATE Update $S$ by descending its stochastic gradient
    $\nabla_{G} L_\text{mimic}$
    \ENDFOR
\end{algorithmic}
% %%removedVspace
\end{algorithm}

Following the $\ell_2$ observation above of Hinton \etal \cite{hinton2015distilling}, we consider the case of $\tau \rightarrow \infty$ when deploying KL-divergence. Hence, it can be written as the $\ell_2$ error between central model logits $\hat{\bm{z}}$ and local aggregated $\bar{\bm{z}}$. Let $\bm{\pi}_k(x) =[\pi_k^1(x),\cdots,\pi_k^C(x)] \in [0,1]^C$ be the per-sample weight, and $\odot$ is Hadamard product, the local ensemble expertise is indicated as follows:
%%removedVspace
\begin{equation}
\label{eq:agglogits}
\begin{aligned}
    A(\bm{z}_1, \cdots, \bm{z}_K, x) = \sum_{k=1}^K \bm{\pi}_k(x) \odot \bm{z}_k,
\end{aligned}
%%removedVspace
\end{equation}
where the central model $S$ is optimized to mimic the local ensemble of expertise, while the generator $G$ is a critic to generate $x$ on which $S$ lags behind local experts. The motivation is that such challenging input will transfer the hard-to-mimic knowledge from local to central. Therefore, we tailor the input data on which the central model produces a result diverged from the local output. Using KL-divergence as a dissimilarity evaluation, we train $G$ and $S$ in an adversarial manner:
\begin{equation}
% %%removedVspace
\label{eq:mimicloss}
\begin{aligned}
    &\max_G \min_S L_\text{mimic}(G, S)
   = \\&\max_G \min_S
   % \fe_{w \in \domw}
   \fe_{w}
   | S(G(w)) - A(T_1(G(w)), \cdots, T_K(G(w)))|^2,
\end{aligned}
\end{equation}
where $A(\cdot)$ is the aggregation function detailed in Eq.~\ref{eq:agglogits}.
% Inspired by \cite{fang2019data, chen2019data, yin2020dreaming},  we optimize the generator $G$ to serve as a critic of central learning, boosting the central model $S$ to mimic locals $T_k$ in an adversarial manner.
%: $\max_{G} \Loss_\text{mimic}$, where $\Loss_\text{mimic}$ will be detailed in the following section.
To sum up, the overall loss function can be written as
\begin{equation}
\label{eq:overallloss}
\begin{aligned}
   &\max_G \min_{D_k} L_\text{gan}^k(G, D_k) +\min_G [L_\text{conf}(G) + L_\text{unqiue}(G)] \\
   & +\max_G \min_S L_\text{mimic}(G, S).
\end{aligned}
%%removedVspace
\end{equation}
And the overall process is explained in Algorithm~\ref{alg}.

\section{Experiments}
We provide comprehensive empirical studies with various heterogeneous FL settings on natural image classification and more privacy-sensitive medical tasks, including brain tumor segmentation and histopathological nuclei instance segmentation.
%Our experiment settings simulate real-world FL scenarios, including local data of different sizes, on different subjects, on different organs, and collected from different institutions.
%\textit{Implementation details and privacy analysis will be provided in the supplementary materials.}

% \begin{table}
% \begin{center}
% \resizebox{\columnwidth}{!}
% % \scalebox{0.90}
% % \scriptsize
% {
% \begin{tabular}{cc|cccccccc}

% \hline
% $L_\text{gan}$  &$\pi_k$ &Avg &Eq.\ref{eq:ganloss} &Eq.\ref{eq:ganloss} &Eq.\ref{eq:ganloss} &Eq.\ref{eq:ganloss} &Eq.\ref{eq:ganloss} &Eq.\ref{eq:ganloss} &Eq.\ref{eq:ganloss}\\ \hline
% \multirow{2}{*}{$L_\text{mimic}$} &$\tau$ &\multirow{2}{*}{\xmark} &\multirow{2}{*}{\xmark} &1 & 1 &$\infty$ &$\infty$ &$\infty$ &$\infty$\\
% &$\pi_k$ & &  &$*$ &Eq.\ref{eq:aggweight} &Eq.\ref{eq:aggweight} &Eq.\ref{eq:aggweight} &Eq.\ref{eq:aggweight} &Eq.\ref{eq:aggweight} \\\hline
% \multicolumn{2}{c|}{$L_\text{conf}$} &\xmark &\xmark &\xmark &\xmark &\xmark &\cmark &\xmark &\cmark\\\hline
% \multicolumn{2}{c|}{$L_\text{unique}$} &\xmark &\xmark &\xmark &\xmark &\xmark &\xmark &\cmark &\cmark\\ \hline
% \multicolumn{2}{c|}{Acc. (\%)} &57.03 &65.05 &59.88 &65.85 &68.49 &69.21 &69.67 &70.36 \\
% \hline
% \end{tabular}}
% %%removedVspace
% \end{center}
% \caption{Ablation study on CIFAR-10 with ResNet-8, $K$=20 and $\alpha$=0.1. For the training of $L_\text{gan}$, we compare our weighting scheme (Eq. \ref{eq:ganloss}) with the typical average ensemble.  For the ensemble scheme of $L_\text{mimic}$, we compare our per-sample, per-class importance weighting (Eq. \ref{eq:aggweight}) with $*$ which represents the weighting scheme commonly used in other FL methods \cite{lin2020ensemble, hsu2020federated}. To compare $\tau$, we only list the result with a typical value $\tau$=1~\cite{hinton2015distilling}.}
% \label{tab:cifarablation}
% %%removedVspace
% \end{table}

% \begin{table}
% \begin{center}
% \resizebox{\columnwidth}{!}
% \scalebox{0.90}
% \scriptsize
% {
% \begin{tabular}{cc|cccccccc}
% \hline
% $L_\text{gan}$  &$\pi_k$ &Avg &Eq.\ref{eq:ganloss} &Eq.\ref{eq:ganloss} &Eq.\ref{eq:ganloss} &Eq.\ref{eq:ganloss} &Eq.\ref{eq:ganloss} &Eq.\ref{eq:ganloss}\\ \hline
% \multirow{2}{*}{$L_\text{mimic}$} &$\tau$ &\multirow{2}{*}{\xmark} &\multirow{2}{*}{\xmark} & 1 &$\infty$ &$\infty$ &$\infty$ &$\infty$\\
% &$\pi_k$ & &   &Eq.\ref{eq:aggweight} &Eq.\ref{eq:aggweight} &Eq.\ref{eq:aggweight} &Eq.\ref{eq:aggweight} &Eq.\ref{eq:aggweight} \\\hline
% \multicolumn{2}{c|}{$L_\text{conf}$} &\xmark &\xmark &\xmark &\xmark &\cmark &\xmark &\cmark\\\hline
% \multicolumn{2}{c|}{$L_\text{unique}$} &\xmark &\xmark &\xmark &\xmark &\xmark &\cmark &\cmark\\ \hline
% \multicolumn{2}{c|}{Acc. (\%)} &57.0 &65.0 &65.9 &68.5 &69.2 &69.7 &70.4 \\
% \hline
% \end{tabular}}
% %%removedVspace
% \end{center}
% \caption{Ablation study on CIFAR-10 with ResNet-8, $K$=20 and $\alpha$=0.1. For the training of $L_\text{gan}$, we compare our weighting scheme (Eq. \ref{eq:ganloss}) with the typical average ensemble.  For the ensemble scheme of $L_\text{mimic}$, we compare our per-sample, per-class importance weighting (Eq. \ref{eq:aggweight}) with $*$ which represents the weighting scheme commonly used in other FL methods \cite{lin2020ensemble, hsu2020federated}. To compare $\tau$, we only list the result with a typical value $\tau$=1~\cite{hinton2015distilling}.}
% \label{tab:cifarablation}
% \end{table}

\begin{table*}[]
\centering
% \fontsize{9.0pt}{10.0pt} \selectfont
% \scriptsize
% \footnotesiz
% \normalize
\resizebox{\textwidth}{!}
{
\begin{tabular}{cc|c|c|cc|cc}
\toprule
\multicolumn{2}{c|}{\multirow{2}{*}{\textbf{Method}}}
% \multicolumn{2}{c|}{\textbf{Privacy Guarantee}} &
%& No Param.
&Model- &Auxiliary  &
\multicolumn{2}{c|}{CIFAR-10} &
\multicolumn{2}{c}{CIFAR-100} \\
& %& Share
&agnostic & Prerequisite
&$\alpha=1$ &$\alpha=0.1$ &$\alpha=1$ &$\alpha=0.1$ \\
\midrule
\multicolumn{2}{c|}{Standalone (mean $\pm$ std)}
%& -
&- &- &65.25$\pm$ 5.14 & 30.92$\pm$  11.17 &27.60$\pm$ 1.58 &16.99$\pm$ 2.46
\\ \midrule
\parbox[t]{6mm}{\multirow{5}{*}{\rotatebox[origin=c]{90}{\shortstack[c]{Parameter- \\ based}}}} &FedAvg~\cite{mcmahan2017communication}
%&\xmark
&\xmark &-
&
78.57$\pm$ 0.22 &
68.37$\pm$ 0.50
&
42.54$\pm$ 0.51 &
36.72$\pm$ 1.50 \\
&FedProx~\cite{li2018federated}
%&\xmark
&\xmark &-
&
76.32$\pm$ 1.95 &
68.65$\pm$ 0.77
&
42.94$\pm$ 1.23 &
35.74$\pm$ 1.00 \\
&FedAvgM~\cite{hsu2019measuring}
%&\xmark
&\xmark &-
&
77.79$\pm$ 1.22 &
68.63$\pm$ 0.79
&
42.83$\pm$ 0.36 &
36.29$\pm$ 1.98 \\
&FedGEN~\cite{zhu2021data}
%&\xmark
&\xmark &task-relevant data
&
80.31$\pm$ 0.97 &
68.13$\pm$ 1.37
&
45.97$\pm$ 0.23&
35.97$\pm$ 0.31\\
&FedDF~\cite{lin2020ensemble}
%&\xmark
&\xmark &task-relevant data
&
\bf{80.69$\pm$ 0.43} &
\bf{71.36$\pm$ 1.07}
&
\bf{47.43$\pm$ 0.45} &
\bf{39.33$\pm$ 0.03} \\\midrule
\parbox[t]{6mm}{\multirow{3}{*}{\rotatebox[origin=c]{90}{\shortstack[c]{Distill- \\ based}}}} &FedMD~\cite{li2019fedmd}
%&\cmark
&\cmark & task-relevant data
&
80.37$\pm$ 0.37 &
{69.23}$\pm$ 1.31
&
\bf{45.83$\pm$ 0.58} &
38.86$\pm$ 0.78 \\
&FedKD \cite{gong2022preserving}
%&\cmark
&\cmark &task-relevant data
&
{80.98}$\pm$ 0.11 & {65.46}$\pm$ 3.45 & {45.55}$\pm$ 0.38& {40.61}$\pm$ 2.54
\\
&FedIOD %&\cmark
&\cmark &None &\bf{82.78$\pm$ 0.18} &\bf{70.08$\pm$ 0.37} &45.36$\pm$ 0.32 &\bf{41.88$\pm$ 0.16} \\
\bottomrule
\end{tabular}
}
% %%removedVspace
%FedAD is trained with a single round 200 epochs, while all the others are trained with 100 communication rounds and 10k steps per round. Note that FedMD, FedDF and ours are based on distillation and use batch normalization (BN), while the others use group normalization (GN) as GN is demonstrated to be more efficien than BN for gradient-based FL methods~\cite{mcmahan2017communication, hsu2019measuring, li2018federated}.}
%%removedVspace
\caption{Accuracy (\%) comparisons on the CIFAR-10 and CIFAR-100 datasets with ResNet-8 and $K$=20. ``Standalone'' indicates the performance of local models trained with individual private data. Several popular FL methods are compared with parameter-based and distillation-based FL prior arts, among which \cite{lin2020ensemble, zhu2021data} employ both parameter exchange and model output distillation.}
%%removedVspace
\label{tab:cifarcompare}
\end{table*}



\subsection{CIFAR-10/100 classification}
We use heterogeneous data splits with Dirichlet distribution following the prior art \cite{hsu2019measuring} for distributed local training sets. The value of $\alpha$ in the Dirichlet distribution controls the degree of non-IIDness: $\alpha \rightarrow \infty$ indicates an identical local data distribution, and a smaller $\alpha$ indicates a higher non-IIDness. We report average accuracy over three split seeds on the corresponding test set.

We conduct experiments following the typical FL setting \cite{lin2020ensemble} under $K$=20 and $\alpha$=1, 0.1 with ResNet-8. %We train each local model individually with SGD and CosineAnnealing, decreasing the learning rate from 0.02 to 0.001 in 500 epochs with a batch size of 16. For distillation, we use the Adam optimizer,  a constant learning rate of 1e-3, and a batch size of 512, 200, and 10 rounds for CIFAR-10  and  CIFAR-100, respectively. The weight decay is 3e-4 and 0 for local training and distillation, respectively. \xnote{G,D,T,S model structure, training details }
$w$ is randomly sampled with a dimension of 100, and $x=G(w)$ has a size of $32 \times 32$. We use a patch discriminator as $D_k$, of which the output is of size $8 \times 8$.
The comparison in Table~\ref{tab:cifarcompare} shows that our method achieves superior or competitive results and a much stronger privacy guarantee.
Without the requirement of auxiliary data or prior knowledge of the local task, our method outperforms relevant-data-dependent distillation-based and parameter-based counterparts.
% Compared to parameter-based prior art~\cite{lin2020ensemble}, we demonstrate much better performance (\eg, 82.78\% over 80.69\%, and 41.88\% over 39.33\%).
% At the same time, it protects privacy by not sharing locally trained model parameters.
Moreover, our method demonstrates other benefits, including eliminating prerequisites of identical local model architecture or task-relevant real data.
% Figure~\ref{fig:bandwidth} compares the communication bandwidth of our FedIOD and the classic FL method like FedAvg.
%In Table~\ref{tab:cifarablation}, we conduct ablation studies to validate the efficacy of each proposed module.

\subsection{Magnetic resonance image segmentation}
We use the dataset from the 2018 Multimodal Brain Tumor Segmentation Challenge (BraTS 2018)~\cite{menze2014multimodal, bakas2018identifying}.
% It contains multi-parametric preoperative magnetic resonance imaging scans of 285 subjects with brain tumors, including 210 high-grade glioma (HGG) and 75 low-grade glioma (LGG) subjects.
Each subject was associated with voxel-level annotations of ``whole tumor", ``tumor core," and ``enhancing tumor."
%Each subject was scanned under the T1-weighted, T1-weighted with contrast enhancement, T2-weighted, and T2 fluid-attenuated inversion recovery (T2-FLAIR) modalities. %
Following the experimental protocol of one prior art, \cite{chang2020synthetic}, we deploy 2D segmentation of the whole tumor on T2 images of HGG cases, among which 170 were for training and 40 for testing. The local data split also follows \cite{chang2020synthetic}.%: we first sort the training cases with tumor size and then divide the training set into ten subsets distributed to 10 local nodes. Overall there are 11,057 slices as training images across all local nodes and 2,616 slices as testing images.

\begin{table}[h]
\centering
% \scriptsize
% \fontsize{9.0pt}{10.0pt} \selectfont
\resizebox{\columnwidth}{!}
{
\begin{tabular}
{c|cccc}
\toprule
&Dice(\%)$\uparrow$ &Sens.(\%)$\uparrow$ &Spec.(\%)$\uparrow$ &HD95(pixel)$\downarrow$ \\ \midrule
{Standalone} &\renewcommand\arraystretch{0.8} \begin{tabular}{@{}c@{}} 65.03 \\ \footnotesize{$\pm$3.31}\end{tabular}
&\renewcommand\arraystretch{0.8} \begin{tabular}{@{}c@{}} 69.27  \\ \footnotesize{$\pm$4.72}\end{tabular}
&\renewcommand\arraystretch{0.8} \begin{tabular}{@{}c@{}} 99.35 \\ \footnotesize{$\pm$0.15}\end{tabular}
&\renewcommand\arraystretch{0.8} \begin{tabular}{@{}c@{}} 24.61 \\ \footnotesize{$\pm$3.62}\end{tabular}  \\\midrule
{Centralized} &74.85 &79.83 &99.55 &12.85 \\ \midrule
% FedAvg \cite{mcmahan2017communication} &0.7235 &0.6963 &\bf{0.9984} &10.43\\
FedAvg &70.71 &67.31 &\bf{99.85} &{11.88}\\
AsynDGAN &70.43 &72.95 &99.57 &14.94\\
FedIOD &\bf{75.38} &\bf{79.47} &99.60 &\bf{11.76}\\
\bottomrule
\end{tabular}}
%%removedVspace
\caption{Comparisons on the BraTS2018 dataset with $K$=10 under the same net with FedAvg and AsynDGAN. ``Centralized'': centralized training when all local data are collected together.
% Sens. and Spec. are the abbreviation of sensitivity and specificity
}
\label{tab:brats}
%%removedVspace
\end{table}

We employ the same network structure of $G$, $D_k$, $S$, and the same data preprocessing as \cite{chang2020synthetic} for a fair comparison. Following its label condition $\domw$, we improve our $L_\text{gan}$ with additional perceptual loss \cite{johnson2016perceptual}.
% we do data augmentation with random crop, flip, and rotation for both the local training and the distillation.
The Dice score, sensitivity (Sens.), specificity (Spec.), and Hausdorff distance (HD95) are used as evaluation metrics, where ``HD95'' represents 95\% quantile of the distances instead of the maximum.
% Taking $\bm{y}, \hat{\bm{y}} \in \{0,1 \}^{H \times W}$ as the ground-truth mask and the segmentation prediction, respectively, Dice evaluates the overlap between the two: $\text{Dice}(\bm{y}, \hat{\bm{y}}) = {2|\bm{y} \cap \hat{\bm{y}} |} / {(|\bm{y}|+|\hat{\bm{y}}|)}$. Sensitivity represents the true positive rate: $\text{Sens}(\bm{y}, \hat{\bm{y}}) = {|\bm{y} \cap \hat{\bm{y}} |} / {|\bm{y}|}$, and specificity represents the true negative rate: $\text{Spec}(\bm{y}, \hat{\bm{y}}) = {|(1-\bm{y}) \cap (1-\hat{\bm{y}}) |} / {|1-\bm{y}|}$. The Hausdorff distance evaluates the shape similarity:
% \begin{equation}
% \label{eq:hausdorff}
%     \text{HD}(\bm{y},\hat{\bm{y}}) = \max \{\sup_{\bm{u} \in \partial{\bm{y}}} \inf_{\bm{\hat{u}} \in \partial{\bm{\hat{y}}}}  |\bm{u}-\hat{\bm{u}}|, \sup_{\bm{\hat{u}}\in \partial{\bm{\hat{y}}}} \inf_{\bm{u} \in \partial{\bm{y}}}    |\bm{u}-\hat{\bm{u}}| \},
% \end{equation}
% where $\partial$ indicates boundary extraction and returns boundary position sets.


Table \ref{tab:brats} compares our method with the prior art of distributed learning \cite{chang2020synthetic} and the classical parameter-based FedAvg method. Ours performs best segmentation on pixel-level overlap metrics (Dice and Sens.) and shape similarity metrics (HD95).
% In particular, our proposed method outperforms the parameter sharing method \cite{mcmahan2017communication} and simultaneously provides a much more secure privacy guarantee. Figure~\ref{fig:bandwidth} shows that our method costs less or equivalent communication bandwidth compared to the parameter-based method such as FedAvg.

\begin{table}
\centering
% \scriptsize
% \fontsize{6.5pt}{8.0pt} \selectfont
\resizebox{\columnwidth}{!}
{
\begin{tabular}
{cc|cccc}
\toprule
& &Dice(\%)$\uparrow$ &Obj-Dice(\%)$\uparrow$  &AJI(\%)$\uparrow$ &HD95(pixel)$\downarrow$\\ \midrule
\parbox[t]{1mm}{\multirow{4}{*}{\rotatebox[origin=c]{90}{\shortstack[c]{{Standalone}}}}}
    &{breast} &77.92 &73.47 &53.64 &12.34\\
    &{liver} &79.16 &75.38 &55.63 &12.47 \\
    &{kidney} &74.99 &69.67 &50.99 &14.64\\
    &{prostate} &77.46 &73.74 &54.40 &15.59\\  \midrule
%\multicolumn{2}{c|}{{Centralized}} &- &0.7833 &  &0.5608 \\ \midrule
    \multicolumn{2}{c|}{FedAvg}  &78.12 &75.05 &55.56 &12.96\\
    \multicolumn{2}{c|}{AsynDGAN} &79.30 &72.73 &56.08 &14.49 \\
    \multicolumn{2}{c|}{FedIOD} &\bf{80.48} &\bf{77.03} &\bf{58.37} &\bf{11.22}\\
    \bottomrule
\end{tabular}}
%%removedVspace
\caption{Comparisons on the TCGA dataset with four cross-organ local nodes. All methods use the same segmentation net provided by \cite{chang2020synthetic} for a fair comparison. }
\label{tab:nucleicrossorgan}
%%removedVspace
\end{table}

% %%removedVspace
\subsection{Histopathological image segmentation}
In real-world medical applications, the heterogeneity of data distributed among medical entities is not limited to the local size of the data or various subjects. Local data held by different clinical sites can be quite a domain variant, \eg, targeting different organs or collected with different infrastructures, which is relatively underexplored in contemporary FL methods. To this end, we evaluate our method in a cross-organ, cross-site setting where locally held data are from different organs and institutes. We experiment on nuclei instance segmentation task with pathological datasets, including TCGA~\cite{kumar2017dataset}, Cell17~\cite{vu2019methods} and TNBC~\cite{naylor2018segmentation}.

\begin{table*}
% %%removedVspace
\centering
% \fontsize{9.0pt}{10.0pt} \selectfont
% \scalebox{0.65}
\resizebox{\textwidth}{!}
% \scriptsize
{
\begin{tabular}{c|ccccc|cccc}
\toprule
%&\multirow{3}{*}{\rotatebox[origin=c]{0}{{Privacy}}} &\multirow{3}{*}{\rotatebox[origin=c]{0}{{Prerequisite}}}
%&NO Param. &\multirow{2}{*}{Utility}
&Test   &\multirow{2}{*}{Dice(\%)$\uparrow$} &\multirow{2}{*}{Obj-Dice(\%)$\uparrow$} &\multirow{2}{*}{{AJI(\%)$\uparrow$}} &\multirow{2}{*}{HD95(pixel)$\downarrow$} &\multicolumn{4}{c}{Average}\\
%& Share &
&Data & & & &  &Dice(\%)$\uparrow$ &{Obj-Dice(\%)$\uparrow$} &{AJI(\%)$\uparrow$} &HD95(pixel)$\downarrow$\\
\midrule
\multirow{3}{*}{FedAvg}
%&\multirow{3}{*}{\xmark} &\multirow{3}{*}{Task-agnostic}
&\cellcolor{gray0}{Cell17}  &\cellcolor{gray0}{68.74} &\cellcolor{gray0}{65.82} &\cellcolor{gray0}{39.37}  &\cellcolor{gray0}{24.15} &\multirow{3}{*}{63.42} &\multirow{3}{*}{64.00} &\multirow{3}{*}{37.29}  &\multirow{3}{*}{54.51} \\
%& &
&\cellcolor{gray1}{TCGA}  &\cellcolor{gray1}{77.57} &\cellcolor{gray1}{72.94} &\cellcolor{gray1}{50.03} &\cellcolor{gray1}{15.87} & & &  \\
%& &
&\cellcolor{gray2}{TNBC}  &\cellcolor{gray2}{43.95} &\cellcolor{gray2}{53.23} &\cellcolor{gray2}{22.48} &\cellcolor{gray2}{123.51} & & &  \\ \midrule
\multirow{3}{*}{AsynDGAN}
%&\multirow{3}{*}{\cmark} &\multirow{3}{*}{Classify/Segment-only}
&\cellcolor{gray0}{Cell17}  &\cellcolor{gray0}{79.82} &\cellcolor{gray0}{59.03} &\cellcolor{gray0}{34.64} &\cellcolor{gray0}{19.27} &\multirow{3}{*}{66.64}   &\multirow{3}{*}{61.15} &\multirow{3}{*}{34.21}  &\multirow{3}{*}{35.46} \\
%& &
&\cellcolor{gray1}{TCGA}  &\cellcolor{gray1}{52.29} &\cellcolor{gray1}{57.12} &\cellcolor{gray1}{26.03} &\cellcolor{gray1}{47.47} & & &  \\
%& &
&\cellcolor{gray2}{TNBC}  &\cellcolor{gray2}{67.80} &\cellcolor{gray2}{67.31} &\cellcolor{gray2}{41.96} &\cellcolor{gray2}{39.63} & & &  \\ \midrule
\multirow{3}{*}{FedIOD}
%&\multirow{3}{*}{\cmark} &\multirow{3}{*}{Task-agnostic}
&\cellcolor{gray0}{Cell17} &\cellcolor{gray0}{86.23} &\cellcolor{gray0}{68.03} &\cellcolor{gray0}{44.75} &\cellcolor{gray0}{7.01}  &\multirow{3}{*}{\bf{79.28}}  &\multirow{3}{*}{\bf{71.58}} &\multirow{3}{*}{\bf{49.52}} &\multirow{3}{*}{\bf{16.41}}  \\
%& &
&\cellcolor{gray1}{TCGA}  &\cellcolor{gray1}{76.59} &\cellcolor{gray1}{72.67} &\cellcolor{gray1}{53.04} &\cellcolor{gray1}{12.69} & & &  \\
%& &
&\cellcolor{gray2}{TNBC}  &\cellcolor{gray2}{75.01} &\cellcolor{gray2}{74.03} &\cellcolor{gray2}{50.76} &\cellcolor{gray2}{29.54} & & &  \\
\bottomrule
\end{tabular}}
%%removedVspace
\caption{Comparisons of cross-site cross-organ nuclei segmentation tasks with Cell17, TCGA, TNBC as distributed local data. For a fair comparison, all methods use the same U-Net architecture as the segmentation model and the same post-processing to convert the semantic prediction into instance segmentation results.
}
%%removedVspace
\label{tab:nucleicrosssite}
\end{table*}



% \noindent\textbf{TCGA:}
% The TCGA dataset~\cite{kumar2017dataset}  was captured from the Cancer Genome Atlas  archive and used in MICCAI 2018 multi-organ segmentation challenge (MoNuSeg). The training set consists of 30 images and around 22,000 nuclei instance annotations, while the test set includes 14 images with additional 7000 nuclei boundary annotations. The images are with $1000$ $\times$ $1000$ pixels and captured at 40$\times$  magnification on hematoxylin and eosin (H\&E) stained tissue. These images show highly varying properties from 18 hospitals and seven organs (breast, liver, kidney, prostate, bladder, colon, and stomach).


% \noindent\textbf{Cell17:}
% The MICCAI 2017 Digital Pathology Challenge dataset~\cite{vu2019methods} (Cell17)  consists of 64 H\&E stained histology images. Both the training and testing sets contain 32 images from four different diseases: glioblastoma multiforme (GBM), lower-grade glioma (LGG) tumors, head, and neck squamous cell carcinoma (HNSCC), and non-small cell lung cancer (NSCLC). The image sizes are either $500 \times 500$ or $600 \times 600$ at $20\times$ or $40\times$ magnification.



% \noindent\textbf{TNBC:}
% The Triple Negative Breast Cancer (TNBC)~\cite{naylor2018segmentation} dataset consists of 50 annotated $512 \times 512$ images at $40\times$ magnification. The images are sampled from 11 patients at the Curie Institute, with three to eight images for each patient. Overall there are  4022 annotated cell instances. %The image data includes low cellularity regions, which can be stromal areas or adipose tissue, and high cellularity areas consisting of invasive breast carcinoma cells.

We cropped the images into patches of size $256 \times 256$ for training and inference. For metrics evaluation, the cropped patches are stitched back into the whole image with the original size. For $G$, $D_k$, and $S$,  we use the same model structure provided by \cite{chang2020synthetic} and the additional perceptual loss \cite{johnson2016perceptual} for $L_\text{gan}$.
We use object-level Dice \cite{chen2016dcan} and Aggregated Jaccard Index (AJI) \cite{vu2019methods} as metrics to evaluate the instance overlap or shape similarities for an individual object. Let $\bm{y}^i$ be the ground truth mask for the $i$-th instance of the total $n$ instances, and $\hat{\bm{y}}^j$ be the predicted mask for the $j$-th instance of the total $\hat{n}$ instances. %The object-level Dice is written as
% \begin{equation}
% \label{eq:diceobj}
% \begin{aligned}
% \footnotesize{
%     \text{Dice}_\text{obj} (\bm{y}, \hat{\bm{y}})= \frac{1}{2}( \sum_{i=1}^n \omega_i \text{Dice}(\bm{y}^i, J(\bm{y}^i))
%     + \sum_{j=1}^{\hat{n}} \hat{\omega}_j \text{Dice}(\hat{\bm{y}}^j, J(\hat{\bm{y}}^j)) ), }
% \end{aligned}
% \end{equation}
$J(\bm{y}^i)= \text{argmax}_{\hat{\bm{y}}^j} {|\bm{y}^i \cap \hat{\bm{y}}^j|} / {|\bm{y}^i \cup \hat{\bm{y}}^j|}$ is the predicted instance that maximally overlaps $\bm{y}^i$, and $J(\hat{\bm{y}}^j)= \text{argmax}_{\bm{y}^i} {|\bm{y}^i \cap \hat{\bm{y}}^j|} / {|\bm{y}^i \cup \hat{\bm{y}}^j|}$ denotes the ground-truth instance that maximally overlaps $\hat{\bm{y}}^j$.
%$\omega_i = |\bm{y}^i|/\sum_{m=1}^{n}|\bm{y}^m|$, and $\bm{\omega}_j = |\hat{\bm{y}}^j|/\sum_{m=1}^{\hat{n}}|\hat{\bm{y}}^m|$ are the instance weight of ground-truth and prediction respectively.
For instance, for shape similarity, we use the Aggregated Jaccard Index (AJI):
\begin{equation}
\label{eq:aji}
\begin{aligned}
    \text{AJI} (\bm{y}, \hat{\bm{y}})= \frac{\sum_{i=1}^{n} | \bm{y}^i \cap J(\bm{y}^i)| }{ \sum_{i=1}^{n} | \bm{y}^i \cup J(\bm{y}^i)| + \sum_{j \in \mathcal{J}} |\hat{\bm{y}}^j|},
\end{aligned}
\end{equation}
where $J(\bm{y}^i)$ is the predicted instance that has maximum overlap with $\bm{y}^i$ concerning the Jaccard index (sorted and nonrepeated). $\mathcal{J}$ is the set of predicted instances that have not been assigned to any ground-truth instance.

\begin{figure}[h]
%%removedVspace
\centering
\includegraphics[width=1.0\linewidth]{fig/fig3.pdf}
%%removedVspace
\caption{Qualitative comparisons on cross-site cross-organ nuclei segmentation tasks. The three rows visualize instance contours on test images from Cell17, TCGA, and TNBC.}
\label{fig3}
%%removedVspace
\end{figure}

\textbf{Cross-organ scenario.}
We first focus on cross-organ settings where each distributed local node holds only the data of one organ. Following \cite{chang2020synthetic}, from the TCGA dataset, we take 16 images of the breast, liver, kidney, and prostate for training and eight images of the same organs for testing.
%The network structure of $G$, $D_k$, $S$ and data augmentation keep consistent with
Table \ref{tab:nucleicrossorgan} shows the experimental results of this cross-organ setting and compares them with the baseline method \cite{chang2020synthetic} and the classical FedAvg. We can note that our method achieves the best results on semantic segmentation (Dice and Hausdorff) and instance segmentation (object-level Dice and AJI) metrics. %In particular, our method demonstrates superior segmentation performance and stronger privacy guarantees compared to the parameter-sharing method \cite{mcmahan2017communication}.



\textbf{Cross-site cross-organ scenario.}
We also conduct experiments on more challenging settings with cross-site cross-organ datasets, where locally held data are from different organ nuclei datasets.
Taking the training set of Cell17, TCGA, and TNBC as private data distributed over local nodes, we evaluate on the corresponding test sets. %textit{The evaluation of each locally trained model will be provided in the supplementary.}
Table \ref{tab:nucleicrosssite} compares our method with two prior arts \cite{chang2020synthetic, mcmahan2017communication} on various segmentation metrics to evaluate semantic/instance level overlap and shape.
% AsynDGAN \cite{chang2020synthetic} uses synthetic pairs to train the central model, thus limiting applications only to the classification or segmentation task.
% In contrast, our method is model-agnostic and task-agnostic without any application constraints.
Our proposed FedIOD outperforms the prior art on all these metrics for overlap and shape evaluation,
%Notably, compared with the last experiment only with cross-organ local splits, we achieve more obvious improvement over the counterparts under these cross-site cross-organ settings
demonstrating our efficacy in coping with heterogeneous FL scenarios. The qualitative comparisons shown in Figure \ref{fig3} also demonstrate the superiority of our method over its counterparts.

\begin{table}[h]
% %removedVspace
    \centering
    \scalebox{1.0}{
    \begin{tabular}{c|ccccc}
    \hline
        \multicolumn{2}{c}{{Privacy budget $\varepsilon$ $\downarrow$}} & 3.5    & 6.0	 & 7.7	  & 10.0\\
        \hline
        \multirow{2}*{FedKD}
        & w/ DP $\uparrow$         &45.64	&56.08	&61.80	&70.90 \\
        & w/o DP $\uparrow$        &66.79   &79.30  &80.28  &81.55 \\\hline
        \multirow{2}*{FedIOD }
        & w/ DP $\uparrow$         &44.45	&58.96	&62.14	&73.58\\
        & w/o DP $\uparrow$        &74.31   &80.02  &82.03  &82.69 \\
    \hline
    \end{tabular}}
% %removedVspace
\noindent\caption{Compare FedIOD and FedKD in terms of accuracy (\%) on CIFAR10 ($K$=20, $\alpha$=1) under same privacy cost. %We set the privacy budget $\varepsilon$ based on the experience of prior arts.
}
\label{tab:privacy-utility}
% %removedVspace
\end{table}

\section{Privacy Analysis}
\textbf{Comparison with data-dependent distillation-based FL.}
The significant difference between ours and typical FL based on distillation is that FedIOD generates data for knowledge distillation, while others rely on auxiliary real data. We adopt the differential privacy (DP) analysis in DP-CGAN \cite{torkzadehmahani2019dp} and GS-WGAN \cite{chen2020gs} to measure the privacy cost of the gradients used to train the generator. For a fair comparison, we apply PATE \cite{papernot2018scalable} on the local model output and then transfer them to the server to satisfy DP for both FedIOD and our counterpart FedKD~\cite{gong2022preserving}. Table \ref{tab:privacy-utility} compares FedIOD with FedKD in terms of accuracy under a series of rigid differential privacy protections ($\varepsilon <$10).
%Our method satisfies $(\varepsilon, \delta)$-differential privacy by implementing clipping $\mathcal{C}$ and Gaussian-noise $\mathcal{N}(0,\sigma^2\mathcal{C}^2I)$ to the shared gradients.

% -------------- Table 2 DP Analysis -----------------------------------


\begin{figure}[h]
\centering
\includegraphics[width=0.95\linewidth]{fig/fig6.pdf}
% %removedVspace
\caption{ Comparison of FID scores between FedIOD and FedAvg on (a) 9 randomly selected local clients; and (b) average score under CIFAR10 ($K$=20, $\alpha$=1) FL setting. A larger FID indicates a stronger privacy guarantee.
}
\label{fig:fid}
\end{figure}

% \begin{figure}[h]
% \centering
% \includegraphics[width=\linewidth]{fig/fig5.pdf}
% %removedVspace
% \caption{ Comparisons of communication cost for (a) CIFAR10 ($K$=20, $\alpha$=0.1) classification; and (b) BraTS2018 segmentation to reach certain performance.
% }
% \label{fig:bandwidth}
% %removedVspace
% \end{figure}

 \textbf{Comparison with parameter-based FL.}
% Sharing parameters makes it vulnerable to white-box attacks \cite{chang2019cronus, zhu2019deep, geiping2020inverting}, while our distillation-based method only has black-box attack risks.
% Although it's intuitive that distillation-based FL is more secure than parameter-based FL, the synthetic images used in distillation-based FedIOD may raise privacy concerns.
% We use the similarity between synthetic images and privately held local data as a quantization of privacy leakage.
We use DLG \cite{zhu2019deep} as an attacker to recover private data using its iterative shared model parameters for parameter-based FL. We then measure the quality of the recovered data using Fréchet Inception Distance (FID).
We assume a larger FID, \ie, a larger distance between the recovered data and private data, indicates a stronger privacy guarantee. For our method, we measure the FID between the synthetic images and the private images. The comparison in Figure \ref{fig:fid} shows that our method has a much higher FID, thus far more privacy protected than the FL parameter-sharing method such as FedAvg \cite{mcmahan2017communication}.

\textit{Please refer to the ``Privacy Analysis'' section in the appendix for more details.}
% Figure~\ref{fig:bandwidth} compares the communication bandwidth of our FedIOD and the classic FL method like FedAvg.
% In particular, our proposed method outperforms the parameter sharing method \cite{mcmahan2017communication} and simultaneously provides a much more secure privacy guarantee. Figure~\ref{fig:bandwidth} shows that our method costs less or equivalent communication bandwidth compared to the parameter-based art \cite{mcmahan2017communication} .

%%removedVspace
\section{Conclusions}
In this work, we propose a novel federated learning framework, FedIOD, that protects local data privacy by distilling input and output to transfer knowledge from locals to the central server. To cope with the highly non-i.i.d. data distribution across local nodes, we learn the input on which each local achieves both consensual and unique results to represent individual heterogeneous expertise.
We conducted extensive experiments with natural and medical images on classification and segmentation tasks in a variety of real, in-the-wild, heterogeneous FL settings.
All demonstrate the efficacy of FedIOD, showing its superior privacy-utility trade-off to others and significant flexibility in FL scenarios without any prerequisite of prior knowledge or auxiliary real data.

\section{Acknowledgment}
This research was supported in part by Zhejiang Provincial Natural Science Foundation of China under Grant No. D24F020011, Beijing Natural Science Foundation L223024, National Natural Science Foundation of China under Grant 62076016, the National Key Research and Development Program of China (Grant No. 2023YFC3300029) and “One Thousand Plan” projects in Jiangxi Province Jxsg2023102268 and a generous gift from Amazon.

Deleniti ad reprehenderit laboriosam quidem laborum, quod id harum, ipsum nam excepturi voluptate.Earum error facilis quod dolorum laboriosam laborum temporibus excepturi consequuntur quia totam, quas tempore labore, sequi maxime necessitatibus optio consequatur voluptates natus quisquam nobis, perferendis optio assumenda amet vitae a labore in quam iure omnis.Voluptates voluptatum voluptate iste quasi provident, tempora neque molestiae sapiente ea quibusdam aut non nisi, porro quibusdam eveniet nobis, assumenda sapiente perspiciatis quidem repellat doloremque labore voluptas molestias ipsam, vel corrupti temporibus ad ipsum quas id velit quaerat alias est sit.Impedit nesciunt minima omnis tempora possimus quaerat recusandae beatae id delectus, quae amet accusamus, a non eaque quibusdam ipsum asperiores exercitationem corrupti illum, voluptatibus perspiciatis ut assumenda praesentium tempore maxime?Pariatur totam nesciunt consectetur, assumenda dicta exercitationem atque accusamus vero dignissimos, aut dignissimos voluptatibus natus dicta veniam eligendi modi earum vel?Atque animi enim porro iure odio beatae minima id dolores, repudiandae aliquid iure dolorum quibusdam, minus nobis quidem odio itaque corrupti laborum quae asperiores sed consectetur, blanditiis molestiae illum dolorum necessitatibus sit tenetur nobis?Dicta optio numquam provident eligendi doloribus dolorem ratione voluptatem, illum quod laboriosam consequuntur voluptas beatae quidem veritatis deserunt enim maxime, similique atque nulla, voluptatibus dolorum mollitia vel optio consequuntur ullam in molestias nostrum.Autem et vero possimus laudantium deleniti asperiores tenetur libero, porro dignissimos alias vel nulla aspernatur, illum pariatur neque sed modi consequuntur nesciunt quas, perferendis modi cupiditate molestiae deserunt, cupiditate odio amet sint expedita.Adipisci doloremque vitae voluptatum asperiores eius nisi illo, expedita sunt fugit nesciunt rem doloribus blanditiis voluptatum at ea fugiat, sunt reiciendis quidem perferendis fugiat aliquid laudantium.Vero dolor maiores, maxime aliquam iste fuga quaerat saepe provident cum libero consequuntur facilis in, consequatur ipsam omnis obcaecati recusandae impedit numquam magnam, animi officia veniam esse dolorum non odio ea similique accusantium?Odit fugiat quis, nihil repellat architecto totam facilis cupiditate dolorem magnam hic praesentium, libero cumque omnis deleniti esse quidem recusandae, incidunt tempora pariatur odio veniam quasi architecto hic at mollitia dolores.Qui ex quaerat nesciunt doloribus quas autem quibusdam aperiam, animi adipisci aspernatur impedit laudantium, mollitia earum sapiente eaque odio minima voluptatibus?Accusantium modi autem cupiditate consequatur recusandae sapiente, dolore tempore reiciendis aliquid praesentium inventore odit, cumque temporibus dignissimos ad quaerat nulla nisi nesciunt nihil repudiandae rem.Odit dignissimos eveniet ad nostrum quam quis tenetur quos sit quia, perferendis ipsam soluta fugit illo provident obcaecati tempora, eius dolorem reprehenderit ipsum quas?Minima praesentium iste ab saepe ut, at odit quisquam voluptate officiis recusandae magni, qui sed rerum magni maxime sapiente inventore, praesentium odio asperiores quo neque facilis autem facere, earum doloremque dolor accusantium minus sit voluptas animi et officia fuga?Error deserunt ea possimus illum amet iure expedita illo laboriosam veniam maxime, ducimus quo labore laudantium suscipit fugit quia nemo magnam dolore sint?Harum accusamus quo fuga pariatur consectetur explicabo, minus officia tenetur porro facere?Eligendi excepturi ab quia incidunt perspiciatis vitae aspernatur voluptas aperiam atque, nobis hic accusantium alias.Eum ad nihil, adipisci repellendus veritatis, recusandae facilis praesentium mollitia optio sapiente vero dolorem eos accusantium.Odit labore ducimus soluta voluptatum tenetur itaque esse dolore, incidunt expedita sequi eum accusantium modi veritatis beatae sit quidem, dolore sint voluptates aut.Exercitationem dolorem cupiditate magnam vitae iste reiciendis perspiciatis soluta eligendi, sit nihil tenetur aliquid qui consectetur modi, totam ipsum quisquam commodi, libero possimus officia suscipit dolore minima.Voluptas recusandae illum perspiciatis odio saepe, deleniti corporis libero eos autem modi.Voluptatum inventore nulla illum, enim delectus eaque, doloremque voluptates in explicabo assumenda atque sapiente dicta aspernatur ad et voluptatibus?Ea tenetur quasi dignissimos illum quibusdam nemo labore id numquam neque laborum, molestias aliquam obcaecati optio laboriosam molestiae maxime, dolorum error laboriosam consectetur sed fuga aliquam voluptate inventore nam, officiis iste hic, neque accusantium illum repudiandae sed suscipit explicabo eos consequuntur magnam placeat animi.Odit quisquam consequuntur deleniti aut debitis at aliquid, temporibus cum sed quos aliquid dignissimos id exercitationem velit doloremque, ex reprehenderit voluptatem in molestias pariatur dolorum iste vitae labore?Debitis quisquam fugiat ex, dolores et dignissimos sapiente, facilis praesentium sed soluta veniam assumenda, nihil suscipit vitae iure asperiores porro fuga, ipsum aliquam mollitia eius obcaecati laborum.Perferendis repellat neque ipsam iure laudantium dolore harum dolorem ex voluptatibus in, ea veritatis fugiat consequuntur quasi labore quibusdam error voluptate consectetur, mollitia omnis dolor officiis illum magni beatae, at ipsum mollitia possimus ratione eius neque tenetur voluptate?Vitae quae illum facilis quia, aut saepe accusamus sit repellendus reprehenderit quod cum praesentium sapiente et.Labore explicabo delectus sequi earum tempora deserunt minima quod quos laudantium sapiente, est aspernatur voluptas.Incidunt error sed culpa temporibus cum molestiae aliquid, deserunt obcaecati sit tempore iure architecto optio quis?Itaque aliquam id sunt aliquid veniam distinctio ratione accusantium, aperiam assumenda esse repellat sapiente blanditiis expedita distinctio ullam reprehenderit, veniam ipsa dolor praesentium fugit harum quia delectus perspiciatis?Itaque corporis animi quidem rerum odio nemo quas enim quia, dolore deleniti tempore, repellat eveniet nesciunt accusantium laboriosam ex ipsam, officiis ipsa ex quos facilis aperiam?Reprehenderit soluta modi nobis omnis explicabo fugiat ipsam incidunt optio reiciendis, id excepturi eveniet eius blanditiis sint, magni dicta adipisci voluptate ipsa, magni culpa corporis libero illum animi ex cupiditate accusamus veritatis ipsum pariatur, quisquam harum itaque?Iure qui aut dolores minus, voluptatum eos sunt, tempore corrupti eveniet, aliquam laudantium voluptatibus quam aspernatur doloremque labore sint repellat pariatur doloribus harum, dolorem aut doloremque eum nulla dolore vero.Repudiandae animi eaque eligendi quasi fugiat optio consequatur, enim asperiores perspiciatis doloribus beatae eaque optio maxime possimus repellat at exercitationem, sequi quidem obcaecati animi non itaque assumenda corrupti id commodi.Aliquam illum commodi exercitationem magni pariatur maxime totam repudiandae, enim voluptate deserunt facere eligendi vel quidem ipsam modi laboriosam, dolorum dolore fugit commodi, ducimus ea quisquam atque fugit vel quae nisi earum.Neque corrupti minima quaerat, quidem veniam dolores, quaerat explicabo eius placeat fugiat?Facere distinctio in atque quo quisquam, numquam nulla eveniet esse dolorem modi facilis praesentium sit, architecto praesentium a ex, deleniti aliquam iusto ducimus vel, eligendi suscipit voluptatem doloremque ad tempora unde possimus at?Enim ut veritatis totam ipsa sapiente libero aspernatur, nesciunt eius sit pariatur quo odio modi, dolore dolorem fuga, quidem minus quis tempore aliquam magnam, vel maxime exercitationem totam accusantium magnam distinctio nisi atque corrupti odio.Dolore nesciunt vel quaerat error dicta eligendi dolores, molestias qui maiores sit necessitatibus enim quam facilis praesentium reiciendis iure possimus.Autem modi natus architecto voluptate ipsam ex, aliquid dolor sequi dicta, aperiam amet omnis consequatur tempora nesciunt et corporis illo fugit tenetur accusantium, molestiae dolor dignissimos sed quae ab?Minus nulla quam minima facilis modi atque blanditiis officiis voluptatibus temporibus, possimus quas ipsa autem consectetur tempore animi laborum impedit culpa perspiciatis, ipsa corrupti sequi, ab est itaque deserunt accusantium?Quo magni totam obcaecati magnam sed, quas itaque accusamus ex commodi et consequatur enim dolore harum saepe, expedita natus dignissimos, laborum tempore totam qui fugiat repellendus esse excepturi accusamus.Ad reiciendis perferendis esse, odit fugiat harum, corporis similique in numquam nostrum placeat, sed saepe perspiciatis velit voluptatum esse sint dolores natus, illum quos laborum enim at debitis perferendis ea quas.Error aspernatur fugiat explicabo velit, fugit voluptatem alias sint dolores obcaecati cupiditate repellendus culpa ut nisi?Magni voluptas expedita, iste cum quia earum repudiandae?Perspiciatis doloremque suscipit incidunt, officia iste provident iusto quo excepturi enim labore, odio culpa maxime beatae recusandae eius magnam quam suscipit pariatur maiores quo, voluptas exercitationem molestiae.Doloremque tenetur rem, quos a nemo facilis quasi aut minima sed repellat quae laudantium?Assumenda error ullam numquam, placeat dolorem incidunt consectetur id ratione doloribus nulla.Labore deserunt quaerat aperiam nostrum quas reiciendis non laboriosam est dolorem at, ducimus earum natus fuga quae, in sed odit perspiciatis, delectus dolore dignissimos aut quis repudiandae, necessitatibus perspiciatis tempore nobis magnam at sed repellendus?Natus cum alias deleniti fugit quibusdam iure repellendus a tempora qui libero, commodi ad illum eligendi provident nisi?Fugit repellat blanditiis vitae excepturi consequuntur enim quidem dolor asperiores, quo tempore ipsam minus.Quasi recusandae nulla debitis neque labore nam numquam in sint, pariatur officia sapiente voluptate, vitae facere iusto?Odit velit est commodi dolore sit eveniet tempora accusamus repudiandae, vel sit id eligendi possimus nemo in culpa, similique ut ab incidunt?Incidunt sapiente consectetur, quidem sunt perferendis, eveniet obcaecati explicabo eius, architecto sint animi quis, ullam incidunt obcaecati mollitia eius numquam ea ipsam ab necessitatibus.Ad eveniet incidunt debitis blanditiis accusamus voluptates sed, harum corrupti facilis sit vel odit, labore alias magnam aut mollitia doloremque placeat, commodi earum tempore vel sint?Provident quo illo, expedita consequatur sunt laboriosam illum facere excepturi necessitatibus voluptatem ea cupiditate commodi, deleniti laborum qui ullam vitae accusantium repellendus deserunt, at officia reiciendis similique saepe, a quos dolor necessitatibus deleniti cum.Aliquid cumque dolores consequatur labore, cumque explicabo eaque, eveniet quam quisquam vero asperiores sint eligendi assumenda adipisci excepturi veritatis molestiae.Doloribus blanditiis provident iure porro molestias nulla in quam, distinctio ea inventore architecto perferendis, excepturi eligendi tempora cupiditate nihil consequatur corrupti, quasi velit suscipit, suscipit et minima aliquid possimus?Blanditiis nihil consequuntur amet vero totam tempora, hic ducimus molestias nihil nobis sapiente odio expedita sint cumque non.A beatae dolor incidunt possimus atque aspernatur illo distinctio animi dignissimos obcaecati, omnis voluptate error ad doloribus sint voluptates velit iusto vitae, obcaecati quod aperiam eos tempore earum veniam suscipit repellendus.Temporibus deleniti laboriosam dignissimos provident suscipit adipisci dolorum, dolor dolorum sint soluta natus ab, necessitatibus mollitia magni dolorem pariatur atque illum, molestias incidunt tempore at nihil, corrupti voluptatibus quo deserunt sunt illo pariatur rerum quis veritatis perspiciatis.Vel mollitia odio quasi tempora rem ut harum architecto sit, iusto accusantium tempora aliquam dolorem fugiat ea voluptatibus impedit molestias minus numquam, neque suscipit in provident harum nam maxime fugiat praesentium sit officiis, rerum a eius molestias quia culpa reprehenderit architecto dicta, incidunt impedit dolores ut voluptate labore ipsa facere eum?Praesentium neque enim illo eos commodi quia necessitatibus, totam natus accusamus eligendi voluptatibus dolorem aliquam architecto autem praesentium, omnis qui maxime inventore nam minima nobis quia eius quaerat assumenda rerum, vitae rerum assumenda maxime doloribus illo architecto non accusantium minima ipsum eos, dicta molestias expedita sequi tenetur pariatur aliquam beatae aperiam voluptas.Eveniet atque officiis porro, excepturi officia similique atque nihil, modi ex nesciunt perferendis officiis, aliquid sit deserunt itaque sed, vel eligendi ratione amet odit.Eligendi doloribus quas commodi reiciendis unde ipsum quis amet at tenetur maxime, ratione amet officia neque necessitatibus esse hic aut aspernatur nisi dolor rerum, nesciunt atque iure laboriosam dicta asperiores, molestiae perferendis rem assumenda eveniet tenetur quasi ratione debitis quas nostrum beatae, est quod dignissimos ducimus cupiditate.Doloribus nemo repudiandae aspernatur atque esse, necessitatibus debitis fugit in magni eius quas expedita minima, maxime repudiandae dignissimos illum quis, tempore a iste harum labore quo delectus porro aliquid sed?Ipsam fuga doloremque vel dolore, quis velit laborum atque amet, nemo error quaerat libero atque iusto consequatur.Nisi quaerat laudantium vel ipsum minima aut velit doloremque sapiente sit eligendi, velit voluptates atque error porro fugiat, nulla eos asperiores exercitationem consequatur unde blanditiis, rem aspernatur soluta vel esse, molestiae architecto distinctio?Delectus illo voluptates aut obcaecati magnam culpa ullam cupiditate, impedit omnis suscipit praesentium facere aut ex neque ab quasi consectetur quas, perferendis facere expedita nulla placeat excepturi ipsa repellat.Mollitia quia quis a ex dolore tempore perferendis quaerat dicta harum eius, tempore mollitia ut a quae cupiditate dolore repellendus quia, dolore perferendis debitis delectus officiis reiciendis porro nihil vero omnis sunt quaerat, aspernatur dolorum soluta est reiciendis quas ab commodi, explicabo eaque quos amet veniam deserunt recusandae autem vero esse?Expedita consectetur inventore quis unde a iste accusamus, ab voluptate quisquam asperiores inventore impedit rem harum voluptates?Quos ducimus quae eius dolorum amet voluptatibus dolor, ipsam ullam quia adipisci fugiat accusantium harum vero, doloremque praesentium nobis amet saepe odio possimus pariatur sed deleniti, rem eaque suscipit doloremque sed nam vero ratione non, in iste placeat doloremque pariatur excepturi aliquid ullam cupiditate?Deserunt exercitationem quae ad itaque voluptatem libero recusandae modi minus enim impedit, quae quod aperiam suscipit rerum officia tempore odio voluptate ex earum beatae?Labore neque molestias illum fugit, facilis placeat ducimus dolore impedit provident illo, blanditiis numquam vero sint ducimus nostrum minima.Eius placeat dolor consequuntur culpa quaerat, expedita eum ea quod inventore dolorem quia sint voluptas ipsum, minima nemo voluptatum?Impedit nisi nihil ipsa non ipsam, obcaecati ipsum alias fuga sequi modi consectetur deleniti ducimus, alias aspernatur voluptates quasi, possimus accusantium aliquid laudantium officia asperiores quaerat quis eius velit quae expedita?Quas dolore omnis quaerat quod accusamus reprehenderit tempore rem atque eligendi iste, laborum nesciunt hic sed vero eos ducimus illo, exercitationem quis ipsam neque rerum doloribus mollitia, excepturi eligendi ad?Dolore deleniti nam distinctio omnis est animi vero perferendis, temporibus animi recusandae at sint voluptatibus culpa neque id illo, quam deleniti et esse error officiis.Nulla tempora deleniti illum quas ad quasi quae sequi, repellat sunt veniam illo saepe iusto molestiae obcaecati aspernatur officiis facere non, quod at saepe.Obcaecati architecto assumenda sapiente eius, voluptatibus quis quidem officia.Doloribus recusandae cumque, ipsum tenetur accusamus mollitia facilis consequuntur similique, repellendus sed ab quae ex odit odio natus culpa ullam aliquid tempore?\clearpage
\bibliography{aaai24}




%%%%%%%%%%%%%%%%%%%%%%%%%%%%%%%%%%%%%%%%%%%%%%%%%%%%%%%%%%%%


% \newpage

\section*{\LARGE\textbf Appendix}

We provide materials supplementing the main manuscript, including the implementation details as well as some additional experiment results. %Then we compare the synthetic transfer data between the baseline method and the proposed method both quantitatively and qualitatively.

\section{CIFAR-10/100 Classification}
The network of $G$ and $D_k$ adopt the same architecture of the generator and discriminator as that in \cite{fang2021mosaicking}. For a fair comparison, the network of $T_k$ and $S$ employ ResNet-8 following the prior art \cite{lin2020ensemble}.

In the first stage, we train each local task model $T_k$ individually with SGD as optimizer and 0.0025 as learning rate. We adopt cross-entropy loss function and a batch size of 16 for 500 epochs.
In the second stage, we update the generator $G$, discriminators $D_k$, and the central model $S$ simultaneously. We use Adam optimizer and Cosine Annealing decreasing the learning rate from 0.001 to 0 with a batch size of 64 for 300 epochs. We conduct an additional ablation study in table \ref{tab:cifarablation} to demonstrate the efficacy of each proposed module.

In Table \ref{tab:cifars1}, we further show quantitative comparisons of the inception score (IS) of the synthetic transfer data. Typical inception score use InceptionNet \cite{szegedy2017inception} pretrained on ImageNet\cite{deng2009imagenet} to compute KL-divergence between the conditional and marginal probability distributions of the output. We adapt the inception score by inferring generated data with the locally trained model $T_k$ to evaluate its quality (each sample strongly classified as one class) and diversity (the overall probability of the generated data on each class of  $T_k$ tends to have even distribution).

\begin{table}[b]
\begin{center}
\resizebox{\columnwidth}{!}
% \scalebox{0.90}
% \scriptsize
{
\begin{tabular}{cc|cccccccc}

\hline
$L_\text{gan}$  &$\pi_k$ &Avg &Eq.2 &Eq.2 &Eq.2 &Eq.2 &Eq.2 &Eq.2 &Eq.2\\ \hline
\multirow{2}{*}{$L_\text{mimic}$} &$\tau$ &\multirow{2}{*}{\xmark} &\multirow{2}{*}{\xmark} &1 & 1 &$\infty$ &$\infty$ &$\infty$ &$\infty$\\
&$\pi_k$ & &  &$*$ &Eq.8 &Eq.8 &Eq.8 &Eq.8 &Eq.8 \\\hline
\multicolumn{2}{c|}{$L_\text{conf}$} &\xmark &\xmark &\xmark &\xmark &\xmark &\cmark &\xmark &\cmark\\\hline
\multicolumn{2}{c|}{$L_\text{unique}$} &\xmark &\xmark &\xmark &\xmark &\xmark &\xmark &\cmark &\cmark\\ \hline
\multicolumn{2}{c|}{Acc. (\%)} &57.03 &65.05 &59.88 &65.85 &68.49 &69.21 &69.67 &70.36 \\
\hline
\end{tabular}}
% %removedVspace
\end{center}
\caption{Ablation study on CIFAR-10 with ResNet-8, $K$=20 and $\alpha$=0.1. For the training of $L_\text{gan}$, we compare our weighting scheme (Eq. 2) with the typical average ensemble.  For the ensemble scheme of $L_\text{mimic}$, we compare our per-sample, per-class importance weighting (Eq. 8) with $*$ which represents the weighting scheme commonly used in other FL methods \cite{lin2020ensemble, hsu2020federated}. To compare $\tau$, we only list the result with a typical value $\tau$=1~\cite{hinton2015distilling}.}
\label{tab:cifarablation}
% %removedVspace
\end{table}

\begin{table}[t]
\begin{center}
\resizebox{\columnwidth}{!}
% \scalebox{0.90}
% \scriptsize
{
\begin{tabular}{c|ccccc}
\hline
$L_\text{gan}(G)$  &\cmark &\cmark &\cmark &\cmark &\cmark \\\hline
{$L_\text{mimic}(G)$} &\xmark  &$*$ &Eq.8  &Eq.8  &Eq.8 \\\hline
{$L_\text{conf}(G)$} &\xmark &\xmark &\xmark  &\cmark &\cmark\\\hline
{$L_\text{unique}(G)$} &\xmark &\xmark &\xmark  &\xmark &\cmark\\\hline
{Inception Score $\uparrow$} &2.40 &2.90 &2.95 &2.64 &3.57 \\
{Adapted Inception Score $\uparrow$} &2.30 &2.19 &2.35 &2.74 &2.82 \\
\hline
\end{tabular}}

\end{center}
\caption{Ablation study on the fidelity of the generated data, with $K$=20 and $\alpha$=0.1 on CIFAR-10. For the ensemble scheme of $L_\text{mimic}$, we compare our per-sample per-class importance weighting (Eq. 8) with the weighting scheme commonly used in other FL methods (represented with $*$) \cite{lin2020ensemble, hsu2020federated}. We compare both typical inception score and adapted inception score  evaluated by each locally trained model $T_k$ (taking average of all local models).}
\label{tab:cifars1}
\end{table}
% %removedVspace


\begin{figure}[h]
\centering
\includegraphics[width=\linewidth]{fig/sfig1.pdf}
\caption{Visualization of testing results on BraTS2018 dataset with $K$=10. We compare ours with AsynDGAN~\cite{chang2020synthetic} and FedAvg~\cite{mcmahan2017communication}.  We highlight the contours extracted from each method's segmentation prediction as well as the ground-truth. The zoomed part is shown at the left-bottom of each image and demonstrates that our method achieves much closer prediction to the ground-truth.
}
\label{fig:s1}
% %removedVspace
\end{figure}

\section{Magnetic resonance image segmentation}
The 2018 Multimodal Brain Tumor Segmentation Challenge (BraTS 2018)~\cite{menze2014multimodal, bakas2018identifying} contains multi-parametric preoperative magnetic resonance imaging scans of 285 subjects with brain tumors, including 210 high-grade glioma (HGG) and 75 low-grade gliomas (LGG) subjects.
Each subject was associated with voxel-level annotations of “whole tumor”, “tumor core”, and “enhancing tumor”. Each subject was scanned under the T1-weighted, T1-weighted with contrast enhancement, T2-weighted, and T2 fluid-attenuated inversion recovery (T2-FLAIR) modalities. Following the experimental protocol of one prior art \cite{chang2020synthetic}, we deploy 2D segmentation of the whole tumor on T2 images of HGG cases, among which 170 were for training and 40 for testing. The local data split also follows \cite{chang2020synthetic}: we first sort the training cases with tumor size and then divide the training set into ten subsets distributed to 10 local nodes. Overall there are 11,057 slices as training images across all local nodes and 2,616 slices as testing images. Following \cite{chang2020synthetic}, the network structure of $G$ employs a 9-block ResNet \cite{he2016deep}, and each discriminator $D_k$ employs the same structure as the patch discriminator in \cite{isola2017image}. The segmentation net for $T_k$ and $S$ follow the same U-Net \cite{ronneberger2015u} structure as that in \cite{chang2020synthetic}.

In the first stage of local training,
we employ Adam optimizer and a learning rate of 0.002 to train each local model $T_k$ using cross-entropy loss and dice loss. The batch size is 16 and the total number of training epochs is 50. During training, we crop and resize the image to $224 \times 224$ following the same procedure as that in \cite{chang2020synthetic}. In the second stage of distillation, we use the label condition with size $240 \times 240$ following \cite{chang2020synthetic} and improve our $L_\text{gan}$ with additional perceptual loss \cite{johnson2016perceptual}.
We adopt the Adam optimizer with a learning rate of 0.0002 and batch size of 2 for 400 epochs. We randomly crop the generated image $x$ to $224 \times 224$ and randomly rotate and flip images as data augmentation during distillation.

The Dice score, sensitivity, specificity, and Hausdorff distance are used as evaluation metrics. Taking $\bm{y}, \hat{\bm{y}} \in \{0,1 \}^{H \times W}$ as the ground-truth mask and the segmentation prediction, respectively, Dice evaluates the overlap between the two: $\text{Dice}(\bm{y}, \hat{\bm{y}}) = {2|\bm{y} \cap \hat{\bm{y}} |} / {(|\bm{y}|+|\hat{\bm{y}}|)}$. Sensitivity represents the true positive rate: $\text{Sens}(\bm{y}, \hat{\bm{y}}) = {|\bm{y} \cap \hat{\bm{y}} |} / {|\bm{y}|}$, and specificity represents the true negative rate: $\text{Spec}(\bm{y}, \hat{\bm{y}}) = {|(1-\bm{y}) \cap (1-\hat{\bm{y}}) |} / {|1-\bm{y}|}$. The Hausdorff distance evaluates the shape similarity:
\begin{equation}
\label{eq:hausdorff}
    \text{HD}(\bm{y},\hat{\bm{y}}) = \max \{\sup_{\bm{u} \in \partial{\bm{y}}} \inf_{\bm{\hat{u}} \in \partial{\bm{\hat{y}}}}  |\bm{u}-\hat{\bm{u}}|, \sup_{\bm{\hat{u}}\in \partial{\bm{\hat{y}}}} \inf_{\bm{u} \in \partial{\bm{y}}}    |\bm{u}-\hat{\bm{u}}| \},
\end{equation}
where $\partial$ indicates boundary extraction and returns boundary position sets. ``HD95'' represents 95\% quantile of the distances instead of the maximum.

In Figure \ref{fig:s1}, we show qualitative results on the segmentation performance of central segmentation net $S$. We can note that our method outperforms the other two counterparts \cite{mcmahan2017communication, chang2020synthetic} on tumor shape segmentation with much more closer prediction to the ground-truth.


\section{Histopathological image segmentation}
\subsection{Datasets}
\textbf{TCGA:}
The TCGA dataset~\cite{kumar2017dataset}  was captured from the Cancer Genome Atlas  archive and used in MICCAI 2018 multi-organ segmentation challenge (MoNuSeg). The training set consists of 30 images and around 22,000 nuclei instance annotations, while the test set includes 14 images with additional 7000 nuclei boundary annotations. The images are with $1000$ $\times$ $1000$ pixels and captured at 40$\times$  magnification on hematoxylin and eosin (H\&E) stained tissue. These images show highly varying properties from 18 hospitals and seven organs (breast, liver, kidney, prostate, bladder, colon, and stomach).

\noindent \textbf{Cell17:}
The MICCAI 2017 Digital Pathology Challenge dataset~\cite{vu2019methods} (Cell17)  consists of 64 H\&E stained histology images. Both the training and testing sets contain 32 images from four different diseases: glioblastoma multiforme (GBM), lower-grade glioma (LGG) tumors, head, and neck squamous cell carcinoma (HNSCC), and non-small cell lung cancer (NSCLC). The image sizes are either $500 \times 500$ or $600 \times 600$ at $20\times$ or $40\times$ magnification.

\noindent \textbf{TNBC:}
The Triple Negative Breast Cancer (TNBC)~\cite{naylor2018segmentation} dataset consists of 50 annotated $512 \times 512$ images at $40\times$ magnification. The images are sampled from 11 patients at the Curie Institute, with three to eight images for each patient. Overall there are  4022 annotated cell instances. The image data includes low cellularity regions, which can be stromal areas or adipose tissue, and high cellularity areas consisting of invasive breast carcinoma cells.


\begin{table*}
\centering
% \fontsize{9.0pt}{10.0pt} \selectfont
\scalebox{0.9}
% \resizebox{\textwidth}{!}
% \scriptsize
{
\begin{tabular}{cccccc|cccc}
\toprule
%&\multirow{3}{*}{\rotatebox[origin=c]{0}{{Privacy}}} &\multirow{3}{*}{\rotatebox[origin=c]{0}{{Prerequisite}}}
Train
&Test   &\multirow{2}{*}{Dice(\%)$\uparrow$} &\multirow{2}{*}{Obj-Dice(\%)$\uparrow$} &\multirow{2}{*}{{AJI(\%)$\uparrow$}} &\multirow{2}{*}{HD95(pixel)$\downarrow$} &\multicolumn{4}{c}{Average}\\
Data &Data & & & &  &Dice(\%)$\uparrow$ &{Obj-Dice(\%)$\uparrow$} &{AJI(\%)$\uparrow$} &HD95(pixel)$\downarrow$\\
\midrule
% \parbox[t]{1mm}{\multirow{9}{*}{\rotatebox[origin=c]{90}{\shortstack[c]{Standalone}}}}
%\multirow{9}{*}{Standalone}
\multirow{3}{*}{Cell17}  &\cellcolor{gray0}{Cell17} &\cellcolor{gray0}{85.90} &\cellcolor{gray0}{67.24} &\cellcolor{gray0}{44.26} &\cellcolor{gray0}{8.21} &\multirow{3}{*}{59.37}  &\multirow{3}{*}{50.59} &\multirow{3}{*}{27.42}  &\multirow{3}{*}{32.89}\\
&\cellcolor{gray1}{TCGA} &\cellcolor{gray1}{43.03} &\cellcolor{gray1}{33.83} &\cellcolor{gray1}{11.75} &\cellcolor{gray1}{34.75} & & &  \\
&\cellcolor{gray2}{TNBC} &\cellcolor{gray2}{49.18} &\cellcolor{gray2}{50.70} &\cellcolor{gray2}{26.26} &\cellcolor{gray2}{55.71} & & &  \\ \midrule
\multirow{3}{*}{TCGA}   &\cellcolor{gray0}{Cell17} &\cellcolor{gray0}{55.10} &\cellcolor{gray0}{48.25} &\cellcolor{gray0}{29.42} &\cellcolor{gray0}{31.94} &\multirow{3}{*}{49.20}  &\multirow{3}{*}{48.41} &\multirow{3}{*}{32.94} &\multirow{3}{*}{57.07}  \\
&\cellcolor{gray1}{TCGA} &\cellcolor{gray1}{75.75} &\cellcolor{gray1}{71.58} &\cellcolor{gray1}{51.91} &\cellcolor{gray1}{13.14} & & &  \\
&\cellcolor{gray2}{TNBC}  &\cellcolor{gray2}{16.77} &\cellcolor{gray2}{25.42} &\cellcolor{gray2}{7.50} &\cellcolor{gray2}{126.14} & & &  \\ \midrule
\multirow{3}{*}{TNBC} &\cellcolor{gray0}{Cell17} &\cellcolor{gray0}{70.42} &\cellcolor{gray0}{56.79} &\cellcolor{gray0}{36.59} &\cellcolor{gray0}{24.33} &\multirow{3}{*}{61.36}  &\multirow{3}{*}{52.65} &\multirow{3}{*}{33.35} &\multirow{3}{*}{27.78} \\
&\cellcolor{gray1}{TCGA} &\cellcolor{gray1}{35.59} &\cellcolor{gray1}{24.32} &\cellcolor{gray1}{5.65} &\cellcolor{gray1}{35.31} & & &  \\
&\cellcolor{gray2}{TNBC} &\cellcolor{gray2}{78.08} &\cellcolor{gray2}{76.84} &\cellcolor{gray2}{57.81} &\cellcolor{gray2}{23.72} & & &  \\
\bottomrule
\end{tabular}}
\caption{The performance of locally trained models under the cross-site cross-organ nuclei segmentation setting with Cell17~\cite{vu2019methods}, TCGA~\cite{kumar2017dataset}, TNBC~\cite{naylor2018segmentation} as distributed local data.
}
% %removedVspace
\label{tab:nucleis2}
\end{table*}

\begin{figure}[h]
\centering
\includegraphics[width=\linewidth]{fig/sfig2.pdf}
\caption{Visualization of synthetic data on cross-organ TCGA dataset with $K$=4. We compare ours with AsynDGAN~\cite{chang2020synthetic}.  We zoom the instance region at the left-bottom of each image where our method succeeds to generate corresponding nuclei instances while the counterpart fails.
}
\label{fig:s2}
% %removedVspace
\end{figure}

\subsection{Implementation details and results}
Following \cite{chang2020synthetic}, the network structure of $G$ employs a 9-block ResNet \cite{he2016deep}, and each discriminator $D_k$ employs the same structure as the patch discriminator in \cite{isola2017image}. The segmentation net for $T_k$ and $S$ follow the same U-Net \cite{ronneberger2015u} structure as that in \cite{chang2020synthetic}.

In the first stage of local training, each local model $T_k$ is trained with Adam optimizer and a constant learning rate of $2.5\times10^{-4}$. The batch size is 4 and the overall number of training epochs is 100. We employ weighted cross-entropy loss where the foreground and the contour region are given more weight than the background region.
The data augmentation during training includes random rotation, random cropping ($256 \times 256$), and random flip both horizontally and vertically.
In the second stage of distillation, we use the same label as condition with size $256 \times 256$ following \cite{chang2020synthetic} and improve our $L_\text{gan}$ with additional perceptual loss \cite{johnson2016perceptual}. $G$ and $D_k$ are pretrained for 100 epochs and then trained together with $S$ for another 300 epochs. We use Adam optimizer with a learning rate of 0.0001 and batch size of 8.


Figure \ref{fig:s2} shows the visualization of synthetic images under the cross-organ experiment setting. From the comparisons of the highlighted region, we can note that our synthetic data used for knowledge transfer achieves much better qualitative results (clear instance generation given the instance contour) over the counterpart \cite{chang2020synthetic}. Table \ref{tab:nucleis2} shows the performance of locally trained models under the cross-site cross-organ nuclei segmentation (corresponding to Table 4 in the main manuscript).

\begin{figure*}
\centering
\includegraphics[width=\linewidth]{fig/sfig3.pdf}
\caption{Visualization of MRI image reconstruction with IXI, BraTS2020 and fastMRI as locally held data. We compare ours with two other FL methods: FedAvg and FL-MRCM. Each FL method trains with T1/T2-
weighted IXI, BraTS2020, fastMRI as local data and tests on T1  IXI, T2  IXI, T1 BraTS2020, T2 BraTS2020, T1 fastMRI, T2 fastMRI test set respectively. The second column of each sub-figure is the error map (absolute difference) between the reconstructed images and the ground truth (GT).
}

\label{fig:s3}
% %removedVspace
\end{figure*}


\section{Brain MRI reconstruction}
%Crucially, our method is by no means limited to just classification/segmentation problems.
The proposed FedIOD framework can be used for other tasks, \eg, magnetic resonance image reconstruction. Following the prior-art experiment protocol \cite{Guo_2021_CVPR, gong2022federated}, we use fastMRI \cite{zbontar2018fastMRI}, IXI \footnote{\url{https://brain-development.org/}}, BraTS\cite{hdtd-5j88-20} as private data distributed across local nodes and evaluate the corresponding test sets.

% \noindent\textbf{fastMRI} \cite{zbontar2018fastMRI}:  For fastMRI T1-weighted images, we use 2,583 subjects for training and 860 for testing. For T2-weighted images, 2,874 subjects are used for training and 958 for testing. Each subject consists of approximately 15 axial cross-sectional images of brain tissues.

% \noindent\textbf{BraTS2020} \cite{hdtd-5j88-20}: BraTS2020 consists of 494 subjects for both  T1 and T2-weighted modalities. There are 369 subjects for training and 125 subjects for testing. Each subject includes approximately 120 axial cross-sectional images of brain tissues for both modalities.

% \noindent\textbf{IXI} \cite{ixi}:
% IXI T1-weighted images include 436, 55, and 90 subjects for training, validation, and testing, respectively.
% For the T2-weighted modality, there are 434, 55, and 89 subjects for training, validation, and testing.  Each subject includes approximately 150 and 130 axial cross-sectional images of brain tissues for T1 and T2-weighted respectively.

\begin{table}
% %removedVspace
\centering
% \scalebox{0.7}
\resizebox{\columnwidth}{!}
% \scriptsize
{
\begin{tabular}{cccc|cc|cc}
\toprule
%&\multirow{3}{*}{\rotatebox[origin=c]{0}{{Privacy}}} &\multirow{3}{*}{\rotatebox[origin=c]{0}{{Prerequisite}}}
&\multirow{2}{*}{NO shared } &\multirow{3}{*}{Prerequisite}
&Test &\multicolumn{2}{c|}{{{T1-weighted}}} &\multicolumn{2}{c}{{{T2-weighted}}}  \\
\cmidrule{5-8}
% & & & & & & & & \\
&Param. &  &{Data} &{SSIM$\uparrow$} &{PSNR$\uparrow$} &{SSIM$\uparrow$} &{PSNR$\uparrow$}  \\
\midrule
%\parbox[t]{1mm}{\multirow{4}{*}{\rotatebox[origin=c]{0}{\shortstack[c]{\cite{Guo_2021_CVPR}}}}}
\multirow{3}{*}{FedAvg}
&\multirow{3}{*}{\xmark} %&\multirow{3}{*}{\rotatebox[origin=c]{90}{\shortstack[c]{Online}}}
&{Identical} &\cellcolor{gray0}B &\cellcolor{gray0}{0.9317} &\cellcolor{gray0}{34.91}
% &\multirow{3}{*}{0.9447} &\multirow{3}{*}{34.59}
&\cellcolor{gray0}{0.9173} &\cellcolor{gray0}{30.68}
% &\multirow{3}{*}0 &\multirow{3}{*}0
\\
& &{model}  &\cellcolor{gray1}F  &\cellcolor{gray1}{0.8803} &\cellcolor{gray1}30.48 &\cellcolor{gray1}{0.8782} &\cellcolor{gray1}{30.09} \\
& &{structure} &\cellcolor{gray2}I &\cellcolor{gray2}{0.9232} &\cellcolor{gray2}{32.31} &\cellcolor{gray2}0.8597 &\cellcolor{gray2}29.89 \\
\midrule

\multirow{3}{*}{FL-MRCM}
&\multirow{3}{*}{\xmark} %&\multirow{3}{*}{\rotatebox[origin=c]{90}{\shortstack[c]{Online}}}
&{Identical} &\cellcolor{gray0}B &\cellcolor{gray0}{0.9577} &\cellcolor{gray0}{36.88}
% &\multirow{3}{*}{0.9447} &\multirow{3}{*}{34.59}
&\cellcolor{gray0}{0.9308} &\cellcolor{gray0}{34.28}
% &\multirow{3}{*}0 &\multirow{3}{*}0
\\
& &{model} &\cellcolor{gray1}F &\cellcolor{gray1}0.9023 &\cellcolor{gray1}{33.63} &\cellcolor{gray1}0.8974 &\cellcolor{gray1}31.24 \\
& &{structure} &\cellcolor{gray2}I &\cellcolor{gray2}{0.9362} &\cellcolor{gray2}{33.29} &\cellcolor{gray2}0.8778 &\cellcolor{gray2}30.44  \\
\midrule \midrule

%\parbox[t]{1mm}{\multirow{4}{*}{\rotatebox[origin=c]{0}{\shortstack[c]{\cite{gong2022federated}}}}}
\multirow{3}{*}{FedAD}
&\multirow{3}{*}{\cmark} %&\multirow{3}{*}{\rotatebox[origin=c]{90}{\shortstack[c]{Offline}}}
&Auxiliary &\cellcolor{gray0}B &\cellcolor{gray0}0.9111 &\cellcolor{gray0}34.55
% &\multirow{3}{*}{0.9155} &\multirow{3}{*}{33.21}
&\cellcolor{gray0}0.9199 &\cellcolor{gray0}34.06
% &\multirow{3}{*}0 &\multirow{3}{*}0
\\
& &task-relevant  &\cellcolor{gray1}F &\cellcolor{gray1}{0.9182} &\cellcolor{gray1}33.37 &\cellcolor{gray1}{0.9374} &\cellcolor{gray1}{32.76}\\
& &image  &\cellcolor{gray2}I &\cellcolor{gray2}0.9173 &\cellcolor{gray2}31.72  &\cellcolor{gray2}{0.9058} &\cellcolor{gray2}{30.93}\\
\midrule
\multirow{3}{*}{FedIOD}
&\multirow{3}{*}{\cmark}
&Auxiliary &\cellcolor{gray0}B
&\cellcolor{gray0}{0.9326}  &\cellcolor{gray0}{36.08}
% &\multirow{3}{*}{0.9083} &\multirow{3}{*}{32.91}
&\cellcolor{gray0}0.9064 &\cellcolor{gray0}34.39
% &\multirow{3}{*}{90.57} &\multirow{3}{*}{34.39}
\\
& &task-relevant  &\cellcolor{gray1}F &\cellcolor{gray1}0.8725 &\cellcolor{gray1}30.53 &\cellcolor{gray1}0.9057 &\cellcolor{gray1}30.45  \\
& &label  &\cellcolor{gray2}I &\cellcolor{gray2}{0.9198} &\cellcolor{gray2}{32.15} &\cellcolor{gray2}0.8650 &\cellcolor{gray2}29.69 \\
\bottomrule
\end{tabular}}
% %removedVspace
\caption{Results on cross-domain MRI image reconstruction with fastMRI, BraTS2020, and IXI as locally held data (abbreviated as F, B, I respectively). We compare SSIM and PSNR with parameter-based methods, FedAvg and FL-MRCM,  as well as distillation-based prior art FedAD.
}
\label{tab:mriin}
% %removedVspace
\end{table}

We use the same preprocessing and U-Net \cite{ronneberger2015u} architecture for the reconstruction networks as \cite{Guo_2021_CVPR, gong2022federated}. %Comparison in Table \ref{tab:mriin} shows that our method achieves comparable results with the counterparts, including the latest state-of-the-art \cite{Guo_2021_CVPR, gong2022federated},
Compared to distillation-based methods, we achieve competitive results
with far fewer prerequisites: our counterpart \cite{gong2022federated} relies on additional brain MRI images with the same modalities for distillation. At the same time, ours only utilizes the contour of the foreground of the brain as a condition of $G$, demonstrating much more relaxation and flexibility.  In addition, our method achieves comparable SSIM and PSNR with parameter-based methods while simultaneously demonstrating other benefits, including protecting privacy by not sharing local parameters.
\subsection{Datasets}
\textbf{fastMRI} \cite{zbontar2018fastMRI}:  For fastMRI T1-weighted images, we use 2,583 subjects for training and 860 for testing. For T2-weighted images, 2,874 subjects are used for training and 958 for testing. Each subject consists of approximately 15 axial cross-sectional images of brain tissues.

\noindent \textbf{BraTS2020} \cite{hdtd-5j88-20}: BraTS2020 consists of 494 subjects for both  T1 and T2-weighted modalities. There are 369 subjects for training and 125 subjects for testing. Each subject includes approximately 120 axial cross-sectional images of brain tissues for both modalities.

\noindent \textbf{IXI}:
IXI T1-weighted images include 436, 55, and 90 subjects for training, validation, and testing, respectively.
For the T2-weighted modality there are 434, 55, and 89 subjects for training, validation, and testing.  Each subject includes approximately 150 and 130 axial cross-sectional images of brain tissues for T1 and T2-weighted respectively.

\subsection{Implementation details and results}
The architecture of $G$ and $D_k$ are the same as those used in brain tumor image segmentation. And the reconstruction network $T_k$ and $S$ follow the same U-Net architecture as that in \cite{Guo_2021_CVPR}.

For local training, we train each $T_k$ with Adam optimizer and a constant learning rate of 0.0001 for 20 epochs following \cite{gong2022federated}.
For the second stage of distillation, we update the networks with Adam optimizer and a constant learning rate of 0.0001 for 100 epochs. In Figure \ref{fig:s3} we show qualitative results of the reconstructed images as well as the comparisons with two other FL methods  \cite{mcmahan2017communication, Guo_2021_CVPR}.

\section{Privacy Analysis}
\textbf{Comparison with data-dependent distillation-based FL.}
The major difference between ours and typical FL based on distillation is that FedIOD generates data for knowledge distillation, while others rely on auxiliary real data. Although eliminating such a prerequisite of real data, the gradients backpropagated to train the generator might raise security concerns. To this point, we adopt the differential privacy (DP) analysis in DP-CGAN \cite{torkzadehmahani2019dp} and GS-WGAN \cite{chen2020gs} to measure the privacy cost of the gradients used to train the generator. By clipping and adding Gaussian noise to these gradients, it satisfies $(\varepsilon, \delta)$-differential privacy: it allows a small probability ($\delta = 10^{-5}$) for the privacy budget $\varepsilon$. For a fair comparison, we apply PATE \cite{papernot2018scalable} on the local model output and then transfer them to the server to satisfy DP for both FedIOD and our counterpart FedKD~\cite{gong2022preserving}. Table \ref{tab:privacy-utility} compares FedIOD with FedKD in terms of accuracy under a series of rigid differential privacy protections ($\varepsilon <$10). We can see that our method (a) eliminates the requirements of prior knowledge of the local task and task-relevant public data during federated distillation; (b) and at the same time achieves superior or equivalent performance to FedKD under the same privacy cost.
%Our method satisfies $(\varepsilon, \delta)$-differential privacy by implementing clipping $\mathcal{C}$ and Gaussian-noise $\mathcal{N}(0,\sigma^2\mathcal{C}^2I)$ to the shared gradients.

% -------------- Table 2 DP Analysis -----------------------------------
% \begin{table}[h]
% % %removedVspace
%     \centering
%     \scalebox{1.0}{
%     \begin{tabular}{c|ccccc}
%     \hline
%         \multicolumn{2}{c}{{Privacy budget $\varepsilon$ $\downarrow$}} & 3.5    & 6.0	 & 7.7	  & 10.0\\
%         \hline
%         \multirow{2}*{FedKD}
%         & w/ DP $\uparrow$         &45.64	&56.08	&61.80	&70.90 \\
%         & w/o DP $\uparrow$        &66.79   &79.30  &80.28  &81.55 \\\hline
%         \multirow{2}*{FedIOD }
%         & w/ DP $\uparrow$         &44.45	&58.96	&62.14	&73.58\\
%         & w/o DP $\uparrow$        &74.31   &80.02  &82.03  &82.69 \\
%     \hline
%     \end{tabular}}
% %removedVspace
% \noindent\caption{Compare FedIOD and FedKD in terms of accuracy (\%) on CIFAR10 ($K$=20, $\alpha$=1) under same privacy cost. %We set the privacy budget $\varepsilon$ based on the experience of prior arts.
% }
% \label{tab:privacy-utility}
% % %removedVspace
% \end{table}

% \begin{figure}
% \centering
% \includegraphics[width=0.9\linewidth]{fig/fig6.pdf}
% % %removedVspace
% \caption{ Comparison of FID scores between FedIOD and FedAvg on (a) 9 random selected local clients; and (b) average score under CIFAR10 ($K$=20, $\alpha$=1) FL setting. A larger FID indicates a stronger privacy guarantee.
% }
% \label{fig:fid}
% \end{figure}

\begin{figure}[b]
\centering
\includegraphics[width=\linewidth]{fig/fig5.pdf}
% %removedVspace
\caption{ Comparisons of communication cost for (a) CIFAR10 ($K$=20, $\alpha$=0.1) classification; and (b) BraTS2018 segmentation to reach certain performance.
}
\label{fig:bandwidth}
% %removedVspace
\end{figure}

 \textbf{Comparison with parameter-based FL.}
Sharing parameters makes it vulnerable to white-box attacks \cite{chang2019cronus, zhu2019deep, geiping2020inverting}, while our distillation-based method only has black-box attack risks. Although it is intuitive that distillation-based FL is more secure than parameter-based FL, the synthetic images used in distillation-based FedIOD may raise privacy concerns. We use the similarity between synthetic images and privately held local data as a quantization of privacy leakage.
For parameter-based FL, we use DLG \cite{zhu2019deep} as an attacker to recover private data using its iterative shared model parameters. We then measure the quality of the recovered data using Fréchet Inception Distance (FID). We assume a larger FID, \ie, a larger distance between the recovered data and private data, indicates a stronger privacy guarantee. For our method, we measure the FID between the synthetic images and the private images. The comparison in Figure \ref{fig:fid} shows that our method has a much higher FID, thus far more privacy protected than the FL parameter-sharing method such as FedAvg \cite{mcmahan2017communication}.
% Figure~\ref{fig:bandwidth} compares the communication bandwidth of our FedIOD and the classic FL method like FedAvg.
In particular, our proposed method outperforms the parameter sharing method \cite{mcmahan2017communication} and simultaneously provides a much more secure privacy guarantee. Figure~\ref{fig:bandwidth} shows that our method costs less or equivalent communication bandwidth compared to the parameter-based art \cite{mcmahan2017communication} .





% \begin{table*}
% % %removedVspace
% \centering
% \resizebox{\textwidth}{!}
% % \scriptsize
% {
% \begin{tabular}{c|c|cc|cc|cc|cc|cc|cc}
% \toprule
% \multirow{3}{*}{Method} &\multirow{3}{*}{Privacy} &\multicolumn{2}{c|}{Prerequisite} &\multicolumn{2}{c|}{Data}  &\multicolumn{4}{c|}{T1-weighted}  &\multicolumn{4}{c}{T2-weighted}\\%\cmidrule{4-13}
% & &Net Structure &Auxiliary &\multirow{2}{*}{Train} &\multirow{2}{*}{Test} &\multirow{2}{*}{SSIM$\uparrow$} &\multirow{2}{*}{PSNR$\uparrow$} &\multicolumn{2}{c|}{Average}  &\multirow{2}{*}{SSIM$\uparrow$} &\multirow{2}{*}{PSNR$\uparrow$}   &\multicolumn{2}{c}{Average}\\
% & &Identity &Data & & & &  &SSIM$\uparrow$ &PSNR$\uparrow$  & &  &SSIM$\uparrow$ &PSNR$\uparrow$ \\\midrule
% \multirow{3}{*}{FedAvg\cite{mcmahan2017communication}}  &\multirow{3}{*}{\xmark} &\multirow{3}{*}{Y} &\multirow{3}{*}{N} &B,F &\cellcolor{gray0}I &\cellcolor{gray0}0.9086 &\cellcolor{gray0}31.46  &\multirow{3}{*}{0.8907}  &\multirow{3}{*}{32.39}  &\cellcolor{gray0}0.8532 &\cellcolor{gray0}29.42 &\multirow{3}{*}{0.8717} &\multirow{3}{*}{30.87} \\
% & & & &F,I &\cellcolor{gray1}B &\cellcolor{gray1}0.9268 &\cellcolor{gray1}34.69 & & &\cellcolor{gray1}0.9062 &\cellcolor{gray1}33.11 & & \\
% & & & &B,I &\cellcolor{gray2}F &\cellcolor{gray2}0.8369 &\cellcolor{gray2}31.03 & & &\cellcolor{gray2}0.8556 &\cellcolor{gray2}30.09 & & \\
% \midrule

% \multirow{3}{*}{FL-MRCM\cite{Guo_2021_CVPR}}  &\multirow{3}{*}{\xmark}  &\multirow{3}{*}{Y}  &\multirow{3}{*}{N} &B,F &\cellcolor{gray0}I &\cellcolor{gray0}0.9157 &\cellcolor{gray0}31.74  &\multirow{3}{*}{0.8995} &\multirow{3}{*}{33.16} &\cellcolor{gray0}0.8354 &\cellcolor{gray0}29.28 &\multirow{3}{*}{0.8666} &\multirow{3}{*}{30.69} \\
% & & & &F,I &\cellcolor{gray1}B &\cellcolor{gray1}0.9486 &\cellcolor{gray1}35.78  & & &\cellcolor{gray1}0.9041 &\cellcolor{gray1}33.15  \\
% & & & &B,I &\cellcolor{gray2}F &\cellcolor{gray2}0.8354 &\cellcolor{gray2}31.96  & & &\cellcolor{gray2}0.8604 &\cellcolor{gray2}29.66  \\
% \midrule

% \multirow{3}{*}{\textcolor{black}{FedDF\cite{lin2020ensemble}}} &\multirow{3}{*}{{\xmark}} &\multirow{3}{*}{Y} &\multirow{3}{*}{Y} &\textcolor{black}{B,F} &\cellcolor{gray0}\textcolor{black}{I} &\cellcolor{gray0}\textcolor{black}{0.9209} &\cellcolor{gray0}\textcolor{black}{32.11}  &\multirow{3}{*}{0.9083} &\multirow{3}{*}{33.46} &\cellcolor{gray0}\textcolor{black}{0.8781} &\cellcolor{gray0}\textcolor{black}{30.20}  &\multirow{3}{*}{0.8903} &\multirow{3}{*}{31.51} \\
% & & & &\textcolor{black}{F,I} &\cellcolor{gray1}\textcolor{black}{B} &\cellcolor{gray1}\textcolor{black}{0.9561} &\cellcolor{gray1}\textcolor{black}{35.92} & & &\cellcolor{gray1}\textcolor{black}{0.9154} &\cellcolor{gray1}\textcolor{black}{33.96}\\
% & & & &\textcolor{black}{B,I} &\cellcolor{gray2}\textcolor{black}{F} &\cellcolor{gray2}\textcolor{black}{0.8479} &\cellcolor{gray2}\textcolor{black}{32.36}  & & &\cellcolor{gray2}\textcolor{black}{0.8775} &\cellcolor{gray2}\textcolor{black}{30.37}  \\
% \midrule

% \multirow{3}{*}{FedAD \cite{gong2022federated}} &\multirow{3}{*}{\cmark} &\multirow{3}{*}{N} &\multirow{3}{*}{Y} &B,F &\cellcolor{gray0}I &\cellcolor{gray0}0.9141 &\cellcolor{gray0}31.26 &\multirow{3}{*}{0.8908} &\multirow{3}{*}{32.04} &\cellcolor{gray0}0.8883 &\cellcolor{gray0}29.92 &\multirow{3}{*}{0.8884} &\multirow{3}{*}{31.15 } \\
% & & & &F,I &\cellcolor{gray1}B &\cellcolor{gray1}0.9052 &\cellcolor{gray1}33.25  & & &\cellcolor{gray1}0.8964 &\cellcolor{gray1}33.24 & &\\
% & & & &B,I &\cellcolor{gray2}F &\cellcolor{gray2}0.8533 &\cellcolor{gray2}31.62  & & &\cellcolor{gray2}0.8805 &\cellcolor{gray2}30.31  & & \\
% \midrule
% \multirow{3}{*}{FedIOD} &\multirow{3}{*}{\cmark} &\multirow{3}{*}{N} &\multirow{3}{*}{N} &B,F &\cellcolor{gray0}I &\cellcolor{gray0}0.8902 &\cellcolor{gray0}31.13 &\multirow{3}{*}{0.8509} &\multirow{3}{*}{31.56} &\cellcolor{gray0}0.7974 &\cellcolor{gray0}28.25 &\multirow{3}{*}{0.8496} &\multirow{3}{*}{30.28}  \\
% & & & &F,I &\cellcolor{gray1}B &\cellcolor{gray1}0.8907 &\cellcolor{gray1}34.85  & & &\cellcolor{gray1}0.8926 &\cellcolor{gray1}33.72 & &\\
% & & & &B,I &\cellcolor{gray2}F &\cellcolor{gray2}0.7719 &\cellcolor{gray2}28.70 & & &\cellcolor{gray2}0.8588 &\cellcolor{gray2}28.89  & & \\
% % \midrule
% % \midrule
% % \multirow{3}{*}{Centralized} &\multirow{3}{*}{\textcolor{black}{-}} &\multirow{3}{*}{-} &B,F &\cellcolor{gray0}I &\cellcolor{gray0}0.9015 &\cellcolor{gray0}31.03  &\multirow{3}{*}{0} &\multirow{3}{*}{0}  &\cellcolor{gray0}0.8763 &\cellcolor{gray0}29.22 &\multirow{3}{*}{0} &\multirow{3}{*}{0} \\
% % & & &F,I &\cellcolor{gray1}B &\cellcolor{gray1}0.9246 &\cellcolor{gray1}34.75 & & &\cellcolor{gray1}0.9045 &\cellcolor{gray1}33.07& & \\
% % & & &B,I &\cellcolor{gray2}F &\cellcolor{gray2}0.8747 &\cellcolor{gray2}32.65  & & &\cellcolor{gray2}0.8827 &\cellcolor{gray2}30.33 & & \\
% \bottomrule
% \end{tabular}}
% \caption{Results on cross-domain testing sets for MRI image reconstruction.}
% \label{tab:mricross}
% %removedVspace
% \end{table*}




\end{document}