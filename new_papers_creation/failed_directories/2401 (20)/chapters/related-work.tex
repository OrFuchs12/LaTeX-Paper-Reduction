\section{Related Work}
\label{sec:related-work}

 \paragraph{Voxel-based EM Neuron Segmentation.}
 Given a terabyte- or perabyte-scale EM image volume, many learning-based methods have been proposed for automatic segmentation. A popular series of methods train voxel-based convolutional neural networks to learn affinity maps~\cite{lee2021learning, sheridan2022local, funke2018large} and then apply non-parametric watershed transformation and agglomeration~\cite{beier2017multicut, wolf2018mutex}. 
 Flood-filling networks (FFN)~\cite{FFN} combine these two steps into one by gradually expanding segments from seed voxels. 
 Compared to affinity-based methods, FFN outperforms by a substantial margin, but at the expense of two orders of magnitude higher computational complexity. 
 In this work, we choose to correct the split errors from FFN segmentation results, instead of modifying the dense voxel-wise segmentation.  
 
 
 \paragraph{Error Correction Methods.}

Although the afore-mentioned approaches determine most cell boundaries correctly, the remaining connection errors still require extensive human proofreading and correction. 
A few methods attempt to detect and correct errors automatically. \citet{zung2017error} firstly study error detection and correction for 3D neuron reconstruction. They train a 3D CNN to predict the map of split and merge errors and formulate error correction as an object mask pruning task. Other works focus on merge error correction using predefined neuron structural rules~\cite{VJain-MICCAI-2020, celii2023neurd}. 
\citet{matejek2019biologically} apply biological prior to
detect potential split errors, and then employ 3D CNNs to predict connection probability between the candidate pairs based on the segment morphology. We further explore the split error correction problem by introducing a comprehensive dataset and extensively studying different image and morphology representations for connectivity prediction. 

 \paragraph{Deep Metric Learning.}
The goal of metric learning is to learn a distance metric or embedding vectors such that similar samples are pulled closer and dissimilar samples are pushed away. Deep metric learning with pairwise~\cite{hadsell2006dimensionality} or triplet loss \cite{schroff2015facenet} performs favorably in image retrieval \cite{yang2018retrieving} and geo-localization \cite{shi2020optimal}. In the semantic and instance segmentation area, many algorithms employ metric learning to explore structural relations between pixels in the embedding space~\cite{zhou2022rethinking, lee2021learning, Wang_2021_ICCV}. The extracted pixelwise vectors of dense image embedding is significant helpful for downstream tasks. In this work, we also employ dense embeddings as a delicate image feature representation. The embedding vector of each voxel captures its neighboring appearance features, subsequently fused with point or voxel-based morphological representations to predict segment connectivity. Different from existing dense metric learning approaches, without requiring pixel-wise annotations, our method learns dense embeddings from sparse segment connectivity directly through a novel connectivity-aware contrastive loss. 

