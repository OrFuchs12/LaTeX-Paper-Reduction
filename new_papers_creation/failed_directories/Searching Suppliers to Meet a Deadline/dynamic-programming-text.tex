
% Dynamic programming
where the choice of the best supplier for the $i^{th}$ task relies on the previously computed best suppliers for the first $(i-1)$ tasks. 


This approach, however, does not work for solving an \samd problem. 
Let $\langle \{T_1,T_2\}, S, X, 2\rangle$ be the example \samd problem described in Section~\ref{sec:def} with a deadline 2. The optimal supplier assignment for this problem is $\varphi_{2,4}$. Now, consider a reduced problem that consists of only the first task with $T'=\{T_1\}$, $S'=\{s_1,s_2\}$, and the task completion distributions $X'$, which are equal to those of $X$ for the suppliers of the reduced problem. The optimal supplier assignment for $\langle T', S', X', 2\rangle$ is $\varphi_{1}$ for which $M(T',\varphi_{1}, 2)=1$. Obviously, supplier $s_1$ that constitutes the optimal assignment for the first task $T_1$ is not part of the optimal assignment of the extended two-task problem. 

More generally, in \samd an optimal supplier assignment for an \samd problem that consists the first $(i-1)^{th}$ task and the same deadline $d$ is not necessarily an optimal assignment for these tasks for an \samd problem with all tasks and the same deadline. 
This property -- that a prefix of an optimal solution is an optimal solution to the corresponding subproblem -- is called the \emph{optimal substructure} property. The absence of optimal substructures prevents the use of efficient dynamic programming approaches (cf. the \emph{relax} operation when solving combinatorial search problem~\citep{bellman1958routing,dijkstra1959note}). 

\Roni{I'm not 100\% happy with the above, and we might remove it altogether. I fear this will not be understood by a common reviewer.}