\section{Introduction}
Scholars have long recognized the broad implications of mediating the flow of information in society \citep{McCombs1972,Entman1993}. Moderation of content along this flow is often referred to as \textit{gatekeeping}, ``culling and crafting countless bits of information into the limited number of messages that reach people each day'' \citep{Shoemaker2009}. Gatekeeping power constitutes ``a major lever in the control of society'' \citep{Bagdikian1983}, especially since prominent topics in the news can become prominent topics among the general public \citep{McCombs2005}. 

The flow of information across society has become more complex through the digitization, algorithmic curation, and intermediation of the news \citep{Diakopoulos2019}. Sociotechnical content curation systems like Google, Facebook, YouTube, and Apple News play significant roles as gatekeepers; their algorithms, processes, and content policies directly influence the public's media exposure and intake. Considering this powerful influence on society, our work asks: \textit{How might we systematically characterize the gatekeeping function of algorithmic curation platforms?} To address this question, we develop a framework for auditing algorithmic curators, including aspects of the curation \textit{mechanism} (e.g. churn rate and adaptation) and the curated \textit{content} (e.g. outputted sources and topics). We apply this framework in an audit of Apple News. 

With more than 85 million monthly active users \citep{Feiner2019}, Apple News now has a significant (and growing) influence on news consumption. Our audit focuses on two of the most prominent sections of the app: the ``Trending Stories'' section, which is algorithmically curated, and the ``Top Stories'' section, which is curated by an editorial staff. We observe minimal evidence of localization or personalization within the Trending Stories section. We also find that, compared to Top Stories, Trending Stories exhibit higher source concentration and gravitate towards content related to celebrities, entertainment, and the arts -- what journalism scholars classify as ``soft news'' \citep{Reinemann2012}.

This paper presents (1) a conceptual framework for auditing content aggregators, (2) an application of that framework to the Apple News application, with tools and techniques for synchronizing crowdsourced data collection and using app simulation to get around a lack of APIs, and (3) results and analysis from our audit, which both corroborate and nuance the current understanding of how Apple News curates information. We discuss these findings and their implications for future audit studies of algorithmic news curators. 

%In fact, these modern-day gatekeepers have already pulled their proverbial lever. While Facebook initially resisted being framed as a media company \citep{Segreti2016}, it has begun pressing in to its gatekeeping practices. For example, publishing more detailed explanations of the platform's content moderation and news feed ranking methodology.

%As these platforms stretch wider and deeper across society, their values and biases must be held in check. Researchers have shown with increasing clarity that our technical tools are ``heterogeneous and diffuse sociotechnical systems, rather than rigidly constrained and procedural formulas'' \citep{Seaver2017}. Contrary to the notion of ``neutral'' or ``purely technical'' systems, they often embed, embody, and propagate values and biases \citep{Friedman1996,Friedman2008,Introna2000,ONeil}. To better articulate these biases researchers have called for algorithmic accountability: the notion that, having such an influence on society, algorithms deserve critical attention \citep{Gillespie2014,Diakopoulos2015,Garfinkel2017}, as do the larger socio-technical assemblages in which they operate \citep{Geiger2014,Binns2017,Kitchin2016}. In making them more transparent, we can better understand the normative role they currently play in society, as well as the role they \textit{should} play.

%Because of their role as gatekeepers, transparency and accountability is especially needed for sociotechnical systems that rank, search, and aggregate news. In our view, there are two aspects of an intermediary such as Apple News that should be made transparent: its mechanism and its content.

%While particularities of the mechanism may seem inconsequential on the surface, they play a demonstrably normative role in content curation. For example, the update frequency of a platform can determine whether it emphasizes \textit{recent} content or \textit{relevant} content \citep{Chakraborty}, that is, information that is of present importance or long-term importance. The churn rate for popularity-ranked content also has effects on the quality of that content, since over-reliance on  popularity signals can undermine the wisdom of the crowd \citep{Ciampaglia2018}.

%The political implications of personalization mechanisms have been pointed out more frequently. A recent wave of concern about the ``filter bubble'' pointed out various threats to democratic efficacy \citep{Bozdag2015,Helberger2018,Flaxman}, ranging from diminished individual autonomy to undermined civic discourse. While these concerns were recently met with evidence that the filter bubble phenomenon proved relatively minimal in practice \citep{Haim2018,Trilling2018}, the looming ramifications of platform personalization require continued investigation.

%In addition to mechanistic qualities such as update frequency and content adaptation, a platform's inclusion and exclusion of content should also be transparent. As \citep{Schudson1995} puts it, the selection of news headlines represents ``a hierarchy of moral salience.'' This is because almost all editorial decisions play a role in agenda-setting, sending topics from the mass media's attention into the general public's attention \citep{McCombs2005}. This relationship to the public discourse means that a media system's content has significant \textit{political} implications.

%The content in a given media system can be evaluated for source diversity, content diversity, and exposure diversity: variation in a system's providers, topics/ideas, and audience consumption, respectively. While the relationship between the three is somewhat muddled, investigating them is crucial to understanding ''increasingly complex, interconnected, and convergent media systems'' \citep{Napoli2011}, and thus how they are impacting public discourse and democratic efficacy \citep{Nechushtai2019}.

%In addition to political implications, the content in a platform such as Apple News also has \textit{economic} implications. Businesses have come to recognize the economic benefits of ranking well in Google search results, which drives more clicks, advertisement views, and therefore more revenue. Similarly, stories that appear prominently in Apple News will convert to revenue for their publishers. With many news organizations already reeling from industry shifts in the last decade or so \citep{Abernathy2016}, further economic complications should be approached cautiously.

%Considering the political and economic implications a platform's mechanism and content, this paper addresses the following research questions about Apple News, an emerging news aggregator:

\begin{comment}
\begin{itemize}
\item RQ1: \textbf{Mechanism}. How is the aggregator operating?
\begin{itemize}
\item 	RQ1a: How often does the aggregator update the content displayed?
\item 	RQ1b: To what extent does the aggregator adapt content across users: is content personalized, localized, or uniform?
\end{itemize}

\item RQ2: \textbf{Content}. What content does the aggregator direct attention to?
\begin{itemize}
\item 	RQ2a: What sources does the aggregator display, and with what frequency?
\item 	RQ2b: What topics does the aggregator display, and with what frequency?
\end{itemize}

\item (possibly) RQ3: \textbf{Effect} How does the aggregator affect attention?
\begin{itemize}
\item 	(possibly) RQ3a: Does traffic to a story change when that story appears in the aggregator?
\end{itemize}
\end{itemize}
\end{comment}



