\documentclass[letterpaper]{article} % DO NOT CHANGE THIS
\usepackage{aaai20}  % DO NOT CHANGE THIS
\usepackage{times}  % DO NOT CHANGE THIS
\usepackage{helvet} % DO NOT CHANGE THIS
\usepackage{courier}  % DO NOT CHANGE THIS
\usepackage[hyphens]{url}  % DO NOT CHANGE THIS
\usepackage{graphicx} % DO NOT CHANGE THIS
\urlstyle{rm} % DO NOT CHANGE THIS
\def\UrlFont{\rm}  % DO NOT CHANGE THIS
\usepackage{graphicx}  % DO NOT CHANGE THIS
\frenchspacing  % DO NOT CHANGE THIS
\setlength{\pdfpagewidth}{8.5in}  % DO NOT CHANGE THIS
\setlength{\pdfpageheight}{11in}  % DO NOT CHANGE THIS

\usepackage[utf8]{inputenc}
\usepackage{bbold}
\usepackage{amssymb}
\usepackage{amsmath}
\usepackage{balance}
\usepackage{tikz}
\usepackage{subfig}
\usepackage{bbm}
\usepackage{enumitem}
\usepackage{appendix}

\DeclareMathOperator*{\argmax}{arg\,max}
\providecommand{\TODO}[1]{\textcolor{red}{\textbf{#1}}}



%\title{Automating Product Placement in Retail via Stochastic Demand Simulation}

\title{A Probabilistic Simulator of Spatial Demand for Product Allocation}

%Your title must be in mixed case, not sentence case. 
% That means all verbs (including short verbs like be, is, using,and go), 
% nouns, adverbs, adjectives should be capitalized, including both words in hyphenated terms, while
% articles, conjunctions, and prepositions are lower case unless they
% directly follow a colon or long dash
\author{Porter Jenkins \textsuperscript{\rm 1}, Hua Wei \textsuperscript{\rm 1}, J. Stockton Jenkins \textsuperscript{\rm 2}, Zhenhui Li \textsuperscript{\rm 1} \\ 
\textsuperscript{\rm 1} Penn State University \\
\textsuperscript{\rm 2} Brigham Young University \\%If you have multiple authors and multiple affiliations
% use superscripts in text and roman font to identify them. For example, Sunil Issar,\textsuperscript{\rm 2} J. Scott Penberthy\textsuperscript{\rm 3} George Ferguson,\textsuperscript{\rm 4} Hans Guesgen\textsuperscript{\rm 5}. Note that the comma should be placed BEFORE the superscript for optimum readability
}


\begin{document}
\maketitle

\begin{abstract}
Connecting consumers with relevant products is a very important problem in both online and offline commerce. In physical retail, product placement is an effective way to connect consumers with products. However, selecting product locations within a store can be a tedious process. Moreover, learning important spatial patterns in offline retail is challenging due to the scarcity of data and the high cost of exploration and experimentation in the physical world. To address these challenges, we propose a stochastic model of spatial demand in physical retail. We show that the proposed model is more predictive of demand than existing baselines. We also perform a preliminary study into different automation techniques and show that an optimal product allocation policy can be learned through Deep Q-Learning. 

\end{abstract}



\section{Introduction}
%%%%%%%%%%%%%%%%%%%%%%%%%%%%%%
% 1.定义image captioning任务 
%%%%%%%%%%%%%%%%%%%%%%%%%%%%%%
Image captioning is a fundamental task in vision-language understanding that involves generating natural language descriptions for a given image. It plays a critical role in facilitating more complex vision-language tasks, such as visual question answering \cite{Agrawal2015VQAVQ,gqa,okvqa} and visual dialog \cite{Das2016VisualD,Niu2018RecursiveVA,llava}.
%%%%%%%%%%%%%%%%%%%%%%%%%%%%%%
% text-only training 的介绍
%%%%%%%%%%%%%%%%%%%%%%%%%%%%%%
The mainstream image captioning methods \cite{conimgcap4,conimgcap1,conimgcap3,conimgcap2} require expensive human annotation of image-text pairs for training neural network models in an end-to-end manner. Recent developments in Contrastive Image Language Pre-training (CLIP) \cite{clip} have enabled researchers to explore a new paradigm, zero-shot image captioning, through text-only training. In particular, CLIP learns a multi-modal embedding space where semantically related images and text are encoded into features with close proximity. As such, if a model learns to map the CLIP text features to their corresponding texts, it is feasible to generate image captions from the CLIP image features without needing supervision from caption annotations.

%%%%%%%%%%%%%%%%%%%%%%%%%%%%%%
% text-only training 的优势
%%%%%%%%%%%%%%%%%%%%%%%%%%%%%%

One main advantage of this zero-shot captioning paradigm is that it enables a Large Language Model (LLM) \cite{gpt3, Zhang2022OPTOP} with image captioning capabilities using only text data and affordable computational resources. Despite the impressive performance achieved by recent powerful multimodal models \cite{miniGPT4,llava}, they typically require large-scale, high-quality human-annotated data and expensive computational resources for fine-tuning an LLM. Zero-shot captioning methods can significantly reduce such costs, which is particularly important in situations of data scarcity and limited resources. Moreover, recent work \cite{Guo2022FromIT, Changpinyo2022AllYM,Tiong2022PlugandPlayVZ} demonstrates that other vision-language tasks, such as VQA, can be addressed by LLMs and image captions. Consequently, the paradigm of zero-shot captioning has the potential to pave the way to solving complex vision-language tasks with LLMs through efficient text-only training. 


%%%%%%%%%%%%%%%%%%%%%%%%%%%%%%
% zero-shot image captioning via text-only training 的challenge
%%%%%%%%%%%%%%%%%%%%%%%%%%%%%%
A critical challenge in zero-shot image captioning through text-only training is to mitigate a widely observed phenomenon known as the \textit{modality gap}. While the features of paired texts and images are close in the CLIP embedding space, there remains a gap between them \cite{MindGap}. This gap often results in inaccurate mappings from the image embeddings to the text ones. Consequently, without fine-tuning with paired data, it significantly impairs the performance of zero-shot image captioning.
%%%%%%%%%%%%%%%%%%%%%%%%%%%%%%
% current works intro
%%%%%%%%%%%%%%%%%%%%%%%%%%%%%%
Several works have attempted to address the modality gap in zero-shot image captioning, relying mainly on two strategies: (1) The first strategy leverages a memory bank from training text data to project visual embeddings into the text embedding space \cite{DeCap}. However, this projection prevents it from representing any semantic content outside the distribution of the memory bank features and introduces extra inference costs; (2) The second approach injects noise during training to encourage the visual embeddings to be included inside the semantic neighborhood of the corresponding text embeddings \cite{CapDec}. Nonetheless, the noise injection tends to diffuse the distribution of visual inputs at the cost of weakening the semantic correlation between paired images and text embeddings. 

%However, in the first strategy, the projection of visual embeddings prevents them from  For the second strategy, noise injection during training diffuses the input distribution at the cost of degrading the semantic correlation between paired images and text embeddings.

%Previous attempts \cite{CapDec,DeCap} to reduce the modality gap in zero-shot image captioning can be summarized into two aspects: (1) Decap\cite{DeCap} leverages a memory bank from training text data to project visual embeddings into text embedding space. However, the projection of visual embeddings prevents it from representing any semantic content outside the distribution of the memory bank and introduce extra inference cost. (2) CapDec\cite{CapDec}proposes to inject noise during training to encourage the visual embedding to be included inside the text embedding space. 
% current work weakness
%Nevertheless, noise injection during training diffuses the input distribution at the cost of degrading the semantic correlation between paired images and text embeddings.


%%%%%%%%%%%%%%%%%%%%%%%%%%%%%%
% 我们工作的流程
% 分析得到两个结论:1.subregion带来更好的匹配2.image text gap符合高斯分布
%%%%%%%%%%%%%%%%%%%%%%%%%%%%%%
To tackle these challenges, we first conduct a thorough analysis of the CLIP feature space, leading to two key observations. First, most text descriptions are unable to fully capture the content of their paired images. However, we empirically find that the visual embedding of certain local regions of an image, named image subregions, have closer proximity to the text embedding of the paired caption. Integrating such image subregions with the global image representation generates a tighter alignment between image and text. Additionally, we analyze the distribution of the gap between the CLIP features of image or subregion-text pairs and find that it closely resembles a zero-mean Gaussian distribution.
%initiate our investigation by conducting a thorough analysis of the CLIP latent space. Building upon the insights from the work \cite{MindGap}, we identify a key factor contributing to the existence of a modality gap. Due to the inherent disparities between textual and visual modalities, text is incapable of comprehensively describing the information within an image. However, we empirically demonstrate that the CLIP embedding of some part of image, named image subregions, exhibit closer proximity to the CLIP embedding of the paired caption. The integration between image subregion information and global image feature leads to more compact image text alignment. Besides, we collect the statistics of the gap between CLIP image and text feature. The results demonstrate the gap is close to gaussian distribution. 

%%%%%%%%%%%%%%%%%%%%%%%%%%%%%%
% 我们的方法简略介绍
%%%%%%%%%%%%%%%%%%%%%%%%%%%%%%

Based on our findings, we propose a novel zero-shot image captioning framework, named \textit{\textbf{M}ining Fine-Grained Image-Text \textbf{A}lignment in \textbf{C}LIP for \textbf{Cap}tioning} (MacCap), to address the aforementioned challenges. In this framework, we introduce a region-aware cross-modal representation based on CLIP and an effective unimodal training strategy for an LLM-based caption generator. Our cross-modal representation maps an input image into the language space of LLMs and consists of two main components. First, we design a \textit{sub-region feature aggregation} module to fuse both global and subregion-level CLIP image features, resulting in a smaller gap between the corresponding CLIP text embedding. Next, we introduce a learnable adaptor-decoder to transform the CLIP representation into the LLM's language space.
To train our model with text-only data, we develop a robust procedure to learn a projection from the CLIP embedding space to a language representation, enabling the LLM to reconstruct captions. Specifically, our learning procedure first injects noise into our region-aware CLIP-text representation, mimicking the modality gap between image and text features. This is followed by a multiple sampling and filtering step that leverages the CLIP knowledge to improve the quality of the captioning.
%tackles the problem from three key perspectives. Firstly, we focus on learning a robust projection from CLIP embedding space to language model space by text reconstruction training, which enable model to generate text based on both CLIP image and text feature. The region noise injection in training alleviate the \textit{modality gap} between image and text feature, which makes the projection works for both image and text features. Secondly, we design \textit{sub-region feature aggregation} to obtain a more compact CLIP image feature, which is based on the observation that CLIP subregion feature exhibit closer disntance with corresponding text feature. Third, we propose multiple sampling and filtering to mitigate the drawbacks of noise injection, which leverage CLIP knowledge to further boost caption performance. Finally, we design a pipeline for zero-shot VQA to demonstrate the extensibility of ouir methods to more intricate vision-language tasks.
In addition to the image captioning task, we further extend our framework to build a zero-shot VQA pipeline, demonstrating the generality of our cross-modal representation for more complex vision-language tasks.

%%%%%%%%%%%%%%%%%%%%%%%%%%%%%%
% 我们的方法简略介绍
%%%%%%%%%%%%%%%%%%%%%%%%%%%%%%

We evaluate our framework on several widely-adopted image captioning benchmarks, such as MSCOCO \cite{mscoco} and Flickr30k \cite{Flickr30k}, as well as a standard VQA benchmark, VQAV2 \cite{vqav2}. Our extensive experiments cover multiple vision-language tasks, including zero-shot in-domain image captioning, zero-shot cross-domain image, and zero-shot VQA. The results not only demonstrate the superiority of our methods but also validate our findings on the CLIP embedding space.

% demonstrate through experiments that our proposed methods outperform previous approaches on popular captioning benchmarks, such as MSCOCO, Flickr30k, which further verify our understanding of \textit{concept region}



% Specifically, we evaluate the distribution of the image and text embedding space under hyperspherical coordinates and observe a geometric phenomenon \textit{concept region} 
% where semantically correlated image and text embedding tend to clustering despite the \textit{modality gap}.
% 我们基于concept region的观察提出的方法:concept region和modality gap的cause里面有mismatch pair data导致的semantic ambiguity,总体思路是在train的时候模拟在concept region。在training的时候,我们给text embedding加上region noise,具体而言就是以原本text embedding为中心,一定范围内的多个随机sample的related text embedding,这样的获得的text embedding全都是在输入text对应的concept region内部。在zs captioning的inference时,部分image sub-region inforamtion 会比global image 对text匹配度更高,因此我们基于部分image sub-region inforamtion
% Motivated by the semantic ambiguity of mismatched data observed in \textit{concept region}, we propose two 
% an image sub-region information aggregation strategy for .In detail

% result summary

\section{Problem Definition}\label{prob-def}
In the following section, we provide a formal definition of the optimal allocation problem. Additionally, we define the necessary components of our reinforcement learning agent: the state space, action space, reward function, and state transition function.
\subsection{Optimal Allocation Problem}

In a physical retail environment $\mathcal{R}$ with a set of $n$ spatial regions, we represent the environment with a spatial graph $\mathcal{R} = (\mathcal{V}, \mathcal{E})$, where each region $r_i\in \mathcal{V}$ is a vertex in the graph, the spatial neighboring relation between two regions $r_i$ and $r_j$ are represented as $e_{ij}\in \mathcal{V}$. From $\mathcal{G}$, we can construct the adjacency matrix, $\textbf{A}$.

Additionally, we observe a set of $k$ products, $\mathcal{M} = \{m_j : 0 < j <=k\}$ that are sold. For each product, $m_j$, we know the retail price, $p_j$. 

The decision process faced by the retailer is to allocate each product in $\mathcal{M}$ across regions in $\mathcal{R}$. We define the allocation policy as a function $f$:

\begin{equation}
    f: \mathcal{R} \times \mathcal{M} \rightarrow \mathcal{Z}
\end{equation}
\begin{equation}
    \mathcal{Z} = \{\langle r_i, p_j \rangle , ... \langle r_w, p_q \rangle \}
\end{equation}

Where $\mathcal{Z}$ is the set of selected product region, such that $w <= n$, $q <= k$ and $\mathcal{Z} \subseteq \mathcal{R} \times \mathcal{M}$. This function is typically dynamic over time, which we denote as $f^{t}$. To simplify computation, we treat $\mathcal{Z}^{t}$ as an $(n \times k)$ grid and refer to it as the board configuration at time, $t$. An optimal retail strategy is to find the allocation policy that maximizes revenue:

\begin{equation}
    f^{\ast} = \sum_{t}^{T} \argmax_{f^{t}} \sum_{i, j \in f^{t}(\mathcal{R}, \mathcal{M})} p_j q_i
\end{equation}

where $p_j$ is the price for product $m_j$, and $q_i$ is the quantity sold in region $r_i$ and $T$ is the future time horizon of analysis. The main idea of the current work is to discover the long-term, optimal allocation policy, $f^{\ast}$ from data.

\subsection{Optimal Allocation as a Markov Decision Process}
We believe that the optimal allocation problem is well suited for reinforcement learning because the RL agent is designed for sequential decision making that maximizes expected discounted reward over time. We frame the inputs as a Markov Decision Process (MDP). An MDP is defined by the tuple $\langle \mathcal{S}, \mathcal{A}, P, r, \delta  \rangle$, where $\mathcal{S}$ is the state space, $\mathcal{A}$ is the set of possible actions, $P$ is the (typically unkown) state transition function, $r$ is the reward function and $\delta \in [0,1]$ is the discount factor. 

\begin{itemize}
    \item \textbf{State} At each time, $t$, we observe the state of the retail environment, $\mathcal{E}$. We define the state, $s_t \in \mathcal{S}$, as the tuple of state features, $s_t = \langle \mathcal{Z}^{{t}}, d^{t}, \textbf{g}^{(t-1)}  \rangle$, where $\mathcal{Z}^{{t}}$ is the current board configuration, $d^t$ is the current day of the week (e.g., Sunday $\rightarrow$ 0), and $\textbf{g}^{(t-1)}$ is a vector denoting the revenue at the previous time, $(p_j q_i)^{(t-1)} \forall z \in \mathcal{Z}^t$

    \item \textbf{Action} We define the action space  $\mathcal{A} = \mathcal{R} \times \mathcal{M} \times \{-1, 1\} \cup \{0\}$, indicating ``to place'', ``take way'' or ``do nothing'' for each product, $m_j$ in each region, $r_i$.
    \item \textbf{Reward} The reward function in this case is the total product revenue at time $t$, constrained by the monetary cost, $c$, of placing a set of products in each region:
    \begin{equation}
        r(t) = \sum_{i=1}^n \sum_{j=1}^k p_j q_{ij}^{t} - c \sum_{i=1}^n \mathbbm{1}_{\mathcal{Z}}(r_i)
    \end{equation}
    
    \item \textbf{State transition function}: The state transition, $P$ is defined as $p(s^{t+1} | s^t, a^t): \mathcal{S} \times \mathcal{A} \times \mathcal{S} \rightarrow [0,1]$, which gives the probability of moving to state, $s^{(t+1)}$ given the current state and action. In the optimal allocation problem the exact transition function, $P$ is unknown since the current state, $s^t$ depends on the results of the previous time, $\textbf{g}^{(t-1)}$. We model this transition as a stochastic process.
\end{itemize}
\section{Commonsense for Zero-Shot NLVL}
\label{sec:proposedSection}

\subsection{Problem Formulation}
We denote an input video as $V$, and its grounding annotations as \(\left( Q,V_{\text{span}}\right) \), where $Q$ is the query representation and \(V_{\text{span}}\!=\!\left( t_{s},t_{e}\right)\) is the corresponding video moment span annotation, with \(t_{s}\) and \(t_{e}\) representing the start and end timestamps, respectively. Learning to localize a video moment conditioned on a query entails maximizing the expected log-likelihood of the model parameterized by \(\theta\). In its typical setting, this can be formulated as follows:
\begin{equation}
\label{eq:groundingOriginal}
    \theta ^{\ast }=\arg \max _{\theta } \mathbb{E}\left[ \log p_{\theta }\left(  V_{\text{span}} | V,Q\right) \right]. 
\end{equation}
In the zero-shot setting, the goal is to learn this task without parallel video-query annotations. Hence, the query and video moment annotations are derived from $V$, using a dynamic video moment proposal method followed by a pseudo-query generation mechanism. Formally,  \(V_{\text{span}}\,\!{=}\!\,f_{\text{span}}(V)\) and \(Q\,\!{=}\!\,f_{pq}(V_{\text{span}})\), where $f_{\text{span}}$ and $f_{\text{pq}}$ are video moment proposal and pseudo-query generation mechanisms, respectively. Given that $f_{\text{span}}$ and $f_{\text{pq}}$ are responsible for generating the annotations, the performance of the localization model heavily depends on the quality of these modules. Existing methods face challenges in aligning \(Q\) to \(V_{\text{span}}\) due to noise introduced by ungrounded pseudo-query generation mechanisms. 
To address this, we simplify \(f_{\text{pq}}\) while augmenting cross-modal understanding by leveraging external information in the form of a commonsense graph \(G_{C}(C, E)\) with \(n_c\) nodes, where \(C\!=\!\left\{c_{1}, c_{2}, \dots, c_{n_{C}}\right\}\) are the concept node vector representations and \(E\) is the set of weighted directed edges, respectively. Accordingly, learning can be formulated as
\begin{equation}
\label{eq:groundingOurs}
    \theta ^{\ast }=\arg \max _{\theta } \mathbb{E}\left[ \log p_{\theta }\left(  V_{\text{span}}| V,Q,G_{C}\right) \right].
\end{equation}

\noindent Figure \ref{fig:approach} shows both training and inference flows.
\subsection{Pseudo-supervised Setup}
\modelname first processes a raw video with a video moment proposal $f_{\text{span}}$ module that extracts important video segments capturing key events, and a pseudo-query generation $f_{\text{pq}}$ that generates text query annotations corresponding to the extracted video segments.

\paragraph{Dynamic Video Moment Proposal ($f_{\text{span}}$).}
We adopt the dynamic video moment proposal approach proposed by \citet{nam_zero-shot_2021}. Specifically, $f_{\text{span}}$ primarily comprises a k-means clustering mechanism that groups semantically similar and temporally proximal video frame features together to extract atomic moments. To obtain frame features, we consider the columns of a frame-wise similarity matrix derived from the CNN features of individual frames. We enforce temporal proximity by concatenating the frame index to the features. Composite video moments are then formed by combining neighboring atomic moments, and a subset of all possible combinations is sampled uniformly at random. The resulting set of video moments corresponds to $V_{\text{span}}$.

\paragraph{Pseudo-query Generation ($f_{\text{pq}}$).} The pseudo-query is constructed as a collection of objects present in the video. To generate the pseudo-query, we employ an off-the-shelf object detector, enabling the extraction of pertinent objects in \(V_{\text{span}}\). We adopt a top-$k$ strategy to sample the $k$ most probable object predictions associated with the query \query.

\paragraph{Video Encoder.}
We uniformly sample $T$ frames from $V$ and extract their CNN (\eg, I3D~\cite{qian_locate_2022}) features. These features are contextually encoded using a video encoder ${\phi}_{v}$ to yield frame features ${\phi}_{v}(V)\!=\!\left\{ v_{1},v_{2},\ldots,v_{T}\right\}$ where $v_{i}\in\mathbb{R}^{d}$, and $d$ is the common video/query encoding dimension. We implement ${\phi}_{v}$ as a GRU-based encoder.

\paragraph{Query Encoder.}
Our pseudo-query $Q$, composed of up to $k$ tokens, is encoded using a query encoder ${\phi}_{q}$ that generates query embeddings ${\phi}_{q}(Q)\!=\!\left\{ q_{1},q_{2},\ldots,q_{k}\right\}$, for the top-$k$ detected objects extracted from the pseudo-query generation. Here, $q_{i}\in \mathbb{R}^{d}$ and $d$ is the common video/query encoding dimension. We implement ${\phi}_{q}$ as a bi-directional GRU-based encoder preceded by a trainable embedding layer. 

\subsection{Commonsense Enhancement Module}
\label{sec:cem}
To enrich the encoded video and query features with information grounded in commonsensical knowledge, we introduce a Commonsense Enhancement Module (CEM), pictorially described in Figure~\ref{fig:cem}. This enhancement helps inject necessary information into video and query representations, which can not just help bridge the gap between the available visual and textual cues but also provide rich information to the downstream span localization module. 

\begin{figure}[t!]
    \centering
    \includegraphics[width=0.8\linewidth]{figures/figure_files/Cem.pdf}
    \caption{\modelname Commonsense Enhancement Module (CEM). CEM comprises a concept encoder and an enhancement mechanism that uses the previously encoded concept vectors to update a given input vector (video/query vectors). The concept encoder employs a Graph Convolution Network for encoding the nodes (concepts) of \(G_C\). 
    }
  \label{fig:cem}
\end{figure}

CEM includes a set \(C\!=\!\left\{c_{1}, c_{2}, \dots, c_{n_{C}}\right\}\) of \(n_{C}\) concept vectors, where \(c_{i} \in \mathbb{R}^{d}\) and \(d\) is the concept feature dimension (same dimension as $\forall v_i \in V$ and $\forall q_i \in Q$). In general, given source feature vectors $S\!=\!\left\{ s_{1},s_{2},\ldots,s_{n}\right\}$ with individual feature vectors $s_{i \in [1,n]} \in \mathbb{R}^{d}$, the enhanced feature vectors $S_{C}$ are obtained using a commonsense enhancement mechanism $\phi_{C}$.
We implement this commonsense enhancement step $\phi_{C}$ as a cross-attention mechanism that enriches source input features, attending over $S$ guided by the commonsense concept vectors $C$, \ie, 
\begin{equation}
\label{eq:cenhance}
\scalemath{1}{
    }
    S_{C} = S + \phi_{C}(S) = S + \sigma \left( \frac{SW_{Q}(CW_{K})^{T}}{\sqrt{d}} \right) C W_{V},
\end{equation}
where $\sigma$ is a softmax activation, \(W_{Q}\), \(W_{K}\), \(W_{V}\) are trainable matrices and \(d\) is the common dimension of the vectors \(S\) and \(C\). In our setting, the source feature vectors $S$ are either video $V$ or pseudo-query $Q$ features. We build separate enhancement mechanisms for $V$ and $Q$, \ie, the projection matrices \(W_{Q}\), \(W_{K}\), \(W_{V}\) are not shared between $Q$ and $V$. We elaborate more on the rationale in the appendix.
The enriched video and pseudo-query features are denoted as \(V_{C}\!=\!\phi_{C_{\text{vid}}}(V)\) and \(Q_{C}\!=\!\phi_{C_{\text{pq}}}(Q)\), respectively.

\paragraph{Concept Encoder.}
The concept vectors \(C\) mentioned above are feature representations that internally form the nodes of the commonsense graph, \(G_C\). Accordingly, graph \(G_{C}\) is represented as a matrix, where \(G_{C(i,j)}\) represents the total number of directed relational edges between \(c_{i},c{j} \in C\) that start at \(c_i\) and end at \(c_j\). To encode the commonsense information, we employ Graph Convolutional Networks (GCN) \cite{hammond_wavelets_2011}. The concept encoder is composed of $L$ graph convolution layers, each of which performs a convolution step
\begin{equation}
\scalemath{1}{
    C^{\left(l+1\right)}=\sigma \left( AC^{\left(l\right) }W^{\left( l\right) }\right),
    }
\end{equation}
where $C^{\left(l\right)}$ are node (concept) features and $W^{\left( l\right)}$ trainable weight matrix of layer $l \in [1, L]$, $\sigma$ is a nonlinear activation function, and $A$ is the adjacency matrix obtained by normalizing graph $G_C$ with the degree matrix $D$. Since $G_C$ is a directed graph, normalization can be formulated as $A\!=\!D^{-1}G_{C}$.

\paragraph{Commonsense Information.}
We use ConceptNet \cite{speer_conceptnet_2017}, a popular knowledge graph that provides information spanning various types of relationships such as physical, spatial, behavioral, \etc To ensure that the ConceptNet information utilized is relevant to themes found in the video data, we consider the set of objects available in pseudo-queries and include the top-$k$ most frequently occurring objects to be the seed concept set \(C\). We extract the  ConceptNet subgraph that includes all edges incident between the concepts in \(C\). 
We filter the edge types based on a pre-determined relation set \(R\), which is compiled to involve relations that are relevant to the nature of the video localization task, \eg, spatial (\textit{AtLocation}, \etc) and temporal (\textit{HasSubevent}, \etc) relations are useful for video understanding, while \textit{RelatedTo} and \textit{Synonym} are fairly generic relations that add little information to the localization task. Table \ref{tab:relations} shows the relations included in \(G_C\).

\paragraph{Cross-Modal Interaction Module.} The commonsense enriched video and query features, \(V_{C}\) and \(Q_{C}\), are fused with a multi-modal cross-attention mechanism. We employ a two-step fusion process. First, Query-guided Video Attention (QVA) is applied to attend over video $V_C$, and Video-guided Query Attention (VQA) attends over query $Q_C$ guided by video $V_C$, resulting in updated features $V_C'$ and $Q_C'$, respectively. Both QVA and VQA utilize Attention Dynamic Filters~\cite{rodriguez_proposal-free_2020} that adaptively modify video features, dynamically adjusting them in response to the query, and vice versa. Next, the attended features are fused using a cross-attention mechanism over $V_C'$ guided by $Q_C'$, resulting in localized video features $V_{C_{\text{loc}}}$.

\paragraph{Temporal Regression Module.}
The final step involves a regression layer that approximates $\hat{V}_{\text{span}}$. We employ attention-guided temporal regression to estimate the span of the target video moment. To find important temporal segments relevant to the query, the fused features $V_{C_{\text{loc}}}$ are temporally attended based on the query features to obtain $V_{\text{ta}}$. Then, the span boundaries are localized using a regressor implemented as a Multi-Layer Perceptron (MLP).

\begin{align}
{o}_i = \sigma\left({W}_{1} V_{C_{\text{loc}_i}} + {b}_{{1}}\right) \\
V_{\text{ta}} = \sum_{i=1}^{T} o_i V_{C_{\text{loc}_{i}}} \\
[\hat{t}_s, \hat{t}_e] = {W}_2 {V}_{\text{ta}} + {b}_{2}.
\end{align}
Here, ${W}_{1}$ and ${b}_1$ are the weight matrix and bias vector of the temporal attention MLP, $\sigma$ represents the sigmoid activation function, $V_{C_{\text{loc}_i}}$ stands for the encoded localized video features, ${V}_{\text{ta}}$ represents the temporally attended video features, ${W}_2$ and ${b}_2$ denote the weight matrix and bias vector of the regression MLP, and $[\hat{t}_s, \hat{t}_e]$ correspond to the start and end timestamps of the predicted video span $\hat{V}_{\text{span}}$.

\begin{table}[t!]
\centering
\resizebox{\linewidth}{!}{
\begin{tabular}{ll}
\toprule
\textbf{Category} & \textbf{Relations}                                                                                         \\ \toprule
Spatial           & AtLocation, LocatedNear                                                                                    \\ \midrule
Temporal          & \begin{tabular}[c]{@{}l@{}}HasSubevent, HasFirstSubevent, HasLastSubevent, HasPrerequisite\end{tabular} \\ \midrule
Functional        & UsedFor                                                                                                    \\ \midrule
Causal            & Causes                                                                                                     \\ \midrule
Motivation        & MotivatedByGoal,  ObstructedBy                                                                             \\ \midrule
Other             & CreatedBy, MadeOf                                                                                          \\ \midrule
Physical          & \begin{tabular}[c]{@{}l@{}}HasA, HasProperty, Antonym, SimilarTo\end{tabular}                      
\\ \bottomrule
\end{tabular}
}

\caption{Relations in the Commonsense Enhancement Module (CEM) grouped by categories.}
\label{tab:relations}

\end{table}
\subsection{Training and Inference}
The training objective is 
$\mathcal{L}_{loc} = \mathcal{L}_{treg}+\lambda \mathcal{L}_{ta},$ where \(\lambda\) is a balancing hyperparameter, \(\mathcal{L}_{ta}\) is a temporal attention guided loss and \(\mathcal{L}_{treg}\) is the regression loss.  The temporal attention-guided loss is defined as
\begin{equation}
\label{tatt}
\mathcal{L}_{ta} = \frac{\sum^{T}_{i=1}g_{i}\log \left( a_{i}\right)}{\sum^{T}_{i=1}g_{i}},
\end{equation}
where \(a_{i}\) is the attention weight for video frame \(v_{i}\) and \(g_{i}\) is the attention mask for \(v_{i}\), that is assigned to \(1\) if \(v_{i}\) is inside the target video segment, and \(0\) otherwise. 
This objective encourages the model to produce higher attention weights for video segments that are relevant to the query. 
On the other hand, \(\mathcal{L}_{treg}\) dictates the video span boundary regression and is the sum of smooth $\ell_1$ distances between start and end timestamps of the ground truth and predicted spans, \ie,
\begin{equation}
\label{treg}
\mathcal{L}_{treg} = \text{smooth}{\ell_1}(t_{s}, \hat{t}_{s}) + \text{smooth}{\ell_1}(t_{e}, \hat{t}_{e}).
\end{equation}
Here, $t_{s}$ and ${t}_{e}$ represent the ground truth start and end timestamps and $\hat{t}_{s}$ and $\hat{t}_{e}$ the predicted start and end timestamps, respectively.
The integration of a smoothing mechanism enhances training stability and improves the model's ability to handle outliers. Finally, during inference, we employ an off-the-shelf part-of-speech tagger to extract nouns from the text input query and feed them as query input to the trained \modelname video localizer.
\section{Assessment}
\label{sec:assessment}
\subsection{Experimental Setup}
We implement our PCDNet in PyTorch \cite{paszke2019pytorch} and train it for 300 epochs with the batch size of 32 on two NVIDIA GeForce RTX 3090 GPUs. We use stochastic gradient descent (SGD) \cite{amari1993backpropagation} with a momentum of 0.937 and a weight decay of $5 \times 10 ^{-4}$ during training. The initial learning rate is set to 0.01 and decayed to 0.001 using a cosine annealing schedule. We initialize PCDNet randomly and load the weights of CSPDarknet53 \cite{wang2020cspnet} pre-trained on ImageNet \cite{imagenet_cvpr09} for the encoder part. To increase the diversity and complexity of the training samples, we apply data augmentations including random cropping, random flipping, and mosaic \cite{redmon2018yolov3}. We use the evaluation metrics of Microsoft COCO \cite{lin2014microsoft} for validation.

\begin{table}[ht]
\caption{Quantitative comparison against state-of-the-art polarization-based detectors ($\star$), single-stage detectors ($\dag$), two-stage detectors ($\ddag$), anchor-based detectors ($\triangle$), anchor-free detectors ($\circ$), and self-supervised method ($\S$).}
\small
\centering
\renewcommand\arraystretch{0.9}
\setlength{\tabcolsep}{2.6pt}
\begin{tabular}{lccccc}
\hline\hline
Methods	&	Pub'Year	&	Backbone	&	AP	&	AP50	&	AP75	\\
\hline
Faster R-CNN$^{\ddag\triangle}$ 	&	NeurIPS'15	&	Res50	&	44.8	&	75.4	&	45.4	\\
SSD$^{\dag\circ}$ 	&	ECCV'16	&	VGG16	&	25.5	&	52.6	&	22.6	\\
Cascade R-CNN$^{\ddag\triangle}$ 	&	CVPR'18	&	Res50	&	45.8	&	73.2	&	47.8	\\
CornerNet$^{\dag\circ}$ 	&	ECCV'18	&	Res50	&	19.8	&	47.4	&	29.6	\\
P-SSD I$^{\star\dag\circ}$ 	&	ITSC'19	&	VGG16	&	25.9 	&	53.1	&	22.7	\\
P-SSD S$^{\star\dag\circ}$ 	&	ITSC'19	&	VGG16	&	23.0 	&	48.9	&	20.1	\\
FCOS$^{\dag\circ}$ 	&	ICCV'19	&	Res50	&	23.1	&	50.9	&	18.4	\\
DH R-CNN$^{\ddag\triangle}$ 	&	CVPR'20	&	Res50	&	32.7	&	65.3	&	28.2	\\
Dynamic R-CNN$^{\ddag\triangle}$ 	&	ECCV'20	&	Res50	&	46.2	&	74.2	&	48.0	\\
EfficientDet$^{\ddag\triangle}$ 	&	CVPR'20	&	D3	&	45.3	&	73.0	&	46.3	\\
VarifocalNet$^{\dag\circ}$  & CVPR'21 & Res50 & 44.2 &	73.5 &	44.4	\\
D-DETR$^{\dag\circ}$ 	&	ICLR'21	&	Res50	&	43.8	&	74.9	&	44.3	\\
DDOD$^{\dag\circ}$ 	&	MM'21	&	Res50	&	43.5	&	73.0	&	43.3	\\
TOOD$^{\dag\triangle}$ 	&	ICCV'21	&	Res50	&	44.3	&	74.3	&	44.6	\\
YOLOX$^{\dag\circ}$ 	&	arXiv'21	&	YOLOX-l	&	54.3	&	82.5	&	56.7	\\
YOLOv7$^{\dag\triangle}$	&	arXiv'22	&	Dark53	&	57.6	&	84.3	&	60.3	\\
RTMDet$^{\dag\circ}$ 	&	arXiv'22	&	RTMDet-l	&	53.9	&	81.4	&	56.7	\\
DINO$^{\dag\circ\S}$ 	&	ICLR'22	&	Res50	&	52.7	&	81.8	&	54.8	\\
YOLOv8$^{\dag\circ}$ 	&	-'23	&	YOLOv8-l	&	56.8	&	83.6	&	59.0	\\
\hline
\textbf{PCDNet$^\star$}	&	\textbf{Ours}	&	Dark53	&	\textbf{58.5}	&	\textbf{85.2}	&	\textbf{61.5}	\\
\hline\hline
\end{tabular}
\label{tab:comparison}
\end{table}

\begin{figure*}[htp]
    \centering
    \begin{center}
        % \includegraphics[width=\linewidth]{figure/comparison.pdf}
        \includegraphics[width=\linewidth,height=10.5cm]{figure/comparison.pdf}
    \end{center}
    \caption{Qualitative comparison of PCDNet against state-of-the-art detectors retrained on RGB-P Car dataset.} 
    \label{fig:comparison}
\end{figure*}

\subsection{Qualitative and Quantitative Evaluation}
We extensively compare our PCDNet with 19 state-of-the-art methods by retraining and testing all methods on the RGB-P Car dataset using their original settings. The compared methods include two-stage detectors such as EfficientDet \cite{tan2020efficientdet} and the R-CNN family \cite{Ren_2017, Cai_2019, zhang2020dynamic}, and one-stage detectors such as SSD \cite{liu2016ssd}, and YOLO family \cite{ge2021yolox, wang2022yolov7, ultralytics2023yolov8}. These methods also comprise anchor-based methods such as the R-CNN family and YOLOv7 \cite{wang2022yolov7}, and anchor-free methods such as CornerNet \cite{law2018cornernet}, VarifocalNet \cite{zhang2021varifocalnet}, and YOLOv8 \cite{ultralytics2023yolov8}. Some detectors use traditional convolutional networks such as FCOS \cite{tian2019fcos} and RTMDet \cite{lyu2022rtmdet} while others use transformer structures, such as DeformableDETR \cite{zhu2020deformable} and DINO \cite{zhang2022dino} that employs self-supervised learning. We also include the P-SSD \cite{blin2019road} that utilizes polarization information. The quantitative evaluation results are reported in Tab. \ref{tab:comparison}. We can see that our method outperforms all competing state-of-the-art methods. 

Fig. \ref{fig:comparison} further qualitatively demonstrates the benefits of our method: a) in poorly lit indoor parking lots, distinguishing black cars behind pillars is extremely challenging (the first two rows). The compared methods tend to conflate the shadow and the black car (\textit{i.e.}, merging cars on either side of the pillar into a single entity or treating partial views of the car as one object) while our PCDNet can handle such ambiguities; b) in the third example, all methods except our PCDNet fail to detect a partially visible car obstructed by another car or misplace it with the previous car; c) in the fourth example, RGB-based methods wrongly identify distant pedestrians as cars, but our PCDNet method can effectively eliminate such interference with the help of polarization cues; d) the fifth and sixth examples depict black cars in an outdoor parking lot at night which are very hard to be distinguished in the RGB image. Despite the enhancement through ZeroDCE \cite{guo2020zero}, the sixth example remains unclear. By contrast, polarization imaging is robust to low light conditions, enabling our robust car detector PCDNet; and e) the last row shows a virtual car reflected in a mirror located at the upper-left corner of the image. The mirrored virtual car and the rest of the mirror regions exhibit similar and smooth AoLP, providing useful cues for PCDNet to recognize this region as background. 


\subsection{Ablation Study}
\textbf{Impact of Spectral Intensity and Polarization Cues.} We conduct a series of ablation experiments to demonstrate the effects of spectral intensity and polarization cues on car detection (Tab. \ref{tab:abl_input}).
The results show that: a) combining different forms of polarization cues with RGB as the input of PCDNet can improve the car detection accuracy (\textit{C}, \textit{D}, \textit{F}, \textit{G}, \textit{K} and \textit{L} are higher than \textit{B}); b) DoLP cues have a greater impact than AoLP cues (\textit{D}, \textit{J} and \textit{L} are better than \textit{C}, \textit{I} and \textit{K}, respectively); c) stacking AoLP and DoLP on RGB in the channel dimension does not boost performance (\textit{E} is slightly lower than \textit{B}), possibly because the characteristic gap between different modalities hinders effective features extraction; d) spectral intensity and polarization are more beneficial than monochromatic intensity and polarization for car detection (comparing paired \textit{B} and \textit{H}, \textit{C} and \textit{K}, \textit{D} and \textit{L}, \textit{I} and \textit{K}, \textit{J} and \textit{L}); e) enhancing RGB image via ZeroDCE \cite{guo2020zero} is less effective than introducing polarization (\textit{M} performs worse than \textit{C}-\textit{G}, \textit{K} and \textit{L}).
Fig. \ref{fig:abl_input} provides visual support for these observations.

\begin{table}[t]
\small
\centering
\caption{Quantitative comparisons of ablation with different inputs. ``stacked I'' denotes the stacked intensity measurements with a linear polarization angle of 0$^{\circ}$, 45$^{\circ}$ and 135$^{\circ}$ and ``stacked S'' refers to the stacked Stokes elements S0, S1 and S2 \cite{blin2019road}.}
\begin{tabular}{clccc}
\hline\hline
	&	PCDNet Input	&	AP	&	AP50	&	AP75	\\
 \hline
\textit{A}	&	RGB, AoLP and DoLP (original)	&	58.5 	&	85.2 	&	61.5 	\\
\hline
\textit{B}	&	RGB only	&	57.6 	&	84.3 	&	60.2 	\\
\textit{C}	&	RGB and AoLP	&	58.0 	&	84.6 	&	60.7 	\\
\textit{D}	&	RGB and DoLP	&	58.3 	&	85.4 	&	61.1 	\\
\textit{E}	&	stacked RGB, AoLP and DoLP	&	57.5 	&	84.3 	&	59.9 	\\
\textit{F}	&	RGB and stacked I	&	58.0 	&	84.1 	&	61.0 	\\
\textit{G}	&	RGB and stacked S	&	57.8 	&	84.8 	&	60.4 	\\
\textit{H}	&	Gray only	&	57.4 	&	84.3 	&	60.0 	\\
\textit{I}  &   Gray and mono AoLP & 57.5 & 84.5 & 60.5 \\
\textit{J}  &   Gray and mono DoLP & 57.6 & 84.9 & 60.1 \\
\textit{K}	&	RGB and mono AoLP	&	57.9 	&	84.6 	&	60.5 	\\
\textit{L}	&	RGB and mono DoLP	&	58.2 	&	84.9 	&	60.6 	\\
\textit{M}  &   Enhanced RGB & 57.4 & 84.0 & 60.0 \\
\hline\hline
\end{tabular}
\label{tab:abl_input}
\end{table}

\begin{figure}[t]
    \centering
    \includegraphics[width=1\linewidth]{figure/abl_input.pdf}
    \caption{Qualitative comparison of ablation with different inputs. The model with RGB intensity only is susceptible to interference from ghost car caused by water on the road.}
    \label{fig:abl_input}
\end{figure}

\textbf{Influence of PCDNet Components.}
First, we investigate the performance of different strategies for fusing AoLP and DoLP inputs. From Tab. \ref{tab:abl_module}(\textit{A}-\textit{D}), we observe that our PI module is more effective than the simple fusion methods including concatenation, addition and element-wise multiplication.
Second, by removing MP module \ref{tab:abl_module}(\textit{E}) from the original PCDNet (A), the detection performance declines. This demonstrates that exploring the polarized material features of cars across all learning samples is useful. We also explore the influence of applying MSP and MCP on different levels of features. The results in Tab. \ref{tab:abl_module}(\textit{A},\textit{F}-\textit{G}) show that applying MSP on shallower features and MCP on deeper features can yield better performance.
Finally, we validate the effectiveness of CDDQ module.
Removing the CDDQ module (\textit{I}) from PCDNet (\textit{A}), which causes the feature extraction processes of the RGB and polarization to be independent from each other, leads to the performance drop. We also demonstrate the benefits of the CWDA and SDMD in the CDDQ module by removing either of them (\textit{J} and \textit{K}). 

\begin{table}[t]
\small
\centering
\caption{Quantitative comparisons of ablation with different modules demonstrate that all component of PCDNet contributes to the overall performance. We used sequences of three letters separated by '-' and enclosed in parentheses to represent different combinations of MSP and MCP.}
\begin{tabular}{clccc}
\hline\hline
	&	Ablation	&	AP	&	AP50	&	AP75	\\
 \hline
\textit{A}	&	PCDNet (original)	&	58.5 	&	85.2 	&	61.5 	\\
\hline
\textit{B}	&	Input RGB and [AoLP DoLP]	&	58.2 	&	85.4 	&	60.9 	\\
\textit{C}	&	Input RGB and AoLP+DoLP	&	58.1 	&	84.8 	&	60.5 	\\
\textit{D}	&	Input RGB and AoLP*DoLP	&	58.1 	&	84.8 	&	60.5 	\\
\hline
\textit{E}	&	A \textit{w/o} MP	&	56.9 	&	84.2 	&	59.2 	\\
\textit{F}	&	A \textit{w/} M(S-S-S)P	&	58.2 	&	85.2 	&	60.8 	\\
\textit{G}	&	A \textit{w/} M(S-C-C)P	&	58.2 	&	85.0 	&	60.9 	\\
\textit{H}	&	A \textit{w/} M(C-C-C)P	&	58.1 	&	85.0 	&	61.1 	\\
\hline
\textit{I}	&	A \textit{w/o} CDDQ	&	58.0 	&	84.7 	&	60.8 	\\
\textit{J}	&	A \textit{w/o} SDMD	&	58.2 	&	85.2 	&	60.8 	\\
\textit{K}	&	A \textit{w/o} CWDA	&	58.3 	&	85.1 	&	61.1 	\\
\hline\hline
\end{tabular}
\label{tab:abl_module}
\end{table}

\subsection{Limitations}

When both the RGB intensity and the polarization measurement yield weak car signals, our method's effectiveness declines. Specifically, in low-light scenarios, when a car approaches on an unlit road, the strong light from its headlights can create a ``hole'' in both the RGB and polarization and obscure the entire car. We illustrate such an example in Fig. \ref{fig:failure} where the extreme HDR and heavy motion blur in the captured image limit its depiction of both RGB and polarization. In these challenging scenarios, prior RGB-based methods and even human vision are powerless.

\begin{figure}[t]
    \centering
    \includegraphics[width=1\linewidth]{figure/failure.pdf}
    \caption{PCDNet has limited ability to handle extreme HDR or heavy motion blur cases.}
    \label{fig:failure}
\end{figure}

\section{Related Work}
\label{sec:related-work}

\paragraph{Datasets.}
The lack of cross-file and cross-project (e.g. dependencies) information is a general issue in current evaluation datasets for code.
In terms of code completion, common choices are Py150 \citep{raychev2016probabilistic} for Python and Github Java Corpus \citep{allamanis2013mining} for Java. Both datasets are constructed at file level, where source files are isolated from their project and dependencies and no consideration of project separation is taken in constructing training and test sets.
\citet{lu2022reacc} constructed a code completion dataset from CodeNet \citep{puri2021project}, which contains coding problems and solutions from online judge websites and also lacks project context. 
\citet{clement2021long} presented a real-world Python method generation task based on CodeSearchNet \citep{husain2019codesearchnet} but the auxiliary information they extract still comes from within a local file. 
\citet{svyatkovskiy2021fast} constructed a completion dataset based on top Python repositories on GitHub and released the URLs for these repositories. 
However, those repositories are not write-protected and can change over time. Besides, setting up the dependency environments at scale for further analysis is not easy. 
Both make their dataset difficult to reproduce.
In the contrast, we release the code and the dependencies for the projects to ensure reproducibility.
Apart from code completion, datasets for other code tasks such as Cloze test \citep[e.g.][]{feng2020codebert}, code refinement \citep[e.g.][]{tufano2019empirical, yasunaga2021break, haque2022fixeval}, and generating code from text descriptions \citep[e.g.][]{chen2021evaluating, hendrycks2021measuring, austin2021program}, are often small and mostly without project-level code context. 
Beyond-local information is beneficial for programmers to solve programming tasks in real-world settings. The lack of such information in the current dataset would restrict the progress into high-level semantic understanding and reasoning in the code domain.


\paragraph{Code language models.}
Encouraged by the success of pretrained language models in natural language processing \citep{devlin2019bert, liu2019roberta, lewis2019bart, raffel2020exploring} and the promise of naturalness in code \citep{hindle2016naturalness, allamanis2018survey}, we have seen rising adaptations of language models for code. For example, CuBERT \citep{kanade2020learning} and CodeBERT \citep{feng2020codebert} are pretrained based on masked language modeling. GPT-C \citep{svyatkovskiy2020intellicode} and CodeGPT \citep{lu2021codexglue} are both pretrained based on unidirectional language modeling. PLBART \citep{ahmad2021unified} and CodeT5 \citep{wang2021codet5} are pretrained encoder-decoder structures which adopts denoising objectives and can support code understanding and code generation. UnixCoder \citep{guo2022unixcoder} combines the above three pretraining objectives for a unified pretrained model. 




\paragraph{Code completion.}
Code completion is an essential feature for modern IDEs and an important topic for code intelligence. 
In recent years, deep neural networks \citep{liu2016neural, li2018code, alon2020structural, liu2020multi, kim2021code}, especially pretrained language models \citep{svyatkovskiy2020intellicode, lu2021codexglue} become the mainstream solution to this task. 
Still, incorporating additional information proved beneficial.
One popular choice is abstract syntax tree, e.g. \citet{kim2021code, peng2021could, guo2022unixcoder}. 
However, \citet{lopez2022ast} suggested that pretrained code language models may have already encoded the syntax.  
Other proposals seek to use data flow graph, control graph, and various graph relations, e.g. \citet{guo2020graphcodebert, hellendoorn2019global}.
However, information is still restricted from a single file.
We instead try to enhance the model with out-of-file information, similar to what is accessible in a development environment.

For project-level analyzer induced information, \citet{svyatkovskiy2021fast} described a way to use a static analyzer to refine completion candidates from neural methods.
\citet{weyssow2020combining} considered leveraging the project-wise contexts via embeddings for better function call completion performance.
Other than code completion, project-level information has been utilized for methods name prediction~\citep{wang2021lightweight} and generating code from text descriptions~\citep{lyu2021embedding}.
However, none of them tested their approaches with pretrained code language models. 
In terms of incorporating additional context through concatenation,
\citet{clement2021long} reported improvements from prioritize certain parts of in-file context.
Recently, \citet{lu2022reacc} proposed to enhance code language models by concatenating similar code fragments retrieved by a neural network. Despite the general similarity, we 1) use a simple lightweight way to retrieve auxiliary information instead of training a heavy retriever; 2) do not restrict ourselves on similar code fragments and show that dissimilar code fragments (function implementation) can be helpful; 3) explore task-specific fine-tuning with retrieved information for better completion.





% \vspace{-1em}
\section{Conclusions}
% \vspace{-1em}
In this paper, we introduced a benchmark task for commonsense reasoning that aims at uncovering unspoken intents that humans can easily uncover in a given statement by making presumptions supported by their common sense. In order to solve this task, we propose
CORGI (COmmon-sense ReasoninG by Instruction),  a neuro-symbolic theorem prover that performs commonsense reasoning by initiating a conversation with a user. CORGI has access to a small knowledge base of commonsense facts and completes it as she interacts with the user. We further conduct a user study that indicates the possibility of using conversational interactions with humans for evoking commonsense knowledge and verifies the effectiveness of our proposed theorem prover.
% We defined common-sense reasoning as the process of finding a chain of reasoning in a logic program given an if/then/because statement. We showed that obtaining the because statement is crucial in extracting a relevant chain of reasoning given an if/then statement. Moreover, we introduced a soft backward chaining algorithm that allows us to combat variations in natural language by learning embeddings for the facts and rules in the knowledge base. This algorithm combines symbolic AI with neural approaches allowing us to bridge a gap between symbolic AI and the recent advances in deep learning.

\bibliographystyle{aaai}
\bibliography{main}

\end{document}
