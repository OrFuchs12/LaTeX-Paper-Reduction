
\section{Error Analysis}
\label{sec:error_analysis}

In this section we analyze errors %
on the ending square prediction task. 
Incorrect predictions for this task %
can be (exhaustively) categorized into four categories:

\begin{itemizesquish}
	\itemsep0em 
	\item {\em Unreachable}: The predicted ending square cannot be reached %
	by any possible piece type %
	at the starting square regardless of the board state. 
	\item {\em Syntax}: The predicted ending square cannot be reached %
	by the piece type present at the starting square regardless of the board state. This error indicates failure at tracking the piece type present at the starting square. 
	\item {\em Path Obstruction}: The predicted ending square cannot be reached 
	because there are other pieces obstructing the %
	path. This error indicates failure at tracking other pieces on the board or a lack of %
	understanding that for all piece types except the knight, the path %
	must be clear. %
	For example, in Figure~\ref{fig:error_path}, the pawn at \pos{c6} blocks the bishop's move from \pos{e4} to \pos{b7}.
	\item {\em Pseudo Legal}: 
	The move is illegal because the moving player's king is in check at the end of the move. 
\end{itemizesquish}
Table~\ref{tab:error_analysis} shows error counts for the ending square prediction task. 
For brevity we omit unreachable errors since they are rare ($< 5$ for all models).


Errors across all categories decrease with more training data. For syntax errors this reduction is particularly dramatic, decreasing by roughly an order of magnitude when moving from Train-S to Train-M. %
In contrast, both path obstruction and pseudo legal errors decline more gradually.
Determining whether a path is blocked or if the king is in check requires a computation involving multiple piece locations which all need to be computed from the move  history. 
These trends suggest that identifying the piece type at a starting square 
requires data but is learnable, while keeping track of all \emph{other} pieces  on the board remains challenging even with large training sets.

UCI + RAP %
consistently outperforms %
UCI in syntax errors, %
the differences being largest for the small training sets. This validates our hypothesis that RAP can aid the model in piece tracking (Section~\ref{sec:rap_board}). Across other error categories we don't see consistent trends, suggesting piece tracking improvements do not necessarily translate to other error categories. The Performer generally makes more errors than the transformers, especially in the syntax category. The partial attention in the Performer may be limiting its ability to attend to the most relevant prior positions to determine the piece type at the given starting square. 

Predicting ending squares for the actual move made (``Actual'') is easier than for a randomly chosen legal move (``Other''). However, the syntax errors are comparable between the two settings, while there are many more path obstruction and pseudo legal errors for the Other instances. 
	The higher error rate for these categories could be because:
	\begin{itemizesquish}
		\item Avoiding path obstruction and check are difficult functions to learn and may therefore be being ``mimicked'' from training data rather than being learned as a general algorithmic function.
		\item The model is trained on only actual games with emphasis on meaningful moves rather than legal moves. We observe that some of the Other instances lack any ``meaningful" continuations (Appendix~\ref{sec:app_error_analysis}).
		\item There's a distribution shift between piece types moved in actual moves vs randomly chosen legal moves. 
		For example, the End-Actual task has only about 13\% prompts for moves made by king in comparison to the 33\% for the End-Other task (Appendix~\ref{sec:data_stats}).  We find that moves made by king have a higher chance of resulting in pseudo legal errors in comparison to other piece types (Appendix~\ref{sec:pseudo_legal}).  
	\end{itemizesquish} 






