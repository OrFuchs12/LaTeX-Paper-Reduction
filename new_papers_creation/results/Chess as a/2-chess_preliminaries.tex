
\section{Chess Preliminaries}
\label{sec:chess}


We represent moves using Universal Chess Interface (UCI) notation, which combines the starting square and the destination square to represent a move.\footnote{For more details see \url{https://en.wikipedia.org/wiki/Universal_Chess_Interface}} 
The move in Figure~\ref{fig:move_notation} is represented as \texttt{f1b5} in UCI where \texttt{f1} indicates the starting square and \texttt{b5} denotes the ending square.
While the SAN notation is the standard choice for gameplay, we prefer UCI %
(see Appendix~\ref{sec:san} for why we pick UCI over SAN). 

For training language models, we first tokenize games represented in UCI notation using a simple regular expression based tokenizer, which considers a board square symbol such as \texttt{b1} as a single token.
This gives us a vocabulary of 77 token types, 
which includes the 64 squares, piece type symbols, and other special symbols (see Table~\ref{tab:model_vocab}).\footnote{In initial experiments we used a delimiter token to indicate move boundary. However, removing it did not degrade performance and made training faster due to reduced sequence length.}
For example, the move sequence ``\pos{e2e4 e7e5 g1f3}" is tokenized to ``\pos{e2}, \pos{e4}, \pos{e7}, \pos{e5}, \pos{g1}, \pos{f3}". We then train an autoregressive language model on these move sequences, using the standard maximum likelihood objective.

\begin{table}[t]
\centering{
\begin{tabular}{llc}
    \toprule
    Type & Examples & Count \\
    \midrule
    Square names & \pos{e4}, \pos{d1} & 64 \\
    Piece type & \pos{P}, \pos{K}, \pos{Q}, \pos{R}, \pos{B}, \pos{N} & \phantom{1}6\\
    Promoted Pawn Piece type & q, r, b, n & \phantom{1}4 \\
    Special symbols & BOS, EOS, PAD & \phantom{1}3 \\
    \midrule
    Total & & 77\\
    \bottomrule
\end{tabular}
}
\caption{Model Vocabulary}
\label{tab:model_vocab}

\end{table}

\begin{table*}
	\centering{
		\begin{tabular}{lll}
			\toprule
			Notation 		& Training 					& Inference \\\midrule
			UCI		 		& \pos{e2}, \pos{e4}, \pos{e7}, \pos{e5}, \pos{g1}, \pos{f3}     						& \pos{e2}, \pos{e4}, \pos{e7}, \pos{e5}, \pos{g1}, \pos{f3} 		\\
			UCI + RAP 15 	& \pos{e2}, \pos{e4}, \pos{P}, \pos{e7}, \pos{e5}, \pos{g1}, \pos{f3} 							& \pos{e2}, \pos{e4}, \pos{e7}, \pos{e5}, \pos{g1}, \pos{f3} 		\\
			UCI + RAP 100 	& \pos{P}, \pos{e2}, \pos{e4}, \pos{P}, \pos{e7}, \pos{e5}, \pos{N}, \pos{g1}, \pos{f3}							& \pos{e2}, \pos{e4}, \pos{e7}, \pos{e5}, \pos{g1}, \pos{f3} 		\\
			UCI + \piecetype 	& \pos{P}, \pos{e2}, \pos{e4}, \pos{P}, \pos{e7}, \pos{e5}, \pos{N}, \pos{g1}, \pos{f3}							& \pos{P}, \pos{e2}, \pos{e4}, \pos{P}, \pos{e7}, \pos{e5}, \pos{N}, \pos{g1}, \pos{f3}		\\
			\bottomrule
		\end{tabular}
	}
		\caption{Token sequences corresponding to the move sequence \pos{e2e4 e7e5 g1f3} for different notations during training and inference. Notice that regardless of the RAP probability used during training, at inference time the token sequences have no piece types.}
	\label{tab:token_seq}

\end{table*}
