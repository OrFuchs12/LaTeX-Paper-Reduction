\section{Conclusion}

In this paper, we propose a novel GNN-RNN framework to innovatively incorporate both geospatial and temporal knowledge into crop yield prediction, through graph-based deep learning methods. To our knowledge, our paper is the first to take advantage of the spatial structure in the data when making crop yield predictions, as opposed to previous approaches which assume that neighboring counties are independent samples. We conduct extensive experiments on large-scale datasets covering 41 US states and 39 years, and show that our approach substantially outperforms many existing state-of-the-art machine learning methods across multiple datasets. Thus, we demonstrate that incorporating knowledge about a county's geospatial neighborhood and recent  history can significantly enhance the prediction accuracy of deep learning methods for crop yield prediction. 

%, and demonstrate its superiority to existing models. large-scale crop yield dataset \textit{USCrop}, covering 48 states and 39 years. On \textit{USCrop}, we broadly compare the popular AI methods in crop yield prediction and provide solid baselines on three well-known metrics. Furthermore, we propose a novel GNN-RNN framework to innovatively incorporate both geospatial and temporal knowledge into prediction, through graph-based deep learning methods, and demonstrate its superiority to existing models. As far as we know, \textit{USCrop} is so far the most comprehensive crop yield dataset both spatially and temporally for machine learning. We will keep maintaining the dataset by including more features such as satellite images and progress data, and updating the dataset every year. We hope \textit{USCrop} can become a solid benchmark for crop yield study and an inspiration for novel machine learning techniques in spatiotemporal data modeling. We will also explore 
% more effective ways to better understand the influence of climatic variations. 

% In this paper, we present a large-scale crop yield dataset \textit{USCrop}, covering 48 states and 39 years. On \textit{USCrop}, we broadly compare the popular AI methods in crop yield prediction and provide solid baselines on three well-known metrics. Furthermore, we propose a novel GNN-RNN framework to innovatively incorporate both geospatial and temporal knowledge into prediction, through graph-based deep learning methods, and demonstrate its superiority to existing models. As far as we know, \textit{USCrop} is so far the most comprehensive crop yield dataset both spatially and temporally for machine learning. We will keep maintaining the dataset by including more features such as satellite images and progress data, and updating the dataset every year. We hope \textit{USCrop} can become a solid benchmark for crop yield study and an inspiration for novel machine learning techniques in spatiotemporal data modeling. We will also explore 
% more effective ways to better understand the influence of climatic variations. 