\section{Introduction}

Climate change \cite{houghton1990climate} has become a real and pressing challenge that poses many threats to our everyday life. Besides the evident extreme events \cite{trenberth2015attribution}, climatic variations also gradually impact the yields of major crops \cite{zhao2017temperature}. Crop production is  vulnerable and sensitive to fluctuations in climatic factors such as temperature, precipitation, soil, moisture and many other factors \cite{ortiz2018growing}. As the planet gets warmer, all of these swiftly-changing factors could perturb annual crop yields. Many recent works urge rethinking crop production practices under climate change \cite{reynolds2010adapting,raza2019impact}, which motivates the crop yield prediction problem \cite{van2020crop}. Crop yield prediction can help with food security \cite{shukla2019climate}, supply stability \cite{garrett2013land}, seed breeding \cite{ansarifar2020performance}, and economic planning \cite{horie1992yield}.

However, crop yield depends on numerous complex factors including weather, land, water, etc. While there exist specialized process-based models to simulate crop growth \cite{shahhosseini2021coupling}, they often produce highly-biased predictions, require strong assumptions about management practices, and are computationally expensive \cite{leng2020predicting}.
%Modeling all of them using hand-crafted agricultural models may be costly \cite{khaki2019crop}.
Therefore, in recent years, powerful yet inexpensive machine learning methods have been widely adopted in crop yield prediction and demonstrated impressive results \cite{romero2013using, dahikar2014agricultural, marko2016soybean, you2017deep, khaki2020cnn, khaki2020predicting}. Machine learning models (especially deep learning models) benefit from the large capacity, sophisticated non-linearity and mature techniques inherited from other application domains.

Despite the enormous amount of machine learning papers for crop yield prediction, many of them share similar methods. Among around 70 papers we surveyed, 48 used neural networks, 10 used tree-based methods (e.g. decision tree, random forest), and 10 used linear regressions (e.g. lasso). These methods often only differ in location (US, Brazil, India), study granularity (province, county, site/farm), crop types (soybean, corn), and time range (weeks to years). Similar findings are also reported in \cite{van2020crop}. 
For many relatively small self-collected datasets, simpler models are preferred. But these models may not perform well on a large and diverse region like the entire US.
%One major reason causing this is the lack of a publicly-available large and comprehensive crop yield dataset, and thus for many relatively small-size self-collected datasets, simple models are preferred.

In this paper, we compare various machine learning techniques on a nationwide scale, and propose a novel graph-based framework, GNN-RNN, which integrates both geospatial and temporal knowledge into inference. We train and compare these methods on over 2,000 counties from 41 states in the US mainland, with data covering years 1981 to 2019. There are up to 49 climatic and soil factors for each county, including precipitation, temperature, wind, soil moisture, soil quality, etc. Most of these factors vary across time within the year, or vary across different soil layers. All features are publicly available from sources such as PRISM, NLDAS, and gSSURGO. Furthermore, although not every county has crops planted, USDA provides corn yield labels from 41 states, and soybean yields from 31 states. 
%All features are preprocessed to vector form. 

Recent works using machine learning demonstrate promising results for crop yield prediction \cite{khaki2020cnn}. Nevertheless, these methods treat each county as an i.i.d. sample in their models, which is plausible in small regions but may not fully utilize the spatial structure of a larger region. For instance, if one county has a splendid harvest, the neighborhood counties are very likely to have high yields as well, which violates the independence assumption. It is also problematic to treat counties in the northern US and southern US as i.i.d. samples, as the distribution of their climatic and soil conditions is very different. We hence introduce \textbf{graph neural networks} (GNN) \cite{wu2020comprehensive} to take into account the geographical relationships among counties. When the model makes a prediction for a county, it can combine the features from neighboring counties with its own features to boost the predictive power. GNN models have been successful in many tasks such as election prediction \cite{jia2020residual} and COVID forecasting \cite{kapoor2020examining}. Additionally, we show that GNNs can work synergistically with RNNs to combine both geospatial and temporal information for prediction. We will show the novel \textbf{GNN-RNN} model can achieve superior performance in experiments. As far as we know, our work is the first to incorporate geographical knowledge into crop yield prediction. 
%It is also worth noting that our work shows that crop yield prediction is reciprocally a good test field for GNN field. \textit{USCrop} could provide an extra scenario beyond the conventional transportation or chemistry applications.
To further lay a solid foundation in this task, we compare various widely adopted machine learning methods with our method, including lasso \cite{tibshirani1996regression}, gradient boosting tree \cite{friedman2001elements}, CNN, RNN, CNN-RNN \cite{khaki2020cnn}. These are also predominant methods among the papers we surveyed. The experimental results show that GNN-based methods consistently outperform these existing models on the nationwide benchmark. On both RMSE and $R^2$, our GNN-RNN outperforms the state-of-the-art CNN-RNN model by 10\%.
