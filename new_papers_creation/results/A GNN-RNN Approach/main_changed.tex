\def\year{2022}\relax
%File: formatting-instructions-latex-2022.tex
%release 2022.1
\documentclass[letterpaper]{article} % DO NOT CHANGE THIS
\usepackage{aaai22}  % DO NOT CHANGE THIS
\usepackage{times}  % DO NOT CHANGE THIS
\usepackage{helvet}  % DO NOT CHANGE THIS
\usepackage{courier}  % DO NOT CHANGE THIS
\usepackage[hyphens]{url}  % DO NOT CHANGE THIS
\usepackage{graphicx} % DO NOT CHANGE THIS
\urlstyle{rm} % DO NOT CHANGE THIS
\def\UrlFont{\rm}  % DO NOT CHANGE THIS
\usepackage{natbib}  % DO NOT CHANGE THIS AND DO NOT ADD ANY OPTIONS TO IT
\usepackage{caption} % DO NOT CHANGE THIS AND DO NOT ADD ANY OPTIONS TO IT
\DeclareCaptionStyle{ruled}{labelfont=normalfont,labelsep=colon,strut=off} % DO NOT CHANGE THIS
\frenchspacing  % DO NOT CHANGE THIS
\setlength{\pdfpagewidth}{8.5in}  % DO NOT CHANGE THIS
\setlength{\pdfpageheight}{11in}  % DO NOT CHANGE THIS

%
% These are recommended to typeset algorithms but not required. See the subsubsection on algorithms. Remove them if you don't have algorithms in your paper.
\usepackage{algorithm}
\usepackage{algorithmic}
\usepackage{comment}
\usepackage{xcolor}
\usepackage{amsmath}
\usepackage{amssymb}
\usepackage{amsfonts}
\usepackage{subfig}
\usepackage{makecell}
\usepackage{arydshln}

\newcommand{\std}[1]{\fontsize{9}{10}\selectfont \emph{(#1)} \fontsize{10}{10}\selectfont}
\newcommand{\junwen}[1]{\textcolor{blue}{ #1 (Junwen)}}
\newcommand{\joshua}[1]{\textcolor{purple}{ #1 (Joshua)}}

\newcommand{\carla}[1]{\textcolor{magenta}{ #1}}
%
% These are are recommended to typeset listings but not required. See the subsubsection on listing. Remove this block if you don't have listings in your paper.
\usepackage{newfloat}
\usepackage{listings}
\lstset{%
	basicstyle={\footnotesize\ttfamily},% footnotesize acceptable for monospace
	numbers=left,numberstyle=\footnotesize,xleftmargin=2em,% show line numbers, remove this entire line if you don't want the numbers.
	aboveskip=0pt,belowskip=0pt,%
	showstringspaces=false,tabsize=2,breaklines=true}
\floatstyle{ruled}
\newfloat{listing}{tb}{lst}{}
\floatname{listing}{Listing}
%
%\nocopyright
%
% PDF Info Is REQUIRED.
% For /Title, write your title in Mixed Case.
% Don't use accents or commands. Retain the parentheses.
% For /Author, add all authors within the parentheses,
% separated by commas. No accents, special characters
% or commands are allowed.
% Keep the /TemplateVersion tag as is

% DISALLOWED PACKAGES
% \usepackage{authblk} -- This package is specifically forbidden
% \usepackage{balance} -- This package is specifically forbidden
% \usepackage{color (if used in text)
% \usepackage{CJK} -- This package is specifically forbidden
% \usepackage{float} -- This package is specifically forbidden
% \usepackage{flushend} -- This package is specifically forbidden
% \usepackage{fontenc} -- This package is specifically forbidden
% \usepackage{fullpage} -- This package is specifically forbidden
% \usepackage{geometry} -- This package is specifically forbidden
% \usepackage{grffile} -- This package is specifically forbidden
% \usepackage{hyperref} -- This package is specifically forbidden
% \usepackage{navigator} -- This package is specifically forbidden
% (or any other package that embeds links such as navigator or hyperref)
% \indentfirst} -- This package is specifically forbidden
% \layout} -- This package is specifically forbidden
% \multicol} -- This package is specifically forbidden
% \nameref} -- This package is specifically forbidden
% \usepackage{savetrees} -- This package is specifically forbidden
% \usepackage{setspace} -- This package is specifically forbidden
% \usepackage{stfloats} -- This package is specifically forbidden
% \usepackage{tabu} -- This package is specifically forbidden
% \usepackage{titlesec} -- This package is specifically forbidden
% \usepackage{tocbibind} -- This package is specifically forbidden
% \usepackage{ulem} -- This package is specifically forbidden
% \usepackage{wrapfig} -- This package is specifically forbidden
% DISALLOWED COMMANDS
% \nocopyright -- Your paper will not be published if you use this command
% \addtolength -- This command may not be used
% \balance -- This command may not be used
% \baselinestretch -- Your paper will not be published if you use this command
%  -- No page breaks of any kind may be used for the final version of your paper
% \columnsep -- This command may not be used
%  -- No page breaks of any kind may be used for the final version of your paper
% \pagebreak -- No page breaks of any kind may be used for the final version of your paper
% \pagestyle -- This command may not be used
% \tiny -- This is not an acceptable font size.
% \vspace{- -- No negative value may be used in proximity of a caption, figure, table, section, subsection, subsubsection, or reference
% \vskip{- -- No negative value may be used to alter spacing above or below a caption, figure, table, section, subsection, subsubsection, or reference

\setcounter{secnumdepth}{0} %May be changed to 1 or 2 if section numbers are desired.

% The file aaai22.sty is the style file for AAAI Press
% proceedings, working notes, and technical reports.
%

% Title

% Your title must be in mixed case, not sentence case.
% That means all verbs (including short verbs like be, is, using,and go),
% nouns, adverbs, adjectives should be capitalized, including both words in hyphenated terms, while
% articles, conjunctions, and prepositions are lower case unless they
% directly follow a colon or long dash
\title{A GNN-RNN Approach for Harnessing Geospatial and Temporal Information:\\  Application to  Crop Yield Prediction}
% \title{\carla{A GNN-RNN Approach for Harnessing Geospatial and Temporal Information:\\  Application to  Crop Yield Prediction}}



% \author{%
%   Joshua Fan\thanks{Equal contribution.} \\
%   Computer Science\\
%   Cornell University\\
%   Ithaca, NY 14853 \\
%   \texttt{jyf6@cornell.edu}
%   % examples of more authors
%   \And
%   Junwen Bai\footnotemark[1] \\
%   Computer Science \\
%   Cornell University \\
%   Ithaca, NY 14853 \\
%   \texttt{jb2467@cornell.edu} \\
%   \And
%   Zhiyun Li\footnotemark[1] \\
%   Applied Economics and Management \\
%   Cornell University \\
%   Ithaca, NY 14853 \\
%   \texttt{zl547@cornell.edu}
%   \And
%   Ariel Ortiz-Bobea \\
%   Applied Economics and Management \\
%   Cornell University \\
%   Ithaca, NY 14853 \\
%   \texttt{ao332@cornell.edu}
%   \And
%   Carla Gomes \\
%   Computer Science \\
%   Cornell University \\
%   Ithaca, NY 14853 \\
%   \texttt{gomes@cs.cornell.edu}
% }


%Example, Single Author, ->> remove \iffalse,\fi and place them surrounding AAAI title to use it
\iffalse
\title{My Publication Title --- Single Author}
\author {
    Author Name
}
\affiliations{
    Affiliation\\
    Affiliation Line 2\\
    name@example.com
}
\fi

%\iffalse
% %Example, Multiple Authors, ->> remove \iffalse,\fi and place them surrounding AAAI title to use it
% \title{My Publication Title --- Multiple Authors}
% \author {
%     % Authors
%     First Author Name,\textsuperscript{\rm 1}
%     Second Author Name, \textsuperscript{\rm 2}
%     Third Author Name \textsuperscript{\rm 1}
% }
% \affiliations {
%     % Affiliations
%     \textsuperscript{\rm 1} Affiliation 1\\
%     \textsuperscript{\rm 2} Affiliation 2\\
%     firstAuthor@affiliation1.com, secondAuthor@affilation2.com, thirdAuthor@affiliation1.com
% }

%\iffalse
\author{
    %Authors
    % All authors must be in the same font size and format.
    Joshua Fan\thanks{Equal contribution.}\textsuperscript{\rm 1}, Junwen Bai\footnotemark[1]\textsuperscript{\rm 1},  Zhiyun Li\footnotemark[1]\textsuperscript{\rm 2},  Ariel Ortiz-Bobea\textsuperscript{\rm 2}, Carla P. Gomes\textsuperscript{\rm 1}\\
}
\affiliations{
    %Afiliations
    \textsuperscript{\rm 1} Department of Computer Science, Cornell University, USA\\
    \textsuperscript{\rm 2} Department of Applied Economics \& Management, Cornell University, USA\\
    % If you have multiple authors and multiple affiliations
    % use superscripts in text and roman font to identify them.
    % For example,
    % Sunil Issar, \textsuperscript{\rm 2}
    % J. Scott Penberthy, \textsuperscript{\rm 3}
    % George Ferguson,\textsuperscript{\rm 4}
    % Hans Guesgen, \textsuperscript{\rm 5}.
    % Note that the comma should be placed BEFORE the superscript for optimum readability
    % 2275 East Bayshore Road, Suite 160\\
    % Palo Alto, California 94303\\
    % email address must be in roman text type, not monospace or sans serif
    \{jyf6, jb2467, zl547, ao332\}@cornell.edu, gomes@cs.cornell.edu
}
%\fi


% REMOVE THIS: bibentry
% This is only needed to show inline citations in the guidelines document. You should not need it and can safely delete it.
\usepackage{bibentry}
% END REMOVE bibentry


\begin{document}

\maketitle

\begin{abstract}
Climate change is posing new challenges to crop-related concerns, including food insecurity, supply stability, and economic planning. Accurately predicting crop yields is crucial for addressing these challenges. However, this prediction task is exceptionally complicated since crop yields depend on numerous factors such as weather, land surface, and soil quality, as well as their interactions. In recent years, machine learning models have been successfully applied in this domain. However, these models either restrict their tasks to a relatively small region, or only study over a single or few years, which makes them hard to generalize spatially and temporally. In this paper, we introduce a novel graph-based recurrent neural network for crop yield prediction, to incorporate both geographical and temporal knowledge in the model, and further boost predictive power. Our method is trained, validated, and tested on over 2000 counties from 41 states in the US mainland, covering years from 1981 to 2019. As far as we know, this is the first machine learning method that embeds geographical knowledge in crop yield prediction and predicts crop yields at the county level nationwide. We also laid a solid foundation by comparing our model on a nationwide scale with other well-known baseline methods, including linear models, tree-based models, and deep learning methods. Experiments show that our proposed method consistently outperforms the existing state-of-the-art methods on various metrics, validating the effectiveness of geospatial and temporal information.

%Climate change is posing new challenges to crop-related concerns including food insecurity, supply stability and economic planning. As one of the central challenges, crop yield prediction has become a pressing task in the machine learning field. Despite its importance, the prediction task is exceptionally complicated since crop yields depend on various factors such as weather, land surface, soil quality as well as their interactions. In recent years, machine learning models have shown many successful applications in the crop domain. However, these models either restrict their tasks to a relatively small region, or only span over a single or few years, which makes them hard to generalize spatially and temporally. In this paper, we present a new crop yield dataset comprising of plenteous weather, land, soil variables across over 3000 counties from all 48 states in the US mainland, covering years from 1981 to 2019. As far as we know, this is the first nationwide dataset with such a long time span in the crop domain for machine learning. On this dataset, we lay a solid foundation for machine learning tasks by applying well-known linear models, tree-based models, deep learning methods and comparing their performance. Moreover, we introduce a novel graph-based recurrent neural network for crop yield prediction, to incorporate both geographical and temporal knowledge in the model, and further boost the predictive power.
\end{abstract}

\section{Introduction}
%%%%%%%%%%%%%%%%%%%%%%%%%%%%%%
% 1.定义image captioning任务 
%%%%%%%%%%%%%%%%%%%%%%%%%%%%%%
Image captioning is a fundamental task in vision-language understanding that involves generating natural language descriptions for a given image. It plays a critical role in facilitating more complex vision-language tasks, such as visual question answering \cite{Agrawal2015VQAVQ,gqa,okvqa} and visual dialog \cite{Das2016VisualD,Niu2018RecursiveVA,llava}.
%%%%%%%%%%%%%%%%%%%%%%%%%%%%%%
% text-only training 的介绍
%%%%%%%%%%%%%%%%%%%%%%%%%%%%%%
The mainstream image captioning methods \cite{conimgcap4,conimgcap1,conimgcap3,conimgcap2} require expensive human annotation of image-text pairs for training neural network models in an end-to-end manner. Recent developments in Contrastive Image Language Pre-training (CLIP) \cite{clip} have enabled researchers to explore a new paradigm, zero-shot image captioning, through text-only training. In particular, CLIP learns a multi-modal embedding space where semantically related images and text are encoded into features with close proximity. As such, if a model learns to map the CLIP text features to their corresponding texts, it is feasible to generate image captions from the CLIP image features without needing supervision from caption annotations.

%%%%%%%%%%%%%%%%%%%%%%%%%%%%%%
% text-only training 的优势
%%%%%%%%%%%%%%%%%%%%%%%%%%%%%%

One main advantage of this zero-shot captioning paradigm is that it enables a Large Language Model (LLM) \cite{gpt3, Zhang2022OPTOP} with image captioning capabilities using only text data and affordable computational resources. Despite the impressive performance achieved by recent powerful multimodal models \cite{miniGPT4,llava}, they typically require large-scale, high-quality human-annotated data and expensive computational resources for fine-tuning an LLM. Zero-shot captioning methods can significantly reduce such costs, which is particularly important in situations of data scarcity and limited resources. Moreover, recent work \cite{Guo2022FromIT, Changpinyo2022AllYM,Tiong2022PlugandPlayVZ} demonstrates that other vision-language tasks, such as VQA, can be addressed by LLMs and image captions. Consequently, the paradigm of zero-shot captioning has the potential to pave the way to solving complex vision-language tasks with LLMs through efficient text-only training. 


%%%%%%%%%%%%%%%%%%%%%%%%%%%%%%
% zero-shot image captioning via text-only training 的challenge
%%%%%%%%%%%%%%%%%%%%%%%%%%%%%%
A critical challenge in zero-shot image captioning through text-only training is to mitigate a widely observed phenomenon known as the \textit{modality gap}. While the features of paired texts and images are close in the CLIP embedding space, there remains a gap between them \cite{MindGap}. This gap often results in inaccurate mappings from the image embeddings to the text ones. Consequently, without fine-tuning with paired data, it significantly impairs the performance of zero-shot image captioning.
%%%%%%%%%%%%%%%%%%%%%%%%%%%%%%
% current works intro
%%%%%%%%%%%%%%%%%%%%%%%%%%%%%%
Several works have attempted to address the modality gap in zero-shot image captioning, relying mainly on two strategies: (1) The first strategy leverages a memory bank from training text data to project visual embeddings into the text embedding space \cite{DeCap}. However, this projection prevents it from representing any semantic content outside the distribution of the memory bank features and introduces extra inference costs; (2) The second approach injects noise during training to encourage the visual embeddings to be included inside the semantic neighborhood of the corresponding text embeddings \cite{CapDec}. Nonetheless, the noise injection tends to diffuse the distribution of visual inputs at the cost of weakening the semantic correlation between paired images and text embeddings. 

%However, in the first strategy, the projection of visual embeddings prevents them from  For the second strategy, noise injection during training diffuses the input distribution at the cost of degrading the semantic correlation between paired images and text embeddings.

%Previous attempts \cite{CapDec,DeCap} to reduce the modality gap in zero-shot image captioning can be summarized into two aspects: (1) Decap\cite{DeCap} leverages a memory bank from training text data to project visual embeddings into text embedding space. However, the projection of visual embeddings prevents it from representing any semantic content outside the distribution of the memory bank and introduce extra inference cost. (2) CapDec\cite{CapDec}proposes to inject noise during training to encourage the visual embedding to be included inside the text embedding space. 
% current work weakness
%Nevertheless, noise injection during training diffuses the input distribution at the cost of degrading the semantic correlation between paired images and text embeddings.


%%%%%%%%%%%%%%%%%%%%%%%%%%%%%%
% 我们工作的流程
% 分析得到两个结论:1.subregion带来更好的匹配2.image text gap符合高斯分布
%%%%%%%%%%%%%%%%%%%%%%%%%%%%%%
To tackle these challenges, we first conduct a thorough analysis of the CLIP feature space, leading to two key observations. First, most text descriptions are unable to fully capture the content of their paired images. However, we empirically find that the visual embedding of certain local regions of an image, named image subregions, have closer proximity to the text embedding of the paired caption. Integrating such image subregions with the global image representation generates a tighter alignment between image and text. Additionally, we analyze the distribution of the gap between the CLIP features of image or subregion-text pairs and find that it closely resembles a zero-mean Gaussian distribution.
%initiate our investigation by conducting a thorough analysis of the CLIP latent space. Building upon the insights from the work \cite{MindGap}, we identify a key factor contributing to the existence of a modality gap. Due to the inherent disparities between textual and visual modalities, text is incapable of comprehensively describing the information within an image. However, we empirically demonstrate that the CLIP embedding of some part of image, named image subregions, exhibit closer proximity to the CLIP embedding of the paired caption. The integration between image subregion information and global image feature leads to more compact image text alignment. Besides, we collect the statistics of the gap between CLIP image and text feature. The results demonstrate the gap is close to gaussian distribution. 

%%%%%%%%%%%%%%%%%%%%%%%%%%%%%%
% 我们的方法简略介绍
%%%%%%%%%%%%%%%%%%%%%%%%%%%%%%

Based on our findings, we propose a novel zero-shot image captioning framework, named \textit{\textbf{M}ining Fine-Grained Image-Text \textbf{A}lignment in \textbf{C}LIP for \textbf{Cap}tioning} (MacCap), to address the aforementioned challenges. In this framework, we introduce a region-aware cross-modal representation based on CLIP and an effective unimodal training strategy for an LLM-based caption generator. Our cross-modal representation maps an input image into the language space of LLMs and consists of two main components. First, we design a \textit{sub-region feature aggregation} module to fuse both global and subregion-level CLIP image features, resulting in a smaller gap between the corresponding CLIP text embedding. Next, we introduce a learnable adaptor-decoder to transform the CLIP representation into the LLM's language space.
To train our model with text-only data, we develop a robust procedure to learn a projection from the CLIP embedding space to a language representation, enabling the LLM to reconstruct captions. Specifically, our learning procedure first injects noise into our region-aware CLIP-text representation, mimicking the modality gap between image and text features. This is followed by a multiple sampling and filtering step that leverages the CLIP knowledge to improve the quality of the captioning.
%tackles the problem from three key perspectives. Firstly, we focus on learning a robust projection from CLIP embedding space to language model space by text reconstruction training, which enable model to generate text based on both CLIP image and text feature. The region noise injection in training alleviate the \textit{modality gap} between image and text feature, which makes the projection works for both image and text features. Secondly, we design \textit{sub-region feature aggregation} to obtain a more compact CLIP image feature, which is based on the observation that CLIP subregion feature exhibit closer disntance with corresponding text feature. Third, we propose multiple sampling and filtering to mitigate the drawbacks of noise injection, which leverage CLIP knowledge to further boost caption performance. Finally, we design a pipeline for zero-shot VQA to demonstrate the extensibility of ouir methods to more intricate vision-language tasks.
In addition to the image captioning task, we further extend our framework to build a zero-shot VQA pipeline, demonstrating the generality of our cross-modal representation for more complex vision-language tasks.

%%%%%%%%%%%%%%%%%%%%%%%%%%%%%%
% 我们的方法简略介绍
%%%%%%%%%%%%%%%%%%%%%%%%%%%%%%

We evaluate our framework on several widely-adopted image captioning benchmarks, such as MSCOCO \cite{mscoco} and Flickr30k \cite{Flickr30k}, as well as a standard VQA benchmark, VQAV2 \cite{vqav2}. Our extensive experiments cover multiple vision-language tasks, including zero-shot in-domain image captioning, zero-shot cross-domain image, and zero-shot VQA. The results not only demonstrate the superiority of our methods but also validate our findings on the CLIP embedding space.

% demonstrate through experiments that our proposed methods outperform previous approaches on popular captioning benchmarks, such as MSCOCO, Flickr30k, which further verify our understanding of \textit{concept region}



% Specifically, we evaluate the distribution of the image and text embedding space under hyperspherical coordinates and observe a geometric phenomenon \textit{concept region} 
% where semantically correlated image and text embedding tend to clustering despite the \textit{modality gap}.
% 我们基于concept region的观察提出的方法:concept region和modality gap的cause里面有mismatch pair data导致的semantic ambiguity,总体思路是在train的时候模拟在concept region。在training的时候,我们给text embedding加上region noise,具体而言就是以原本text embedding为中心,一定范围内的多个随机sample的related text embedding,这样的获得的text embedding全都是在输入text对应的concept region内部。在zs captioning的inference时,部分image sub-region inforamtion 会比global image 对text匹配度更高,因此我们基于部分image sub-region inforamtion
% Motivated by the semantic ambiguity of mismatched data observed in \textit{concept region}, we propose two 
% an image sub-region information aggregation strategy for .In detail

% result summary


% \begin{figure}[t]
    \centering
    \subfloat[relationship among scenes]{\resizebox{0.22\textwidth}{!}{
        \includegraphics[]{figure/chord.png}
    }}
    \hspace{4mm}
    \subfloat[car instance $ln$ distribution]{\resizebox{0.22\textwidth}{!}{
        \includegraphics[]{figure/polar.png}
    }}
    \caption{The images in our RGB-P Car dataset vary in terms of (a) scenarios and (b) the number of car instances.}
    \label{fig:dataset}
\end{figure}

\begin{table}[tp]
\caption{Comparison of existing car detection datasets with polarization measurements.}
\small
\centering
\setlength{\tabcolsep}{2.6pt}
\begin{tabular}{c|c|c|c|c}
\hline\hline
Datasets         & Pol. & \begin{tabular}[c]{@{}c@{}}Pixel\\ align\end{tabular} & \begin{tabular}[c]{@{}c@{}}Num.images \\ Train / Test\end{tabular} & \begin{tabular}[c]{@{}c@{}}Num. cars \\ Train / Test \end{tabular} \\ 
\hline
PolarLITIS       & Mono & $\times$                                                     & \begin{tabular}[c]{@{}c@{}}2569 \\ 1640 / 929 \end{tabular}         & \begin{tabular}[c]{@{}c@{}}17428\\ 6061 / 11367 \end{tabular}    \\
\hline
\textbf{RGBP-Car (Ours)} & Tri  & \checkmark                                                     & \begin{tabular}[c]{@{}c@{}}2601 \\ 1611 / 990 \end{tabular}         & \begin{tabular}[c]{@{}c@{}}31234 \\ 19582 / 11652 \end{tabular}   \\ 
\hline\hline
\end{tabular}
\label{tab:datasetcomp}
\end{table}


\section{RGB-P Car Detection Dataset}
\label{sec:dataset}
We construct the first pixel-aligned RGB-polarization car detection dataset called RGBP-Car with trichromatic polarization measurements. We record cars in diverse traffic scenes using FLIR-Blackfly-S, a polarized color camera that simultaneously obtain pixel-aligned polarization measurements in four linear polarization directions (0$^\circ$, 45$^\circ$, 90$^\circ$, and 135$^\circ$) for each color channel (\textit{i.e.}, R, G, and B). RGBP-Car contains 2601 RGB, AoLP, and DoLP image triplets. Each image has manually labeled bounding boxes indicating the position and size of each car. To ensure the diversity and challenge of our dataset, we take the RGB-P images under different weather conditions (clear and rainy), different lighting conditions (daytime and nighttime), different driving environments (indoor, outdoor, road and parking lot), and different car densities. 
Fig. \ref{fig:samples} gives representative examples and Fig. \ref{fig:dataset} analyzes (a) the relationship among different scenes and (b) the density distribution of car instances. Tab. \ref{tab:datasetcomp} further shows the superiority of our RGBP-Car over existing car detection datasets with polarization measurements.



\section{Methods}

\subsection{Problem Formulation}
In crop yield prediction, we denote each county's climatic features by $\mathbf{x}_{c,t}$ and ground-truth crop yield (for a particular crop) by $y_{c,t}\in \mathbb{R}$, where $c$, $t$ represent county and year respectively. Each $\mathbf{x}_{c,t}$ contains four types of features (detailed descriptions of these features can be found in the Experiments section): weather features $\mathbf{x}_{c,t}^w\in \mathbb{R}^{n_w\times 52}$, land surface features $\mathbf{x}_{c,t}^l\in \mathbb{R}^{n_l\times 52}$, soil quality features $\mathbf{x}_{c}^s\in \mathbb{R}^{n_s\times 6}$, and some extra features (e.g. crop production index) $\mathbf{x}_{c}^e\in \mathbb{R}^{n_e}$. Namely, $\mathbf{x}_{c,t}=(\mathbf{x}_{c,t}^w, \mathbf{x}_{c,t}^l, \mathbf{x}_{c}^s, \mathbf{x}_{c}^e)$. We denote the number of weather, land surface, soil quality, and extra variables as  $n_w, n_l, n_s, n_e$ respectively. Among these features, $\mathbf{x}_{c,t}^w, \mathbf{x}_{c,t}^l$ change both spatially and temporally, while $\mathbf{x}_{c}^s, \mathbf{x}_{c}^e$ are county-specific and remain stable over time. The goal is to predict $y_{c,t}$ given $\mathbf{x}_{c,t}$. Recent work \cite{khaki2020cnn} also showed features from past years can help with the prediction, so we reformulate our task as predicting $y_{c,t}$ with $\{\mathbf{x}_{c,t},\mathbf{x}_{c,t-1},...,\mathbf{x}_{c,t-\Delta t}\}$. $\Delta t$ is the length of year dependency. If $\Delta t=0$, the model will not consider features from prior years. 

\subsection{Per-Year Embedding Extraction}
Regardless of whether the models use historical features or not, the first step is always to extract an embedding for each year from $\mathbf{x}_{c,t}$. Then a prediction can be made based on the embedding from the current year or the embeddings from the last few years.

The four types of features $\mathbf{x}_{c,t}^w, \mathbf{x}_{c,t}^l, \mathbf{x}_{c}^s, \mathbf{x}_{c}^e$ have different structures. Using a uniform neural network to extract the embedding may not effectively exploit the structure in the raw data. For example, weekly features $\mathbf{x}_{c,t}^w, \mathbf{x}_{c,t}^l$ naturally incorporate a temporal order, but county-specific soil features $\mathbf{x}_{c}^s$ do not change temporally and are measured at different depths underground. Therefore, we use separate neural networks to process the differently structured-parts from $\mathbf{x}_{c,t}$:
\begin{equation}
\label{eq:cnn}
\begin{aligned}
&\mathbf{h}_{c,t}^{wl}=f_{wl}(\mathbf{x}_{c,t}^w, \mathbf{x}_{c,t}^l) \\
&\mathbf{h}_{c}^s=f_s(\mathbf{x}_{c}^s) \\
&\mathbf{h}_{c,t}=(\mathbf{h}_{c,t}^{wl}, \mathbf{h}_c^s, \mathbf{x}_{c}^e)
\end{aligned}
\end{equation}
$f_{wl}(\cdot)$ handles the features that vary over time. Since land surface features like soil moisture from $\mathbf{x}_{c,t}^l$ are weekly data closely related to weather, we concatenate $\mathbf{x}_{c,t}^l$ and $\mathbf{x}_{c,t}^w$ before further passing to $f_{wl}$. Given the temporal order, an RNN or a CNN can be used for $f_{wl}$ to facilitate information aggregation along the time axis. On the other hand, $f_s(\cdot)$ aggregates information along soil depths. We use CNN as the architecture for $f_s$. $\mathbf{x}_{c}^e$ only contains six scalar values, so we directly pass it to the output embedding. The final embedding $\mathbf{h}_{c,t}$ is the concatenation of $\mathbf{h}_{c,t}^{wl}, \mathbf{h}_c^s, \mathbf{x}_{c}^e$.


\begin{figure*}[t]
\centering
\begin{minipage}[c]{7cm}
\includegraphics[width=6.9cm]{figs/cnn.png}
\label{fig:cnn}
\end{minipage}
\begin{minipage}[c]{10cm}
\includegraphics[width=9.9cm]{figs/gnn-rnn.png}
\label{fig:gnn-rnn}
\end{minipage}
\caption{\textbf{Left}: The CNN model used for per-year embedding extraction. \textbf{Right}: Our overall GNN-RNN framework. For each county $c$ and year $t'$, the CNN extracts an embedding $\mathbf{h}_{c, t'}$. Then we apply a GNN to refine each year's embedding by aggregating information from neighboring counties, producing a new embedding $\mathbf{z}_{c, t'}$. Finally, an LSTM processes the embeddings from each year and outputs the yield prediction $\widehat{y}_{c, t}$.}
\end{figure*}


\subsection{Temporal Dependency}
Though new crops are planted every year and yields primarily depend on climatic factors within one year, it has been observed that the trend and variations captured by recent history can be very informative for prediction \cite{khaki2020cnn}. For example, crop yields have tended to increase over the past few decades due to improvements in technology and genetics \cite{ortiz2018another}. While data on the underlying technological improvement is unavailable \cite{khaki2020cnn}, we can observe recent trends in crop yield. Our per-year embedding extraction makes it easy to incorporate  historical knowledge. All we need is an RNN that reads the per-year embeddings from the current year and several prior years. The output from the last time step would be our prediction for the crop yield of the current year: 
\begin{equation}
\label{eq:rnn}
\begin{aligned}
\widehat{y}_{c,t}=r(\mathbf{h}_{c,t-\Delta t}, ..., \mathbf{h}_{c,t-1}, \mathbf{h}_{c,t})
\end{aligned}
\end{equation}
where $r(\cdot)$ is an RNN, and $\mathbf{h}_{c,t'}$ is the embedding from year $t'$ for county $c$. The model described so far follows the CNN-RNN framework, which has previously been shown to outperform single-year NN models \cite{khaki2020cnn}.


\subsection{Incorporating Geographical Knowledge}
Eq.~\ref{eq:rnn} shows how one can extend the use of embeddings from Eq.~\ref{eq:cnn} temporally. Then a natural question is, Can we take advantage of the embeddings geospatially as well? Intuitively, if a county has good yields, nearby counties tend to have good yields as well. The weather and soil conditions should also transition smoothly across the continent. The additional features from neighboring counties could boost the prediction if used properly. A recent success in COVID-19 forecasting \cite{kapoor2020examining} with similar insights could further support incorporating geographical knowledge, where the graph-based representation learning greatly improves case prediction. 

\subsubsection{Graph Neural Network}
Graph Neural Network (GNN) \cite{zhou2020graph} is a novel type of neural network proposed to unravel the complicated dependencies inherent in graph-structured data sources.
%GNN allows more flexibility and a wider representation space to embed the node and edge information from the graph for inference.
Given its strong power in representation learning, GNN has demonstrated prominent applications in chemistry \cite{gilmer2017neural}, traffic \cite{cui2019traffic}, biology \cite{fout2017protein}, and computer vision \cite{satorras2018few} with sophisticated model architectures \cite{kipf2016semi,hamilton2017inductive,velivckovic2017graph}. Formally, a graph is denoted by $G=(V,E)$ where $V$ is the set of nodes and $E$ is the set of edges between nodes. In our crop yield prediction task, each node is a county. $E$ is represented as a symmetric adjacency matrix $A\in \{0,1\}^{N\times N}$ where $A_{i,j}=1$ if two counties $v_i, v_j\in V$ border and $A_{i,j}=0$ otherwise. $N$ is the total number of counties. Each node is associated with $\mathbf{x}_{c,t}$ for every year. 

\subsubsection{GraphSAGE} 
A popular GNN model, GraphSAGE, \cite{hamilton2017inductive} is a general framework that leverages node feature information and learns node embeddings through aggregation from a node's local neighborhood. Unlike many other methods based on matrix factorization and normalization \cite{jia2020residual}, GraphSAGE simply aggregates the features from a local neighborhood, and is thus less computationally expensive. The features can be aggregated from a different number of hops or search depth. Therefore the model often generalizes better. GraphSAGE is suitable for crop yield prediction because most counties only border a few others and the adjacency matrix is sparse. It also provides flexible aggregation methods.

Formally, for the $l$-th layer of GraphSAGE, 
\begin{equation}
\label{eq:gnn}
\begin{aligned}
&\mathbf{a}_{c,t}^{(l)} = g_l(\{\mathbf{z}_{c',t}^{(l-1)},\forall c'\in\mathcal N(c)\})\\
&\mathbf{z}_{c,t}^{(l)} = \sigma(\mathbf{W}^{(l)}\cdot (\mathbf{z}_{c,t}^{(l-1)}, \mathbf{a}_{c,t}^{(l)}))
\end{aligned}
\end{equation}
where $\mathbf{z}_{c,t}^{(0)}=h_{c,t}$ from Eq.~\ref{eq:cnn}, and $l\in\{0,1,...,L\}$. $\mathcal N(c)=\{c', \forall A_{c,c'}=1\}$ is the set of neighboring counties for $c$. The aggregation function for the $l$-th layer is denoted $g_l(\cdot)$, which could be mean, pooling, or graph convolution (GCN) function. In practice, we found mean or pooling are effective and computationally efficient. $\mathbf{a}_{c,t}^{(l)}$ is the aggregated embedding from the bordering counties. We concatenate $\mathbf{a}_{c,t}^{(l)}$ with the last layer's embedding $\mathbf{z}_{c,t}^{(l-1)}$ before the transformation using $\mathbf{W}^{(l)}$. $\sigma(\cdot)$ is a non-linear function.

\subsubsection{GNN-RNN}
The output embedding from GNN's last layer $\mathbf{z}_{c,t}^{(L)}$ thus extracts the information (e.g., weather, soil) from the whole local neighborhood for year $t$. To integrate the historical knowledge, we can do the same as in Eq.~\ref{eq:rnn}, by taking the GNN output embeddings from prior years:
\begin{equation}
\label{eq:gnn-rnn}
\begin{aligned}
\widehat{y}_{c,t}=r(\mathbf{z}_{c,t-\Delta t}^{(L)}, ..., \mathbf{z}_{c,t-1}^{(L)}, \mathbf{z}_{c,t}^{(L)})
\end{aligned}
\end{equation}
where $\mathbf{z}_{c,t'}^{(L)}$ is the GNN embedding from year $t'$.

\subsubsection{Loss Function}
We use log-cosh function as our objective:
\begin{equation}
\begin{aligned}
L(\widehat{y}_{c,t}, y_{c,t})=\log(\text{cosh}(\widehat{y}_{c,t}-y_{c,t}))
\end{aligned}
\end{equation}
Log-cosh works similarly to mean square error, but is not as strongly affected by the occasional wildly incorrect prediction. It is also twice differentiable everywhere. Mini-batch training is adopted during optimization. Batch loss is the average log-cosh loss of all samples in a batch. 


%!TEX root =  main.tex
\section{Related Work}\label{sec:related}

The matching market model characterizes many applications such as labor market \citep{roth1984evolution}, house allocation \citep{abdulkadirouglu1999house}, college admission and marriage problems \citep{gale1962college}, among which the  many-to-one setting is very common and widely studied \citep{roth1992two}. 
Responsiveness and substitutability are most generally known conditions to guarantee the existence of a stable matching \citep{kelso1982job,roth1984stability,roth1992two,abdulkadirouglu2005college} and the deferred acceptance (DA) algorithm is a classical offline algorithm to find a stable matching under this property \citep{kelso1982job,roth1984stability}. 


For simplicity, we refer to the setting where one-side participants (players) have unknown preferences as the online setting. 
This line of works relies on the technique of MAB, a classical online learning framework with a single player and $K$ arms \citep{lattimore2020bandit}. 
The explore-then-commit (ETC) \citep{garivier2016explore}, upper confidence bound (UCB) \citep{auer2002finite}, Thompson sampling (TS) \citep{agrawal2012analysis} and elimination \citep{auer2010ucbelimination} algorithms are common strategies to obtain $O(K\log T/\Delta)$ regret where $\Delta$ is the minimum suboptimality gap among arms. 

Multiple-player MAB (MP-MAB) generalizes the standard MAB problem by considering more than one player in the environment. 
In this setting, each player selects an arm at each time and a player will receive nothing if it collides with others by selecting the same arm. 
The MP-MAB problem has been studied in both homogeneous and heterogeneous settings. The former assumes different players share the same preference over arms \citep{rosenski2016multi,bubeck2021cooperative} and the latter allows players to have different preferences \citep{bistritz2018distributed,shi2021heterogeneous}. 
Both settings aim to minimize the collective cumulative regret of all players. 
% which is compared with the maximum collective reward received by all players. 


The matching market problem is different from above MP-MAB framework by considering that each arm also has a preference ranking over players.  
When multiple players select one arm, the player preferred most by the arm would not be collided and would gain a reward.   
The objective in this setting is to learn a stable matching and minimize the stable regret for players. 
\citet{das2005two} first introduce the bandit learning problem in one-to-one matching markets and explore the empirical performances of the proposed algorithms in the market where all participants on each side have the same preferences. 
Recently, \citet{liu2020competing} study a variant of the problem and present the first theoretical guarantees in a centralized setting where a central platform assigns allocations to players in each round. 
Later,  \citet{sankararaman2021dominate}, \citet{basu21beyond} and \citet{maheshwari2022decentralized} successively study this setting in a decentralized manner where players make their own decisions without a central platform. 
These works additionally assume the preferences of participants satisfy some constraints to ensure the uniqueness of the stable matching.  
For a general decentralized one-to-one market, \citet{liu2021bandit} and \citet{kong2022thompson} propose UCB and TS-type algorithms with guarantees for player-pessimal stable regret, respectively. 
Until recently, the theoretical analysis for the stronger player-optimal stable regret objective has been derived \citep{zhang2022matching,kong2023player}. 

Due to the generality when modeling real applications, \citet{wang2022bandit} start to study the bandit problem in many-to-one settings. They assume arms have responsive preferences and derive algorithms both in centralized and decentralized settings. For the decentralized setting, they only guarantee the player-pessimal stable regret with the upper bound $O(N^5K^2\log^2 T/(\varepsilon^{N^4}\Delta^2))$ where $\varepsilon\in(0,1)$ is a hyper-parameter. 
Please see Table \ref{table:comparison} for a comprehensive comparison among these works. 

% \shuai{introduce table 1 somewhere. table 1 does not introduce N,K,Delta}

\section{EXPERIMENTS}
In this section, we conduct experiments on two widely used datasets to answer the following research questions:
\begin{itemize}
	\item RQ1: How does GRMM perform compared with different retrieval methods (typically traditional, local interaction-based, and BERT-based matching methods)?
	\item RQ2: How effective is the graph structure as well as the long-dependency in ad-hoc retrieval?
	\item RQ3: How sensitive (or robust) is GRMM with different hyper-parameter settings?
\end{itemize}

\subsection{Experiment Setup}
\subsubsection{Datasets.}
We evaluate our proposed model on two datasets: Robust04 and ClueWeb09-B.
\begin{itemize}
    \item Robust04\footnote{https://trec.nist.gov/data/cd45/index.html} is a standard ad-hoc retrieval dataset with 0.47M documents and 250 queries, using TREC disks 4 and 5 as document collections.
    \item ClueWeb09-B\footnote{https://lemurproject.org/clueweb09/} is the "Category B" subset of the full web collection ClueWeb09. It has 50M web pages and 200 queries, whose topics are accumulated from TREC Web Tracks 2009-2012.
\end{itemize}
Table \ref{tab:1} summarises the statistic of the two collections. For both datasets, there are two available versions of the query: a keyword title and a natural language description. In our experiments, we only use the title for each query.


\subsubsection{Baselines.}
To examine the performance of GRMM, we take three categories of retrieval models as baselines, including traditional (QL and BM25), local interaction-based (MP, DRMM, KNRM, and PACRR), and BERT-based (BERT-MaxP) matching methods, as follows: 

\begin{itemize}
    \item \textbf{QL} (Query likelihood model) \cite{zhai2004study} is one of the best performing language models that based on Dirichlet smoothing.
    \item \textbf{BM25} \cite{robertson1994some} is another effective and commonly used classical probabilistic retrieval model.
    \item \textbf{MP} (MatchPyramid) \cite{pang2016text} employs CNN to extract the matching features from interaction matrix, and dense neural layers are followed to produce final ranking scores.
    \item \textbf{DRMM} \cite{guo2016deep} performs a histogram pooling over the local query-document interaction signals. 
    \item \textbf{KNRM} \cite{xiong2017end} introduces a new kernel-pooling technique that extracts multi-level soft matching features.
    \item \textbf{PACRR} \cite{hui2017pacrr} uses well-designed convolutional layers and $k$-max-pooling layers over the interaction signals to model sequential word relations in the document.
    \item \textbf{Co-PACRR} \cite{hui2018co} is a context-aware variant of PACRR that takes the local and global context of matching signals into account.
    \item \textbf{BERT-MaxP} \cite{dai2019deeper} applies BERT to provide deeper text understanding for retrieval. The neural ranker predicts the relevance for each passage independently, and the document score is set as the best score among all passages.
\end{itemize}


\begin{table}[]
	\footnotesize
	\begin{tabular}{@{}ccccc@{}}
		\toprule
		\textbf{Dataset}     & \textbf{Genre} & \textbf{\# of qrys} & \textbf{\# of docs} & \textbf{avg.length} \\ \midrule
		\textbf{Robust04}    & news           & 250                 & 0.47M                & 460                         \\
		\textbf{ClueWeb09-B} & webpages       & 200                 & 50M                 & 1506                        \\ \bottomrule
	\end{tabular}
	\caption{Statistics of datasets.}
	\label{tab:1}
\end{table}


\subsubsection{Implementation Details.}
All document and query words were white-space tokenised, lowercased, and lemmatised using the WordNet\footnote{https://www.nltk.org/howto/wordnet.html}. We discarded stopwords as well as low-frequency words with less than ten occurrences in the corpus. Regarding the word embeddings, we trained 300-dimensional vectors with the Continuous Bag-of-Words (CBOW) model \cite{mikolov2013distributed} on Robust04 and ClueWeb-09-B collections. For a fair comparison, the other baseline models shared the same embeddings, except those who do not need. Implementation of baselines followed their original paper.

Both datasets were divided into five folds. We used them to conduct 5-fold cross-validation, where four of them are for tuning parameters, and one for testing \cite{macavaney2019cedr}. The process repeated five times with different random seeds each turn, and we took an average as the performance.

We implemented our method in PyTorch\footnote{Our code is at https://github.com/CRIPAC-DIG/GRMM}. The optimal hyper-parameters were determined via grid search on the validation set: the number of graph layers $t$ was searched in \{1, 2, 3, 4\}, the $k$ value of $k$-max-pooling was tuned in \{10, 20, 30, 40, 50, 60, 70\}, the sliding window size in \{3,5,7,9\}, the learning rate in \{0.0001, 0.0005, 0.001, 0.005, 0.01\}, and the batch size in \{8, 16, 32, 48, 64\}.
Unless otherwise specified, we set $t$ = 2 and $k$ = 40 to report the performance (see Section \ref{sec:neighbouraggre} and \ref{sec:featureelect} for different settings), and the model was trained with a window size of 5, a learning rate of 0.001 by Adam optimiser for 300 epochs, each with 32 batches times 16 triplets. All experiments were conducted on a Linux server equipped with 8 NVIDIA Titan X GPUs.

\subsubsection{Evaluation Methodology.}
Like many ad-hoc retrieval works, we adopted a re-ranking strategy that is more efficient and practical than ranking all query-document pairs. In particular, we re-ranked top 100 candidate documents for each query that were initially ranked by BM25. To evaluate the re-ranking result, we used the normalised discounted cumulative gain at rank 20 (nDCG@20) and the precision at rank 20 (P@20) as evaluation matrices. 


\subsection{Model Comparison (RQ1)}
Table \ref{tab:2} lists the overall performance of different models, from which we have the following observations:
\begin{itemize}
	\item GRMM significantly outperforms traditional and local interaction-based models, and it is comparable to BERT-MaxP, though without massive external pre-training. To be specific, GRMM advances the performance of nDCG@20 by 14.4\% on ClueWeb09-B much more than by 5.4\% on Robust04, compared to the best-performed baselines excluding BERT-MaxP. It is reasonably due to the diversity between the two datasets. ClueWeb09-B contains webpages that are usually long and casual, whereas Robust04 contains news that is correspondingly shorter and formal. It suggests that useful information may have distributed non-consecutively, and it is beneficial to capture them together, especially for long documents. GRMM can achieve long-distance relevance matching through the graph structure regardless of the document length. 
	
	\item On the contrary, BERT-MaxP performs relatively better on Robust04 than on ClueWeb09-B. We explain the observation with the following two points. First, since the input sequence length is restricted by a maximum of 512 tokens, BERT has to truncate those long documents from ClueWeb09-B into several passages. It, therefore, loses relations among different passages, i.e. the long-distance dependency. Second, documents from Robust04 are generally written in formal languages. BERT primarily depends on the pre-trained semantics, which could naturally gain benefit from that. 
	
	\item Regarding the local interaction-based models, their performances slightly fluctuate around the initial ranking result by BM25. However, exceptions are DRMM and KNRM on ClueWeb09-B, where the global histogram and kernel pooling strategy may cause the difference. It implies that the local interaction is insufficient in ad-hoc retrieval task. Document-level information also needs to be considered. 
	
	\item Traditional approaches like QL and BM25 remain a strong baseline though quite straightforward, which means the exact matching of terms is still of necessity as \citet{guo2016deep} proposed. These models also avoid the problem of over-fitting, since they do not require parameter optimisation. 
\end{itemize}                       

\label{sec:modelcompare}
\begin{table}[]
	\fontsize{9.3pt}{11pt}\selectfont
    \begin{tabular}{@{}cllll@{}}
    \toprule
    \multirow{2}{*}{Model} & \multicolumn{2}{c}{Robust04}                           & \multicolumn{2}{c}{ClueWeb09-B}                        \\ \cmidrule(l){2-5} 
                           & \multicolumn{1}{c}{nDCG@20} & \multicolumn{1}{c}{P@20} & \multicolumn{1}{c}{nDCG@20} & \multicolumn{1}{c}{P@20} \\ \midrule
    QL                     & 0.415$^-$                   & 0.369$^-$                & 0.224$^-$                   & 0.328$^-$                \\
    BM25                   & 0.418$^-$                   & 0.370$^-$                & 0.225$^-$                   & 0.326$^-$                \\ \midrule
    MP                     & 0.318$^-$                   & 0.278$^-$                & 0.227$^-$                   & 0.262$^-$                \\
    DRMM                   & 0.406$^-$                   & 0.350$^-$                & 0.271$^-$                   & 0.324$^-$                \\
    KNRM                   & 0.415$^-$                   & 0.359$^-$                & 0.270$^-$                   & 0.330$^-$                \\
    PACRR                  & 0.415$^-$                   & 0.371$^-$                & 0.245$^-$                   & 0.278$^-$                \\
    Co-PACRR               & 0.426$^-$                   & 0.378$^-$                & 0.252$^-$                   & 0.289$^-$                \\ \midrule
    BERT-MaxP              & \textbf{0.469}                       & -                        & 0.293                       & -                        \\ \midrule
    GRMM                   & 0.449                        & \textbf{0.387}                    & \textbf{0.310}                       & \textbf{0.354}                    \\ \bottomrule
    \end{tabular}
	\caption{Performance comparison of different methods. The best performances on each dataset and metric are highlighted. Significant performance degradation with respect to GRMM is indicated (-) with p-value $\leq$ 0.05.}
	\label{tab:2}
\end{table}

\subsection{Study of Graph Structure (RQ2)}
\label{sec:graphstructure}
To dig in the effectiveness of the document-level word relationships of GRMM, we conduct further ablation experiments to study their impact. Specifically, we keep all settings fixed except substituting the adjacency matrix with: 
\begin{itemize}
	\item \textbf{Zero matrix}: Word nodes can only see themselves, and no neighbourhood information is aggregated. This alternative can be viewed as not using any contextual information. The model learns directly from the query-document term similarity.
	\item \textbf{Word sequence}, the original document format: No words are bound together, and they can see themselves as well as their previous and next ones. This alternative can be viewed as only using local contextual information. It does not consider long-distance dependencies. 
\end{itemize}


\begin{figure}[h]
	\centering
	\includegraphics[width=.47\textwidth]{./pics/graph_ablation.png}
	\caption{Ablation study on graph structure of GRMM.}
	\label{fig:3} 
\end{figure}

Figure \ref{fig:3} illustrates the comparison between the original GRMM and the alternatives. We can see that:
\begin{itemize}
    \item GRMM (zero matrix) performs inferior to others in all cases. Since it merely depends on the junior term similarities, the model becomes approximate to term-based matching. Without contextualised refinement, some words and their synonyms can be misleading, which makes it even hard to discriminate the actual matching signals. 
    \item GRMM (word sequence) promotes GRMM (zero matrix) by fusing local neighbourhood information but still underperforms the original GRMM by a margin of 2-3 points. This observation resembles some results in Table \ref{tab:2}. It shows that such text format could advantage local context understanding but is insufficient in more comprehensive relationships. 
    \item  From an overall view of the comparison, the document-level word relationships along the graph structure is proved effective for ad-hoc retrieval. Moreover, a relatively greater gain on ClueWeb09-B indicates that longer texts can benefit more from the document-level respective field.
\end{itemize}

\subsection{Study of Neighbourhood Aggregation (RQ2 \& RQ3)}
\label{sec:neighbouraggre}
Figure \ref{fig:4} summarises the experimental performance w.r.t a different number of graph layers. The idea is to investigate the effect of high-order neighbourhood aggregations. For convenience, we notate GRMM-0 for the model with no graph layer, GRMM-1 for the model with a single graph layer, and so forth for the others. From the figure, we find that:

\begin{figure}[h]
	\centering
	\includegraphics[width=.47\textwidth]{./pics/num_of_layers.png}
	\caption{Influence of different graph layer numbers.}
	\label{fig:4} 
\end{figure}

\begin{itemize}
	\item GRMM-1 dramatically boosts the performance against GRMM-0. This observation is consistent with Section \ref{sec:graphstructure} that propagating the information within the graph helps to understand both query-term interaction and document-level word relationships. The exact/similar query-document matching signals are likely to be strengthened or weakened according to intra-document word relationships. 
	\item GRMM-2 improves, not as much though, GRMM-1 by incorporating second-order neighbours. It suggests that the information from 2-hops away also contributes to the term relations. The nodes serving as a bridge can exchange the message from two ends in this way.
	\item However, when further stacking more layers, GRMM-3 and GRMM-4 suffer from slight performance degradation. The reason could be nodes receive more noises from high-order neighbours which burdens the training of parameters. Too much propagation may also lead to the issue of over-smooth \cite{kipf2017semi}. A two-layer propagation seems to be sufficient for capturing useful word relationships.
	\item Overall, there is a tremendous gap between using and not using the contextual information, and the model peaks at layer $t$ = 2 on both datasets. The tendency supports our hypothesis that it is essential to consider term-level interaction and document-level word relationships jointly for ad-hoc retrieval. 
\end{itemize}

\subsection{Study of Graph Readout (RQ3)}
\label{sec:featureelect}
\begin{figure}[h]
	\centering
	\includegraphics[width=.47\textwidth]{./pics/k.png}
	\caption{Influence of different $k$ values of $k$-max pooling.}
	\label{fig:5} 
\end{figure}

We also explored the effect of graph readout for each query term. Figure \ref{fig:5} summarises the experimental performance w.r.t different $k$ values of $k$-max-pooling. From the figure, we find that: 
\begin{itemize}
	\item The performance steadily grows from $k$ = 10 to $k$ = 40, which implies that a small feature dimension may limit the representation of terms. By enlarging the $k$ value, the relevant term with more matching signals can distinguish from the irrelevant one with less. 
	\item The trend, however, declines until $k$ = 70, which implies that a large feature dimension may bring negative influence. It can be explained that a large $k$ value may have a bias to the document length, where longer documents tend to have more matching signals. 
	\item Overall, there are no apparent sharp rises and falls in the figure, which tells that GRMM is not that sensitive to the selection of $k$ value. Notably, almost all performances (except $k$ = 70) exceed the baselines in Table \ref{tab:2}, suggesting that determinative matching signals are acquired during the graph-based interactions before feeding into the readout layer. 
\end{itemize}

\section{Conclusion}

In this paper, we propose a novel GNN-RNN framework to innovatively incorporate both geospatial and temporal knowledge into crop yield prediction, through graph-based deep learning methods. To our knowledge, our paper is the first to take advantage of the spatial structure in the data when making crop yield predictions, as opposed to previous approaches which assume that neighboring counties are independent samples. We conduct extensive experiments on large-scale datasets covering 41 US states and 39 years, and show that our approach substantially outperforms many existing state-of-the-art machine learning methods across multiple datasets. Thus, we demonstrate that incorporating knowledge about a county's geospatial neighborhood and recent  history can significantly enhance the prediction accuracy of deep learning methods for crop yield prediction. 

%, and demonstrate its superiority to existing models. large-scale crop yield dataset \textit{USCrop}, covering 48 states and 39 years. On \textit{USCrop}, we broadly compare the popular AI methods in crop yield prediction and provide solid baselines on three well-known metrics. Furthermore, we propose a novel GNN-RNN framework to innovatively incorporate both geospatial and temporal knowledge into prediction, through graph-based deep learning methods, and demonstrate its superiority to existing models. As far as we know, \textit{USCrop} is so far the most comprehensive crop yield dataset both spatially and temporally for machine learning. We will keep maintaining the dataset by including more features such as satellite images and progress data, and updating the dataset every year. We hope \textit{USCrop} can become a solid benchmark for crop yield study and an inspiration for novel machine learning techniques in spatiotemporal data modeling. We will also explore 
% more effective ways to better understand the influence of climatic variations. 

% In this paper, we present a large-scale crop yield dataset \textit{USCrop}, covering 48 states and 39 years. On \textit{USCrop}, we broadly compare the popular AI methods in crop yield prediction and provide solid baselines on three well-known metrics. Furthermore, we propose a novel GNN-RNN framework to innovatively incorporate both geospatial and temporal knowledge into prediction, through graph-based deep learning methods, and demonstrate its superiority to existing models. As far as we know, \textit{USCrop} is so far the most comprehensive crop yield dataset both spatially and temporally for machine learning. We will keep maintaining the dataset by including more features such as satellite images and progress data, and updating the dataset every year. We hope \textit{USCrop} can become a solid benchmark for crop yield study and an inspiration for novel machine learning techniques in spatiotemporal data modeling. We will also explore 
% more effective ways to better understand the influence of climatic variations. 

\section{Acknowledgements}

This research was supported by USDA Cooperative Agreement 58-6000-9-0041 and USDA NIFA Hatch Project 1017421. We would like to thank Rich Bernstein for constructive suggestions and Samuel Porter for help in processing the gSSURGO dataset.

%
Esse maiores eos, placeat porro fuga beatae id neque odit minus nulla a, dolores fugit esse temporibus nemo omnis iste autem incidunt deleniti veritatis consequatur.Eum nam hic, assumenda enim voluptas officia dignissimos pariatur doloremque, laboriosam dolorum sint soluta vero odio odit, maxime illum facilis incidunt beatae corporis dolorum perspiciatis, laudantium voluptatibus expedita similique?Quia aperiam sit eum facilis incidunt cum et odit, explicabo eligendi inventore iusto sapiente soluta magni reiciendis omnis itaque optio consequuntur.Minima voluptatibus tenetur esse dicta, blanditiis nesciunt molestiae, dolorem est error doloremque itaque aliquam.Ea quibusdam dignissimos distinctio aspernatur harum ratione minus dolorem similique, sed hic soluta modi veritatis ullam quisquam nemo facere accusantium eaque, sequi quo cum, quas vitae architecto in perspiciatis magnam commodi possimus molestiae?Nam aspernatur vitae sapiente culpa modi ut velit assumenda odit alias unde, laudantium possimus ipsa suscipit et dolore dolorem quae dignissimos pariatur officia, a
\bibliography{ref}


\end{document}