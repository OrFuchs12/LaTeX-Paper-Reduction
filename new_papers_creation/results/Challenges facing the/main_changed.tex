\documentclass[letterpaper]{article} % DO NOT CHANGE THIS
\usepackage{aaai23}  % DO NOT CHANGE THIS
\usepackage{times}  % DO NOT CHANGE THIS
\usepackage{helvet}  % DO NOT CHANGE THIS
\usepackage{courier}  % DO NOT CHANGE THIS
\usepackage[hyphens]{url}  % DO NOT CHANGE THIS
\usepackage{graphicx} % DO NOT CHANGE THIS
\urlstyle{rm} % DO NOT CHANGE THIS
\def\UrlFont{\rm}  % DO NOT CHANGE THIS
\usepackage{natbib}  % DO NOT CHANGE THIS AND DO NOT ADD ANY OPTIONS TO IT
\usepackage{caption} % DO NOT CHANGE THIS AND DO NOT ADD ANY OPTIONS TO IT
\frenchspacing  % DO NOT CHANGE THIS
\setlength{\pdfpagewidth}{8.5in} % DO NOT CHANGE THIS
\setlength{\pdfpageheight}{11in} % DO NOT CHANGE THIS
%
% These are recommended to typeset algorithms but not required. See the subsubsection on algorithms. Remove them if you don't have algorithms in your paper.
\usepackage{algorithm}
\usepackage{algorithmic}

%
% These are are recommended to typeset listings but not required. See the subsubsection on listing. Remove this block if you don't have listings in your paper.
\usepackage{newfloat}
\usepackage{listings}
\DeclareCaptionStyle{ruled}{labelfont=normalfont,labelsep=colon,strut=off} % DO NOT CHANGE THIS
\lstset{%
	basicstyle={\footnotesize\ttfamily},% footnotesize acceptable for monospace
	numbers=left,numberstyle=\footnotesize,xleftmargin=2em,% show line numbers, remove this entire line if you don't want the numbers.
	aboveskip=0pt,belowskip=0pt,%
	showstringspaces=false,tabsize=2,breaklines=true}
\floatstyle{ruled}
\newfloat{listing}{tb}{lst}{}
\floatname{listing}{Listing}


%
% Keep the \pdfinfo as shown here. There's no need
% for you to add the /Title and /Author tags.

% DISALLOWED PACKAGES
% \usepackage{authblk} -- This package is specifically forbidden
% \usepackage{balance} -- This package is specifically forbidden
% \usepackage{color (if used in text)
% \usepackage{CJK} -- This package is specifically forbidden
% \usepackage{float} -- This package is specifically forbidden
% \usepackage{flushend} -- This package is specifically forbidden
% \usepackage{fontenc} -- This package is specifically forbidden
% \usepackage{fullpage} -- This package is specifically forbidden
% \usepackage{geometry} -- This package is specifically forbidden
% \usepackage{grffile} -- This package is specifically forbidden
% \usepackage{hyperref} -- This package is specifically forbidden
% \usepackage{navigator} -- This package is specifically forbidden
% (or any other package that embeds links such as navigator or hyperref)
% \indentfirst} -- This package is specifically forbidden
% \layout} -- This package is specifically forbidden
% \multicol} -- This package is specifically forbidden
% \nameref} -- This package is specifically forbidden
% \usepackage{savetrees} -- This package is specifically forbidden
% \usepackage{setspace} -- This package is specifically forbidden
% \usepackage{stfloats} -- This package is specifically forbidden
% \usepackage{tabu} -- This package is specifically forbidden
% \usepackage{titlesec} -- This package is specifically forbidden
% \usepackage{tocbibind} -- This package is specifically forbidden
% \usepackage{ulem} -- This package is specifically forbidden
% \usepackage{wrapfig} -- This package is specifically forbidden
% DISALLOWED COMMANDS
% \nocopyright -- Your paper will not be published if you use this command
% \addtolength -- This command may not be used
% \balance -- This command may not be used
% \baselinestretch -- Your paper will not be published if you use this command
%  -- No page breaks of any kind may be used for the final version of your paper
% \columnsep -- This command may not be used
%  -- No page breaks of any kind may be used for the final version of your paper
% \pagebreak -- No page breaks of any kind may be used for the final version of your paperr
% \pagestyle -- This command may not be used
% \tiny -- This is not an acceptable font size.
% \vspace{- -- No negative value may be used in proximity of a caption, figure, table, section, subsection, subsubsection, or reference
% \vskip{- -- No negative value may be used to alter spacing above or below a caption, figure, table, section, subsection, subsubsection, or reference

\setcounter{secnumdepth}{0} %May be changed to 1 or 2 if section numbers are desired.

% The file aaai23.sty is the style file for AAAI Press
% proceedings, working notes, and technical reports.
%

% Title

% Your title must be in mixed case, not sentence case.
% That means all verbs (including short verbs like be, is, using,and go),
% nouns, adverbs, adjectives should be capitalized, including both words in hyphenated terms, while
% articles, conjunctions, and prepositions are lower case unless they
% directly follow a colon or long dash
\title{Challenges facing the explainability of age prediction models:

case study for two modalities}
\author{
       Mikoaj Spytek${}^1$\,\,\,\, Weronika Hryniewska-Guzik${}^1$\, \\ Jarosaw ygierewicz${}^2$\,\,\,\, Jacek Rogala${}^2$\, \\  Przemysaw Biecek${}^{1,2}$
}
\affiliations{
    %Afiliations
    ${}^1$  Warsaw University of Technology \\
    ${}^2$  University of Warsaw \\
}


% REMOVE THIS: bibentry
% This is only needed to show inline citations in the guidelines document. You should not need it and can safely delete it.
\usepackage{bibentry}
% END REMOVE bibentry

\providecommand{\keywords}[1]{\textbf{Keywords:} #1}


\begin{document}

\maketitle

\begin{abstract}
The prediction of age is a challenging task with various practical applications in high-impact fields like the healthcare domain or criminology.

Despite the growing number of models and their increasing performance, we still know little about how these models work.
Numerous examples of failures of AI systems show that performance alone is insufficient, thus, new methods are needed to explore and explain the reasons for the model's predictions.

In this paper, we investigate the use of Explainable Artificial Intelligence (XAI) for age prediction focusing on two specific modalities, EEG signal and lung X-rays. We share predictive models for age to facilitate further research on new techniques to explain models for these modalities.

\end{abstract}

\begin{keywords}
  Explainable Artificial Intelligence, Age prediction, EEG, X-rays
\end{keywords}

\section{Introduction}

Models for age prediction have many interesting applications. Age can be a direct target of a model, for example in age prediction for forensics to identify the age of the offender, as well as indirect, for example, as an input feature. The list of examples is extensive, just to name a few: age prediction is useful feature in cancer diagnostics \cite{cancer21}, analyses of brain ageing lead to relevant features for predicting neurological disorders, such as schizophrenia or bipolar disorder \cite{BrainAGE17} or cognitive impairment \cite{NeuroImage17}.

Age prediction can be performed on modalities that differ in terms of structure and format. Just to list some of these modalities: 2d images of a person's face (features such as wrinkles, crow's feet, and age spots can be used to predict a person's age \cite{CVPRWage15}), 1d voice recordings (features such as pitch, tone, and accent can be used to predict a person's age), series of nucleotides such as methylation data \cite{methylclock20}, multiple series as in EEG \cite{EEGage19, ColeBrain17} or single-channel images X-rays \cite{Thodberg2017}.

This variety of modalities creates a major challenge when trying to explain predictive models. It is because the most popular methods of explainable AI are agnostic to the structure of the model but they are not agnostic to the modality of the data \cite{Holzinger2022, ema21}. If they are designed for tabular, image, or text data, it is difficult to transfer them to EEG signals or grayscale images.

To address this problem, we created and shared a collection of benchmark models for two modalities that currently do not have representations suitable for explanations.


\section{Materials and methods}

A series of models has been trained for the age prediction problem and are available on GitHub repository \url{https://github.com/pbiecek/challenges-xai-aging-aaai23}.

The models vary in structure, ranging from linear regression models, which are  interpretable by design, to tree based boosted models and multi-layer neural networks. The following sections summarise the data on which these models were trained.

\subsection{Age prediction with EEG data}

The first use case concerns predicting the metrical age of patients based on EEG data. The dataset containing 20,365 observations of patients aging from 16 to 70 has been gathered over the past 15 years across 32 hospitals and medical institutes in Poland. All recordings contain measurements from 19 electrodes placed according to the international 10-20 system \cite{system10-20} with sampling rates of 250, 256, or 500 Hz, depending on the used hardware. For the purposes of this analysis, only the data from healthy patients was used.

The raw recording data has been preprocessed to extract 1,653 features for each observation. The first 1,368 features represent each of 8 frequency bands multiplied by the 171 values from the lower triangle of the coherence matrix. The remaining features correspond to the normalized band power from 15 frequency bands for each of the 19 channels.

Using the extracted features two machine learning models were fitted, with the task of predicting the age of the patient -- a Linear Regression model and a Multi-Layer Perceptron. Predictions on a holdout test set of 5,091 observations are presented in Figure~\ref{fig:eeg-models}. The linear regression achieves a Mean Absolute Error of 7.93 years, whereas the MLP network scores 7.23 in this metric. %Interestingly, for both models the predictions flatten out for older patients. This can mean that starting from a certain age, the recorded brain activity is a worse predictor of the patient's age or that the differences in brain activity of older people are less pronounced.



\subsection{Age prediction with x-ray lung data}

The second use case is based on a new B2000 dataset with X-rays images. The data contains cases from two hospitals in Poland it is composed of adults and children X-rays. Each image is annotated with labels from 6 different classes: normal, pneumothorax, airspace opacification, fluid, cardiomegaly, and mass/nodule. As mentioned in~\cite{checklist}, images of children and adults should not be mixed in one class in the standard classification task. However, embeddings are representations of images, so we decided to use one of the few-shot learning methods, namely siamese networks. Few-shot learning helps to deal with a problem when a few examples in each class are present.

We trained the Siamese network with 2,158 images for 20 epochs and a learning rate 0.0005. The backbone was EfficientNet-B0 of input size 224x224. The baseline vector (anchor vector) is compared with a positive and negative one, and the resulting values are passed to the triplet loss. The generated embeddings were vectors of length 512.

The XGBoost regressor with squared loss and the CatBoost regressor were trained on embeddings retrieved from the model of the lowest validation loss to predict the age of a patient. After training, we clipped the age values below zero to zero. Obtained results are presented in Figure \ref{fig:xray-models}.

\section{Conclusions}

To facilitate the development of XAI techniques for regression models that are needed to explain age prediction models in the healthcare domain, it is crucial to share with the XAI community specific cases on which such techniques can be benchmarked and validated.

In this work, we have developed and shared age prediction models operating on two modalities for which there are currently no good explanatory techniques, that is, multidimensional series in the EEG signals and grayscale X-ray images.

We hope that the availability of these models and validation samples will help in the development of new explanatory techniques for age prediction.


\begin{figure*}
\centering
    \includegraphics[width=0.35\textwidth]{images/linreg_new.png}
    \includegraphics[width=0.35\textwidth]{images/mlp_new.png}
\caption{
Diagnostic plots for developed age prediction models for EEG data. A) predictions from a Linear Regression model, B) predictions from a Multi-Layer Perceptron}
\label{fig:eeg-models}
%\end{figure*}

%removedVspace

%\begin{figure*}[h]
\centering
    \includegraphics[width=0.35\textwidth]{images/xrayage.png}
    \includegraphics[width=0.35\textwidth]{images/xrayage2.png}
\caption{
Diagnostic plots for developed age prediction models for X-ray data.
A) predictions from XGBoost Regression model, B) predictions from Cat Boost Regression model}
\label{fig:xray-models}
\end{figure*}

%\begin{figure*}[h]
%\centering
 %   \includegraphics[width=0.35\textwidth]{images/eeg-tsne.png}
  %  \includegraphics[width=0.35\textwidth]{images/xray-tsne.png}
%\caption{
%Graphs showing the results of dimensionality results using TSNE A) on EEG data, B) on embeddings of X-ray images}
%\label{fig:tsne}
%\end{figure*}

%\section{Explanation of age prediction models}

%To explain the EEG model, we can divide the input data into subgroups based on frequency bands. However, even if the explanation method will show which frequency bands influences the model prediction, it will not be easily interpretable for practitioners. The features by which a person diagnoses are very different from the features that the model can learn.

%In case of Siamese networks trained on X-rays, very often visualizations are treated as explanations. It is a consequence of the fact that there is a difficulty in interpreting multidimensional vector values. Although explanation methods can be used to show that the n-th value in the embedding contributes the most to the prediction, it will not be interpretable for a human being.

%Currently, the only easily-interpretable method of explanation is to show whether patients in the same age groups are separable. In Fig. \ref{fig:tsne}, there are results of TSNE visualization for both use-cases. The visualization for EEG is obtained from model input data. For X-rays, Fig. \ref{fig:tsne} shows the 2D representation of embedding generated by siamese network. Thus, presented data are embeddings obtained from trained deep learning model for disease prediction, before being delivered to the machine learning model for age regression. Therefore, siamese network not only learned to differentiate well the classes on which it was trained, i.e., the patient's diseases, but also the age.

%Unfortunately, there are currently no explanatory methods available to see how models trained on both use-cases behave. Various attempts can be made to explain each use-case separately, but they will not provide human-interpretable results.


% Acknowledgements should go at the end, before appendices and references

%\acks{We would like to acknowledge support for this project
%from the National Science Foundation (NSF grant IIS-9988642)
%and the Multidisciplinary Research Program of the Department
%of Defense (MURI N00014-00-1-0637). }

% Manual newpage inserted to improve layout of sample file - not
% needed in general before appendices/bibliography.


\vskip 0.2in
Nesciunt nam minus velit sit sint ipsam officiis perspiciatis consectetur delectus, temporibus ipsam voluptates sapiente corporis sunt quasi libero error sed quos?Harum aspernatur magnam eligendi excepturi assumenda tenetur dolor, nisi inventore sunt suscipit beatae repellendus enim quos animi necessitatibus distinctio, ut possimus cum, dolorum ratione minus.Dolor doloremque ullam ipsam maxime repellendus beatae amet illo nemo, modi molestias placeat quo incidunt minima necessitatibus, commodi quas officia eveniet necessitatibus ducimus officiis voluptas in deserunt hic qui?Suscipit veritatis consequuntur corporis deleniti soluta a impedit quidem, laborum adipisci culpa nihil at qui, eum blanditiis quasi fuga architecto consectetur a at corrupti placeat officia?Corporis quas maiores labore aliquam nobis, animi explicabo itaque ipsum ad quisquam, architecto autem beatae facere?Suscipit quam cumque quis culpa recusandae, fuga nulla illum dolorum aspernatur obcaecati quis alias eum laborum et?Optio adipisci debitis odit, eius nemo similique sunt deserunt quam eos enim ratione commodi, praesentium error sed est dicta quis impedit ducimus non?Neque officiis minus vero, autem deleniti voluptate quos laudantium vero maiores velit, inventore qui facilis earum enim rerum animi tenetur.Voluptatum expedita recusandae ab rerum quam ea ullam quo quidem, perferendis placeat libero, aspernatur assumenda dicta sit, voluptates beatae quas sit reiciendis eum impedit fuga.Voluptatibus nulla officia voluptatem ullam sunt rerum quas quis dicta aspernatur, debitis a iste cupiditate, ad quae numquam minima consectetur illum placeat laudantium hic?Cumque at dignissimos excepturi tenetur alias suscipit pariatur minima earum numquam, qui sed magnam officiis fugiat, laborum molestiae architecto sit saepe.Esse qui minus iure placeat harum tempora quia a explicabo, necessitatibus fuga tenetur recusandae temporibus ex cum facere enim debitis hic doloribus, reiciendis architecto similique?Cumque aspernatur ab itaque sapiente a commodi quibusdam doloribus, natus facilis delectus rem ex possimus velit omnis, eligendi aut fuga commodi rerum, odio laboriosam autem?Sapiente iusto quia ab cum beatae tempore atque adipisci ut minus blanditiis, quibusdam eaque fuga quam exercitationem sunt similique aut quis quas voluptate, recusandae consequatur obcaecati fuga?Aut officia commodi pariatur excepturi impedit temporibus eveniet odio iusto porro, dignissimos quae dolore ipsum necessitatibus animi porro eius magni corrupti obcaecati exercitationem, architecto corporis eum in sed laboriosam ipsa ea quos nisi?Obcaecati saepe mollitia fugiat repellat rem sequi quisquam magnam, ullam corporis laborum numquam tempore recusandae, fuga nemo vel illo ipsa in nostrum enim quis sed sequi.Fuga dolor laudantium dolores perspiciatis suscipit, ab nulla dignissimos, voluptatum repudiandae nihil dolor optio accusamus molestiae blanditiis iste non nam quaerat, vitae veniam harum quo vel sed optio ea quasi fugit.Non debitis porro eligendi sunt in, aliquam mollitia qui voluptatem repellendus velit amet voluptatum quas nihil excepturi, quis doloribus quae animi nulla voluptate obcaecati?Odio quas minus vero et at aut enim accusamus quia distinctio, necessitatibus nisi quod, eveniet placeat vel suscipit nam recusandae doloribus quos voluptatibus assumenda adipisci, et neque cumque nemo dicta velit culpa id quo quod, ratione explicabo facilis ab vel mollitia esse natus eligendi tempora.Adipisci minima sit, esse ipsum mollitia quia itaque ut odit ipsam sint, magni aliquam possimus, obcaecati ratione eum sint iusto officia quia iure, doloremque ipsam atque voluptatem aperiam itaque fuga nihil repellat quaerat nemo et?Quia esse optio vel aspernatur fuga, cumque consectetur officia maiores eligendi.Placeat eaque qui sapiente velit repellat iusto labore non voluptatum, distinctio nemo dolorem deleniti nulla ipsam animi non, soluta repellendus magni recusandae, eligendi magni corporis voluptatum provident eius quibusdam debitis?Voluptatum sit iusto impedit deserunt culpa voluptatem, voluptates recusandae sit labore ipsam facilis aspernatur, nisi asperiores quibusdam ea voluptatum quae obcaecati?Veritatis voluptas quia ratione, corporis impedit doloribus perferendis ipsam corrupti labore, et eos qui quia repellendus?Hic amet recusandae veniam eum facilis vel reiciendis, aut distinctio doloribus molestias cupiditate possimus mollitia dolores quae quos, quod reprehenderit quis pariatur facere repellat.Nulla doloribus labore excepturi animi dolorum inventore eum soluta fugiat, alias sed iusto vitae perferendis accusantium iure sunt quia, optio ducimus praesentium non fugit velit deserunt, culpa voluptatibus ipsum atque odit rem nesciunt?Expedita nam omnis sequi vitae dolorem, tempora facere tenetur maiores quae cum dignissimos laudantium,
\bibliography{sample}

\end{document}