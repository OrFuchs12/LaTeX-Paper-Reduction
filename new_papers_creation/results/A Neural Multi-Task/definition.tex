This section gives formal definitions of the two tasks to be investigated: MER and MEN.

\subsection{Medical Named Entity Recognition} 
The medical named entity recognition (MER) task is to find the boundaries of mentions from medical text. It differs from general NER in several ways. A large number of synonyms and alternate spellings of an entity cause explosion of word vocabulary sizes and reduce the efficiency of dictionary of medicine. Entities often consist of long sequences of tokens, making harder to detect boundaries exactly. It is very common to refer to entities also by abbreviations, sometimes non-standard and defined inside the text. Polysemy or ambiguity is pronounced: proteins (normally
class GENE) are also chemical components and depending on the context occasionally should be classified as class CHEMICAL; tokens that are sometimes of class SPECIES can be part of a longer entity of class DISEASE referring to the disease caused by the organism or the specialization of disease on the patient species.
In this work, we follow the setup of the shared subtasks of BioCreative \cite{Wei2015Overview}. Given a sentence $s$, i.e., a word sequence $w_1, ..., w_n$, each word is annotated with a predicated MER tag (e.g., ``B-DISEASE"). Therefore, we consider MER as a sequence-labeling task. 

\subsection{Medical Named Entity Normalization}
Medical named entity normalization (MEN) is to map obtained medical named entities into a controlled vocabulary. It
 is usually considered as a follow-up task of MER because MEN is usually conducted on the output of MER. In other words, MER and MEN are usually considered as hierarchical tasks in previous studies. In this paper, we consider MEN and MER as parallel tasks. MEN takes the same input with MER and have different output, i.e., for each word sequence $w_1, ..., w_n$, MEN outputs a sequence of tags from a different tag set. Therefore, we also consider MEN as a sequence-labeling task with the same input with MER. 
 
MER and MEN are not independent. MER and MEN are essentially hierarchical tasks but their outputs potentially have mutual enhancement effects for each other. Specifically, the output of MER, such as ``B-DISEASE", is a clear signal indicating the beginning of a disease entity, making MEN to map the code of a disease. Conversely,  the output of MEN, such as ``D054549-P" which is a disease code tag, is very helpful to recognize it as a part of a disease named entity.  