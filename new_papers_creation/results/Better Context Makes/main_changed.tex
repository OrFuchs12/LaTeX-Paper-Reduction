\documentclass[letterpaper]{article} %
\usepackage{aaai23}  %
\usepackage{times}  %
\usepackage{helvet}  %
\usepackage{courier}  %
\usepackage[hyphens]{url}  %
\usepackage{graphicx} %
\urlstyle{rm} %
\def\UrlFont{\rm}  %
\usepackage{natbib}  %
\usepackage{caption} %
\frenchspacing  %
\setlength{\pdfpagewidth}{8.5in}  %
\setlength{\pdfpageheight}{11in}  %
\usepackage{algorithm}
\usepackage{algorithmic}

\usepackage{newfloat}
\usepackage{listings}
\DeclareCaptionStyle{ruled}{labelfont=normalfont,labelsep=colon,strut=off} %
\lstset{%
	basicstyle={\footnotesize\ttfamily},%
	numbers=left,numberstyle=\footnotesize,xleftmargin=2em,%
	aboveskip=0pt,belowskip=0pt,%
	showstringspaces=false,tabsize=2,breaklines=true}
\floatstyle{ruled}
\newfloat{listing}{tb}{lst}{}
\floatname{listing}{Listing}


\setcounter{secnumdepth}{2} %


\input{math_commands.tex}

\newif\ifaaai
\newif\ifamlc
\aaaitrue

\usepackage{comment}
\usepackage{multirow}
\usepackage{booktabs}
\usepackage{graphicx}
\usepackage{adjustbox}
\usepackage{url}
\newcommand{\todo}[1]{\textcolor{red}{[todo] {#1}}}
\usepackage{ulem} %
\usepackage[inline]{enumitem} %

\usepackage{xcolor}
\usepackage{listings}

\usepackage{xparse}

\lstset{language=Python,basicstyle=\ttfamily\small,keywordstyle={},morekeywords={assert}}
\newcommand{\codeinline}[1]{\lstinline{#1}}
\NewDocumentCommand{\codeword}{v}{%
\texttt{\textcolor{blue}{#1}}%
}

\renewcommand{\emph}[1]{\textit{#1}}

\def\ContinueLineNumber{\lstset{firstnumber=last}}

\usepackage{xspace} %
\def\CallArgs{\textsc{CallArgs}\xspace}
\def\PyEnvs{\textsc{PyEnvs}\xspace}

% \newlabel{app:dataset_details}{{A}{9}}
\newlabel{sec:query}{{A.1}{9}}
\newlabel{sec:criteria}{{A.2}{9}}
\newlabel{sec:isolation}{{A.3}{10}}
\newlabel{sec:saturate}{{B.3}{11}}
 % comment out when including Appendix


\title{Better Context Makes Better Code Language Models: \\
A Case Study on Function Call Argument Completion}
\author{
    Hengzhi Pei\textsuperscript{\rm 1}\thanks{Work done while interning at Amazon Web Services.},
    Jinman Zhao\textsuperscript{\rm 2},
    Leonard Lausen\textsuperscript{\rm 2},
    Sheng Zha\textsuperscript{\rm 2},
    George Karypis\textsuperscript{\rm 2}
}
\affiliations{
    \textsuperscript{\rm 1} University of Illinois Urbana-Champaign\\
    \textsuperscript{\rm 2} Amazon Web Services\\


    hpei4@illinois.edu,
    \{jinmaz,lausen,zhasheng,gkarypis\}@amazon.com
}


\begin{document}

\maketitle

\begin{abstract}
Pretrained code language models have enabled great progress towards program synthesis. However, common approaches only consider in-file local context and thus miss information and constraints imposed by other parts of the codebase and its external dependencies. Existing code completion benchmarks also lack such context. To resolve these restrictions we curate a new dataset of permissively licensed Python packages that includes full projects and their dependencies and provide tools to extract non-local information with the help of program analyzers. We then focus on the task of function call argument completion which requires predicting the arguments to function calls. We show that existing code completion models do not yield good results on our completion task. To better solve this task, we query a program analyzer for information relevant to a given function call, and consider ways to provide the analyzer results to different code completion models during inference and training. Our experiments show that providing access to the function implementation and function usages greatly improves the argument completion performance. Our ablation study provides further insights on how different types of information available from the program analyzer and different ways of incorporating the information affect the model performance.


\end{abstract}

\section{Introduction}
Justice et al. \cite{justiceguide} state in their book that ``Children develop their knowledge of the world around them as they interact with their environment directly and indirectly. The direct experiences children have in their homes, schools and communities certainly provide the greatest amount of input to the world knowledge base.''. This knowledge arises from both physical and conversational interactions. In this paper, we test the hypothesis that just like a human child, machines need interaction to acquire world knowledge and develop commonsense reasoning abilities, and we study the effect of conversational interactions on this knowledge acquisition. Most of the literature on commonsense reasoning 
relies %rely [kmm- most-> relies]
on extracting the largest possible snapshot of 
%the [kmm- removed]
world knowledge and either 
query %query [kmm- on-> extracting and querying]
it or 
propose %propose [kmm- most-> proposes][could also parse as 'relies on-> proposing' or 'querying or proposing', may be better to restructure the sentence][fa- it was the later, so i restructured]
automated knowledge base completion methods for it. We argue that it is necessary to equip reasoning engines with an interaction strategy facilitating the extraction of just-in-time information needed for reasoning. 
%, through conversation with a human user [kmm- removed; conversation is covered by 'interaction' earlier in the sentence]
In this paper, we 
take up %take a few steps towards [kmm- rephrase (take steps/take steps repetitive)]
this grand goal, %[kmm- comma added]
and although we do not solve the whole challenge, we take the first steps needed for addressing it. 
Specifically, here we propose a ``soft'' commonsense reasoning engine and solve targeted knowledge base completion problems based on the information provided by the user through a conversational interface.

% We state this as our overarching grand research goal and mention carefully that we are taking a few steps towards this grand goal. Although it does not solve all of it but it is a step towards achieving this goal. This is just a first step however its a part of a very well reasoned and ambitious project. Then we also carefully describe the limitations of the project
% In other words, our overarching goal is having a human construct a reasoning system that does not have commonsense and extract commonsense from the user through conversation.
% \amoscomment{I think that it might be better saying something like: this work takes the first step towards ... I think that the paper could also benefit from adding a few sentences at the beginning.} \facomment{Is this resolved now?}

We believe that this is the right time for this proposal specifically since conversational agents such as Siri, Google home, Alexa and Cortana among others are starting to enter our daily lives. Therefore, it is plausible to assume that 
such agents %we [kmm- rephrase]
have access to conversation with a human for extracting commonsense knowledge. In this paper, we work with the Learning by Instruction Agent (LIA) \citep{azaria2016instructable,labutov2018lia} and develop a commonsense reasoning system for her called CORGI (\textbf{CO}mmonsense \textbf{R}easonin\textbf{G} by \textbf{I}nstruction). In what follows, we present our definition of commonsense reasoning for LIA after briefly introducing her. % It is worth noting, however, that the proposed method is not limited to a specific conversational agent. 
% \kmcomment{Anthropomorphizing LIA (referring to the agent as 'her') is a somewhat political choice -- it's okay to make it, but make it consciously.}

LIA is an intelligent agent that operates on 
a user's smartphone. %the phone [kmm- rephrase (you do not call LIA; there are other agents where you call in so it's important to make the distinction)]
%and can be taught new commands through user instructions. [kmm- removed (covered in the very next sentence)]
End users add new functionalities to LIA through verbal instructions and teach her how to perform new tasks. For example, the user can tell LIA, ``whenever it snows at night, wake me up 30 minutes early''. If LIA does not understand how to perform this task, she will ask the user to instruct her by breaking the task down into a set of steps in a teaching session. In this case, the user can say, ``(first) open the weather app, (second) see if the night weather condition is snow, (third) if true then adjust my alarm to 30 minutes earlier''. After this teaching session, LIA can perform this task. 

One phenomenon we have noticed in collecting these types of ``Whenever $S$ occurs, then do $A$'' instructions is that people often {\em underspecify} the precondition $S$. For example, one instructor might want to wake up early when it snows because they are concerned about getting to work on time.  For this user, the implied precondition is not really ``whenever it snows,'' but instead ``whenever it snows enough to cause traffic slowdowns, and it's a workday.'' The point is %Amos: I think that "the point is" doesn't sound good. How about "Naturally,"?
that people often fail to specify  such detailed conditions, perhaps because they are used to speaking to other people who possess the common sense needed to infer the more specific intent of the speaker.

Our goal for LIA is to use background commonsense knowledge to reason about the user's more specific intent, and to discuss this with the user in order to create the correct preconditions for the recommended action.  Therefore, we assume LIA can obtain statements from the user that fit the logical template ``Whenever $S$ occurs, do $A$ because I want to achieve goal $G$.''\footnote{Note in LIA's conversational setting, if the user gives an instruction of the form ``Whenever $S$ occurs, do $A$.'' and omits the reason, then LIA can simply respond ``Why do you want to do that?'' in order to prompt for the missing reason $G$.}
%LIA then generalizes from this statement to other actions. For example, if the user says, ``if the weather is rainy tomorrow then set an alarm for 1 hour later'', LIA can perform this action without needing to be taught again. However, this generalization has some limitations which 
%stem %stems [kmm- limitations->stem]
%from the lack of reasoning capabilities in LIA. 
For example consider the following two statements: %, [kmm- colon replaces comma]
\begin{itemize}
\item Whenever it snows at night, wake me up 30 minutes early because I don't want to be late to work
\item Whenever it snows at night, wake me up 30 minutes early because I have never seen the snow before 
\end{itemize}
Note that in the first statement, the user will not want to wake up early on a weekend or a holiday (assuming that they do not work then) whereas in the second scenario, the user will want to wake up early regardless of the date in order to see snow for the first time -- but might not want to wake up early once she has seen snow for the first time.

In CORGI, the role of commonsense reasoning is to derive the intended condition to use in place of the stated $S$ given an ``If $S$ then do $A$ because $G$'' statement from the user. Its general approach is to derive an explanation of how action $A$, performed in state $S$ will achieve goal $G$, and then to derive the intended precondition $S$ by collecting the preconditions on $S$ that allow this explanation to hold.  CORGI has access to a limited amount of general background knowledge about the world, represented in a logic programming language. Reasoning reduces to using this background knowledge to perform multi-hop logical inference. If no reasoning path is found, CORGI initiates a conversation with the user to extract relevant background knowledge and adds it to its underlying understanding of the world.  This newly acquired background knowledge will be used in future user interactions with CORGI. In essence, we are performing knowledge base completion through conversation, on a need-driven basis. Note that in earlier work Hixon et al. \cite{hixon2015learning} perform relation extraction using human interaction for question answering. Although the general idea of using human interaction is similar to our proposal, the information extraction method and the problem studied in \cite{hixon2015learning} differs from our setting. To the best of our knowledge, CORGI is the first conversational assistant that targets completing reasoning paths.
% \amoscomment{'their' seems like a typo, not sure what you are saying} --> resolved
% Therefore, our reasoning system is a commonsense reasoning by instruction engine. 

% \amoscomment{I find it hard to understand when 'LIA' refers to the agent from previous work, and when it refers to new capabilities added by this work.} \facomment{is this resolved now, Amos?} %Yes, Thanks!

% In this paper we develop a reasoning system for LIA that is capable of commonsense reasoning in order to generalize correctly given if-then user commands through the because statement.

CORGI's main reasoning component is the multi-hop inference system. Since the knowledge is represented in a logic programming language, the underlying inference algorithm is backward chaining. However, backward chaining in its traditional form is not robust to variations in natural language. This is specifically of importance since CORGI allows open-domain dialog with the user
to reduce the startup cost of the user having to learn a %so that the user is not limited to a [kmm- is this rephrase correct?]
specific grammar or vocabulary. Therefore, there is no parsing algorithm to resolve these variations. For example, in 
%the [kmm- removed]
traditional backward chaining, the statements ``if the forecast is snow tonight'' and ``if the weather is snowy tonight'' are thought of as two different statements whereas we want them both to map to the same representation. In order to address this, we propose a ``soft backward chaining'' algorithm that learns continuous representations or embeddings of the logical statements in the background knowledge. This will allow CORGI to indicate the equivalence of semantically similar statements based on the distance of their learned representations in the vector space. This soft backward chaining allows us to bridge a gap between symbolic AI and neural approaches using the best of both worlds.

% CORGI's soft backward chaining algorithm is end-to-end differentiable and is trained by looking at the proof traces of similar 

% kmm: resolve AA's confusion here with "compatible with deep-learning techniques"

% . This multi-hop reasoning system is end-to-end differentiable and supports soft multi-hop reasoning to account for natural language variations. \amoscomment{I might be missing something, but what does it mean being end-to-end differentiable, are you referring to differentiable functions (those that have a derivative), is this required in order to train the system? Or do you mean that the system obtains knowledge piece by piece. I guess you mean the former, but I did struggle with this.}

% \tmcomment{There are two main themes: 1. claiming that the reasoning can help get the generalization right, 2. how to do the reasoning in a way that is correct}

% \tmcomment{why are we doing reasoning this way and how can we make sure we can do it successfully. we need to compare it with the approximate inference and probabilistic inference methods for performing reasoning}

% \tmcomment{Our contributions are two fold. one is that we are proposing a reasoning strategy through conversation and are proposing to extract the missing information just in time to perform the correct reasoning. No one has the capacity to store the world's largest kb and until now everyone has tries to maintain the largest knowledge bases that there are. However, we are proposing a new way of doing this and it is to extract the correct part of the missing knowledge from the user. This is our grand goal and we have performed a set of small steps towards it... [layout the steps]. Another contribution is the soft unification part. In order to make this work we need to combine symbolic AI with neural approaches to bridge the gap and use the best of both worlds.}

% \tmcomment{reviewer question: How do we know if our method scales? No one has a large enough knowledge base that contains all the information there is in the world. And currently everyone in the field is trying to do this. However, we are proposing a method for extracting the right information just in time needed to perform the reasoning}

% \tmcomment{We do not know the user will give us the right answer even if we ask the right question} \kmcomment{Focus less on ``right'' answer/question here; there are many-to-many possible question/answer pairs that will give a good result. Make a definition of what success means in this context.}

% \tmcomment{Our goal is to have a conversation with the user and the main goal is to have the user give us the missing part of the information and in a funny/not so funny way this is a feature of the system}

% \tmcomment{consider the problem of learning procedures including triggers by conversation. When humans give instructions they are imprecise. In this project we are interested in having the human construct a reasoning system that does not have the commonsense and we want to use conversation to extract the commonsense from the user. We state this as our overarching grand research goal and mention carefully that we are taking a few steps towards this grand goal. Although it does not solve all of it but it is a step towards achieving this goal. This is just a first step however its a part of a very well reasoned and ambitious project. Then we also carefully describe the limitations of the project.}


\begin{figure}[t]
    \centering
    \subfloat[relationship among scenes]{\resizebox{0.22\textwidth}{!}{
        \includegraphics[]{figure/chord.png}
    }}
    \hspace{4mm}
    \subfloat[car instance $ln$ distribution]{\resizebox{0.22\textwidth}{!}{
        \includegraphics[]{figure/polar.png}
    }}
    \caption{The images in our RGB-P Car dataset vary in terms of (a) scenarios and (b) the number of car instances.}
    \label{fig:dataset}
\end{figure}

\begin{table}[tp]
\caption{Comparison of existing car detection datasets with polarization measurements.}
\small
\centering
\setlength{\tabcolsep}{2.6pt}
\begin{tabular}{c|c|c|c|c}
\hline\hline
Datasets         & Pol. & \begin{tabular}[c]{@{}c@{}}Pixel\\ align\end{tabular} & \begin{tabular}[c]{@{}c@{}}Num.images \\ Train / Test\end{tabular} & \begin{tabular}[c]{@{}c@{}}Num. cars \\ Train / Test \end{tabular} \\ 
\hline
PolarLITIS       & Mono & $\times$                                                     & \begin{tabular}[c]{@{}c@{}}2569 \\ 1640 / 929 \end{tabular}         & \begin{tabular}[c]{@{}c@{}}17428\\ 6061 / 11367 \end{tabular}    \\
\hline
\textbf{RGBP-Car (Ours)} & Tri  & \checkmark                                                     & \begin{tabular}[c]{@{}c@{}}2601 \\ 1611 / 990 \end{tabular}         & \begin{tabular}[c]{@{}c@{}}31234 \\ 19582 / 11652 \end{tabular}   \\ 
\hline\hline
\end{tabular}
\label{tab:datasetcomp}
\end{table}


\section{RGB-P Car Detection Dataset}
\label{sec:dataset}
We construct the first pixel-aligned RGB-polarization car detection dataset called RGBP-Car with trichromatic polarization measurements. We record cars in diverse traffic scenes using FLIR-Blackfly-S, a polarized color camera that simultaneously obtain pixel-aligned polarization measurements in four linear polarization directions (0$^\circ$, 45$^\circ$, 90$^\circ$, and 135$^\circ$) for each color channel (\textit{i.e.}, R, G, and B). RGBP-Car contains 2601 RGB, AoLP, and DoLP image triplets. Each image has manually labeled bounding boxes indicating the position and size of each car. To ensure the diversity and challenge of our dataset, we take the RGB-P images under different weather conditions (clear and rainy), different lighting conditions (daytime and nighttime), different driving environments (indoor, outdoor, road and parking lot), and different car densities. 
Fig. \ref{fig:samples} gives representative examples and Fig. \ref{fig:dataset} analyzes (a) the relationship among different scenes and (b) the density distribution of car instances. Tab. \ref{tab:datasetcomp} further shows the superiority of our RGBP-Car over existing car detection datasets with polarization measurements.



\section{Commonsense for Zero-Shot NLVL}
\label{sec:proposedSection}

\subsection{Problem Formulation}
We denote an input video as $V$, and its grounding annotations as \(\left( Q,V_{\text{span}}\right) \), where $Q$ is the query representation and \(V_{\text{span}}\!=\!\left( t_{s},t_{e}\right)\) is the corresponding video moment span annotation, with \(t_{s}\) and \(t_{e}\) representing the start and end timestamps, respectively. Learning to localize a video moment conditioned on a query entails maximizing the expected log-likelihood of the model parameterized by \(\theta\). In its typical setting, this can be formulated as follows:
\begin{equation}
\label{eq:groundingOriginal}
    \theta ^{\ast }=\arg \max _{\theta } \mathbb{E}\left[ \log p_{\theta }\left(  V_{\text{span}} | V,Q\right) \right]. 
\end{equation}
In the zero-shot setting, the goal is to learn this task without parallel video-query annotations. Hence, the query and video moment annotations are derived from $V$, using a dynamic video moment proposal method followed by a pseudo-query generation mechanism. Formally,  \(V_{\text{span}}\,\!{=}\!\,f_{\text{span}}(V)\) and \(Q\,\!{=}\!\,f_{pq}(V_{\text{span}})\), where $f_{\text{span}}$ and $f_{\text{pq}}$ are video moment proposal and pseudo-query generation mechanisms, respectively. Given that $f_{\text{span}}$ and $f_{\text{pq}}$ are responsible for generating the annotations, the performance of the localization model heavily depends on the quality of these modules. Existing methods face challenges in aligning \(Q\) to \(V_{\text{span}}\) due to noise introduced by ungrounded pseudo-query generation mechanisms. 
To address this, we simplify \(f_{\text{pq}}\) while augmenting cross-modal understanding by leveraging external information in the form of a commonsense graph \(G_{C}(C, E)\) with \(n_c\) nodes, where \(C\!=\!\left\{c_{1}, c_{2}, \dots, c_{n_{C}}\right\}\) are the concept node vector representations and \(E\) is the set of weighted directed edges, respectively. Accordingly, learning can be formulated as
\begin{equation}
\label{eq:groundingOurs}
    \theta ^{\ast }=\arg \max _{\theta } \mathbb{E}\left[ \log p_{\theta }\left(  V_{\text{span}}| V,Q,G_{C}\right) \right].
\end{equation}

\noindent Figure \ref{fig:approach} shows both training and inference flows.
\subsection{Pseudo-supervised Setup}
\modelname first processes a raw video with a video moment proposal $f_{\text{span}}$ module that extracts important video segments capturing key events, and a pseudo-query generation $f_{\text{pq}}$ that generates text query annotations corresponding to the extracted video segments.

\paragraph{Dynamic Video Moment Proposal ($f_{\text{span}}$).}
We adopt the dynamic video moment proposal approach proposed by \citet{nam_zero-shot_2021}. Specifically, $f_{\text{span}}$ primarily comprises a k-means clustering mechanism that groups semantically similar and temporally proximal video frame features together to extract atomic moments. To obtain frame features, we consider the columns of a frame-wise similarity matrix derived from the CNN features of individual frames. We enforce temporal proximity by concatenating the frame index to the features. Composite video moments are then formed by combining neighboring atomic moments, and a subset of all possible combinations is sampled uniformly at random. The resulting set of video moments corresponds to $V_{\text{span}}$.

\paragraph{Pseudo-query Generation ($f_{\text{pq}}$).} The pseudo-query is constructed as a collection of objects present in the video. To generate the pseudo-query, we employ an off-the-shelf object detector, enabling the extraction of pertinent objects in \(V_{\text{span}}\). We adopt a top-$k$ strategy to sample the $k$ most probable object predictions associated with the query \query.

\paragraph{Video Encoder.}
We uniformly sample $T$ frames from $V$ and extract their CNN (\eg, I3D~\cite{qian_locate_2022}) features. These features are contextually encoded using a video encoder ${\phi}_{v}$ to yield frame features ${\phi}_{v}(V)\!=\!\left\{ v_{1},v_{2},\ldots,v_{T}\right\}$ where $v_{i}\in\mathbb{R}^{d}$, and $d$ is the common video/query encoding dimension. We implement ${\phi}_{v}$ as a GRU-based encoder.

\paragraph{Query Encoder.}
Our pseudo-query $Q$, composed of up to $k$ tokens, is encoded using a query encoder ${\phi}_{q}$ that generates query embeddings ${\phi}_{q}(Q)\!=\!\left\{ q_{1},q_{2},\ldots,q_{k}\right\}$, for the top-$k$ detected objects extracted from the pseudo-query generation. Here, $q_{i}\in \mathbb{R}^{d}$ and $d$ is the common video/query encoding dimension. We implement ${\phi}_{q}$ as a bi-directional GRU-based encoder preceded by a trainable embedding layer. 

\subsection{Commonsense Enhancement Module}
\label{sec:cem}
To enrich the encoded video and query features with information grounded in commonsensical knowledge, we introduce a Commonsense Enhancement Module (CEM), pictorially described in Figure~\ref{fig:cem}. This enhancement helps inject necessary information into video and query representations, which can not just help bridge the gap between the available visual and textual cues but also provide rich information to the downstream span localization module. 

\begin{figure}[t!]
    \centering
    \includegraphics[width=0.8\linewidth]{figures/figure_files/Cem.pdf}
    \caption{\modelname Commonsense Enhancement Module (CEM). CEM comprises a concept encoder and an enhancement mechanism that uses the previously encoded concept vectors to update a given input vector (video/query vectors). The concept encoder employs a Graph Convolution Network for encoding the nodes (concepts) of \(G_C\). 
    }
  \label{fig:cem}
\end{figure}

CEM includes a set \(C\!=\!\left\{c_{1}, c_{2}, \dots, c_{n_{C}}\right\}\) of \(n_{C}\) concept vectors, where \(c_{i} \in \mathbb{R}^{d}\) and \(d\) is the concept feature dimension (same dimension as $\forall v_i \in V$ and $\forall q_i \in Q$). In general, given source feature vectors $S\!=\!\left\{ s_{1},s_{2},\ldots,s_{n}\right\}$ with individual feature vectors $s_{i \in [1,n]} \in \mathbb{R}^{d}$, the enhanced feature vectors $S_{C}$ are obtained using a commonsense enhancement mechanism $\phi_{C}$.
We implement this commonsense enhancement step $\phi_{C}$ as a cross-attention mechanism that enriches source input features, attending over $S$ guided by the commonsense concept vectors $C$, \ie, 
\begin{equation}
\label{eq:cenhance}
\scalemath{1}{
    }
    S_{C} = S + \phi_{C}(S) = S + \sigma \left( \frac{SW_{Q}(CW_{K})^{T}}{\sqrt{d}} \right) C W_{V},
\end{equation}
where $\sigma$ is a softmax activation, \(W_{Q}\), \(W_{K}\), \(W_{V}\) are trainable matrices and \(d\) is the common dimension of the vectors \(S\) and \(C\). In our setting, the source feature vectors $S$ are either video $V$ or pseudo-query $Q$ features. We build separate enhancement mechanisms for $V$ and $Q$, \ie, the projection matrices \(W_{Q}\), \(W_{K}\), \(W_{V}\) are not shared between $Q$ and $V$. We elaborate more on the rationale in the appendix.
The enriched video and pseudo-query features are denoted as \(V_{C}\!=\!\phi_{C_{\text{vid}}}(V)\) and \(Q_{C}\!=\!\phi_{C_{\text{pq}}}(Q)\), respectively.

\paragraph{Concept Encoder.}
The concept vectors \(C\) mentioned above are feature representations that internally form the nodes of the commonsense graph, \(G_C\). Accordingly, graph \(G_{C}\) is represented as a matrix, where \(G_{C(i,j)}\) represents the total number of directed relational edges between \(c_{i},c{j} \in C\) that start at \(c_i\) and end at \(c_j\). To encode the commonsense information, we employ Graph Convolutional Networks (GCN) \cite{hammond_wavelets_2011}. The concept encoder is composed of $L$ graph convolution layers, each of which performs a convolution step
\begin{equation}
\scalemath{1}{
    C^{\left(l+1\right)}=\sigma \left( AC^{\left(l\right) }W^{\left( l\right) }\right),
    }
\end{equation}
where $C^{\left(l\right)}$ are node (concept) features and $W^{\left( l\right)}$ trainable weight matrix of layer $l \in [1, L]$, $\sigma$ is a nonlinear activation function, and $A$ is the adjacency matrix obtained by normalizing graph $G_C$ with the degree matrix $D$. Since $G_C$ is a directed graph, normalization can be formulated as $A\!=\!D^{-1}G_{C}$.

\paragraph{Commonsense Information.}
We use ConceptNet \cite{speer_conceptnet_2017}, a popular knowledge graph that provides information spanning various types of relationships such as physical, spatial, behavioral, \etc To ensure that the ConceptNet information utilized is relevant to themes found in the video data, we consider the set of objects available in pseudo-queries and include the top-$k$ most frequently occurring objects to be the seed concept set \(C\). We extract the  ConceptNet subgraph that includes all edges incident between the concepts in \(C\). 
We filter the edge types based on a pre-determined relation set \(R\), which is compiled to involve relations that are relevant to the nature of the video localization task, \eg, spatial (\textit{AtLocation}, \etc) and temporal (\textit{HasSubevent}, \etc) relations are useful for video understanding, while \textit{RelatedTo} and \textit{Synonym} are fairly generic relations that add little information to the localization task. Table \ref{tab:relations} shows the relations included in \(G_C\).

\paragraph{Cross-Modal Interaction Module.} The commonsense enriched video and query features, \(V_{C}\) and \(Q_{C}\), are fused with a multi-modal cross-attention mechanism. We employ a two-step fusion process. First, Query-guided Video Attention (QVA) is applied to attend over video $V_C$, and Video-guided Query Attention (VQA) attends over query $Q_C$ guided by video $V_C$, resulting in updated features $V_C'$ and $Q_C'$, respectively. Both QVA and VQA utilize Attention Dynamic Filters~\cite{rodriguez_proposal-free_2020} that adaptively modify video features, dynamically adjusting them in response to the query, and vice versa. Next, the attended features are fused using a cross-attention mechanism over $V_C'$ guided by $Q_C'$, resulting in localized video features $V_{C_{\text{loc}}}$.

\paragraph{Temporal Regression Module.}
The final step involves a regression layer that approximates $\hat{V}_{\text{span}}$. We employ attention-guided temporal regression to estimate the span of the target video moment. To find important temporal segments relevant to the query, the fused features $V_{C_{\text{loc}}}$ are temporally attended based on the query features to obtain $V_{\text{ta}}$. Then, the span boundaries are localized using a regressor implemented as a Multi-Layer Perceptron (MLP).

\begin{align}
{o}_i = \sigma\left({W}_{1} V_{C_{\text{loc}_i}} + {b}_{{1}}\right) \\
V_{\text{ta}} = \sum_{i=1}^{T} o_i V_{C_{\text{loc}_{i}}} \\
[\hat{t}_s, \hat{t}_e] = {W}_2 {V}_{\text{ta}} + {b}_{2}.
\end{align}
Here, ${W}_{1}$ and ${b}_1$ are the weight matrix and bias vector of the temporal attention MLP, $\sigma$ represents the sigmoid activation function, $V_{C_{\text{loc}_i}}$ stands for the encoded localized video features, ${V}_{\text{ta}}$ represents the temporally attended video features, ${W}_2$ and ${b}_2$ denote the weight matrix and bias vector of the regression MLP, and $[\hat{t}_s, \hat{t}_e]$ correspond to the start and end timestamps of the predicted video span $\hat{V}_{\text{span}}$.

\begin{table}[t!]
\centering
\resizebox{\linewidth}{!}{
\begin{tabular}{ll}
\toprule
\textbf{Category} & \textbf{Relations}                                                                                         \\ \toprule
Spatial           & AtLocation, LocatedNear                                                                                    \\ \midrule
Temporal          & \begin{tabular}[c]{@{}l@{}}HasSubevent, HasFirstSubevent, HasLastSubevent, HasPrerequisite\end{tabular} \\ \midrule
Functional        & UsedFor                                                                                                    \\ \midrule
Causal            & Causes                                                                                                     \\ \midrule
Motivation        & MotivatedByGoal,  ObstructedBy                                                                             \\ \midrule
Other             & CreatedBy, MadeOf                                                                                          \\ \midrule
Physical          & \begin{tabular}[c]{@{}l@{}}HasA, HasProperty, Antonym, SimilarTo\end{tabular}                      
\\ \bottomrule
\end{tabular}
}

\caption{Relations in the Commonsense Enhancement Module (CEM) grouped by categories.}
\label{tab:relations}

\end{table}
\subsection{Training and Inference}
The training objective is 
$\mathcal{L}_{loc} = \mathcal{L}_{treg}+\lambda \mathcal{L}_{ta},$ where \(\lambda\) is a balancing hyperparameter, \(\mathcal{L}_{ta}\) is a temporal attention guided loss and \(\mathcal{L}_{treg}\) is the regression loss.  The temporal attention-guided loss is defined as
\begin{equation}
\label{tatt}
\mathcal{L}_{ta} = \frac{\sum^{T}_{i=1}g_{i}\log \left( a_{i}\right)}{\sum^{T}_{i=1}g_{i}},
\end{equation}
where \(a_{i}\) is the attention weight for video frame \(v_{i}\) and \(g_{i}\) is the attention mask for \(v_{i}\), that is assigned to \(1\) if \(v_{i}\) is inside the target video segment, and \(0\) otherwise. 
This objective encourages the model to produce higher attention weights for video segments that are relevant to the query. 
On the other hand, \(\mathcal{L}_{treg}\) dictates the video span boundary regression and is the sum of smooth $\ell_1$ distances between start and end timestamps of the ground truth and predicted spans, \ie,
\begin{equation}
\label{treg}
\mathcal{L}_{treg} = \text{smooth}{\ell_1}(t_{s}, \hat{t}_{s}) + \text{smooth}{\ell_1}(t_{e}, \hat{t}_{e}).
\end{equation}
Here, $t_{s}$ and ${t}_{e}$ represent the ground truth start and end timestamps and $\hat{t}_{s}$ and $\hat{t}_{e}$ the predicted start and end timestamps, respectively.
The integration of a smoothing mechanism enhances training stability and improves the model's ability to handle outliers. Finally, during inference, we employ an off-the-shelf part-of-speech tagger to extract nouns from the text input query and feed them as query input to the trained \modelname video localizer.

\section{Assessment}
\label{sec:assessment}
\subsection{Experimental Setup}
We implement our PCDNet in PyTorch \cite{paszke2019pytorch} and train it for 300 epochs with the batch size of 32 on two NVIDIA GeForce RTX 3090 GPUs. We use stochastic gradient descent (SGD) \cite{amari1993backpropagation} with a momentum of 0.937 and a weight decay of $5 \times 10 ^{-4}$ during training. The initial learning rate is set to 0.01 and decayed to 0.001 using a cosine annealing schedule. We initialize PCDNet randomly and load the weights of CSPDarknet53 \cite{wang2020cspnet} pre-trained on ImageNet \cite{imagenet_cvpr09} for the encoder part. To increase the diversity and complexity of the training samples, we apply data augmentations including random cropping, random flipping, and mosaic \cite{redmon2018yolov3}. We use the evaluation metrics of Microsoft COCO \cite{lin2014microsoft} for validation.

\begin{table}[ht]
\caption{Quantitative comparison against state-of-the-art polarization-based detectors ($\star$), single-stage detectors ($\dag$), two-stage detectors ($\ddag$), anchor-based detectors ($\triangle$), anchor-free detectors ($\circ$), and self-supervised method ($\S$).}
\small
\centering
\renewcommand\arraystretch{0.9}
\setlength{\tabcolsep}{2.6pt}
\begin{tabular}{lccccc}
\hline\hline
Methods	&	Pub'Year	&	Backbone	&	AP	&	AP50	&	AP75	\\
\hline
Faster R-CNN$^{\ddag\triangle}$ 	&	NeurIPS'15	&	Res50	&	44.8	&	75.4	&	45.4	\\
SSD$^{\dag\circ}$ 	&	ECCV'16	&	VGG16	&	25.5	&	52.6	&	22.6	\\
Cascade R-CNN$^{\ddag\triangle}$ 	&	CVPR'18	&	Res50	&	45.8	&	73.2	&	47.8	\\
CornerNet$^{\dag\circ}$ 	&	ECCV'18	&	Res50	&	19.8	&	47.4	&	29.6	\\
P-SSD I$^{\star\dag\circ}$ 	&	ITSC'19	&	VGG16	&	25.9 	&	53.1	&	22.7	\\
P-SSD S$^{\star\dag\circ}$ 	&	ITSC'19	&	VGG16	&	23.0 	&	48.9	&	20.1	\\
FCOS$^{\dag\circ}$ 	&	ICCV'19	&	Res50	&	23.1	&	50.9	&	18.4	\\
DH R-CNN$^{\ddag\triangle}$ 	&	CVPR'20	&	Res50	&	32.7	&	65.3	&	28.2	\\
Dynamic R-CNN$^{\ddag\triangle}$ 	&	ECCV'20	&	Res50	&	46.2	&	74.2	&	48.0	\\
EfficientDet$^{\ddag\triangle}$ 	&	CVPR'20	&	D3	&	45.3	&	73.0	&	46.3	\\
VarifocalNet$^{\dag\circ}$  & CVPR'21 & Res50 & 44.2 &	73.5 &	44.4	\\
D-DETR$^{\dag\circ}$ 	&	ICLR'21	&	Res50	&	43.8	&	74.9	&	44.3	\\
DDOD$^{\dag\circ}$ 	&	MM'21	&	Res50	&	43.5	&	73.0	&	43.3	\\
TOOD$^{\dag\triangle}$ 	&	ICCV'21	&	Res50	&	44.3	&	74.3	&	44.6	\\
YOLOX$^{\dag\circ}$ 	&	arXiv'21	&	YOLOX-l	&	54.3	&	82.5	&	56.7	\\
YOLOv7$^{\dag\triangle}$	&	arXiv'22	&	Dark53	&	57.6	&	84.3	&	60.3	\\
RTMDet$^{\dag\circ}$ 	&	arXiv'22	&	RTMDet-l	&	53.9	&	81.4	&	56.7	\\
DINO$^{\dag\circ\S}$ 	&	ICLR'22	&	Res50	&	52.7	&	81.8	&	54.8	\\
YOLOv8$^{\dag\circ}$ 	&	-'23	&	YOLOv8-l	&	56.8	&	83.6	&	59.0	\\
\hline
\textbf{PCDNet$^\star$}	&	\textbf{Ours}	&	Dark53	&	\textbf{58.5}	&	\textbf{85.2}	&	\textbf{61.5}	\\
\hline\hline
\end{tabular}
\label{tab:comparison}
\end{table}

\begin{figure*}[htp]
    \centering
    \begin{center}
        % \includegraphics[width=\linewidth]{figure/comparison.pdf}
        \includegraphics[width=\linewidth,height=10.5cm]{figure/comparison.pdf}
    \end{center}
    \caption{Qualitative comparison of PCDNet against state-of-the-art detectors retrained on RGB-P Car dataset.} 
    \label{fig:comparison}
\end{figure*}

\subsection{Qualitative and Quantitative Evaluation}
We extensively compare our PCDNet with 19 state-of-the-art methods by retraining and testing all methods on the RGB-P Car dataset using their original settings. The compared methods include two-stage detectors such as EfficientDet \cite{tan2020efficientdet} and the R-CNN family \cite{Ren_2017, Cai_2019, zhang2020dynamic}, and one-stage detectors such as SSD \cite{liu2016ssd}, and YOLO family \cite{ge2021yolox, wang2022yolov7, ultralytics2023yolov8}. These methods also comprise anchor-based methods such as the R-CNN family and YOLOv7 \cite{wang2022yolov7}, and anchor-free methods such as CornerNet \cite{law2018cornernet}, VarifocalNet \cite{zhang2021varifocalnet}, and YOLOv8 \cite{ultralytics2023yolov8}. Some detectors use traditional convolutional networks such as FCOS \cite{tian2019fcos} and RTMDet \cite{lyu2022rtmdet} while others use transformer structures, such as DeformableDETR \cite{zhu2020deformable} and DINO \cite{zhang2022dino} that employs self-supervised learning. We also include the P-SSD \cite{blin2019road} that utilizes polarization information. The quantitative evaluation results are reported in Tab. \ref{tab:comparison}. We can see that our method outperforms all competing state-of-the-art methods. 

Fig. \ref{fig:comparison} further qualitatively demonstrates the benefits of our method: a) in poorly lit indoor parking lots, distinguishing black cars behind pillars is extremely challenging (the first two rows). The compared methods tend to conflate the shadow and the black car (\textit{i.e.}, merging cars on either side of the pillar into a single entity or treating partial views of the car as one object) while our PCDNet can handle such ambiguities; b) in the third example, all methods except our PCDNet fail to detect a partially visible car obstructed by another car or misplace it with the previous car; c) in the fourth example, RGB-based methods wrongly identify distant pedestrians as cars, but our PCDNet method can effectively eliminate such interference with the help of polarization cues; d) the fifth and sixth examples depict black cars in an outdoor parking lot at night which are very hard to be distinguished in the RGB image. Despite the enhancement through ZeroDCE \cite{guo2020zero}, the sixth example remains unclear. By contrast, polarization imaging is robust to low light conditions, enabling our robust car detector PCDNet; and e) the last row shows a virtual car reflected in a mirror located at the upper-left corner of the image. The mirrored virtual car and the rest of the mirror regions exhibit similar and smooth AoLP, providing useful cues for PCDNet to recognize this region as background. 


\subsection{Ablation Study}
\textbf{Impact of Spectral Intensity and Polarization Cues.} We conduct a series of ablation experiments to demonstrate the effects of spectral intensity and polarization cues on car detection (Tab. \ref{tab:abl_input}).
The results show that: a) combining different forms of polarization cues with RGB as the input of PCDNet can improve the car detection accuracy (\textit{C}, \textit{D}, \textit{F}, \textit{G}, \textit{K} and \textit{L} are higher than \textit{B}); b) DoLP cues have a greater impact than AoLP cues (\textit{D}, \textit{J} and \textit{L} are better than \textit{C}, \textit{I} and \textit{K}, respectively); c) stacking AoLP and DoLP on RGB in the channel dimension does not boost performance (\textit{E} is slightly lower than \textit{B}), possibly because the characteristic gap between different modalities hinders effective features extraction; d) spectral intensity and polarization are more beneficial than monochromatic intensity and polarization for car detection (comparing paired \textit{B} and \textit{H}, \textit{C} and \textit{K}, \textit{D} and \textit{L}, \textit{I} and \textit{K}, \textit{J} and \textit{L}); e) enhancing RGB image via ZeroDCE \cite{guo2020zero} is less effective than introducing polarization (\textit{M} performs worse than \textit{C}-\textit{G}, \textit{K} and \textit{L}).
Fig. \ref{fig:abl_input} provides visual support for these observations.

\begin{table}[t]
\small
\centering
\caption{Quantitative comparisons of ablation with different inputs. ``stacked I'' denotes the stacked intensity measurements with a linear polarization angle of 0$^{\circ}$, 45$^{\circ}$ and 135$^{\circ}$ and ``stacked S'' refers to the stacked Stokes elements S0, S1 and S2 \cite{blin2019road}.}
\begin{tabular}{clccc}
\hline\hline
	&	PCDNet Input	&	AP	&	AP50	&	AP75	\\
 \hline
\textit{A}	&	RGB, AoLP and DoLP (original)	&	58.5 	&	85.2 	&	61.5 	\\
\hline
\textit{B}	&	RGB only	&	57.6 	&	84.3 	&	60.2 	\\
\textit{C}	&	RGB and AoLP	&	58.0 	&	84.6 	&	60.7 	\\
\textit{D}	&	RGB and DoLP	&	58.3 	&	85.4 	&	61.1 	\\
\textit{E}	&	stacked RGB, AoLP and DoLP	&	57.5 	&	84.3 	&	59.9 	\\
\textit{F}	&	RGB and stacked I	&	58.0 	&	84.1 	&	61.0 	\\
\textit{G}	&	RGB and stacked S	&	57.8 	&	84.8 	&	60.4 	\\
\textit{H}	&	Gray only	&	57.4 	&	84.3 	&	60.0 	\\
\textit{I}  &   Gray and mono AoLP & 57.5 & 84.5 & 60.5 \\
\textit{J}  &   Gray and mono DoLP & 57.6 & 84.9 & 60.1 \\
\textit{K}	&	RGB and mono AoLP	&	57.9 	&	84.6 	&	60.5 	\\
\textit{L}	&	RGB and mono DoLP	&	58.2 	&	84.9 	&	60.6 	\\
\textit{M}  &   Enhanced RGB & 57.4 & 84.0 & 60.0 \\
\hline\hline
\end{tabular}
\label{tab:abl_input}
\end{table}

\begin{figure}[t]
    \centering
    \includegraphics[width=1\linewidth]{figure/abl_input.pdf}
    \caption{Qualitative comparison of ablation with different inputs. The model with RGB intensity only is susceptible to interference from ghost car caused by water on the road.}
    \label{fig:abl_input}
\end{figure}

\textbf{Influence of PCDNet Components.}
First, we investigate the performance of different strategies for fusing AoLP and DoLP inputs. From Tab. \ref{tab:abl_module}(\textit{A}-\textit{D}), we observe that our PI module is more effective than the simple fusion methods including concatenation, addition and element-wise multiplication.
Second, by removing MP module \ref{tab:abl_module}(\textit{E}) from the original PCDNet (A), the detection performance declines. This demonstrates that exploring the polarized material features of cars across all learning samples is useful. We also explore the influence of applying MSP and MCP on different levels of features. The results in Tab. \ref{tab:abl_module}(\textit{A},\textit{F}-\textit{G}) show that applying MSP on shallower features and MCP on deeper features can yield better performance.
Finally, we validate the effectiveness of CDDQ module.
Removing the CDDQ module (\textit{I}) from PCDNet (\textit{A}), which causes the feature extraction processes of the RGB and polarization to be independent from each other, leads to the performance drop. We also demonstrate the benefits of the CWDA and SDMD in the CDDQ module by removing either of them (\textit{J} and \textit{K}). 

\begin{table}[t]
\small
\centering
\caption{Quantitative comparisons of ablation with different modules demonstrate that all component of PCDNet contributes to the overall performance. We used sequences of three letters separated by '-' and enclosed in parentheses to represent different combinations of MSP and MCP.}
\begin{tabular}{clccc}
\hline\hline
	&	Ablation	&	AP	&	AP50	&	AP75	\\
 \hline
\textit{A}	&	PCDNet (original)	&	58.5 	&	85.2 	&	61.5 	\\
\hline
\textit{B}	&	Input RGB and [AoLP DoLP]	&	58.2 	&	85.4 	&	60.9 	\\
\textit{C}	&	Input RGB and AoLP+DoLP	&	58.1 	&	84.8 	&	60.5 	\\
\textit{D}	&	Input RGB and AoLP*DoLP	&	58.1 	&	84.8 	&	60.5 	\\
\hline
\textit{E}	&	A \textit{w/o} MP	&	56.9 	&	84.2 	&	59.2 	\\
\textit{F}	&	A \textit{w/} M(S-S-S)P	&	58.2 	&	85.2 	&	60.8 	\\
\textit{G}	&	A \textit{w/} M(S-C-C)P	&	58.2 	&	85.0 	&	60.9 	\\
\textit{H}	&	A \textit{w/} M(C-C-C)P	&	58.1 	&	85.0 	&	61.1 	\\
\hline
\textit{I}	&	A \textit{w/o} CDDQ	&	58.0 	&	84.7 	&	60.8 	\\
\textit{J}	&	A \textit{w/o} SDMD	&	58.2 	&	85.2 	&	60.8 	\\
\textit{K}	&	A \textit{w/o} CWDA	&	58.3 	&	85.1 	&	61.1 	\\
\hline\hline
\end{tabular}
\label{tab:abl_module}
\end{table}

\subsection{Limitations}

When both the RGB intensity and the polarization measurement yield weak car signals, our method's effectiveness declines. Specifically, in low-light scenarios, when a car approaches on an unlit road, the strong light from its headlights can create a ``hole'' in both the RGB and polarization and obscure the entire car. We illustrate such an example in Fig. \ref{fig:failure} where the extreme HDR and heavy motion blur in the captured image limit its depiction of both RGB and polarization. In these challenging scenarios, prior RGB-based methods and even human vision are powerless.

\begin{figure}[t]
    \centering
    \includegraphics[width=1\linewidth]{figure/failure.pdf}
    \caption{PCDNet has limited ability to handle extreme HDR or heavy motion blur cases.}
    \label{fig:failure}
\end{figure}


\section{Related Work}
\label{sec:related-work}

\paragraph{Datasets.}
The lack of cross-file and cross-project (e.g. dependencies) information is a general issue in current evaluation datasets for code.
In terms of code completion, common choices are Py150 \citep{raychev2016probabilistic} for Python and Github Java Corpus \citep{allamanis2013mining} for Java. Both datasets are constructed at file level, where source files are isolated from their project and dependencies and no consideration of project separation is taken in constructing training and test sets.
\citet{lu2022reacc} constructed a code completion dataset from CodeNet \citep{puri2021project}, which contains coding problems and solutions from online judge websites and also lacks project context. 
\citet{clement2021long} presented a real-world Python method generation task based on CodeSearchNet \citep{husain2019codesearchnet} but the auxiliary information they extract still comes from within a local file. 
\citet{svyatkovskiy2021fast} constructed a completion dataset based on top Python repositories on GitHub and released the URLs for these repositories. 
However, those repositories are not write-protected and can change over time. Besides, setting up the dependency environments at scale for further analysis is not easy. 
Both make their dataset difficult to reproduce.
In the contrast, we release the code and the dependencies for the projects to ensure reproducibility.
Apart from code completion, datasets for other code tasks such as Cloze test \citep[e.g.][]{feng2020codebert}, code refinement \citep[e.g.][]{tufano2019empirical, yasunaga2021break, haque2022fixeval}, and generating code from text descriptions \citep[e.g.][]{chen2021evaluating, hendrycks2021measuring, austin2021program}, are often small and mostly without project-level code context. 
Beyond-local information is beneficial for programmers to solve programming tasks in real-world settings. The lack of such information in the current dataset would restrict the progress into high-level semantic understanding and reasoning in the code domain.


\paragraph{Code language models.}
Encouraged by the success of pretrained language models in natural language processing \citep{devlin2019bert, liu2019roberta, lewis2019bart, raffel2020exploring} and the promise of naturalness in code \citep{hindle2016naturalness, allamanis2018survey}, we have seen rising adaptations of language models for code. For example, CuBERT \citep{kanade2020learning} and CodeBERT \citep{feng2020codebert} are pretrained based on masked language modeling. GPT-C \citep{svyatkovskiy2020intellicode} and CodeGPT \citep{lu2021codexglue} are both pretrained based on unidirectional language modeling. PLBART \citep{ahmad2021unified} and CodeT5 \citep{wang2021codet5} are pretrained encoder-decoder structures which adopts denoising objectives and can support code understanding and code generation. UnixCoder \citep{guo2022unixcoder} combines the above three pretraining objectives for a unified pretrained model. 




\paragraph{Code completion.}
Code completion is an essential feature for modern IDEs and an important topic for code intelligence. 
In recent years, deep neural networks \citep{liu2016neural, li2018code, alon2020structural, liu2020multi, kim2021code}, especially pretrained language models \citep{svyatkovskiy2020intellicode, lu2021codexglue} become the mainstream solution to this task. 
Still, incorporating additional information proved beneficial.
One popular choice is abstract syntax tree, e.g. \citet{kim2021code, peng2021could, guo2022unixcoder}. 
However, \citet{lopez2022ast} suggested that pretrained code language models may have already encoded the syntax.  
Other proposals seek to use data flow graph, control graph, and various graph relations, e.g. \citet{guo2020graphcodebert, hellendoorn2019global}.
However, information is still restricted from a single file.
We instead try to enhance the model with out-of-file information, similar to what is accessible in a development environment.

For project-level analyzer induced information, \citet{svyatkovskiy2021fast} described a way to use a static analyzer to refine completion candidates from neural methods.
\citet{weyssow2020combining} considered leveraging the project-wise contexts via embeddings for better function call completion performance.
Other than code completion, project-level information has been utilized for methods name prediction~\citep{wang2021lightweight} and generating code from text descriptions~\citep{lyu2021embedding}.
However, none of them tested their approaches with pretrained code language models. 
In terms of incorporating additional context through concatenation,
\citet{clement2021long} reported improvements from prioritize certain parts of in-file context.
Recently, \citet{lu2022reacc} proposed to enhance code language models by concatenating similar code fragments retrieved by a neural network. Despite the general similarity, we 1) use a simple lightweight way to retrieve auxiliary information instead of training a heavy retriever; 2) do not restrict ourselves on similar code fragments and show that dissimilar code fragments (function implementation) can be helpful; 3) explore task-specific fine-tuning with retrieved information for better completion.






\section{Conclusion}
We curated \PyEnvs, a collection of permissively licensed Python packages along with isolated development environments.
Upon that, we built a function call argument completion dataset \CallArgs containing analyzer-induced information for each function call.
We experimented feeding auxiliary information as additional input context to various pretrained code language models for call argument completion during training and inference.
Results show that access to the function implementation and function usages universally improves the model performances.
We further provide insights on the effect of different types of models and different types of additional information on this task.
In the future, we can use \PyEnvs to construct new datasets for other code-related tasks to further study the benefits from cross-file and cross-project information.

Minus voluptatem tenetur at inventore odio iusto explicabo autem, iure repellendus saepe, officiis nihil aliquam debitis dolor minima suscipit et aperiam, dolor quia odio fugiat molestiae ea laudantium ipsam expedita aperiam.Sequi libero accusamus sapiente, quas ut dolores debitis perspiciatis non voluptatibus, quasi debitis eos adipisci nemo sit praesentium?Dolorem laudantium obcaecati numquam saepe perspiciatis voluptatem veritatis, quo molestiae fuga corrupti officia quidem eaque assumenda?Porro odit corporis doloribus aut recusandae delectus quaerat eligendi, amet officia ab fugiat eaque sit totam pariatur minus dignissimos.Corporis vel ab, nesciunt accusantium quibusdam ratione consequatur asperiores, earum alias aliquid repellat iusto quia nulla.Nesciunt voluptatibus perspiciatis, quod nulla amet ipsum neque vitae necessitatibus perferendis, nam quos molestias voluptates dignissimos ex doloremque.Nulla impedit optio exercitationem, repudiandae architecto molestias et sunt, similique fugit quam iure iste?Impedit fugit inventore laudantium laborum tempora possimus perspiciatis eligendi incidunt, temporibus similique laudantium consectetur soluta dignissimos necessitatibus modi distinctio, nesciunt magnam in eius?Praesentium molestias quo in soluta, tempora fugit accusantium eius aliquam, eos vero laboriosam adipisci ducimus recusandae sit in aperiam amet voluptatum, iusto doloremque dolores nihil quod reprehenderit corporis dolorum illo vel.\clearpage
\bibliography{main}



% Appendix is omitted for the camera-ready submission.


\end{document}